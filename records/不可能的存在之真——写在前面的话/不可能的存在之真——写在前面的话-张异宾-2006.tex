\documentclass{article}
\usepackage[utf8]{inputenc}
\usepackage{ctex}

\title{不可能的存在之真——写在前面的话\footnote{Click to View:\url{https://web.archive.org/web/20221026091213/https://rentry.co/naxep}}}
\author{张异宾}
\date{2006-02}

% \setCJKmainfont[BoldFont = Noto Sans CJK SC]{Noto Serif CJK SC}
% \setCJKsansfont{Noto Sans CJK SC}
% \setCJKfamilyfont{zhsong}{Noto Serif CJK SC}
% \setCJKfamilyfont{zhhei}{Noto Sans CJK SC}
% \setlength\parindent{0pt}

\begin{document}
\CJKfamily{zhkai}

\maketitle


\Large

对中国人而言,生活从未像今天这样被剧烈地撕裂着。物化市场的魔域中,生存已在照亮了宗法土地的太阳(理性之光)阴影中断为两截:一种是在这个沙漏般的人与物“最佳配置方式”的竞争世界中悬临于空中的“成功”人士(过去叫”布尔乔业”),另一种则是跌落入沙漏底层的“弱势”众生(过去叫“普罗泰利特”)。在第一种人那里,“他”会是开着宝马,拥着美人,银行户头上有无数的金钱,甚至还揣着博士学位证书,并拥有众多令人仰慕的学术头衔和官品,在象征关系舞台的重重射灯打探之下,诗意地“在”着。而在第二种人那里,炫目灯光下功成名就的“他”成了“我”毕生不懈奋斗的镜像,这神化了的另一个“他”就应该是“我”。这种“他”对“我”的理想性自居,使十分赢弱无力的“我们”更加举步维艰,一次又一次,我们在泥泞的沙漏边缘
\newpage
攀爬,滑倒,再攀爬一一如加缪笔下那个荒谬而不屈的西西弗斯。“我”,永远向上推举着不断落下的存
在巨石。 

对第一种人来说,你以为你是人,可是你却可能真的不是。这个真相只有在金钱散去、香车美人飞离时大突然显现。这一格式塔伪境,也,许会是在“他”离不可能的存在之真一一拉康哲学映像而对后者来讲,穷尽一生,“我”总以为自己还不是“人”,犹自固执地向着第一种人的镜像埋头奋斗,却阵然不知与之苦战的唐·吉诃德式的风车和西西弗斯之巨石的真在。这是作者想通过拉康故事的道说所讲的一个普通的事理(此理,形同曹雪芹《红楼梦》中的“好
了歌”) 

当然,拉康的故事从总体而言,肯定还是一种错误。因为,如果没有了拉康所骂的想象和象征关系的织入,特别是被拉康遮蔽掉的人对世界的感性活动关系的历史构境,人的存在也就真的没有了。然而,拉康之骂对那些无法认清自己的人却又不见得是一种谬论。遗憾的是,今天这个世界上不能正确认识自己
\newpage
的人恐怕占了绝大多数,因此,拉康之骂对于我们又
会是过于真实的真在。 

所以,拉康的故事固然只是一则离奇的哲学寓言,可也的确诉说了一种常人不可见的真实。依我看,听懂它的言说并不见得必然使我们走向悲观的虚无(我的一位朋友在听了拉康的故事之后,跺足大叫:“这正是佛、老之无!”)。它无非是让我们多一种清醒,以正确认识自己,认真对待我们存在中的种种物化和异化之自我疏远,自省生命中的能指之漂浮和本己的不可能性罢了。那样,这个世界上大抵真会少些不自知的疯狂和看似正常的精神疾病。

\end{document}
