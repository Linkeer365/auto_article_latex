\documentclass{article}
\usepackage[utf8]{inputenc}
\usepackage{ctex}

\title{万火归一\footnote{Click to View:\url{https://web.archive.org/web/20230121032908/https://www.99csw.com/book/10596/383031.htm}}}
\author{胡利奥·科塔萨尔}
\date{1966}

% \setCJKmainfont[BoldFont = Noto Sans CJK SC]{Noto Serif CJK SC}
% \setCJKsansfont{Noto Sans CJK SC}
% \setCJKfamilyfont{zhsong}{Noto Serif CJK SC}
% \setCJKfamilyfont{zhhei}{Noto Sans CJK SC}
% \setlength\parindent{0pt}

\begin{document}
\CJKfamily{zhkai}

\maketitle


\Large

等哪天自己立了雕像就会是这个模样,总督一面不无讽刺地想着,一面举起胳膊,摆出问候的姿势,僵立在一连两个小时的竞技和高温后依然毫无倦意、欢呼不止的公众之中。时候到了,这是他允诺过的意外惊喜;总督放下手臂,看向他的妻子,妻子带着节庆日那种空洞的微笑回望着他。伊蕾妮并不知道接下来的节目,却显出了然于心的神色,一旦学会了用总督所憎恶的那种无动于衷去承受主子的各种奇思怪想,即便是意外她也都能习以为常。她不用转过身去看竞技场,便预见到大局已定,接下来的事情会一如既往地残忍。葡萄园主利卡斯和他妻子乌拉妮娅最先喊出了一个人的名字,人群立即不断应和。“这是我特意为你准备的惊喜,”总督说,“大家向我保证,你喜欢这个角斗士的风格。”伊蕾妮脸上依然挂着不变的微笑,点点头表达谢意。“尽管讨厌这样的游
\newpage
戏,你还是出席了,我们大家都倍感荣幸,”总督接着说道,“我唯有努力把最讨你欢心的东西献给你,才恰如其分。”“你是世间的盐!”利卡斯高声喊道,“你把战神的化身带到我们这个卑微的行省竞技场!”“你现在看到的还只是一半。”总督借葡萄酒杯润了润嘴唇,再把酒杯递给妻子。伊蕾妮饮了一大口,仿佛淡淡的酒香能驱走那久久不散的浓烈血腥和粪便味。竞技场上突然一片寂静,全场期待中,马尔科的身影异常鲜明,他走到了场地中央;一缕阳光透过古老的帷幔斜射下来,在他的短剑上映出一道寒光,他漫不经心地在左手上提了一面青铜盾牌。“你该不会是想让他同斯米尔纽的胜利者对决一场吧?”利卡斯兴奋异常地问道。“不只如此,”总督答道,“我想让你们这个行省因这场比赛而记住我,也想让我妻子不再无聊。”乌拉妮娅和利卡斯鼓起掌来,期待着伊蕾妮的回应,可她只是默默地把酒杯递给了奴隶,全然不在意第二名角斗士上场而引发的如雷欢声。马尔科一动不动,同样漠然地身处对手得到的欢呼之中
;他用剑尖轻轻敲击着金色胫甲。 

“你好。”罗兰·勒努瓦边说边拣出一支香烟
\newpage
,这是他每次拿起电话以后必做的动作。听筒里有串线的杂音,有个人在念数字,突然间又沉寂下来,可这沉寂比电话遮在耳孔上的黑暗还要难以捉摸。“你好。”罗兰又说了一遍,把烟架在烟灰缸边,在睡衣口袋里摸火柴。“是我。”电话里传来让娜的声音。罗兰眯起眼睛,乏乏地伸展身体,换了个更舒服的姿势。“是我。”让娜徒劳地重复道。罗兰一直不答话
,她又加了一句:“索尼娅刚从我这儿离开。” 

按照规矩他此刻应该把目光投向帝王包厢,像往常一样行礼致敬。他知道自己必须这样做,这样他将看见总督夫人,当然还有总督,也许总督夫人会像最近几场竞技时一样,对他莞尔一笑。他不用思考,他几乎不会思考了,可本能告诉他这个场地不好,在这巨大的古铜色环形场地上,栅栏和棕榈树叶掩映着一条条弯曲的信道,此前打斗的留痕使得信道更为幽暗。前一天夜里他梦见一条鱼,梦见残破的柱子之间有一条孤零零的小路;就在他披挂上阵的时候,有人在低声说,总督不会付给他金币。马尔科懒得去打听,又有个人不怀好意地大笑起来,没有向他转身,便直接走远了;后来,第三个人对他说,刚才那位是他
\newpage
在马西利亚杀死的那个角斗士的兄弟,可这时他已经被推上了信道,推向外面人声鼎沸的竞技场。天气热得让人受不了,头盔沉甸甸的,把阳光反射到帷幔和看台上。一条鱼,残破的柱子;那晦涩难懂的梦境,遗忘的深渊使他无从解读。帮他穿戴盔甲的人告诉过他,总督不会付给他金币;今天下午总督夫人也许不会冲他微笑。他对场上的欢呼无动于衷,因为此时的欢呼是为了他的对手,相较而言,不如片刻前为他发出的欢呼热烈,可其中又夹杂着若干惊呼,马尔科扬起头,向包厢看去,而伊蕾妮已经转过身,正同乌拉妮娅说话,总督在包厢里漫不经心地做了个手势,他立刻绷紧全身,手紧紧地握住了短剑的剑柄。现在他只要把目光投向对面的过道;但他的对手并没有出现在那里,却是平时放出猛兽的那个黑黢黢的通道口的栅栏门升起来,嘎吱嘎吱地响着,终于,马尔科看见了努比亚持网角斗士巨大的身影,在此之前布满苔藓的石壁隐匿了对手的身形;突然间他确定无疑地知道,总督是不会付给他金币的,他猜到了鱼和残破的柱子的含义。与此同时,对他来说,对手和他谁胜谁负已经不重要了,这是他们的职业,是命运,但他的身体还是绷紧了,仿佛他在害怕,他身体里有什么东西
\newpage
在问,为什么对手会从猛兽的通道出来,观众的欢呼中夹杂着同样的疑问,利卡斯向总督提出了这个问题,总督对这个出其不意的安排笑而不答,利卡斯笑着抗议,觉得有必要把赌注下在马尔科一边;不用听他们接下来的对话,伊蕾妮就知道总督一定会加倍下注赌那个努比亚人赢,然后和蔼可亲地看着她,让人给她上一杯冰镇葡萄酒。而她也一定会边喝酒边与乌拉妮娅评论一番那个努比亚持网角斗士的身材,评论他有多凶猛;每一个动作都已经事先设定好了,尽管人们自己并不知道,尽管细节有些出入,比如也许会没有这杯葡萄酒,或者乌拉妮娅欣赏那个彪形大汉时嘴型不同。利卡斯无数次观看过这类竞技赛事,是位行家,他会指给她们看那努比亚人穿过关猛兽的栅栏门时,头盔甚至擦过了高悬在门顶端、离地面足有两米的铁刺,他也会大加赞赏那人把鳞状渔网搭在左臂上的动作多么干净利落。自从很久以前的那个新婚之夜起,伊蕾妮就让自己缩回到内心的最深处,这一次也一如既往,同时表面上她顺从着,微笑着,甚至在尽情享受;在那自由而了无生气的深处,她感受到了死亡的征兆,总督将它伪装在一场公众娱乐的意外惊喜中,唯有她,也许还有马尔科,能领会这征兆,可此
\newpage
刻的马尔科,严峻,沉默,机械,他是不会明白的了,他的身体,在另一个午后的竞技场上她曾如此渴望的身体(这一点总督早已猜到,他从第一刻起就猜到了,一如既往,无须他那些巫师的帮助),将为纯然的幻想付出代价,因为她多看了一眼那个被一剑封喉
而死的色雷斯人的尸体。 

在给罗兰打电话之前,让娜的手翻过一本时尚杂志,把玩了一小瓶安定药片,还摸了摸蜷缩在沙发上的那只猫的脊背。接着罗兰的声音说:“你好。”声音带些困倦,突然间让娜有种荒谬的感觉,她想对罗兰说的话会让自己变成一个不折不扣的电话怨妇,而那唯一的听众面带嘲讽,在屈尊俯就的沉默中抽着烟。“是我。”让娜说,这句话更像是对她自己说的,而不是对着电话那头的寂静说,在这片寂静里,些许杂音仿若声音的火花在跳动。她看了看自己的手,这只不经意地摸过小猫又拨出号码(电话里不是还能听见号码的声音吗?难道不是有一个遥远的声音在向某个人报着数字,而那个听的人一言不发,在顺从地抄写吗?)的手,这只刚刚举起又放下镇静剂药瓶的手,她不愿意相信这就是她自己的手,也不愿意相信
\newpage
那个刚刚又说了一遍“是我”的声音就是自己的声音,这是她的最后一道防线了。为了尊严,什么话也别说,慢慢把电话挂上,一个人待着,干净利落。“索尼娅刚从我这儿离开。”让娜说,防线崩溃,荒谬开
始,安逸舒适的小小地狱。 

“哦。”罗兰说,一边擦着了火柴。让娜清清楚楚地听见了擦火柴的声音,就好像同时看见了罗兰的脸,他吸着烟向后靠去,两眼半睁半闭。渔网从那黑巨人手中扬起,像是一道波光粼粼的河流,马尔科堪堪避开。要是在从前——总督心中有数,他侧过头去,只让伊蕾妮看见他的笑容——马尔科一定会在瞬息之间抓住持网角斗士的弱点,用盾牌格挡长长的三叉戟的威胁,逼上前去,发出闪电般的一击,直扑对手毫无防备的胸膛。可马尔科仍然待在战圈之外,他弯曲着双腿,仿佛准备一跃而起,这时努比亚人飞快地把渔网收了起来,准备发动新的一击。“他完了。”伊蕾妮想道,她并没有看总督,后者正从乌拉妮娅递过来的盘子里挑拣甜点。“这不像之前的他了。”利卡斯想着,心疼自己下的赌注。马尔科微微弯下腰,两眼紧盯围着他打转的努比亚人;所有人都预感到
\newpage
的结局,只有他一无所知,他蹲伏着,无疑是在等待下一次机会,只是此前没能完成他的技艺所要求的行动让他有些迷茫。他需要更多的时间,比如胜利之后在酒馆的欢庆时刻,也许到那时才能理解为什么总督不会付给他金币。他沉着脸,等待下一个有利的时机;也许只有到了最后,等他把一只脚踏在持网角斗士的尸体之上时,他才能再一次看见总督夫人的微笑;可现在他没有这样想,而这样想的人却不再相信马尔
科的脚能踏上被割喉的努比亚人的胸膛。 

“有话快说,”罗兰说,“除非你想让我整整一下午都听这家伙给鬼知道是谁的什么念数字。你听见了吗?”“听见了,”让娜答道,“这声音听上去好远。三百五十四,二百四十二。”有那么一会儿,只余这个遥远单调的声音。“不管怎么着,”罗兰说,“他总归拿着电话在做点事情。”回答是可以预想到的,她会说出第一声抱怨,可让娜令沉默延续了几秒钟,才重复道:“索尼娅刚从我这儿离开。”她迟疑了片刻,又补充说:“她大概快到你家了。”罗兰大吃一惊,索尼娅没什么道理要到他家来。“别撒谎。”让娜说这话时,猫从她手里跳了出去,恼怒地看
\newpage
着她。“这不是谎话,”罗兰说,“我说的是时间,不是说她来或者不来。索尼娅知道,我不喜欢这个时间有人来访或者打电话。”八百零五,远远地,那个声音还在报数。四百一十六。三十二。让娜闭上双眼,等待着那个匿名者的声音第一次停顿,让她能够把余下的唯一的话说完。要是罗兰把电话挂了,至少在电话线的远方依然有那个声音,她还可以把电话附在耳边,在沙发里越陷越深,抚摸那只重新趴回她身旁的小猫,把玩药瓶,聆听数字,直到那个声音也累了,最后归于虚无,纯然的虚无,仿佛在手指间越来越沉重的不是听筒,而是某种死亡之物,应该看也不看就立刻丢掉。一百四十五,那声音还在报着数。更遥远的地方,有个人,仿佛一幅小小的铅笔素描,可能是一个腼腆的女人,在两声杂音之间问了句:“北站
?” 

他第二次躲开了渔网,可在向后一跃时估算失误,在一摊湿漉漉的沙土上滑了一下。马尔科颇有些费力地用短剑划出一道弧线,挡开了渔网,又伸出左臂,用盾牌截住了三叉戟重重的一击,观众的心一下子被提到半空。总督没去理睬利卡斯兴奋不已的评点
\newpage
,把头转向了不为所动的伊蕾妮。“机不可失,时不再来。”总督说道。“没机会了。”伊蕾妮回答道。“这不像之前的他了,”利卡斯又说了一遍,“这样下去他要吃亏的,那个努比亚人不会再给他机会了,看看他的样子。”远处,马尔科几乎一动不动,他好像已经意识到了自己所犯的错误;他高高举起盾牌,眼睛一眨不眨地盯住已经被收回的渔网,盯住在他眼前两米远处晃动着、仿佛在施加催眠术的三叉戟。“你说得有道理,这确实不像之前的他了,”总督说,“你是把赌注下在他身上了吧,伊蕾妮?”马尔科伏下身,随时准备跃起,他在皮肤上、在胃的深处感觉到,他已经被人群抛弃了。假使他能有片刻的镇定时间,他也许能解开那让他手足无措的心结,解开那看不见摸不着的锁链,那锁链来自他身后遥远的、他不知所在的某处,有时是总督的殷勤,是一笔非同寻常的酬金的许诺,也是一个梦境,梦里有一条鱼,而在这已经容不得他有半点迟疑的时刻,眼前晃动的渔网仿佛把从天幕缝隙里漏进来的每一缕阳光都网罗其中,他感到自己正是梦中的那条鱼。一切都是锁链,一切都是陷阱;他威胁似的猛然直起身,观众报以掌声,而那持网角斗士第一次向后退了一步,马尔科选择
\newpage
了唯一的路,困惑和汗水和血腥味,以及面前必须战胜的死亡;有人在微笑的面具后替他把什么都想到了,有人越过那个奄奄一息的色雷斯人的躯体安排了一切。“毒药,”伊蕾妮想,“我总会找到毒药,可现在,接受他递来的这杯酒吧,变得比他强大,等候你的时机。”遥远的声音重复着数字,断断续续地回响在那条阴暗凶险的通道里,通道不断延长,通话的停顿随之延长。让娜一直笃信人们真正想传递的信息往往在话语之外;对那些用心聆听的人来说,或许这些数字蕴藏着更丰富的含义,超过了其他所有的表达,就像索尼娅香水的味道,她临走前手掌在自己肩头的轻抚,比她的话更意味深长。但索尼娅自然不会满足于加密的信息,她想要的是一字一句、淋漓尽致。“我懂,对你来说,这很残酷,”索尼娅再一次说道,“可我讨厌装模作样,我宁可跟你实话实说。”五百四十六,六百六十二,二百八十九。“她去不去你家我不在乎,”让娜说,“现在我什么都不在乎了。”没有报另一个数字的声音,只有一阵长长的寂静。“你在听吗?”让娜问道。“我在听。”罗兰说着把烟头扔进烟灰缸,又从容地去够白兰地酒瓶。“我不明白的是……”让娜开了个头。“拜托,”罗兰说,“
\newpage
事到如今谁都弄不明白,亲爱的,再说,就算明白了又能怎么样呢。我很抱歉,索尼娅太着急了,这事情不该由她来告诉你的。该死,这些数字怎么没完没了的?”那小小的声音让人想到组织严密的蚂蚁世界,在那片渐渐迫近、越发厚重的寂静之下,那声音继续有条不紊地报数。“可是你,”让娜毫无章法地说,
“总之,你……” 

罗兰啜了口白兰地。他一向喜欢斟酌字句,不讲一句多余的话。而让娜会把同一句话翻来覆去地讲,每次将重音放在不同的地方;就让她讲吧,让她一遍又一遍地讲好了,正好让他组织起最简洁明智的回答,理顺她可悲的感情冲动。在一次佯攻和侧面冲击后,他深吸了一口气,站直身子;冥冥之中有什么告诉他,努比亚人会改变进攻的顺序,这一回他会先出三叉戟,后撒开渔网。“看好了,”利卡斯给他的妻子解说道,“我在阿普塔·尤利亚看他玩过这一手,他总能把对手耍得晕头转向。”马尔科不做防备,全然不顾自己已经进入对手渔网的攻击范围,径直向前猛扑,最后关头才举起盾牌,去抵挡从努比亚人手中闪电般抛出的那片亮闪闪的河流。他拦截住了渔网的
\newpage
边缘,可那三叉戟在下路刺来,马尔科的腿上喷射出鲜血,而他的剑太短,只是徒劳地架住了三叉戟的木柄,发出一声闷响。“你看,我说吧。”利卡斯大声喊道。总督全神贯注地看着那条受伤的腿,鲜血已经染红了金色胫甲;他几乎是有点怜悯地想,伊蕾妮会很想爱抚这条腿,寻找这腿上的力量和温度,她会发出呻吟,就像每次他把她紧紧搂住弄疼她的时候一样。今天晚上他就要把这话讲给她听,研究她的面孔,寻找那张完美面具的破绽,这样会很有趣,她肯定还会故作漠不关心,一装到底,就像现在,在这突如其来的结局刺激下,满场的平民都在兴奋地号叫,她却还能装出一副对这场决斗饶有兴致的斯文模样。“好运已经抛弃了他,”总督对伊蕾妮说,“我甚至有些自责,把他带来这个行省竞技场;看得出,他把他的一部分丢在罗马了。”“他身上剩下的东西就要丢在这里了,连同我下在他身上的赌注。”利卡斯笑道。“拜托,别这样,”罗兰说道,“我们明明今晚就可以见一面,却还这么在电话里说来说去,这太荒唐。我再说一遍,是索尼娅太着急了,我本来不想让你受这样一个打击。”蚂蚁停止了听写数字,让娜的声音清晰,从中听不出要哭的意思,罗兰本以为会面对她
\newpage
疾风暴雨般的指责,也预备好一套说辞,这一来倒很出乎他意料。“不想让我受打击?”让娜说,“撒谎,当然了,你无非就是想再多骗我一次。”罗兰叹了口气,放弃了原先准备好的答话,那样下去只会让这令人厌倦的谈话没完没了。“对不起,不过你要是一直这样,我就挂电话了。”他说,话里第一次有了点儿亲切的口吻。“要不我明天过去看看你,不管怎么说,我们都是文明人吧,真是活见鬼了。”远处,那蚂蚁又念叨开了:八百八十八。“你别来,”让娜说,她的话和数字混在一起听上去挺好玩的,你八百别八十八来,“你永远都不要再来了,罗兰。”那一套夸张的戏剧,可能会拿自杀相威胁,就像玛丽·若瑟那回一样无聊,像所有那些把一切都搞得悲悲切切的女人一样无聊。“别犯傻了,”罗兰劝道,“明天你就会想通的,这对我们两个人都好。”让娜沉默,蚂蚁这回念的全都是整数:一百,四百,一千。“那好,明天见。”罗兰说这话时正欣赏着索尼娅身上的裙子,她刚刚打开门,站在那儿,脸上半是疑问半是嘲笑。“她倒挺会抓紧时间给你打电话的。”索尼娅边说边放下手提包和一本杂志。“让娜,明天见。”罗兰重复了一遍。电话线里的沉默像一张绷得紧紧的弓
\newpage
,直到一个遥远的数字将其打断:九百零四。“我真是受够这些愚蠢的数字了!”罗兰用尽全身气力喊道,在把听筒从耳际拿开之前,他听见另一端传来咔嚓一声,那张弓射出了毫无敌意的一箭。马尔科无法动弹,他知道那张渔网随即就会把他裹住,而他无从躲避,他面对着那个努比亚巨人,过短的剑举在他伸直的臂膀的尽头。努比亚人两次把渔网张开又收起,找寻着最佳位置,他抡起渔网打旋,仿佛是想让全场观众继续吼叫、鼓动他干掉对手,他压低三叉戟,侧身发力强势一击。马尔科高举盾牌,径直向渔网扑了过去,他像一座塔楼撞碎在黑色躯体之上,短剑插进了什么东西里面,那东西发出一声嘶吼;沙土扑进了他
的嘴巴和双眼,渔网徒然地落在垂死的鱼身上。 

小猫对让娜的爱抚无动于衷,它无法感觉出让娜的手在微微颤抖,越来越凉。手指拂过它的皮毛又停下,忽然间一阵抽搐,接着抓了一下,小猫发出高傲的抗议;之后它仰面躺下,挥舞着爪子,每一次它这样让娜都会笑出声来,可这次它的期待落空了。让娜的手一动不动地搭在小猫旁边,只有一根手指好像还在寻找着小猫身上的体温,从皮毛上一划而过,停
\newpage
在了小猫身侧和滚到那里的药瓶之间。胃部被刺中的努比亚人惨叫着后退一步,在这最后的时刻,疼痛化作仇恨的火焰,全身正离他而去的气力都汇聚到一只臂膀之上,他把三叉戟深深扎进趴在地上的对手的后背。他倒在了马尔科身上,一阵抽搐使他滚到了一边;马尔科一条胳膊缓慢地动了动,身体被钉在沙土之
中,活像一只巨大的闪闪发光的虫子。 

“这可不常见,”总督转过身子朝伊蕾妮说道,“这么棒的两个角斗士同归于尽。我们真该庆幸自己有眼福,看到这么难得一见的场面。今天晚上我要把这件事写信告诉我的兄弟,这对身陷乏味婚姻的他
来说,也算点安慰。” 

伊蕾妮看见马尔科的胳膊动了一下,缓缓地,徒劳地,仿佛是想把扎进自己肾脏里的三叉戟拔出来。她想象着这会儿是总督赤着身子躺在竞技场上,也有一支三叉戟深深地没入他的身体,只剩木柄还在外面。可总督绝不会带着这最后的尊严动一动自己的胳膊;他只会大喊大叫,像只野兔一样蹬着双脚,向着群情激愤的观众请求宽恕。伊蕾妮扶着丈夫伸过来的
\newpage
手站起来,又一次顺从;那胳膊已经不动了,她现在唯一能做的就是面带微笑,用机智把自己保护起来。小猫似乎不喜欢让娜的静默,依然仰面躺着等待爱抚;然后,仿佛支在皮毛上的那根手指烦到了它,它发出一声不快的喵呜,翻身离开,漫不经心,已然困乏
了。 

“很抱歉我这个时间来,”索尼娅说,“我看见你的车停在门口,这诱惑太强烈了。她给你打电话了,是不是?”罗兰找着香烟。“这事儿你做得不对,”他说,“这种事通常是男人去做的,不管怎么说,我和让娜相处了两年多,她是个好姑娘。”“哦,可我觉得好玩,”索尼娅边说边给自己倒了一杯白兰地,“她总是那么天真幼稚,我一直都很受不了,就是她那个样子最让我恼火。我告诉你吧,她一开始还笑了,以为我是在逗她玩。”罗兰看了一眼电话机,他又想起了蚂蚁。让娜随时可能再打电话来,这会很尴尬,因为索尼娅已经在他身边坐了下来,爱抚着他的头发,同时还胡乱翻看着一本文学杂志,仿佛是在寻找哪一幅插图。“这事儿你做得不对。”罗兰重复道,把索尼娅搂到自己身边。“你是指我不该这个时
\newpage
候过来吗?”索尼娅笑着,顺从地让那双手笨拙地摸索,解开自己衣服最上面的拉链。紫色的纱巾罩住伊蕾妮双肩,她背对观众,等候着总督完成对公众的最后致意。欢呼的声浪当中混杂着人群开始挪动的声音,已经有人争先恐后地向出口挤去,想先一步到达下面的通道。伊蕾妮明白,此时一定会有几个奴隶正在把两具尸体拖走,她没有转过身来;她满意地想到总督已经接受了利卡斯的邀请,到他家的湖畔庄园共进晚餐,在那里,夜晚的空气会帮她忘记这里平民百姓臭烘烘的气味,忘记那最后的惨叫,忘记那只手臂是怎样缓缓地挪动,仿若爱抚这大地。对她来说遗忘并不难,尽管总督对种种往事心存芥蒂,还会经常旧事重提来折磨她;总有一天伊蕾妮会找到一种办法让他也永远忘掉这些事,而人们只会觉得他已经死了。“您会尝到我们家厨师的新花样,”利卡斯的妻子说,“他让我丈夫恢复了胃口,而且一到夜里……”利卡斯大笑起来,一面不断跟他的众多朋友打招呼,他在等总督结束致意,走向过道,可总督磨磨蹭蹭,继续观看场地上奴隶们用钩子搭住尸体拖走,好像很享受这场景。“我太幸福了。”索尼娅边说边把脸颊伏在昏昏欲睡的罗兰的胸口。“别说这样的话,”罗兰嘟
\newpage
囔了一句,“听起来像是客套。”“你不相信我?”索尼娅笑了。“我当然相信你。可这会儿别说这样的话。我们还是抽根烟吧。”他在小矮桌上摸索着,找到了香烟,他放了一支在索尼娅嘴唇中间,又把自己的凑过去一起点着。睡意沉沉,他们几乎没去看对方,罗兰晃了晃火柴,把它甩到小矮桌上,那里某处有只烟灰缸。索尼娅先睡着了,他慢慢地从她嘴边取下香烟,和自己的并在一起,扔在了小矮桌上,然后,他靠在索尼娅身旁,滑入了沉重而无梦的酣睡。烟灰缸旁,一条纱巾先被燎着了,没有燃起明火,而是慢慢地、一点一点地烧焦,又落在地毯上,旁边是一堆衣服和一杯白兰地。一群观众吵吵嚷嚷地挤在下面的台阶上;总督再一次致意,然后冲卫兵们做了一个手势,示意为他开路。最先发现不对的是利卡斯,他指向那古老帷幔最远处的一道,帷幔开始碎裂,火花雨点般落在慌慌张张涌向出口的人群当中。总督大声发出命令,推了伊蕾妮一把,伊蕾妮依然背对着他,一动不动。“快跑,下面的通道马上就要挤满人了。”利卡斯抢到妻子前面边跑边叫。伊蕾妮第一个闻见油燃烧的气味,是地下仓库着火了;在她身后,帷幔倾覆在争先恐后逃生的人们背上,过道本就很窄,这时
\newpage
已经挤成一团。成百上千的人跳到了竞技场地,想另寻生路,但滚滚的油烟很快把他们的身影吞没了,总督还没来得及躲进通往帝王包厢的过道,一绺布条便带着火头落在了他的身上。听见他的惊叫声,伊蕾妮转过身来,翘起两根手指,仪态万方地为他拿走了那烧焦的布条。“我们都逃不出去了,”她说,“下面已经挤成一团,像一群野兽。”索尼娅尖叫起来,竭力想挣脱那条在睡梦中把她搂得死死的燃烧的臂膀,她的第一声号叫同罗兰的混在了一起,后者正徒劳地挣扎着爬起,浓浓的黑烟呛住了他。他们放声呼救,但声音渐渐微弱了,这时消防车穿过满是好奇看客的街道全速赶到。“十楼,”队长说,“不太好救啊,
还刮着北风。上吧。” 



\end{document}
