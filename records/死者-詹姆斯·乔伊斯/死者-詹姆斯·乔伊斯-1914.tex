\documentclass{article}
\usepackage[utf8]{inputenc}
\usepackage{ctex}

\title{死者\footnote{Click to View:\url{https://web.archive.org/web/20230614132756/https://www.99csw.com/article/4501.htm}}}
\author{詹姆斯·乔伊斯}
\date{1914}

% \setCJKmainfont[BoldFont = Noto Sans CJK SC]{Noto Serif CJK SC}
% \setCJKsansfont{Noto Sans CJK SC}
% \setCJKfamilyfont{zhsong}{Noto Serif CJK SC}
% \setCJKfamilyfont{zhhei}{Noto Sans CJK SC}
% \setlength\parindent{0pt}

\begin{document}
\CJKfamily{zhkai}

\maketitle


\Large

看楼人的女儿莉莉简直是双脚离地在飞跑了,她刚刚把一位先生带进底层营业所后面的餐具间,帮他脱掉大衣,断断续续的前门门铃可又响起来了,她只得匆匆奔过空荡荡的过道,给另一位客人开门。幸亏不要她也伺候女客人。凯特小姐和朱莉娅小姐想到了这一层,把楼上的浴室改做女客们的化妆室了。凯特小姐和朱莉娅小姐现在正在那儿,聊着天,笑着,大惊小怪地没事儿瞎忙着,还轮番走到楼梯口,从扶手栏杆上向下张望,朝楼下对莉莉大声喊着,问她
是谁来了。 

这从来都是件大事情,莫坎家的几位小姐每年一次的舞会。她们所有的熟人都来参加,家庭的成员,家里的老朋友,朱莉娅唱诗班里的队员,凯特教过的一些已经长大成人的学生,甚至玛丽·简的学生有
\newpage
的也来参加。没有哪回不是尽欢而散的。就人们记忆所及,好多好多年了,这舞会一向是开得很成功的;自从她们的哥哥帕特去世,凯特和朱莉娅从斯托尼·巴特那幢房子里搬出来,带上玛丽·简,她们唯一的侄女儿,一块住在阿雪岛上这幢幽暗、冷落的房子里以来,一直是这样。她们从楼下做粮食生意的富勒姆先生手里租下了楼上一层,已经有足足三十个年头了。玛丽·简那时候还是个穿短衫裤的小丫头,如今已是家里的台柱子了。海丁顿街上的管风琴归她弹。她从专科学校毕业,还每年一度在老音乐厅的楼上开一次学生演奏会。她的好多学生都是金斯顿和达尔基一带上等人家的子女。她的姨妈们虽然老成那样了,也都在尽自己的一份力。朱莉娅,尽管已经两鬓斑白,仍然是“亚当与夏娃”唱诗班的第一女高音,凯特,因为身体太弱,不能多跑动,就在后屋那架老式方型大钢琴上给启蒙学生教音乐课。莉莉,看楼人的女儿,给她们做女仆的工作。虽然她们生活得简朴,她们主张要吃的好;样样都买顶好的:带梭形骨头的牛腰肉、三先令一磅的茶叶和上等的瓶装黑啤酒。莉莉照吩咐做事,极少有差错,所以她跟三位女主人处得挺好。她们都爱大惊小怪,如此而已。不过她们唯一不
\newpage

能忍受的是跟她们顶罪。 

当然喽,这样一个晚上,她们大惊小怪是有充分理由的。早就过了十点钟,可是加布里埃尔跟他妻子还不见影儿。此外,她们还非常害怕弗雷狄·马林斯可能喝醉了酒来的。她们怎么也不希望玛莉·简的哪个学生看见他醉醺醺的样子;而他要是这样子,有时还很难对付呢。弗雷狄·马林斯总是迟到,然而她们奇怪加布里埃尔会让什么事拖着呢:这就是为什么她们隔上两分钟便要走到楼梯扶手处,问莉莉加布里
埃尔或是弗雷狄来了没有。 

“噢,康罗伊先生,”莉莉为加布里埃尔开门时对他说,“凯特小姐和朱莉娅小姐还以为您不会来
了呢。晚上好,康罗伊太太。” 

“我保证她们会这么想的,”加布里埃尔说,“可是她们忘记了,我这位太太真要命,得花三个钟
头打扮自己呢。” 

他立在擦鞋垫上,把他套鞋上的雪往下蹭,这
\newpage
时莉莉把他妻子陪到楼梯口,喊了一声:“凯特小姐
,康罗伊太太来了。” 

凯特和朱莉娅马上蹒跚地从幽暗的楼梯上走下来0她俩都吻了加布里埃尔的妻子,说她一定给活活
冻坏了吧,又问加布里埃尔是否跟她一道来了。 

“我在这儿,跟邮件一样准时呢,凯特姨妈!
上楼吧,我这就来,”加布里埃尔在暗处大声说。 

三个女人说笑着往楼上女化妆室走去,他还在继续使劲儿地蹭他的脚。薄薄一层雪绕边儿盖在他大衣的肩头上,像条披肩似的;盖在他的套鞋上,像鞋头上的花纹似的;他咯吱咯吱地解开冻硬的粗呢大衣上的纽扣,这时一阵室外的芳香的寒气从他衣服的缝
隙和褶皱中散发出来。 


“又下雪了吗,康罗伊先生?”莉莉问。 

她领他走进餐具间,去帮他脱大衣。加布里埃尔听她称呼自己姓时发出的那三个音节,微微一笑,
\newpage
瞧了她一眼。她是个细长身材,正在抽条儿的姑娘,面色发白,头发是干草色。小房间里的煤气灯把她照得更苍白了。当她还是个小孩子,老是抱着个破布娃娃坐在楼梯最低一级上的时候,加布里埃尔已经认识
她了。 

“又下了,莉莉,”他回答,“我看得下一整
夜呢。” 

他抬头望望餐具间的天花板,楼上脚步的踢踏和拖曳震得天花板都在摇晃了,他听了一会儿钢琴声,然后又瞧瞧这个姑娘,她正在搁板的另一头小心地
把他的大衣叠好。 

“告诉我,莉莉,”他口气和蔼地说,“你现
在还上学吗?” 

“噢,不了,先生,”她回答,“我今年不上
学了,往后也不再上了。” 

“喔,那么,”加布里埃尔快活地说,“我看
\newpage
哪个好日子,我们该去参加你跟你那个年轻人的婚礼
了吧,嗯?” 

女孩回过头瞧他一眼,非常辛酸地说:“现在的男人都只会说废话,把你身上能骗走的东西全骗走
。” 

加布里埃尔脸红了,仿佛他觉得自己做错了事情似的,他眼睛不朝她看,把自己的套鞋甩脱下来,
一个劲儿地用他的厚手套擦着他的漆皮鞋。 

他是个壮实的、高高个儿的青年人。他双颊上红润的血色甚至向上延展到他的额际,在那儿泛作几片不成形状的淡红色;在他没有胡须的面庞上,一副眼镜屏挡着他一双灵敏的、永不宁静的眼睛,眼镜上光洁的镜片和铮亮的镀金框架也在永不宁静地闪耀着光辉。他那有光泽的黑头发从中间分开,又长又弯地梳向耳后,在帽子压出的一道纹路下轻微地卷曲着。
 

把皮鞋擦得发亮了,他便站直身子,把背心向
\newpage
下拉一拉,使他更贴身地罩在他丰满的躯体上。然后
他从口袋里迅速地掏出一枚硬币来。 

“喔,莉莉,”他说着,把钱塞进她手里,“
过圣诞节了,是吗?不过是……一点小意思……” 


他赶快向门外走去。 

“噢,不,先生!”女孩子大声说,跟他走过
来。“真的,先生,我不要。” 

“过圣诞节了!过圣诞节了!”加布里埃尔说着,一边几乎是小跑步地向楼梯走去,同时向她挥动
一只手,要她把钱收下。 

女孩见他已经走下楼梯了,便在他身后高声说
:“那么,谢谢您了,先生。” 

他在客厅门外等着这支华尔兹结束,听着衣裾从门边擦过和脚步在地板上拖动的声音。女孩子那句辛酸而意外的回话让他仍然心绪不宁。这句话让他显
\newpage
得抑郁,他把袖口拉拉平,把领结整一整,试图驱散这种气氛。然后他从背心口袋里掏出一张小纸片,看了看他为自己的演讲所列的提纲。他还拿不定主意要不要用罗伯特·勃朗宁的几行诗,因为他怕这超出了听他讲话的人们的知识水平。引几段他们能知道是莎士比亚或是歌曲集上的字句会更好些。这些人的鞋跟的粗俗的磕碰声和鞋底在地板上的拖曳声使他想起,他们的文化等级跟他的不同。对他们引用他们所不能懂的诗,只能使自己显得滑稽。他们会想,他在炫耀自己高人一等的教育。他跟他们打交道就会失败,就像他在楼下餐具间里跟那个姑娘打交道失败一样。他把调子定错了。他整个演讲从头到尾都错了,是个彻
底的失败。 

这时候,他的姨妈们和他的妻子从女客化妆室出来了。他的姨妈是两位身材矮小,衣着朴素的老妇人。朱莉娅姨妈大约高上一英寸左右。她的头发向下披着盖住耳朵尖,是灰白色的;她那张脸宽大松弛,也是灰白色的,但是脸上有几处颜色比较深。虽然她体格结实,立得端端正正,她迟钝的眼睛和分开的嘴唇使她看起来是个不知道自己身在何处,也不知道该
\newpage
往何处去的女人。凯特姨妈比较有生气。她的面色比她妹妹的健康,脸上尽是皱纹和褶子,好像一只干缩了的红苹果,她的头发也用同样老式的样子编起来,
还没有失去成熟的胡桃颜色。 

她俩都真诚地吻了加布里埃尔。他是她们心爱的侄子,死去的姐姐爱伦的儿子,她嫁的是港口船坞
公司的特·捷·康罗伊。 

“格莉塔给我说,你们今儿晚上不打算坐出租
马车回蒙克斯顿了,加布里埃尔。”凯特姨妈说。 

“不了,”加布里埃尔说,转身向她的妻子,“我们去年可受够了,是吗?你记不记得,凯特姨妈,格莉塔给冻成什么样子了?马车窗子一路上咯咯咯地响,车过梅里翁之后,东风就往车里灌,可真够呛
的。格莉塔害了一次重感冒。” 

凯特姨妈一本正经地皱着眉,他说每句话她都
点一次头。 

\newpage

“非常对,加布里埃尔,非常对,”她说。“
你尽可能仔细总是不错的。” 

“可是要说格莉塔她呀,”加布里埃尔说,“
要是依着她,她准会冒着雪走回家去的。” 


康罗伊太太笑了。 

“您别听他的,凯特姨妈,”她说,“他可真烦死人了,什么为了汤姆的眼睛晚上要用绿灯罩呀,要让他练哑铃呀,强迫伊娃吃麦片粥呀。可怜的孩子!她简直见了麦片粥就恨!……哦,可你们怎么也猜
不出,他现在逼我穿些什么!” 

她发出一串响亮的笑声,对她丈夫瞧了瞧,他爱慕和幸福的眼光正从她的衣服上移到她面孔和头发上。两位姨妈也亲切地笑着,因为加布里埃尔的婆婆
妈妈的作风,向来是她们的笑柄。 

“套鞋!”康罗伊太太说,“这是最新的玩意儿。只要路上有点潮湿,我就得穿上套鞋。甚至今儿
\newpage
晚上,他也要我穿上,可是我不肯。下次他要给我买
的,该是一套潜水服了。” 

加布里埃尔神经质地笑着,接着好像要让自己安心似的的拍拍领结,这时凯特姨妈笑得都直不起腰了,这个笑话让她非常开心。朱莉娅姨妈脸上的笑容不久便消逝了,她闷闷不乐的眼神转向她侄儿的脸庞。停了一会儿,她问:“套鞋是什么呀,加布里埃尔
?” 

“套鞋吗,朱莉娅!”她姐姐惊讶地说。“天哪,你难道不知道套鞋是什么?你把它穿在你……穿
在你的靴子上,格莉塔,是吗?” 

“是的,”康罗伊太太说,“用古塔胶做的。我们俩现在都各有一双了。加布里埃尔说大陆上人人
都穿的。” 

加布里埃尔皱皱眉头说,似乎稍微有点生气:“这没有什么奇怪的嘛,可是格莉塔认为非常好笑,她说,套鞋这个词儿让她想起克瑞斯蒂剧团①的演员
\newpage


“可是,告诉我,加布里埃尔,”凯特姨妈思路敏捷、措词得体地说,“你当然找好房间了,格莉
塔刚刚说……” 

“噢,房间没问题,”加布里埃尔回答。“我
在格列沙姆订好一间。” 

“说真的,”凯特姨妈说,“办得好极了。还
有孩子们哪,格莉塔,你不为他们担心吗?” 

“哦,就一个晚上嘛,”康罗伊太太说。“再
说,贝茜会照顾好他们的。” 

“说真的,”凯特姨妈又说了,“有个像她那样的保姆该多称心,一个你能靠得住的人!瞧那个莉莉,我敢说,我不知道这阵子她怎么啦。她简直跟从
前完全不一样了。” 

加布里埃尔正想就这一点向姨妈问几个问题,然而她突然停住话,目送她妹妹走开去,朱莉娅晃晃
\newpage
悠悠地往楼下走,正从楼梯扶手上伸长脖子朝下望。

“啊,我问你,”她几乎是烦躁地说,“朱莉娅上哪儿去了?朱莉娅!朱莉娅!你上哪儿去呀?”

朱莉娅已经下了一段楼梯了,又走回来,态度
温顺地报告说:“弗雷狄来了。” 

同时传来一阵掌声和钢琴手的最后的装饰性乐段,说明华尔兹舞结束了。客厅门从里向外打开,几对舞伴走了出来。凯特姨妈急忙把加布里埃尔拉向一边,俯在他耳边悄悄说:“溜下楼去,加布里埃尔,求求你,看他对不对头,要是喝醉了,就别让他上楼
来。我敢说他是喝醉了的。我敢说他是的。” 

加布里埃尔走到楼梯旁,从扶手栏杆上往下倾听。他能听见两个人在餐具间谈话的声音。然而他听出了弗雷狄·马林斯的笑声。他脚步很重地走下楼去
。 

“真让人宽心,”凯特姨妈对康罗伊太太说,
\newpage
“有加布里埃尔在这儿。有他在这儿,我总是觉着安心点儿……朱莉娅,瞧,戴丽小姐跟鲍尔小姐得吃点儿点心。谢谢您弹得漂亮的华尔兹舞曲,戴丽小姐。
真叫人觉着愉快。” 

一个高高的,面容干瘪的人,生一撮硬挺的灰白小胡髭,皮肤黝黑,正跟他的舞伴打客厅出来从旁边走过,说道:“我们也来点儿点心好吗,莫坎小姐

“朱莉娅,”凯特姨妈当即说,“这是布朗先生和弗朗小姐。朱莉娅,陪他们跟戴丽小姐和鲍尔小
姐一道去。” 

“我是个讨女士们喜欢的人,”布朗先生说,嘴巴噘得小胡子都翘直了,把满脸的皱纹都笑出来了。“您知道,莫坎小姐,她们那么喜欢我的原因是…
…” 

他没说完这句话,马上就陪三位女客往后屋去了,因为他见凯特姨妈听不清他说话。后屋正当中摆了两张拼在一起的方桌,朱莉娅姨妈正跟看楼人一块
\newpage
儿把一张大台布拉直,铺在桌子上。餐具柜上整齐地排列着杯盘碗碟和一束束的刀叉和汤匙。方型大钢琴合上盖子,顶上也当餐具柜用,放着各种菜肴和甜食。屋角一只小些的餐具柜前有两个年轻人站着,在喝
苦味蛇麻子啤酒。 

布朗先生把他受托照管的女士们引到那里,开玩笑地请她们三位都尝点女宾用的混和甜饮料,又热,又浓,又甜。她们说她们从没喝过烈性的饮料,他便为她们开了三瓶柠檬水。然后他请年轻人当中的一位让一让,拿起有玻璃塞的细颈酒瓶,给自己满满儿斟了一杯威士忌。当他呷一口酒品品味道的时候,两
个年轻人恭敬地看着他。 

“上帝帮助我,”他笑眯眯地说,“正是医生
吩咐我喝的。” 

他干瘪的面庞上展出一副比较开朗的笑容,三位女士对他的诙谐报以音乐般的笑声,笑得前后摇晃着身子,肩膀激动地抽搐着。其中最勇敢的一位说:“噢,布朗先生呀,我敢说医生从来不会这样吩咐的
\newpage


布朗先生把他的威士忌又啜了一口,侧身做了个鬼脸,说道:“啊,你们瞧,我就是那位大名鼎鼎的卡西迪太太,据说她讲过:‘喂,玛丽·格兰姆斯,假若我不喝,您就强迫我喝,因为我感觉我需要喝
。’” 

他发热的面孔向前探得有点儿太亲热了,他又装出一副非常俗的都柏林腔调,所以这些年轻女士们,出于同一种本能,都一声不响听着他。弗朗小姐,她是玛丽·简的一个学生,问戴丽小姐她弹的那支华尔兹舞曲叫什么名字;布朗先生发觉人家不注意他了,便立即转向两位青年,他们比她们更能赏识他一些

一位红面孔的年轻女人,穿一身蓝紫色衣裳,走进屋里来,激动地拍着说大声说:“跳四对舞了!
跳四对舞了!” 


凯特姨妈紧跟她进来,大声说: 


\newpage

“两位先生,三位女士,玛丽·简!” 

“哦,这儿有伯金先生和克里根先生,”玛丽·简说,“克里根先生,您和鲍尔小姐跳舞好吗?弗朗小姐,让我给您找位舞伴吧,伯金先生。哦,现在
都好了。” 


“三位女士,玛丽·简,”凯特姨妈说。 

两位年轻人恭请三位女士跳舞,玛丽·简转向
戴丽小姐。 

“噢,戴丽小姐,您真是太好、太好了,已经给两场舞伴奏过,可是我们今晚上的确是太缺少女舞
伴了。” 


“我一点儿不在意呢,莫坎小姐。” 

“不过我有一位出色的舞伴介绍给您,巴特尔·达西先生,那位男高音。回头我还要请他唱一个。
整个都柏林都在入迷地谈论他呢。” 

\newpage

“漂亮的嗓子,漂亮的嗓子!”凯特姨妈说。

钢琴已经两次弹起第一节舞的序曲,玛丽·简便把她请到的几位急忙带出这间屋。他们刚出去,朱
莉娅姨妈就慢腾腾地踱进来,向身后望着什么。 

“怎么回事儿,朱莉娅?”凯特姨妈急切地问
。“是谁呀?” 

朱莉娅正拿进一卷餐巾来,转过身向着她姐姐简单地说,仿佛这个问题让她出乎意外似的:“是弗
雷狄,凯特,加布里埃尔陪着他呢。” 

其实,已经看到加布里埃尔就在她身后了,正引着弗雷狄·马林斯跨过楼梯口的平台。后者是一个约莫四十岁左右的年轻人,身段和体格都和加布里埃尔相似,两个肩头很圆。他的面孔肥胖而苍白,只有厚厚的两只向下挂着的耳垂上和两扇鼻翼上才有点血色。他相貌粗俗,一只塌鼻子,额头凸出又向后斜缩回去,嘴唇是肿胀而噘出的。他的眼皮厚重的眼睛和稀疏的头发的凌乱样子,显出一副没睡醒的神气。他
\newpage
在楼梯上给加布里埃尔讲一个故事,刚讲到关键的地方,他正在为此开心地笑着,同时用他左手拳头的指
关节来回擦着他的左眼。 


“晚上好,弗雷狄,”朱莉娅姨妈说。 

弗雷狄·马林斯向几位莫坎小姐说了声晚上好,态度好像很简慢,因为他一向说起话来是噎声噎气的,随后,看见布朗先生立在餐具柜边向他裂开嘴笑,便脚步不稳地穿过房间,重新开始低声讲起他刚刚
告诉过加布里埃尔的故事来。 

“他不是那么糟糕吧,是吗?”凯特姨妈对加
布里埃尔说。 

加布里埃尔皱着眉头,然而他立即舒展开来,
回答说:“哦,不是,几乎看不出。” 

“他不是个极糟的家伙吗?”她说,“他可怜的妈妈在除夕晚上要他起过誓的。不过,走吧,加布

\newpage
里埃尔,咱们去客厅吧。” 

在她跟加布里埃尔一块走出这间屋之前,她皱皱眉头,来回挥动食指向布朗先生打暗号,提醒他。布朗先生点点头作答,等她走了,他便对弗雷狄·马林斯说:“那么,特狄,让我给您满满来一杯柠檬水
,给您提提精神吧。” 

弗雷狄·马林斯的故事快要讲到高潮,不耐烦地挥挥手,不听他的,然而布朗先生先是提醒弗雷狄·马林斯注意他衣服有个地方不整齐,然后倒上满满一杯柠檬水递给他。弗雷狄·马林斯用左手机械地接下玻璃杯,因为右手正忙于机械地调整着他的衣服。布朗先生又一次笑得满脸皱纹,给他自己斟了一杯威士忌,这时,弗雷狄·马林斯的故事正要讲到高潮,突然爆发出一阵高声的咳嗽般的大笑,他把还没喝过的、满得溢出来的杯子放下,开始用他左手拳头的指关节来回擦着左眼睛,尽管他还在发出阵阵的笑声,
还极力要把他最后一段话再重复一遍。 

玛丽·简给客厅里寂静的听众演奏她学院式的曲子,其中满是速奏和困难的乐段,加布里埃尔不能
\newpage
听进去。他喜欢音乐,但是她正弹的这首曲子他觉得没有旋律,他并且怀疑其他听众是否会觉得有什么旋律,虽然是他们请求玛丽·简弹点儿什么的。四个年轻人从吃点心的房间出来,听到钢琴声便立在门边,几分钟后又两个两个地走开了。似乎只有两个人能够领略这音乐,一个是玛丽·简自己,她的两只手在键盘上飞快地移动,或在停顿时从键盘上拎起来,好像一个女术士在诅咒的瞬间里的两只手,另一个是凯特
姨妈,她立在玛丽·简肘边为她翻乐谱。 

涂满蜂蜡的地板在庞大的枝型吊灯照耀下闪闪发光,把加布里埃尔的眼睛刺激得难受,他便向钢琴上方的墙壁望去。那儿挂着一幅画,画的是《罗密欧与朱丽叶》中阳台上一场,旁边是一副关于伦敦古堡中两王子被害的画②,这是朱莉娅姨妈年轻时用红、蓝、褐三色绒线绣的。大概在她们小时候上的学校里,这类活计要教一学年。他母亲曾给他做过一件紫色波纹毛葛背心当生日礼物,上边有些小狐狸头花样,褐色段子衬里,还有圆形的深紫红色扣子。真奇怪,他母亲居然没有音乐才能,虽然凯特姨妈总是称她作莫坎家的智囊。她和朱莉娅两人一直好像为她们这位
\newpage
贵妇般的姐姐感到有些骄傲。她的照片摆在穿衣镜前。她膝头上放一本打开的书,正在把书里的什么指给康斯坦丁看,他穿一身海军服躺在她脚边。她儿子们的名字都是她起的,因为她对于家庭生活中的尊严是非常敏感的。多亏她,康斯坦丁现在在巴尔不里干③当高级牧师,也多亏她,加布里埃尔自己在皇家大学取得了学位。当他回想起她绷着脸反对他婚姻的情景时,他脸上掠过一层阴影。她那时用过的几个轻蔑字句至今隐隐在他的记忆中引起怨恨;有一回她谈到格莉塔,说她像乡下人似的做作,而这对格莉塔是完全不真实的。她最后在蒙克斯顿他们家里长期卧病的期
间,全都是格莉塔伺候她的。 

他知道玛丽·简一定是快要弹完她的曲子了,因为她又重新弹起了开头时的旋律,每一小节后面都来一段溜音节的速奏,当他在等待结束时,那种怨恨情绪在他心里渐渐消逝了。乐曲以一段高音部八度颤音和一段结尾的低音部八度音阶而告终。一阵热烈的掌声向玛丽·简表示祝贺,她红着脸,神经紧张地收起乐谱,从屋里逃出去。最热烈的掌声来自门口那四个年轻人,他们在曲子开始时走开到吃点心的房间里
\newpage

去了,而当琴声停止时又回来了。 

跳四对舞的人都安排定了。加布里埃尔发现给他安排的舞伴是艾弗丝小姐。她是个为人坦率的、健谈的年轻小姐,脸上有雀斑,一双棕黄色的眼睛突出来。她没有穿低领的紧身胸衣,领子正面别着一枚大
大的胸针,上面刻有爱尔兰文题铭和格言。 

当他们站好位置时,她突如其来地说:“我有
件事情要想跟您问明白。” 


“跟我?”加布里埃尔说。 


她严肃地点点头。 

“什么事情?”加布里埃尔对她一本正经的态
度微微一笑。 

“加·康这个人是谁?”艾弗丝小姐回答。转
过眼睛瞧着他。 

\newpage

加布里埃尔脸红了,正打算把眉毛一拧,装作好像他不了解似的,这时她单刀直入地说:“噢,天真无邪的小姑娘!我发现您在给《每日快报》写文章
呢。嘿,您就不觉得害臊吗?” 

“我干嘛要害臊呢?”加布里埃尔问,眨眨眼
睛,试图笑一笑。 

“我可为您害臊呢,”艾弗丝小姐直率地说。“您怎么会给报纸写那种东西。我从前没想到,您是个西布立吞人。”④加布里埃尔脸上露出一种迷惑的表情。的确,他每星期三为《每日快报》文学评论栏写一篇文章,人家为此付给他十五个先令。但这绝不会使他变成一个西布立吞人。比起那张数目小得可怜的支票来,他对收到的那些送来让他评论的书更欢迎。他爱抚摸新出版的书封面,翻翻其中的书页,差不多每天当他在学院里的教学工作结束后,他习惯于去沿码头一带那些旧书店逛逛,去巴切勒路的希基书店,去阿斯顿码头上的韦布书店或梅西书店,或是去附近一条小街道上的奥克洛希西书店。他不知道怎样对付她的指责。他想说,文学是超政治的。然而,他们
\newpage
是多年的朋友了,他们的经历是彼此相似的,先是读大学,后来当教师:他不能冒险对她说一句大话。他继续眨巴眼睛,试图显出笑容,而且笨拙地喃喃说,
他认为写书评同政治不相干。 

轮到他俩转到对面去的时候,他还是不知所措和漫不经心。艾弗丝小姐热情地一把抓紧他的手,又用温柔而友好的口气说:“当然,我不过是开开玩笑
。来吧。咱们该过去了。” 

等他俩又到了一块儿,她谈起大学的问题,于是加布里埃尔感到自在多了。她的一位朋友把他评勃朗宁诗歌的文章拿给她看。她就是这样发现这个秘密的:但是她非常喜欢这篇评论。后来她突然说:“噢,康罗伊先生,您今年夏天到阿兰岛⑤来做次短途旅行好吗?我们要在那儿住整整一个月。去大西洋里呆一呆可真美呢。您一定要来。克兰西先生要来的,还有基尔肯尼和凯斯林·卡尼。格莉塔也准会觉得美极
了,如果她来的话。她是康诺特人⑥吧,是吗?” 

“她老家在那儿,”加布里埃尔简略地回答。
\newpage


“可是您回来的,是吗?”艾弗丝小姐说着,
用她的一只温热的手热切地按住他的肩膀。 

“事实是这样,”加布里埃尔说,“我刚安排
了要上……” 


“上哪儿?”艾弗丝小姐问道。 

“啊,您知道,我每年都跟几个人出去兜一圈
,这样可以……” 


“可是上哪儿呢?”艾弗丝小姐问。 

“啊,我们通常是去法国,或者是比利时,或
者也许是德国,”加布里埃尔尴尬地说。 

“您为什么要去法国和比利时呢,”艾弗丝小
姐说,“而不去您自己的土地上看看呢?” 

“啊,”加布里埃尔说,“一部分是为了能跟
\newpage

那几种语言保持接触,一部分是为了换换空气。” 

“难道您就没有自己的语言——爱尔兰语,需
要保持接触吗?”艾弗丝小姐问。 

“啊,”加布里埃尔说,“要说起这个,您知
道,爱尔兰语不是我的语言。” 

他们两旁的人都转过来倾听这场盘问了。加布里埃尔紧张地左边望望,右边望望,他已经被折磨得额头上泛起红晕,力图在这种情况下保持自己的好情
绪。 

“您难道没有自己的土地可以去看看吗?”艾弗丝小姐接着说,“您对它一无所知的土地,您自己
的人民,您自己的祖国?” 

“噢,跟您说真话吧,”加布里埃尔突然顶撞
她说,“我的祖国已经让我厌烦了,厌烦了!” 


\newpage

“为什么?”艾弗丝小姐问。 

加布里埃尔没有回答,因为他这句顶撞话是他
自己激动了。 


“为什么?”艾弗丝小姐又问一次。 

他俩得一块去看看,再说,既然他也没有回答她,艾弗丝小姐便兴奋地说:“当然咯,您没法回答

加布里埃尔试图掩饰他的激动,就非常卖力地跳舞。他避开她的眼光,因为他见她脸上有一种愠怒的表情。然而当大家连成一串,而他又挨着她的时候,他惊奇地感到他的手被紧紧地握着。她从眉毛下古怪地望了他一会儿,直望到他微微一笑。然后,正当排成一串的人要重新散开时,她踮起脚尖,凑近他耳
朵悄声说:“西布立吞人!” 

四对舞跳完了,加布里埃尔走开去,来到远处一个屋角里弗林斯·马林斯的母亲在那儿坐着。她是一位矮胖、虚弱的白头发老太太。她的嗓音跟她儿子的一样,有点儿发噎,所以她稍微有些口吃。人家已
\newpage
经告诉她弗雷狄来了,说他差不多是完全正常的加布里埃尔问她渡海峡时情况怎样。她跟她出嫁的女儿住在格拉斯哥,每年来都柏林玩一趟。她温和地回答说,她渡海峡时平稳极了,船长对她非常照顾。她还谈起她的女儿在格拉斯哥住的房子多漂亮,谈起他们那儿所有的朋友们。当她在唠唠叨叨地说的时候,加布里埃尔在力图把他和艾弗丝小姐的一场不愉快的插曲从头脑里清除掉。这个女孩,或者说女人,不管她是什么吧,当然是个热心人,可是说话做事总得看个时候才对。也许他不该像那么样来回答她。可是她没权利当众叫他西布立吞人呀,哪怕是开玩笑吧。她是想让他在人们面前出丑,她当众诘难他,还用她一双家
兔似的眼睛瞪着他。 

他看见他妻子正从一双双华尔兹舞伴中间向他走来。她走到他身边,她对着他的耳朵说:“加布里埃尔,凯特姨妈想知道,是不是还像往年一样由你来
切鹅肉。戴丽小姐切火腿,我来切布丁。” 


“好的,”加布里埃尔说。 

\newpage

“这场华尔兹以结束,她就先把年轻客人送过
去,这样餐桌旁边就只是我们了。” 


“你跳舞了吗?”加布里埃尔问。 

“当然跳了。你没看见我吗?你跟莫莉·艾弗
丝俩嚷嚷些什么?” 


“没嚷嚷,怎么?她说我嚷嚷了?” 

“好像是的。我在想法儿让那位达西先生唱歌
。他满以为自己了不起呢,我觉得。” 

“没嚷嚷过,”加布里埃尔不愉快地说,“只
是她要我去爱尔兰西部玩一趟,我说我不去。” 


她妻子兴奋地一拍手,轻轻一跳。 

“哦,去呀,加布里埃尔,”她喊着说。“我
真想再看看高尔韦呢。” 

\newpage

“你要喜欢你就去,”加布里埃尔冷冷地说。

她瞧了他一会儿,就转向马林斯太太说:“您
瞧这个丈夫有多好!马林斯太太。” 

她穿过房间回到原处去了,马林斯太太并没在意人家打断她的话,接着对加布里埃尔谈苏格兰有什么美丽的去处和美丽的风景。她女婿每年都带她们去湖泊区游览,她们每次都钓鱼。她女婿是个钓鱼的能手。一天他捉到一条美丽的大鱼,旅馆的主人还给他
们烧好,当菜吃呢。 

加布里埃尔几乎听不见她说些什么。马上就要用晚餐了,他又开始想他的演讲和引文。当他看见弗雷狄·马林斯穿过屋子走来见他的母亲,加布里埃尔就从椅子上站起来,让他坐,自己退到窗口的斜墙旁。这间屋已经收拾干净,从后屋里传来盘子和刀叉的磕碰声。留在客厅里的人看来也不想再跳舞了,聚成小堆在悄悄交谈。加布里埃尔用热乎乎、颤巍巍的手指轻轻弹着冰冷的窗玻璃。外面该有多冷啊!假如一个人出去,先沿着河岸,再穿过公园散散步,该多舒
\newpage
服!树枝上一定覆盖着雪花,威灵顿⑦纪念碑上面一定堆成了一顶明亮的帽子。要是在那儿,要比在晚餐
桌旁舒服多少啊! 

他匆匆温习了一下他的讲演的提纲:爱尔兰人的殷勤好客、悲哀的回忆、赐人以美丽和快乐的三女神、帕里斯⑧、所引的勃朗宁的诗句。他自言自语地说了一遍他在评论中写过的句子:“你觉得正在听一段扰人心绪的音乐。”艾弗丝小姐赞扬过这篇评论。她是真心的吗?在她那一套宣传后边,是不是真正有她自己的生活?这个晚上之前,他们之间不曾有过什么敌意。一想到她会在晚餐桌旁,当他发言的时候,用她那批评和嘲弄的眼光朝上望着他,他就不安。也许她看到他演讲失败,不会感到惋惜吧。一个想法出现在他脑子里,这给了他勇气。他会暗暗提到凯特姨妈和朱莉娅姨妈说:“女士们,先生们,我们中间现在正处于衰退的一代人可能有缺点,但是就我来说,我认为他们是有某些优秀品质的,像殷勤好客、幽默和慈爱,而这些品质依我看来,正是在我们周围成长着的、非常严肃、受过太多教育的新的一代人所缺少的。”好极了,这段话是说给艾弗丝小姐听的。他的
\newpage
姨妈们只不过是两个没有学识的老太太,有什么可关
心的? 

房间里的一阵低语声吸引了他的注意。布朗先生满带骑士风度地陪着朱莉娅姨妈从房门口走来,她倚在他的手臂上,微笑着,低垂着头。一阵不争气的噼里啪啦的掌声,一直送她来到钢琴面前,玛丽·简在琴凳上坐稳后,朱莉娅姨妈就不再微笑,半转过身子以便使她的声音能清楚地投进房间,这是掌声才渐渐平息下来。加布里埃尔听出了那个序曲。她嗓子在音调上是有力而又清晰的,精神十足地配合着一段段使曲调华丽的速奏。虽然她唱得很快,却甚至连一个最小的装饰音也没漏掉。倾听着歌声,不看歌唱者的面容,就能感受并且分享迅疾而可靠的灵感引起的激情。加布里埃尔和其他人一块儿在歌声终止时大声地鼓掌,从看不见的晚餐桌旁也传来了响亮的掌声。掌声听来是那样真诚,以致当朱莉娅姨妈俯身把封面上有她名字的第一个字母的旧皮面歌本放回乐谱架上时,一抹微微的红晕泛上了她的脸颊。弗雷狄·马林斯斜着脑袋好听得更清楚些,人家都停住了,他还在大声鼓掌,并且热烈地对他母亲谈论着,他母亲则庄重
\newpage
地、慢悠悠地点着头表示默许。最后,等他没法再鼓掌了,他便突然站起身来,匆匆穿过房间走到朱莉娅姨妈面前,双手抓住她的胳膊,摇着,不只是因为太激动了,还是因为他嗓子里的噎声太多,他说不出话
来。 

“我刚才还在对我母亲说,”他说,“我从没听见您唱得这么好,从没有听见过。没有,我从没听见您的嗓子像今天晚上这样好。好!现在您信吗?是真的。我敢用名誉担保,是真的。我从没听见您的嗓子那么清亮,那么……那么优美和清亮,从没听见过

朱莉娅姨妈把自己的手从他手中抽回来,大方地笑了笑,轻轻说了些不敢当的话。布朗先生把手向她伸过去,手心摊开,用一种演出主持人向听众介绍一个天才演员的架势对近旁的人说:“朱莉娅·莫坎
小姐,我最新的发现!” 

他正在自顾自地大笑,弗雷狄·马林斯转身向他,说道:“好了,布朗,你如果认真去发现,还可能发现你的发明并不高明。我所能说的仅仅是,打我
\newpage
到这儿来,我就从没听见她唱得有一半这么好。这是
千真万确的话。” 

“我也没听见过,”布朗说,“我认为她的嗓
子大有进步。” 

朱莉娅姨妈耸了耸肩,温顺而自傲地说:“三
十年前,跟一般嗓子比,我的嗓子并不坏。” 

“我常对朱莉娅说,”凯特姨妈断然地说,“在那个合唱队里,人家简直就不把她当回事儿。可是
她从来不肯听我的。” 

她转过身来好像在求助于其他人的高见,帮她来对付一个倔强的孩子似的,这时,朱莉娅姨妈双目
朝前凝视,脸上隐隐显出一种缅怀往昔的笑容。 

“不啊,”凯特姨妈接着说,“她谁的话也不听从,白天黑夜,黑夜白天地在那个唱诗班里给人家苦干。圣诞节早晨六点钟就去唱!都是为了什么?”

\newpage

“啊,难道不是为了上帝的荣耀吗,凯特姨妈
?”玛丽·简在琴凳上转了个身,微笑着问道。 

“上帝的荣耀我全知道,玛丽·简,可是我认为,把唱诗班里苦了一辈子的女人们都赶走,让一群妄自尊大的小男孩子骑在她们头顶上,对于教皇来说,根本不是件荣耀的事情。我想假如教皇那样做了,那是为了教会的好处。可那是不公平的,玛丽·简,
那是不对的。” 

她说得激动起来,还想再说下去,为她的妹妹争几句,因为这是一个让她伤心的话题,但玛丽·简
见所有跳舞的人都回来了,便和解地把话打断。 

“哎,凯特姨妈,你是在惹布朗先生生气呢,
他的宗教信仰跟您的不同。” 

凯特姨妈转向布朗先生,他听见人家提到自己的宗教,正在裂开嘴笑,凯特姨妈连忙说:“噢,我并不怀疑教皇做得对。我不过是个傻老太婆,我也不敢这样做,不过还有日常的礼貌和感谢这些人人知道
\newpage
的事情呀。要是我处在朱莉娅的地位上,我就会面对
面地向那个希利神父说……” 

“再说,凯特姨妈,”玛丽·简说,“我们大
家真是都饿了,我们一饿就都好吵架。” 

“我们渴了也好吵架呢,”布朗先生添上一句
说。 

“所以我们最好去吃饭,”玛丽·简说,“以
后再来结束这场讨论吧。” 

在客厅门外的过道上,加布里埃尔发现他的妻子正在设法说服艾弗丝小姐留下来吃饭。但是艾弗丝小姐已经戴上帽子,正在扣斗篷扣子,不肯留下来。她一点儿都不觉得饿,并且她已经超过了她该呆的时
间。 

“不过十分钟嘛,莫莉,”康罗伊太太说,“
不会耽误你事儿的。” 

\newpage

“吃一点嘛,”玛丽·简说。“跳了那么多的
舞。” 


“我真是不能再呆了,”艾弗丝小姐说。 

“我怕你玩得一点儿也不开心呢,”玛丽·简
无奈地说。 

“非常开心呢,我想你保证,”艾弗丝小姐说
,“不过你得让我现在就走才行。” 


“可你怎么回家呢?”康罗伊太太说。 


“噢,沿码头走几步就到了。” 


加布里埃尔犹豫了一会儿,说: 

“假如你愿意,艾弗丝小姐,我送您回家吧。
假如您真是非走不可的话。” 


\newpage

但是艾弗丝小姐突然从他们身边走开了。 

“我不听这个,”她嚷道。“看老天爷份上,吃你们的晚饭去,别管我了。我还好好儿的,能照管
我自己。” 

“唉,你真是个怪里怪气的姑娘,莫莉,”康
罗伊太太率直地说。 

“晚安,亲爱的,”艾弗丝小姐笑着嚷了一句
,奔下楼梯。 

玛丽·简凝视着她的背影,脸上显出阴郁、迷惑的表情,康罗伊太太靠在扶梯把手上听过道里响起开门声。加布里埃尔在问自己,他是不是她突然离去的原因。但是她看上去并没有不高兴——她一路笑着
走去的嘛。他从楼梯口上茫然望下去。 

这时,凯特姨妈跌跌撞撞地从开晚餐的房间里
出来,几乎是绝望地绞着两只手。 

“加布里埃尔在哪儿?”她嚷道。“加布里埃
\newpage
尔到底在哪儿呀?大家全等在那儿,虚位以待呢,没
人来切鹅了!” 

“我在这儿呢,凯特姨妈!”加布里埃尔猛地活跃起来,喊着:“需要的话,我可以切整整一群鹅

一直棕黄色的肥鹅摆在桌子的一端,另一端:在一个装饰着欧芹细枝的皱纹纸垫上,摆着一只大火腿,已经剥了皮,撒满了干面包粉,胫骨处套着一个精美的纸花边,火腿旁边是一块五香牛腿肉。在这相对的两端之间是平行的两列其他菜肴:高高两堆果子冻,一红一黄;一只浅底盘满盛着大块的牛奶冻和红色果酱,一个绿色带梗状柄的叶形大盘,里边是一枝枝紫色葡萄干和去皮的杏子,另一只同样的盘子里,是堆成整齐的长方形的士麦那⑨无花果,一盘上面撒有豆蔻沫的牛奶蛋糊,满满一小盆包着金银纸的巧克力和糖果,一只玻璃花瓶里插着一些长长的芹菜茎。桌子正中立着两只矮胖的老式雕花细颈玻璃瓶,一只盛着白葡萄酒,另一只盛着深色的雪利酒,它们像卫兵似的守卫着一只水果盘,盘子托起尖尖的一堆橘子和美洲苹果。在盖拢的方形钢琴上有一只还没上桌的
\newpage
用大黄盘盛着的布丁,它后边是三排烈性黑啤酒、淡啤酒和矿泉水,像士兵一样依照它们各自制服的颜色分别排列成队。前两排是黑色的,贴着咖啡和红色标签,第三排也是最短的一排是白色的,瓶上横系着绿
色的饰带。 

加布里埃尔大模大样地坐在首席上,看了看刀锋,便把叉子稳稳地插进了鹅身上。这会儿他觉得相当自在,因为他是个运刀的能手,顶喜欢坐在丰盛餐
桌的首席上。 

“弗朗小姐,给您来点什么?”他问,“一个
翅膀呢,还是一片脯子肉?” 


“一小片脯子肉就行了。” 


“希金斯小姐,您呢?” 


“随您便吧,康罗伊先生。” 

加布里埃尔和戴丽小姐把盛着鹅肉的盘子和盛
\newpage
着火腿跟五香牛肉的盘子对调,莉莉端着一盘包在白餐巾纸里的粉嘟嘟的热土豆沿桌送给客人,这是玛丽·简的主意,她还建议过要给鹅肉浇上苹果沙司,可是凯特姨妈说,她一向觉得没有苹果沙司的本色烤鹅就很好了,她只希望她永远别吃到比这更坏的鹅肉。玛丽·简照应着她的学生们,要他们都吃上最好的一片。凯特姨妈和朱莉娅姨妈从钢琴上把黑啤酒、淡啤酒和矿泉水一瓶瓶打开,递过来,啤酒是为男宾们准备的,矿泉水是为女宾们准备的。笑声和喧哗声,让菜声和辞谢声,刀叉声和软木塞、玻璃塞的打开声乱成一团。加布里埃尔给大家分完了第一份,没给自己切一份,马上又开始分第二份。每个人都向他大声抗议,他不得不妥协,喝了一大口黑啤酒,因为他发现切鹅肉也是件费劲的事。玛丽·简一声不响地坐在那儿用她的晚餐,可是凯特姨妈和朱莉娅姨妈仍旧跌跌撞撞地围着桌子转,一会儿这个在前面,一会儿那个在前面,互相挡住去路,不让人注意地互相吩咐些事情,但是她们说,时间还多着呢,最后,弗雷狄·马林斯先生站起身捉住凯特姨妈,在一片哈哈的笑声中
,扑通一下把她按在椅子上。 

\newpage

给每个人都分好了,加布里埃尔笑着说:“嗯,要是哪位客人想再来点儿俗人们说的鹅肚皮里的填
馅儿,就请说话。” 

大家齐声请他自己开始用晚餐,莉莉拿着三个
她专为他留下的土豆走过来。 

“好极了,”加布里埃尔又喝了一口酒开开胃,亲切地说,“女士们,先生们,请你们在几分钟之
内忘了我的存在吧。” 

他开始吃晚餐,不介入桌上的谈话,趁人们谈话时,莉莉在收拾桌上的菜盘。谈话的题目是当时正在皇家剧院演出的歌剧团。男高音巴特尔·达西先生,一个留着潇洒的小胡子的深肤色的年轻人,高度赞扬剧团的首席女低音,可是弗朗小姐认为她的表演风格很俗气。弗雷狄·马林斯说,在舞剧《欢乐》的第二部分里,有个黑人队长唱歌,他的嗓子是他听到过
的最好的男高音之一。 

“您听过他唱吗?”他隔着桌子问巴特尔·达
\newpage

西先生。 

“没有,”巴特尔·达西先生漫不经心地回答

“因为,”弗雷狄·马林斯解释说,“我很想
知道您对他的意见。我认为他的嗓子美极了。” 

“真正的好东西总是要特狄来发现的,”布朗
先生放肆地对桌上的客人们说。 

“为什么他不能也有条好嗓子呢?”弗雷狄·
马林斯尖锐地发问。“就因为他只是个黑人吗?” 

没人来答复这个问题,于是玛丽·简把大伙引回到正统歌剧上来。她的一个学生送她一张《迷娘》⑩的免费入场券,当然啦,非常好,她说,但是它使她想起了可怜的乔治娜·伯恩斯。布朗先生还要扯起许多往事呢,他扯到了过去常到都柏林来的那些老意大利剧团——梯让斯,伊尔玛·德·莫尔兹卡,康帕尼尼,伟大的特列别里,久格里尼,拉维里,阿拉布罗,他说,那些日子才能在都柏林听到像样的歌声,
\newpage
他还谈到老皇家剧院的顶层楼座从前是怎样地每夜客满,一天晚上,一个意大利男高音怎样在听众的要求下一连唱了五遍“让我像士兵那样倒下”,每一遍都唱出了一个高音C,顶楼上的男孩子们有时怎样热情奔发,从某个有名的歌剧女演员的马车下解下马来,自己给她拉车,招摇过市,把她送回旅馆里。他问道:干吗他们现在不上演那些堂皇的歌剧了,比如《迪诺拉》,《鲁克列齐亚·波尔吉亚》⑪?因为他们找
不到好嗓子唱这些歌剧,这就是原因。 

“噢,啊,”巴特尔·达西先生说,“依我看
,现在还是有像当年一样的好歌唱家的。” 

“他们在哪儿呢?”布朗先生针锋相对地问。

“伦敦、巴黎、米兰都有,”巴特尔·达西先生激动地说。“比如,我认为卡鲁索就也挺好,假不
比您刚才提到的那些人更好的话。” 

“也许是这样,”布朗先生说,“但是我可以

\newpage
告诉您,我非常怀疑这一点。” 

“噢,我只要能听卡鲁索唱歌,什么都肯给,
”玛丽·简说。 

“要我说呀,”正在那儿剔一根骨头肉的凯特姨妈发言了,“只有一个男高音。我的意思是,能使我满意的。可是我想你们中间大概没人听他唱过歌。
” 

“他是谁呀,莫坎小姐?”巴特尔·达西先生
彬彬有礼地问。 

“他叫,”凯特姨妈说,“帕金森。我是在他顶红的时候听他唱的,我认为他那时候的嗓子,是最
棒的男高音嗓子了。” 

“奇怪,”巴特尔·达西先生说。“我从没听
人说起过他。” 

“对,对,莫坎小姐说得对,”布朗先生说。“我记得听过老帕金森唱歌,不过他对我说来是太远
\newpage

太远的往事了。” 

“一个美丽、纯净、甜蜜而又圆润的英格兰男
高音,”凯特姨妈热情地说。 

加布里埃尔吃完了,那只硕大的布丁移到了桌上,重又响起叉匙的碰击声。加布里埃尔的妻子舀出一匙匙布丁,把碟子沿桌往下传。半路上,由玛丽·简接着,在碟子里浇满木莓冻,或橘子冻,或牛奶冻和果酱。布丁是朱莉娅小姐做的,四面八方都在夸她
做得好。她自己说,这布丁烤得还不够黄。 

“啊,莫坎小姐,”布朗先生说,“但愿您认为我是够黄的人,因为您知道,我是个黄人儿呀。⑫

除了加布里埃尔之外,所有的男客们都出于对朱莉娅姨妈的赞美才吃了点布丁。加布里埃尔因为从来不吃甜食,所以芹菜就留给他吃。弗雷狄·马林斯也取了一枝芹菜便就布丁吃。他听说,芹菜是补血的,他现在正在就医。在晚餐桌旁一直沉默着的马林斯太太说,她儿子过一个星期左右要去梅勒里山。就餐
\newpage
的人便谈起梅勒里山来了,那儿的空气是多么清新,那儿的修士是多么好客,他们是怎样从来不向客人收
一文钱。 

“你们的意思是不是说,”布朗先生不相信地问,“一个家伙可以上那儿去,当旅馆似的住下来,
大吃大喝一场,然后一钱不付就走掉吗?” 

“噢,大多数人走时都要布施一点给修道院的
,”玛丽·简说。 

“但愿我们的教会也有这么个规矩,”布朗先
生坦率地说。 

他听说那些修士从来不讲话,早上两点多就起床,夜里睡在自己的棺材里,感到惊讶。他问他们这
么做是为什么。 

“那是修士会规定的,”凯特姨妈坚决地说。


\newpage

“是啊,可是为什么呢?”布朗先生问。 

凯特姨妈又说一遍,这是规定,就是这样。布朗先生似乎仍旧不了解。弗雷狄·马林斯尽可能地向他解释说,修士是在尽力弥补外界所有罪人们犯下的罪行。解释并不很清楚,因为布朗先生裂开嘴笑着说:“我非常欣赏这种做法,但是,难道惬意的弹簧床
对他们不是和棺材一样好睡吗?” 

“棺材嘛,”玛丽·简说,“是提醒他们要记
住自己最终的结局。” 

因为话题越来越阴郁,大家沉默下来了,在沉默中,只听见马林斯太太模模糊糊地小声对她邻座的说:“他们都是好人呢,那些修士,都是非常虔诚的
人。” 

葡萄干、杏子、无花果苹果、橘子、巧克力和糖果这会儿在满桌传递着,朱莉娅姨妈请客人们都来点葡萄酒,要不就雪利酒。开头,巴特尔·达西先生一样都不喝,但是他的一位邻座用胳膊肘碰碰他,对他小声讲了点什么,于是,他同意把酒杯斟满。渐渐
\newpage
地,等最后一只酒杯斟满,谈话也停了下来,大家静了一会儿,只等喝酒声和椅子移动声打破沉默。莫坎小姐们,一共三位,垂下眼睛望着台布。有人咳了一两声嗽,接着有几位先生轻轻敲了敲桌子作为保持安静的信号。完全静下来了,加布里埃尔朝后推推他的
椅子,站起来。 

为了鼓励他,桌子立即敲得更响了,接着,大家都停下不敲了。加布里埃尔把他十个抖动的手指按在台布上,紧张地对大家笑了笑。他的眼光遇到一排仰起的面孔,于是他便抬头望着枝型吊灯。钢琴弹奏出一支华尔兹舞曲,他能听得见裙子扫在客厅门上的声音。也许这会儿正有人站在外面码头上的雪地里,凝视着窗里的灯光,倾听着华尔兹乐曲呢。外边的空气清新的。远处是公园,公园里的树上压着雪。威灵顿纪念碑戴着一顶微微发亮的雪帽,由那里向西是一
片十五英亩的雪原在闪着白光。 


他开始了。 

“女士们,先生们,我有幸在今天晚上,和往
\newpage
年一样,来履行一项令人愉快的职责,但我恐怕我作为一个演说家的能力是微薄了,与这项职责实在太不
相称。” 


“不啊,不啊!”布朗先生说。 

“可是无论怎样微薄吧,今晚我只好请各位谅解我是心有余而力不足,恭请各位耐心听我讲一会儿,让我尽力用言词向各位表达一下我在这个场合的感
受。 

“女士们,先生们,我们大家聚在这好客的人家里,围坐在这张好客的餐桌边,已经不是第一次了。我们作为几位好客的女士的款待的受用者,或者我
顶好说是受害者吧,也不是第一次了。 

他用手臂在空中划了个圈,停顿了一下。每个人都朝凯特姨妈、朱莉娅姨妈和玛丽·简大笑或者微笑,她们却高兴得脸色绯红。加布里埃尔更加大胆地继续说下去:“一年又一年,我愈来愈强烈地感受到,我们的国家没有哪一种传统像好客传统一样给国家
\newpage
带来了那样多的荣誉,同时又需要国家那样小心翼翼地来加以保护。就我的经历所及,在现代国家中(我访问过不少国家),我们这个传统是独一无二的。也许有人会说,对于我们,这个传统与其说它值得夸耀,倒不如说它是一种弱点好。但是就算如此吧,我认为,它是一种高贵的弱点,并且是一种我坚信将在我们中间长久培养下去的弱点。有一点,至少,我是有把握的。只要前面讲到的这几位好心的女士还住在这幢屋子里——我从心底祝愿她们能住许多许多年——我们的祖先传给我们、而我们一定要再传给我们的子子孙孙的这种真诚、热心、殷勤的爱尔兰式的好客传
统就一直会在我们中间保持着。” 

一阵诚心诚意的赞同的低语声在餐桌四周传开。这声音使加布里埃尔突然想到,艾弗丝小姐不在了,她很不礼貌地走掉了:于是他充满自信地说:“女士们,先生们,在我们中间,新的一代正在成长,这是由新思想和新原则激励的一代人。这些新思想是严肃而热情的,它的热情,甚至使用不当时,大体上,我相信,也都是诚挚的。但我们是生活在一个怀疑论的,要是我能使用这个词儿的话,一个令人思绪烦乱
\newpage
的时代;有时我担心,这新的一代人,这个受过教育的,或者像他们现在的情况,受过太多教育的一代人,会缺乏那些属于过去的日子的仁爱、好客和善意诙谐的品质。今天晚上我听到了好些过去大歌唱家的名字,我得承认,我似乎觉得,我们是生活在一个不够宽敞的时代。而那些日子,可以毫不夸张地被称之为是宽敞的日子;假如它们已一去不返了,那么让我们希望,至少在像今天这样的聚会中,我们将仍旧怀着自豪与亲切的感情谈到它们,将仍旧在心头缅怀着对于那些去世的伟大人物的记忆,这个世界将不会甘心
让他们的美名就此消亡的。” 


“对啊,对啊!”布朗先生高声说。 

“然而,”加布里埃尔继续讲下去,他的声音变得更为柔和了,“在类似今天这样的聚会上,总有些这一类的比较悲哀的思想会出现在我们的脑海里:关于过去、关于青春、关于变革、关于早已不存在而我们今晚在这儿思念的他们那些张面孔。我们的生活道路上铺满了这类悲哀的记忆;但是,假如我们老是念念不忘于这些记忆,我们就会不忍心在活着的人们
\newpage
当中勇往直前地去进行我们的工作。我们在生活中人人都有责任所在和情之所钟,而这些东西要求我们,
完全有权利要求我们去奋发努力。 

“所以,我不能停留于过去而徘徊不前。今晚我不能让任何一种阴郁的说教来侵扰我们。我们从日常生活的奔波和忙碌之中解脱出来,在这儿短短地聚上一小会儿。我们在这儿相会,本着情长谊深的精神作为朋友,同时在某种程度上,本着真正的志同道合的精神作为同事,并且作为——我该怎么称呼她们呢?——都柏林音乐世界中的三位优雅女神的客人。”

来宾们听到这个比喻爆发出一阵鼓掌声和笑声。朱莉娅姨妈徒劳地向她的邻座们一个个打听,要他
们告诉她加布里埃尔说的是什么。 

“他说我们是希腊神话里给人以美丽和欢乐的
三位女神呢,朱莉娅姨妈。”玛丽·简说。 

朱莉娅姨妈并没有听懂,但是她微笑着抬起眼睛来注视着加布里埃尔,他以同样的调子继续讲:“
\newpage
女士们,先生们,今天晚上,我并不企图去扮演帕里斯在另一个场合扮演的角色。我并不企图在她们中间去进行选择。这项任务是叫人厌恶的,也是我的能力所不能企及的。因为当我依次看着她们的时候,不论是我们主要的女主人本人,她的善良心地,她那过于善良的心地,已经成了每个任何她的人的笑柄了;或是她的妹妹,她看来是天生赋有永不凋谢的青春的,今晚她的歌声使我们所有在座的人惊叹不已和出乎意料;或是,最末的但不是最不重要的一位,我们最年轻的女主人,我认为她是天才的、快活的、勤劳的,是天下最好的一位侄女儿,我承认,女士们和先生们,我不知道该把奖品赠给她们之中的哪一位才是。”

加布里埃尔向下瞟了一眼他的两位姨妈,看见朱莉娅姨妈脸上开朗的笑容和凯特姨妈眼眶里已经涌起的泪珠,边赶忙结束他的讲话。他风度翩翩地举起他的一杯葡萄酒,同时每个人也都端起酒杯,期待他说下去,他大声说:“让我们向她们三位一道祝酒。让我们为她们干杯,祝她们健康、富有、长寿、快乐、幸运,并且长久保持她们靠自己努力在职业上取得的骄傲地位,和她们在我们心坎上取得的荣耀而亲切
\newpage

的地位。” 

所有的客人都站起身来,手持酒杯,转向三位坐着的女士,齐声歌唱,布朗先生领唱:他们都是快活的哥儿们呀,他们都是快活的哥儿们呀,他们都是
快活的哥儿们呀,这点没人能否认。 

凯特姨妈毫不掩饰地用手帕擦起了眼泪,甚至朱莉娅姨妈似乎也感动了。弗雷狄用他的布丁叉子打拍子,唱歌的人转过身去面面相对,好像在音乐会里
一样,大家着重地唱:除非他撒谎,除非他撒谎。 

接着再一次转向他们的女主人们,唱道:他们都是快活的哥儿们呀,他们都是快活的哥儿们呀,他
们都是快活的哥儿们呀,这点没人能否认。 

晚餐房间门外的其他客人们也应声欢呼和鼓掌,并一次又一次地重新爆发,弗雷狄·马林斯像个军
官似的高擎着他的叉子。 

他们站在楼下的前厅里,沁人心脾的清新空气
\newpage
从门外涌进来,因此凯特姨妈说:“谁去把门关上呀
。马林斯太太可要害重感冒了。” 


“布朗出去了,凯特姨妈,”玛丽·简说。 


“布朗到处乱窜,”凯特姨妈放低了声音。 


她的口气让玛丽·简笑了起来。 

“说真的,”她调皮地说,“他可殷勤呢。”

“整个圣诞节,”凯特姨妈以同样的口气说,
“他就像煤气一样装在这儿。” 

这回她自己高兴地笑了,接着很快补充说:“不过叫他进来吧,玛丽·简,把门关上。但愿他没听
见我的话才好。” 

这时候,过道门开了,布朗先生从门外的石阶上走进来,笑得好像他的心都要裂开似的。他穿一件绿色长大衣,镶着仿阿斯特拉罕羔皮的袖口和领子,
\newpage
头戴一顶椭圆形的皮帽。他用手指着下边覆盖着白雪
的码头,从那儿传来一阵拖长的刺耳的呼啸声。 

“特狄要把都柏林所有的出租马车都喊出来了
,”他说。 

加布里埃尔从营业所后边的小餐具间里走出来,正往他的长大衣里伸袖子,看了看四周,说:“格
莉塔还没下来?” 

“她在穿衣服,加布里埃尔,”凯特姨妈说。


“谁在那儿弹琴呢?”加布里埃尔问。 


“没人。全走了。” 

“噢,不,凯特姨妈,”玛丽·简说,“巴特
尔·达西先生和奥卡拉汉小姐还没走。” 

“有人在钢琴上乱七八糟弹着玩呢,”加布里

\newpage
埃尔说。 

玛丽·简对加布里埃尔和布朗先生瞟了一眼,打了个冷颤说:“看见你们两位先生裹成这个样,我也觉得冷了。在这个钟点我可不愿意走一趟你们回家
去的那段路。” 

“这会儿,除了在野外美美儿逛逛,或者轻车快马地奔一阵子,”布朗先生豪壮地说:“这是我最
喜欢的事儿了。” 

“从前我们家有过一匹非常好的马和一辆双轮
轻便车的,”朱莉娅姨妈伤感地说。 

“那个永远都忘记不了的姜尼,”玛丽·简笑
着说。 

“怎么,什么姜尼呀的稀奇事儿?”布朗先生
问。 

“是故世的帕特里克·莫坎,我们的祖父的,”加布里埃尔解释道,“晚年人家都称呼他老先生的
\newpage

,是个做熬胶生意的。” 

“噢,我说,加布里埃尔呀,”凯特姨妈笑着
说,“他还有座粉坊。” 

“啊,熬胶也罢,粉坊也罢,”加布里埃尔说,“老先生有一匹马,名叫姜尼。姜尼在老先生的磨坊里干活,一圈又一圈地拉磨。一切都很美好;可是后来姜尼不幸的时候到了。一个大晴天,老先生想,他要摆起上流人士的架势,到公园里去参观军事检阅

“上帝怜悯他的灵魂吧,”凯特姨妈同情地说

“阿门,”加布里埃尔说,“于是这位老先生,就像我说的,套上姜尼,戴上自己最好的高顶礼帽,穿上自己最好的硬领,然后,堂而皇之地驾车驶出
了他的祖宅,那房子是在后街附近吧,我想。” 

看着加布里埃尔的样子,大家都笑了,连马林斯太太都笑了,凯特姨妈说:“噢,我说呀,加布里埃尔,他不住在后街呢,真的。只是磨坊在那儿。”
\newpage


“他把姜尼套在车上,驶出他的祖宅。”加布里埃尔继续说下去,“直到姜尼走到它望见比利大帝雕像的地方以前,一切都非常顺利:不知是它爱上了比利大帝骑的那匹马呢,还是它以为又回到了磨坊里
,反正它就围着雕像转起圈儿来了。” 

加布里埃尔在其余人的大笑声中,穿着套鞋在
前厅里踱了一个圈儿。 

“它走了一圈又一圈,”加布里埃尔说,“而这位老先生,他是个自视颇高的老先生,非常地愤慨。‘向前走,老兄!你这是什么意思?老兄!姜尼!
姜尼!真是莫名其妙!这马是怎么回事儿?’” 

加布里埃尔的模仿引起了一连串大笑声,被前门上一声响亮的敲击声打断了。玛丽·简跑去开门,进来的是弗雷狄·马林斯。弗雷狄·马林斯,帽子贴在后脑勺上,肩膀冷得耸起来,正累得直喘,冒着热
气。 

\newpage


“我只能弄到一辆出租马车,”他说。 

“噢,我们沿着码头还能再找到一辆的。”加

“是啊,”凯特姨妈说,“最好别让马林斯太
太老是站在风口上。” 

马林斯太太由她儿子和布朗先生扶着走下门前的台阶,忙乱了一阵,把她扶上了马车。弗雷狄·马林斯跟着她爬上了车,花了好些时间才把她安顿在座位上,布朗先生给他出主意帮忙。终于,把她舒舒服服安顿好了,弗雷狄·马林斯请布朗先生也上车来。又说了一大阵子乱七八糟的话,布朗先生才上了车。马车夫把一条毯子盖在他们膝头上,然后弯下腰问他们上哪儿去。说话愈加乱七八糟了,弗雷狄·马林斯和布朗先生各自把头从马车的一个窗户里伸出来,让马车夫往不同的方向走。难是难在不知道布朗先生在中途什么地方下车好,凯特姨妈、朱莉娅姨妈和玛丽·简也站在门口台阶上帮忙讨论,七嘴八舌,相互矛盾,笑个不停。至于弗雷狄·马林斯,他是笑得一句话也说不出了。他把脑袋在马车窗子里伸进伸出,告
\newpage
诉他母亲,讨论进展得如何,每进出一回,他的帽子都得冒一次极大的风险,到最后,布朗先生压倒众人的喧声,向已被弄糊涂了的马车夫喊道:“你知道三
一学院吗?” 


“知道,先生,”马车夫回答说。 

“好,你就冲着三一学院的大门撞吧,”布朗先生说,“然后我们再告诉你上哪儿去。现在懂了吗


“懂了,先生,”马车夫说。 


“那就像鸟儿一样向三一学院飞吧。” 


“遵命,先生,”马车夫说。 

鞭子一响,马车在一阵笑声和再见声中沿着码
头隆隆而去。 

加布里埃尔没跟其他人一块到门口去。他在过道的一个暗处盯着楼梯望。一个女人站在靠近第一段
\newpage
楼梯拐弯的地方,也在阴影里。他看不见她的脸,可是他能看见她裙子上赤褐色和橙红色的拼花,在阴影中显得黑一块白一块的,那是他的妻子。她倚在楼梯扶手上,在听着什么。加布里埃尔见她一动不动的样子,感到惊奇,便也竖起耳朵听。但是除了门前台阶上的笑声和争执声、钢琴弹出的几个和音和几个男人
的歌唱声音之外,就再也听不出什么了。 

他静静地站在过道的暗处,试图听清那声音所唱的是什么歌,同时盯着他的妻子望。她的姿态中有着优雅和神秘,好像她就是一个什么东西的象征似的。他问自己,一个女人站在楼梯上的阴影里,倾听着远处的音乐,是一种什么象征。如果他是个画家,他就要把这个姿势画出来。她的蓝色毡帽可以在幽暗的背景上衬托出她青铜色的头发,她裙子上的深色拼花衬托出那些浅色的来。他要把这幅画叫做《远处的音
乐》,假如他是个画家的话。 

大门关上了,凯特姨妈、朱莉娅姨妈和玛丽·
简回到过道里,仍旧在笑着。 

\newpage

“啊,弗雷狄真糟糕,对不?”玛丽·简说,
“他真是糟透了。” 

加布里埃尔什么也没说,只是朝楼梯上他妻子站的地方指了指。现在大门关上了,歌声和钢琴声也就听得更清了。加布里埃尔举起手来示意她们安静。听来这歌是用爱尔兰老调子唱的,歌唱者无论对他的歌词还是对他的嗓子都没有把握。由于距离,也由于歌者的嗓子嘶哑,声音显得哀伤,歌声隐隐地传出了节奏和吐露悲痛的句子:哦,雨点打着我浓密的头发,露珠儿沾湿我的皮肤,我的婴儿寒冷地躺着……“噢,”玛丽·简大声说。“是巴特尔·达西在唱,他不会唱一个通宵的。噢,我要让他唱一支歌再走。”


“噢,行啊,玛丽·简,”凯特姨妈说。 

玛丽·简擦过其他人跑向楼梯,可是她还没到
楼梯上,歌声就停止了,钢琴也碰地一声关上了。 

“哦,真可惜!”她叫道。“他下来了吗,格

\newpage
莉塔?” 

加布里埃尔听见他妻子应了一声是,看见她朝他们走下来。她身后几步就是巴特尔·达西先生和奥
卡拉汉小姐。 

“噢,达西先生,”玛丽·简叫道,“我们都听得正入迷呢,您这样突然不唱了,简直是太不应该
了。” 

“整个晚上我都在他身边的,”奥卡拉汉小姐说。“康罗姨太太也是,他跟我们说他感冒得厉害,
没法唱。” 

“噢,达西先生,”凯特姨妈说,“那么这是
撒了个很妙的小谎咯?” 

“你没发觉我哑得像乌鸦吗?”达西先生粗声
粗气地说。 

他急忙走进餐具间,穿上长大衣。其他人被他这句粗鲁的话顶回去,不知说什么好了。凯特姨妈皱
\newpage
皱眉头暗示其余的人别谈这个了。达西先生正站着仔
细围他的围脖,一脸不高兴的样子。 

“是天气不好呀,”听了一会儿,朱莉娅姨妈

“是啊,人人都感冒,”凯特姨妈马上接着说
,“人人都感冒。” 

“人家说,”玛丽·简说,“三十年没下过这样大的雪了,我今天早晨在报纸上看到,这场雪整个
爱尔兰都下遍了。” 


“我喜欢看下雪,”朱莉娅姨妈伤感地说。 

“我也喜欢,”奥卡拉汉小姐说,“我觉得除
非地上有雪,否则圣诞节就不像真正的圣诞节。” 

“可是可怜的达西先生就不喜欢雪呢,”凯特
姨妈笑着说。 

达西先生从餐具间走出来,脖子裹得严严实实
\newpage
,扣子扣得整整齐齐,用一种悔过的口气向他们谈起自己感冒的经过。大家都给他出主意,说是真的太遗憾了,极力劝他,在晚上户外可要加意保护他的喉咙。加布里埃尔注视着他的妻子,她没有加入谈话。她恰巧站在布满灰尘的扇形气窗下,煤气灯的火光照亮她深青铜色的头发,几天前,他见她在炉前烤干她的这头美发。她还是方才那个姿势,似乎没察觉到她身边的谈话。最后,她向他们转过身去,加布里埃尔看见她面颊上泛起红色,她的眼睛闪着光。一种突然的
快乐从他心底涌出。 

“达西先生,”她问,“您刚才唱的那支歌叫
什么名字?” 

“叫《奥格里姆的姑娘》,”达西先生说,“
可是我记不太清了。怎么,你知道它吗?” 

“《奥格里姆的姑娘》,”她重复着说,“我
想不起这个歌名了。” 

“这支歌子非常美,”玛丽·简说,“你今晚
\newpage

嗓子不好,真遗憾。” 

“我说,玛丽·简,”凯特姨妈说,“别去打
扰达西先生了。我不愿让他觉着烦。” 

看见大家都已做好出发的准备,她便送他们来到门口,在那儿道了晚安:“好,晚安,凯特姨妈,
谢谢您给了我们这么一个快乐的夜晚。” 


“晚安,加布里埃尔,晚安,格莉塔!” 

“晚安,凯特姨妈,真太感谢了。晚安,朱莉
娅姨妈。” 


“噢,晚安,格莉塔,我没看见你呢。” 

“晚安,达西先生。晚安,奥卡拉汉小姐。”


“晚安,莫坎小姐。” 


\newpage

“晚安,再一次祝您晚安。” 


“大家都晚安。一路平安。” 


“晚安,晚安。” 

清晨还是很幽暗的。暗淡的黄光低覆在房屋上和河面上;天好像在往下沉一样。脚下是半融的雪,只有一道道,一片片的雪盖在屋顶上、码头的护墙上和围绕码头一带的栏杆上。街灯仍在黑沉沉的空气中红红地燃着,河那边,四院大厦⑬,咄咄逼人地唉低
沉的天空背景上显现出来。 

她和巴特尔·达西先生一块在他前面走着,她的鞋子包成个褐色的小包,夹在一只胳膊下,双手把裙子从泥泞的雪地上提起。她的姿态已不像方才那么优雅了,可是加布里埃尔的眼睛依然因幸福而发亮。血液在他的血管中流涌,他思潮起伏,澎湃激荡,自
豪,欢乐,温柔,英勇。 

她在他前面走得那样轻捷,挺拔,使他很想不声不响地追上她,抓住她的肩膀,在她耳边说点什么
\newpage
傻气的、充满深情的话。在他看来,他是那样地脆弱,他渴望能够保护他不受任何东西的侵犯,并且和她单独在一起。他俩私生活的一些片段突然像星星一样在他的记忆中亮起来。一只紫红色信封放在他早餐杯子旁,他正在用手抚摸着它。鸟儿在常春藤上鸣啭,他幸福得东西也吃不下,他俩站在挤满人的月台上,他正把一张票塞进她手套的暖和的掌心里。他和她一块儿站在冷风中,从一扇有隔栅的窗子外面望进去,看一个男子在呼呼响的熔炉前做瓶子。那天冷极了。她的脸,在冰冷的空气中发出芬芳,和他的脸那么贴近,突然他向那个熔炉前的人叫道:“那火很旺吗?

可是那人因为炉子的响声而没有听见。也好。
他很可能回答得相当粗鲁呢。 

一阵更为温柔的快乐从他心底迸出,随同温暖的血液,在他的动脉里流着。如同星星的柔和的光,他们共同生活中的一些瞬间,没有人知道,也永远不会有人知道的瞬间,突然出现了,照亮了他的记忆。他急于想要让她回想起那些瞬间,让她忘记那些他俩沉闷地共同活着的年月。而只记住他们这些心醉神迷
\newpage
的瞬间。因为他觉得,岁月并没有能熄灭他或她的心灵。他们的孩子、他的写作、她的家务操劳,都没有能熄灭他们心灵的柔情之火。在他那时写给她的一封信中,他说:“为什么这些词句让我觉得好像是那么迟钝而冰冷?是不是因为世界上没有一个词温柔得足
以用来称呼你呢?” 

像远处的音乐声一般,这些他多年前写过的字句,从过去向他驶来。他非常想能跟她两人单独在一起。等别人都走开了,等他和她到了他们所住的旅馆房间里,他们就单独在一起了。他要温柔地喊她一声
:“格莉塔!” 

也许她不会马上听见;她可能在换衣裳。后来他的声音里某种东西引起她的注意。她转过身来,瞧着他……在酒店街的转角上,他们遇上一辆出租马车。辚辚的车轮声让他高兴,因为这就省得他去参加谈话了。她向车窗外望着,显得困倦。其他人只说过三两句话,指出到了某幢建筑或街道。马儿疲乏地疾驰在早晨阴霾的天空下,拖着格格作响的旧车厢,加布里埃尔又跟她坐在一辆马车中,赶去乘船,赶去度蜜
\newpage

月。 

当马车驰过奥康内尔桥时,奥卡拉汉小姐说:“人家说,你每回过奥康内尔桥都会看见一辆白色的
马。” 

“这回我看见了一个白色的人,”加布里埃尔


“在哪儿?”巴特尔·达西先生问。 

加布里埃尔指指雕像,它身上盖着一片片的雪
。他像同熟人打招呼似的向他点点头,挥挥手。 


“晚安,丹,”他快活地说。 

当马车来到旅馆前,加布里埃尔跳下车,不顾巴特尔·达西先生的抗议,付了车钱。他多给了车夫一个先令。车夫敬个礼,并且说:“祝您新年如意,
先生。” 

“也祝您新年如意,”加布里埃尔衷心地说。
\newpage


她下车时,站立在路边镶砌的石块上向其他人告别时,在他手臂上靠了一会儿。她那么轻轻地靠在他的手臂上,轻得像几个钟头之前他搂着她跳舞时似的。那时他感到骄傲和幸福,幸福,因为她是他的,骄傲,因为她的美和她那做妻子的仪态。然而此刻,在那许多记忆重新激起之后,一接触到她的身体,这音乐般的、奇异的、方向的身体,他立刻周身感到一种强烈的情欲。趁她默默无声时,他把她的手臂拉过来紧贴着自己,他俩站在旅馆的门前,他感到他俩逃脱了他们的生活和责任,逃脱了家和朋友,两人一块,怀着两颗狂乱的、光芒四射的心跑开了,要去从事
一次新的冒险。 

门厅里,一位老人在一只椅背顶端突出的大椅子上打瞌睡。他在柜台间点燃一支蜡烛,领他俩上楼去。他俩一声不响地跟着他。脚步在铺了厚地毯的楼梯上发出轻轻的声音,她在看守人的身后登楼,她的头在向上走时垂着,她娇弱的两肩弓起,好像有东西压在背上,她的一群紧紧贴着她身体。他本来要伸出两只手臂去拥住她的臀部,抱着她的身体,只是他手
\newpage
指甲使劲抵在手掌心上才止住了他身体的这种狂热的冲动。看守人在楼梯上停了一下,收拾他淌泪的蜡烛。他俩也停在他身后的下一步梯级上。寂静中,加布里埃尔能够听见融化的蜡油滴进烛盘里的声音,和他
自己的心脏撞在肋骨上的声音。 

看守人领他俩经过一道走廊,打开一扇门。然后他把摇摇晃晃的蜡烛放在梳妆台上,问早上几点钟
喊醒他们。 


“八点,”加布里埃尔说。 

看守人指指电灯开关,咕哝着道歉起来,但是
加布里埃尔打断了他。 

“我们不需要灯。街上照进来的光就足够了。我说,”他指指蜡烛,又添了一句,“您不妨把这个
漂亮的玩意儿拿走吧,求求您。” 

看守人又把蜡烛拿在手里,但是动作很缓慢,因为他对这样一个新鲜的念头感到惊奇。然后他嘟哝
\newpage

了一声晚安就走了。加布里埃尔锁上门。 

一道长长的苍白的街灯光照进屋来,从一个窗口直照到房门,加布里埃尔把长大衣和帽子甩在一只长沙发上,穿过房间走回窗前。他向下面的街道上望望,想使自己的情绪平静一点儿。然后他转过身,靠在一只五斗橱上,背向光。她已经除掉帽子和披风,正立在一面很大的转动穿衣镜前,解开她腰上的搭扣。加布里埃尔踌躇了一会儿,望着她,然后说:“格
莉塔!” 

她慢慢地从镜子前转过身来,沿着那道光向他走过来。他的脸显得那么严肃而疲倦,使得加布里埃
尔没法开口说话。不,还没到时间。 


“你好像累了,”他说。 


“我是有点儿累,”她回答道。 


“你不觉得不舒服或是虚弱吗?” 

\newpage


“不,是累了;就是这个。” 

她继续向前走到窗下,立在那儿,向外望。加布里埃尔又等了一会儿,后来,生怕羞怯会战胜自己
,他就突然一下子说:“听我说,格莉塔!” 


“什么事儿?” 

“你认识那个可怜人儿,马林斯吗?”他急速
地问。 


“认识呀,他怎么啦?” 

“哎,可怜的家伙,不过说到底,他还是正派人,”加布里埃尔用一种不自然的嗓音继续说道,“他把我借给他的一英镑硬币还了我,而我并没有想要他还,说真的。可惜他不肯躲开那个布朗,因为他也
不是个坏人,说真的。” 

他这时烦恼得浑身颤抖。为什么她看起来那么心不在焉?他不知道怎么开头才好。她也因为什么事
\newpage
在烦恼吗?她要是能转身向着他或是自个儿上他这儿来该多好!像她现在这样去搂她是粗鲁的。不,他必须现在她眼睛里看见一点儿热烈的感情才行。他急于
掌握住她的奇特的情绪。 

“你什么时候借给他那个英镑的?”她在片刻
的无言之后说。 

加布里埃尔极力控制自己,不要猛烈间对酒鬼马林斯和他的一个英镑这件事说出粗鲁的话。他急于想从灵魂深处对她发出呼喊,急于把她的身体紧紧搂抱在自己的怀里,急于要制服她。然而他说:“哦,圣诞节时候,他开了那个小贺年片商店,在亨利街上

他正处在冲动和情欲的狂热之中,连她从窗前走过来也没听见。她在他面前站了一会儿,目光奇异地瞧着他。然后,她忽然踮起脚尖来,两只手轻轻地
搭在他的肩头,吻了吻他。 

“你是个很大方的人,加布里埃尔,”她说。

\newpage

加布里埃尔在颤栗,因为她突然的一吻和她说这句时的仪态让他欣喜,他把两手放在她的头发上,把它向后抚平,手指几乎没有接触到头发。这头发洗得又美又光亮。他心里的幸福已经满得溢出来了。正在他想要的时候,她自己走到他这儿来了。也许她的思想跟他的不谋而合吧。也许他感觉到了他心中急切的情欲吧,所以她就有了一种顺从的心情。现在,她这样轻易地自己迎上来,他倒奇怪他方才怎么会那样
胆怯。 

他站着,两手抱着她的头。然后,一条手臂急速滑过她的身体,把她搂向自己,柔情地说:“格莉
塔,亲爱的,你想要什么?” 

她没有回答,也没有完全顺从他的手臂。他又柔情地说:“告诉我,格莉塔。我觉得我知道你在想
些什么。我知道吗?” 

她没有马上回答。然后她说话了,眼泪夺眶而
出。 

\newpage

“噢,我在想那支歌,《奥格里姆的姑娘》。

她从他手中挣脱,跑向床边,两条手臂伸过床架的栏杆,把脸埋起来。加布里埃尔惊讶地立了一会儿,一动也不动,然后跟在她后面走过去。当他经过转动穿衣镜的时候,他看见自己的整个身影,看见他宽阔的、填得好好的硬衬胸,看见自己的脸孔,每当他在镜子中看见它的表情时总不免感到惑然,看见他亮闪闪的金丝眼镜,他在离她几步远的地方停下来,
说:“那支歌怎么啦?怎么会让你哭起来?” 

她从臂弯里抬起头来,像个孩子似的用手臂擦干眼泪。他的声音里渗入了一种他本来不曾想有的更
亲切的调子、“怎么啦,格莉塔?”他问。 

“我想起一个很久以前的人,他老是唱这支歌
的。” 

“这位很久以前的人是谁?”加布里埃尔微笑
着问。 

\newpage

“是我在高尔韦住的时候认识的,那时候我跟
我奶奶住在一块儿,”她说。 

笑容从加布里埃尔脸上消逝了。已故阴沉的怒气开始在他思想深处聚集,而他那股阴沉的情欲的烈
火也开始在他血管中愤怒地燃烧。 


“是一个你爱过的人吧?”他讥笑地说。 

“是一个我从前认识的年轻人,”她回答说,“名字叫迈克尔·富里。他老是唱那支歌的。《奥格
里姆的姑娘》。他很不俗气。” 

加布里埃尔一声不响。他不希望她认为,他对
这个不俗气的年轻人感到兴趣。 

“我可以那么清楚地看见他,”过了一会儿,她说。“他有那么一双眼睛,大大的、黑黑的眼睛!
眼睛里还有那么一种表情——那么一种表情!” 

“哦,这么说,你那时候爱他了?”加布里埃
\newpage

尔说。 

“我常跟他出去散步,”她说,“我住在高尔
韦的时候。” 


一个思想从加布里埃尔头脑中闪过。 

“也许就因为这个,你想跟那个叫艾弗丝的姑
娘行高尔韦去吧?”他冷冰冰地说。 


“去干嘛?” 

她的眼光让加布里埃尔感到尴尬。他耸耸肩头
说:“我怎么知道?去见他呗,也许。” 

她把眼光从他身上移开,沿着地上那道光,默
不做声地向窗口望去。 

“他死了,”她终于说,“他十七岁就死了。
难道这么年轻就死,不可怕吗?” 

\newpage

“他是干什么的?”加布里埃尔问,还是讥诮
的口气。 


“他在煤气厂工作,”她说。 

加布里埃尔感到丢脸,因为讽刺落了空,又因为从死者当众扯出这么个人来,一个在煤气厂干活的年轻人。他正满心都是他俩私生活的回忆,满心都是柔情、欢乐和欲望的时候,她却一直在心里拿他跟另一个人做比较。一阵对自身感到羞惭的意识袭击着他。他看见自己是一个滑稽人物,一个给姨妈们跑个腿儿,赚上一两个便士的小孩子,一个神经质的、好心没好报的感伤派,在一群俗人面前大言不惭地讲演,把自己乡巴佬的情欲当作美好的理想,他看见自己是他刚才在镜子里瞟到一眼的那个可怜又可鄙的愚蠢的家伙。他本能地把脊背更转过去一些,更多地挡住那
道光,别让她看见自己羞得发烧的额头。 

他试图仍然用他那冷冰冰的盘问语气讲话,可
是开起口来,他的声音却是谦卑的、淡漠的。 

\newpage

“我想你跟这个迈克尔·富里谈过恋爱吧,格
莉塔,”他说。 


“我那时候跟他很亲密,”她说。 

她的声音是含糊而悲伤的。加布里埃尔感觉到,现在如果想把她引到他原先打算的方向上去,会是多么徒劳,他抚摸着她的一只手,也很哀伤地说:“那么他怎么那样年轻就死了呢,格莉塔?痨病吧,是
吗?” 


“我想他是为我死的,”她回答。 

一听到这个回答,加布里埃尔感到一阵朦胧的恐惧,似乎是在他渴望达到目的的时刻里,有某个难以捉摸的、惩罚性的东西正出来跟他作对,正在它那个朦胧的世界里聚集力量反对他。然而他依靠理性努力甩开了这种恐惧,继续抚摸她的手。他没有再问她,因为他觉得她会自己告诉他的。她的手温暖而潮湿:这手对他的抚摸不作反应,但是他继续抚摸着它,恰像他在那个春天的早晨抚摸她的第一封来信一样。
\newpage


“那是个冬天,”她说,“大约是冬天开始的时候,我正要离开奶奶家,上这儿的修道院来。那时候他正在高尔韦他的住处生病,不能出门,人家已经给他在奥特拉尔德的亲人们写信去了。他生的是肺结
核,人家说,或者这一类的病。我一直不清楚。” 


她沉默了一会,叹了一口气。 

“可怜的人儿,”她说。“他非常喜欢我,他又是那么个文雅的年轻人。我们时常一块出去,散散步,你知道,加布里埃尔,在乡下人家都是这样的。要不是因为他的健康,他就去学唱歌了。他嗓子非常
之好,可怜的迈克尔·富里。” 


“那么,后来呢?”加布里埃尔问。 

“后来我从高尔韦到修道院来的时候,他病得更厉害了,人家不让我见他。我就给他写封信,说我要去都柏林了,到夏天回来,希望他到时候会好起来

\newpage

她停了一会儿,为了控制自己的声音,然后又说下去:“后来我动身的前一天夜里,我在尼古岛上我奶奶家里,正收拾着东西,我听见有小石块掷上来打在我窗上的声音。窗子湿得很,我看不见,我就跑下楼,我从房后溜出去,到了花园里,看见这可怜的
人正立在花园的一头,浑身发抖。” 


“你没让他回去吗?”加布里埃尔问。 

“我求他马上回家去,告诉他,这样立在雨地里会要他命的。可是他说,他不想活了。我现在能清清楚楚、清清楚楚看见他的眼睛!他站在围墙尽头,
那地方有一棵树。” 


“那么他回家了吗?”加布里埃尔问。 

“嗯,他回家了。等我到修道院还没一礼拜,他就死了,埋在奥特拉尔德,那儿是他老家。噢,那
一天,我听说他死了的那一天!” 

她停止了,她抽噎得说不出话来,她无法克制
\newpage
激动,脸朝下扑倒在床上,脸埋在被子里呜咽,加布里埃尔把她的手又握了一阵,不知如何是好,后来,不敢在她的悲痛的时候打扰她,他轻轻放下她的手,
静悄悄地走向窗前。 


她睡熟了。 

加布里埃尔斜靠在臂肘上,心平气和地对她乱蓬蓬的头发和半开半闭的嘴唇望了一会儿,倾听着她深沉的呼吸。这么说,在她一生中曾有过那段恋爱史。一个人曾经为她而死去。此刻想起他,她的丈夫,在她一生中扮演了一个多么可怜的角色,他几乎不太觉得痛苦了。她安睡着。他在一旁观望,仿佛他和她从没象夫妻那样一块生活过。他好奇的眼光长久地停留在她的面庞上,她的头发上:他想着,在她有着最初少女美好的那个时候,她该是什么模样,这时,一种奇异的、友爱的、对她的怜悯进入他的心灵。甚至对自己,他也不想说她的面孔如今已不再漂亮了,然而他知道,这张面孔已不再是那张迈克尔·富里不惜
为之而死的面孔。 

\newpage

也许她没把事情全告诉他。他的眼光移向那把椅子,那上面她撂了几件衣服。衬裙上的一条带子垂在地板上。一只靴子直立着,柔软的鞋帮已经塌下去了;另一只躺在它的旁边。他奇怪自己在一小时前怎么会那样感情激荡。是什么引起的?是他姨妈家的晚餐,是他那篇愚蠢的讲演,是酒和跳舞,在过道里告别时的说笑,沿着河在雪地里走时的快乐心情,是这些引起的。可怜的朱莉娅姨妈!她自己不久后也要变成跟帕特里克·莫坎的幽灵和他的马在一道的幽灵了。当她唱着《打扮新娘子》的时候,他在刹那间从她面孔上发现了那种形容枯槁的样子,不久以后,也许他会坐在那同一间客厅里,穿了丧服,绸帽子放在膝盖上。百叶窗关着,凯特姨妈坐在他身边,哭着,擤着鼻涕,告诉他朱莉娅是怎么死的。他搜索枯肠,想找出一些可以安慰她的话,而却只找到一些笨拙的、用不上的话。是的,是的:这不要多久就会发生了。

屋里的空气使他两肩感到寒冷。他小心地钻进被子,躺在他妻子身边。一个接一个,他们全都将变成幽灵。顶好是正当某种热情的全盛时刻勇敢地走到那个世界去,而不要随着年华凋残,凄凉地枯萎消亡
\newpage
。他想到,躺在他身边的她,怎样多少年来在自己心头珍藏着她情人告诉她说他不想活的时候那一双眼睛
的形象。 

泪水大量地涌进加布里埃尔的眼睛。他自己从来不曾对任何一个女人有过那样的感情,然而他知道,这种感情一定是爱。泪水在他眼睛里积得更满了,在半明半暗的微光里,他在想象中看见一个年轻人在一棵滴着水珠的树下的身形。其他一些身形也渐渐走近。他的灵魂已接近那个住着大批死者的领域。他意识到,但却不能理解他们变幻无常、时隐时现的存在。他自己本身正在消逝到一个灰色的无法捉摸的世界里去:这牢固的世界,这些死者一度在这儿养育、生
活过的世界,正在溶解和化为乌有。 

玻璃上几下轻轻的响声吸引他把脸转向窗户,又开始下雪了。他睡眼迷蒙地望着雪花,银色的、暗暗的雪花,迎着灯光在斜斜地飘落。该是他动身去西方旅行的时候了。是的,报纸说得对:整个爱尔兰都在下雪。它落在阴郁的中部平原的每一片土地上,落在光秃秃的小山上,轻轻地落进艾伦沼泽,再往西,
\newpage
又轻轻地落在香农河黑沉沉的、奔腾澎湃的浪潮中。它也落在山坡上安葬着迈克尔·富里的孤独的教堂墓地的每一块泥土上。它纷纷飘落,厚厚积压在歪歪斜斜的十字架上和墓石上,落在一扇扇小墓门的尖顶上,落在荒芜的荆棘丛中。他的灵魂缓缓地昏睡了,当他听着雪花微微地穿过宇宙在飘落,微微地,如同他们最终的结局那样,飘落到所有的生者和死者身上。


注释: 

①克瑞斯蒂剧团:十九世纪美国人乔治·克瑞斯蒂在纽约创办的一种剧团,有白人扮演黑人演唱黑人歌曲,直到二十世纪初,人们仍习惯称这种剧团为
“克瑞斯蒂”剧团。 

②伦敦古堡是座监狱。理查三世在古堡中杀害
两王子。详见莎士比亚《理查三世》。 

③巴尔不里干:都柏林郡北部沿海的一个镇名

④西布立吞人:古代盎格鲁-撒克逊人入侵以
\newpage
前住在不列颠岛上的凯尔特族人,后被迫退入西部山地,逐渐形成近代威尔士人:一部分渡海迁居高卢的阿尔魔利卡。故西布立吞人即指威尔士人。此处艾弗
丝只是讽刺加布里埃尔的行为不像个爱尔兰人。 

⑤阿兰岛:爱尔兰岛东北,大西洋中的一个小
岛名。 


⑥康诺特:爱尔兰的一个省。 

⑦威灵顿(1769-1852):英国统帅。在反对拿破仑战争中,为反法联盟统帅之一,以指
挥滑铁卢战役闻名。 

⑧帕里斯:希腊神话中,由特洛伊王子帕里斯判断三位女神哪一位最美丽,后来故事发展引起特洛
伊战争。 


⑨士麦那:土耳其港口。 

⑩《迷娘》:歌德原著,法国马思耐谱为歌剧
\newpage

的名作。 

⑪鲁克列齐亚·波尔吉亚传说是文艺复兴时教皇亚历山大六世之女,用她的故事写的剧本不止一个。蒂诺拉是德国音乐家迈尔贝尔作曲的意大利语歌剧

⑫布朗说的是句俏皮话,因为布朗(brow
n)在英语里作“黄褐色”解。 

⑬四院大厦:爱尔兰都柏林的著名建筑。

\end{document}
