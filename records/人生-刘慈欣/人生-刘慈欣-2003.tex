\documentclass{article}
\usepackage[utf8]{inputenc}
\usepackage{ctex}

\title{人生\footnote{Click to View:\url{https://web.archive.org/web/20220906000137/https://www.bilibili.com/read/cv18327836/}}}
\author{刘慈欣}
\date{2003-09-27}

% \setCJKmainfont[BoldFont = Noto Sans CJK SC]{Noto Serif CJK SC}
% \setCJKsansfont{Noto Sans CJK SC}
% \setCJKfamilyfont{zhsong}{Noto Serif CJK SC}
% \setCJKfamilyfont{zhhei}{Noto Sans CJK SC}
% \setlength\parindent{0pt}

\begin{document}
\CJKfamily{zhkai}

\maketitle


\Large


母亲:“我的孩儿,你听得见吗?” 


胎儿:“我在哪里?!” 

母亲:“啊孩儿,你听见了?!我是你妈妈啊
!” 

胎儿:“妈妈!我真是在你的肚子里吗?我周
围都是水……” 


母亲:“孩儿,那是羊水。” 

胎儿:“我还听到一个声音,咚咚的,像好远
的地方在打雷。” 

\newpage

母亲:“那是妈妈的心跳声……孩儿,你是在
妈妈的肚子里呢!” 

胎儿:“这地方真好,我要一直呆在这里。”
 

母亲:“那怎么行?孩儿,妈要把你生出来!
” 

胎儿:“我不要生出去,不要生出去!我怕外
面!” 

母亲:“哦,好,好孩子,咱们以后再谈这个
吧。” 

胎儿:“妈,我肚子上的这条带子是干什么的
?” 

母亲:“那是脐带,在妈的肚子里时你靠它活
着。” 

\newpage

胎儿:“嗯……妈,你好像从来也没到过这种
地方。” 

母亲:“不,妈也是从那种地方生出来的,只是不记得了,所以你也不记得了……孩儿,妈的肚子
里黑吗?你能看到东西吗?” 

胎儿:“外面有很弱的光透进来,红黄红黄的
,像西套村太阳落山后的样子。” 

母亲:“我的孩儿啊,你还记得西套村?!妈
就生在那儿啊!那你一定知道妈是什么样儿了?” 

胎儿:“我知道妈是什么样儿,我还知道妈小小的时候是什么样儿呢?妈,你记得什么时候你第一
次看到自己吗?” 

母亲:“不记得了,我想肯定是从镜子里看到的吧,就是你爷爷家那面好旧好旧的,破成三瓣又拼
到一块儿的破镜子……” 

\newpage

胎儿:“不是,妈,你第一次是在水面儿上看
到自个儿的。” 

母亲:“嘻……怎么会呢?咱们老家在甘肃那
地方,缺水呀,满天黄沙的。” 

胎儿:“是啊,所以爷爷奶奶每天都要到很远的地方去挑水。那天奶奶去挑水,还小不点儿的你也跟着去了。回来的时候太阳升到正头上,毒辣辣的,你那个热那个渴啊,但你不敢向奶奶要桶里的水喝,因为那样准会挨骂,说你为什么么不在井边喝好?但井边那么多人在排队打水,小不点儿的你也没机会喝啊。那是个旱年头,老水井大多干了,周围三个村子的人都挤到那口深机井去打水……奶奶歇气儿的时候,你扒到桶边看了看里面的水,你闻到了水的味儿,
感到了水的凉气儿……” 


母亲:“啊,孩儿,妈记起来了!” 

胎儿:“……你从水里看到了自个儿,小脸上满是土,汗在上面流得一道子一道子的……这可是你
\newpage

记事起第一次看到自个儿的模样儿。” 


母亲:“可……你怎能记得比我还清呢?” 

胎儿:“妈你是记得的,只是想不起来了,在我脑子里那些你记得的事儿都清楚了,都能想起来了
。” 


母亲:“……” 


胎儿:“妈,我觉得外面还有一个人。” 

母亲:“哦,是莹博士。本来你在妈妈肚子里是不能说话的,羊水里没有让你发声的空气,莹博士
设计了一个机器,才使你能和妈妈说话。” 

胎儿:“噢,我知道她,她年纪比妈稍大点儿
,戴着眼镜,穿着白大褂。” 

母亲:“孩儿,她可是个了不起的有学问的人

\newpage
,是个大科学家。” 


莹博士:“孩子,你好!” 


胎儿:“嗯……你好像是研究脑袋的。” 

莹博士:“我是研究脑科学的,就是研究人的大脑中的记忆和思维。人类的大脑有着很大的容量,一个人的脑细胞比银河系的星星都多。以前的研究表明,大脑的容量只被使用了很少的一部分,大约十分之一的样子。我领导的项目,主要是研究大脑中那些未被使用的区域。我们发现,那大片的原以为是空白的区域其实也存贮着巨量的信息,进一步的研究提示了一个令人震惊的事实:那些信息竟然是前辈的记忆
!孩子,你听得懂我的话吗?” 

胎儿:“懂一点儿,你和妈妈说过好多次,她
懂了,我就懂了。” 

莹博士:“其实,记忆遗传在生物界很普遍,比如蜘蛛织网和蜜蜂筑巢之类我们所说的本能,其实都是遗传的记忆。现在我们发现人类的记忆遗传,而
\newpage
且是一种比其它生物更为完整的记忆遗传。如此巨量的信息是不可能通过DNA传递的,它们存贮在遗传介质的原子级别上,是以原子的量子状态记录的,于
是诞生了量子生物学……” 


母亲:“博士,孩儿听不懂了。” 

莹博士:“哦,对不起,我只是想让你的宝宝知道,与其他的孩子相比他是多么幸运!虽然人类存在记忆遗传,但遗传中的记忆在大脑中是以一种隐性的、未激活的状态存在的,所以没有人能觉察到这些
记忆的存在。” 

母亲:“博士啊,你给孩儿讲得浅些吧,因为
我只上过小学呢。” 

胎儿:“妈,你上完小字后就在地里干了几年
活儿,然后就一个人出去打工了。” 

母亲:“是啊,我的孩儿,妈在那连水都是苦

\newpage
的地方再也呆不下去了,妈想换一种日子过。” 

胎儿:“妈后来到过好几个城市,在当过饭店服务员,当过保姆,在工厂糊过纸盒,在工地做过饭
,最难的时候还靠捡破烂过日子……” 


母亲:“嗯,好孩子,往下说。” 


胎儿:“反正我说的妈都知道。” 


母亲:“那也说,妈喜欢听你说。” 

胎儿:“直到去年,你在莹博士的研究所当勤
杂工。” 

母亲:“从一开始,莹博士就很注意我。她有时上班早,遇上我在打扫走廊,总要和我聊几句,问我的身世什么的。后来有一天,她把妈叫到办公室去
了。” 

胎儿:“她问你‘姑娘,如果让你再生一次,

\newpage
你愿意生在哪里?’” 

母亲:“我回答‘当然是生在这里啦,我想生
在大城市,当个城里人。’” 

胎儿:“莹博士盯着妈看了好半天好半天,笑了一下,让妈猜不透的那种笑,说:‘姑娘,只要你
有勇气,这真的有可能变成现实。’” 

母亲:“我以为她在逗我,她接着向我讲了记
忆遗传那些事。” 

莹博士:“我告诉你妈妈,我们的研究已经形成了这样一项技术,修改人类受精卵的基因,激活其中的遗传记忆,这样,下一代就能够拥有这些遗传记
忆了!” 

母亲:“当时我呆呆地问博士,他们是不是想
让我生这样一个孩子?” 

莹博士:“我摇摇头,告诉你妈妈:‘你生下

\newpage
来的将不是孩子,那将是……’” 

胎儿:“‘那将是你自己。’你是这么对妈妈
说的。” 

母亲:“我傻想了好长时间,明白了她的话:如果另一个人的脑子里记的东西和你的一模一样,那他不就是你吗?但我真想不出那是一个什么样的娃娃

莹博士:“我告诉她,那不是娃娃,而是一个有着婴儿身体的成年人,他(她)一生下来就会说话(现在看来还更早些),会以惊人的速度学会走路和掌握其它能力,由于已经拥有一个年轻人的全部知识和经历,他(她)在以后的发展中总比别的孩子超前二十多年。当然,我们不能就此肯定他(她)会成为一个超凡的人,但他(她)的后代肯定会的,因为遗传的记忆将一代代地积累起来,几代人后,记忆遗传将创造出我们想像不到的奇迹!由于拥有这种能力,人类文明将出现一个飞跃,而你,姑娘,将做为一个
伟大的先驱者而名垂青史!” 

母亲:“我的孩儿,就这样,妈妈有了你。”
\newpage



胎儿:“可我们都还不知道爸爸是谁呢?” 

莹博士:“哦,孩子,由于技术方面的原因,你妈妈只能通过人工授精怀孕,精子的捐献者要求保密,你妈妈也同意了。孩子,其实这并不重要,与其他孩子相比,父亲在你的生命中所占的比例要小得多,因为你所遗传的全部是母亲的记忆。本来,我们已经掌握了将父母的遗传记忆同时激活的技术,但出于慎重只激活了母亲的,因为我们不知道,两个人的记
忆共存于一个人的意识中会产生什么后果。” 

母亲(长长地叹息):“就是只激活我一个人
的,你们也不知道后果啊。” 

莹博士(沉默良久):“是的,也不知道。”

母亲:“博士,我一直有一个没能问出口的问题:你也是个没有孩子的女人,也还年轻,干嘛不自
己生一个这样的孩子呢?” 

\newpage

胎儿:“阿姨,妈妈后来觉得你是一个很自私
的人。” 


母亲:“孩儿,别这么说……” 

莹博士:“不,孩子说的是实情,你这么想是公平的,我确实很自私。开始我是想过自己生一个记忆遗传的孩子,但另一个想法让我胆怯了:人类遗传记忆的这种未激活的隐性很让我们困惑,这种无用的遗传意义何在呢?后来的研究表明它类似于盲肠,是一种进化的遗留物。人类的远祖肯定是有显性的、处于激活状态的记忆遗传的,只是在后来的漫长岁月中,遗传的记忆才渐渐变成隐性。这是一个不可理解的进化结果:一个物种,为什么要在进化中丢弃自己的一项巨大的优势呢?但大自然做的事总是有它的道理,它肯定是意识到了某种危险,才在后来的进化中关
闭了人类的记忆遗传。” 

母亲:“莹博士,我不怪你,这都是我自愿的
,我真的想再生一次。” 

\newpage

莹博士:“可你没有,现在看来,你腹中怀着的并不是自己,而仍然是一个孩子,一个拥有了你全
部记忆的孩子。” 

胎儿:“是啊,妈,我不是你,我能感觉到我脑子里的事都是从你脑子里来的,真正是我自己的记住的东西,只有周围的羊水,你的心跳声,还有从外
面透进来的那红黄红黄的弱光。” 

莹博士:“我们犯了一个致命的错误,竟然认为复制记忆就能从精神层面上复制一个人,看来完全不是这么回事,一个人之所以成为自己,除了大脑中的记忆还有许多其它的东西,许多无法遗传也无法复制的东西。一个人的记忆像一本书,不同的人看到时有不同的感觉。现在糟糕的是,我们把这本沉重的书
让一个还未出生的胎儿看了。” 

母亲:“真是这样!我喜欢城市,可我记住的
城市到了孩儿的脑子中就变得那么吓人了。” 

胎儿:“城市真的很吓人啊,妈,外面什么都
\newpage

吓人,没有不吓人的东西,我不生出去!” 

母亲:“我的孩儿,你怎么能不生出来呢?你
当然要生出来!” 

胎儿:“不啊妈!你……你还记得在西套村时
,挨爷爷奶奶骂的那些冬天的早晨吗?” 

母亲:“咋不记得,你爷爷奶奶常早早地把我从被窝拎出来,让我跟他们去清羊圈,我总是赖着不起,那真难,外面还是黑乎乎的夜,风像刀子似的,有时还下着雪,被窝里多暖和,暖和得能孵蛋,小时
候贪睡,真想多睡一会儿。” 

胎儿:“只想多睡一会儿吗?那些时候你真想
永远在暖被窝里睡下去啊。” 


母亲:“……。好像是那样。” 


胎儿:“我不生出去!我不生出去!!” 

\newpage

莹博士:“孩子,让我告诉你,外面的世界并不是风雪交加的寒夜,它也有春光明媚的时候,人生
是不容易,但乐趣和幸福也是很多的。” 

母亲:“是啊孩儿,莹博士说的对!妈活这么大,就有好多高兴的时候:像离开家的那天,走出西套村时太阳刚升出来,风凉丝丝的,能听到好多鸟在叫,那时妈也真像一只飞出笼子的鸟……还有第一次在城市里挣到钱,走进大商场的时候,那个高兴啊,
孩儿,你怎么就感觉不到这些呢?” 

胎儿:“妈,我记得你说的这两个时候,记得很清呢,可都是吓人的时候啊!从村子里出来那天,你要走三十多里的山路才能到镇子里赶上汽车,那路好难走的。当时你兜里只有十六块钱,花完了怎么办呢?谁知道到外面会遇到什么呢?还有大商场,也很吓人的,那么多的人,像蚂蚁窝,我怕人,我怕那么
多的人……” 


沉默…… 

\newpage

莹博士:“现在我明白了进化为什么关闭人类的记忆遗传:对于在精神上日益敏感的人类,当他们初到这个世界上时,无知是一间保护他们的温暖的小屋。现在,我们剥夺了你的孩子的这间小屋,把他扔
到精神的旷野上了。” 

胎儿:“阿姨,我肚子上的这根带子是干什么
的?” 

莹博士:“你好像已经问过妈妈了。那是脐带,在你出生之前它为你提供养料和氧气,孩子,那是
你的生命线。” 


两年以后一个春天的早晨。 

莹博士和那位年轻的母亲站在公墓里,母亲抱
着她的孩子。 


“博士,您找到那东西了吗?” 

“你是说,在大脑中的记忆之外使一个人成为
\newpage

自己的东西?”莹博士像自言自语地问道。 

初升的太阳照在她们周围的墓碑群上,使那无
数已经尘封的人生闪动着桔黄色的柔光。 


“爱情啊你来自何方,是脑海还是心房?” 

“您说什么?”年轻的母亲迷惑地看着莹博士
。 

“呵,没什么,这只是莎士比亚的两句诗。”
莹博士说着,从年轻母亲的怀中抱过婴儿。 

这不是那个被激活了遗传记忆的孩子,那孩子的母亲后来和研究所的一名实验工人组成了家庭,这
是他们正常出生的孩子。 

那个拥有母亲全部记忆的胎儿,在那次谈话当天寂静的午夜,拉断了自己的脐带,值班医生发现时,他那尚未开始的人生已经结束了。事后,人们都惊奇他那双小手哪来那么大的力量。此时,两个女人就
\newpage

站在这个有史以来最小的自杀者小小的墓前。 

莹博士用研究的眼光看着怀中的婴儿,但孩子却不是那种眼光,他忙着伸出细嫩的小手去抓晨雾中飞扬的柳絮,从黑亮的小眼睛中迸发出的是惊喜和快乐,世界在他的眼中是一朵正在开放的鲜花,是一个美妙的大玩具。对前面漫长而莫测的人生之路,他毫
无准备,因而准备好了一切。 

两个女人沿着墓碑间的小路走去,年轻母亲从
莹博士怀中抱回孩子,兴奋地说: 

“宝贝儿,咱们上路了!”

\end{document}
