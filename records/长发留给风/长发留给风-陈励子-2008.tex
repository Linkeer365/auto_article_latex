\documentclass{article}
\usepackage[utf8]{inputenc}
\usepackage{ctex}

\title{长发留给风\footnote{Click to View:\url{https://web.archive.org/web/20221025122945/https://rentry.co/uwte9}}}
\author{陈励子}
\date{2008-10}

% \setCJKmainfont[BoldFont = Noto Sans CJK SC]{Noto Serif CJK SC}
% \setCJKsansfont{Noto Sans CJK SC}
% \setCJKfamilyfont{zhsong}{Noto Serif CJK SC}
% \setCJKfamilyfont{zhhei}{Noto Sans CJK SC}
% \setlength\parindent{0pt}

\begin{document}
\CJKfamily{zhkai}

\maketitle


\Large


离别时,转瞬间的彻悟。 


羊皮纸 

追忆往事时,我们不无诧异地发现原来每个人
生命的轨迹都早早地画好在一张古旧的羊皮纸上 

那张神秘丰腴的图纸藏在造物主积尘的书橱的一角。是的,一张浅棕色柔韧绵软的羊皮,将解决我们关于生命尽头的一切困惑一一而这些曾经困扰着苏格拉底的问题仍困扰着我们,并且一一据那些先贤说
一一它们仍将让后人也世代代迷惘下去。 

我惊愕于造物主的精明:正是这些我们穷尽毕生之力仍无法解决的问题,残忍地划定了人类永远不
\newpage
能逾越的界限。甚至当智慧的小部分人已清醒地认识到了那张羊皮纸的存在时,他们对此依然无能为力。
 

我们一生下来就踏上了寻找羊皮纸的旅程。孩提时父辈告诉我们,人生是由无数的岔路口构成的,有一条通向终点。于是,我们终身穿梭在生命的迷宫里,始终为每一个岔路口百分之几的几率奔波着,殊不知这无穷无尽的偶然串成的神秘巧合,本来就带有某种必然的意味。最终,我们来到了迷宫的出口,片刻的雀跃之后,暮然发现迷宫的人口和出口原来是同一个地方:向想起旅途中我们曾为每一次岔道口前的抉择精打细算、钩心斗角……我们猛然意识到原来那些曾把我们踩在脚下、曾被我们踩在脚下的人最终都会和我们走到同一个终点:只是有的人先到了,有的
人还没到罢了。 

我们又意识到,我们本是可以说说笑笑地走出
迷宫的,全然不必这么大费周折。 

终于,我们又意识到,或许我们根本就没有必
\newpage
要走进去一一这些为偶然忙碌终身的人,无不融化在了生命尽头浩渺的必然之中,多少巧取豪夺、快意恩仇,泯然一笑。人们在意识到这一点的那一刻,通体透亮耀眼,痛苦和欢欣在一瞬间都变得无比尖锐、明澈,我们感到了来自天堂的呼唤,于是,整个躯体在明艳的碧空中化作片片白羽,轻软如丝,扶风而上一一这是明净的彻悟,我们甩开了生来即背负的那沉重的十字架,随即身轻如燕温榆怕坐汽车。关于汽车的噩梦始于她小时候每礼拜六坐车进城学琴的经历,那时她大概四五岁。她记得她坐的是一种现在已不多见的小公共汽车,车上约摸有二十来个人的座位,过道里总是塞得满满当当的。她还记得那豆绿色油迹斑斑的软座椅总有一股烟头熏过的苦味,茶色的贴窗纸总是卷了边儿。其他的她都不记得了一这不能怪她,对于痛苦的经历我们总是趋向于忘记她怕汽车里那种裹着灰尘和汗臭的汽油味,这种浓烈的气味首先引起的是她浑身上下皮肤紧绷那种疙疙瘩瘩的不舒服,还有脖颈里毛毛糙糙的异样躁动接有是变换姿势一平仰、侧比,总之怎么都不得劲,然后是额头上冰凉的层薄汗,手心里擦紧的汗渍,其次是胃里的痉挛,再次才是不断涌到口腔里的酸水。晕车的人忌出汗,你越惦
\newpage
记着晕车就越是要晕,莫不如什么都不去想上车就睡觉。这简单的道理是别人告诉给她的,具体是谁她也记不清了。令她恶心的主要是那种混杂浊重的气味,于是她学会了屏气,像感冒了的人那样只用嘴呼吸。她强迫自己不去想关于晕车和呕吐的种种一一就像那人说的一一上车就闭上眼睛睡觉。这招很奏效,于是她练就了一番人睡的本领。在以后的人生中,无论道路如何坎坷、心情如何杂乱,她总能叹息一声便安然人梦。今天看来,这该是她小时候苦苦学琴的最大收获。要说再有让她无法忘怀的那该是下了车后开始正常呼吸的瞬间那种令人反胃的汽油味和尘土味一一是的,那是噩梦般的小公共汽车的残迹。尽管这样,这一瞬间的恶心让她觉得很兴奋,因为最艰难的终于过去了,以后的空气清新而芳香,享用不尽一她有种大难不死的侥幸。以后的十几年里,她的生命始终是被各种各样似曾相识的气味点燃的。气味像酒窖一样藏有她最独特神秘的记忆缘定这东西有种苍凉的华美,仿佛是熏着香草的大红花鸟绣缎,曳在身后柔曼的一袭,海浪般饱胀光滑,衔住了猎猎风声,把风声的棱角磨得浑圆厚实温榆不妨就是披着大红绣缎的古典美人,风中她青丝妩媚纷乱的背影和窈窕绰约的身姿,
\newpage
让你无法想象那张苍白的脸和顾盼流转着清冷的孤独
的眼神 


卡列尼娜 

温榆的一生都用来想象别人的生命如何度过,沉溺于对别人生活的构思是她最后的奢侈。这也许是她格外爱黑夜的原因,黑夜给了她肆无忌惮地想象的机会。她爱自己故事中的主人公,用最最醇美的字眼给他们起芳香的灵性的名字。最初的那些情节是最唯美坦荡不过的,湖蓝、草绿拼接成淡雅明丽的色块伊甸园里亚当夏娃雪白的胴体。唯美的故事是很容易穷尽的,因为情节诞生于人类的邪念。突然有一天温榆厌恶了亚当夏娃的冰清玉洁,她要悲剧。这时来的不是苹果和毒蛇,而是安娜·卡列尼娜。小孩子对事物的判断标准永远是幼稚得惊人而又准确得可怕的。温榆在第一时间内爱上了藏蓝色封面上安娜忧郁的眼神和这个黑色天鹅绒股质地的名字,于是,她故事里女主人公的名字就叫安娜。即使到了后来,温榆仍然执拗地爱着安娜·卡列尼娜这个名字。有些名字固然也是美的,但两三个飘逸忧伤的字眼凑在一起,难免滞
\newpage
涩而伤于尖巧,美得单薄,美得弱不禁风,让人一眼便看穿了人物瑟瑟发抖的命运,是怨女还是弃妇。安娜·卡列尼娜则不然,她可以是农场主家快活能干的胖姑娘,可以是浪荡的欢场女子,甚至可以是深褐色眼睛的精明的总统夫人…安娜·卡列尼娜,平凡而温婉的刻痕,猜不透人物身后究竟是悲剧还是喜剧·许多年后的一个大雪天,她风尘仆仆地赶到火车站,结果发现她要坐的那列火车刚刚开走,远处只有隐隐约约的汽笛声。漫天的飞雪、黑貂大衣、笨重的木箱和青白色的霜让她的心忽然揪疼了一下,她想到了呼啸的暴风雪和浓烟滚滚的火车,安娜和年轻的军官渥伦斯基他们的初遇并不美丽,安娜的死仿佛是为了修正
他们相识那刻残留的遗憾 

一那点微弱的烛光在刹那间把整个世界照得恍如白昼,然后便无声无息地熄灭了,她的世界从此坠入死寂。安娜·卡列尼娜温榆想到了自己的名字。她感到了舌尖那种淡青色的苦味,仿佛是悬着红丝绳的一小块温润的清凉的玉坠儿。直觉告诉她,生橄榄色调的女子的命运是凄凉的。她想到了一连串名字:霍青桐、阿碧、温青青、公孙绿萼、越女阿青……这些
\newpage
名字念久了嘴唇都会发苦发涩。好在她并不害怕凄凉,与孤独一样,凄凉是种迷狂的意绪,醉人的美丽。


玉山 

温榆清晰地记得军训时她第一次在水房里见到司徒玉山时的情景。已经是深夜了,四处都静悄悄的,只有水房里还灯火通明,有谁在那儿洗东西。温榆觉得眼睛很干涩,于是就搭着毛巾走进了水房。她从小就如此,眼晴整夜整夜地疼,时刻网着淡红的血丝一也许那是黑夜深情的啄痕。温榆进来时,司徒玉山正在水池边用力地洗什么东西,水开得很大,哗哗的水流激起的水珠弹子一样霹雳飞散,喷溅得到处都是,水龙头连着锈迹斑斑的水管咯岐咯咕地颤抖着。她的头垂得很低,温榆看不见她的脸,只看见她瘦削的肩膀和枯黄的长发随着手上的动作上下抖动着。温榆敏锐地想到这是个藏着心事的女孩儿。如果没猜错的话,此刻大概是心乱如麻一一震耳欲聋的水声掩饰不了她粗重的喘气声。温榆没跟她打招呼,把旁边的水龙头轻轻扭开了一股涓涓的水流,把毛巾打湿了敷眼睛。过了一会儿,她把毛巾从眼睛上拿开,觉得双目
\newpage
凉沁沁的很舒服。温榆拧干了毛巾,不由得盯着司徒玉山看了一会儿,那水龙头的抖动让她觉得莫名其妙
的不舒服 


“我走了,你洗完早点睡吧。嗯?” 

温榆说完后有意顿了一顿,期待着她的回答不知是水声太响没听见还是她有意不想说话,她依然固执地搓着衣服,头也不抬一下。一缕长发垂到手腕,被水打湿了,她用湿淋淋的手挽了一下,把头发别在耳后。温榆看到了她苍白素淡的面容和过于尖巧的鼻子,心头一丝院隐的失落一也许她期待的本应是位艳若桃李冷若冰霜的冷美人,可偏偏看到的是张最平淡
不过的面孔。 


知道她叫司徒玉山是后来的事了。 

温榆喜欢揣度别人名字的含义。她想到了李白《襄阳歌》里“清风朗月不用一钱买,玉山目倒非人推”的句于。她依稀记得里头的典故说的是一位美男子,萧疏朗轩,亭亭如白玉铸成的高山,他喝醉了,
\newpage
醉倒在地,便如“玉山倒地”,依然是那样俊美。她素来喜欢美丽的传说,不知该是怎样多情的人才能讲出这样醉人的故事一一尽管如此,单从名字讲,“玉
山”两个字用作女孩子的名字,不免太潦草了。 

司徒玉山留给温榆的印象是游丝飞絮一般轻灵
不可捉摸的。 

和她在一起的那一段日子格外地忧伤美丽,而仔细想来却又并没有留下什么细节可供回忆,似乎有
人把一切都轻轻拂去了一样。 

回忆时脑海中那片刺眼的空白让温榆觉得很不可思议。许多年后,当读到张爱玲“一恨鲋鱼多刺,二恨海棠无香”时,温榆突然想到了司徒玉山和她曾经写下的半阕词。海棠美艳而香气寡淡,若即若离欲言又止。只在烟雨濠濠时,隔着一抹嫣红,你才可以
嗅到它满腹平淡如水的心事 


葡萄 

\newpage

玉山病了。其实刚开始只是一个小小的沾水的伤口,一夜水房里砭人肌骨的寒冷。她发烧了,军训完了之后一直烧着,伤口也没好。她母亲死了,父亲酗酒,玉山也没把发烧的事跟谁说,只是一直病着。


一个月后,玉山病重了。 

那些破碎了的东西,司徒玉山总是舍不得扔掉。纵然完好时平淡无奇,一旦残缺便也焕发出一种异世的凄美。她藏有一个深紫的丝绒匣子,里面塞着撑断的发带,青花碗的碎片,豁了口的染着墨渍的金笔尖,指甲刮破的丝巾,虫蛀了的康乃馨,还有半阕词。她对温榆说她常常能从紧锁的匣子里听到一阵遥远神秘的哭泣和灵歌,或许那哭声一直都在,而她只是
偶尔听到。 

那是个冬日的黄昏,枯瘦荒寂,淡淡的日影洒满了病床。雪白的被单被落日罩上了一层淡淡的米色,玉山苍白的脸上描满了扶疏的树影。温榆坐在床边听玉山喃喃地讲话,她讲她深紫色的丝绒的小匣子,一件一件地讲里面东西的来历,讲她听到的灵歌和哭
\newpage
声。玉山对温榆说,对于美好的东西我们充满怜爱仿佛是爱一嘟噜紫莹莹的葡萄,爱它鲜亮的水色和扑鼻的甜香;而对那些破碎的东西,我们更像是爱一窖醇厚的葡萄酒,固然爱它酒香绵密,更爱它身后尘封的神秘岁月和古旧温婉的气韵。因此她爱残损,因为唯
有破碎才是永恒的。 

在从医院回家的路上,温榆想到了刚才玉山的比喻。说实话,玉山的比喻是拙劣的,修辞的苍白让人联想到她平淡的面容。然而,无论是对面目还是对
修辞,温榆都是宽容的。 

是的,玉山说唯有永恒才是美的,这正是她一生苦难的源头。在认定了美的同时用最脆弱的永恒去解释美,她经历的种种不幸简直带有某种愚弄的意味司徒玉山之所以会把温榆当成她倾吐的对象,她一生唯一的朋友,只是因为温榆也认定了美,而温榆对美
的解释是全然不同于她的。 

在温榆看来,但凡不是永恒的东西,多多少少都具备了某种成为美的可能性,偏偏永恒是永远和美
\newpage
不搭界的,或许,如果有一种例外的话,被时光遗弃,独守一隅的永恒天生便是魅惑的,比如王国维所说有“众芳芜秽,美人迟暮之感”的《楚辞》。而这种美感形成的根源恰恰在于它终于肯放下骄矜,从永恒的神龛坠落红尘。温榆也爱残损,但在片刻的心碎和悲凉之后,她感到的是由衷的释然和满目的青葱,在
打碎了永恒的一刹那,她也原谅了人间。 

因此,温榆绝不爱历久弥新的葡萄酒,葡萄酒没有对于衰老和死亡的迷惑没有那种对沉甸甸的岁月的珍重和感念。是的,没有顾虑和忧愁的美味与过于一帆风顺的爱情和刚留下悬念就知道谜底的故事一样,被剥夺了那种让人意乱情迷、忽忽如狂的美。因此,她宁愿对着一串玛瑙般丰美饱满的葡萄默默垂泪为它今朝的容颜老于昨晚。第二天清晨,温榆接到了从医院打来的电话,电话那边是一个中年男子沧桑的声音,可能是玉山的父亲。他说他很抱歉这么早打电话来,他说玉山在凌晨时死了。片刻的沉默之后,他告诉温榆玉山死的时候没有痛苦,嘴角还带着微笑。他说-辈子也没见玉山那么笑过,一点悲苦的意思也没有。他还说要谢谢她,他说她的电话号码是玉山给他
\newpage
的,玉山死之前只想见她一个人,要把她的小匣子给她。温榆紧握着电话的手是冰凉的,说不出一句话来。她觉得口渴,嗓子疼得要着火了。那种如鲠在喉的感觉是那么真切,但她的眼睛没湿,真奇怪她居然一点也不想流泪。她平静地说叔叔您别伤心,玉山从来没有比任何人少活哪怕一点儿,该有的快乐该有的痛苦她全有了,她没什么遗憾的。电话那头的男人开始抽泣了,他说他对不起玉山。温榆说生死有命,谁也没有办法,您好好地活着,玉山才能安心地去。显然那个男人没有听进去温榆的话,他一直低声说,他对不起玉山,对不起玉山……上学路上,温榆顺路去医院取回了司徒玉山留给她的那个深紫色的丝绒小匣子:撑断的发带,青花碗的碎片,豁了口的染着墨溃的金笔尖,指甲刮破的丝巾,虫蛀了的康乃馨,小纸片上的半阕词。一样不少。她看了那半阕词,《玉楼春》,是玉山自己填的:海棠无花踏雪寻,半卷嫣然半盏吟,红颜自古多薄命。她笑了,韵用得不对,想了想把“玉楼春”三个字画了,结末添上一句便成了一
首七绝。她小声地念了一遍: 


\newpage

海棠无花踏雪寻,半卷嫣然半盏吟。 

红颜自古多薄命,一顿寒香对水云。后来温榆还在小匣子里发现了一踏枯黄的长发。她相信玉山这样做是别有用意的,可是她不想去猜晚上睡觉时,温榆忽然想到了玉山说的葡萄和葡萄酒。也许,她想,我们本是应该给玉山葡萄酒的,给她永不凋谢的芳醇的迷醉。要么我们就什么也不给。偏偏我们许诺给了她远方的葡萄酒,带来的却是一串串的葡萄。我们把希望和绝望一并赋予了她。是的,司徒玉山想要的是葡萄酒,即使这酒在温榆看来不过是潺潺动听的美丽
的谎言。 

是的,我们本应该给玉山葡萄酒的。葡萄酒不会改变她的命运,但至少这样一来她的死还能算得上
是悲剧。而如今呢,玉山是彻头彻尾的一出喜剧。 


旗袍 

病床上的玉山曾和温榆聊起同班的女孩郁祺祺

平心而论,郁祺祺并非如何惊世骇俗地美,她
\newpage
的迷人之处在于对自我之美的自觉留意。举手投足之间恰到好处的拿捏,不经意间流露出的对自己外表的
欣赏和满足让郁祺祺变得精致起来。 

这固然为真假道学之辈所不容,然而在这个时代,关乎道德的种种都被重新定义了,风流女子只是
许多女子中的类罢了。 

郁祺祺的可爱在于她身上真实的媚俗,波俏的人间风骨。你可以对她敬而远之,但你永远不能否认自己身上若隐若现的那个郁祺祺。本质上讲,高雅的女性都是胚胎中的郁祺祺。关于这类女人的一切,看到的都是假的。或者说永远只能看到她想让你看到的那一部分,而这一部分往往与真实情况相悖。比如,她们喜欢在落地橱窗前驻足,并不要欣赏橱窗里的名
表,而是为了照照镜子。 

温榆对郁祺祺充满了依恋,那种甜美的堕落深深地诱惑着温榆一一用一根发簪松松地盘起头发,有意散在脸旁柔柔的一缕,隐隐约约,垂到腮边,剪开的秋水一般妩媚而清悠的眸子,酒红色的唇间一根明
\newpage
明灭灭的香烟一一这是属于上个世纪初大上海的古典的诱惑,流金的岁月,如歌的情怀,让那些道德上的
质问显得苍白无力 

欲望,叫嚣它无耻的人多半也在脸红心跳。旗袍,这是并不让人反感的欲望。长期压制欲望的最终结果是毁灭性的聚变的灾难,上个世纪德意志民族的癫狂证明了这一点;而欲望横流又的确让人吃不消。于是,我们小心翼翼地把束紧得如纺梭一般的旗袍割开一条长长的口子,精明的裁缝懂得适可而止。落剪之处,风情款款又不肆无忌惮,这样,欲望一旦被冠以高雅而节制的头衔,也就无可非议了。节制的欲望和清醒的疯狂,这中间是微妙而惊险的一跃,雾霭沉
沉的矛盾下面紧紧地勾着清晰纯美的逻辑线索 

是的,美丽总是以矛盾的形式展出。人之多情和岁月之无情之间的矛盾:现实之紧人滞重和理想之空阔清爽之间的矛盾;未来之翘首茫茫和回忆之顾盼郁郁之间的矛盾:诗词、废墟、秋千、落叶……隐现的都是这样让人深陷其中无以遣怀的困境。文章信美却漫赢得天涯羁旅的姜白石曾做悲凉之叹:“酒醒明
\newpage
月下,梦逐潮声去。”梦如若真是“有情风万里卷潮来,无情送潮归”,那得失反倒让人心平气和、了无牵挂了。你的梦碎了,偏偏你还要用一腔的热血去研磨,把棱角分明的梦磨成了鲜红的面粉。终于,你长叹一声转身离去,这时鲛人衔来蔷露滴落在你的梦上,碧烟袅袅龙涎香。后人至此,惊愕于此香之神异脱尘却不见来人。哪知你刚刚走远,一身风尘、满鬓霜华,不忍顾盼。抛开封建卫道士如何扼杀纯洁爱情这样痴人说梦的话不谈,“春宽梦窄的确称得上吴梦窗的惊世之言,同时这也是旗袍之魅惑的最好诠释。郁祺祺的放荡是旗袍的放荡,春宽梦窄四个字,淡淡的矛盾郁结其中,吴梦窗自己不也说么,“西园日日扫林亭,依旧赏新晴”,这份幽隐俳侧之情自有一番深美曲折,却并不愁眉苦脸。放荡。是的。如果有一场无需考虑任何后果的惊险欲望游戏,温榆愿意在游戏中充当郁祺祺的角色,在旗袍的放荡中撕破那些逻辑陷阱。然而,人生的可怜之处就在于没有任何修正的机会,换句话说,没有一场游戏是没有赌注的这份赌注太危险,而温榆又太懦弱。温榆善于冷嘲热讽人生的虚幻,而不善对峙这份虚幻。司徒玉山则不然。她永远在对峙而不善嘲讽,因此她的对峙苍白无力,一
\newpage
旦意识到这种对峙的空虚,她便在错愕的清醒中头重脚轻,茫然跌落谷底。如果司徒玉山有温榆的那份儒弱,而温榆有司徒玉山的对峙的勇气,她们也许会交换彼此的那份羊皮纸,而司徒玉山将会比现实中幸福的温榆更幸福,温榆将会比现实中悲惨的司徒玉山更悲惨。当然,就像上面说到的那样,人生没有假设的机会,因此,一切结论都是也许。郁祺祺和司徒玉山是同班同学。温榆和司徒玉山初次相逢的那个晚上,水房里,她执意要洗去郁棋祺掸落在她床单上的烟灰。以后每一次见到郁祺棋,她说,她都会被烟味呛到
。温榆笑了,说,你是被人间烟火呛到了。 


静夜 

三岁那年,在日光灯的强光下,温榆开始了人生中第一次真正意义上的清晰的哲学思考。当她母亲咔嗒一声拉动了灯绳,平静地说该起床了温榆时,温榆的眼前一明一暗闪动了好几次,直到尖锐刺眼的光芒冰凉地扎穿了整个房间的每一个毛孔。她不愿动一一这时她正眯缝着眼睛蜷缩在被窝里盯着天花板上雪白的日光灯着,刺眼的灯光让她觉得很不舒服。母亲
\newpage
见她赖在床上不起来,就掀走了她的被子一如果是别的孩子,没有被窝的温暖,准会乖乖起床。可是温榆从不眷恋被窝的温暖,她眷恋的是黑暗。只属于黑夜的温厚和宁静每每让她幸福得有鼻酸的感动,而强光的到来让她觉得粗鲁可笑。她爱黑夜:黑夜是被梦境和白天遗弃的世界。白天的遗弃是坦然地擦肩而过,而梦的遗弃是委婉的背叛。是的,梦境小心翼翼地背叛了黑夜:黑夜无法渗入人们的梦境,而绝大多数人们的梦里阳光普照,处处是流光溢彩的贪婪和欲望。黑夜与梦同床异梦,梦瞒着黑夜怀上了白天的孩子。这样灵妙的念头如果说出来是会让成年人脸红的。只是那时她不知道什么是沉睡的自我的苏醒,她更想不到造物主赋予她的那种神秘力量一一深藏于这个早熟少女心底的自我,一旦被唤醒便一发不可收拾。温榆用最微妙的方式小心翼翼地保护着夜的尊严。强光是对夜的强暴,因此她拥一盏清丽的烛,让摇曳的烛光茶匙一般调匀了夜色,满室稠厚的黑夜被烛光的搅动调稀了,空灵玄妙,令人心旌摇动。黯淡的色彩把夜撑开了,墨黑的舌头又从房间的四角咕溜溜地舔舐掉不断浸染开的朦胧的光晕,在这花开花谢一般的明暗起伏中,四处涨落着淡淡的夜的哀伤。夜的魅惑之处
\newpage
在于明净一点流水孤云流淌出的冷冷寒意,有种碎玉裂帛的美。夜给了温榆冰冷而尖锐的指尖,指尖是她的触角,夜用冰凉的唇吻别了她被日光灼伤的触角。这样凄清的夜让人清醒而疯狂,是的,我们完全可以清醒着疯狂一一疯狂的对立面不是清醒而是麻木。从人生中的某一刻起当我们开始学会抖拗精神,意气昂扬地为荣誉而战时,我们便陷入了一生的麻木,无路可逃。终于,有人察觉到了这种严肃背后的荒谬,宣称人生苦短自当及时行乐一要么是追名逐利的清醒的麻木,要么是及时行乐的不清醒的疯狂,儒道之争自古已然,从来都是五十步笑百步,这真是个可笑的世界大多是在深夜,是的,一种无法言说的失落排山倒海地朝我们压过来。这样的时刻,我们的心境与那种美丽的夜的哲学是如此契合,那种来自清醒之疯狂的召唤是如此恳切,生命终极意义的磁场与我们如此接近,在强烈的吸引下我们的整个身心都在剧烈地颤抖着……失落背后是羊皮纸上困扰我们的大秘密可是我们没有勇气转过身去。静夜揭示了种人类无法逃遁的
生存困境。 


\newpage

墓地 

五年后的命运是铺天盖地袭来的,让人措手不及一一昨天还纤渥凝碧的蓉蓉绿草,南风中一弯琴弦般的银色河水,今天清晨就被芦荻点染成了寒凉凄迷
的一岸风烟。 

初秋的清展。玉山的基,丛丛的衰草翻阅着碑
石额前令人心悸的荒凉 

玉山裹着一袭纯白的浪漫在温榆的视线中渐行渐远,悠然圣洁如神庙前初沐握发的希腊女子,如今,玉山消失了,那袭白袍在秋光中泛黄,散发出隔世
的催泪的悲凉。 

终于温榆明白,生者凭吊死者,永远不是为了回忆,而是为了遗忘。她静静地垂手伫立在玉山基前,关于玉山的一切稀释在四周淡淡芜杂的水样秋树中唯有滤干这秋阳晨雾、抽尽这微尘造音,玉山的记忆
方能熔铸成一滴浓金,可惜温榆做不到,她做不到 

当秋阳殁晨雾散,微尘落建音淡,玉山依然零
\newpage
落成点点的微光,摇曳在草尖,嵌在粗砾基石的沟沟壑壑上。然后,月光烟树鹰胧,玉山便真的消散在光
和影中了。 

浓淡有致的记忆诱惑着温榆,她的心跳收缩着堇色季节的质感,她试图晚猩记忆,仿佛是救助站在为一个垂死的老人做人工呼吸,她以自己吐气的节奏匀舒地挤压着老人嶙峋的胸,期待着冰冷的身体的回声,可是她做不到,做不到,往昔的记忆已经僵硬成
了一尊雪雕,再多的努力也是徒劳· 

关于玉山的记忆飞散在了秋日的墓地,温榆走了,玉山从没这样遥远而阳生过,陌生到迎面走来只
需点头致意,微笑都可忽略不计 


墓地在郊外。 

回家的路上,温榆晕车了,连自己也觉得莫名其妙,怎么会,十年来的第一次,有关晕车的记忆早已淡漠了。恐慌中的颠簸、手心里的薄汗、汨汨分泌的酸水、油腻的座椅和汽油味……就在这一瞬间的眩
\newpage
晕中,温榆感到了一种奇特的、从未经历过的要冲破胸腔的颤抖,在歇斯底里的战栗中抛回到了十五年前清晰明净而苦楚的记忆中,那个不愿坐车学琴的小姑娘,面色苍白跌跌撞撞地下了车,扶住一棵树哇哇地
吐,身后是滚滚红尘蚂蚁般的人群。 

就在这一瞬间,温榆被拍岸而来的悲凉席卷到了时光渐渐柔缓的滔滔细浪中,温存而凄楚的酸风打磨着她的逻辑,直到这一切消失殆尽,只剩下赤裸的意识和更加猛烈的心跳,移植的情境、似曾相识的气味孕育出的神秘的宿命。无力对抗的记忆压得她胸口发闷,她大喊停车,便跌跌撞撞地走到了路边,扶住一棵树哇哇地吐起来,然后就蹲下来放声大哭。墓地清冽了她十五年在时光中熟稔的记忆,时光中她清清白白如同一片处女的唇,依然是那个小姑娘,晕车,
喜欢黑夜,讨厌学琴。 

玉山死了,这个撕破了伪装的事实鲜红如血。

放声大哭。温榆想她恐怕是玉山死后唯一来过的人,至于玉山的父亲,一年前到了南方,据说从此
\newpage
滴酒不沾,就这样把女儿彻彻底底托付给了衰草斜阳古冢这样也好,衰草斜阳无情,却凝然远方,古冢本身是固化的名利场,却从不屑于尘世的欺骗,无心机的司徒玉山,托付给自然的无情,阴森的死亡,远胜于托付给修辞的谎言。荒草,流淌着枯荣中神秘的宇秩序。玉山死了。

\end{document}
