\documentclass{article}
\usepackage[utf8]{inputenc}
\usepackage{ctex}

\title{再造\footnote{Click to View:\url{https://web.archive.org/web/20220317113849/https://linkeer365.github.io/Linkeer365ColorfulLife3/4141866999/}}}
\author{Anther}
\date{2022-08-04}

% \setCJKmainfont[BoldFont = Noto Sans CJK SC]{Noto Serif CJK SC}
% \setCJKsansfont{Noto Sans CJK SC}
% \setCJKfamilyfont{zhsong}{Noto Serif CJK SC}
% \setCJKfamilyfont{zhhei}{Noto Sans CJK SC}
% \setlength\parindent{0pt}

\begin{document}
\CJKfamily{zhkai}

\maketitle


\Large

诡异的梦,梦中的我好像变成了女生(更准确的说是变成了小女孩的那种身心结构),跟我妈妈从儿时的小道上一蹦一跳地走回来,妈妈也变回了年
轻时候的样子。 

我妈妈说单位有人养了热带鱼,其中有一条特别迟钝特别笨拙,每次有人向着水面招手,所有的鱼都会屁颠屁颠飞快地出来抢吃的,但是它根本反应不过来,所以就只能吃剩下的。不光这样,别的鱼会在箱庭里到处游来游去,打打闹闹上蹿下跳,这条鱼就只喜欢躲在一个三角形的由石头砌成的区域当中孤零
零地呆着。 

距那位养鱼的同事在讲,这条鱼之所以会有这种性格,可能跟它一直被别的鱼欺负但又不知道怎么
\newpage
反抗有关,它原本的尾巴是一条长长的丝带,现在都被啄得只剩可怜的几缕,不仅这样,每次一到投食的时候,总会有几条鱼总是乐于让它吃不到东西,它们联合起来扇动尾巴,让它根本看不到、接触不到新鲜
的鱼料。 

“太可怜了,但是xx你知道吗,要让人们为此其他所有的鱼是很难的,要让人们单独捞出那条小鱼倒还好说,所以我跟他们说,我家里有个多余的鱼缸,要不然就让这个可怜的小家伙好好地呆着吧,他
们很快就同意了。” 

“但是这样还不是终点哦,顶多算一个小小的休止符,妈妈知道它不是天生孤僻的,就算是孤僻的小家伙也一定期望着有人能理解他陪伴他,所以我又去跑了几家水族店,专门向老板问了有没有特别孤僻、特别不好养活的那种鱼,好多老板都以为我疯了,但是最后一家老板愿意帮我找,找得特别久,出了好
多的汗。” 

“他问,你在哪里上班,在哪里上学,我就把
\newpage
情况跟他说了,他随后又说,二十年前有个家里介绍的小妹妹特别黏我,也是那种会为了这些奇怪的小事东奔西跑最后特别狼狈地回去的人,可惜当时嫌弃她学历低,脑子笨,不漂亮,说话还有一股浓浓的广西口音,就没有跟她在一起。要是她能有你这么好的学
历就好了。” 

“就算那样大概也没有用的吧,当时的我轻狂浅薄,总觉得能遇到一个条件差不多的,特别是要能在亲戚朋友面前大充门面的人,后来才知道某些人一旦错过,再也不会有类似的人出现了,你觉得这条小鱼已经很难找了,但是像你一样像她一样的人,更是
可遇不可求。” 

“嗯嗯,老板就是这种话很多很有热情的老板,我看他在账本上还写着梅涅劳斯定理的证明呢,应该也是一个很聪明的老板,总之我们先把这条小鱼安顿好,然后看看这两条小鱼合不合拍,好不好啊。”
 

我说“好”(正是在这个时候发现自己的发音
\newpage
方式不对,就算小时候我也不会用这么高的部位发音,但并没有因此醒来,只好任由其继续),然后我妈妈搬出旧鱼缸,先是仔仔细细擦了一遍鱼缸,往水里鼓入氧气,用离子布把水体的硬度降下来等等操作,然后她就让我先去睡个午觉,她还有不少工作没有准
备清楚。 

等我来到床上时妈妈也正好要睡下了,接着她就很自然地把我搂在怀里,就在此刻我发现情况不对劲了,这样的搂抱有点舒适得过头了,甚至有些令人晕眩,在现实中我跟别人拥抱的时候大多都是礼节性的,再亲密一点也就只是时间久一些,但我从没想过被抱在怀里的时候会感觉整个后脑都像要炸裂一样的
舒适。 

在那个当下我什么都无法思考,只是想着被拥抱得久一点再久一点,快乐像某种气泡一样从周身一点一点地涌上来,仿佛时间就此停滞了,所有的流光在我鼻尖缓缓流过,疯狂的同时是一片曼妙的寂静,过了好一阵子等到一切复归平静的时候,还有一点淡

\newpage
淡的、接近于泫然而泣的感伤情绪。 

之后的梦简要概括一下,就是我走到了一间文学馆,里面有各式各样的图书,但最吸引我的还是阁楼里散发着面包香味的绘画,我穿着袜子一级一级地向上爬去(注意到脚几乎是我现在大小的1/3),楼顶上长鼻子的老画家,他从花里拿出一枚戒指给我,说是能赐予我智慧,结果我一戴上那枚戒指就开始
呼吸困难。 

那个老画家也很慌张,高声喊着“完了完了,快来人那”(然而并没有人来,另外你倒是说清楚具体发生了什么啊),随后冲上来用刻刀和抹布想帮我把戒指剔下来,但是每刮一次戒指就紧一点,我也就感到更加难以呼吸,(并且这个时候我已经知道是在梦里了就更加绝望),最后我只能求求他把我的小指
砍下来。 

之前是真的不知道作为女孩子被搂抱在怀里会有这么奇妙的感觉,并且这个应该还跟环境的气氛有
关,在特定的气氛下进行才能有比较好的效果。 

\newpage

既然拥抱会有这么强烈的反应,一旦被冷落应该也会很痛苦,这就非常难办了,因为现在社会是比较原子化的,像那种家庭式的劳动场景已经变得很少了,进而拥抱等等肌肤接触的情形也会变少,如此一来习惯了拥抱的这些人就会变本加厉地在亲人爱人面
前进行求索,以期补偿自己在相应需求上的亏空。 

还有一点就是过于令人陶醉的温暖很容易让人误以为这就是什么不言而喻的情、一往而深的爱的了,其实这只是身体在特定的情况下做出的一些反应而已,也有环境的因素,比如这次就是在枕头上滴了几滴花露水而已。 带给你温暖、令你感觉舒适的人不一定是对的,很多时候只是被身体的反应和环境的气
氛诱惑了。 

有人说会不会跟睡姿有关,我那时候的睡姿可参考这张,枕头是搁在手臂下面的。

\end{document}
