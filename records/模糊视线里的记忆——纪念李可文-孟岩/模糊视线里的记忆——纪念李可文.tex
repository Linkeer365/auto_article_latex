\documentclass{article}
\usepackage[utf8]{inputenc}
\usepackage{ctex}

\title{模糊视线里的记忆——纪念李可文}
\author{孟岩}
\date{2007-07-22}

% \setCJKmainfont[BoldFont = Noto Sans CJK SC]{Noto Serif CJK SC}
% \setCJKsansfont{Noto Sans CJK SC}
% \setCJKfamilyfont{zhsong}{Noto Serif CJK SC}
% \setCJKfamilyfont{zhhei}{Noto Sans CJK SC}
% \setlength\parindent{0pt}

\begin{document}
\CJKfamily{zhkai}

\maketitle


\Large

6月中旬以来我的眼睛就一直有病,近20多天都不能读书和使用电脑,整个世界在我眼睛里都是模糊的,各种各样的眼药水像下雨一样往眼里滴。我常常怀念CSDN的blog,听说这里每天都有上千的新帖子,真的有点跃跃欲试。

不过我原本以为,我重返CSDN Blog的第一个帖子应该是“劝君注意眼睛健康”之类的,没想到今天来到CSDN,却看见了可文的死讯,而且竟然已经过去了五天...

把字体调到最大,屏幕看上去仍是模模糊糊,眼睛里涌着一些东西,不知道是刚刚点下去的抗病毒干扰素,还是什么别的。也许这个时候使用电脑对我很不好,但是我觉得应该写些什么,为这位可爱的朋友。

我认识可文是在联想工作的时候。我的一个同事是猫扑上的常客,也是模拟器界小有名气的人士,当时我们商量在Pocket PC 2000/2002上开发一款任天堂红白机的模拟器。有了这样的模拟器,我们就可以把几十个70年代人耳熟能详的经典游戏移植到我们的产品上,这无疑将大大提高我们产品的吸引力。我对游戏是一窍不通,但很清楚simulator的开发对一个程序员而言意味着什么。我于是大为质疑这项计

\newpage 

划的可行性,直到同事把可文带到面前。

可文给我的第一印象,就是怪异,我至今也保留着这个第一印象,他天生就应该是做游戏的,因为他长得就像游戏里的人。他头颅很大,在孱弱的肩膀上,显得负担过重。所有的头发都向上竖起,好像哪些专门做过发型的歌星。鼻子上永远有一颗红色的包,艳若桃李。他第一次见我的时候,已经做过手术,身体很虚弱,穿着淡淡竖条的白色长袖上衣,最上面两个扣子是松开的。下面是淡淡竖条的深色长裤。这深色的长裤里是他的腿,我不曾见过的,虚弱的腿。他不能长时间走路,然而当他走路的时候,两条腿绵软无力,面条一样绵软。给我的感觉,那不是一个寻常生活中的人,完全应该是一个漫画家笔下的人物,是为了什么特殊的使命出现的人物。

我们请他吃了一顿饭,事情就这么定了。他面对一个崭新的平台,没有表现出任何的畏难情绪,也没有提太多的要求。开发样机送过去之后不久,一个可以run起来的程序就寄了过来。这真的令我很惊讶。然而这只是一连串惊讶的开始。在两三个月的时间里,他就

\newpage 

基本完成了全部的开发工作,DreamNES for Pocket PC,这个技术难度最大的软件项目,最后跑在我们所有人前面ready。记得那段时间,研发部里所有的同事,一到午休时间就捧着我们的产品,叮叮当当地打着游戏,“魂斗罗”、“双截龙”、“绿色兵团”、“三只小猪”...,那些儿时熟悉的游戏音乐在办公室里此起彼伏,好不热闹。

那段时间我们又见了几次,逐渐熟悉起来,我们有的时候发现了bug,就大叫一声,那个李可文死了没有?没死就让他赶快改!于是一片笑声。其实我们当时不了解他的病情,并不知道这种玩笑话,其实对他来说并不幽默,甚至有些残忍。当着他的面,我无非是关照他多注意身体。其实这种话是善意的废话,或许你说的时候很诚心,对于一个无可奈何的病人来说却没有太多的意义,人开朗的话可能客套几句,不开朗的人可能根本懒得反映什么。他不算是开朗的人,所以对于病情不太爱说什么,但是一提起游戏就不同了,两个眼睛里立刻充满了光,北京人的侃劲也上来了。他和我的那位游戏迷同事,经常针对某个游戏的某一关中的某一个细节开展讨论,滔滔江水连

\newpage 

绵不绝,黄河泛滥一发而不可收。对于我关心的模拟器实现技术,他的谈兴则小得多,也许在他看来,游戏才是目的,模拟器不过是个手段而已。不过从跟他少量的交谈中,我了解到模拟器的开发工作是非常艰苦的,不但涉及到软硬件技术和关键技术的突破,而且更需要无比的耐心和毅力。大多数情况下没有任何资料和指导,全凭自己的经验,遇到问题经常要连蒙带猜,反复攻关。很多个别bug根本没有道理可言,只能出现一个解决一个。这种工作,没有十二分的毅力和决心是决不可能完成的。我有的时候想,他的这份毅力是从哪里来的?为什么我们身上没有?难道非要我们的腿也软的像面条一样,非要我们也被死亡所威胁,才能够拿出抵死不悔的气概吗?又或者,对于我们大多数人来说,到那一天可能更加消沉和绝望。

最后一次见面,记不清是什么时候了,大概是去年上半年的一天。简单地打了个招呼,开了一个玩笑,就挥手永别了。之后不久我来到CSDN,曾经跟他联系,让他写写文章,讲讲模拟器开发的技术。他问了问情况,考虑之后拒绝了

\newpage 

。我可以理解。他对于现在时兴的那些新理论与新方法没什么兴趣,“东西都是做出来的”,模拟器程序的规模不太大,有固定的模式和结构,难度主要是在于具体问题的应对和处理上,他不觉得有什么可写的,“没什么意思”,他说。是的,可文首先是个gamer,然后才是个programmer,但是却是最纯粹的programmer。

我是无神论者,我不相信有什么天堂、上帝、天使、佛、来生等等,所以也不会用这些美丽的虚幻的字眼来祝福死去的可文。他去了,消失了,几天之后连他的躯体也将化作青烟,不复存在了。我再也不能跟他开玩笑了,也没有机会邀请他给我写文章了。我遗憾,我难过,但是这却是铁一般的事实。

偶尔考虑生死的问题。迟早有一天,我们这世上所有的人都将死去。这不是一件坏事,死亡决不是一件坏事。可文向我们证明了死亡可以是辉煌生命的一个篇章,死亡可以激发我们的生命,让它迸发出光芒,而不仅仅是让我们迫不及待地去饮食男女,声色犬马。人随时可能死去,就算不死去,眼睛也可能会瞎,肢体可能会残废,器官可能会患病。任何一种可能性发生,我们都将开始

\newpage 

一种完全不同的生活,今天还让你极为烦躁的事情,可能随时会成为永远的不可能。体会到这种紧迫感,生命也许就不再仅仅是赚钱吃饭睡觉阿谀奉承勾心斗角了。

每次见到可文的时候,他身边总有一位北大的女孩子,好像是兰州人,姓氏不记得了,名字好像是晶晶。她搀扶着可文,总是爽朗的笑。告别可文那天她应该在场吧,她一定会哭的吧,她哭得时候,泪水一定是晶莹的吧。

\end{document}
