\documentclass{article}
\usepackage[utf8]{inputenc}
\usepackage{ctex}

\title{世界之血,祛除永昼——评林跃行刺案\footnote{Click to View:\url{https://web.archive.org/web/20220325011235/https://twitter.com/ultramarine471/status/1506996242093514753}}}
\author{群青}
\date{2022-03-24}

% \setCJKmainfont[BoldFont = Noto Sans CJK SC]{Noto Serif CJK SC}
% \setCJKsansfont{Noto Sans CJK SC}
% \setCJKfamilyfont{zhsong}{Noto Serif CJK SC}
% \setCJKfamilyfont{zhhei}{Noto Sans CJK SC}
% \setlength\parindent{0pt}

\begin{document}
\CJKfamily{zhkai}

\maketitle


\Large

由此上溯到一千八百四十年,从那时起,为了反对内外敌人,争取民族独立和人民自由幸福,在历次斗争中牺牲的人民英雄们永垂不朽。——《人民
英雄纪念碑·碑文》 

我只感到唏嘘,其实事情根本不至于此的,保安固然态度恶劣,但真正的巨奸大恶是坐在高处的,就算你的刀有40米长,也捅不到远在国外遥控的他
们。 

不过如果保安态度本身很恶劣,死掉像条狗这么随便也是很正常的吧,至少对上面的人来说,的确是可能产生一些类似于“秦王目眩良久”的症状的。
但秦王再怎么目眩良久,最终也依靠强大的军
\newpage
队灭亡六国,虽说后面也因暴政二世而亡,但终归没有亡在自己手上,正义的指标被后人抢走了,心里还
是挺过意不去的。 

才杀一个人就被统治阶级拿下,其实是并不值得的,这顶多只能勉强算一个及格分,考虑到最近恶性事件的频率与烈度,出于良好的表率与示范,就再加个5分好了。
杀一个人有什么意思,要杀就整片整片、整群整群的杀,在杀的过程中明确杀的标准杀的范围,团
结志趣相同、利益一致的人,让自己成为法律。 

当下的中国,最普遍的一个现象是政府心甘情愿地给资本力量做小,以身相许、自荐枕席还不够,因为五湖四海内,各种不同的“解放生产力”的方式对应着种种不同的资本力量,所以还要做到人尽可夫
。 

我这边没有瞧不起性工作者的意思,怡红院这去处虽然饱受诟病,但人家知道出卖身体也是一种劳动,应当有合适的劳动所得,嫖完给钱天经地义;
\newpage

但到我国政府那边就不是这么个情况了,但凡有点资本算个皮包公司,都愿意用你最喜欢的姿势来迎合你,给你各种好处不说,还能亲手帮你摆平刁民
,享受特色优势。 

不过这也解决了困扰我多年的一个问题,即“有事钟无艳,无事夏迎春”的钟夏儿女到底能不能融合成一人,取长补短,在不同场合扮演不同角色。后来我才发现,这个难题早就被我国政府的各项举措中被轻松化解,也让我不禁感叹,原来那么用力用心地去讨好某个人的时候,一个人的底线能有多低、姿态
能有多卑微。 

不要说头部的那些股份,就这种垃圾企业,一年贡献gdp不到百万分之一,能给政府纳tm多少税、带来多少财政收入啊,转移支付之后实际分到你手上的又能有几何,你还卑微到这个地步,还把相信着你、深爱着你的底层劳动人民通通当作婢子通房送过去,遭受无尽的凌辱以后,怯懦懦地吞下不良资产
的肮脏精液。 

\newpage

我们之所以还会感到痛,还会感到屈辱,是因为我们身上还残存着烙印,它代表我们光辉的从来和英雄的历史,代表所有成功或失败、光明或卑劣、扎实或投机的一切抗争,这些不屈的物质融进了我们的血。正是因为带着这样的烙印,流着这样的血,我们才感到有义务用我们的抗争与牺牲来换取后辈们的光
明与希望。 


世界之血,祛除永昼。 

有人觉得,那我就是想不到那么多啊,我就只知道往我身边作威作福的人捅上一刀,没法考虑那么多啊,完全是出于了结私怨的角度出发了,这样可以吗。
这是完全可以的,毕竟有个及格分在那里,又比如你说要忍气吞声,让矛盾的气球吹大一点再捅破
,以期达到更大的烈度,这也可以。 

又有人说,我这辈子没啥出息,一遍懂得正义却一遍拿不出反抗的勇气,就只能加加油助助威什么的,这也是可以的,人性向来如此嘛,只要不做坏事
\newpage

都有保底的得分。 

我的判断标准其实很明确,假如这世上完全没有道德这一说,这样一来站在你的角度会怎么选、怎么做,一般来说这就是可取的。毕竟多数情况下我们都跟资本家非亲非故,没有说情感上过意不去,或者非得跟老板穿一条裤子才甘心的。
况且杀人怎么就不道德,道德牌肉铺是全球t
m独你一家别无分店是吧。 

我对道德是没有兴趣的,我只对哪些人把控甚至垄断了道德的定义和使用权感兴趣。我对多数人政治和少数人政治也是没有兴趣的,我只在意现有秩序能不能促使社会进步,能不能带领人们走向他们认知
中的、具体到方方面面的美好生活。 

我们当然很快会死,但与无尽的永寂相比,属世界的那一刹那,又能比我们的生命长上多少呢。

\end{document}
