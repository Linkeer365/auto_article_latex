\documentclass{article}
\usepackage[utf8]{inputenc}
\usepackage{ctex}

\title{政治对文学的正确和错误应用\footnote{Click to View:\url{https://web.archive.org/web/20220729152919/http://libgen.is/book/index.php?md5=8C7F488ACD514AE29C2450B88A76ECFC}}}
\author{卡尔维诺}
\date{1976-02-25}

% \setCJKmainfont[BoldFont = Noto Sans CJK SC]{Noto Serif CJK SC}
% \setCJKsansfont{Noto Sans CJK SC}
% \setCJKfamilyfont{zhsong}{Noto Serif CJK SC}
% \setCJKfamilyfont{zhhei}{Noto Sans CJK SC}
% \setlength\parindent{0pt}

\begin{document}
\CJKfamily{zhkai}

\maketitle


\Large

在收到贵方座谈会的发言邀请时,我的第一个想法,同时也是此类情况下习惯性的想法,就是试图回想是否最近某篇关于文学和政治的文章,或者众多关于这个主题的讨论中的某个发言,可以给你们读一读。然后,我发现自己并没有写过任何成熟的东西:很多年来,我都没有针对这个主题写过或者说过什
么。 

再想想,这件事情非常奇怪。在我青年时代的那些年,也就是从1945年开始,到整个50年代甚至以后,最重要的问题正是作家与政治之间的关系。可以说,任何讨论都围绕着这一点展开。我这一代人,可以被定义为同时开始涉足文学和政治的一代。
 

\newpage

最近几年,我经常关心政治上的事情正在如何发展,文学上的事情进展又如何。然而,当我想到政治,我仅仅会考虑政治;当我想到文学,也仅仅会考虑文学。今天,当要同时面对这两个问题的时候,我会产生两个范畴彼此分离的感觉,而且,两种都是空虚感:一个我能够相信的空洞的政治设想和一个我也
能够相信的空洞的文学设想。 

不过,从更深的层次来看,我意识到,青年时代引绊我们的那个关于政治和文学关系的结,时至今日尚未解开。它的剩余部分尽管松垮和磨损,却仍旧影响着我们前进的脚步。20世纪60年代所发生的事情,深刻地改变了很多我们曾经打过交道的概念,尽管我们仍然在用以前的名字称呼它们。这一切对于我们社会的未来会产生怎样根本性的影响,我们仍不得而知。不过我们明白,一场思想上的革命,一个知识层面的转折已经发生。假如我们需要为这个过程下一个概括性的定义,可以说,把人作为历史主体的想法已经过时了,将人拉下宝座的对立者,还应该被称作人,不过那是一个与之前完全不同的人。这意味着在整个星球上以指数形式增长的“数量众多”的人类
\newpage
;意味着大都市的爆炸,社会和经济的无法管理,不论它们属于怎样的体制;意味着经济和意识形态上的欧洲中心论;也意味着所有被排斥压迫、遗忘和沉默的人们提出的,获得所有权利的要求。我们用来对世界下定义、分类和规划的所有参照、种类对立面,都遭到了质疑。不仅仅是那些与历史价值紧密相关的因素,还包括从人类学角度来看好像稳定的因素(理性与神话、工作与生存、男性与女性),甚至是意义相反的基本词汇(肯定与否定。上与下,主体与客体)
。 

在最近的几年,我在政治和文学方面所关心的,是它们的能力不足以完成思想变化强加给它们的任
务。 

我认为,或许首先要更好地定义意大利文学这个微型世界内部的形势,才能解释60年代为我们带
来的新东西。 

在50年代,意大利文学(尤其是小说)雄心勃勃地希望能够表现当代意大利的伦理和社会意识。
\newpage
在60年代,这个奢望从两个方面遭到了攻击。在文学形式上,或者确切地说不仅是形式上的,还包括认识论和来世学的角度,新先锋对叙事文学展开了进攻和抗议,指责它多愁善感、过时、进行虚伪的安慰;只有通过语言空间和时间上的决裂,叙事文学才能够
对当代社会进行反映,并消除它的种种幻想。 

同时,在具有政治倾向的评论界这条阵线上,评论家中最极端的一派,对具有政治和社会立场的文学所奢望的典型性发起了进攻和破坏,指责它是民粹主义。在这条阵线上,先锋派,或者无论如何可以称作否定文学,同样在为他们的复仇做准备。这里所指的是那种并不奢望进行积极的教育,而仅仅是作为我
们所处境况的一个信号的文学态度。 

我们还需要考虑与这两条阵线同时存在的第三条进攻阵线,它也具有同样的重要性,那就是意大利文学的文化背景也正在进行完全的更新:语言学、信息理论,大众社会学。人种学和人类学、对于神话的结构研究、符号学、通过新方式使用的分析心理学,通过新方式使用的马克思主义,所有这些都成为惯用
\newpage
的工具,可以拆外文学物件,把它分解为最基本的元
素。 

我认为,在那个时刻,文学面临着一种前所未有的希望。在它的领域内,在战后论辩中,那些沉重而模棱两可的重大问题已被肃清。对于文学作品的分解,可以开启一条新评价和新结构化的道路。结果是什么呢?什么结果也没有,或者说恰恰与可以期待的结果相反。其中的原因,既来自文学运动的内部,也
来自它的外部。 

1968年学生运动中新的政治激进主义,在意大利表现为一种对文学的拒绝。他们的建议并非是“否定文学”,而是对于文学的否定。文学主要被指责为浪费时间,而与之相反的是唯一重要的事情则是行动。对于行动的崇拜,首先是文学的一个古老的灵感源泉,然而对于这一点的理解(或者正在理解)非
常缓慢。 

我想说,这种态度并非是完全错误的:它意味着拒绝一种所谓社会性的平庸文学,拒绝接受赋予有
\newpage
社会责任感的作家的一种错误形象。如此,就可以通过某种方式,而不是通过任何对于文学的传统而错误
的崇拜,去接近对于文学之社会功能的正确评价。 

不过,这也曾经(在这里,我用的是过去式,是因为我觉得有些事情已经发生了变化)标志着自我
限制、视野的局限和没有能力看到事物的复杂性。 

当政治家和政客对文学过于感兴趣,这就是不祥之兆(主要是对于文学,这是个不祥之兆),因为这将是文学处境最为危险的状况。不过,假如他们不愿意听人谈到文学,那同样是不祥之兆。这种情况不仅会发生在更加传统和麻木的资产阶级政治人物身上,也会发生在那些具有意识形态特色的革命者身上。后面这个不祥之兆主要是对于政治家而言,他们会对于任何可能对其语言的确定性造成疑问的语言的使用
表现出恐惧。 

无论如何,文学和政治上的两种新先锋派的相遇并没有发生。先锋派文学自感失去了他们期待的潜在的读者储备。50年代已经失败的作家恢复了原来
\newpage
的地位。文学不能留有空闲的位置:情况最糟糕的时候,这些位置会被蹩脚的作家占领;而最好的时候,
占据这些位置的会是传统型作家。 

最近几年,所有将问题过于简单化的政治态度都遭到了失败,越来越多的人意识到我们生活的这个社会的复杂性,尽管没有人能够夺望从口袋里掏出解决问题的方法。在当今的意大利,一方面,我们的机构状况不断恶化和腐败;另一方面,也存在着一种集
体的成熟和对自治之路的探索。 

在这种形势下,文学的位置到底在哪里?我要说,文学领域的形势并不比政治领域更有秩序。意大利小说,尤其当它涉及政治和近期历史的时候,在国内拥有广泛的读者群;并且,文学创作不再遵循三十年前的说教形式,而是采用提出问题的方式。另外,作家还面临着媒体的压力,促使他在报纸上发表文章,参加电视里的圆桌讨论,针对所有他可能明白或者不明白的事情阐述观点。作家或许可以在能够理解的政治讨论中,占据一个闲置的空间。这个使命看上去轻而易举(仅仅阐述一些普遍性的观点,而没有任何
\newpage
实际的责任,这样做太容易了),然而,它应该是作家肩负的使命中最为艰难的一项。政治语言越是变得抽象和令人厌烦,我们就越是能够感觉到需要一种不同的更加个人化的和直接的语言,尽管这种需要并没有被表达出来。这种语言也更具挑衅性。在当今的意大利,挑衅是最需要的公共功能。帕索里尼的生命,死亡,以及死亡之后造成的影响,使作家的角色被确
定为挑衅者。 

这一切当中存在着一个根本性的错误。我们要求作家在一切都以非人性的形式存在的世界里,保证被称作人性的讨论能够存活下去,还要安奈我们,因为任何其他的讨论和关系都已经失去了人性。可是人性意味着什么?通常情况下,它意味着情绪化的,感情的、天真的和不严肃的东西。很少有人相信文学具有严肃性。其实文学的严肃性,超越了如今统治世界
那些语言虚假的严肃性,而且与之形成对立。 

今年,诺贝尔奖授予埃乌杰尼奥·蒙塔莱。但是,如今很少有人记得,他的诗歌所具有的力量就是低声细语。他的诗歌作品,从来没有任何形式的强调
\newpage
,而是使用谦卑和怀疑的腔调。正是通过这种方式,诗人才得以使很多人倾听他的声音,而且对三代读者产生了巨大的影响。文学正是用这种方法开辟它的道路:假如文学的“效率”和力量”存在的话,那么它
们就会是这样的。 

然而在当今社会,假如作家希望被倾听,也和希望他们提出的想法能在公众中产生影响,并将他们每个自发的反应极端化,那么就要提高嗓门。然而,即使是最为答人听闻和爆炸性的观点,读者也会充耳不闻:一切仿佛如同风声,都不存在。评论界最多像是对待淘气的孩子那样,摇一摇头。所有人都知道,文字仅仅是文字,不会对周围世界造成任何影响,也不会对作家或者读者造成任何危险。在汪洋大海般的印刷品或者口头传播的文字中,诗人或者作家的文字
消失殆尽。 

这是文学力量的悖论:好像文学只有在受到人迫害的地方,才会表现出它真正的力量,并且向权威挑战。在我们这个宽容的社会和普遍性的文字泛滥当中,文学觉得自己只是被用来制造基种令人愉快的反
\newpage
差(尽管如此,难道我们疯癫到要为此而抱怨吗?但愿上天也希望专断势力能够理解,要想清除书面语言
的危险,最好的方法就是把它看得一文不值!) 

首先我们要记得,假如作家受到迫害,也就意味着不仅仅是文学受到迫害,很多其他类型的讨论和思想(首先是政治思想)也会受到禁止。在那些国家,叙事文学。诗歌和文学评论,特别是在政治上获得了一种特殊的分量,因为这些形式给予了所有无声的人话语权。我们生活在可以自由地进行文学创作的条件下,明白这种自由意味着一个社会处于运动当中,很多事情都在发生改变(变好或者变坏,这是另外一个问题)。即使在这种情况下,问题还是在于文学传递的讯息与社会之间的关系,或者更确切地说,是讯息与创造一个接受讯息的社会的可能性之间的关系。真正重要的是这种关系,而不是文学与政治当权者之间的关系。如今,统治者不能说已经将对于社会的领导权掌握在手中,无论是右派还是左派所进行的,民主还是专制的统治,都不能这样说。文学是一个社会自我意识的工具之一。当然,它不是唯一的工具,但它是最根本的工具,因为文学的渊源与很多类型的知
\newpage
识、准则,以及各种评论思想的形式的渊源彼此相连

总之,我相信,针对文学对政治的有用性,存
在两种错误的看法。 

一种是奢望文学能够闹明政治已经获得的真理,也就是相信所有政治价值的总和是首先产生的,文学只需要与它们相适应。这种观点隐含着一个可能会带来灾难的想法,那就是文学是某种装饰和多余的东西,政治却是某种固定和自信的东西。我觉得,只有糟糕的文学,或者粳糕的政治,才会想到类似的政台
教育功能。 

另外一种错误的看法认为,文学是由各种永恒的人类情感拼凑而成,如同政治倾向于忘记但又需要不时记起的一种人类语言的真理。从表面上看,这种想法给文学留下了更多的空间,实际上却是赋予它一种使命,也就是对已知事物进行确认;又或者仅仅是进行天真而初级的挑旦,包含着青年人的快乐所具有的那种新鲜与自发性。在这个概念背后,是对于文学需要维护的一系列既定价值的想法,存在着一个传统
\newpage
而一成不变的想法,那就是文学是特定真理的保管员。假如文学接受了这个角色,就会将自身的功能局限于安奈、维护、倒退。而在我看来,这种功能更多是
有害而非有利的。 

这就意味着政治对文学的任何使用都是错误的吗?我认为不是。就像存在两种错误的看法一样,还
存在着两种正确的看法。 

对于政治来说,文学的必要性首先在于给予无声者一种声音,赋予无名者一个名字,尤其是对于政治语言排斥或者试图排斥的那些东西。我所指的,是外在和内在世界的各种特征、形势和语言,是受到个人和社会压抑的倾向。文学就如同是一只能够听到政治所能理解的语言以外声音的耳杀;如同一双能够看到政治所能觉察的色阶以外颜色的眼睛。正因为作家是在孤独的状况下工作,才能探索任何人都不曾探索的地域,无论是自己的内心世界,还是外部的世界;还能发现早晚会成为集体意识的重要研究领域的东西

这还仅仅是文学的一个非常间接的,并非有意
\newpage
,而是偶然的功能。作家遵循着自己的道路,同时,偶然现象或者社会与心理的确定因素,也会带领他发现某些有可能对政治和社会行为具有重要性的东西。政治和社会观察者的任务,就在于不能让任何偶然的事件发生,要将自己的方法应用于文学实践当中,以
免忽视任何事情。 

不过,我认为文学还有另外一种影响。我不知道它是否更加直接,但是它一定更能代表文学的意愿,那就是制定语言、观点、想象、脑力劳动,以及事实之间关系模式的能力,总之,是创造(我所说的创造就是组织和选择)那种类型的价值模式。同时,它们又是美学和伦理学方面的价值模式,是每个行动规划,特别是政治生活中的关键性价值模式。所以,在排除了文学的政治教育功能之后,我还要重申,我相信存在着一种通过文学来进行的教育,一种只有通过困难的和间接的方式,只有当其中隐含着为实现文学严肃性而做出的艰苦努力,才能达成其效果的教育类
型。 

文学达到的任何结果,只要是严肃的,那么对
\newpage
于任何实践活动,对于旨在建立一种如此坚固和复杂,甚至能够包含无序世界的精神秩序的人,对于想要创造一种如此微妙和坚韧的方法,甚至可以认为是方法的缺失的人来说,这种结果就可以被认为是一套活
动纲领。 

我已经讲了两种正确使用文学的方法,但是,现在我要明确地指出第三种方法,它与文学那种以批评的眼光看待自己的方法相关。假如说文学曾经被看作一面反映社会的镜子,或者直接表达情感的方式,如今我们已经无法忘记,书籍是由词语、符号,以及创作方法构成的,我们永远不能忘记,书籍所传递的东西,有时候连作者本身也没有意识到,因为有的时候,书籍真正讲述的与本来决定要讲述的东西,可能会有所不同。在每本书中,都有一部分是作者的创作
,还有一部分是匿名和集体创作的结果。 

这种意识不仅仅影响到文学,也会有助于政治,令它发现自已有多大一部分仅仅是文字的结构,是神话和文学的类型。像文学一样,政治首先应该了解

\newpage
自己和怀疑自己。 

我要说的最后一点是,假如今天没有人能够自认为是无事的,假如在每个人所做或者所说的任何事情当中,我们都能够发现一个秘密的原因:作为白人,作为男性,作为享受既得利益的人,属于某种特定经济体系的人,或者是遭受某种神经官能症情结折磨的人,我们不应该因为这些而背负某种普遍的负罪感
,或者持有某种普遍性的指责态度。 

当我们觉察到自己的疾病或者秘密动机的时候,就已经开始将它们置于危机当中。重要的是,我们要接受这些动机,学会在它们的危机中生活。只有这样,我们才有可能变得与自己不同,也就是说,这是
唯一能够创造新的存在方式的方法。 



\end{document}
