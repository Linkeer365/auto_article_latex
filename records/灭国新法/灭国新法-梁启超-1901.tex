\documentclass{article}
\usepackage[utf8]{inputenc}
\usepackage{ctex}

\title{灭国新法\footnote{Click to View:\url{https://web.archive.org/web/20221012132904/http://www.bwsk.com/mj/l/liangqichao/000/020.htm}}}
\author{梁启超}
\date{1901-07-16}

% \setCJKmainfont[BoldFont = Noto Sans CJK SC]{Noto Serif CJK SC}
% \setCJKsansfont{Noto Sans CJK SC}
% \setCJKfamilyfont{zhsong}{Noto Serif CJK SC}
% \setCJKfamilyfont{zhhei}{Noto Sans CJK SC}
% \setlength\parindent{0pt}

\begin{document}
\CJKfamily{zhkai}

\maketitle


\Large

今日之世界,新世界也:思想新,学问新,政体新,法律新,工艺新,军备新,社会新,人物新,凡全世界有形无形之事物,一一皆辟前古所未有,而别立一新天地。美哉新法!盛哉新法!人人知之,人人慕之,无俟吾论。吾所不能已于论者,有灭国新
法在。 

灭国者,天演之公例也。凡人之在世间,必争自存,争自存则有优劣,有优劣则有胜败。劣而败者,其权利必为优而胜者所吞并,是即灭国之理也。自世界初有人类以来,即循此天则,相搏相噬,相嬗相代,以迄今日而国于全地球者,仅百数十焉矣。灭国之有新法也,亦由进化之公例使然也。昔者以国为一人一家之国,故灭国者必虏其君焉,潴其宫焉,毁其宗庙焉,迁其重器焉。故一人一家灭而国灭。今也不
\newpage
然,学理大明,知国也者一国人之公产也,其与一人一家之关系甚浅薄,苟真欲灭人国者,必灭其全国,
而不与一人一家为难。 

不宁惟是,常借一人一家之力,以助其灭国之手段。故昔之灭之人国也,以挞之伐之者灭之;今之灭人国也,以噢之咻之者灭之。昔之灭人国也骤,今之灭人国也渐。昔之灭人国也显,今之灭人国也微。昔之灭人国也,使人知之而备之;今之灭人国也,使人亲之而引之。昔之灭国者如虎狼,今之灭国者如狐狸。或以通商灭之,或以主债灭之,或以代练兵灭之,或以设顾问灭之,或以通道路灭之,或以煽党争灭之,或以平内乱灭之,或以助革命灭之。其精华已竭、机会已熟也,或一举而易其国名焉,变其地图之颜色焉;其未竭未熟也,虽袭其名仍其色,百数十年可也。呜呼,泰西列强以此新法施于弱小之国者,不知
几何矣!谓余不信,请举其例: 


一,征诸埃及。……(编者删) 


\newpage

其二,征诸波兰。……(编者删) 


其三,征诸印度。……(编者删) 


其四,征诸波亚。……(编者删) 


其五,征诸菲律宾。……(编者删) 

以上所列,略举数国,数之不遍,语之不详。虽然,近二百年来,所谓优胜人种者,其灭国之手段,略见一斑矣。莽莽五洲,被灭之国,大小无虑百数十,大率皆入此彀中,往而不返者也。由是观之,安睹所谓文明者耶?安睹所谓公法者耶?安睹所谓爱人如己、视敌如友者耶?西哲有言:“两平等者相遇,无所谓权力,道理即权力也;两不平等者相遇,无所谓道理,权力即道理也。”彼欧洲诸国与欧洲诸国相遇也,恒以道理为权力;其与欧洲以外诸国相遇也,恒以权力为道理。此乃天演所必至,物竞所固然,夫何怪焉!夫何怼焉!所最难堪者,以攘攘优胜之人,托于岌岌劣败之国,当此将灭末灭之际,其将何以为
情哉?其将何能已于言哉? 

\newpage

天下事未有中立者也,不灭则兴,不兴则灭,何去何从,间不容发。乃我四万万人不讲所以兴国之策,而窃窃焉冀其免于灭亡,此即灭亡之第一根源也。人之爱我何如我之自爱,天下岂有牺牲己国之利益,而为他国求利益者乎?乃我四万万人,闻列强之议瓜分中国也,则眙然以忧;闻列强之议保全中国也,
则释然以安;闻列强之协助中国也,则色然以喜。 

此又灭亡之第二根原也。吾今不欲以危言空论,惊骇世俗,吾且举近事之一二,与各亡国之成案,
比较而论之。 

埃及之所以亡,非由国债耶?中国自二十年前,无所谓国债也;自光绪四年,始有借德国二百五十万圆,周息五厘半之事;五年,复借汇丰银行一千六
百十五万圆,周息七厘; 

十八年,借汇丰三千万圆,十九年,借渣打一千万元,二十年,借德国一千万元,皆周息六厘;廿一年,借俄、法一万万五千八百二十万元,周息四厘;廿二年,借英、德一万万六千万元,周息五厘;廿
\newpage
四年,借汇丰、德华、正金三银行一万万六千万圆,周息四分五厘。盖此二十年间(除此次团匪和议赔款未设),而外债之数,已五万万四千六百余万元矣。大概总计,每年须偿息银三千万圆。今国帑之竭,众所共知矣。甲午以前,所有借项,本息合计,每年仅能还三百万,故惟第一次德债,曾还本七十五万,他无闻焉。自乙未和议以后,即新旧诸债,不还一本,而其息亦须岁出三千万。南海何启氏曾将还债迟速之
数,列一表如下: 

债项五万万元,周息六厘,一年不还,其息为三千万元,合本息计,共为五万万三千万元。使以五万万三千万元,再积一年不还,则其息为三千一百八
十万元,本息合计五万万六千百八十万元。 

再以五万万六千百八十万元,积八年不还,则其息为三万万三千三百万元有奇,本息合计,为八万
万九千五百万元有奇。 

再以八万万九千五百万圆有奇,积十年不还,则其息为七万万零八百万元有奇,本息合计,为十六
\newpage

万万零三百万元有奇。 

再以十六万万零三百万元有奇,积十年不还,则其息为十二万万六千八百万元有奇,本息合计,为
二十八万万七千一百万元有奇。 

然则不过三十年,而息之浮于本者几五倍,合本以计,则六倍于今也。夫自光绪五年至十八年,而不能还一千六百余万元之本,则中东战后三十年,其不能还五万万元之本明矣。在三十年以前之今日,而不能还三千万元之息,则三十年后,其不能还二十三万万元之息又明矣。加以此次新债四万万五千万两,又加旧债三之一有奇,若以前表之例算之,则三十年后,中国新旧债本息合计,当在六七十万万以上。即使外患不生,内忧不起,而三十年后,中国之作何局面,岂待蓍龟哉?又岂必待三十年而已,盖数年以后,而本息已盈十万万,不知今之顽固政府,何以待之
? 

夫使外国借债于我,而非有大欲在其后也,则何必互争此权,如蚁附膻,如狗夺骨,而彼此寸毫不
\newpage
相让耶?试问光绪廿一年之借款,俄罗斯何故为我作中保?试问廿四年之借款,俄英两国何故生大冲突,几至以干戈相见?夫中国政府,财政困难,而无力以负担此重债也,天下万国,孰不知之?既知之而复争之若鹜焉,愿我忧国之士一思其故也。今即以关税、厘税作抵,或未至如何启氏之所预算,中国庞然大物,精华未竭,西人未肯遽出前此之待埃及者以相待。而要之债主之权,日重一日,则中央财政之事,必至尽移于其手然后快,是埃及覆辙之无可逃避者也。而庸腐奸险、貌托维新之疆臣如张之洞者,犹复以去年开督抚自借国债之例,借五十万于英国,置兵备以残同胞,又以铁政局之名,借外债于日本。彼其意岂不以但求外人之我信,骤得此额外之巨款,以供目前之挥霍,及吾之死也,或去官也,则其责任非复在我云尔?而岂知其贻祸于将来,有不可收拾者耶?使各省督抚皆效尤张之洞,各滥用其现在之职权,私称贷于外国,彼外国岂有所惮而不敢应之哉?虽政府之官吏百变,而民间之脂膏固在,彼搤我吭而揕我胸,宁虑本息之不能归赵?此乐贷之,彼乐予之,一省五十万,二十行省不既千万乎?一年千万,十年以后不既万万乎?此事今初起点,论国事者皆熟视无睹焉,而不
\newpage
知即此一端,已足亡中国而有余,而作俑者之罪,盖擢发难数矣。中央政府之有外债,是举中央财权以赠他人也;各省团体之有外债,是并举地方财权以赠他人也。吾诚不忍见我京师之户部、内务府,及各省之市政使司、善后局,其大臣长官之位,皆虚左以待碧眼虬髯辈也。呜呼!安所得吾言之幸而不中耶?吾读
埃及近世史,不禁股栗焉耳。 

不宁惟是,国家之借款,犹曰挫败之后,为敌所逼,不得不然。乃近者疆吏政策,复有以借款办维新事业为得计者,即铁路是其已事也。夫开铁路,为兴利也,事关求利,势不可不持筹握算,计及锱铢。而凡借款者,其实收之数,不过九折,而金钱涨价,还时每须添一二成。即以一成而论,其入之也,十仅得九,其还之也,十须十一,是一转移间,已去其二成,而借万万者短二千万矣。此犹望金价平定,无大涨旺,然后能之。若每至还期,外国豪商高抬金价,则不难如光绪四、五年时之借项,借百万者几还二百万,是借款断无清还之期,而铁路前途,岂堪设想耶?夫铁路之地,中国之地也,借洋债以作铁路,非以铁路作抵不可;路为中国之路,非以国家担债不可。
\newpage
即今暂不尔,而他日稍有嫌疑,则债主且将执物所有主之名,而国家之填偿,实不能免。以地为中国之地也,又使今之债主,不侵路权,而异时一有龃龉,则债主又将托办理未善之说,而据路以取息,势所必然。以债为外洋之债也,以此计之,凡借款所办之路,其路必至展转归外人之手而后已。路归外人,而路所经地及其附近处,岂复中国所能有耶?(以上一段,多采何氏《新政治基》之议,著者自注。)试观苏彝士河之股份,其关系于英国及埃及主权之嬗代者何如

呜呼,此真所谓自求祸者也!此所以芦汉铁路由华俄银行经理借款,而英国出全力以抗之;牛庄铁路之借款于汇丰银行,而俄国以死命相争也。诚如是也,则中国多开一铁路,即多一亡国之引线。又不惟铁路,凡百事业,皆作如是观矣。今举国督抚,亦竞言变法矣。即如其所说,若何而通道路,若何而练陆军,若何而广制造,若何而开矿务,至叩其何所凭借以始事,度公私俱竭之际,其势又将出于借款。若是则文明事业,遍于国中,而国即随之而亡矣。呜呼,往事不可追,吾犹愿后此之言维新者,慎勿学张之洞

\newpage
、盛宣怀之政策以毒天下也。 

俄人之亡波兰也,非俄人能亡之,而波兰之贵官豪族,三揖三让以请俄人之亡之也。呜呼,吾观中国近事,抑何其相类耶!团匪变起,东南疆臣,有与各国立约互保之举,中外人士,交口赞之,而不知此
实为列国确定势力范围之基础也。 

张之洞惧见忌于政府,乃至电乞各国,求保其
两湖总督之任; 

又恃互保之功,蒙惑各领事,以快其仇杀异党之意气;僚官之与己不协者,则以恐伤互保为名,借外人之力以排除之。岂有他哉?为一时之私利,一己之私益而已。而不知冥冥之中,已将长江一带选举、黜陟、生杀之权,全移于外国之手。于是扬子流域之督抚,生息于英国卵翼之下,一如印度之酋长,盖自此役始矣。第四次惩治罪魁名单,荣禄等以广大神通,借俄法两使之力,以免罪谴。于是京师、西安之大吏,生息于俄人卵翼之下,一如高丽之孱王,又自此役始矣。一国之中,纷纷扰扰,若者为英日党,若者为俄法党,得附于大国,为之奴隶,则栩栩然自以为
\newpage
得计。噫嘻,吾恐非至如俄人筑炮台以临波兰议院之时,而衮衮诸公,遂终不悟也。人不能瓜分我,而我先自分之,开群雄以利用之法门。彼官吏之自为目前计则得矣,而遂使我国民自今以往,将为奴隶之奴隶而万劫不复。官吏其安之矣,抑我国民其安之否耶?
 

呜呼!吾观天下最奇最险之现象,则未有如拳匪之役者也。列强之议瓜分中国也,十余年于兹矣。事机相薄,妖孽交作,无端而有义和团之事,以为之口实。皮相者流,孰不谓瓜分之议将于今实行乎?而岂知不惟不行而已,而环球政治家之论,反为之一大变,保全支那之声,日日腾播于报纸中;而北京公使会议,亦无不尽变其前此威吓逼胁之故技,而一出以温柔噢咻之手段。噫嘻,吾不知列强自经此役以后,何所爱于中国,而方针之转变,乃如是其速也?一面骂吾民之野蛮无人性,绘为图画,编为小说,尽情丑诋,变本加厉,惟恐不力;一面抚摩而煦妪之,厚其貌,柔其情,视畴昔有加焉。义和团之为政府所指使,为西后所主持,亦既万目共见,众口一词矣,而犹然认为共主,尊为正统,与仇为友,匿怨相交,欢
\newpage
迎其谢罪之使,如事天神,代筹其偿款之方,若保赤子。噫嘻,此何故欤?狙公之饲狙也,朝三暮四则诸狙怒,朝四暮三则诸狙喜。中国人之性质,欧人其知之矣,以瓜分为瓜分,何如以不瓜分为瓜分?求实利者不务虚名,将大取者必先小与。彼以为今日而行瓜分也,则陷吾国民于破釜沈舟之地,而益其独立排外之心,而他日所以箝制而镇抚之者,将有所不及。今日不行瓜分而反言保全也,则吾国民自觉如死囚之获赦,将感再造之恩,兴来苏之颁,自化其前此之蓄怨积怒,而畏折、歆羡、感谢之三种心,次第并起,于是乎中国乃为欧洲之中国,中国人亦随而为欧洲之国民。吾尝读赫德氏新著之《中国实测论》,((P<R>OBERT HARTAS ESBSAYSON THECHINESE VISITATION,去年西十一月出版,因义和团事而论西人将来待中
国之法者也。)其大指若曰: 

今次中国之问题,当以何者为基础而成和议乎?大率不外三策:一曰分割其国土,二曰变更其皇统,三曰扶植满洲政府是也。然变更皇统之策,终难实行,因今日中国人无一人有君临全国之资望,若强由
\newpage
此策,则骚扰相续,迄舞宁岁耳。策之最易行者,莫如扶植满洲朝廷;而漫然扶植之,则亦不能绝后来之祸根。故论中国最终之处分,则瓜分之事,实无所逃避,而无奈瓜分政策,又不可遽实行于今日。盖中国人数千年在沈睡之中,今也大梦将觉,渐有“中国者中国人之中国也”之思想,故义和团之运动,实由其爱国之心所发,以强中国、拒外人为目的者也。虽此次初起,无人才,无器械,一败涂地;然其始羽檄一飞,四方响应,非无故矣。自今以往,此种精神,必更深入人心,弥漫全国。他日必有义和团之子孙,辇格林之炮,肩毛瑟之枪,以行今日义和团未竟之志者。故为今之计,列国当以反分为最后之一定目的,而现时当一面设法,顺中国人之感情,使之渐忘其军事思想,而倾服于我欧人,如是则将来所谓“黄祸”(西人深畏中国人,向有黄祸之语互相警厉。)者,可以烟消烬灭矣。云云。(此乃撮译全书大意,非择译
一章一节。作者自注。) 

呜呼,此虽赫德一人之私言,而实不啻欧洲各国之公言矣。由此观之,则今日纷纷言保全中国者,其为爱我中国也几何?不宁惟是,彼西人深知夫民权
\newpage
与国权之相待而立也,苟使吾四万万人能自起而组织一政府,修其内治,充其实力,则白人将永不能染指于亚洲大陆。又知夫民权之兴起,由于原动力与反动力两者之摩荡,故必力压全国之动机,保其数千年之永静性,然后能束手以待其摆布,故以维持和平之局为第一主义焉。又知夫中国民族,有奴事一姓、崇拜民贼之性质也,与其取而代之,不如因而用之,以中国人而自凌中国人、自制中国人,则相与俯首帖耳,谓我祖若宗以来,既皆如是矣,习而安之,以为分所当然,虽残暴桎梏,十倍于欧洲人,而民气之靖依然
也。故尤以扶植现政府为独一无二之法门焉。 

吾今请以一言正告四万万人曰:子毋虑他人之颠覆而社稷、变置而朝廷也。凡有谋人之心者,必利其人之愚,不利其人之明;利其人之弱,不利其人之强;利其人之乱,不利其人之治。今中国之至愚至弱
而足以致乱者,莫今政府若也。 

使从而稍有所变易,无论其文野程度何若,而必有以胜于今政府;而彼之所以谋我者,必不若今之易易。列强虽拙,岂其出此?且同是压制也,同是凌
\newpage
辱也,出之于已,则已甚劳而更受其恶名;假手于人,则己甚逸而且藉以市惠。各国政治家,其计之熟矣。使以列强之力,直接而虐我民,民有抗之者,则谓之抗外敌,谓之为义士,为爱国,而镇扶之也无名;使用本国政府之力,间接而治我民,民有抗之者,则谓之为抗政府,谓之为乱民,为叛逆,而讨伐之也有辞。故但以政府官吏为登场傀儡,而列强隐于幕下,持而舞之。政府者,外国之奴隶,而人民之主人也。主人既见奴于人,而主人之奴,更何有焉?印度之酋长,印度人之主人也;英皇,则印度主人之主人也。安南之王,安南人之主人也;法总统,则安南主人之主人也。吾中国之有主人也,主人之尊严而可敬畏也,是吾国民所能知也;主人之复有其主人也,主人即借其主人之尊严以为尊严也,是非吾国民所能知也。今论者动忧为外国之奴隶,而不知外国曾不屑以我为奴隶,而必以我为其奴隶之奴隶。为奴隶则尚或知之
,尚或忧之,尚或救之; 

为奴隶之奴隶,则冥然而罔觉焉,帖然而相安焉,栩然而自得焉。呜呼!此真九死未悔,而万劫不复者矣。灭国新法之造妙入神,至是而极矣。虽然,
\newpage
惟蝍蛆为能甘粪,惟韲臼为能受辛,彼列国亦何足责?亦何足怪?彼自顾其利益,自行其政略,例应尔尔也,而独异乎四百兆蚩蚩者氓,偏生成此特别之性质,以适足供其政略之利用,而至今日,已奔走相庆,趋跄恐后,以为列强爱我、恤我、抚我、字我,不我瓜分,而我保全,我中国亿万年有道之长,定于今日矣。此则魔鬼所为掀髯大笑,而天帝所为爱莫能助者
也。 

凡言保全支那者,必继之以开放门户(OPEN THE DOREIN CHINA,译意谓将全国尽开为通商口岸也)。。夫开放门户,岂非美事

彼英国实门户全开之国也。而无如吾中国无治外法权,凡西人商力所及之地,即为其国力所及之地。夫上海、汉口等号称为租界者,租界乎?殖民地耳!举全国而为通商口岸,即举国而为殖民地。西人之保全殖民地,有不尽力者乎?其尽力以保全支那,固
其宜也。保全支那者,必整理其交通机关。 

今内河既已许外国通行小轮,而列国所承筑之
\newpage
铁路,必将实施速办,而此后更日有扩充矣。夫他人出资以代我筑当筑之铁路,岂不甚善?而无如路权属于人,路与土地有紧密之关系,路之所及,即为兵力之所及,二十行省之路尽通,而二十行省之地,已皆非吾有矣。保全支那者,必维持其秩序,担任其治安。和议成后,必有为我国代兴警察之制度者。夫警察为统治之要具,昔无今有,宁非庆事?而无如此权委托于外人,假手于顽固政府,施德政则无寸效,挫民
气则有万能。 

昔波兰之境内,俄人警察之力,最周到焉,其福波兰耶,其祸波兰耶?又今者俄国本境警察严密,为地球冠,俄政府所以防家贼者则良得矣,而全俄之民,呻吟于专制虐政之下,沈九渊而不能复。俄民永梏,而俄政府亦何与立于天地乎?而况乎法制严明、主权确定之远不如俄者也。故以警察力而保全支那,是犹假强盗以利刃而已。保全支那者,必整顿其财政。夫中国之财富,浮积于地面,阗塞于地中者,天下莫及焉。浚而出之,流而布之,可以操纵万国,雄视五洲矣。而无如商权、工权、政权,既全握于他人之手,此后富源愈开,而吾民之欲谋衣食者,愈不得不
\newpage
仰鼻息于彼族。不见乎今日欧美之社会乎,大公司既日多,遂至资本家与劳力者,划然分为两途,富者愈富,贫者愈贫,而中间无复隙地以容中等小康之家。今试问中国资本家之力,能与西人竞乎?既不能为资本家,势不得不为劳力者,畴昔小康之家遍天下,自此以往,恐不能不低首下声、胼手胝足,以求一劳役于各省洋行之司理人矣。保全支那者,必兴教育。教育固国民之元气也,顾吾闻数月以来,京师及各省都会,其翻译通事之人,声价骤增,势力极盛,于是都人士咸歆而慕之,昔之想望科第者,今皆改而从事于此途焉。而达官华胄,有出其娇妻爱女,侍外国将官之颦笑,以为荣幸者矣。吾知此后外国教育之势日涨,而此等之风气亦日开,所以偿义和团之损失者,如是而已。教育一也,而国民教育与奴隶教育,其间有一大鸿沟焉;而奴隶之奴隶教育,更有非言思拟议所能及者矣。嗟乎,列国之所以保全支那者,如斯而已乎!支那之所以自保全者,如斯而已乎!夫熟知瓜分政策,容或置之死地而获生;夫孰知保全政策,实乃使其鱼烂而自亡乎!新法乎,新法乎,前车屡折,而来轸方遒;饮鸩如饴,而灰骨不悔。吾又将谁尤哉!又将谁尤哉!
\newpage


\end{document}
