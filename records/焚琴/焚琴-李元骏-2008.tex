\documentclass{article}
\usepackage[utf8]{inputenc}
\usepackage{ctex}

\title{焚琴\footnote{Click to View:\url{https://web.archive.org/web/20221121014159/https://rentry.co/iqwyk}}}
\author{李元骏}
\date{2008-10}

% \setCJKmainfont[BoldFont = Noto Sans CJK SC]{Noto Serif CJK SC}
% \setCJKsansfont{Noto Sans CJK SC}
% \setCJKfamilyfont{zhsong}{Noto Serif CJK SC}
% \setCJKfamilyfont{zhhei}{Noto Sans CJK SC}
% \setlength\parindent{0pt}

\begin{document}
\CJKfamily{zhkai}

\maketitle


\Large

引子:常言道“诗无达话”,其实音乐亦然。若非要对着一首无标题无歌词的曲子大谈其赏析,
少不得弄出个精神失常 

想当初,俞伯牙还是个什么都不懂的小屁孩,每天只知道与别的小屁孩追逐打闹,混沌终日。有一天,他和同伴玩捉迷藏,别人藏好让他找。我们的小俞生性愚钝,找了半天居然摸进一片松林里,还迷了路。一想到顽童们讥笑自己的神情,俞伯牙就窘得满脸通红,险些急哭了。正在此刻,隐约有一种天赖传进了他的耳朵。前伯牙迷迷糊糊地顺着声音走去,只见林间空地里有个长须白衣一身仙气的老者在弹琴。俞伯牙躲着听了良久,越听越觉得自已平日所为俗不可耐,因找不到玩伴而产生的窘迫感似乎也消失了。不知不觉一曲告终白衣老者忽然唤道:“小孩出来。
\newpage
”伯牙给吓傻了,略征了一下,随后幡然醒悟,激动地冲出去,一面大喊:“老爷爷!我对您的景仰犹如滔滔江水绵绵不绝,又如黄河泛滥一发不可收,我要
拜你为师!” 

于是,俞伯牙成了大琴师连成的学生。连成,也就是那位老者,终日乾乾诲人不倦;伯牙也好好学习天天向上,奈何天资有限,终究难以大成一一偏又渐觉自己高古脱俗,甚至开始置疑于连成先生的品位来。一日他摆出琴,让连成先生猜测他所弹之景物。连成异之,说即使尽琴音高低、张弛之能事,也仅能表现出物体的小部分特征,叫人怎样猜度?伯牙却执
意一试,连成只好答应。 

伯牙便开始拂弦,心中胡思乱想一番,想起高山,于是自以为是地用琴描绘起高山来。弹了好一会儿,心中想到了流水,乃转调再弹了一阵。最后他停下来,问连成先后所弹各为何物。连成迟疑再三道:“前者跌宕起伏,应是军旅之声;后者流畅欢快,或为宴饮之乐。”伯牙大笑:“虽跌宕起伏,却是状山势之高;虽流畅欢快,却是状水流之疾。”连成不同
\newpage
意:“若说是高山流水,你的曲子却分明名了几味烟人气”,“胡说,”伯牙先是大怒,继而一脸轻蔑之色“怕是先生在为自己的错误诡辩吧?我伯牙已经学
成,现在出师算了!’ 

于是,伯牙挟琴回乡,先找来旧时的玩伴卖弄技艺。他仍弹那高山流水让人来猜。当年的玩伴现在生意做得颇有起色,对于弹琴却是一窍不通,半大说不出个所以然。伯牙一脸不屑地嘲卉:“前面跌宕起伏,当然是高山了;后面流畅欢快,自然是流水了。你这家伙满身铜臭却是艺术白痴,这怎么行呢?'那老朋友还真被唬得一楞一愣的,连声说要让自己的儿子跟俞伯牙学琴。伯牙刮肆意训关,“你以为准都配得弹太吗?小像找这般超凡脱俗的、又哪学传会琴?”说罢仰天大笑出门去,觉得过瘾之极一一凭什么别
人家财万贯,自己却家徒四壁,早就跟他不爽了。 

既然过瘾,俞伯牙索性日日到街上让人猜自己的高山流水。他抱琴坐在路边,一见路人就拉住对方,折腾一番然后大笑着把人赶跑,以享受那一瞬间的快感。一部分路人听出了跌宕起伏到流畅欢快的琴音
\newpage
之变,却没有答出高山流水,便一律被他斥为大俗人。“俗人”见得多了,我们的俞伯牙先生就愈觉得“举世皆浊我独清”,居然渐有“英雄寂寞”、“知音难求”之感。钟子期傍晚打柴回来,隔老远就望见前面路旁有个很猥亵的身影抱着琴如思妇般幽怨地仁立很快他就被伯牙拦了下来。伯牙开始弹他的琴,子期却心不在焉。他一介樵夫银子赚得艰难:今天上山砍柴,苦于山高攀爬不上,劳累之极;下山时又不慎失足掉进溪水里,身上一半的木柴都被浸湿了,收人大打折扣。不知不觉伯牙已经弹完了,摇醒钟于期问那个古老的问题。钟子期脱口而出:“先是高山后是流
水。”边说边想:他刚才问我什么来着? 

随着钟子期话音落地,四野陷人了冗长的宁静,只听见几声归鸦的哇哇悲鸣和一只兔子走错路撞在树桩上的声音。四目对视了不知多久,前伯牙忽然爆
发了:“知音啊啊啊啊啊……” 

从此,俞伯牙和钟子期形影不离,每天缠着钟子期听他弹琴。伯牙先弹K文王操》,弹毕请子期鉴赏。钟子期一头雾水,不知所措,连忙回想当初自己
\newpage
是怎样让伯牙欣喜若狂的。终于记起了少许,乃曰:“先是高山,后是流水。伯牙面露疑色一一自己弹的是一首描写周文王德行的曲子,并没有想着高山流水啊。伯牙再弹描写蚩尤与黄帝大战的《逐鹿野》。子期闭上眼睛,摇头晃脑作陶醉状:“先是高山,后是流水。”伯牙的世界忽然变得一片黑暗:难道我花这么多时间找来的人竟不是知音?不可能的,想来不是他听错了,而是我弹错了于是伯牙重弹高山流水,这次子期总算鉴赏得当了。伯牙听到自己的曲子被人理解,那个高兴啊……遂决定,以后每天只弹那首《高山流水》,而且重复不断地弹,以享受被理解的快感
。 

从此,子期一天要说十几遍“先是高山,后是流水”,没过几天就接近崩溃了。要是求伯牙放自己一马吧,以伯牙目前的癫狂程度来看成功率约等于零要是忍辱负重吧,估计不出几日就要气绝身亡了。前
思后想,钟子期计上心来 

第二天早上天还没亮,伯牙就被门外的喧闹吹打之声吵醒了。伯牙骂骂咖列地推开门,只见一队人
\newpage
正披麻戴孝地送葬呢一一众人肩上的红漆棺材格外醒目,上面写着:钟子期之灵柩。怎样描述伯牙当时的心情呢?总之,他摔门进屋骂了老婆打了孩子杀了三只鸡扔了九个臭鸡蛋,然后坐在地上声泪俱下:“子期啊你死得好惨啊……”等到声嘶力竭泪水流干之后,他颤巍巍地站起来,走到墙边取下挂在墙上的琴。“子期啊,你走了,还有谁听得懂我的琴?这琴不弹也罢!”随着一声长叹,伯牙手中的琴远射壁炉三分
命中,很快就干柴烈火了。 

等到伯牙的琴焚成最后-一缕灰烬,门外的棺材里神奇地飞出一个鲜活生猛的钟子期。子期和他的家人一路狂奔并一路狂笑,跑了一天一夜终于逃离了
那个有伯牙的地方。 

从此,钟子期一家过着幸福的生活。

\end{document}
