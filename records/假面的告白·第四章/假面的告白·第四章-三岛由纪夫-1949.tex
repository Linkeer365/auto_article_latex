\documentclass{article}
\usepackage[utf8]{inputenc}
\usepackage{ctex}

\title{假面的告白·第四章\footnote{Click to View:\url{https://web.archive.org/web/20230203015314/https://www.kanunu8.com/files/world/201104/2645/63051.html}}}
\author{三岛由纪夫}
\date{1949-04-27}

% \setCJKmainfont[BoldFont = Noto Sans CJK SC]{Noto Serif CJK SC}
% \setCJKsansfont{Noto Sans CJK SC}
% \setCJKfamilyfont{zhsong}{Noto Serif CJK SC}
% \setCJKfamilyfont{zhhei}{Noto Sans CJK SC}
% \setlength\parindent{0pt}

\begin{document}
\CJKfamily{zhkai}

\maketitle


\Large

意外的是,我提心吊胆的日常生活目前没有开始的迹象。社会陷入了一种内乱,好象人们不去考虑
“明天”,竟比战争期间还要甚。 

借给我大学制服的老校友从军队回来了,我把东西归还给了他。于是,我一时陷入了错觉,以为自
己摆脱了回忆乃至过去,自由了。 

妹妹死了。当我知道自己同时是一个可以流泪的人后,得到了浅薄的安心。园子和某个男人见了面订了婚。我妹妹死后不久,她结婚了。我这时的感觉好比是肩头的担子落了地。我一蹦三跳地自己对着自己乐。“这不是她甩了我,而是我甩了她的必然结果
。”我不无自负。 

\newpage

我爱把命运对我的驱使牵强地作为自身意志或理性的胜利。这一积年的恶习已经发展成疯狂的妄自尊大。被我叫做“理性”的特点中,似乎有一种不道德的感觉,有一种飘忽不定的偶然使假皇帝得意登基似的感觉。这个驴一样的假皇帝,连愚蠢专制可能导
致的复仇结果也不能预知。 

我在暧昧、乐天的心情下,度过了接下来的一年。泛泛的法律学习、机械的走读、机械的返家……我什么都不听,什么也都不听我。我学会了年轻僧侣那老于世故的微笑。我感觉不出自己是死了还是活着。我忘了,好象忘记了。我那天然自然的自杀——在
战争中死去——的希望不是早已被斩断了吗? 

真正的痛苦是徐徐到来的。它恰似肺结核,待自觉症状出现时,病则已经发展到了十分严重的地步
。 

一天,我站在新书日益增多的书店的书架前,抽出了一本装订粗糙的译作。是一个法国作家的饶舌的随笔集。偶然翻开一页,一行文字烙印似地射入眼
\newpage
中。可是,一股不快的不安涌上心头,我合上书放回
书架。 

第二天早上,突然间想起,于是,我在去学校的路上,顺道走入那家离学校正门不远的书店,买下了昨天的书。民法课刚开讲,我就立即悄悄取出它,放在展开的笔记本旁,开始寻找那一行。正是那一行
给了我比昨日更加鲜明的不安。 

“……女人力量的大小,惟独取决于其惩罚恋
人的不幸的能力的程度。” 

我在大学结识了一个亲密的朋友。他是某家老字号点心铺掌柜的儿子。乍看上去,像个老实巴交勤奋好学的学生,可他对于人对于人生所流露出的“哼哼”式的感触以及他那与我十分相似的虚弱的体格唤起了我的共鸣。我出于自我保护和虚张声势,学会了同样的玩世不恭。比起我来,他似乎在这一点上更加具有不伴随危险的自信。这自信心来自何处呢?我想。一段时间后,他用识破我童贞的、令我感到压抑的自嘲和优越的口吻,坦白了他出入不良场所的经历,
\newpage

并且邀我下次同去。 


“想去就打电话找我。本人随时奉陪。” 

“嗯。如果要去的话。……多半……快了。我
会尽快决定的。” 

我答道。他不好意思地抽了一下鼻子。那张脸告诉我,我现在的心理状态他一清二楚,这反过去唤起了他的羞耻心,使他想起了完全同于我目前状况的过去的他。我感到焦躁。这是一种试图把他眼中的我
和现实中的我完全统一起来的老掉牙了的焦躁。 

所谓洁癖,就是一种受欲望指使的任性。我原来的欲望是隐秘的欲望,它甚至不允许存在直截了当的任性。我假想的欲望——即,对于女人的既单纯又抽象的好奇心——被赋予了冷淡的自由,任性在其中将没有活动的余地。好奇心没有什么道德可言。或许
这就是人类可以拥有的最不道德的欲望。 

我开始了痛苦的秘密练习。我凝视着****
\newpage
女人像,试验自己的欲望。——再明白不过了,我的欲望横竖不吱声。先从不想任何图影开始,再从想象女人最下流的姿势努力,我尝试着调教自己。我有时仿佛感觉到了成功。然而,这成功却留下了心碎的扫
兴。 

“豁出去了!”我下定决心。于是打电话告诉朋友,让他星期日5点在一家咖啡馆等我。那是战争
结束后第二个新年的元月中旬。 

“终于下决心了?”他在电话上嘿嘿发笑,“
好,我一定去。中途变卦我可不答应哟。” 

——笑声留在耳朵里。我清楚,我惟有那谁也无法觉察的、僵硬的微笑能与之抗衡。可是,我还有一线希望,确切地说,我仍怀有一丝迷信。一种危险的迷信。惟有虚荣能使人冒险。就我来讲,那是一种
不甘心被人视为23岁的童贞的通常的虚荣。 


想来,我下定决心的日子就是我的生日。 

\newpage

——我们相互用刺探的眼神看对方。他也知道今天一本正经和嘿嘿傻笑同等滑稽,烟从他的嘴角一口接一口喷出。接着,就这家店铺的点心的差劲,他发表了两三句没话找话似的看法。我没有注意听他讲
话,说道: 

“想必你也有思想准备吧。第一次带到那地方的人,要么成为你的终生朋友,要么成为你一生的仇
敌哟。” 

“别吓唬人好不好?你知道我胆小。我可不适
合当他妈的一生的仇敌。” 

“你自己能认识到这一点就好。”我故意说话
老三老四的。 

“是的,那么……”他摆出一副司仪的面孔。他又说:“在什么地方喝几口再去。第一次去,一点
酒不喝怕是够戗。” 

“不,我不想喝。”我感到自己的面部发凉,
\newpage

“走。一口也不喝。这点儿胆量还是有的。” 

接下来是,昏暗的都营电车,昏暗的私营铁路,陌生的车站。陌生的街道。在简易木板房林立的一角,紫色红色的电灯把一张张女人的脸映得像一个个纸灯笼。化霜后的湿渍渍的街道上,嫖客们无言地你来我往,明明穿着鞋却发出了像光脚走路一样的脚步声。没有任何欲望,惟有不安如同闹着要吃零食的孩
子一样催促着我。 

“随便哪里都行。你听见没有?随便哪里都行
。” 

我想尽快逃离女人们故作苦闷的“过来,过来
嘛”的声音。 

“这家的妞危险呢。这模样好吗?那边比较安
全。” 


“管她模样好坏呢。” 

\newpage

“那我就选个相对漂亮的吧。以后可别埋怨我

——我们刚一上前,两个女人就像着了魔似地站起身来。这是个直起腰简直要碰到天花板一样的小矮房。龇着的金牙咧出牙床笑着,一个满嘴东北话的大个子女人把我诱骗到了只有三张榻榻米的小房间。
 

义务观念促使我抱住了女人。搂住肩膀正要接
吻,她笑得肥肩直晃。 

“得了吧。会整得你满嘴通红呢。得这么着。
” 

娼妇张开口红勾边、镶有金牙的大嘴,伸出像木棒一样强壮的舌头。我呀模仿着伸出了舌头。舌尖碰上了舌尖。……外人概莫能知其味,即:没有感觉恰似剧烈的疼痛。我感到我的全身,由于剧烈的疼痛而且是全然感觉不出的疼痛而麻木了。我上床躺下。

10分钟后,证实了我的不行。耻辱使我的双
\newpage

膝发抖了。 

在朋友没有察觉的假定下,接下来的几天,毋宁说我置身于痊愈的自我堕落的感情中。就像生怕患上什么不治之症的人,病名确定后反而可以体会到的一时的安心感,尽管他清楚那安心不过是暂时的,而且,心底期待着更加无处可逃的、绝望的、因而是永久性的安心。可以说,我也衷心期待着更加无处可逃的打击,换句话说,期待着那更加无处可逃的安心。

接下来的一个月中间,我多次在学校见到那个朋友。相互都没有提及那件事。一个月后,他偕一名同样和我要好的、喜欢女人的朋友来访。这人是一个经常吹牛说15分钟就可以把女人搞到手的爱炫耀的
青年。不多时,话题落脚到了应落脚的地方。 

“我已经受不了了。自己控制不住自己。”喜欢女人的同学目不转睛地盯着我,又说,“如果我的朋友中有人阳痿,我真羡慕。岂止羡慕,简直是敬仰

带我去玩过的朋友见我脸色突变,改变了话题
\newpage

,问好色的朋友: 

“以前说好要向你借马赛·普鲁斯特的书的,
有意思吗?” 

“啊,有意思。普鲁斯特是个Sodomy,
他和他的男仆有关系。” 


“什么?Sodomy是什么意思?” 

我知道自己在拼命挣扎,企图靠佯装不懂,靠小小的提问来获得自己的失态还未觉察的反证的线索

“Sodomy就是Sodomy。你不清楚
吗?是鸡奸者。” 

“第一次听说普鲁斯特是着种人。”我感到我的声音发颤。如果怒形于色,就等于把证据交给了对方。我对自己能够忍受这可耻的表面平静感到极度畏惧。我的那个朋友显然嗅出了什么。也许是我的神经

\newpage
过敏,好象他的视线正有意识地避开我的脸。 

夜晚11点,令人诅咒的来访者离去。我一直在屋里闷到天亮。我抽泣。最后,惯有的血腥幻想来临,安慰了我。我完全委身于这最贴身最亲密的残无
人道的幻影。 

我需要安慰。我经常去老朋友家参加聚会。虽然我知道这只能给我留下空洞无物的对话和索然无味的回忆。因为,这种和大学的朋友不同的体面人济济一堂的聚会反倒可以使我感到轻松。这里有异常矫揉造作的千金小姐,有女高音歌唱家,有未来的女钢琴家和新婚不久的年轻夫人。跳舞,喝点儿酒,做无聊的游戏,玩多少有些色情味道的捉迷藏,这样,有时
竟通宵达旦。 

黎明时分,我们往往跳着入睡。为驱赶睡意,别有一番游戏。地上扔下几块坐垫,以骤然停止的音乐为信号,当音乐突然停止时圆圈舞的圈立即散开,一男一女为一组分别坐向坐垫,如果坐歪了屁股沾了地板,必须露一手以壮余兴。因为站着跳舞的人必须扭在一起坐向地板上的坐垫,所以热闹至极。三番五
\newpage
次以后,女人们也就顾不得举止仪容了。一位最漂亮的小姐和人缠在一起摔了个仰面叉的一刹那,裙子翻到了大腿根。或许是有些醉意了,她丝毫没有觉察地
笑个不停。 

如果是以前的我,必定会使用须臾不忘的一贯演技,模仿着其他青年,从欲望处背过身去,猛地转移视线的。然而,自从那天,我和以前的我不同了。我全无一丝羞耻——即:全无一丝所谓的天生意义上的羞耻——目不转睛地,像看某种物质似的,盯视着那雪白的大腿。陡然间,从凝视中来并从凝视中收敛的痛苦降临了。痛苦告诉我:“你不是人。你不能与人相交。你是某种非人类的、既奇怪又可悲的生物。

恰巧,官吏录用的应考越来越紧张。它尽情地把我变为枯燥无味的学习的俘虏,我自然得以远离了折磨我身心的事端。但,这只是起初的时候。随着那一夜的失落感向我生活的每一个角落的蔓延,我连续几日郁闷不已,什么也不愿去干。我觉得,正式自己能行的必要日见紧迫,如果不能正式,我再也无法活下去。虽说如此,却无处寻觅那天生就不道德的手段
\newpage
。在这个国家里,甚至没有以更稳妥的形式满足我异
常欲望的机会。 

春天来了,我貌似平静的背后,积蓄了疯狂的焦躁。这季节像是对我怀有敌意,要不,怎么就刮起这尘土飞扬的烈风呢?每当汽车从我身旁掠过,我就
在心中高声怒吼:“你为什么不轧我?!” 

我爱用强制性的学习和强制性的生活约束自己。学习之余走在街上,我多次感受到了向我充满血丝的眼投来的疑惑的目光。或许在别人眼里乃至社会上,说我严谨诚实一贯如此。可是,我仅仅知道疲劳,那种被自我堕落、放荡、没有明天的生活、馊透了的惰性而腐蚀的疲劳。然而春天即将结束的一天下午,在都营电车上突然,一种窒息般的清冽的悸动向我袭
来。 

我透过乘客站立的空隙,在对面的作为上看见了园子的身影!天真的眉毛下面,有一对正直谨慎、无可言喻、深情温存的眼睛。我差点儿站了起来。一名站着的乘客松开了吊环,向出口走去。这时,我看
\newpage

清了女人的脸的正面。原来不是园子。 

我的心仍扑通普通跳个不停。把这悸动解释为一般吃惊或内心有愧很容易,可是,这种解释却无法推翻那刹那间的激动的纯洁性。我猛然间想起了3月9日早晨在站台发现了园子时的激动。这时与那时完全相同,绝无二致。就连如同被砍倒一样的悲哀也那
么相似。 

这个小小的记忆变得难以忘怀,给以后的几天带来了生气勃勃的动摇。不会的,我不会还爱着园子的。照理讲,我是不能爱女人的。这种反省反倒成了需要唆使的抵抗。尽管到昨天为止,这种反省一直是
忠实、顺从于我的唯一的东西。 

这样,回忆突然在我的内心复辟了,这次政变采取了明显的痛苦的形式。按说我在两年前就已经处理利索了的“小小的”回忆,恰似长大成人后出现的私生子一样,发育成异常大的东西,在我的眼前复苏了。这回忆既没有我时不时虚构出的“甜蜜”的状态,也没有我其后作为权宜之计所持的“事务性”态度
\newpage
,甚至它的每一角落都贯穿了明显的痛苦。假若着是悔恨,那么,众多的前辈业已为我们发现了忍耐之路。只是,这痛苦竟不是悔恨,而是异常明晰的痛苦,如同被人逼迫着从窗口俯视那把马路截然分开的夏天
的烈阳一样的痛苦。 

梅雨季节,一个阴天的下午,我趁着办事,在平素不太熟悉的麻布大街上散步。忽然有人从身后喊我的名字。那是园子的声音。回头发现了她的我,并没有像在电车上错把别人看成她时那样吃惊。这次偶然相遇十分自然,我仿佛觉得尽在预料之中。好像这
一瞬间很早以前便已知晓。 

只见她身穿除胸前的花边外别无其他首饰的、雅致的、壁纸一样花纹的连衣裙,丝毫看不出阔太太的样子。看来她是去了配给所,手里提着篮子,一名同样提着篮子的老太随后跟着。她先将老太打发回家
,和我边走边谈。 


“您瘦了。” 

\newpage


“是啊,忙着应付考试。” 


“是吗?请保重身体。” 

我们沉默了片刻。太阳渐渐照到宅邸町悠闲的路上。一只浑身湿漉漉的鸭子笨拙地走出某家的厨房门,嘎嘎叫着从我们的前面走过,然后顺沟口而去。
我感到了幸福。 


“现在读什么书呢?”我问她。 

“是问小说吗?《各有所好》……还有——”


“没看《A》吗?” 

我说出了眼下的畅销书《A……》的小说名。


“是那本有女人胴体的书吗?”她问。 


“噢?”我不无惊讶地反问。 

\newpage


“挺讨厌的……我是说封面上的画。” 

——两年前的她可不是能当面使用“女人胴体”一类词语的人。从这席位言词的一端就能痛感到园
子已不纯洁。来到拐角处时,她止住了脚步。 


“我家从这里拐个弯到头就是。” 

分手让人心酸,我便把垂下的目光移向篮子。篮子里,日晒后的魔芋挤在一起。那颜色看上去像是
女人海水浴后被晒黑了的肌肤。 


“晒得太厉害,魔芋要坏的。” 

“是啊,责任重大。”园子用带有鼻音的高嗓
门说。 


“再见!” 


“好,一路平安!”她转过身去。 

\newpage

我叫住她,问她回不回娘家。她轻松地告诉我
这个星期六回去。 

分手以后,我发觉了过去一直没有发觉的重大问题。看来,今天的她宽恕了我。为什么要宽恕我呢?有超过这种宽恕的污辱吗?然而,如果让我再一次
明确地碰上她的污辱,说不定我的痛苦会消失。 

星期六到来得太慢太慢。刚巧,草野从京都大
学回到了家中。 

星期六的下午,去访草野。我们俩正在交谈,我突然怀疑起自己的耳朵来。因为传来了钢琴声。那
幼稚的音色已经没有了,它圆润奔逸,充实辉煌。 


“谁?” 


“园子。她今天回来了。” 

一无所知的草野这样回答。我满怀痛苦,把所有的记忆一个一个唤回心中。关于我当时的婉言拒绝
\newpage
,草野其后只字不提。我深深地感觉到了他的善意。我希望得到园子当时曾经为之痛苦的一点点证据,而不愿承认我不幸的某种对应物。但是,“时间”的杂草已经在草野、我、园子中间茂盛生长,那种无须什么固执、什么虚荣、什么客套的感情表白已被彻底禁
止。 

琴声止住了。“我去带她来吧。”草野善解人意地说。不多时,园子和哥哥一起走进这房间。园子的丈夫在外务省工作,三人议论了一番外务省的熟人,无缘无故地笑了。草野被母亲叫走后,于是,就像
两年前的某一天一样,只剩下了园子和我两个人。 

她孩子似地不无骄傲地把草野家的财产由于她丈夫的鼎力相助才幸免于被没收的事讲给我听。在她还是少女时,我就喜欢听她的自我夸耀。过分谦虚的女人,与傲慢的女人同样没有魅力。可是,园子那端庄的、恰到好处的自我夸耀,洋溢着既天真又可人意
的女人味。 

“我说,”她平静地接着说,“有件事早就想
\newpage
、早就想问,可一直没问成。我们怎么就不能结婚呢?我从哥哥那里看到您的来信后,对这世上的事全懵了。每天只是考虑来考虑去,结果还是不明白,即使现在,我也搞不懂,为什么你我就不能结婚呢?……”她像生了气似地把微微泛起红晕的面颊朝向我,然后,一边侧脸一边朗诵似地说道:“……您是讨厌我
吗?” 

这当然也可以理解为“事务性的寻问式的口气罢了”,可是,我的心对于这单刀直入的提问却以剧烈而凄惨的喜悦来响应。然而,顷刻间,这可恶的喜悦蜕变为痛苦,一种十分微妙的痛苦。除原本的痛苦外,另有自尊心受到伤害的痛苦,因为两年前的“小小”旧事的重提,强烈地刺痛了我的心。虽然我希望
在她的面前能够自由,可依然没有这种资格。 

“你仍旧丝毫不了解社会。你的优点就在于不谙世故。可是,社会这东西的组成并不是专门为了随时成全相爱者的。就像我给你哥的信中所写的那样。而且……”我感到自己将要开始女人一样的倾诉,于是想沉默下来,但止不住,说:“……而且,我在那
\newpage
封信里根本就没有明确地说不能结婚。因为我那时才21岁,又是学生,太匆忙。哪知道我正在磨蹭,你
却早早地结了婚。” 

“这事我可没有权利后悔,因为我先生很爱我,我也很爱我先生。我真的很幸福,再没有什么奢望了。只是……大概是个坏念头吧?有时候呢,……这么说吧,有时候另外一个我,想象另外一种生活。这样一来,我就懵了。我觉得我简直要说出不该说的话,想不该想的事,心里怕得不行。这时候,我先生就
成了我的大支柱,他像对待孩子一样疼爱我呢。” 

“我的话可能很自负,还是说出来吧。你在上
述情况下,肯定恨我,肯定极端恨我。” 

园子连“恨”的语义也不明白。她做出一副温
柔、认真的怄气状,说: 


“随您怎么想。” 

“再单独见上一面怎么样?”——我像被什么
\newpage
催促似地哀求,“一点儿也不做问心有愧的事。只要能见个面就心满意足了。我已经没有任何资格说话,
沉默着也行,哪怕30分钟也行。” 

“见了面又怎么样?见过一次后,您会要求再见一次的吧?我婆母嘴碎得很,从去处到时间,大事小事都要问个水落石出。这么着提心吊胆地见个面,万一……”她吞吞吐吐起来,“……谁也说不清楚。
人心会怎么变化。” 

“那,谁也说不清楚,不过,你也太煞有介事的了。为什么不能把事物看得更明快、更单纯些呢?
”——我撒了弥天大谎。 

“男的可以这样,可结了婚的女子不行。等您有了太太,会明白的。我想,事情没有慎重过分的。


“这真像是大姐姐式的说教呢。” 


——由于草野的到来,谈话中断了。 

\newpage

即使在谈话期间,我的心也塞满了无限的狐疑。向神保证,我想见园子的心情是真的。但是,它没有掺杂任何的肉欲也是显而易见的。想见上一面的欲求是怎样的一种欲求呢?已经明确了没有肉欲的热情,难道不是欺骗自己的东西吗?好,就算它是真正的热情,也不过是卖弄似地拨挑几下那轻易就可以压灭的微弱的火苗而已。说到底,能有完全不扎根于肉欲
的恋爱吗?这难道不是明明白白地有违常理吗? 

然而,我又想,假如人的热情具有立足于一切反理之上的力量,那么,便难以断言力量不立足于热
情本身的反理之上。 

从那有决定性的一夜以来,我在生活中巧妙地避开了女人。那之后,别说能激起真正肉欲的男性青少年的唇,就连一个女人的唇也没有碰过,即使是在如不接吻反而失礼的场合下。——夏天来了,它比春天还要威胁我的孤独。盛夏,鞭策我肉欲的奔马。它要烤焦、肆虐我的肉体。为保住身体,有时我需要一
日重复5次恶习。 

\newpage

彻底把倒错现象作为单纯的生物学现象而加以说明的希尔休弗尔德的学说,为我启蒙。那决定性的一夜是自然的归结,而不是什么可耻的归结。想象中的对于同性青少年的嗜欲,一次也没有向恶习发展,而是固定在了大体上同等程度的普遍性已被研究者证明了的某种形式上。在德国人中间,有我这种冲动的并不少见。普拉腾伯爵的日记就是最明显的例证。温凯勒曼也同样。在文艺复兴时期的意大利,米开朗基
罗也显然是一个和我有着同样冲动的人。 

然而,这种科学性的领会却没能结束我心中的是生活。倒错现象之所以难以变为现实之物,是因为它在我这里仅仅停留在肉的冲动,白白吼叫白白喘息的阴暗冲动上。我从理想的男性青少年这里也仅能得到被激起的肉欲而已。如果用肤浅的见解来说,则是“灵”依然属于园子。灵肉相克这一中世纪的图式我不会轻易相信,只是为了便于说明才这样讲的。在我这里,这两种东西的分裂既单纯又直接。园子好象是我渴望正常状态之爱、渴望灵性物之爱、渴望永远存
在之爱的化身。 

\newpage

但是,仅此一点问题也不能解决。感情不喜欢固定的秩序。它喜欢好象乙醚中的微粒子一样,自由
自在地飞旋、浮动、发抖。 

……一年之后,我们觉醒了。我通过了官吏录用考试,大学毕了业,在某个政府机关里做起了事务官。一年来,我们有时像偶然似地,有时借故于并不重要之事,每隔两三个月见上一面。这几次都是利用中午的一两个小时,若无其事地见面,若无其事地分手。仅此而已。我做出一副堂堂正正的样子,丝毫不羞于被人看到。除了点滴回忆和有分寸地揶揄目前各自的处境这种话题外,园子也没有谈及其他。这种程度的焦急,别说关系,就是叫做联系都值得打个问号
。我们会面之中,也总是在想这次怎样爽快分手。 

仅这样,我也心满意足。而且,我还面朝某种东西,感谢这断断续续联系的神秘的丰饶。我没有哪一天不想园子,并且每次相见总能享受到平静的幸福。幽会的微妙的紧张和洁净的匀整遍及我生活的每一个角落,给我的生活带来了十分脆弱然而极其透明的

\newpage
秩序。——我想。 

可是,一年过后我们醒悟了。我们已不是孩子而是大人房间里的居住者,那扇只能打开一半的房门必须马上修缮。如同开到一定的程度便再也无法开的房门,我们之间的这种联系早晚需要修正。不仅如此,而且大人不像孩子一样能忍受单调的游戏。我们所经历的几次幽会,只不过像是叠起一看完全相同的纸
牌,大小一样,厚薄一样,千篇一律。 

在这种关系中,我反而尝遍了只有我才能体会到的不道德的喜悦。这是一种比普通的不道德更加微妙的不道德,是像精美的毒物一样的清洁的缺德。我的本质、我的第一义属于不道德。可结果,我反被认为在道德之举上、问心无愧的男女之交上、光明正大的步骤上,是个品德高尚的人。这一切都以它含有的不道德之味,以真正的恶魔一样的味道,向我献媚。

我们相互伸出手支撑着一个东西,这东西信则有,不信则无。是一种气体一样的物质。支撑它的作业,看上去简单,实际上是精确计算的结果。我在这个空间,表现了人工性的“正常”,并把园子诱至一
\newpage
瞬一瞬支撑架空之“爱”的危险的作业之中。看来,她不明实情地协助了这一阴谋。因为她不明真情,所以可以说其协力是有效的。可是,随着时间的推移,园子隐约中感到了无可名状的危险,感到了和普通的粗糙的危险全然不同的、具有精确密度的危险,感到
了它难以摆脱的力量。 

夏末的一天,从高原避暑归来的园子,和我在“金鸡”餐馆见了面。刚见面,我就把自己辞职的事
告诉了她。 


“今后怎么办呢?” 


“听天由命。” 


“哎呀,真叫人吃惊。” 

她没有深问下去,这已经成了我们之间的习惯

由于高原阳光的照晒园子的皮肤失去了胸前的耀眼的白色。因为炎热,戒指上的大颗粒珍珠懒洋洋
\newpage
地阴沉着脸。她那高亮的语调中,原先就有一种哀切和倦怠交合的音乐色彩,听起来与眼下的季节十分协
调。 

我们又开始了无意义的、总是兜圈子的、不认真的对话,并持续了一阵儿。这对话太像是在转圈玩,又像是在听别人交谈。是一种——快要睡醒时,不愿中断自己的梦而急着再次进入梦乡,这努力反倒不能把梦唤回——的心情。我发现,那佯装一无所知闯进心中的觉醒的不安,那就要醒来时梦的虚无的欢愉,正像某种病菌一样侵蚀着我们的心。疾病如同践约一般几乎同时来到了我们的心中。它反作用似地使我们快活起来。我和园子话追话话赶话地开起玩笑来。

阳光晒黑的脸稍许搅扰了她发下的静谧,但园子那优雅而高耸的发型下,一如既往地、庄重地分布着稚气的眉、温情脉脉秋水无尘的眼、几分厚实的唇。就餐的女客人关注着她,从餐桌旁走过。招待手捧银盘往来穿梭,盘中有只大的冰天鹅,天鹅的冰背上放着冰点心。只见她戒指闪亮的指头轻轻弹了一下塑

\newpage
料手提包的卡子。 


“已经厌倦了是不是?”我问。 


“您快别这么说。” 

听得出她的语气里有种不可思议的倦怠,似和“娇艳”相差无几。她的视线向窗外的夏日的街道移
去,继而缓缓说道: 

“我常常犯迷糊。这么着和您见面到底是为了
什么呢?迷糊归迷糊,可仍免不了要见您。” 

“因为它至少不是没有意义的负数吧。即便肯
定是没有意义的正数。” 

“我是个有先生的人。就算是没有意义的正数
,我也没有多少正的余地呢。” 


“真是绕人的数学。” 

——我悟出,园子终于来到了疑惑的门口。我
\newpage
开始感觉到放任不管那扇只能半开的门已经不行。说不定,现在的这种严谨的敏感已经占据了我和园子之间的共鸣的绝大部分。我距离能使一切维持原状的年
龄,还远着哩。 

另外,好象明确的证据突然把两种事态推到了我的面前:可能我的无法表达的不安已在不知不觉间传染了园子,还可能只有这不安的氛围才是我们之间的唯一的共有物。园子继续讲她方才的意见。我努力不让她的话进入我的耳朵,可我的嘴却偏偏轻佻作答

“您觉得照这样下去会怎么样呢?您不认为我
们已经进退两难了吗?” 

“我敬重你,对谁都问心无愧。朋友之间见个
面又有何妨呢?” 

“过去是这样,完全像您说的一样。我认为您很好。可是,我不知道以后咱们会怎么样。尽管没做什么丢人的事,可我常常做噩梦。每当这时,我就觉

\newpage
得神灵正在惩罚我未来的罪孽呢。” 

“未来”这个词的掷地有声之响使我战栗了。

“我想,这样下去双方总有一天会痛苦的。单等到痛苦以后,不就晚了吗?我们现在做的不就是在
玩火吗?” 


“玩火?玩火指什么?” 


“我想这包括很多。” 


“这怎么是玩火呢。大概是玩水吧。” 


她没有笑,一时无语,嘴唇弯曲紧绷着。 

“最近,我开始觉得自己是个可怕的女人,一心想着自己是精神肮脏的坏女人。我要让自己在做梦的时候也不想我先生以外的男人。我下决心今年秋天
受洗。” 

我透过园子半是自我陶醉的懒洋洋的告白,反
\newpage
而揣测到了她“循着女人特有的爱说反话的心理正准备讲出不该讲的话”的下意识的希求。对此,我既没有权利高兴也没有资格悲伤。丝毫不嫉妒她丈夫的我,怎能动用、怎能否定、又怎能肯定这资格这权利呢?我沉默。盛夏之中,我见自己的手白嫩软弱,使我
绝望了。 


“现在怎么样?”我问。 


“现在?” 


她伏下头去。 


“现在,在想谁?” 


“……我先生。” 


“这么说,就没有接受洗礼的必要了呀。” 

“有必要。……我是怕,我觉得我仍然动摇得

\newpage
厉害。” 


“那么,现在怎么想?” 


发问并不朝向任何人似的,园子抬起了认真的视线。这眸子之美,世间罕见。是一对如同泉水,始终歌唱感情涓流的、深挚的、凝视的宿命式的眸子。面对明眸,我总是失语。我猛地把大半截香烟戳进远处的烟灰缸。细瘦的花瓶一下歪倒,餐桌上到处是水

招待走来擦水。看着起水皱的桌布被擦来拭去,我们的心情糟透了。这给了我们提前走出店门的机会。夏日的街道乱乱哄哄让人焦躁。一对对胸脯高挺的健康的恋人袒露着胳膊从身边走过。我感受到了来
自一切的污辱。污辱像夏日的烈阳一样烤我。 

再过30分钟,我们分手的时刻就要来临。难以准确地说它来自分别的心酸,一种貌似热情的黯然的神经质的焦躁,使我生出了想用油画的浓涂料重重涂抹这30分钟的心情。扩音器把变调的伦巴舞曲撒满街道,我在舞厅前止住了脚步。因为我忽然间想起
\newpage

了曾经读过的诗句: 


……然而,即便如此,它, 


也是没有终了的交际舞。 

其余部分忘记了。大概是安德烈·萨尔门的诗句。园子向我点点头,为跳30分钟的舞,随我走进
了这极少出入的舞厅。 

随便把公司的午休延长一两个小时仍在跳舞的常客把舞厅搞得一片混乱。一股热气迎面扑来。换气装置本来就不完备,又加上一层厚实的窗帘,因此,只见场内沉淀的令人窒息的酷热,混浊地翻动灯光映照的雾一样的灰尘。散发着汗臭、廉价香水味、廉价发油味。旁若无人地扭动着的顾客的类型,不言自明
。我真后悔把园子带进这地方。 

然而,返身出去,现在的我却不能。我们勉强地进入那跳动的人群之中。稀疏的电风扇也没有送来正二八经的风。舞女和身穿夏威夷衫的年轻人紧贴着
\newpage
满是汗水的额头跳在一起。舞女的鼻梁两侧出现两道黑,被汗浸湿了的白粉变成粒状,布在脸上像是长了疖子似的,礼服的背面则比方才的桌布还脏还潮。是跳还是不跳?尚在犹豫之时,汗水已经顺胸流下。园
子难受地急促地吐了口气。 

为了呼吸室外的空气,我们低头穿过假花悬绕的拱门,来到里院,在简陋的长椅上坐下休息。这里尽管有室外之气,但是,阳光晒烫了的混凝土的地面把强烈的热能投向了背阴处的长椅。可口可乐的甜味粘在嘴上。我曾感到的那来自所有东西的污辱的痛苦,同样使园子沉默了。——我觉得。我难以忍受时间
在沉默中推移,于是,把目光转向了我们的周围。 

一个胖姑娘用手帕扇着胸前,无力地倚靠着墙壁。摇滚乐队奏出了压倒一切的快步舞曲。里院的大花盆中的枞树,在干裂的土上倾斜了树身。背阴处的长椅上坐满了人,而向阳处的长椅上到底没人去坐。

有了!只有一组人坐在那象样的长椅上旁若无人地谈笑着:两个姑娘两个小伙子。一个姑娘装模作
\newpage
样地用笨拙的手把还没学会抽的烟送近嘴边,每一次都要轻轻内咳一声。两个姑娘都穿着像是浴衣改做的怪兮兮的连衣裙,袒露出胳膊。其中一个像渔家姑娘,发红的胳膊上斑斑点点有蚊虫叮咬的痕迹。她们听了两个小伙子的下流玩笑,你看我我看你,故意做出一种样子笑个不停。他们好象全然不在乎射在头顶的强烈的夏天的阳光。一个小伙,脸苍白些,显得阴险,身穿夏威夷衫,胳膊却壮得很。下流的笑在他的嘴角时隐时现。他一次次用指尖戳姑娘的胸脯,一次次
逗得对方发笑。 

我的视线被另外一个吸去。是个二十二三岁,脸相粗野、皮肤浅黑然而端正的小伙。他赤裸着上身,汗水湿透了用漂白布做的已变成了浅灰色的围腰。他重新解开围上。他一边凑着热到一边故意慢腾腾地围围腰。袒露的胸现出了丰富结实的筋肉块,深深的立体的筋肉槽从胸部的中央只滑向腹部。粗绳扣似的肉的连锁被左右勒紧,盘踞在肋腹。那光滑的热能沸腾的有质有量的胴体被他用脏了的漂白布围腰紧了又紧地围起来。那阳光晒黑了的光膀像涂了油似的发亮。腋窝下露出的毛丛,在阳光的照耀下鬈曲地放射出
\newpage

金色的光。 

看到这,特别是看到他筋肉紧绷的胳膊上刺着的牡丹时,我欲火中烧。热烈的注视紧紧定在这粗俗野蛮然而无与伦比的美的肉体之上。他在太阳下笑着。向后仰身时,露出了突出的粗大的喉头。奇怪的激动驰过我的胸底。我已不能从他的身上移开我的目光

我忘记了园子的存在。我心中只想象着下面的情景:盛夏,他半裸着走向街头,接着,和流氓弟兄展开搏斗。锋利的匕首穿透那围腰刺入他的胴体;鲜血把那脏围腰点缀得美丽无比;他满身是血的尸体被
抬上门板再次送向这里…… 


“只剩下5分钟了。” 

园子高昂哀切的声音穿透我的耳膜。我不可思
议地回头向园子望去。 

一瞬间,在我的心中有东西被残酷的力量一撕为二,如同雷落树裂一般。我听见了我一直竭尽全力
\newpage
构筑的建筑凄惨崩溃的声音。我好象看见了我的存在接替一种可怕的“不存在”的一刹那。闭上眼睛,顷
刻间,我抓住了冻结的义务观念。 

“还有5分钟是吗?带你到这里来,对不起了。你没生气吧?像你这样的人是不应该看到那帮下贱人的下贱样子的。”据说这个舞厅没有处理好“仁义”问题,所以尽管老板再三谢绝,可那帮人仍免不了
前来白跳。 

然而,看他们的只有我自己。她根本没看。她接受的教育,就是不看不该看的。她只是无意间注意
到了为观看跳舞而汗水湿背的观众。 

虽说如此,这舞厅中的空气似乎在不知不觉中也使园子的心中发生了某种化学反应。不多时,只见她腼腆的嘴角漂浮起微笑的征兆,这是一种未曾开口
先以微笑试探的征兆。 

“想问您一个怪问题:您已经那个了吧。已经

\newpage
知道那事了吧?” 

我没有一点力量了。然而,心中还有一个发条
一样的东西,它使我作出了堂而皇之的回答: 


“嗯。……知道。遗憾得很。” 


“什么时候?” 


“去年春天。” 


“和哪一位?” 

——这优雅的提问使我吃惊不小。她只知道把
我和她自己知道姓名的女人联系在一起考虑。 


“名字不能讲。” 


“哪一位?” 


“别问了。” 

\newpage

大概是听出了我赤裸裸哀求腔调中的弦外之音,她马上大吃一惊似地沉默起来。为了不让她觉察出我的脸正在失去血色,我尽了最大的努力。我们等待着分手的时刻。卑俗的节拍反复揉搓着时间。我们在
扩音器传来的伤感的歌声中一动不动。 


我和园子几乎同时看了手表。 

——时间到了!我再次朝那向阳的长椅投去偷视的目光。几个人像是跳舞去了,空荡荡的长椅在火辣辣的阳光下放置着,桌上洒落的什么饮料一闪一闪
反射出凄热的光。 

(昭和24年7月)

\end{document}
