\documentclass{article}
\usepackage[utf8]{inputenc}
\usepackage{ctex}

\title{nza}
\author{ssz}
\date{1989-02-23}

% \setCJKmainfont[BoldFont = Noto Sans CJK SC]{Noto Serif CJK SC}
% \setCJKsansfont{Noto Sans CJK SC}
% \setCJKfamilyfont{zhsong}{Noto Serif CJK SC}
% \setCJKfamilyfont{zhhei}{Noto Sans CJK SC}
% \setlength\parindent{0pt}

\begin{document}
\CJKfamily{zhkai}

\maketitle

\setlength\parindent{0pt}

\begin{center}

\begin{minipage}{0.5\linewidth}

\Large

最少听见声音的人被声音感动 \\
最少听见声音的人成了声音 \\
头上是巴赫的十二圣咏 \\
是头和数学 \\
沿着黄金风管满身流血 \\
 \\
巴赫的十二圣咏 \\
拔下雷霆的塞子,这星座的音乐给生命倒酒 \\
放下了呼吸,在。 \\
在谁的肋骨里倾注了基础的声音 \\
在晨曦的景色里 \\
这是谁的灵魂?在谁的 \\
最少听见声音的耳鼓里 \\
敲响的火在倒下来 \\
 \\
巴赫的十二圣咏遇见了金子 \\
谁的手斧第一安睡 \\

\end{minipage}

\newpage

\begin{minipage}{0.5\linewidth}

\Large

空荡荡的房中只有远处的十二只耳朵 \\
在火之后万里雷鸣 \\
 \\
我对巴赫的十二圣咏说 \\
从此再不过昌平。 \\
巴赫的十二圣咏从王的手上 \\
拿下十二支雷管\end{minipage}

\end{center}


\end{document}
