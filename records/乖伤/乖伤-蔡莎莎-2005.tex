\documentclass{article}
\usepackage[utf8]{inputenc}
\usepackage{ctex}

\title{乖伤\footnote{Click to View:\url{https://web.archive.org/web/20221026072121/https://rentry.co/wsytq}}}
\author{蔡莎莎}
\date{2005-09}

% \setCJKmainfont[BoldFont = Noto Sans CJK SC]{Noto Serif CJK SC}
% \setCJKsansfont{Noto Sans CJK SC}
% \setCJKfamilyfont{zhsong}{Noto Serif CJK SC}
% \setCJKfamilyfont{zhhei}{Noto Sans CJK SC}
% \setlength\parindent{0pt}

\begin{document}
\CJKfamily{zhkai}

\maketitle


\Large

又是九月。我依然是双手插口袋晃荡在街上,枯黄的叶子纷乱地旋转着往下坠,幼遣卷地伏在肩头,继而悄然滑落。转眼竟是秋天了。红灯亮起,我停在街头看过往的行人,他们的表情很匆忙,带着一丝谨慎与坚固,可是仍能发现里面隐械着的落寞的影子。我习惯性地将手指框成平行四边形,任由视线在拘泥中淡漠地延伸着,试图能找到那个跃动的红色身影。可是没有没有,苍白色将一切掩盖得恰到好处,不留痕迹不动声色,摩实地撑起这个城市的上空。流
苏轻拂、空气里弥散着儿缕淡淡的野菊香。 

我知道我该去看看她了,那个九月来到这个世界又在九月离开的小女孩,找的格子。我想她,我总是在这样的日子里无可休止地想念曾经和我一起在马路上奔跑的她。她站在我的面前,笑腾如花。她说,
\newpage

只要你点点头幸福就不会溜走。 

是的。一直以来,我都不知道该如何落笔。墙壁上不断地渗出甘冽的花朵,就如同魔术师手中的黑色帽子变幻多端,关于格子的往事也浅浅地在心底铺展开来。疼痛。我拉住回忆的手企图让它远走高飞,可到最后却发现是徒劳。那些成长的碎片摊开在我错综复杂的生命线中,承载着旧日沉淀下来的伤痕,在
无法捕捉的未来里蔓延着。 

掰掰手指,我们小学三年级的时候就认识了吧,大雁南回的秋天。我刚从一个美丽的北方城市来到这个并不秀美的南方小城,带着一脸的桀骜不驯。我是新生,也许是天生的傲气惹恼了一些同学,有孩子开始指着我说很难听的话。我表面异常平静地走过去,狼狠地给了其中一个男孩子一耳光响得在教室里起了回声。所有人在那一刻都沉默了。我转过身去,若无其事地走出了教室,操场真大,天空却很小。有乌鸦在上面不停地飞旋。短暂的安静后,身后的教室开
始沸腾了 

\newpage

找去开幢嚣的人群,芽过诺大的操场,坐在角落里的乒乓球台上晃荡看脚丫。风曳过我的脸时,沙子趁势迷住了眼睛,伪装了想要掉下来的泪水。我把腿缩上来,紧紧地抱住,头埋了下去。我保持着这样一个姿势,开始疯狂地想念过去,想念曾经北方的生活。那时候的心情一直不知道怎么来形容,后来看到-句话,想应该是合适的。它说,整个世界用脊梁骨
对着你的时候是很可怕的 

不记得过了多久,我感觉到有人在轻轻地拍我的背,是个小女孩的声音,她说,你不要哭。她说,你不要哭。我抬起头来,干涩的脸庞让她递过来的手帕僵持在半空中。如此近的距离,我却没有看清她的脸,视线是模糊的,因为在手臀上压得太久的缘故。我微微昂起头,用很坚定的表情告诉她,我是不会哭的。她笑了,收回了手帕。我看到格色下她侧过的脸庞,夕阳里清澈的笑声,一点点荡漾开去,宛如梦幻
 

然后我们一直沉默。我是个不大喜欢和陌生人说话的孩子。她也没再说什么,往我手里塞了一架纸
\newpage
飞机就跑开了。她跑的样子真好看,小辫子一潮一翘的,穿着红色毛衣的背影像跃动的火焰,呈现出孩童应有的朝气。找还清楚地记得那是一架用音乐书页叠的飞机,小巧精致,空白处歪歪扭扭地写着:默默,格子要做你的朋友。摊开书页,上面的图商是两个小女孩背萃着背坐在一起,还有一首好听的儿歌《虫儿飞》。看得出,她在很用心地做着这一切,好一个细心的小姑娘。我轻轻地念着她的名字,格子,干净清澈是我喜欢的样子。可是对于这样突如其来的友情,我的内心还是有些抗拒的,刚来的遭遇让我对这个城
市反感极了,骨子里是个倔强的孩子。 

我安静地把那架纸飞机放进了阿婆送给我的布袋里,开始有意无意地观察着那个叫格子的小姑娘,发现她也是独来独往的。每天放学后,她都会跑到教学楼的楼顶上,趴在隔热层上做作业。她做作业的时候极其认真,眉头时而紧登,时而舒展。做完后就靠在栏杆上往下看、隐约听到有小声的歌唱。我就呆在不远的地方看着她。后来,我干脆也在那里完成作业。再后来,是两个人一起,我和格子不记得是哪一天了,我也试着往下着,我想知道是什么样的一个世界
\newpage
让她如此专注。就是在把头伸出去的那一瞬,我决定和格子做朋友了。楼下是操场,三三两两的人群,嬉戏玩耍。就是这样的简单画面,她是在美慕着。她唱
歌,想把自己融人进去。 

我不紧不慢地从书包里抽出一张白纸,也叠成飞机的样子。上面写着:格子,默默愿意做你的朋友。然后捧着放在嘴边哈了口气,带着志杰的心情,轻轻一掷。戴烫悠悠的纸飞机刚好洛在她摊开的作业本上。她先是被从天而降的纸飞机一惊,接着便拿起来仔细地看。还没来得及四处张望,便看到我了,脸上第出浅浅的笑意。她说你好。很大的声音,和着风声,听得我的心里一阵荡源。我说你好。很大的声音,和着风声,说得我的心里一阵舒畅。然后我们成了朋友。我们在楼顶上大声地唱零零碎碎的歌谣,捉迷蔽,跳房子,或者是把皮筋栓在石柱上两个人一起跳,
嘴里还念明着奇怪的数数歌。 

马兰开花二十一,二八二五六,二八二五七,二八二九三十一三八三五六,三八三五七,三八三九

\newpage
四十一…… 

没有人会理会我们,他们大概不会在意到楼顶上还有这样两个小孩。我和格子的书包里总是塞满了五颜六色的糖果,做作业做得累了的时候,就不停地吃,然后把那些透明的彩色的糖纸覆盖在眼睛上,看天看地看树看人看所有的被换了妆的模样。开始逐渐地忘记了下面的世界,沉漫在自己的快乐中。那些日子里,天空的蓝色异常地明朗,轻薄的白云一片一片地铺展开来。它们的默契就像大海和沙滩,天空是地球的大海,白云是地球的沙滩。那么,我们算不算海
里面游来游去的鱼。 

回家的路上,我和格子总是像张开翅膀的巨大鹏鸟一样飞越过这个城市,华丽而张扬。第一次,格子拉起我的手。脱离了依附的树叶不断从眼前晃过,飞舞着往下坠、踩在脚底发出略岐略岐的声响。格子的小手很暖。那个飞累飘扬的秋大于是在我眼里豁然开朗起来。找们偶尔停下采在写了大大的红色拆字的残垣上画画,拿着地上破碎的砖头涂满了那些快要消失的脏脏的空墙。印象最深的是,有一次我们都画了一张孩子的脸,格子画的没有瞳孔,我画的则是没有
\newpage
嘴巴。然后我们惊讶地对视了很久,因为默契,也因为难过。最后,格子还是在上面写下了自己的名字,她说,要一直一直记住自己的模样。我知道,格子是勇敢的,她其实更像一位骑士,一面在凛冽的寒风中驰骋着,一面疼惜着同样奔跑着的爱马。想起即使是在很冷的冬天,穿着厚厚的棉袄,格子也会拽着我去冷饮店买冰激凌,两个人吃得开心死了特别是看到旁边不断地有人投来惊讶的目光时。格子说,难过就吃
冰激凌 

我们顺着长长的街穿过广场,一边吃一边旁若无人地大声朗诵着顾城的诗,那个有着许多童话的诗
人,我们喜欢他。 

我喜欢穿旧衣裳/在默默展开的早展里/穿过广场/一蓬蓬郊野的荒草/从空深中,无声地爆发起来/我不能停留/那些瘦小的黑蟠蟀/已经开始歌唱/我只有十二岁/我垂下目光/早起的几个大人不会注意/一个穿旧衣服的孩子的思想/何况,乌也开始叫了/在远处,马达的鼻子不通/这就足以让几个人

\newpage
快乐或患伤…… 

后来,格子搬来和我住,我们一起精心地料理着家里的小花网。那时候正值盛夏,满目的向日葵绚丽而且张扬,点缀着硕大的花盘和明媚的黄色,轿傲地朝着太阳生长,释放出极致的灿烂。在我的眼里,格子就是那粲然的向日葵,我安静地倚在她身边,寂然生长。格子总是很早地就起床,然后拖着一大堆颜料坐在天台上画画。周末的下午我们会窝在沙发里拼命地打超级玛里奥,谁输了就请客,去那家不远的面馆吃牛肉拉面。我和格子都是很能吃辣的,拿着料子使劲儿往碗里搁,店主在旁边瞪眼吸嘴的却也不能说什么,那叫一个心疼呀。我们低下头边吃边暗自幸灾
乐祸。 

不打游戏的时候,我们就去郊外的铁路边散步,张开双臂来回地走,如同马路上飞翔的天使。郊外的天空很蓝,有各种漂亮的鸟儿在头顶歌唱。远远地听到火车鸣了,格子就拉着我飞快地闪到一边,直直地站着等待它们来检阅我们的怒放、火车呼啸而过的那一瞬,耳边还残留猎猎的风响,纷乱的头发把脸弄

\newpage
得生疼生疼的 

有一次,我趁格子躺在草丛中咪着眼晴看天空的间隙跑到对面去捉蜻蜓。回来的时候刚好有火车通过,我于是着急起来,我怕格子在找我,她也在着急。我大声地喊,轰鸣的声音的掩盖下连我自己都听不见叫喊。我趴在草地上,看火车下面的铁轨,看对面的脚丫,不停地叫格子格子格子。可是没有任何回音。当火车终于完完全全地通过,我看到了对面呆站着的格子,她已经泪流满面,眼神里是我从没有看见过的绝望。我顿时想起了格子画过的那个没有障孔的孩
子。 

从那天起,我暗暗告诉自己,不要,不要轻易离开格子一直没有仔细问过关于格子父母的事情,只知道同我的父母一样,也都在外地工作。记得小时候学校里的说法很多,大抵是说格子是被抛弃的孩子。我向来是讨厌那些人的,他们编顺口溜:默默坏,格子坏,格子默默吊书袋。因为格子骂人,我会打人,然而我们都能拿很漂亮的成绩,得很多的奖。后来升了中学,分散在不同的学校,这些也就随着时间的消

\newpage
磨过去了。 

暑假去外地参加竞赛夏令营了。回来的时候,突然觉得格子消沉淡漠了许多。那几天,格子总是靠在透明的落地窗边,捧着水杯一直不说话。我就看着那个黑色的影子一动也不动,心里勾画着这样一幅画面,一个女核卷曲着腿,在淡淡的月白色背景里,沉
静地俯首。它的名字是“凝神"。 

后来,格子终于开口了。她说,默默,我给你讲个故事好么还是老和尚给小和尚讲故事的那个片段
么。以前格子就是这样让我笑起来的。 

格子的嘴角抽动了一下,苦笑。不是的,是关于一个女孩的故事。我想我是明白了什么,便没再开
口。 

有个小女孩,暂且没有名字吧。她一直以为自已是幸福的,有很爱她的爷谷、奶奶,小姑,尽管爸爸妈妈不在身边。爷爷常常拉着她的小手去散步,奶奶会给超做好看的结实的布鞋,姑姑每次来都带很多的好吃的还有爸爸妈妈的礼物,她觉得自己就是生活
\newpage
在阁楼里的公主。有时侯,看到同学都有父母来接,她就问爷爷,为什么爸爸妈妈从不回来看他。谷符告诉她他们工作都很忙,只有她乖了开心了,爸爸妈妈才能安心工作。于是她相信了,并且自豪着,告诉自己要用最灿烂的笑容迎接每一天。一格子。我突然开口,却一时不知道说什么。格子没有搭理我,继续说
着。 

可是这个美丽的谎言终于被数穿了,在很多年很多年以后。女孩长大了,搬到学校附近姑姑的旧房子那儿去住:屋子很脏乱,站姑把里里外外都打扫了一遍,这才从外观上看起来像个家的样子。可是没有呆多久,女孩住到她的好朋友家里,她也是一个落翼的孩子。和她在一起的日子,是女孩最快乐的时光。有一天,爷爷告诉女孩妈妈快回来了,女孩高兴坏了,忙穿了漂亮的裙子,把头发扎了起来,跑回家里等着,那个时候还听到心紧张地扑扑直跳。可是到天黑的时候他们又说妈妈不回来了。等爷说,回去吧。女孩说不,我要在这里等。女孩就那么一直一直地等着,等到人都麻木了,也没有出现爸爸妈妈的身影。后来女装去爷爷家玩,不小心弄破了手指,到处找云南
\newpage
白药时无意中在抽展里发现一沓汇教单,收款单位是某精神病医院,附百上赫然写着妈妈的名字。女孩突然征住了,揉了揉眼睛,确定自已没有看错。那一瞬,女孩的脑海一片空白。她跑到院子里,缠著爷爷告诉她真相。爷爷的眼驱湿润了,他说孩子你不要问不要问好不好。这是女孩第一次看到爷爷的眼泪,便便地从那苍老的脸庞上滚下来。女孩也哭了,她说,爷爷你告诉我,好么,告诉我。她死死地技住爷爷的衣角,用乞求的眼光看着。可是现在女孩后悔了,她希望自已是什么都不知道的,至少那样心里还能有个美好的幻想可以安慰自己。格子的声音有些颤抖了。我看到她的杯子都空了,可是她还是偶尔地用嘴磕碰着。我起身给她添了杯茶,坐在她对面。格子没有看我爷爷开始对小女孩讲述整个事情的来龙去脉,事情发生在另一个城市。女孩的爸爸妈妈本来确实是在一个遥远的地方工作,两人携手创业,辛勒工作,硕果果累。可是爸爸却有了外遇。妈妈为了家庭为了孩子去闹过,那个第三者张扬得不得了,比母夜叉还厉害,居然动手打了妈妈。当时爸爸就在旁边,却一声不帆。妈妈无奈之下提出离婚,可是爸爸突然回心转意,要和妈妈和好。妈妈始终是对他们的爱情还存有幻想
\newpage
的,答应了,随之答应把所有的家产都寄居在爸爸名下。看,爸爸的目的达到了,于是他悄悄地离开了。第二天,妈妈醒来发现身边空荡荡的,家已经被洗劫一空。她承受不了被曾经深爱的人这样对待的打击,疯狂地哭喊过后,疯了。爷谷于是赶去送女儿进了当地的一家精神病疗养医院。其实那次的回米,只不过是医院传来了能够出院的消息。可是由于另一位想者的刺激,加之妈妈吞食了有寄的物品,妈妈再度疯了,或者说她根本就没好过。女孩知道了这些以后,很想去暂香自己的妈妈那个可怜的女人。她烟给她捶捶背,跟她聊聊天,为她唱唱歌,告诉她外面的世界。可是仅仅是一个月以后,爷爷告诉女孩,妈妈自杀了。女孩蒙了,一下子瘫坐在地上。地不明白妈妈为什么要自杀,难道忘了还有个一直等婚旧来的女儿吗,为什么不继续活下去呢。后来,女孩终于明台了,也许死对于妈妈来说,才是一种真正的解脱。从此,女孩每晚都要给妈妈析祷,圣母玛利亚,圣母玛利亚,请你为我的妈妈带去祝福。说完。格子双手合十,嘴里默念着,温暖而苍凉的姿势我的视线一片模糊,我仿佛看到有一个孩子在拼命地逃,也在拼命地追,可是她没有追到幸福,她“追回"感伤,“追回”了一
\newpage
切的不幸。我递过纸巾,格子拒绝了。于是回到自己的床铺,我知道格子这个时候需要的是什么,然后沉默。半夜醒来,打开灯,发现格子还是倚靠在落地窗边,只是已是熟睡的状态,蜷缩得很紧,像个婴儿。是谁说,婴儿最脆弱。给格子盖被子的时候发现她的嘴唇破了,上面有点点血溃,掩饰着深深的牙印,脸
上还残留着淡淡的泪痕,久久不肯消失。 

在很深很深的夜里,格子抓住我哭拉,她说默默我冷我冷。我紧紧地握住她冰凉的手,告诉她不要害怕不要害怕,有默默陪你。荒凉的天空下,我们相
互依假着。 

默默,当有一天所有人都离开了,你还会不会站在原地等找:会的,会的,我会站在这里等到你出现。这是格子搬回自己家的前一天和找说的话。她还说,有些事情必须学会自己面对。我看着她球珀色的眼睛不说话,然后安静地带她收拾东西。有些事情不
是预料不到,它在到来之前就是有预兆的 

几个星期后,格子告诉我,她想出去走走,到
\newpage
宏村去写生。格子是学画画的,无可非议,只是我总是隐隐地有些担心。那天去送她,熟悉的铁轨边,我们常常站在这里沿着铁轨延伸的方向望去,大声告诉彼此,一定要离开这个厌恶的城市,一起离开,许诺一起。可是亲爱的格子,你源么忽而就抛下我一人了呢。我在车站一直看着格子挥动的手臂逐渐化为一个点时才转身,我流泪了,英名其妙,那天的阳光并不
刺眼。 

格子到了宏村以后,就开始不定期地给我写信,告诉我她的近况,每封信里都不忘塞一架纸飞机,
上面写满了祝愿。 

亲爱的默款,我坐在这个古老的镇子给你写信,安静,破败。身边是高大的石牌坊,一座连著一座就像某种植物绵延成的茂密的森林,解肤然而伟岸,散发着不可盲说的味道。每一座牌坊都记载著一段远去已久但是始终为人津津乐道的故事,我刚刚从悠长逼仄的小巷回来,石板路在梅雨季节里泛装水痕,带着凉意悠悠地扑过来,一呼吸满身都是青涩的苔藓样的味道。巷子总是潮潮的,仿佛承载不动这里的过往
\newpage
。我住在一户村民家,他们见我是学生对我也很好,只收了很少的一部分房租。老人拿着长长的烟斗倚在门口冲我朴实地微笑,有些老婆婆的头上扎着非常奇怪的头饰,小孩们玩的也是我香不懂的古老的游戏。我想我是深深地爱上这里了,我的画架一直都支着,可画板却是空着的,我觉得自己笨拙的双手无法描绘出它们灵魂的模样。默肤,当有一天我离开了所有的人,你还会不会站在原地等我?会的,会的。我会站在这里等到你出现。我在心底轻轻地回答着日子如同过眼云烟很快便消逝,在空气里逃遁得无影无踪。已经开学了,格子还没来报到,可是按照她最近写给我的信,她应该是到家了的。我有些志恋不安了,似乎在害怕和逃避着些什么。那个晴好的下午,同学说传达室有我的信,黄山来的。我心里微微地震了一下,感觉到有什么东西在冥冥之中把找刺伤了,留下一个又一个微小的不易察觉的伤口。然而最后,我发现那何止是伤口,简直是五雷轰顶的打击。格子乘从宏村到市区的车子严重超载而翻进了山沟,意外身亡。那封信即是死亡通知书,还说了些什么不记得了,只知道那两个大大的黑体字“死亡”在我的脑海很长一段时间里挥之不去。还有一张被揉得皱巴巴的纸条,上
\newpage

面写的不知道是歌词还是其他什么。 

妈,给我钥匙,给我那把钥匙我想回家,我肚子很饿我需要一个甜盛的吻一个薄荷的吻一个陌生但是潮湿的咖我想念家里的花园,花朵长得那么美丽就
像你美丽的笑脸,妈妈 

我仍旧能记起回家的路,我仍旧能仍旧能妈妈,昨晚我掉了第一颗牙齿,我没有哭,我想我已经长大了妈妈,牵着我的手,抚摸我黑色的头发我已经是大人了,妈妈,我要带你回家带你回家我已经长大了,然后我很想你妈妈不要把我扔在角落里,街上有那么多的人,但是却找不见你的影子我亲爱的妈妈我总是想起回家的路,我总是一遍遍地想起想起我抬头看天,空气是流动的,天空的颜色很美丽。格子笑,她
说,默默,过几天我就会回来。我说好的 

暮色时分独自去了我们曾经就读的小学,那里已经重修过。兵乓球台换了位置,教学楼也是新盖的,楼顶不能上去了,学校考虑学生安全,给顶楼的铁门加了锁。旧貌换新颜,物非人亦非。我再次一个人
\newpage
坐在乒乓球台上,晃荡着腿,到夫黑。操场上有小孩子在追打嬉闹着,他们的叫声穿透我的耳膜。关于格子的所有,像电影片段一样在我眼前不停地重现,重现。心里面沉沉的忧伤也从那些小孔中慢慢地慢慢地
逃遁出来。 

我记得,是谁在春天还没到来的时候,爬着梯子在屋檐下很细心地砌鸟窝,或是把四方的纸盒子做成家的样子,里面还有棉花铺的温暖的床,可口的食物和充足的水;我记得,我来这里第一次手发病的时候,是谁冲出教室跑到医务室拿膏药,仓促下楼梯的时候跌破了膝盖,然后我的抽展里每天都少不了音药:我记得,是谁在我牙痛的时候跑到山里去找土药方子,结果开得自己一身伤痕;我记得,是谁因为我的皮肤不好不能多接触碱类物质而帮我洗衣服,然后自己的手也开始蜕皮;我记得,是谁第一个送我可以出现飞鸟的杯子,叮嘱我记得喝热水;我记得,是谁陪我一起偷价跑到大堤去慰问抗洪的解放军,用不多的零钱都买了冰棍直到化了也不舍得自己吃;我还记得…我一直记得,都清清楚楚地记得。亲爱的格子,你听见了么。你告诉我你会回来的。可是你就那么不声
\newpage
不响地从我的视线里消失了,如同那握不住的匆匆云
烟。 

我亲爱的格子,请你跑慢些,再慢些,让我好好记住你的脸流年婉转逝。我又退回到沉默不语的样子了,独自站在苍白的岁月里翘首回望。那些写满祝愿的纸飞机,从那个神秘的古镇飞来,渗着格子的清新气息,被我放到一个好看的檀木盒子里。格子说过喜欢这样精致的木质盒子的,可是她把那些钱省下来给我买很贵的热水感应幻化杯了。我的手又开始频繁地疼痛,指尖白色的浮肿已经开始蔓延了,身边空荡
荡的,已经没有人再为我细心地包扎伤口。 

习惯了一个人跑步,奔走在疏离的边缘。习惯了在很热的夏天晃悠在大街上,让太阳把自己晒得很疲惫,好把一路的思想都蒸干。习惯了晚自习后在路上看着自己的影子被路灯拉得很长很长,一个人无声融进夜色,一个人无声哭泣。不需要谁的安慰,所需要的只是一个安宁的空间,安静地养伤。直到遇见了
墨墨 

\newpage

默默,我要做你的朋友。在榕树下的留言版上看到这句话的那一刻,我是幸福的。我知道墨墨是不会轻易说出这句话的,找能想象得到网络那端她郑重的表情。于是又想起了格子,那个梳着羊角辫的安琪儿第一次在我面前认真说出这句话时我有些不屑的态
度。我不可以再伤害墨墨了。 

三月三日,默默出生:六月六日,墨墨出生。九月九日,格子出生。九月,默默认识了格子。九月,格子离开了默默九月,墨塑刚好出现了默默,聚器。我喜欢这些名字,相似的形状,相同的发音,读的时候可以感受得到上下两片嘴唇碰触的快乐的声音。发音轻缓,平和,不需要借助舌头。路房总是在很深很深的夜里用很空灵的声音安慰着我。同样是生活在南方但是深爱着北方的孩子,埋没于浩瀚的题海,在很努力地为自己的蓝色理想奋斗着。尽管这样,墨墨还是会抽出时间来给我写-些小东西。我们总是用白色的A4纸,黑色的钢笔字,干净明朗。透过那些字,仿佛能看到墨墨疲惫却幸福的神情。很庆幸认识了展精这个医生,她医好了我的伤口。上帝是公平的,

\newpage
他在把格子带回天堂的同时,给我留下了墨蟹。 

我在电话里告诉墨墨,我要去看望格子了。帮我送首歌给她吧。墨墨说于是我拿来录音机。听清楚了,是《天堂里没有车来车往》,还有眼泊的流动。墨墨说,我们都是这个美丽世界的孤儿。别哭,亲爱的人。我终于去看格子了。三束野雏菊,一盘磁带,还有装在植木盒子里的纸飞机。当纸飞机漫天飞舞的时候,我看到格子在中间向我招手,她叫我默默,她不停地叫默默,默默。我伸出手去抓,却怎么也抓不
到,只能眼睁睁地看着她慢慢离去。 

泪流满面,泪流满面,转眼之间,全都消失不见我害怕,我无能为力,我不知道墨墨是否有一天也会这样无声地消失晚上临睡前,听到电话铃声,提起话筒,却无人应答,但我清晰地感受到了某个人温暖的呼吸,是整照,没错,我肯定。我的泪水如浩劫般
再次降临,因为听到了这样一首歌《虫儿飞》 

黑黑的天空低垂/亮亮的繁星相随/虫儿飞,虫儿飞/你在思念谁/天上的星星流泪/地上的攻瑰枯萎/冷风吹,冷风吹/只要有你陪/虫儿飞,花儿
\newpage
睡/一双又一对才美/不怕天黑,只怕心碎/不管累累/也不管东南西北……

\end{document}
