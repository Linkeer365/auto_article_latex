\documentclass{article}
\usepackage[utf8]{inputenc}
\usepackage{ctex}

\title{当水变成了冰雪\footnote{Click to View:\url{https://web.archive.org/web/20220710104532/http://libgen.rs/book/index.php?md5=FC18A13E765587CBA9CA756D53F7BFB3}}}
\author{吴建美}
\date{2008-10}

% \setCJKmainfont[BoldFont = Noto Sans CJK SC]{Noto Serif CJK SC}
% \setCJKsansfont{Noto Sans CJK SC}
% \setCJKfamilyfont{zhsong}{Noto Serif CJK SC}
% \setCJKfamilyfont{zhhei}{Noto Sans CJK SC}
% \setlength\parindent{0pt}

\begin{document}
\CJKfamily{zhkai}

\maketitle


\Large

流流环着胳膊,交叉在胸前,眼睛眺望着潋滟的河水,涟漪泛泛。她几乎每天都要骑着自行车走很远的路来到这座小城唯一的桥上,一个人看水。这是个很臭的毛病,奶奶是这么说的,因为流流放学总是站在这里发呆,不回家帮奶奶做家务,有时甚至忘记吃饭。夕阳慢慢泛红,像是被技艺高超的画匠用精巧的工笔手法一点一点小心翼翼地着成了红色。流流脸上的沮丧随河水渐去渐远,心情在被流水慢慢抚平,河水载着橘黄色的光,看上去很美。流流转头望望桥头上的钟表,已经6点半了,糟糕,又要挨奶奶骂了。她跨上自行车奋力飞驰。流流喜欢水,每次和水接触总有特别舒服的感觉,这也正是流流喜欢游泳的原
因。 

流流家住的是将要拆迁的古老平房,几处方方
\newpage
的房子被面对面地图在一个院落里,有点北京四合院的味道。离院落很远时就能听到随风飘来的邻居大妈在家长里短地聊天的声音,流流竖起耳朵,推着自行车朝院里走。突然,流流的嘴角像是被话语硬生生地扯回了平角,一个熟悉得无法再熟悉的名字野亦地挤进了流流的耳朵,刘彻——她的父亲。“刘彻真是短命啊,怎么好端端地就这么死了呢?”、“哎,留下孤儿赛母的,真是可怜,前些天我去商场看见他老婆了,听说嫁了一个有钱人呢。”、“就是前几年跑了的那个女人?……”流流突然觉得自己好像得了耳鸣,慌乱得一时不知作何表情,在这个特殊的时刻,竞没有一个表情适合流流,她只觉得一股电流从身体里蹲过,其实她早就料到会有这样的结果,因为爸爸出去打工已经四年没回来了,而且一个电话也不曾来过,流流只是侥幸地希望自己想的是错的,她总在胡思乱想的时候去问奶奶。每次间的时候,奶奶只是坐在椅子上低着头打毛衣,无论冬夏,一件又一件,是很大很厚实的那一种。流流便也不再追问,只是心里隐隐地痛。她好几次看到搁在桌子上的毛衣沾了许多的水泪。也许是平静的水看多了,痛像水一样柔柔地浸

\newpage
没流流的身体,冰冷而平缓地莫延。 

流流双手紧紧地担着车把,似乎要把所有的痛都变横地强加在车把上面,吃力地推着自行车走进院子,一个眼尖的大婶一见流流,立马干咳了几声,其他人见状就利索地把嘴锁了起来,默契不语,很有八卦的风度啊。流流条件反射似的从嘴角挤出她自己也知道看了令人难以下饭的笑对着她们。奶奶正在公用厨房里煮饭,雾气氤氲,她弯曲着的背仿佛是虾米。开水的呼啸声与狗的叫声呼应着,使得瞬时寂静的小院又沸腾起来,流流麻木地拔掉插销,擒起水壶向暖水瓶里灌水,沸水与瓶壁碰撞的声音纠缠着,流流的
心都要碎了。 

小屋里整晚一句话也没有。夜里,流流望着天花板发呆,耳边不断回响着那句话:“就是前几年跑了的那个女人。”转过脑袋,奶奶只留下一个侧着的瘦弱的背影。流流强迫自己闭上眼睛,想着“那个女人”,那个她管她叫妈的女人,她已经四年没有见过的女人,她又回到了这座小城——那个抛家弃女的女人,大大的泪滴顺着脸颊滴到枕中上,尽管流流自己

\newpage
骗自己说不屑为她流泪。 

流流在奶奶的强迫下,在学校组织的预备初三毕业班的补课班中挣扎着。其实,她并不是一个为了分数的美丽而高兴的女孩子,也并不适应在应试教育的沼泽里挣扎,她喜欢像水一样自由流淌的日子。不过,小时候听和爸爸说,奶奶是个上过私塾的女人,懂得知识的重要。流流心里也明白,现在奶奶是她唯一的亲人,她不想让她最后的亲人失望,虽然她并不赞同奶奶的做法。她很了解自己,知道那些所谓的知识根本看不上自己,又怎会与自己志同道合呢。想到这里,她放下了在指间旋转的中性笔,转头望向窗外,耳边传来的是从化学老师肥硕的双唇里挤出来的字符,缓慢而连贯。“刘——彻。”她艰难而迷茫地在心里默默地重复着这两个字,仿佛要把它们念碎才能表达她的情感。窗外洋洋洒洒地飘起雨来,八月的细
雨是温暖的,除去了闷热的痕迹。 

流流眼里含着泪花,却没有流下来,她不想让别人看到她的脆弱,包括奶奶在内。她答应过自己要做个坚强的人,虽然现在她觉得自己是外强中干。她

\newpage
只是含着泪,在本该记笔记的本上这样写道: 

我只觉得自己很可怜,虽然我一直觉得自怜是
很懦弱的表现。 

流流今天没有去大桥上看水。她突然很想奶奶,一放学就满脸焦急地奔向车棚,不顾好友兼同桌心月在身后千里传言。流流气喘吁吁地把自行车推进院子,看到了奶奶弯曲的后背才心安地笑了。奶奶听到停放车子的声音,回头看到流流红扑扑的脸上还在向下滴汗,便挥舞着炒勺心疼地责难:“你这是干吗呢?怎么弄得满脸都是汗?也不怕中暑,快过来我看看,瞅瞅,这后背都湿透了。”流流看着奶奶唠叨的样子,会心地笑了。“你今天怎么回来这么早啊?知道
早回来了啊?!”流流只是抿着嘴不说话。 


晚上,流流在日记中写道: 

今天下雨了,牛毛般的雨丝,我的心也像雨丝一样被疼痛分割成了缕缕细丝,不同的是雨是暖的,而我的心却是冰冷的。可我知道上天还是怜惜我的,留给了我一个奶奶,和我相依为命的亲人。看着雨,
\newpage
她的样子在我的脑海里模糊了,突然渴望能看到她的身影,迫切地。当我见到她时,我有种心安的感觉,遗失很久的幸福感觉。我知道,她就是我的全部了。
 

流流有时就像个小孩子似的,疼痛就如同水一样从她身上流过,却也不再回来,似乎被她忘记了。
她知道她并不是一个人,最起码她还有个奶奶。 

课间,流流和心月正在用功,小灵通灵灵就横冲直撞地奔了进来。她站在讲台上,一边大口大口地喘着粗气,一边张牙舞爪地比画着,真是能把孙猴子急死。流流的第六感觉得好像有人在看她,她合上课本,眯着眼睛慢慢抬起头来。果然,灵灵一直注视着她。老猫是个急性子,见灵灵半天了还在喘气,就龇着牙发起猫威来了,两手往腰上一又,醒着脖子,拉开腔调,说:“灵灵同志,您老这是干吗?喘气怕大家看不着啊?还站到讲台上喘,有话就快说。”灵灵转过头来,恶狠狠地瞪了老猫一眼,心想,死老猫,阴阳怪气的,讨厌,看我空了怎么收拾你。大家为了满足好奇心,就撩着性子等灵灵运功调息完毕。灵灵
\newpage
总算开口了:“学校这回总算大发慈悲了,我接到准确消息,下个礼拜开学后,咱们初三会举办一次游泳比赛!”大家顿时欢呼汰跃,疯狂得有点像要放假的精神病院病人。灵灵颠颠地蹿到流流面前,说:“流流,你可要加油啊,争取保持三连冠哦。”流流乐呵呵地点点头,低下头看看心月,发现今天心月特别安静,就连听到她最喜爱的游泳比赛也无动于囊,心月紧紧地担着手中的笔,手,就那样停留在半空中。流流感到一股力量使她的心脏颤了一下,刚要开口,心
月微笑的脸却迎面而来。 

游泳比赛是在一个阳光明媚的下午,连续三年的比赛结果就像停滞不前的记忆,被相机定格在原地。女子比赛的冠军依然是流流,虽然她不经常游泳,但她却像是鱼儿,一碰到水就精力充沛,这是体育老
师说的。亚军是心月,季军是季淼,一成不变。 

老猫咧开嘴说:“哈,还是咱们流流厉害,像鱼一样,我就说你会是第一名的,准吧?还不请我吃饭啊?”灵灵轻打了老猫一巴掌,翻白眼说不用你说人家也是第一名,没见过你这么不要脸的呢。老猫故
\newpage
作严厉状说:“怎么?你以为我是胆小的周天一不敢还手?切,我是好男不跟女斗!哎呀,你还逞起英雄来了?!”灵灵瞪着老猫,抬手要打,老猫拔腿便跑,灵灵就追了出去。真是欢喜客家啊,流流摇着头想,见心月在角落里默默地收拾东西,她走过去和心月
说话。 


“心月,干吗呢?” 

“哦,收拾收拾,没事干啊。”心月停下手中
的活,抬起头说。 


“嗯……” 


“你真厉害,还是冠军啊!” 

“嗯,你也不错啊。”流流总算放心了,心月
并没有生她的气哦。 


“其实第几名都无所谓啦。” 

\newpage

晚上,流流把奖杯放在桌上没有言语,奶奶冷冷地瞟了一眼就去洗衣服了,流流知道,奶奶并不喜欢她游泳,可她并不知道为什么。夜里被梦惊醒,蒙胧中见奶奶坐在桌前,双手似乎在抚摩那奖杯。流流揉揉眼睛,发现奶奶的脸上还挂着泪痕。流流知道,奶奶肯定是想爸爸了,记得儿时听和爸爸提起过,他
也疯狂地爱着游泳。 

流流早上起来,发现奶奶还在睡着,便悄悄地
上学去了。 

在路上,流流遇到了灵灵,流流奇怪为什么心月没有和灵灵一起上学,她们俩住对门,都是天天一起上下学的。灵灵知道流流肯定会问起心月的,就先开了口说:“哦,我没有等她,今天我不想和她一起
走。” 


“可为什么啊?”流流很好奇。 

“流流,你觉得心月是一个怎样的人呢?”灵

\newpage
灵似乎并没有听到流流的话。 

“她啊,她是一个臭美的小丫头,花钱大手大
脚的……” 

“她还是个争强好胜的女生。”灵灵打断了流
流的话。 

流流没有介意。“嗯,她是个不服输,喜欢把
事情做到最好的人,怎么了?” 

“那你觉得她介意每次都是你拿冠军吗?”灵
灵开始移动脚步向前走。 

“应该不会啊,昨天我以为她生气了,可是她说第几名都无所谓啊。”其实流流心中也有一些担心
,毕竞心月除了游泳,什么都是最棒的。 

“但愿如此,可是……我还要去值日,先走了
。”灵灵匆勿地跑开了。 

流流慢慢地走着,灵灵是个直肠子,今天说话
\newpage

怎么绕这么大弯子,还吞吞吐吐的? 

流流心事重重地上着课,心月今天却格外活跃,她上课与老师一唱一和的,下课就给流流讲笑话,没等流流乐呢,她自己却已前仰后合了。流流看着心月开心的样子,拍着自己的脑袋责怪自己,我怎么可
以担心心月呢,该死的脑袋。 

放学收拾书包的时候,心月问流流:“明天开
家长会,还是你奶奶来吗?” 


“什么?明天开家长会?谁说的?” 


“老师刚才特别嘱咐的啊。” 


“老师?我怎么没听到呢?真该死啊!” 


“你今天到底怎么了啊?” 


“没什么,有点呆,脑子长锈了。 

\newpage

流流无精打采地回到家里,发现奶奶不在公用厨房里,就丢了魂似的一通乱撞。她踉跄地推开小屋的门,看到奶奶躺在床上,对门的李婶正拿着药过来。听她说,奶奶已经一天没下床了,这都立秋了,早晚有些凉,可能是昨上晚受了风,着了凉。流流来不及放下书包就坐到奶奶身边,奶奶睡着了,李婶把药放下,摸摸奶奶的头轻声说还有点烫,待会李婶醒了,就把这药给她吃了,这两天别让她下地了,人岁数大了生病不容易好,也容易落病根,我先回去做饭了,待会儿再过来。流流点着头,道了谢,目送李婶出屋后,就守在奶奶身旁,还不时地用手摸摸奶奶的额头。流流就这样呆呆地守在奶奶身旁,直至奶奶的咳嗽声把她从沉思中晚醒。她赶忙回过神来,奶奶只说了一个字,饿。流流这才想起来已经7点了,她还没有给奶奶弄吃的呢。公共厨房里出现了一个忙得不可开交的身影,其实也没做什么复杂的食物,只是蒸鸡蛋赣和热牛奶就足以让一个基本上不下厨房的流流忙得四脚朝天。推开小屋的门,奶奶又睡着了,流流坐在煤气灶旁的小丝子上,哭了。煤气灶上的火焰不紧不慢地一边做着自己的本职工作,一边默默地看着流流的泪花闪烁,不时发出腾腾的声音,似乎是在心疼
\newpage

流流——这个可怜的孩子。 

奶奶在吃完了流流煮的东西,满足地又睡着了。流流依旧坐在奶奶身旁看她睡着的样子。流流心里突然好怕,她怕奶奶终有一天会像这样一直睡下去,
再也不会醒来。 

顽皮的太阳终于起床了,它拨开云彩,露出红彤彤的脸蛋。流流睡眼惺忪地坐在教室里,她并没有把开家长会的事情告诉奶奶,只好一边听老师念经,
一边想着如何跟班主任交待。 

心月见流流一上午恹恹的,便问:“流流,你
怎么了,这么没精神?” 

流流几乎是说梦话一般:“我奶奶感冒了,我
昨天照顾她来着。 

“啊?那没什么大碍吧?”心月把眼睛睁得大
大地望着流流。 

\newpage

流流支起沉重的眼皮:“没什么,不过要几天
不出屋,以免再受风。 


“那还好,那她今天不就来不了了?” 

“是呀,我也一直在想这个问题,我要怎么和老师交待呢?”她把右臂杵在桌子上,用手掌支撑着
脑袋说。 

“嗯,你就直接和老师说呗,你奶奶不是本来
就病了吗?这是事实。” 

“关键就在这里,要是你说老师肯定信,可我,老师能信吗?我骗老师已经骗到说实话她也绝对不
信的地步了。” 

心月把眉毛皱成了一个疙瘩,转着手中的笔,没有做声。透明的笔杆在心月纤细的手指间翩距,更像是一个舞者。流流已经知道答案了,她终于有兴趣看看讲台上那个讲得眉飞色舞的数学老师到底在可怜

\newpage
的黑板上做了些什么。 

流流盯着那些单个都很熟悉,合在一起却搞不慌的字符对旁边的心月说:“你们家谁来”你爸爸有空吗?”心月的身世其实也挺特别的,她妈妈多年以前就去世了,爸爸为了支撑这个破碎的家,把所有精力都放在了事业上,经常把心月托付给对门的灵灵的妈妈照顾。心月却很少让灵灵的妈妈照顾,而是自己一个人在家,即使爸爸出差也是这样。后来,心月就变得很要强了,什么事情都会做到最好,不然绝不罢休,这也不知到底是福还是祸。这是流流以前听灵灵
讲的。 

零星的弧线划破了心月的脸,那是她神秘的微笑。流流看得出来,心月是在抑制心中的喜悦,但那幸福却如洪水泛滥般冲人流流的心窝,一阵酸楚。“
秘密,明天我再告诉你吧。” 

流流低头看着恪尽职守的勤奋的表针,想,一
天又这样流逝了。 

流流在办公室门外徘徊时,心月恰巧经过。“
\newpage

流流,你在干吗呢?” 

“哎,我在想怎么向老师开口呢,心月要不你和我一起去吧,老师比较信任你。”流流右手挠头,
用水汪汪的大眼睛望着心月。 

心月摊了摊手,说:“老师在班里呢,你在这
干吗啊?” 

“啊?”流流张大的嘴巴足以塞进一个鸡蛋了
。 

“走吧走吧,去班里,速战速决,等会儿家长会就要开始了。”她一边说,一边用手推着流流下楼

流流深深地吸了一口气后,踏进了教室,本来准备迈向老师的脚,似乎在瞬间被施了葵花点穴手,凝固在半空中。她看到一个穿着粉红色外套很惹眼的女人正微笑着向老师询问她家孩子的情况,从老师微笑的幅度可知她孩子的学习成绩一定远远在流流之上。在停顿了5秒钟之后,她便以迅雷不及掩耳的速度
\newpage
奔出了教室,消失在走廊的尽头,身后只留下心月疑
惑的表情和一连串的呼唤。 

残阳如血一般,慢慢地浸染着周边洁白的云彩。在教学楼间一个夹角里,一个残阳照射不到的角落,流流终于停止了奔跑的脚步,如被人握在手里的生鸡蛋,稍稍一用力,便在一刹那之间崩溃了,她一下子瘫软在地上,坚强的只是外表。泪水再也无法控制,湿了整个面颊,但她只是无声地抽泣,整个身子亦
在不停地颜拌。 

心月不知什么时候已站在了流流的对面,她只是静静地注视着流流。流流的脖子像生了锈似的慢慢抬了起来,泪水再次夺眶而出。“她……那个女人,她……回来了,还是穿着粉……粉红色……衣服。”流流哽咽着。心月的左脚向前迈了一小步,蹲了下来
,表情迷茫而担优。 

流流擦着眼泪,鸣咽着:“心月,你知道吗?那、那个女人,我的妈妈……妈妈,她来开家长会了,给别人的孩子,来、来开家长会了。”她稍微调整
\newpage
了一下激动的语气说:“一个抛弃了我四年的女人,如今却出现在我们班里,为别的,别的孩子开家长会
……” 

可以看出心月震惊了,脸白如纸,她愣了一下,干干的双唇轻轻地碰了碰,问道:“你的妈妈?你
确定吗?是不是眼花了?这怎么可能?” 

“怎么会呢,她是我妈妈,不,她不是我妈妈,我不会认她的,不,那个女人,坏女人……”流流的手在空中胡乱地挥动着,她被刺激得有些语无伦次

心月把手搭在她的肩上,柔柔地拍了两下,说
:“别急,我相信你,你,你先休息一下。” 

流流的双臂环着蜷曲的膝盖将头深深地埋在臂弯里,像是一只受了惊吓的小猫,缄默地坐在地上,
只有双肩还在不停地颤抖。 

心月想让流流稍微冷静一下,就去学校的小商店买了两瓶矿泉水,回来的时候握着矿泉水的手心沁
\newpage
出了细细的水珠,小到心月自己都没有察觉到。“喝点水吧,嗯?”她用一瓶水轻轻磁了碰流流的手臂。流流无力地点点头,把水接了过来,她突然觉得前所
未有的累,连接一瓶水也要费尽全身的力气。 

她们沿着路慢慢地挪动着脚步,许久,只有寂静,残阳邪恶地跟在她们后面,将她们的影子拉得越
来越长。 

“你妈妈……”心月停顿了一下,乜斜地看了一眼流流,“她不是出走很久了吗?你怎么还那么确
定就是她呢?” 

“我们是血脉相连的,就算她再怎么变我也能认出她,更何况她几乎没有变呢。”流流说话的时候,手紧紧地握着水瓶,几乎能听到咯吱咯吱的响声。“那粉红的颜色,她最喜欢的颜色,嗯?她到底来给
谁开家长会?” 

心月似乎在想事情,并没有回答流流的问题。心月跤了几步,她忽然转过头来对流流说:“你还是
\newpage
快回家吧,别的先别想了,你奶奶还病着。我有事,
也得赶紧回去了。” 

流流已经筋疲力尽了,她并没有注意到心月心事重重的样子,而且她急于回去看奶奶,便点点头,
拖着疲惫不堪的身子向家的方向走去。 

天,撑着深沉的蓝色,慢慢暗下来了;她,踩着沉实的路基,缓缓踏进门来。奶奶不知何时已经从床上爬起来了,正趴在凉了许久的饭菜前睡着,锅里的面条似乎知道要被吃掉,恐惧地抱成一团。看着奶奶又瘦了一圈的单薄的脊背,流流的泪水再次夺眶而出,浸湿了睫毛。她蹑手蹑脚地将一件上衣披在奶奶
背上,就再也支撑不下,倒在床上昏睡了过去。 

当她睁开双眼时,奶奶正焦急地望着她。见她醒来了,奶奶激动得几乎哽咽着说:“你总算醒了孩
子,你都睡了一天一夜了,急死奶奶了都!” 

流流眨巴着眼睛,像个受了惊吓的小孩,一下子张开双臂扑到奶奶怀里,双手紧紧地搂着奶奶的脖
\newpage
子。这是这几年以来流流和奶奶最亲密的一次举动。仿佛过了有一个世纪那么长,流流幽幽地在奶奶耳边吐出几个字来:“她,回来了。”有液体滑过她的面类,洒在奶奶从藏蓝洗成浅色的中山服上,轻巧却不乏力度。“那个女人跑到我们班上来给别人开家长会,她,不是来找我的。”流流的声音开始变得颤抖,手用力地抓着奶奶的衣服。月光无情地穿破玻璃,染白了奶奶的头发,飞扬跋扈的光芒,刺得奶奶眼睛红肿,眼眶中闪烁着泪花。“她6月份就回来了,还来找过你。”流流很惊讶,一股脑儿坐了起来,眼神喜忧参半。“那天你在大桥上看水,还没有回来;她来了,说很想你,想接你去新家,因为她刚和另一个男人结了婚。我没同意,我无法忍受她在你最需要她的时候跑掉,现在还好意思来找你,我就把她骂跑了。”奶奶在说这席话时,很是小心翼翼,语速慢得几乎可以让每个字都在空中盘旋好几遍,但还是掩饰不住她的愤怒,她的眼睛一直直视着流流。流流看到奶奶的假牙扣在唇上,泪水亦在嘴角徘徊,她深深下陷的眼窝中,一对炯炯有神的眼睛充满了沧桑。流流可以想到,在四年前那段艰难的日子里,奶奶既要面对失去儿子,白发人送黑发人的现实,又要用她那已经不
\newpage
再结实的臂膀支撑支离破碎的家,而在这时,她的儿
媳又抛弃了她,这是多大的打击啊! 

“奶奶,我不会再理会关于那个女人的任何事了,我妈妈早在四年前就死了。我们好好地生活下去吧,好吗?我会努力学习的。”流流斩钉截铁地说。

“好孩子,我知道你会的。”奶奶用手帮流流擦泪,而她自己也已泪流不止。流流第一次发现,奶
奶的手上竟布满了裂口。 

“奶奶,奶奶,我饿死啦!”流流撤娇地向奶
奶抱怨着。 

“傻孩子,能不饿吗你?睡了一天一夜。”奶
奶微笑地看孙女四年以来第一次向自己撒娇。 

在以后的日子里,流流再也没有向心月提到过那个女人,而是发奋图强起来。她的成绩也像刚上市的绩优股,一路飙升,在短短的两个月里,直逼心月

\newpage
第一的宝座。 

随着中考的逼近,学校的课时也开始不断猛增,每天放学回家都要八点半以后了,流流根本就没有
时间天天去看水了。 

十月底的夜空是深邃而忧郁的,暗含着不为人知的一面。今天灵灵不知为什么没有等心月就先走了。晚上放学后,心月一个人站在校门边徘徊,正巧碰
到了流流。 

“心月,干吗昵,这么晚了怎么还不回家啊?
灵灵呢?”流流推着车子,停下了脚步。 

心月犹瑚了一下,说:“她估计有事吧,先走了。嗯……流流,你能陪我回家吗?我家那边太黑了,我不敢自己回去……没事,你要有事就走吧,没事
的。” 

“走吧,我陪你吧,把你这么一个美女放在路上我也不放心啊。”流流啊开嘴笑了,想,最近小城

\newpage
里这么乱,万一心月有个三长两短可怎么办呀。 

“呵呵,你真好。”心月挽过流流的骆膊,并
排地走着。 

一路上心月像吃了兴奋剂一样说这说那,流流只是静静地听,微笑着。这是最黑的一段路了,没有路灯,路也窄得只能过一辆车,而且凹凸不平,过了
这就是心月的家了。 

心月正在比手画脚地讲一个笑话。恰巧一个醉汉从旁边经过,说他是醉汉为他走路摇摆不定,而且
酒气熏天,老远就能闻到。 

心月在比画一个大西瓜的时候,手一不小心打到了他。他恶狠狠地吼道:“找死呀?”心月连忙收回手臂。“对不起,对不起,天太黑了,没看到你。
“没长眼睛啊?信不信我踩死你们?” 

心月突然也火冒三丈,叫道:“有你这么说话的吗?我好好给你赔礼道新,你却出言不逊,谁是该你骂的呀?”“你个小毛丫头是不是欠搂?”说着,
\newpage
他的拳头就冲着心月飞来了,只听哐啷一声,那拳头扑了个空,说时迟那时快,原来是流流把自行车扔在了地上,三步并作两步地跑过来把心月往后搜了一下,心月站在远处喘着粗气,似乎被吓得说不出话来了。淡淡的月光下根本很难关认出动作,流流只感觉到自己的背部正中一脚,巨大的推力使得她奔出了好几米,最终她还是跌在了地上。厮打了几下后,流流急中生智,开始一边掏电话,一边冲着那人嚷:“你等着,我报警,你信不信我报警啊。”那醉汉似乎还不是全醉,当听到要报警时,就开始往后退,走时还不忘冲着流流的自行车重重地踊了一脚,喷喷着,你报警呀,你敢报一个试试,你们在这等着,有你们好看的。见那醉汉的身影渐去渐远,流流稍微松了一口气,她低头看看掌心,全是冷汗,她哪有什么电话啊!“快走吧,这太危险了。”流流回过头冲着边上已经
1分钟没有动弹的心月说道。 

这种人怎么这么不讲道理啊?下回再让我见到他绝对饶不了他!心月一边快步地走着,一边叨咕着

流流似乎没在听,她回头望了望远处的电线杆
\newpage
,发现电线杆旁边站着一个男生,手里还举着一个方方的类似砖头的东西,身子似乎在抖动,流流觉得这
个身影很熟但也没在意。 

流流一直注视着心月的身影消失在楼道口才转身,然后倚在小区的墙上,慢慢蹲下,汗水顺着流流
的额头大颗大颗地打在地上。 

与此同时,一个影子亦映在流流头上,流流猛地抬起头来,眼睛鼓得像两个李子那么大,是天一。流流看到是他,只是大口大口地喘着粗气,没有叶声。“你还好吧?”她无力地播摇头。“刚才在回家路上看到你俩回家,我不放心就也跟着,结果果然遇到了状况。没想到你会那么多,跟那人大打出手,还把他吓跑了。”他不怕被憋死吗,一口气说这么多话,流流想着,说的却是:“那个举着东西站在电线杆旁边的人是你吗?”“呵,呵呵,是,是啊,我,举了一块砖头,可我还没准备好呢,你就把他给吓跑了。”流流寻思,果然很没用,拿着砖头在那发抖,怪不得同学说他胆子小呢,但流流对他还是感激万分,毕

\newpage
竟人家没有见势就跑啊。 

“你的衣服脏了,你真的没事吗?背后还有脚印呢。”天一扶着流流的自行车跟在流流身后走着。

“哎呀,我的自行车。”当流流回头想和天一
说话时,心痛地发现车筐瘪了。 


“被那人踹的?”其实天一早就看见了。 

“一定是,这个恶棍!”流流气鼓鼓地点头,
拍了拍衣服上的土。 

流流就被这么一个比女人胆子还小的男生护送
回了家。 

流流并没有把今天发生的事情告诉奶奶,她怕
奶奶担心。 

第二天早上,流流一醒来就觉得浑身酸痛无比,好像刚跑完5000米一样。流流刚进班就发现大家都用看怪物的眼光看着她,老猫也噶噶地叫了几声
\newpage

,看得她浑身发毛,鸡皮疙阅都起来了。 

终于下课了,流流觉得很不自在就逃到了走廊里,站在窗边看树枝摇曳。“是心月,她说你昨天晚上和流氓打架了。”流流没有回头,她知道背后站着的是天一,但她只觉得有一股寒风从腰间沿着脊柱向
上游走,彻骨地冷。 

心月莫名其妙地换桌了,连一个质问的机会都
不给流流。 


流流不相信心月会散布这些谣言来诉毁她。 

心月那么单纯、可爱,况且事情不是由心月引发的吗?况且她和心月是最好的朋友。况且,她是为了送心月回家……况且……况且……流流不愿再想,她望着汩汩流动的河水,心里平静了下来,仿佛粼粼
水波可以填平她伤痕累累的心。 

她把自行车停在彩虹超市门口,进去买盐,中午奶奶说没有盐了,晚上等着她买盐下锅。她在货架
\newpage
上徘徊,到底哪种既便宜又好呢?一个古怪的想法窜到了流流的脑子里,要不然列个函数解析式吧,没准行呢?想到这里她就自然地咧开了嘴,头也习惯性地向左摆动,却看到了天一站在另一个货架旁。流流正纳冰,为什么总能碰到他?忽然发现天一正将一块德芙巧克力小心翼翼地装进了裤兜里。天一似乎觉察出有人在注视他,于是机警地四处盯盯,像只闻到猫味儿的大老鼠,目光正好与流流交汇,只愣了一秒钟就
慌张地躲开了。 

流流没有追,只是心事重重地抓起了一袋盐走
向收银台。 

晚上奶奶埋怨着,怎么买了这人么贵的一袋盐
?!流流低头扒着碗里的米饭,什么也没说。 

心月遇到流流时总是马上转过身,继续和别人高声说笑,压根没把流流放在眼里,流流也懒得追问了,清者自清。而天一遇到流流时总是低着头,急忙地跑开。流流实在讨厌这种感党,于是,一天课间她走到天一课桌前,用不大不小的声音说,周天一,放
\newpage
学后我在大桥上等你,咱俩说清楚。说完流流就后悔了,说是说清楚,可是似乎引起了更大的误会。因为话音刚落就收到了齐刷刷很暧昧的目光,这阵势决不
亚于升国旗时的注目礼。 

流流感觉自己的脸在一点一点地扭曲,心想完
了完了,误会大了。 

路边的行人匆匆地闪过,唯独留下流流和天一
,像两个被时间抛弃的人儿站在桥上对视。 

车声人声流水声充斥着银色月光下的小城,有一种说不出来的丐美。一个满头银发的老人的经过唤醒了沉思许久的他们。那老人骑着一辆几乎要散架的二八式自行车,他每奋力地路一步,车子都会痛苦地呻吟一次;这刺耳的声音和这沉静的小城是多么地不
和谐啊! 

“为什么?”流流直直地巍着流水,面无表情
地问。 

\newpage

“……”天一也一直望着水,不知为什么,在夜色的掩护下,水儿似乎变得汹涌了,他可以清楚地
听见流水拍打岸边的声音。 

“你敢不敢从这跳下去!”流流的口吻很奇怪,明明是问句,她却说得如此肯定。这座大桥是在改革开放时期建的,桥身是用一种比较特殊的石料铸成,微微弯曲的弧线横亘在30多米的河道上,桥中央距离水面大概有七八米的距离,由于时间比较久,桥
上的几个栏杆已经断裂了。 

天一脸上的肌肉开始抽搞,带着哭腔叫道:“我多想从这跳下去啊!要是我四年前就从岸边下去,也许我的弟弟就不会死了,那个叔叔也不会死了!”天一此时已经蹲在地上,并且不住地用手捶打自己的
脑袋。 

终于,有路人肯牺牲时间停下来了,但他们只是瞪着再也不能睁得再大的眼睛,欣赏着这难得一见
的莫名举动。 

\newpage

一丝惊讶的表情在流流的脸上闪过,她转过头来看着痛苦不堪的天一,她只是听老猫提过,天一以前有个孪生弟弟,几年前在海边出意外死了。她抓住天一的前臂,尽量保持镇静,还没等她措好词,天一
就继续说了下去。 

“为什么为什么,当初去海里捡球的不是我?明明是我抛过去的球啊……”天一用力挣开流流的手,继续捶打自己的脑袋。周围聚集的观众已经自觉地
围成了一个圈,并且还在窃窃私语。 

流流感到天一的情绪越来越激动,再这样下去局势将无法控制,她脸一绷厉声喝道“懦夫,起来!”便开始拽着天一的胳膊狂奔,他们冲出“包围”,
闯过横道,把尖锐的刹车声和鸣笛声丢在了身后。 

最后他们停在一个废弃的工地上。“啊——啊——啊——”流流冲着空有防的工地大声地叫着,回
声在整个地域中舞动着,和原声交相辉映。 

“啊——弟弟——弟弟——我——对——不—
\newpage

—起——你!”天一也开始宣泄心底的抑郁。 

喊累了,他俩背靠背地坐在地上,呼哧呼哧地
喘着粗气。 

夜空宁静如水,一颗星一颗星地舒展开来。凝望夜空许久之后,天一低语道:“我现在好多了,我可以把整件事都告诉你吗?我真的好想找个人倾诉。
流流没有出声,却在他背后点了点头。 

“四年前,那时我和我弟弟还都在上五年级,我们在同一个班。放暑假时由于父母工作都很忙,就把我们送到Q市的奶奶家去避暑。我奶奶家的房子就建在离海边很近的一个小后上,房子不大却很温馨,我和弟弟在那玩得很尽兴。那一阵子那儿的孩子很流行玩沙滩排球,这也是我们的最爱。就在一天下午……”天一说到这心里似乎很害怕,他回了一下头,他想看看流流到底有没有在偷看他。“就在一个阳光本是很明媚的下午,我和弟弟决定在海边玩排球,就是那该死的海风,不,是该死的我,把排球推进了海里一处很远的地方。球,就在那里随波漂流,弟弟看看
\newpage
我,我也瞅瞅他,他知道我是只旱鸭子,最多只能在岸边踩踩浪花,于是他扑到海里去捡球,可是这一去就再也没有回来。我眼睁睁地看着弟弟游向排球,越来越近了,他终于抓到球了,他兴高采烈地冲我挥手,却突然惊慌失措地挣扎着向下沉,他双手无助地拍打着海水,一点一点消失在海面上,先是肩膀,然后
是头,接着只有手指在水面上挥舞了。 

讲到这里,他已泣不成声,流流感到他背上的冷汗开始浸湿她的衣服了。“我早已被眼前的场景吓傻了,已经面无血色无法思考了。只见一个不是很强壮的男人顾不得脱下衣服就一个猛子跳进了海里,他有力的双臂切割着水面,近了近了,可是弟弟早已消失在茫茫的海面上了。男人来不及多想也钻进了水底。时间似乎是蜗牛背上的壳,沉重地挪动着脚步,逼迫在我的心头,使我无法呼吸。一浪打过一浪,他们还没有踪影,我先是呆呆地桂在那里目不转睛地盯着海面,然后就奔向海里,直到海水没到我的膝盖,我害怕了,不敢再往前走了,我想起了小时候第一次游泳呛水的事件,就在我犹豫之际,那个叔叔腾地从海面上出现了,肩上还驮着呛水的弟弟。看着他们缓缓
\newpage
向我游过来时,我兴奋得手舞足蹈,可是他们突然又停住了,似乎是那个叔叔抽筋了,他痛苦地挣扎着,我眼看着弟弟再次在海面上消失,叔权也慢慢地往下
沉……” 

流流皱着眉偷偷瞄了一眼背后的天一,她已经不敢看他的脸了,她发现他的手,放在侧面的枯瘦的手,青筋纵横,紧紧地拘着一把沙。“我终于有了意识,我一边向回疯跑着一边喊着“来人啊,有人溺水了,快来人啊,救救我弟弟"。当我带着人赶回海边时,只有那个排球在水上漂着。大海依旧在吃哮,似乎什么事都没有发生过。晚了,一切都太晚了。我跪在沙滩上绝望地望着海面,我不相信这是真的。熟悉水性的人开始下海搜寻,一拨又一拨,奶奶也赶了过来,她看到我跪在那里就晕倒了。直至夜幕降临,弟弟最先被抬了上来,他脸色苍白地躺在沙滩上,我冲着人们大叫:“为什么不叫救护车?怎么不救我弟弟?”“晚了,他已经不行了。”一个人回答道。然后,那个叔叔也被找到了,他俩并排地躺着,这对我无疑又是重重的一击。和爸妈来了,然后是葬礼,一连一个礼拜,没有人过问我,我也没和任何人说过话,
\newpage
我整日整夜地充斥在自责之中,为什么,为什么当初不是我去捡球?为什么我的胆子会这么小?这到底是
为什么?” 

天一猛地站了起来,然后又狠狠地跪在了地上
,歇斯底里的喊声响彻整个工地。 

流流已经不知道再用什么办法来安慰天一了,
她只得静静地望着他,感受着他的痛苦。 

不知过了多久,天一开口问道:“我的胆子是不是真的很小?”流流的想法已经不言而喻了,这是个不得不承认的事实。“我知道我的胆子很小,自从弟弟去世后,我就越来越胆小了。我甚至怕老鼠,怕上课发言,怕与取笑我的人争执。我真的好痛苦。我
受够这种日子了。 

流流可以想象他的痛苦。“你偷东西不会是为
了排解痛苦吧?” 

“不,我想练我的胆量,偷东西是需要很大勇
\newpage
气的。”天一跪得太久以至于想换个姿势坐下时腿都
不听使唤了。 

“可是这是犯法的,你不知道吗?”流流一直
盯着天一的眼睛。 


“犯法是需要更大的勇气的。 


“你早晚会把自己毁了的,你……” 

天一抢先说:“我自己早就毁了,父母根本不管我了,看到我只会让他们觉得痛苦,同学也都瞧不起我,我根本没有朋友,我压根就是供人取笑的笑柄。哈,哈哈……”天一仰天苦笑着,泪水不争气地流
进了他张大的嘴里。 

“你这个懦夫,别人看不起你,你就不会让别人看得起你?你偷东西永远不会有人看得起你;你没有朋友我可以做你朋友,但是,我绝对不会跟偷东西的懦夫交朋友,你明白吗?”流流突然有种恨铁不成钢的感觉,赌气地用拳头拼命地砸在地上,地上留下
\newpage

了一片血迹。 

天一被流流的话语和举动震惊了,像个知错的孩子,一个劲地点头:“我知道了,我错了,我错了,我不会再偷东西了。流流,你的手在流血呢。”天
一突然感觉很心疼。 

经过天一的提醒,流流才意识到手在针扎般的
痛着,十指连心啊。 

天一抽出一张纸巾,动作灵敏地把流流的手包
了起来,而后两人会心地笑了起来。 


就这样,流流和天一成了好朋友。 

“哎,为什么最近有这么多事情发生在我身上呢?整得我都没心情学习了。”流流像是在自言自语
,又像是在对身旁推着自行车的天一说话。 

“就是为了让你不要学习啊。”天一看着远方

\newpage
微微泛黄的叶子无意间脱口而出,却也意味深长。 

“什么意思?”流流猛地抓住天一的袖口问道

很不巧,这个动作恰好被旁边的同学看到了。本来就沸沸扬扬的谣言,似乎得到了证实一样,他们
在旁边小声地窃笑着,很是暧昧。 

天一干咳了一声,示意流流把手放开。流流并没有要撒手的意思,看样子她正在等待天一的回答。天一无奈地笃息肩,道:“到前边吧,我就告诉你了
,这里太乱了。” 

“好了,现在你可以说了吧?到底什么意思,你是不是知道什么了?”当他们站在一处安静的胡同
里时,流流略带忧郁地问道。 

她早就感到事情不会这么巧,什么都会不约而同地发生在她身上的,毕竟这又不是拍电影;一定有什么不好的原因,明明已经意识到不好了,但她却无
法抑制住自己去问。 

\newpage

天一的脸色亦不是很好,他有点犯难地问道:
“我怕你会难受,你真的一定要知道吗?” 

流流再坚定不过地点了点头:“关于我的事我
是应该知道的,不是吗?” 

“好吧,本来不打算告诉你的。你还记得那天晚上你送心月回家时出现的状况吗?那个全都是心月一手策划的。”讲最后这句话时,天一一字一顿。流流似乎正在专心致志地玩弄着脚下的一块石子。“因为心月突然之间就不理你了,还到处说你的坏话,我知道你很伤心,就决定和她谈谈。大概一个礼拜之前吧,放学后,我跟着她想找个适当的机会和她说话。结果跟着跟着,就发现了一个熟悉的身影迎面而来,开始和她攀谈起来。隐约听到“我演技还行吧",“没把你打疼吧",“我还踹了她车子一脚呢',“这回她该受到教训了”之类的话。我马上就想起来了,原来是那个那天找你麻烦的人,原来这一切都是心月布的局。”听到这里时,流流脚下的石头停顿了一下,仅仅一下,便又“欢快”地扭动起来。“听到这里,我就惶惶地转头准备回家,我觉得根本没有什么可
\newpage
谈的了。结果一回头差点撞到灵灵的脸,我差点惊叫出来,灵灵赶忙播住我的嘴,把我拉到后边的一棵树后。然后她告诉我,她知道心月肯定要找你麻烦的;自从你抢走了她的游泳冠军,学习成绩对她也造成了严重的威胁后,她就在设计怎么来打扰你,给你点颜色看看。那天她是故意叫灵灵先走的,然后在门口等
你……” 

流流眯着眼睛皮笑肉不笑地问:“我是不是很
蠢?” 

“你不要往心里去,算了,都过去了。早看清她的真面目不是也好?走啦,回家了。”天一在脸上
摞出一筐呆呆的笑,拉着流流继续前行。 

“前行,前方的路是真实的吗?会不会也像朋友一样,下一秒就塌得一塌糊涂,还没真实的敌人来
得壮烈?”流流在心里默默地问着自己。 

奶奶由于过度忧虑与操劳再次病倒了,流流踏进家门的时候整个夜空已经成了星星的天下,奶奶正
\newpage
趴在床上咳嗽,样子憔悴得就像下霜打蔫了的茄子,干瘪地窝在那。流流心疼地想,这回的病似乎更严重
了,该怎么办啊? 

我们的主人公流流在以后的日子多了一副眼镜,黑白配色的镜框,宽宽方方的,架在她的鼻梁上,隐住了她黑色的忧郁的眸子。“嗯?你怎么戴上眼镜了呢?昨天还没戴啊?”课间,灵灵和老猫围在流流身边打听着,其他同学之所以不过来是觉得流流变坏了,又和流氓打架,又交男朋友,有意地避而远之。


“近视。”流流无心答复。 

她似乎一夜之间变了一个人,一颗闪亮的星黯淡了下来;除了天一,她几乎不和任何人说话,每天放学后救火般地奔回家中,守在奶奶身旁,整夜整夜
地熬着,犹如一盏不灭的明灯,燃啊燃啊燃啊。 


但这并未感动上苍,奶奶的病再度恶化。 

在一个下了露水的清晨,救护车惊醒了沉睡的
\newpage
小城,奶奶被放在监护室观察。当天一拥着一大束鲜花走到流流面前时,流流已经一天一夜没有合眼了,泪腺早已干涸,头发杂乱地披散下来,她还穿着那件与她瘦小的身材很不相符的,可以拿来当戏服的校服。她的脸和手像粘了胶水一样粘在玻璃上,她正在目不转睛地注视着病房里奶奶的一举一动,但奶奶只是
就那么躺着,连眼皮也不曾抬过一下。 

天一看到她那样痛苦,心痛得几乎无法呼吸。

他缓缓地把花放在地上,就静静地站在流流的
身后。 

不知过了多久,流流挪动了一下干涸得发皱的嘴唇:“你来了。”天一不知道该说些什么了,只是
用自己宽厚的手掌扶着流流单薄的肩。 

又过了些许时候,他开口了,“这个礼拜日有难得一见的狮子座流星雨,我们去为奶奶许愿吧,就在大桥上,她肯定会好起来的,星星一定会答应。”

\newpage

天一记得流流说过她特别想看流星雨,因为她
有一大堆的愿望要对星星说。 

“星星肯定会答应的。”流流自言自语地重复
着。 

等天一再次出现在医院的时候,流流已经一动不动地坐在那三天两夜了。“流流。”天一看到她已
经憔悴得不成模样时,忍不住叫道。 

流流甚至还来不及回头就因体力不支重重地摔
在了地上。 

模糊地分清天与地后,流流一下子就坐了起来,抓住正在给她输液的护士间:“我奶奶怎么样了?”由于太过突然,把那实习的小护士吓了一跳,几秘钟后才缓过来。“噢,还没有醒过来呢,那个男孩一直守在病人窗外,你还很虚弱,先躺下吧。”听到天一在守护奶奶,流流就放心地躺下了,她实在是太累
了,几天米水未进,现在手上还在输生理盐水呢。 

\newpage

知道流流已经无碍后,天一长长地舒了一口气,虽然他非常担心流流,很想看看她,但他知道,只
有他守护奶奶,流流才能放心地休息。 

天一个箭步冲进来时,李婶正在给流流热粥。

“流流,流流,奶奶她,你奶奶她醒了!”流流几乎是从床上蹦起来的,也来不及拔掉输液管,冲出病房时硬生生地将输液管挣开了。可当她跑到监护室门口时,主治医生却拦住了她的路。“病人病情很不稳定,随时有生命危险,你最好不要打扰她。”医生将生硬的话灌进流流的脑子里。“可是,我奶奶她醒了,她在看我呢,她需要我……”在一番苦苦哀求
之后,医生最终同意了流流进和病房。 

踏进病房的一刹那,血与泪同时滴到了地板上,血是从她刚才挣开输液管时受伤的针孔处涌出的。

流流坐在床边,紧紧地握着奶奶的手,嘴里不停地叫着奶奶。奶奶一直注视着她,费力地吐出几个字来:“流流,奶奶,要,走了……我,放心不下你
\newpage

啊。”一滴泪顺着奶奶的眼角向下游走。 

流流顿时泪如雨下“不,奶奶,我不让你走,我舍不得你走,你还要督促我写作业呢,没有你,我
根本不能活下去啊。” 

“你,要,学会,学会照顾,自己。你,还要
考,考大学。答应奶奶,行吗?” 


“嗯,嗯,我答应,我考大学,我答应。” 

“你知道我,我为什么不让,不让你游泳吗?你爸爸……刘彻,就是为了救一个……溺水的孩子……淹死的,淹死的。我不想你……不像你,重蹈他的
覆辙啊!” 

奶奶用尽最后一口力气,不舍地走了,尽管她
的一辈子活得艰辛不已。 

葬礼的整个过程,流流一滴眼泪也没流。接下来的几天,她只是在小屋子里呆呆地坐着,不眠不休
\newpage

地坐着,没有眼泪,没有表情。 

奶奶的走似乎控空了她的心,使得她没有了任何的反应,她彻彻底底地成了孤儿。天一也像从人间
蒸发了一样,没人见过他。 

星期天的下午,流流终于想要去看水了,她脚踏浮云般地站在桥上,看着结了一层薄薄的冰的水面
,想起了奶奶曾经说过,她看水是个臭毛病。 

奶奶啊奶奶,你现在是否正与爸爸在天上数星星呢?星星?她突然想到今天晚上要和天一起看流星雨的约定。流流用手抬了抬鼻梁上的眼镜,感到了初
冬的寒冷。 

大桥另一头,一对母女正手挽手悠闲地朝流流
走来。 

近了近了,流流的整个人仿佛被寒流冻了起来,是心月和那个女人流流的妈妈。心月似乎也发现了流流,她慌张地搜着那个女人转过身,想往回走,结
\newpage
果由于太过紧张,一不小心失足从断裂的栏杆处啪地掉到了冰河里。只留下那女人的尖叫声在空气中回旋,穿透了无数人的耳膜。由于河水太冷,也太过突然,心月竟在河里挣扎起来,慢慢地向下沉去。流流以百米冲刺的速度奔了过来,停在了那女人面前。女人早已被吓得面色惨白手足无措,但她还没到糊涂的程度,她一见流流,就惊讶地喊道:“流流!”流流只是轻蔑地暼了她一眼,便麻利地摘掉眼镜,扑通一声也跳了下去。她像凶猛的大白鲨一样,急速驶向她的猎物,亦像灵巧的海豚,动作柔美。她像父亲一样,扎进了水里,岸上的人都暗自祈祷着,希望她们能安全归来,在这个队伍里也出现了一张熟悉的面孔,天一,他几乎无法再次面对这样的场景,心脏早已经在超负荷工作了,目光一刻不离地在湖面上搜寻。有人
报了110。 

30秒……1分钟……1分18秒,出现了!流流用肩抵着心月的头出现在人们的视野之中。她甩着有力的右臂,切割着薄薄的冰层,拖着昏迷的心月
向岸边靠近着。 

\newpage

终于靠岸了,人们七手八脚地将心月拽了上去,然后是流流。就在这一刻,流流挣开了那些看似温暖的手掌,静静地,静静地,向河水中间游去。人们再次呆住!“流流!”天一意识到流流将要干什么,他歇斯底里地喊着。流流缓缓地地转过头来,微笑就
挂在嘴角,她给了天一个最最美丽的微笑…… 

警察到的时候,一切都已经结束了。震惊的人们渐渐散去,心月被送去了医院,那个女人站在桥上嘤嘤地哭泣着,警方开始派人在河里打捞尸体,警车
在一旁呼啸着。 

天一沿着桥走着,与流流的每个瞬间在脑海里内过。在断了的栏杆旁,天一捡到了流流的眼镜,这
副流流戴了不到一个月的方框眼镜。 

天一想感受流流的气息,就把眼睛戴了起来。

原来,原来这竟是一副没有度数的眼镜。天一
这才明白,流流的内心是多么的痛苦。 

\newpage

天空中竟洋洋洒洒地飘起雪来。一颗流星疾驰而过,光芒在瞬间绽放。天一蹲在桥上,双手扶着栏杆,对着被雪装点得闪闪发亮的河面喊道:“流流,
你看到流星了吗?” 



\end{document}
