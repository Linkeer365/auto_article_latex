\documentclass{article}
\usepackage[utf8]{inputenc}
\usepackage{ctex}

\title{替死鬼05号\footnote{Click to View:\url{https://web.archive.org/web/20220317093033/https://linkeer365.github.io/Linkeer365ColorfulLife2/3680586287/}}}
\author{卡塔兰}
\date{2020-12-12}

% \setCJKmainfont[BoldFont = Noto Sans CJK SC]{Noto Serif CJK SC}
% \setCJKsansfont{Noto Sans CJK SC}
% \setCJKfamilyfont{zhsong}{Noto Serif CJK SC}
% \setCJKfamilyfont{zhhei}{Noto Sans CJK SC}
% \setlength\parindent{0pt}

\begin{document}
\CJKfamily{zhkai}

\maketitle


\Large


(1) 

我乘坐电梯来到顶楼的目的,是想在崭新的视角下饱览校园景色(要是有三两伙伴同行,此时便是三傻大闹宝莱坞的桥段了,可惜拉加到最后还是蒙着眼,体会不到逐步进入美景的阶段感与层次感),这
是一种疏怀胸臆的新办法,值得尝试 

(说实在不知道是不是因为本命年,近期我的生活在多舛这一点上似乎达到了局部的最值,从学业到亲情,各个领域都向我尽情地展示他们能糟糕到怎样的地步,或者说,它们不断地以我为实例探索着跌
落谷底的深度上限…) 

现在说的话又变得好生奇怪了,不过不管怎么
\newpage
说,还是要变得勇敢坚强,从小事做起从一点一滴做
起培养仪式感以阳光心态面对新生活新挑战 

——尽管天色将晚,而出于鸟瞰的出行,似乎也能牵强地算作新的尝试之一,因此是值得的,是好
的 

——抱着这样的想法的我,到达顶楼推开门把
手,与胭红色的火烧云撞了个满怀 


(2) 

(就冲着这副景象,已经算作值回票价了吧)
 

风是偶或而起,像极了陪同在发小身边的小心
翼翼的同伴 

我数着步子走向栏杆,突然发现在火烧云与楼顶平台的接壤之处,似乎并不仅有那一道颇具古色的围栏,似乎还有一个人影,在渐暗的天色中时隐时现
\newpage

—— 


(3) 

由于高度近视加色盲,夜色里我的视力将失去眼镜能给我的所有加持,于是我调亮手电筒,看到幼小的踮起脚尖的人影——她似乎使劲地用力向外探去
,挪动的身体重心 

我的心骤然提起来,这丫头该不会是…脑海里迅速闪过各种各样的可能性…好在她身高似乎并不足
以越过那道矮矮的围墙 

我捏着脚步向她踱过去,她似乎察觉到了什么似的,向我的右后方平移,我伸出手去,扑了个空—
—情况似乎不对了 

我转身飞奔起来,她一言不发速度却极快,于是场面瞬间变成了一场追逐战——然而以前长跑的底
子可不是盖的,我一个箭步将她猛扑在地上! 

\newpage


(痛,痛) 

女孩子的身体冰冷而坚硬,时不时伴随着一阵
颤抖,我看不清她的脸,只能感受到一阵阵的抽搐 


(不要这样,事情总会好起来的) 

(其实叔叔我呀,也不是没有这么绝望的时候
的,比方说…) 

(比方说现在吗)几个字差点从我口中蹦出,然而这样的程度似乎还可以接受,不知道情感上有没
有变动的给药剂量带来不同感受的说法了… 


(4) 


女孩子仍在嘀嘀地喃着——大概痛的是她的心吧,这么小的姑娘却有这么冰冷而坚硬的躯体,该是有多么沉重的负荷才把她变成这样的?命运的毒手为

\newpage
什么总是如此地信守约言、无远弗届? 

(别怕啊,我啊是能承载好多好多痛苦的花车,不过目的地是在云山的深处,那里的小精灵专门以人的痛苦为食,等他们饱餐一顿后,你就永远得到解
放啦…) 

(我也是一个再平凡不过的存在,就算承包了你的痛苦也只是短暂的宿缘,所以小姑娘啊不要想太多,永远永远都会有新的花车化成各种各样的身体与人形,他们会带走你的痛苦,变成云端深处的一缕炊
烟…) 

(你看看呀,小姑娘,火烧云就算再晚也不会消灭,那就是精灵们开大餐的时段呀,他们用的炉火是五颜六色的,所以云朵就会是各种颜色的—当然要考虑光的偏折,但炉火的五彩总于某种色彩唯一地对
应呀hhh) 


(5) 


\newpage

(青骢结驷,悬火延起!) 

我惊异地发现我的吐字里充满了哭腔,真可笑啊,明明是想要救人一命的人,却自己先陷在悲伤的泥沼里了——我转过身,将女孩子自胯至颈地环抱而
起,怀里的躯体像一头瘦长的羊—— 


我缓缓地退出大门,按下电梯的下行键 


(6) 

我醒来的时候发现自己在医院里,加护病房的
来苏水味已经散去了不少 

(没想到你这家伙,内心里倒是风流得要死)


这熟悉的口气让我定了定神, 

(妈的第一眼怎么就是你这家伙,我怎么在这
里,太多的话请分点概述) 


\newpage

他突然之间烦躁起来,欲言又止 


(磨叽啥呀,别告诉这波我你想趁机表白) 


他突然抬手,在我头上结结实实来了记栗爆 

(有什么话别窝在心里,跟兄弟好好说,成吗?这些年满打满算咱也有2-3年的同居之谊了,有什么不开心的跟兄弟唠唠,不丢人啊,人生在世谁没
有难过的时候呢…) 


他咳几声,用手指揩揩眼角 

(不我是想救一位姑娘的,只是不知道为什么
…) 

他突然猛地一抽,我循着他手指的指向望去—
— 

一个幼小的身影静静地蜷缩在角落的深处,脚尖以相同的频率做着起跳运动,一切都显得如此的自

\newpage
然——除了在她背后安装着的内嵌的电池槽口 


(7) 

(那是【替死鬼05号】,会吸引任何一个走上天台的人的注意力——别跟我扯你只是去哪里看风景,学校内有完善安全保障的鸟瞰选址就有足足7处
,我还带你去过你不可能不知道) 

(只要走上了那处天台,就绝不是看风景这么简单了,就绝对是拿看风景当作借口要干点不理智的
事这样的了) 

他激动起来,突然恶狠狠地咬向自己的手腕—

(我真的好想怪你,骂你,可我现在才发现做不到,我真的以为那些和你喝酒聊天痛骂计组老师的日子能让你开心点,但其实这些都不是你啊,你的痛
苦太深了,以至于你已经根本察觉不出来了) 

(没有痛觉的人,身上的流血最后会流干的吧

\newpage
) 

他再次大声地咳嗽起来,一滴水滴在我的额头


(8) 

住院期间的我同时诊断出数十种精神症状,它们在我的身上、脑海中、灵魂里诗意地栖居了不知多久,随着治疗逐步推进我成了天才演员,在一刻钟内
高频地切换出喜怒哀乐四种状态 

我终于察觉到我的痛苦了,像金阁寺的大火,
又如三界不安之火宅 


(9) 

(原来我才是蒙着眼的那个人,伙伴也有男有女,看来三傻大闹宝莱坞的桥段最终还是成立了呀)
… 

(我一直以为我是花车,然而花车也是需要清扫的,花车自己也要把独属的烦恼交给小精灵才行)
\newpage


(所以我的烦恼就交给你好了,小姑娘,谢谢
你救了我…) 

充满电的(替死鬼05号)趴在窗棂旁踮起脚
尖,远处的火烧云在夜色里依稀可见 


(10) 


额外设定说明: 

其一、(替死鬼05号)是多功能机器人,具备有限的动作与个别的语言能力,同时还具有远景扫描、异地警报、录音等功能;该机器人面部设计优秀
,就算在白天也是使人认不出来的端丽容姿 

其二、(花)指的是痛苦与遗憾的经历,类似于(梅花便落满了南山)与(没有悲伤,也没有花朵
)的暗示,即悲伤就是花朵,花朵就是悲伤 

其三、(我)是体验过鸟瞰风景的,而(我)
\newpage
之所以会知道,一来是因为(兄弟)的普及与带路,二来这里也可以理解为一种(预切创),譬如决意割腕的人并不总能痛快地来上一记,而是先试探性地割出小创口;这些景点的逐一拜访,可以视作(跳楼踩)

\end{document}
