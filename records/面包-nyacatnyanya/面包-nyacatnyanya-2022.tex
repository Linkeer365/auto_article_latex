\documentclass{article}
\usepackage[utf8]{inputenc}
\usepackage{ctex}

\title{面包\footnote{Click to View:\url{https://web.archive.org/web/20230507144330/https://paste.ubuntu.com/p/jjhd6dv7Nr/}}}
\author{nyacatnyanya}
\date{2022-10-28}

% \setCJKmainfont[BoldFont = Noto Sans CJK SC]{Noto Serif CJK SC}
% \setCJKsansfont{Noto Sans CJK SC}
% \setCJKfamilyfont{zhsong}{Noto Serif CJK SC}
% \setCJKfamilyfont{zhhei}{Noto Sans CJK SC}
% \setlength\parindent{0pt}

\begin{document}
\CJKfamily{zhkai}

\maketitle


\Large

今天去洗漱的时候听到旁边的两个女生聊天
。 

说起的是好利来的那个海格蛋糕。一个女生说个头太小,另外一个女生说挺符合好利来的价格的,说好利来本身也不贵。那个女生就说她想去看电脑,但是看了就想买。之后她们就在聊“你很有钱哦?”“不告诉你w”“有多少哇,两万?三万?”之类的
,再之后我就走了。 

我想起来一些无端的事。好利来的蛋糕一直都很精致,漂漂亮亮地摆在精致的橱窗里,我常常从橱
窗前路过。但我想起的是另外的事。 

那时候我刚上大学,和舍友一起去做调研,入
\newpage
秋天黑了就冷下来,我们两个往回走,路过一家叫原麦山丘的面包店。站在门口闻到焦香的面包味混着奶
油的香气,一股股往鼻子里钻。 

我俩站在门口停了一会,说了些“好温暖啊”“灯光好明亮”“好香啊”之类的话,但最终谁都没
有提出进去买一个。 

那时候我们去景区调研,午饭跑到景区边缘买一个八块钱的煎饼果子,告诉老板分两个袋子装一人
一半。面包大概是二三十块一个。 

后来开始拿奖助学金,一部分拿来买药,一部
分存着治这治不好的病。 

我已经极少想起来当时的舍友,当时做过的那
些调研和作业的内容。 

但我偶尔会想起来当时的我,还会对橱窗内的温暖和香气有所期待的我。谈不上怀念或者感伤,只不过是像如今这样从橱窗前匆匆路过的时候会想起来
\newpage
记忆如同落叶,我行走在片叶不沾身的路上。

\end{document}
