\documentclass{article}
\usepackage[utf8]{inputenc}
\usepackage{ctex}

\title{剪刀石头布\footnote{Click to View:\url{https://web.archive.org/web/20221012134418/https://rentry.co/mgizh}}}
\author{舒辉波}
\date{2008-04}

% \setCJKmainfont[BoldFont = Noto Sans CJK SC]{Noto Serif CJK SC}
% \setCJKsansfont{Noto Sans CJK SC}
% \setCJKfamilyfont{zhsong}{Noto Serif CJK SC}
% \setCJKfamilyfont{zhhei}{Noto Sans CJK SC}
% \setlength\parindent{0pt}

\begin{document}
\CJKfamily{zhkai}

\maketitle


\Large

下午要录像了,上午我们在演播厅彩排。在下午的节目中我们有一个环节,需要六个孩子分成两组玩一个游戏,他们用“黑白板”的方式很快就分成了两组,所谓黑白板就是小朋友们有的出手心有的出手背,出手心的叫白板,出手背的叫黑板,出手相同的为一组。我想这样的分组方法好多小朋友们都玩过,我们小时候跳房子或者斗鸡分组的时候也采用这个方法。我以为他们这样分组就算可以了,谁知道他们又嚷着要分个先后顺序,于是每个组派出了一个代表,然后剪刀、石头、布,于是就分出了先后。孩子们兴奋地闹着猜拳的时候,我忽然心中一动,想起了奶奶讲给我的那个故事,关于剪刀、石头和布的故事。那个时候我并不知道剪刀、石头、布是孩子们用来猜
拳的一种游戏。 

\newpage

我把节目的彩排交给了现场导演,一个人坐在空落落的观众席上,在明明灭灭的灯光下重温着奶奶
的故事。 


剪 刀 

剪刀在1938年的鄂北板桥店镇是一只鸡的
名字。 

那个时候奶奶有多大呢?大概是20岁的样子
,因为奶奶说她18岁就出嫁了。 

那年,奶奶的母亲生了一场重病,于是奶奶就抱着剪刀到了太奶奶家,奶奶想把这只名叫剪刀的鸡
送给太奶奶补补身子。 

要知道,奶奶做出这样一个决定是非常不容易的。因为,奶奶和剪刀有着非常深厚的感情。是的,从某种程度上说,这只名叫剪刀的鸡把奶奶当成了它
的妈妈。 

\newpage

剪刀不知道是谁家的,也不知道是从哪儿来的,总之有一天奶奶从菜园子里砍了三根莴苣回来的时候,在她的身后就跟着这样一只黄茸茸的小鸡,小绒球一样滚动着的小鸡张皇地啾啾叫着,在奶奶的身后
,紧跟不舍。 

于是,奶奶就把新鲜的莴苣叶切碎了拌上芝麻喂它,小鸡吃完了之后,把它嫩黄的小嘴在地上左右擦了擦,然后很满足地伸了伸脖子,斜靠在奶奶的裤
脚边,微闭着双眼,不再张皇地啾啾鸣叫。 

就这样,这样一个小绒球就在奶奶的身后滚着滚着,直到身上嫩黄的绒毛换成了白色,再换成了黄色,最后,身为母鸡的它竟然和公鸡一样,在尾巴上支棱着四根红黑的羽毛,交错而生,就像背上扛着一柄漂亮的剪刀,于是,它有了属于自己的名字—剪刀
。 

奶奶抱着剪刀的时候,剪刀的喉间发出一种咕咕的声音,奶奶知道,这是一种信任和满足 ;甚至,当奶奶把雪亮的菜刀架在剪刀脖子上的时候,它也
\newpage

是如此的平和与温顺,它的喉间轻轻地咕咕着。 

于是,奶奶的眼泪就吧嗒吧嗒地落在雪亮的菜刀上,落在剪刀油亮光滑的羽毛上。剪刀睁开眼睛看
着奶奶,杏黄色的眼珠子圆溜溜的。 

剪刀的眼神里没有一丝恐惧与不安,它很平静地把脖子搁在奶奶雪亮的刀口上,平静得就像枕在奶
奶的脚上一样。 

这个时候,大病初愈的太奶奶说话了,她说,
留着它吧,是个伴。 

于是,剪刀就留在了太奶奶的小院里。太奶奶的小院里有棵石榴树,鲜红的石榴花把柔软的枝条都压弯了,剪刀在小院里咯咯地走着,很快,它就接受了这样一个新的环境。或者说,剪刀似乎从来都以为自己是生活在这个小院里,现在只是从奶奶那里回来
了。 


\newpage

太奶奶问奶奶,他有消息了吗? 

他就是奶奶新婚的丈夫,我的爷爷。奶奶结婚不久爷爷就参加了红军游击队,然后就再也没有爷爷
的任何消息了。 


奶奶伤感地摇了摇头。 


太奶奶说,你回去吧。 

奶奶在民国25年她18岁的时候出嫁的,那年是1936年,过了一年,1937年的上半年,太爷爷去世了。到了1937年年底,日本人攻陷了南京。当时奶奶的哥哥在南京国民政府任军职,南京国民政府陷落之后,太奶奶就不知道她唯一的儿子的任何消息了。于是,太奶奶就盼着过年。1937年终于过去了,太奶奶还是没有在春节的时候把她的儿
子盼回来,于是,太奶奶就病倒了。 

在剪刀的陪伴下,太奶奶一点一点地恢复了。剪刀长得更漂亮了,绝对是母鸡之中的美女,而且它特别肯下蛋,剪刀还有剪刀下的蛋,成了太奶奶生活
\newpage

的全部。 

太奶奶在那兵荒马乱的年代守着一个空落落的小院和一只不会讲话的剪刀,守着她那所剩无几的光
阴,她是个孤独的人。 


石 头 

石头的妈妈是一只黑白相间的非常美丽的中国
土狗,这也注定了,石头也是一只狗。 

有一天,石头的妈妈不知道为什么被人追打,当它躲进太奶奶的小院的时候,它的一条腿已经被打断了,嘴角正流着血。能够躲到太奶奶的小院里,并且在墙角躺下,这似乎已经用完了它所有的力量,当太奶奶出现在它面前的时候,虽然它的目光里仍然充
满了恐惧,但它已经没有力气再站起来奔跑了。 

太奶奶转身端来了一瓢水。它看了太奶奶一眼,在太奶奶温暖目光的注视下,它的目光里渐渐消散了那些恐惧。它紧张着的身体彻底放松了,它放心地
\newpage

躺了下来,把头搁在地上,闭上了美丽的眼睛。 

太奶奶发现,它的尾巴下面正在淌血,这是一
只快要分娩的母狗。 

它伸长着粉紫色的舌头,低声却很忧伤地呜咽着,很快,一只小狗从它的尾巴下面滑了出来,接着
又是一只,然后还有一只。 

第三只小狗出生之后,它很想回过头来望一眼它的孩子们,它甚至还伸了伸舌头,想舔一舔它的那些孩子,但是,它已经没有一丝气力了,分娩让它的生命彻底的细若游丝,它嘴角的血迹已经干枯成褐色。它积蓄了很久的力量,才勉强伸了伸舌头,舔了几下瓢里的水,望着太奶奶摇了两下尾巴,然后合上眼
睛,泪水打湿了它脖颈下面的泥土。 


它再也没有睁开眼睛。 

太奶奶和剪刀一起望着这只美丽的母狗在死亡之前生下自己的孩子,太奶奶和剪刀都没有发出任何
\newpage
声音。太奶奶抹了抹眼泪,把闭着眼睛正四处寻找奶
头的三只小狗一只一只地抱了起来,走进屋里。 


剪刀跟在太奶奶的身后,也默默地进了屋。 

后来,那三只小狗只剩下一只还有呼吸,但它一动不动,也不发出任何声音,就像石头一样,于是
,它就有了自己的名字——石头。 

太奶奶在离石榴树不远的地方挖着土,她需要挖一个很大的坑,才能把石头的妈妈以及它的兄弟或姐妹埋进去。当太奶奶沉默地挖着的时候,剪刀一会儿歪着头看太奶奶,一会儿歪着头看躺在地上只有呼
吸而一动不动的石头。 

当太奶奶挖累了,坐在椅子上喝水的时候,剪刀跳进了土坑里,用它的爪子用力地扒拉着,不知道
它是在寻找蚯蚓和虫子还是想帮太奶奶挖土。 

于是,太奶奶就捧着茶碗望着剪刀笑了。剪刀不知道是因为找到了它喜欢的虫子还是因为看见了太
\newpage
奶奶的笑容,它也欢快地咯咯地唱着,唱着唱着剪刀
就拍打着翅膀跳了上来,它去下蛋了。 

后来,石头能够活下来就是因为剪刀下的蛋。太奶奶把剪刀下的蛋和水一起调了蒸成黄澄澄的鸡蛋
羹,然后再一口一口地喂给石头吃。 

就这样,石头渐渐地睁开了眼睛,它漆黑的眼睛也渐渐有了光亮,它粗短的尾巴可以左右地摇晃来表达它快乐的心情,并且它还可以摇摇晃晃地走到剪刀的跟前,瞪着圆圆的黑狗眼,巴巴地仰望着身形俊朗的剪刀。剪刀也歪着头瞪着它美丽的杏黄眼望着这块肥厚得可以摇摆移动的石头,它们对望了好久……石头大概是脖子扬得有点酸了,就摇了摇尾巴,走近一步,蹭着剪刀丰美的羽毛。剪刀的喉间发出了满意
的咕咕声,那声音里充盈着无限的温柔。 

吃着剪刀营养丰富的鸡蛋,石头生长得很快,体型很快超过了剪刀。尽管如此,石头还是喜欢很依恋地蹭着剪刀的腿和脖子,或者绕着石榴树追逐着快乐地咯咯唱歌的剪刀。捧着茶碗的太奶奶忘记了喝茶
\newpage
,张着嘴露出她剩下不多的几颗牙齿呵呵地笑着。也许是受了这笑声的鼓舞,剪刀和石头就都拥了过来,在太奶奶的膝下蹭着,剪刀会轻轻地啄太奶奶的小脚穿着的千层底,而石头则撕咬着太奶奶的裤脚,不停地摆着头呜呜地扯着。太奶奶说你个狗东西,口水把我的裤脚都打湿啦,滚,滚,都给我滚开。于是,剪
刀和石头就相互追逐着跑开了…… 

1938年5月,徐州沦陷,骄狂的日本侵略军将矛头指向了当时的抗战中心—武汉,企图在武汉发起最后一击,实现其三个月灭亡中国的图谋。6月,妄想速战速决的日军调集了14个师团、3个旅团,还有机械化兵团、航空兵团和舰艇120艘,总兵力近40万人,在华中派遣军司令畑俊六的指挥下,分两路沿长江两岸和大别山北麓扑向武汉。1938年10月,各路日军突破了国民党军的外围防线,逼
近武汉。 

10月25日,武汉三镇失守,至此武汉大会战结束。举世瞩目的武汉大会战,历时四个半月,是中国抗战史上中日双方出动兵力最多、规模最大的一
\newpage
次战役。此次会战,中国守军共歼日军10万余人,有力地消耗了日军的有生力量。从此,中国抗日战争
进入相持阶段。 

这段历史,是后来我读书之后了解到的,当时生活在板桥店镇的奶奶和太奶奶们显然并没有如此宏观地知道当时中国的格局,她们只是知道日本人来了,大概日本人觉得小镇太小所以又走了。日本人来的时候,板桥店镇的保长自杀了,但副保长郭麻子则归顺了日本人,成了日本人的伪乡长,负责协助日本人管理小镇。日本人帮助伪乡长的一群乌合之众武装了简单的枪支,从此,郭麻子有了自己心爱的盒子炮和一只正宗的日本军犬。当然这只是我根据历史的一些推测,奶奶只告诉我小镇的街头上,经常可以看见伪乡长郭麻子腰挎盒子炮,手牵日本狗,头望青天,耀武扬威地对着每个人都颐指气使。小镇上所有的人都
敢怒而不敢言,顶多只能“道路以目”。 

日本人的铁蹄过处,给小镇上幸存的每一位居民的心中都留下了创伤,然而逃难归来重新收拾残破家园的小镇人甚至连日本人都没有见过,而那些不幸
\newpage
遭遇过日本人的乡亲们,则几乎无一幸存。所以,小镇上,人们仇恨的目光都望向了日本人的走狗—郭麻
子和那条日本狗。 

石头的妈妈就是因为小镇居民们的家仇国恨而死掉的,石头的妈妈居然怀上了日本狗的种。也就是
说,石头的爸爸就是那只日本军犬。 

我想,当时石头妈妈的主人一定是含着泪和其他人一起追杀着自己心爱的狗,这只美丽的中国狗怎么可以和践踏华夏残杀中国人民的日本狗发生爱情呢
?怎么可以? 

当时,听奶奶讲到这里,我内心的情感非常复杂,以至于对石头,我到底是该爱还是该恨,我自己
都有些说不清楚。 

太奶奶没有把家仇国恨和这样一只名叫石头的小狗联系起来,她只是觉得在这兵荒马乱的年代,在这日薄西山的晚年,在这个安静的小院里有着这样两只生灵的陪伴,她觉得生活里有了许多生气,并且对
\newpage

这鸡飞狗跳的俗世生活生出了几多眷念。 


布 

在一个夜黑风高的寒夜,板桥店镇的青石街上响起了一串清脆激越的马蹄声,在太奶奶小院的门前,一个身材魁梧一身戎装的军人翻身跃马而下,高筒军靴咔嚓咔嚓地打破小镇静寂的暗夜。咯吱一声门响,太奶奶身披棉袄,一手端着在风中闪烁不定的煤油灯,一手抹着眼角滑落的浊泪……只听见高筒皮靴咔嗒一声碰撞,暗夜中的军人一个标准的军礼之后,上前一步,双手抱紧了太奶奶,哽咽地喊道 :“妈妈
……” 

尽管奶奶在给我讲述这个故事的时候并没有这样的细节,但只要我一闭上眼睛,就仿佛看到了这样的画面,犹如电影,在我的脑海里一遍一遍地放映着,以至于直到现在我还相信,当时太奶奶唯一的儿子就是这样回家的,就是这样,见过他的母亲并永别了他的母亲,就这样,骑着战马告别母亲,壮士一去不

\newpage
复返。 

在祖国母亲遭受蹂躏的时候,总有许多儿女跃马扬鞭战死沙场,虽然我们并不知道他们的名字。如果没有奶奶给我讲剪刀、石头和布的故事,我就不知
道,奶奶还有个战死沙场的英雄哥哥。 

我想,当太奶奶知道了自己的儿子依然活着,肯定非常高兴,并且欣慰,这对于太奶奶来说绝对是个惊喜,因为,太奶奶原本以为,她的这个儿子在南
京陷落的时候就已经不在了。 

太奶奶的儿子走了之后,家里多了一只猫,叫
布。 

布通体漆黑,只是四只爪子是白色的,像是穿着美丽的白袜。奶奶说,我见过布,它浑身毛色油光水滑,应该叫缎子才对,怎么能叫布呢?太奶奶当时也很纳闷,说,哪有这样光滑漂亮的布啊?当时奶奶太奶奶们穿衣用的布都是自己纺的土布,自然不会有
多细腻光滑。 

\newpage

现在想来,也许布只是一种发音,布是它的英文名字,只是发音跟布接近,所以就叫作布。当然,
这只是我的推测。 

布的到来,让鸡飞狗跳的剪刀和石头也不由自主地暂时安静了下来。布既不像剪刀那样在土里翻腾,咯咯地唱歌,也不像石头那样蠢蠢地转着身子咬自己的尾巴。布披着它那华贵而美丽的一身黑长毛,在小院里安静地踱步,或者轻轻一跃,跳上屋顶,在有着瓦楞的屋脊上卧下,黑色的皮毛披着霞光显现出一种琉璃的光彩,一双幽蓝的眼睛,冷冷地像是在打量
着这个小院,又像是审视着这个陌生的小镇。 

布有着宽阔的躯干,粗短的四肢和尾巴,脖子和后背上有长长的鬣毛,白袜下面的爪子大而有力,结实强壮的布就像一头小狮子,给人以坚实而有力的
感觉。 


布不怒而自威。 

捧着茶碗的太奶奶越看越觉得布像她那威武有
\newpage
力的军人儿子,更何况布原本就是儿子的爱猫,所以
,太奶奶对布的爱就更深一层。 


可布还是走了。 

布回来的时候已经是料峭的深秋了。布瘦了一圈,油光水滑的皮毛皱皱巴巴的晦涩暗淡,好像是换了一只猫,唯一不变的是布那王者的神情和气质。当太奶奶把布揽在怀里,用她那龟裂的手掌抚摩着它的时候,布闭上了它美丽的蓝眼睛,发出了呜呜声音,如此安详,仿佛它终于把悬着的心放下了,这个家,
它认了。 

奶奶说布是去找她的哥哥了,但是没有找到,
于是它听从了主人的叮嘱,回到了太奶奶的小院。 

这个时候的石头已经壮实得像一头小牛犊,它冷冷地看着伏在太奶奶腿上的布,而剪刀则顺从地依
靠在石头的腿间,也无声地望着布。 

剪刀和石头的关系已经倒置了过来,从前石头
\newpage
像小孩依恋妈妈一样依恋着剪刀,但现在,长大成人的儿子则在前面保护着母亲,剪刀对石头充满了信赖

面对体积是自己几倍大的石头挑衅的目光,布该干什么干什么,一切都从容不迫,没有丝毫的扭捏作态。尽管如此瘦弱,布的身上依然有着凛然不可侵
犯的霸气,这反而让石头和剪刀有些无所适从。 

在太奶奶的精心调养下,布脱掉了晦暗的旧毛,很快就恢复了它原来光滑的皮毛。布对剪刀和石头尽管仍然保持着距离,但却不亢不卑地表达着自己的友善, 因为在布恢复调养的这个过程中,它吃了不少剪刀下的蛋,布似乎也心知肚明。它轻轻地靠近剪刀,用它毛茸茸的耳朵轻轻地蹭着剪刀,或者直立起来,举起前腿,用它那美丽的白袜,轻轻地挠挠石头肥厚的臀部。这个时候的剪刀和石头就很享受地闭了眼睛,接受着布善意的表达。尽管,布与剪刀和石头的关系没有更进一步,但也没有恶化,布与它们相敬
如宾,毕竟,它们现在是一家人。 

雪静静地下着,覆盖了田野,覆盖了小河,覆
\newpage
盖了板桥店镇的每一个角落。太奶奶坐在将熄未熄的火塘边笼着衣袖望着仍在飘落的白雪,布满褶皱的脸上密布着愁云。石头卧在太奶奶的脚边,剪刀缩着翅膀伏在石头的脖颈边,这个家庭里的每一个成员在这样一个雪天尽管仍然可以相互温暖,但却都饿着肚子


布已经有两天没有回来了。 

布回来的时候,这个家喧腾了起来,因为布在一个雪夜衔回了一只比它自己还要大的灰色的兔子。渐近年关的雪天,这个家因为布带回来的这只兔子而有了浓浓的年的味道。也因为这只兔子,石头完全信赖了布,而信赖石头的剪刀也理所当然地跟着石头一
起依顺着布。 

正在长身体的石头跟着布不仅学会了捉野兔,还学会了抓老鼠,除了上树之外,布的本领石头全部学会了。因为石头原本就是一只有着军犬血脉的狗,学会这些很容易。在那个兵荒马乱的年代,在板桥店镇,人们经常看到一只小狮子一样的猫后面,跟着一只小牛犊一样的狗和一只咯咯唱歌的母鸡。人们之所
\newpage
以没有大惊小怪,大概因为人们的注意力都放在战乱
和饥荒之上。 

1939年很快就过去了,1940年的春天
来了。 


这是一个萧索的春天,日本人的铁蹄凶蛮地践踏着中国的锦绣河山,花草树木、村庄田园、亿万子民,在战火硝烟里都失却了春天应有的容颜。除了日本人扶持的汪精卫伪中央政权和地方伪政权之外,所有的中国人都团结了起来,一起在自己的家门口抗击着日本人,华夏儿女以自己的鲜血守护并灌溉着每一
寸国土。 

在这样一个暮春时节的黄昏,几乎足不出户的太奶奶轻轻地捧起了剪刀,搂抱在自己温暖的怀里。剪刀安详地闭上了眼睛, 犹如是在一个迷人的梦境里,它享受着太奶奶久违了的温暖怀抱。太奶奶的身后立着铁塔似的石头,正值青春力壮的石头仿佛一头铜筋铁骨的老虎。灿烂开放着的火红的石榴花像是摇
\newpage
曳在风中的风铃,石榴树下正生机盎然地生长着许多瓜果蔬菜,绿的瓜秧、黄的小花、碧绿的小白菜,清清爽爽地生长着。它们的足下,是长眠于此的石头的妈妈和它的兄弟或姐妹们。布懒懒地躺在石榴树下,虽然微闭着双眼,但是它也感觉到了即将出门的太奶奶此行非同寻常。剪刀、石头还有布,它们都知道太奶奶的生活习惯,如果有一天太奶奶慎重地走出属于她自己的院门,那么肯定有不同寻常的事情要发生。
这对它们三个来说,都是心照不宣的事情。 

在太奶奶轻轻地扣上门环,咔嗒一声把那把古老的铜锁锁上的时候,布跃到了石头的背上,再借力一跃腾上了院墙,它在高高的院墙上轻轻地走着,并斜睇了太奶奶一眼,仿佛在告诉他们,它要留下看家


剪刀和石头就随着太奶奶来到了奶奶家。 


太奶奶抱着剪刀说,剪刀还记得这个家吗? 

剪刀仿佛是睡眼惺忪的样子,懵懵懂懂地从太奶奶的怀里探出了头,圆睁着它的斗鸡眼环顾了一下
\newpage

四周,然后咯咯地表达着自己愉快的心情。 

太奶奶也同着剪刀一起看了一下四周,然后问
奶奶,他呢? 

奶奶也环顾了一下四周,然后小声地说,在地
窖里。 


太奶奶说,活着就好。 

奶奶就抹眼泪,说,他这是负了伤跑回来的,
部队里没有粮了…… 


八尺高的汉子还不到九十斤…… 

太奶奶不说什么了,她轻轻地用自己的手掌捋
着剪刀油亮的脖子。 

去年秋冬时节,剪刀又换了一身新毛,现在除了尾巴上仍然支棱着红黑相间的四根羽毛之外,在脖子上也长了一圈红黑白相间的羽毛,像是戴着一个色
\newpage

泽亮丽的三色针织围脖。 


太奶奶叹息道,多美的大姑娘啊。 


奶奶望着剪刀,眼泪刷刷地滑落。 

太奶奶坚毅而果断地站起了身来,她从厨房里
抽出菜刀,然后在水缸的边沿一下又一下地磨着。 

原本伏在地上的石头忽然就“腾”地站了起来,它紧贴着太奶奶的腿蹭着,并轻轻地叼着太奶奶的裤脚扯着。这样一来,太奶奶站立不稳,险些让刀伤了自己的左臂。于是,太奶奶恼怒地用菜刀的刀背砍向了石头。石头低叫了一声,远远地望着太奶奶,黑
亮的眸子里满是哀求。 

只有剪刀似乎对周遭的一切一无所知,也许,
它早就将生死置之度外了。 

太奶奶在看到石头眼神的那一瞬间,一个激灵,手一软,提着的刀就哗啦一声,顺着缸沿掉了下来
\newpage

,黄昏,声音凄厉而刺耳。 

忽然,石头警觉地跃了起来,掉转头跑向院门
口,并低声呜呜着。 

“哎哟,这年月还有鸡吃啊?肯定是来了贵客对吗?”郭麻子在两个扛枪的喽啰的左拥右护下,左手牵着日本狼狗,右手按着腰间的盒子炮,螃蟹一样
摇了进来。 

奶奶和太奶奶都没有说话,斜阳把她们的身影
镀上悲壮的金色,仿佛两尊沉默的雕塑。 

石头全身的肌肉都紧张了起来,犹如箭在弦上

剪刀紧贴在石头的身边,泰然自若,仿佛一切
皆在预料之中,或者是对一切皆无所预料。 

“婶,”郭麻子笑着问太奶奶,“我表哥只怕
都当上了国民军的军长了吧?” 

\newpage


太奶奶说 :“我没你这个表侄儿。” 

“大表妹,你说妹夫做什么不可以,为什么去当什么新四军?现如今是皇军的天下,哪能跟皇军对
着干啊!” 


奶奶动了动嘴巴,没有说话。 

郭麻子脸色一变说 :“都乡里乡亲的,远近都是个亲戚。你到哪抗日都行,别到我的地盘上来。
如今我是吃皇军的饭,只好得罪了,搜!” 

一阵叮叮咚咚之后,奶奶的家基本上就被砸了
个稀烂。 

郭麻子一支烟还没有抽完,那两个腿子就过来
说,没找见。 

郭麻子一声冷笑,呸地吐掉烟,吐了口黑黄的
浓痰说,嘿嘿,老子有日本军犬…… 

\newpage

松开绳套之后那个军犬就腾上翻下,四处寻找人的气味。就在它循着气味向着地窖嗅过去的时候,
石头像尊铁塔似的挡住了它的去路。 


石头和它的爸爸第一次如此近距离地接触。 

尽管相对于板桥店镇的本地土狗来说,石头已经是庞然大物了,但和这只日本军犬比起来,石头还是显得有些单薄,它能比拼过这只经过严格训练的军
犬吗? 

那只军犬和石头相互环绕着对方转了一圈,然后一起向后退了两三步。很显然,这只军犬第一次觉
得自己遇上了一只与自己旗鼓相当的狗。 

那只军犬将自己粗壮的两只前爪在地上轻轻一按,肩胛和头都低低地放下,然后忽然一个鱼跃冲向石头,而此时石头刚好仰起上半身,没有任何实战经验的石头被这只军犬撞得腾空而起,摔在地上,发出
一声闷响。 

\newpage

“嘿嘿,有意思。”郭麻子坐在椅子上又点燃了一支烟,两个扛枪的喽啰分列左右,都一起看稀奇
一样看着这两只狗的较量。 

太奶奶和奶奶的心都提到了嗓门口,不敢大声出气,甚至都不敢咳嗽,怕把堵在嗓门口的那颗怦怦跳动的心给吐了出来。这样的紧张,既是担心石头,
更是担心地窖里藏身的爷爷被汉奸发现。 

只有剪刀围绕着两只战斗着的狗,微张着翅膀
,左右挪步。 

那只军犬虽然把石头扑倒在地,却并不恋战,
而是很快抢占有利地势准备下一次的进攻。 

在接下来的撕咬中,石头充分发挥了它灵巧的腾挪功夫,总是把那只军犬的迅疾进攻在跳跃中化解

郭麻子手里捏着的那支烟都烧了三分之二了,也没见他抽一口,他大概是看得太投入了,忘记了抽烟,那捏着烟的手就一直举着离嘴巴半尺远,嘴巴半
\newpage

张着,露出口里歪七竖八的那些黄黑的牙齿。 

他忽然说了一句 :“这只狗打架的动作怎么
这么像猫啊?” 

布教会了石头本领,但在和石头嬉戏中,它们从来都不相互攻击对方的要害部位,而且总是点到为
止,这害苦了石头。 

尽管在接下来的争斗中,石头占尽了上风,但石头却并没有怎么咬伤那只军犬,那只军犬虽然只有几次有效攻击,但效果却很明显,因为石头的肚腹和
脖子都在淌血。 

在对豺狼的战斗中,做一只善良的狗是不行的

在军犬的又一次冲刺中,石头高高地跃起,让那只军犬扑了个空,并重重地摔在了地上,与此同时石头紧紧地咬住了军犬的尾巴,军犬挣了几下没有挣
脱。 

\newpage

郭麻子的那支烟烧疼了他的手指,他扔下了只抽了一口的烟,然后骂了一句,哄笑着说:“真是它
妈的笨狗,竟然去咬人家的尾巴……” 

这不是石头一个人在院里玩着的咬尾巴的游戏
,这是一场你死我活的战斗。 

那只军犬拖着尾巴转了几圈之后忽然一声尖叫掉头跃起,它居然主动地扯断了自己的尾巴,与此同时,它紧紧地横向咬住了石头的咽喉,并用一双粗大
的爪子死死地抱住石头的头颅。 


血从石头的脖子里一点一点地流了出来。 


太奶奶背过了淌满眼泪的脸。 


奶奶尖叫一声,几近崩溃。 


郭麻子他们松了一口气。 

就在这个时候,一只大鸟飞身而起,落向僵持
\newpage
争斗着的两只狗,军犬一声惨叫呜咽着跑开,它的一
只眼珠挂在眼角,血顺眼角而下。 

那只大鸟是剪刀,在所有人都忘记了剪刀的存在的时候,剪刀像一只迅猛的鹰从高处跃下,用它那
坚硬的嘴巴啄伤了那只要置石头于死地的军犬。 

一声枪响,腾飞着的剪刀的羽毛雪片一样地飘落。剪刀没来得及再一次歌唱,就这样如同尘埃一样静静地落下,它的那些美丽的羽毛,在落日的余晖中
最后一次展现了它摄人魂魄的美丽。 


剪刀的死就像一朵花的飘落和凋零。 

“老子的枪法,指哪儿打哪儿……”郭麻子提着枪管还在冒烟的盒子炮边说边将枪管对准了太奶奶,“叭——”郭麻子的嘴巴叫了一声“叭——”之后
把枪管又慢慢地移向了奶奶,“叭——” 

“哈!哈!哈!”看见奶奶闭上了眼睛,郭麻子把枪管慢慢地移向了躺在地上的石头,这次他真的
\newpage

要扣动扳机了。 

忽然,小镇的另一头传来了两声清脆的枪响,
郭麻子收起了枪说了声,走,我们看看去! 

郭麻子左手牵着那只日本军犬,右手按在腰间的盒子炮上,如他来时一样,不同的是,他的一个喽
啰的枪刺上挑着血淋淋的剪刀,那 

只军犬凄惨地呜咽着,一扫来时的汹汹霸气。
 


石头伏在地上,眼泪和血一起 地滑落。 


眼泪和血一起沁湿了它伏身的土地。 


布和石头 

月光如水,在这个有着皎皎明月的夜晚,小镇
的万物都浸淫在这如水的月光下。 

\newpage

太奶奶孑然一身,立在自家院门的门口,她停
住了。 

院子里一片狼藉。碧绿的白菜、青绿的瓜秧、瓜秧上顶着的金黄花朵都如碎玉一般,零落成泥,甚至连吐芳绽艳的火红石榴花也被抽打得落了一地,点
点滴滴,犹如斑斑血泪,触目惊心。 

太奶奶的目光越过狼藉一片的院子,望向里屋,一扇门歪倒在一边,屋内的光线虽然有些暗淡,但
仍然可以看见,这个家已经被彻底地砸过。 

原本,国家都在风雨飘摇之中,大家尚且如此
,小家又如何能指望保全呢? 

月光下的太奶奶静静地走进了小院,在石榴树下,默默地用右手扒拉着泥土,很快,就有了一个小坑,然后,她伸展开左手,把那几根沾染了血迹的羽
毛抖落在土坑里,那是剪刀的美丽羽毛。 

太奶奶叹了口气,抬头望了望月亮,然而冷月
\newpage
无声。太奶奶知道,月亮目睹了一切,却永远不会开
口说话。 

太奶奶默默地捧起小坑旁边的泥土,让那些泥
土一点一点地覆盖了剪刀的那些羽毛。 

太奶奶有些颓废地坐在石榴树下,闭上眼睛,那只嗓音甜美喜欢咯咯唱歌的剪刀又出现了,有着美丽的三色围脖,有着剪刀一样美丽的红黑羽翎,喜欢优雅地踱步而且特别肯下蛋的剪刀,它永远地走了。

在掩埋了剪刀的羽毛之后,在月光下坐着的太奶奶忽然觉出了寒意,在她用双手抱紧自己臂膀的时候,心中一个激灵,一下子就想到了布。她焦急地喊
道,布——布—— 


然而四周一片冷寂。 

坐在地上,坐在月光里的太奶奶一下子就哭了,她像个无助的孩子一样,泪水顺着她褶皱的皮肤静静地淌着,温热的泪水,在滑进脖子的时候就已经冰
\newpage

凉…… 

不知道过了多久,沉浸在这如梦一样绵长忧伤中的太奶奶忽然感觉到有什么在舔着她的脚脖子,温
热的,一下,又一下。 

太奶奶睁开眼睛,就看见了布。太奶奶一把将布揽进了怀里,把淌满泪水的脸庞紧紧地贴在布的身上。布轻轻地叫了一声,这一声,虽然尖细,但是太
奶奶还是听出了它叫声的异常。 


果然,布的一条腿受了伤。 

太奶奶站了起来,从坐着的泥土上站了起来。她还有布,还有女儿,还有在战场上战斗的儿子,还有受了伤的新四军女婿和受了伤的石头,还有心头燃烧着的希望,只要有希望在,总能舔干血迹,直面惨
淡的人生。 

第二天,太奶奶的院子里出现了郭麻子的卫队

\newpage
,七八个人,都气急败坏地跟太奶奶要那只猫。 

这个卫队,正是昨天晚上以搜捕新四军爷爷为由而砸了太奶奶家的人。卫队的队长,郭麻子的侄子郭小川的脸上挂着非常恐怖的两道爪印,从眼睛下面越过鼻梁直至嘴角之上,爪子深入皮肉,看来,十分
俊俏的小伙子是被破了相。 

当时讲述这个故事的时候,奶奶叹了口气,说,人心难测啊,其实要说和郭小川我们都是一起玩大的伙伴,怎么日本人来了,他脸一抹就成了心狠手辣的卫队队长,把枪口指向了自己的乡亲,要知道,以
前做小镇老师的时候有好多姑娘都喜欢他啊…… 

我怎么也无法想象一个白面书生如何成了提着枪的日本人的走狗,但直到布在他的脸上留下了两道他永远无法抹掉的抓痕之后,才在自己的心中渐渐有了郭小川的形象,在我的想象中,仿佛有着狰狞的面
容才符合汉奸的形象。 

在郭麻子得知爷爷回到小镇的确切消息之后,立即兵分两路,由他带了两个喽啰去了奶奶家,而另
\newpage

派郭小川带领卫队去了太奶奶家。 

他以为爷爷十有八九是躲在太奶奶家,因为太奶奶的儿子是国民军将领的缘故,对于太奶奶,郭麻
子他们多少有些惧怕。 

就在郭小川指挥着大家大肆搜寻抢砸的时候,在院墙上冷眼观看着的布,突然无声跃落,并以迅雷不及掩耳之势把自己的利爪刺进了这个强盗的皮肉之中。在郭小川的惨叫声中不知道谁向着布连开两枪,布在浓重的暮色之中顺着院墙跃上屋脊,消失在小镇
众多的屋顶之上。 

太奶奶终于明白了原委,紧张地四顾张望,却
忽然不知道布去往了何处。 

就这样,仿佛布知晓郭小川每次不预期的搜捕,总之,郭小川没有一次在太奶奶家碰到过布。郭小川说,无论何时何地,只要我碰到那只猫,立即就地
正法! 

\newpage

太奶奶望着郭小川说,你爱杀谁杀谁,那是你
的事情。 

太奶奶院子的小菜地又生机盎然了,一切都可
以从头再来,只要希望还在。 

太奶奶坐在小院里,坐在石榴树下,坐在一片碧翠的菜地前,坐在黄昏垂暮的霞光里。她捧着茶碗
,若有所思,一动不动地消融在暮色之中。 

布在这段时间总是早出晚归,有时也有好几天
不回来,行踪漂浮的布总让太奶奶很担心。 

咯吱一声门响,太奶奶抬眼望去,虚掩着的院门被打开了,石头伸着脑袋向院子里张望着,当它看见太奶奶的那一瞬间,目光温柔明亮并且充满了温暖

太奶奶揉了揉眼睛说,啊,是石头,是我的石
头回来了! 

太奶奶正准备站起身来的时候,石头进了院子
\newpage

,并回过头去摆了摆尾巴,然后布也走进了院门。 

原本就要站起来的太奶奶大概是禁受不了这样的幸福,又一屁股跌落在凳子上,说,啊,是布,我
的布也回来了! 

石头和布都奔向了太奶奶。太奶奶紧紧地把石头抱在怀里,把自己的头放在石头那厚实的脊背上。石头轻轻地类似呜咽般低叫着,用它温暖的舌头一下又一下地舔着太奶奶龟裂的手掌。与此同时,太奶奶的脚脖子也有什么舐舔着,温热的,一下,又一下。
太奶奶知道,那是布。 

当太奶奶从石头的脊背上扬起她那满是花白发的头的时候,暮色中太奶奶的脸庞上爬满了泪水,可是太奶奶却笑着说,石头瘦了不少啊,不过—多好啊
,多好!一家人又在一起了…… 

在昏黄的油灯下听着奶奶慢慢悠悠地讲着这个故事的时候,我终于忍不住了,急切地问道 :“爷

\newpage
爷呢?爷爷呢?” 

当时爷爷就坐在奶奶的近旁,捧着茶碗,像当年的太奶奶那样,微眯着双眼,像是睡着了一样。他听到我的问话后,睁开了眼睛,我看见爷爷的眼睛里闪着泪花,知道爷爷并没有睡着,他和我一样在静静
地听着奶奶的故事。 

奶奶抬头望了一眼爷爷,目光和那昏黄的油灯的光一样温暖。她低下头,从纺车里变魔术一样扯出一根绵长的白线,然后,右手缓缓地摇着纺车,依然
不紧不慢地讲着…… 

爷爷知道这个小镇已然失去了往日的祥和,日本人的蛇蝎已经遍布小镇,所以在第二天晚上爷爷就
从地窖里爬出,负伤逃亡,去追随自己的部队了。 

爷爷留下了一些他从部队带回的疗伤的神奇药粉—云南白药,正是这些药粉让石头止住了伤口不断
涌出的热血。 

“喵—”轻轻的、轻轻的一声呼唤,躺在地上
\newpage
奄奄一息的石头睁开了眼睛,就像我们在黑暗中点燃了一根蜡烛,石头的眼睛里有了温暖的光亮,它看见
了布。 

在一个清冷的月夜,布从小镇的一方奔往另一方,它轻轻地走向石头,轻轻地呼唤着石头,并且用它的耳朵轻轻地摩挲着石头的鼻子,让石头感觉到,
布与它同在,希望与它同在。 

就这样,在石头养伤的那些日子里,总有布在它的左右,轻轻地叫着。我想,虽然我们听见的只是一声“喵——”但对于石头,也许布的叫声里包含了许多其他的表达,有时是鼓励,有时是宽慰。伴随在
石头左右的布,给了石头生的勇气与希望。 

不仅如此,布还经常从田间地头抓来肥硕的田鼠,轻轻地放在石头的嘴边,用爪子,轻轻地挠挠石头的鼻子,轻轻地叫一声“喵——”然后蹲在石头的旁边,一下一下地清洗自己的爪子和脸庞,布总是那
样干净,一尘不染。 

\newpage

石头只需要歪歪脖子,就可以把身体还保持温
热的田鼠衔在自己的口中,美餐一顿。 

直到石头重新可以站起来,可以奔跑,可以用它粗大的爪子把布温柔地揽在怀里。它们一起躺在地上,微闭着眼睛。太阳每天从东边升起,再到西边落下,花静静地开放再无声地落下,奶奶的小院里一直
空着,除了风偶尔来过,这里只有石头和布。 

奶奶哪儿去了呢?奶奶在爷爷逃走的第三天就被郭麻子关进了小镇的看守所。他们相信,爷爷一定会回来营救奶奶的,而那个时候,正可以不费吹灰之力就拘捕了新四军爷爷。这是郭麻子的如意算盘,他
肯定能够得到皇军的褒奖,加官晋爵。 

在那个初夏的黄昏,太奶奶匍匐在石头的背脊上哭着的时候,她一定是百感交集,在她风烛残年的时候,在国破家碎的时候,温暖她的是并不能开口说
话的石头和布。 

石头和布都静静地停在石榴树下,那个小小的
\newpage

土包里葬着剪刀的羽毛。 

石头伏下身子,半跪在地上,深深地嗅着,也许它闻见了剪刀的气息。过了许久,石头把自己的脖
子高高地扬起,望着苍茫的天空,低低地呜咽着。 

太奶奶和布一起望着石头,顺着石头高扬的头颅,他们看到苍茫的天空渐渐地暗淡了下来,最后一
抹霞光也暗淡了,暗夜吞噬了一切。 


在漆黑的夜里,一高一低,两双明亮的眼睛犹如夏夜里的飞萤,如流星追月般迅疾地向着小镇的镇公所靠近,然后在离公所约五百米远的灌木下无声地停下。石头扭转头看了看布,布也抬眼看着石头,它们交换了一下眼神。然后,布静悄悄地向着那个驻有哨兵的碉楼靠近。在碉楼旁边有一棵高大的樟树,樟树枝叶繁茂,很快,布就消失在枝繁叶茂的樟树之上

过了片刻,石头也越过灌木向碉楼走去,它奔

\newpage
走得很迅捷,却也很谨慎。 

碉楼上有只狗叫了起来,接着,吠声一片,一束强光从碉楼上扫了过来。石头停住了前进的步伐,
在这样的强光之下,它什么也看不见。 

碉楼上传来含糊不清的骂声,在这样的骂声里只听见吠声一片,由远及近。很快,在石头的面前立着五只日本军犬,其中,瞎了一只眼睛的正是石头的
爸爸。 

见了曾经置自己于死地的仇敌,石头肩脊的毛
全竖了起来。 

那只瞎了一只眼睛的军犬也认出了石头,它低
叫一声,其余的四只军犬一起向后退去。 

石头和瞎眼军犬对峙着,并慢慢地围绕着对方打转。在石头背靠碉楼的时候,它觉得视线很好,因为这样的位置是背对来自碉楼的强光,但同时,它也把自己的脊背留给了身后随时可能进攻的四只军犬。

\newpage

但是转到面向碉楼的时候,强光又让石头一无所见。总之,对于石头来说,局势于它极其不利。更何况,石头复仇心切,这样的心理对于石头来说也是
非常不利。这次交锋会让石头再次失败吗? 

布已经从延展着的枝丫上无声地跃上了碉楼。

在一个亮着灯的屋子里,布看见了郭小川和他的卫队,看见了郭麻子,布还看见了十来个日本兵。

随着新四军敌后游击队灵活机动的打击和正面交锋的国民军顽强抵抗,日本人向中国纵深推进的步伐放缓了。为巩固占领区并配合日军的“扫荡”和“清乡”,日本人在板桥店镇驻扎了一个卫队,主要任务就是和伪军一起肃清敌后抗日力量,及时与日本正
规军策应,并消灭零散的新四军抗日游击队队员。 

在碉楼的楼顶,有两名伪军正在值岗,他们一起顺着强光望着楼下两只狗即将开始的战斗,心中充
满了幸灾乐祸的欢喜。 

\newpage

布意识到了石头的危险,立即掉转身子顺碉楼
而下。 

石头最终选择了侧光而立,这样一来,它和瞎眼军犬一样,都可以部分地避开来自碉楼之上的强光,同时又可以兼顾来自另外四只军犬不预期的进攻。


曾经的挫败教会了石头许多。 

突然,石头如飞沙走石般向瞎眼军犬压了过去,还没有准备好进攻的瞎眼军犬被扑倒在地,另外四
只军犬立即站起,蓄势待发。 


瞎眼军犬一个滚地转身踉跄站住。 

这时石头和瞎眼军犬的位置都发生了改变,石头背对碉楼而立,在它的身后正是那四只蓄势待发的军犬,其中一只军犬正无声地向石头靠近,然而石头
所有的注意力全放在对面的这只瞎眼军犬身上了。 

就在布打算从碉楼的二楼跃身而下的时候,它
\newpage
发现了一个黑屋里囚禁着的奶奶,那个小黑屋里,除
了奶奶之外还关着另外三个游击队员的亲眷。 

布在那个窄小的窗口立住,冲着奶奶轻轻地“
喵——”了一声。 

奶奶抬头望见了暗夜里布雪亮的眼睛和缎子一
样的皮毛,奶奶说,布—— 

布在奶奶发现它之后就转身跳下窗户,跃身楼
下,借助樟树转跳地上,悄然无声。 

那只偷袭的军犬离石头只有一步之遥了,它伏了伏身子,准备着它即将开始的出其不意的一记漂亮
偷袭。 


石头也伏了伏身子准备着它的第二次进攻。 

几乎是同时,石头和那只偷袭的军犬一起跃起,它们一起撞向了那只瞎眼军犬。因为逆光的原因,原本就少了一只眼睛视力不济的瞎眼军犬被重重地撞
\newpage
飞了。在石头翻滚站起的过程中,石头回过头来用自己的双爪死死地按住了那只偷袭的军犬的脖子,并将自己强有力的牙齿刺进了它的咽喉。那只偷袭的军犬,在这种突然发生的改变之下没有来得及做出更进一
步的调整,血已经从它的咽喉中流出。 


其余的三只军犬很快围了上来。 

“喵——”布响亮地叫了一声,钻进了密密的
灌木丛中。 

石头听见布的呼叫,放开那只偷袭的军犬,跃
身而过灌木,和布一起消失在黑夜之中。 

碉楼之上的哨兵没来得及做出任何反应,只好招呼一声,那几只军犬一起停止了追赶,迅速地上楼了。躺在地上的那只军犬想努力地站起来,但是没有
成功。 

暗夜里奶奶压抑着的哭声虽然很低细,但是仍然让人觉得有很大的声响。在这个静寂的深夜,在这
\newpage
个逼仄的囚牢,奶奶的哭声让许多人也跟着哭了起来
,奶奶的哭声带给了他们许多的恐慌。 

太奶奶的眼睛已经适应了囚牢里的光线,她和那些熟识的和自己的女儿一样被关押的人打着招呼。自己失踪了好些日子的女儿终于知道了下落,太奶奶显得非常平静,甚至有些欣慰。她说,大家都活着呢
,哭什么啊?别哭,孩子们。 

太奶奶紧紧地握住了奶奶的手,握住了奶奶身旁的熊小荣的手,奶奶又握住了隔壁李红的手,就这样,大家的手都交互地握在了一起,温暖、力量和坚定通过彼此紧握的手相互传递着,大家慢慢地止住了
哭泣。 

奶奶问,布呢,还有石头呢?太奶奶沉默了一
会儿说,它们都还好吧。 

太奶奶说完这句话之后背过了身子在暗夜里默默地流泪,因为,石头和布是否安好,她心里并没有

\newpage
底。 

为了让石头和布能够逃生,太奶奶在预感到自己要被捕的那一刻,她已经把屋子的后门和窗子都打开了,石头和布却很警觉地在太奶奶身后亦步亦趋,
并做好了随时跟那些持枪强盗拼命的架势。 

太奶奶说,郭麻子,你要是有点良心认我这个
表姨就别和这两个畜生过不去,杀我剜我随你。 


郭麻子说,是!是! 

但太奶奶刚出院门就听见郭小川的卫队嚷嚷着
,关门,打狗!还有那只猫…… 

听到枪声的时候,太奶奶的心里一阵抽搐,她闭上了眼睛、太奶奶看见了在枪声里像花瓣一样飞散
的剪刀的羽毛和鲜血。 

石头和布会不会也如剪刀一样呢?它们在哪里
呢?天堂、地狱还是人间? 

\newpage

在这片广袤的土地上,虽然遍布着蛇蝎,但总有一只猫、一只狗的藏身之地,它们没有成为丧家之
犬,因为它们也有自己的使命! 

一高一低,两双明亮的眼睛犹如夏夜里的飞萤,它们轻轻悄悄地飘落在奶奶的小院,那是石头和布,它们在卫队的枪弹中逃逸,在山林里奔走,现在,
它们回来了,在夜里。 

突然,石头很警觉地停了下来,它回头看了布
一眼,它是在告诉布,这个小院里有新的情况。 

布静静地蹲在月夜下,石头循着气味一点一点
地向地窖靠近。 

突然,石头向后面退了十多步,然后一动不动
地盯着地窖的出口。 

地窖上面盖着一捆破旧的芦席,芦席旁堆着一堆柴草,谁都看不出,在柴草旁边的芦席下面有一个地窖。芦席轻轻地被挪动了,露出了一个黑糊糊的洞
\newpage
口,过了一会探出了一个头,那个人向四周望了望没有发现异常情况,就很迅速地从地窖下端出一簸箕泥土,然后把泥土分散在小院的四周。就在这个人打算再进入地窖的时候突然一个激灵,吓了一跳,因为他
看见了蹲在地上一动不动的石头。 


那个人小声地问,是石头吗? 

石头觉得这声音很熟悉,但是它还是没有动。

石头,难道你忘记了吗?我帮你敷过药啊!那
个人仍然小声地说。 

石头站起身来,慢慢地向那个人靠近,在闻见了熟悉的气味之后,高兴的石头一下子把毫无戒备的那个人扑倒在地,那个人就倒在地上紧紧地搂着石头
的脖子,压抑地喊着,石头,石头! 


石头边舔着那个人的脸,边呜呜地低叫。 


\newpage

布也轻轻地走了过来。 

啊,布,布也在这儿!那个人把布也揽在了怀
里。 


那个人就是爷爷。 

虽然,爷爷与布和石头不是太熟,但是石头一直记得自己在昏迷中,这个人是怎样细心地替它剪掉伤口周围的毛,然后用包谷酒清洗伤口,再小心地涂
好药。在伤痛中,石头永远地记住了爷爷的味道。 


没想到,会是这样一种相逢。 

听奶奶讲到这里的时候,我哭了。我想,如果
石头和布也会哭泣, 

那么它们肯定也会如我一样地哭泣。这哭泣里有点喜极而泣的意思,但肯定远远不止这些,肯定交集着更多的情感,因为,石头和布经历了那么多事情的变故,剪刀的死,奶奶和太奶奶的被捕,还有敌人对石头和布的通缉与追捕,石头和布在枪弹中亡命逃
\newpage
逸,几乎是在绝望中遇见亲人,遇见可以信任的人,
这该是怎样的一种情感体验啊! 

爷爷很快把石头和布带进了地窖,因为,外面
太危险。 

这是怎样的一个地窖啊!虽然入口很窄,但却逶迤蜿蜒而出小镇之外,而且在这个地窖里还不止爷爷一个人,还有熊小荣的弟弟、李红的丈夫、林燕的
爹…… 

一高一低,两双明亮的眼睛犹如夏夜里的飞萤,飘忽在镇公所碉楼五百米开外的灌木丛中,那是石头和布。石头蹲在地上伸着长长的舌头,爷爷搂着石头的脖子,对身边的同志点了点头,大家交换了一下
眼神,他们在等待着时机。 

这已经是第三个晚上了,无奈碉楼的看守太过机警和严密,扫来扫去的强聚光灯让任何人都无法进
行成功的偷袭。 

\newpage

不知道什么原因,碉楼上的日本军犬吠声一片,这让石头的耳朵立即竖了起来。突然,石头一个鱼跃,跳过灌木丛向碉楼奔去。很快,碉楼上值守哨兵的聚光灯就发现了石头。一个哨兵悄悄地支起了步枪
,等待着石头跑到他的有效射程之内。 

石头在上次和日本军犬角斗的地方停了下来。

碉楼顶上一阵喧哗,几乎所有的人包括郭麻子和日本鬼子都上了碉楼的楼顶。他们一致决定不用枪,而是和上次一样放下他们的狼狗,他们想看一出好戏,因为皇军对于自己的军犬获取胜利有绝对的信心

很快,那四只军犬并列而立在石头的面前,日本人的两盏聚光灯都照了过去。一个日本人说,这就像在看电影。而那时生活在小镇上的郭麻子之流并不知道电影是什么,但郭麻子他们喜欢看戏,于他们来
说,这绝对是一出好戏。 

虽然上次战败,但是瞎眼军犬显然依旧是首领,它低叫一声,余下的三只军犬蹲在了三米开外的地
\newpage

方,仍然由瞎眼军犬来跟石头对阵。 

有了上次的经验,石头选择了侧光而立,前爪低伏,力量都集中在后爪,后爪深深地嵌入地下,这样就可以更有力地后蹬,这种姿势亦守亦攻以静制动


石头在战斗中成长。 

爷爷和另外几个同志都紧张地望着这场即将开始的厮杀,都捂着胸口怦怦跳动的心,大气都不敢出

这个时候,布轻叫一声,钻过了灌木,向着碉楼奔去。爷爷马上会意,他挥了挥手,其他的同志也
都意识到,这是一个偷袭的好时机。 

爷爷他们循着布的身影,远远地避着那两道强聚光,向着碉楼下面的那棵高大的樟树靠近,碉楼上
所有人的注意力都在这两只对峙而立的狗的身上。 

瞎眼军犬忽然凶猛地向石头冲撞过去,石头有力的后腿蹬地一个腾挪闪开了瞎眼军犬的进攻,瞎眼
\newpage
军犬一个趔趄没有站稳。趁着瞎眼军犬调整步伐的时候,石头忽然跳起,骑在瞎眼军犬的背上,并且从上面咬住了瞎眼军犬的颈子。瞎眼军犬顺势一滚,挣脱
了石头的利齿,血顺着颈子往下流去。 

一个日本人按了按腰间的佩枪,掀起搭扣,把
枪端在手中,骂了一句日本话。 

郭麻子见状也给旁边几个伪军递了个眼色,这
几个伪军也都把枪端了起来,把枪口对向了石头。 

布在高大的樟树下停了下来,它回过头来望了望爷爷他们,然后再回过头去噌噌噌上了树。爷爷他们望了望高大的樟树,相互交换了下眼神,把枪别进
腰间,也噌噌噌地上了樟树。 

受了伤的瞎眼军犬低叫了一声,那并列而立的三只军犬中的一只军犬站起了身,并做好了进攻的准备。而此刻的石头正准备乘胜攻击,对战胜瞎眼军犬
它有了绝对的信心。 

\newpage

就在石头准备再次跃起的时候,来自那只军犬猝不及防的进攻,让石头狠狠地摔倒在地。石头很快
就地滚起,它意识到自己不是在和一只狗战斗。 

在这次有效的攻击之后,另外的两只军犬也站
了起来,它们一起从四个方向把石头围在了中间。 

日本人面露得意色,把枪收进了枪匣,郭麻子也喜形于色,摆了摆手,那几个端枪的伪军也收了枪

布已经跃上了碉楼,它站在那个小窗户那里,
轻轻地叫了一声。 

啊,布!太奶奶和奶奶同时喊了起来,布!布
!是布吗? 

听到奶奶和太奶奶的声音之后,爷爷他们很快就明白了自己的亲属被关押的具体位置了,他们很快
顺着樟树的枝丫攀缘到了碉楼上。 

这边的石头已经被连续地扑倒了三次,每次都
\newpage
是在石头还没有从跌倒中调整好就又被扑倒,而且,
石头不知道下一次的进攻来自何方。 

石头伸长着舌头,气喘吁吁,身上已经多处负
伤。 


这样下去,石头恐怕性命难保。 

碉楼上的人们此刻已经显现出了看戏时的那份
轻松了,他们嬉笑怒骂,胜券在握。 

仅有的一个值守伪军已经被爷爷他们无声地消
灭了,奶奶和太奶奶她们正被相继地接下碉楼…… 

伤痕累累的石头反而在多次的跌倒和挫败中平静了下来,它决定要发起主动攻击,而且,攻击的对
手选定为逆光而立的狗。 

就在石头又一次被扑倒之后,它顺势一滚,立即跃向逆光而立的瞎眼军犬。因为强烈的聚光,瞎眼军犬几乎是一无所见毫无防备地就被石头击倒在地,
\newpage

肚子被拉开了一道口子。 

石头没有回到包围圈中,而是立即又向瞎眼军犬近旁的另外一只军犬扑了过去,它们撕咬在一起,其他的军犬立即围了过去,但却并不攻击,因为这样
很容易攻击到同伴。 

爷爷他们已经把所有的亲人都解救了出来,并
绕过了强聚光悄悄地向着灌木丛靠近。 

太奶奶多想把布揽在怀里啊,可是布呢?布呢
? 


太奶奶低声地呼唤着布。 

布已经靠近了石头,它要唤石头离开这个危险
的地方。 

石头和它撕咬着的那只军犬分开的时候,石头艰难地站了起来,而那只军犬却一动不动地躺在了地

\newpage
上。 

这时,碉楼上所有的枪口都瞄准了石头,日本
人因为又损失了一只军犬而呱呱大骂。 

酣战中的石头仿佛听见了布的呼唤,血正从它身体的各个部位往外淌着,它的躯体如火一般地燃烧着,它已经感觉不到丝毫的疼痛,但来自布的呼唤让它猛醒,它要飞奔追随布而去,离开这里。它用后腿蹬地,它要像风一样地飞跑,离开这里,离开。石头的内心也有个声音这样呼唤着它,离开,它要离开。

它要奔跑,心里的愿望如此强烈,真正奔跑的时候,才知道自己强有力的后腿已经被那些日本狼狗的利齿咬断了,露出了白森森的骨头,每走一步,都
刺痛着自己的灵魂。 

忽然,石头艰难地转身了,它扬起脖子低低地哀鸣着,仿佛是在和布告别。它用尽自己所有的力气向那只瞎眼军犬扑了过去,虽然,它的骨骼已经被那些狼狗咬碎,但是它的牙齿依然锋利,它要把自己锋

\newpage
利的牙齿刺入敌人的咽喉。 


一阵枪声之后,石头以战斗的姿势倒地。 

在这样的枪声中还有布凄厉的叫声和太奶奶痛彻心扉的哭泣。奶奶说,太奶奶当时嘴巴里流着血,不知道是来自胸腔,还是因为伤痛咬破了嘴角,总之,太奶奶昏厥过去了,是爷爷背着太奶奶离开了那个
死地。 


过了这么多年,奶奶讲述这个故事的时候依然会伤痛地哭泣。记得油灯下,泪眼婆娑中我望见奶奶扬着左手,拉着那根长长的棉线,不住地颤抖,右手停止了摇纺车,眼泪顺着她褶皱的面颊无声地滑落…
… 

过了这么多年,当我回想起这个故事的时候,我的心依然像我第一次听到这个故事的时候一样,那样的伤痛像是有许多小虫子,一起噬咬着我的心,让我的灵魂也疼痛地跟着发抖。我多怀念那只我从来未

\newpage
曾见过的狗啊,它的名字叫石头。 

那个晚上,枪声、马蹄声、搜寻追捕的吆喝声打碎了小镇上的安宁,无数的火把让小镇的夜空也燃烧了起来。太奶奶、奶奶、爷爷他们躲在那个曲折蜿蜒的地洞里,外面的追杀于他们像是一场梦,他们一
起在这个地洞里躲过了一场劫难。 


然而当晚布却并不在那个地洞里。 


布呢? 


布呢?太奶奶苏醒后的第一句就问,布呢? 

然而爷爷奶奶他们并没有答案,在那样慌乱的逃生中,人们都忘记了那只猫,布。但是大家都相信
,布仍然活着。 

经过这次劫难,目睹了石头的死掉,再次经历了与布的失散,太奶奶在这样颠沛流离的逃亡日子里没能走到最后。直到太奶奶弥留之际她仍然叨念着她的小院,小院里的茄瓜蔬菜,鸡飞狗跳的剪刀与石头
\newpage
,还有和她的儿子一样神武的布。谁都没有告诉她,
她的小院在逃亡的那个晚上已经被付之一炬。 

那个时候想要有一个安宁的家园是多么奢侈的
一个梦想啊。 

布呢?布到底到哪里去了?小时候我总喜欢这样问,似乎相信任何人或物都自有其最终的归宿,可是奶奶却始终无法回答我,因为,她那时远离了小镇,流亡在外,爷爷他们因为擅自行动,目无纪律,受
到了部队的严厉处罚。 

终于有一天,奶奶告诉了我布最后的故事。奶奶大概是被我的问题缠得无法脱身不得不给出一个答案。现在想来,这个故事也许是奶奶自己编的,但我
却一直相信这是真实的。 

奶奶说,没过多久板桥店镇就被收复了,日本人节节退缩,抗战胜利指日可待,这个时候奶奶和爷爷就回到了小镇。当时镇里流传着这样一个离奇的故

\newpage
事。 

枪声、马蹄声、告密、暗杀、清乡、围剿……日本人和郭麻子把板桥店镇带进了没有尽头的寒冷的冬天。如果没有枪声、马蹄声和日本军犬搜捕吠叫的声音,那么这个小镇就像陷入了无边无际的静寂之中,似乎连空气都是凝固的,如果是在黑夜里,这样的
静寂更加一倍。 

传言就是在这样的情境下风生水起,说是有一只黑色的怪兽,专门在暗夜里刺杀携带枪支或在镇公所值勤的伪军,被袭击的人,要么被刺瞎了眼睛,要么就被破了相。起初,镇上的人们不太相信,但逐渐郭麻子的队伍里独眼龙多了,而且,被破了相的人也
多了起来,人们这才相信传言的真实。 

而郭麻子他们则如临大敌,都认为是有鬼怪神明在惩罚汉奸卖国贼,一时间人心惶惶。而且,后来的传说说这只黑色的怪兽有两只像蝙蝠一样的肉质翅膀,可以飘忽飞翔,还有,所有被袭击的人的枪都在
扣动扳机的时候失灵…… 

\newpage

这样的故事把我听得如痴如醉,大快心意,我忍不住说道,肯定是布,布就是黑色的。可是布怎么
能飞呢?它又是如何长出翅膀的呢? 

又一个暗夜来临的时候,不知道哪家的孩子因为饥饿睡不着,打开窗户望着外面,然后扭转头对妈
妈说,妈妈,好多的萤火虫。 

妈妈不以为然地说,冬天怎么会有萤火虫?但是当她也站在窗边的时候,她半张着嘴巴,一个字也
说不出来。 

她看见在板桥店镇的街道上,有无数的萤火虫一样摇曳漂移的眼睛,这些眼睛潮水一般向着镇公所流泻而去,街镇上的石板路上传来冰凉的细碎的声音,那是肉质的脚掌踩踏青石板行进的声音,那样的声音只有在如此的静寂中才能听得见。妈妈揉了揉眼睛
,她看见了无数的猫一起无声地奔跑。 

郭麻子跟往常一样,眯着眼睛喝着酒,因为传说中有怪兽的袭击,所以即使在喝酒的时候,他的右
\newpage
手也按在腰间的盒子炮上,而是用左手举杯。他一直不相信枪会在怪兽的面前成了哑巴。他想,如果他遇见了,肯定会像枪击那只名叫剪刀的鸡一样,漂亮而利索。就在这个时候,他听见了一声凄厉的猫叫。他
心中一喜,道,来了! 

他上到碉楼的时候发现值守的伪军和日本人一起半张着嘴巴,惊恐地望着碉楼下,手中的枪也垂了
下来。 

郭麻子也抬眼望去,不禁打了个寒战,他望见了下面数不清的如夏夜萤火虫一样繁多的眼睛。起初,他想数一数,看到底有多少猫,但很快,他就放弃
了,他根本就不可能数得清。 

日本人从最初的惊恐和慌乱中安静了下来,聚
光灯向碉楼下扫了过去。 

地上密密匝匝的全是猫,不同颜色的猫都汇集到了这里。它们无一例外地都弓着背,背上的毛都奓立起来,随时准备投入战斗,随时准备开始撕咬。当
\newpage
聚光灯扫到那棵樟树上的时候,只见樟树的枝枝丫丫
上都立满了猫。 

碉楼上的机枪很快架了起来,空气绷紧得像一
根根冰冷的铁丝,刺痛着每个人的神经。 

咆哮着的凶残的军犬被放了下去,但是当那些狼狗真正站立在猫群之中后,它们只剩下战栗的哀吠
,但是它们已经退无去路。 

就在碉楼上的人要发令射击的时候,他们发现在他们的腿边立满了猫,每只猫都弓背龇牙,目露凶光,碉楼上的每一个人都陷入巨大的惊恐之中却都安静得几乎失去了呼吸,因为惊恐,他们想大声地呼救
,却无法叫出声来。 

屋顶上、墙角里,甚至连架好的机枪架上都是
猫,猫的眼睛照亮了这个蛇蝎盘踞的碉楼。 

几只军犬凄厉的惨叫打破了这死一样安静的对峙,就在那些手指即将扣动扳机的时候,碉楼上的人
\newpage

们几乎是同时感觉到了脸上火辣辣的刺痛。 

一瞬间,这千万只猫同时消遁在冷寂的暗夜中,如它们的到来一样,仿佛是一场梦。除了郭麻子他们脸上仍然清晰的刺痛还那么真实之外,一切都像是
一场梦。 

碉楼下的三只日本军犬因为全都失去了眼睛,
在无边的黑暗中哀号奔突。 

怎么会有那么多的猫呢?我至今都想不明白为什么会有那么多猫,它们能够会聚起来,听从一只猫的调度,听从一只名字叫布的猫的调度。而布,仿佛就如它最先前的主人一样,原本就是一位驰骋疆场指
挥千军的将军。 

奶奶讲完之后有点忐忑地看着我,她有点担心我是否会怀疑。但是我闭着眼睛,沉醉在奶奶的讲述中。我的毫不怀疑让奶奶放下心来,因为我相信,布从来都是一个传奇,指挥千军万马的事情,只有布能够完成。然而现在想来,也许,这样的结局只是小镇
\newpage
人们的一个愿望,或者说只是奶奶的一个愿望。真实的情况也许是这样的,布可能会袭击那些伤害过剪刀和石头的敌人,但毕竟,它只是一只猫,在那样烽火连天的战争岁月,人们会很快忘记一个猫最终会流落何方。这样想过之后,心中自然盈满感伤,我还是愿意相信,布指挥着那些猫战斗,再指挥着那些猫撤退。它们去了哪里呢?这个世界这么大,总有许多可以让猫们自由生活的家园。因为直到现在,我还是能够在我们的周围看见许多猫,虽然它们没有主人,但它们一样自由地生活着,而且它们的身上是那样干净,
不像流浪狗的身上总是脏脏的,并且无助地彷徨。 

当我在演播厅的一个角落里重温着奶奶讲给我的有关剪刀、石头和布的故事之后,我决定把这个故事写下来,来纪念我的奶奶。我的奶奶于1998年世,那时正是我读大学二年级的暑假。

\end{document}
