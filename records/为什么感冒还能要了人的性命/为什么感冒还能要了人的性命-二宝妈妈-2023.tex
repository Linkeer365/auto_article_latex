\documentclass{article}
\usepackage[utf8]{inputenc}
\usepackage{ctex}

\title{为什么感冒还能要了人的性命\footnote{Click to View:\url{https://web.archive.org/web/20230326092311/https://paste.ubuntu.com/p/T3dGkCNRPh/}}}
\author{二宝妈妈}
\date{2023-03-25}

% \setCJKmainfont[BoldFont = Noto Sans CJK SC]{Noto Serif CJK SC}
% \setCJKsansfont{Noto Sans CJK SC}
% \setCJKfamilyfont{zhsong}{Noto Serif CJK SC}
% \setCJKfamilyfont{zhhei}{Noto Sans CJK SC}
% \setlength\parindent{0pt}

\begin{document}
\CJKfamily{zhkai}

\maketitle


\Large


直到现在,我还难以相信这是真的。 

我的职业是律师,平时工作比较忙。家中两个宝贝女儿,大宝12岁,上五年级,二宝5岁,在幼
儿园上中班。 

2023年3月3日(星期五)下午2:00多,我突然接到二宝幼儿园医务室的电话,说二宝发烧了,38.5度,让家长把孩子接回家。由于我单位离幼儿园比较远,我就立即联系了孩子奶奶,让她把孩子接回来,不用回家,直接到家门口的火箭军医院。我直接从单位到了火箭军医院,并给孩子挂好儿科的号。没多久,奶奶就带着孩子到了儿科。此时,客户的电话不断打来,我就一个又一个地接电话,奶奶带着孩子完成了看诊及抽血化验。据奶奶说,抽血
\newpage
时,因孩子血管太细,护士扎了几次都失败了,孩子哭得撕心裂肺,后来改为抽指血,化验结果也很快出来了。奶奶拿着化验单找医生开药时,我进了医生的诊室,询问医生是否要做一下甲流检测,最近甲流比较流行。医生说,无论甲流还是乙流,医院都是同样开药,只开了小儿豉翘,还有一个止咳药,家中有美林,所以就没开。此前,我在网上买了奥司他韦,物
流显示第二天会送达。 

当天晚上,我给孩子吃了小儿豉翘,药不好吃,但是孩子还是很听话的吃了,晚上给孩子测过两次体温,都在38度左右,所以也就没吃美林。周六(3月4日)早上6点左右,孩子体温38.8,我给她吃了一次美林,然后孩子起床时体温已经退到38度以下,还吃了一些早饭,一上午的精神状态都不错,还在喋喋不休地给我讲幼儿园的各种事情,说惊蛰老师会带着她们种地,她带的种子是绿豆,不知道能不能发芽,还说想种一些土豆。上午10点多,又给她吃了一次小儿豉翘,吃完药之后她说有点困,想睡觉,我就让她在我的大床上睡了,大约睡了40多分钟,孩子突然就吐了,全是小儿豉翘的颜色,黄呼呼
\newpage
的。奶奶就把孩子抱到他们的房间去睡觉,我开始清洗被污染的衣物和床单。在奶奶房间睡觉时,奶奶想给她换个被子,她还回答不要换。期间,孩子提问又有升高,爸爸就给孩子又吃了一次美林,吃完美林之后大约1小时,爷爷再次给她测量了体温,这是发现是41.5度,爷爷以为自己看错了,又让我确认一遍,确实是41.5度。(在吃完美林之后,孩子爸爸陪着姐姐去家附近的金鼎轩吃午饭了,我还跟爸爸说,看看附近的药房是否有奥司他韦,怕物流今天无法送达。)发现孩子体温超高之后,我赶紧让爸爸回
来开车带孩子去医院。 

我知道家门口的火箭军医院没有奥司他韦,而此前在家长群中看到别的家长说儿童医院急诊排队也要3小时以上,另外有一个家长说北大妇幼的人不算多,我查看了一下路程,只有4.8公里,然后我就决定去北大妇幼。在去医院的路上,孩子再次呕吐,期间奶奶发现孩子精神状态不好,想让我们以最快的速度掉头回到火箭军医院,我说那个医院太差劲,还
是去北大妇幼吧。 

\newpage

到了北大妇幼,我先下车去挂号,奶奶随后抱着孩子下车,爸爸去停车。我在排队挂号时,爸爸打来电话,说孩子精神状态不对,脸上是很诡异的笑容,我赶紧出去接奶奶,看到看孩子也是在笑,但是眼神已经不对了,我赶紧让护士安排抢救室。到了抢救室发现,孩子下肢僵硬,医生说是惊厥,然后直接打了一针6毫升咪达唑仑。下肢僵硬稍有好转,护士又给孩子测了一下甲流抗原,立即显示了暗红色的阳性。我又按照护士安排去挂号和缴费,没多久,奶奶再次发现孩子下肢僵硬,医生又安排了6毫升的咪达唑仑注射,然后我跟着医生去办理住院手续。下午15
:26分,孩子住进PICU。 

然后管床医生安排了一次谈话,大致内容是,因甲流导致超高热,引发了高热惊厥,会安排孩子进行一次腰穿检查,如果没有问题则可放心一半,但是不排除后续会有脑水肿、脑疝等极端情况的出现。直到这时,我都很冷静,为孩子交了住院费用,到医院门口买了纸尿裤、护理垫、护肤霜、干湿纸巾等用品,安慰奶奶不用担心,高热惊厥也是常见现象,一般

\newpage
都不会有事的。 

晚上六点多,医生让我们回家,PICU不让探视,留在医院没有意义。我们回家的路上,PICU打来电话,说孩子无法自主呼吸,需要气管插管,让我们知情并同意,我们立即同意。晚上八九点钟,PICU再次来电,说孩子心脏两次停止跳动,通过心肺复苏都抢救回来了,同时医院组织了各科专家会诊,让我们回到医院再次沟通病情。晚上十点左右,医生告知情况如下:初步判断是甲流引发的急性坏死性脑病,该病死亡率高,预后差,绝大多数都有严重的后遗症,植物人、脑瘫、行动或语言障碍,只有极小的概率可以完全恢复。此时我已经意识到问题的严重性,爸爸和奶奶已经哭得肝肠寸断。我没有掉一滴眼泪,不断用手机检索关于急性坏死性脑病的各种相关知识。晚上十一点多,奶奶让我给一对做医生的姐姐和姐夫打电话,寻求一下他们的帮助,姐姐二话没说,直接到了医院,姐夫则通过他的渠道不断在了解孩子的病情,姐姐不断在和姐夫通话,然后就跟我说,什么都不要想,相信医院,相信医生,只要孩子能抢救过来就行,别想其他的。姐姐一直陪我们到凌晨2点多,医生让我们先回去,已经针对孩子的情况进
\newpage
行了对症治疗,包括使用托珠单抗、激素、以及其他一切应对措施,但是孩子无法脱离呼吸机,并且脑电波出现了大慢波,证明脑电活动很弱。我们让医生尽力救治,并且保证不在医院打扰医生治疗,也让医生
姐姐回家休息了。 

凌晨3点多,PICU再次来电,说要进行血液置换,让我们再回到医院签署知情同意书,我们立
即回去签字。 

一夜无眠。3月5日早上,爷爷做好了早饭,但是所有人都无法下咽,我让大家务必保重身体,二宝未来康复还需要所有人的参与。我对大宝说,如果我们离世之后无法照顾妹妹,你作为姐姐也要照顾她,大宝哭得稀里哗啦,不断地说,妹妹不会有事,她
平时活蹦乱跳的,让我们不要吓唬她。 

我们再次到医院等待消息,中午的时候,爸爸说嗓子疼,发烧,然后我们到北大一院的发热门诊给爸爸挂号检查,爸爸的甲流抗原检测也是阳性,但是颜色比较浅。医生给开了奥司他韦和布洛芬。这一天
\newpage
医生主要是告知孩子脑电活动已经没有了,是一条直
线了,肝肾功能都在衰竭。 

3月6日,医生说孩子救活的希望不大,目前已经没有其他可以使用的更有效的救治方法了,目前其实已经是脑死亡状态,但是还是需要进行脑死亡的评定,小儿脑死亡评定需要间隔12个小时。当天晚上,我们同意医院进行脑死亡评定。同时经过申请,我们得以进到PICU看了她一次,小小的身体,插满了管子,鼻饲管、呼吸机、尿管、血液透析等等。我摸了她的小手,还是肉肉的,护士已经把她的小辫子梳得高高的并且编起来了。我看到了他周围的那么多叫不上名字的机器,我知道她在用这些仪器维持着生命。我跟她说出院之后可以去种地,说幼儿园的小朋友都想她了。看了没多久,血氧维持不住了,医生让我们出去。后来告知我们因为有痰液,吸出来就好了。那天晚上,医生再次跟我们交流病情,我知道医生已经尽力了,我就跟医生再一次把孩子生病的整个过程详细地描述了一遍,我希望给医生提供更多的信息,希望他在以后的诊疗过程中可以有更多可参考的信息。管床医生是个年轻的小伙子,他说自己也快要
\newpage

当爸爸了,他说能够理解我们的痛苦。 

当天晚上,我在京东给她买了她喜欢的公主裙和一双闪亮亮的公主鞋。晚上,爸爸问我如果宣告脑死亡了,我们还要继续坚持吗?我第一次放声大哭,我说我要继续,只要她的手还是热的,只要她还存在,我真的太爱她了......低头痛哭的时候,眼泪会滴在眼镜上,一层一层的眼泪,干了之后在眼镜片上留下一片片白色的无机盐,我已经看不清楚这个
世界了。 

3月7日早上,大宝起床时说嗓子疼,我立即给她向老师请假,并告知家中妹妹及爸爸确诊甲流的情况,老师说如果确诊甲流需要向学校报备。我在北大妇幼给大宝挂了急诊号,中午的时候大宝开始发烧。大宝到医院后先到PICU门口跟门口的护士央求进去看妹妹一眼,但是护士没有同意,告诉她妹妹现在不方便看。大宝一边哭,一遍迷迷糊糊地跟我去急诊那边候诊。医院进行甲流抗原检测是阴性,但是医生结合妹妹和爸爸确诊甲流的情况认为大宝肯定也是甲流,也给开了奥司他韦和止咳药。在陪着大宝就诊
\newpage
的过程中,PICU来电说二宝情况十分不稳定,可能过不了今晚了。我们跟医生咨询了孩子的器官是否还有可以捐献的,如果她身体的任何一个部分可以留在这个世界上,对我们来说都是一种安慰。医生告诉我们,她的器官都已经衰竭,并且血液中的炎性因子
太多,遍布全身了,所有的器官都不能用了。 

我们把大宝送回家,同时带着新买的衣服和鞋子到了医院。我们又一次进到PICU,她还是静静地躺在那里,我握着她那热乎乎的小手,多希望她能睁开眼睛看我们一眼,又是很短的时间,她的血氧再
次无法维持,我们只能出去等待。 

不到半个小时,医生出来跟我们说孩子已经走了,让我们再等一下,护士处理完毕我们就可以进去给她穿衣服了。傍晚6点多,我们进去的时候,她的小手还是软软肉肉的,小裙子很漂亮,当天晚上6:30本来是她上舞蹈课的时间,我给她穿上了白色的舞蹈袜,我还没来得及向舞蹈学校给她请假,我把她
喜欢的那些漂亮的橡皮筋都放在她旁边了。 

\newpage

后面就是跟着太平间的工作人员办理手续,一切都办完之后也就7点多。我和爸爸不敢回家,不敢面对爷爷奶奶那些急切的询问,我们从医院步行回家,到家也没敢上楼,在车里坐到十点多,预计大宝该
睡觉了,我们才上楼回家。 

大宝确实已经上床了,爷爷奶奶还在等我们,我们告诉他们二宝已经走了,爷爷奶奶哭得直跺脚,奶奶认为是火箭军医院给耽误了病情,要去医院找那个医生同归于尽,我们告诉她们这是小概率事件,不是任何人的错误,不要自责。爷爷奶奶的哭声把大宝
吵醒了,大宝自己藏在被窝里泪流满面。 

3月8日上午,我们到八宝山殡仪馆了解一些殡葬事宜,殡仪馆告知6岁以下的小朋友是没有骨灰的。当天下午,我们通知了孩子的叔叔和婶婶,带着爷爷奶奶和姐姐一起到太平间看了二宝最后一次。当天下午3:00多,PICU 通知可以开具死亡证明了,同时办理了出院结算,交了10万元的住院押金,各种费用总计9.5万左右,医保报销6万多。

\newpage
当天联系了殡仪馆的车辆并预约了火化时间。 

3月9日早上6点,殡仪馆的车辆准时到了太平间,爷爷在家陪着姐姐。叔叔婶婶爸爸、奶奶和我一起到的太平间,这次把二宝一年四季的衣物鞋袜都带去了,铺在那个红红的小棺材底下。还有姐姐给她用彩泥捏的“皮医生”,姐姐英语课上表现好老师给她的糖果,这些都是姐姐特意让我们给她带去的。因为无法保留骨灰,爸爸想到要留下一些东西,所以我们用剪刀剪下了她的一截头发,爸爸说,如果未来技术发达,或许可以用头发克隆出来一个二宝。我跟在
灵车里,爸爸开车带着其他人一起到的殡仪馆。 

我和爸爸在办理手续的过程中,叔叔婶婶陪着奶奶,奶奶说一只小喜鹊从他们面前低低地飞过,落在旁边的松树上,不断的扭头看着奶奶,奶奶说,那
是二宝不想走,化作小鸟来看看奶奶。 

因为没有通知亲友,没有遗体告别,没有骨灰,所以整个流程不到1小时就全都办完了,送走一个
宝贝,原来可以这么简单。 

\newpage

当天下午,我通知了幼儿园,办理了退园手续,带回了幼儿园储物柜上的照片和她在幼儿园的物品,老师和园长都跟着一起流泪,只叹缘分太浅,后来幼儿园老师把她在幼儿园两年来的所有照片和视频发
给了我。 

晚上的时候,爷爷做了她喜欢吃的三个菜,用三个小碟子装好叫她回家吃饭,然后一家人再次哭成一团,我跟爷爷奶奶说,不太再做这些仪式了,徒增伤感。晚上不到8点,或许是太累了,我在卧室好像睡着了,但是可以听到爷爷奶奶和爸爸在客厅说话,我想起床可是身体无法动,想喊可是却喊不出声音,好像过了很久才喊出爸爸的名字,他们问我怎么了,我说可能是二宝回来找我了,奶奶说有些事情,不能
不信,7天之内,孩子肯定还在家里呢。 

单位同事知道了我的情况,主任和我的师姐一起来家里看望我,我说不用安慰,孩子的性格像我,一切都干脆利落,绝不拖累别人,我下周一就可以到单位正常工作了,这一星期耽误了太多工作,很多客户还不了解情况,我需要尽快投入工作,这样才会更
\newpage

快地忘记这些伤痛。 

上周末,我和爸爸带着姐姐在京西林场参加了义务植树,二宝喜欢门头沟的山山水水,我们希望门头沟继续山清水秀,二宝可以随时回去看看她喜欢的
那片山、那湾水。 

爸爸保留着二宝所有的涂鸦,爷爷不断地把手机里二宝的照片冲洗出来,他说这样才能更长久地保存;奶奶经常会精神恍惚,过马路不看交通灯,甚至二宝那二万多的压岁钱她数了几个小时都数不明白了;姐姐似乎还是跟从前一样疯疯闹闹,但是我知道,她的表面快乐只是想安慰爷爷奶奶,她怕家里太安静
了。 

我不断地给自己增加工作,以前都是由助手起草的文件,我现在都在自己亲自动手来起草,只有不断地工作,我可能才会不那么痛苦。前天傍晚在客户那里开会结束,回家的路上,再次路过了过去每天送二宝上学的那条小路,二宝以前总是自己跑在前面,然后藏在一个配电箱后面,让我过去找她,我知道再
\newpage
也不会有人喊“妈妈你来找我呀”。昨天早上,我刚到办公室,就看到一只喜鹊落在外面的窗台上,还扭头看了我一眼,这一眼,我就相信是二宝回来看我了。我想起了二宝以前总会扭着屁股说的儿歌“你们都给我变鸟,变很多很多的鸟,变各种各样的鸟,鸟儿鸟儿满天飞”,二宝,你真的变成那个小喜鹊了吗?
 

又到周末了,以前总会带着二宝去门头沟爬山,但是今天我就躲到单位加班了。上午写完了客户需要的文件,但是我还不想回家,就把这二十多天以来所有事情记录一下吧。

\end{document}
