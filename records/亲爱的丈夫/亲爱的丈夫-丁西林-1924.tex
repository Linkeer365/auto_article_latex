\documentclass{article}
\usepackage[utf8]{inputenc}
\usepackage{ctex}

\title{亲爱的丈夫\footnote{Click to View:\url{https://web.archive.org/web/20221014020423/https://rentry.co/rofeq}}}
\author{丁西林}
\date{1924}

% \setCJKmainfont[BoldFont = Noto Sans CJK SC]{Noto Serif CJK SC}
% \setCJKsansfont{Noto Sans CJK SC}
% \setCJKfamilyfont{zhsong}{Noto Serif CJK SC}
% \setCJKfamilyfont{zhhei}{Noto Sans CJK SC}
% \setlength\parindent{0pt}

\begin{document}
\CJKfamily{zhkai}

\maketitle


\Large


剧中人物 


原先生 


老刘 


任太太 


任先生 

布景:一间中国旧式的厅屋,后面左右两边,各有一独扇门,通过道。室之右壁为旧式长格,室之中央,置一小桌,四围置轻便小椅数张,桌上放一花瓶,内置鲜花。室之左前部装一洋式白铁火炉,炉旁置茶几沙发及安乐椅,椅上皆有腰枕。一切家具陈设
\newpage
新而精致,处处表现出一个新成立的新家庭的气象。一个年在二十岁以上的青年(原,)带了帽子,穿了大衣,立在电灯之下看晚报。少停,一个听差的(刘
)由右门走进,手捧茶具,肘下挟了两本杂志。 

刘:老爷在书房里写信,一会儿就来。(将茶具放在桌上,将杂志送到原的面前。)这是刚寄来的
杂志。——这屋里很热,您不把大氅脱了么? 

原:(讲晚报塞进大衣袋里,以帽与之。)太
太在家么? 


刘:(代原脱衣。)在家。要不要请去? 

原:不用。(拿了一本杂志,做到茶几旁边的
一张大椅上。)这几天太太出门没有? 

刘:(将衣帽挂到衣架上。)我们太太不爱出
门。 


\newpage

原:(随便的翻阅杂志。)你怎么知道? 

刘:(倒茶。)来了两个月,一共才出了三次门。上一次连雍和宫打鬼都没有去看。(送茶。)您喝茶。嗳,不过也挺好,那么一点地方,那么多的人
,又都是没有受过教育的,一点意思也没有。 

原:(将报纸放在一旁,取了茶杯。)太太不
出门,在家做点甚么? 

刘:(正在拨火炉。)太太?太太的事多得很。早上看了我们收拾屋子,下午看看书,写写字,空
的时候做点活计。 


原:太太还会做活计么? 

刘:您看看这屋里的窗纱和桌布······
 


原:那都是太太自己做的么? 

刘:(擦了一擦手,椅上取了一个腰枕,送到
\newpage

原的手里。)连这上的花儿都是太太自己绣的。 

原:这样的能干?(接了腰枕随便的看了一看
。)这花儿做得好不好? 


刘:我们那儿配知道这样的东西。 


原:客气,客气。 

刘:(取回了腰枕,手指着上面的绣花。)您不要看不起它小,嗳,俞小愈难,又要漂亮,又要脱俗,难就难在这个上面。(将腰枕拍了一拍,放还原
处。整理其余的椅饰及室内的家具。) 


原:(短停顿以后。)太太的脾气好不好? 


刘:很有点规矩。 

原:(误会了他的意思。)啊,没有从前那样
的自由,嗳? 

\newpage

刘:我们做下人的怕的不是做主子的讲规矩·
····· 


原:喔?——怕的是甚么? 


刘:怕的是做主子的没有身份! 


原:喔!——你们太太有没有身份? 


刘:从来没有骂过我们······ 


原:啊! 


刘:讲话非常的客气。 


原:所以你们对她也得客气客气,对不对? 

刘:那是应该的。比方拿穿衣服的一件事来讲
,······ 


\newpage

原:她要你穿大褂儿? 


刘:我向来不着短衣。 


原:喔,对不起,我忘了。 

刘:她最不愿意看见我们穿脏的衣服,前天我穿了一件旧大褂儿,其实不能算脏,太太看见了,她说:“老刘,你这件衣服应该换了。”我说:“太太,那一件洗了还没干。”她说:“你只有这两件么?”我说:“是。”她说:“两件怎么够穿?过天叫裁
缝来替你做几件新的吧。” 


原:叫裁缝来做了没有? 

刘:(指了身上现穿的新长衫。)一齐做了两
件。(取了纸烟缸子。) 


原:啊,这就是你所说的做主子的身份! 


刘:(点了一点头,送烟。)你抽烟。 

\newpage

原:(取了一支烟,刘代他燃了火,原喷了一
口烟。)所以你的新主子比旧主子强得多啊! 

刘:喔,大少爷,您不用那么讲。您知道,我
是不愿意出来的。 

原:不错,不错。(由桌上拿了纸烟缸送到刘
的面前。)抽烟么? 


刘:(拒绝。)喔,大少爷! 


原:怎么?戒了烟么? 


刘:您应当知道,这不是在您的敷上啦。 

原:啊,不错,我不应该拿人家的烟请客。(从自己怀里拿了烟盒子,再送去。)这是我自己的,金马牌子,真正的国货,尝一尝,看味儿好不好?(刘依然拒绝,无意中向门看了一看。)喔,不要紧,你们老爷和太太都是有身份的主子,客人请你抽烟,

\newpage
绝不会见怪的。 

刘:(接了一支烟。)谢谢您。(自己燃了火,吸烟,这一支烟,引起了两人旧日的感情,所以以下的谈话,愈加的亲密了。)大少爷,您以前认识不
认识我们的太太? 


原:不认识。为甚么问我? 

刘:(犹豫。)因为——我老想同您说,总没
有空儿。(走近。)我觉得我们太太很有点奇怪。 


原:有什么奇怪? 


刘:我对您说了,您可不要对老爷太太讲。 


原:你说吧。 


刘:我们太太······ 


原:太太怎么样? 

\newpage


刘:是老妈子说的。 


原:老妈子说甚么? 

刘:(走近原的身旁低了声。)自从太太来了
之后,我们老爷还没有在太太房里歇过。 


原:喔,傻东西! 


刘:傻东西?您不信么? 


原:我?我很信! 


刘:您不觉得奇怪? 

原:我也觉得很奇怪,不过这不是你们太太的奇怪,这是老爷的奇怪。你们老爷是怎样的一个人你
知道不知道? 


刘:不知道。 

\newpage

原:啊,你们老爷是一个诗人。你知道不知道怎样叫做诗人?啊,你不知道。一个诗人是:人家看不见的东西,他看得见;人家看得见的东西,他看不见;人家想不到的东西,他想得到;人家想得到的东西,他想不到;人家做得出的事,他做不出;人家做
不出的事,他做得出。 


刘:(不信。)太太每晚睡觉都把门锁上。 

原:那算得甚么?你们老爷袋里有把钥匙!(
刘摇头。) 


刘:厨子说······ 


原:厨子说甚么? 

刘:(低了声。)厨子说我们太太一定是个·
·····(走到原的耳边,讲了两个字。) 


原:胡说!混账东西,滚出去! 

\newpage

刘:(带笑。)老爷写完了信就来。(将纸烟
的火头捻熄,带了烟走出。) 

(原坐在椅上看杂志,少停,任太太由左门走
进,手里拿了未做完的毛织物。) 


原:啊,任太太。(起立。将书放下。) 


太太:原先生。 


原:这几天好吧? 

太太:多谢,很好,请坐。(两人同坐下。)原先生什么时候来的,我简直不知道。我们的听差,
真是一点规矩都不知道。(走去代原倒茶。) 


原:规矩?啊,不错。 


太太:怎么? 

原:我们,你们的听差,很知道规矩。我一进
\newpage
来,他就要去请你,我因为恐怕你有事,所以没有要
他。(接了茶。)多谢。 


太太:他就让你一个人坐在这里? 

原:没有没有,老刘在这里陪了我半天,我们谈得非常的有趣。我正在这里想,觉得中国真是什么人才都缺乏,像他这样的一个听差的,就找不出第二
个来。 


太太:这是你的成绩啊。 


原:怎么是我的成绩? 

太太:因为他是你教出来的,你是他的旧主人
。 

原:我教出来的?我从来就没有教过他,什么事都是他教我。从我七、八岁在书房里上学的时候,他就是我的顾问。记得有一天,先生出了一个对子,叫:“笼中鸟”,要我对。我想来想去想不出,我就
\newpage
去问他。他叫我对了一个“虎离山”。先生说,字面
虽然不工,意思还不错! 


太太:哈,他甚么都知道。 

原:是的,甚么都知道,不过他的专门知识,是中国的旧戏。所有北京的一班唱戏的,你如果问他,谁是谁的徒弟,谁是谁的亲戚,谁是谭派,谁是汪派,谁的拿手好戏是《三打》,谁的拿手好戏是《三斩》,他可以原原本本的背给你听,还可以包你没有
一点错儿。(任先生由右门走进。)哈啰。 

任:哈啰,对不起。(向任太太。)我不知道
你在这里。(想去倒茶。) 

太太:(起来代任倒茶。)让我来替你倒吧。你一做起文章来,就甚么都不管,客人来了,也不来
招呼。(将茶送给任,并替他整理衣领衣扣。) 


任:谢谢。客人?谁是客人? 

\newpage


太太:原先生已经来了好久,······ 

任:原先生?喔,我们不把他当客人,他自己也从来不把自己当客人;就假定他是一个客人,他也不是我的客人,因为他不是来看我的,他是来看你的


太太:静庵! 

任:是的,他们都是来看你的,他们来的时候,都说是来看我的,但是他们都是来看你的。(向原。)老朋友,赶快的结婚吧,一个人一结了婚,从来
不来看你的朋友,就都来看你了。 

原:不要以为个人有你这样好的运气,不要忘了,有的人,一结了婚,从来不看他的人,就都去看他;还有一种人,一结了婚,从来去看他的人,就都
不去看他了。 

太太:那他最喜欢了。他最讨厌的就是客人。


\newpage

任:素贞! 

刘:(由右门走进。)太太,一个姓胡的请您
接电话。 


太太:姓胡的电话?那一个姓胡的? 

刘:他不肯讲。他说提到姓胡的,您就知道。

太太:(面上现出不安的样子,但顷刻之间,恢复了常态。向刘。)再添一点茶来。(刘取了茶壶走出。任太太取了手工亦随后走出。原与任两人的目光,都不知不觉的跟随了她。直等到她出门之后,两
人同时回过脸来,目光恰恰相遇。) 

原:任太太走路,走得真好,从来不曾看见一
个女人走路,有她这样走得好看的。 

任:(慨叹。)嗳,她的好处多的很。你同她
还没有十分的熟识,等你同她处久了,你才知道。 

原:用得着么?从今晚我走进这间屋子,到现
\newpage

在,我的知识,就已经长进了许多。 

任:(摇头。)世界上有两种女人:一种,只有旁人觉得到她的好处;一种,只有她的男人觉得到她的好处。。只有旁人觉得到她的好处的女人,一百
个人里面,你可以找到十个;只 

有她的男人觉得到她的好处的女人,一百个人
里面,你难找到一个。 

原:文章是自己的好,老婆是人家的好。——
是的,一个人常见了,就看不出她的好处来。 

任:常见了就看不出好处来?没有这么一回事

原:那么,为甚么同是一个人,在订了婚的时候,你总觉得非常有趣;等到结了婚,那味儿就淡了
呢? 

任:是的,同是一个人,不错,但是在订了婚的时候,她只是专门的打扮了给你看。等到她和你结
\newpage
了婚,她只是专门打扮了给你的朋友看。专门打扮了给你的朋友看,本来也是一件很好的事,不过她还要
专门的不打扮给你看。 


原:这正是老婆的好处! 


任:这正是人家老婆的好处! 

刘:(手里拿了茶壶及一张名片走进。将茶壶
放桌上,名片向任送去。)会您的。 

任:(看了名片。)我不认识他。是怎样一个
人? 


刘:是步军统领衙门派来的。 

任:步军统领衙门派来的?他到这里来干甚么
? 

刘:问过他。他说——他说有一件要紧的事,

\newpage
要见你。 


原:是怎样的一个人? 


任:他现在在那里? 


刘:在客厅里。 

任:(向原。)坐一会儿。(不愿意的走出。
) 

刘:(将门关好,仓皇地走到原的面前。)喔
,大少爷,大少爷! 


原:(一惊。)什么事? 

刘:这个人是统领衙门里派来拿人的。门外还
有好几个卫兵! 


原:拿人的,拿什么人? 


\newpage

刘:拿——拿我们的太太! 


原:胡说! 

刘:一点都不胡说。今天是汪大帅老太太的生日,家里有堂会。把北京所有的名角儿都叫去了,只有黄凤卿没有到,大帅生了气,要办他,所以派了人
来拿他。 


原:拿他,拿他,到底拿谁? 


刘:您还不明白么? 


原:明白?不知道你说的是怎么一回事。 

刘:喔,大少爷,那一天他们结了婚,一回到
家,我就看了出来。 

原:(不耐烦。)看了出来,看了甚么出来?


刘:我们的太太,是黄凤卿扮的! 

\newpage


刘:您不信?喔,他那眼,他那嘴。他那笑法
,他那走路! 


原:是那儿听来的这些瞎话? 

刘:喔,他们先到他的家里,拿了他的伙计,问他老板在那儿,他说,生病了,问他在那儿病的,其初他还不肯说,后来——后来他们要枪毙他,他方
才说了出来。 


原:说了甚么出来? 


刘:说他老板在我们家里。 


原:吁! 

刘:告诉您,一定是他。您想,自从老爷结了
婚,他就没有唱过戏。 

原:(好笑。)弄清楚了,谁没有唱过戏?黄
\newpage

凤卿没有唱过戏,还是你们太太没有唱过戏? 

刘:(也不耐烦。)喔,他们两个人,就是一
个人。 

任:(手里拿了一张字条,由右门走进。)荒谬绝伦的事!说黄凤卿在我家里养病。看你懂不懂这
上面说的是什么话。 

原:(读字条。刘在旁边静听。)“汪大帅把我押在班房,如不把您交出,立刻就要枪毙我,这不是闹玩意儿,我上有一个老娘,下有五个小的,请您可怜我这条小命,见字即来,为要为要。胡升扣头。
” 


任:这是怎么回事? 

原:(向刘看了一看,刘正在倒茶,会意走出。)听说今天是汪大帅老太太的生日,把北京所有的名角儿都叫了去唱戏。现在什么人都到了,就只黄凤

\newpage
卿没有到。现在他们要拿他。 


任:拿他,为甚么拿到我这里来? 

原:那我可不知道。(慢慢的。)也许——任
太太长得有点像黄凤卿······ 


任:岂有此理! 

原:不要忙,有人还说任太太就是黄凤卿扮的
呢! 


任:混账。 

原:说句老实话,任太太的样子,倒实在有点像黄凤卿,——嗳,简直竟可以说很像很想。如果任
太太不是一个女人,······ 

任:(恐怖起来。)甚么,你说她不是个女人

原:(急辩。)我没有说。我说如果任太太不是一个女人,连我也都可以相信。现在我们知道黄凤
\newpage

卿是个男人,任太太是个女人,所以······ 


任:(气恼。)你怎么知道她是一个女人? 

原:(意料所不及。)我怎么知道她是一个女
人!难道她——她不是一个女人么? 


任:(怒。)我怎么知道? 

原:你怎么知道!!!喔,天呀!(跳起。)
结了婚两个月,不知道······ 


(任太太走进,任两手抱头,坐椅上。) 

太太:(由左门走进,满面的悲愁。刘随其后。向刘。)叫车夫把车子拉出去,我即刻就要出门。


刘:(犹如预知。)是。(由右门走出。) 

太太:(走到任的面前,悲伤的。)静庵。我有一个女朋友病了,刚才来电话要我就去看她去,·
\newpage



太太:······病得非常的厉害。医生说,如果今天晚上要不发生变故,或者有几分希望,不
然,恐怕有点危险。 


原:他们要把他枪毙,是不是? 


太太:甚么?我不懂你的意思。 

原:这不是我的意思,这是那字条儿上的意思
。(手指桌上的字条。) 

太太:字条儿?(取了字条,读了一遍,其初略有为难之色,但立刻转为镇静,露出笑容。)啊。


任:(跳起,粗鲁的把她拉住。)你是谁? 


太太:我?我是你的太太。 

原:不错,问题是:是个男太太,还是个女太
\newpage

太? 

任:(将她震摇。)你到底是谁?你到底是谁

太太:(撒娇。)怎么十个男人,就有九个是
野蛮的。你们就都会欺负女人。 


原:喔,女人! 

太太:女人,是的,一个纯粹的女人,一个理
想的女人。 


任:我问你,为甚么走到这里来? 

太太:(愤慨。)为甚么走到这里来?好像害了你似的!我来了两个月,把你的屋子弄整齐了,把你的起居饮食弄舒服了,把你的头发剪短了,把你的衣服刷新了;请问,我有甚么对不住你的地方?从来新婚夫妻没有享过的幸福,你享尽了;从来男子没有享过的女子的爱情,你享足了。不相信,等你再结婚,一次,两次,十次,一百次,那时你就要想念到你
\newpage
的第一个老婆;因为她们只能恭维你,伺候你,服从你,倚赖你,怕你,怨你,悲你,痛你,哭你,殉你
,她们永远不会像我这样的爱你。 

任:(自嘲。)为甚么把这样大的幸福,加在
我的身上来? 

太太:坐下来,让我讲给你听。(三人同坐下。)有一次,北京的“文人才子”,在中央公园,开了一个辩论大会,讨论一个重大的问题,就是:“中国的旧戏有无男女合演的必要。”那时间,赞成的也有,反对的也有。正当辩论紧急的时候,忽然有一个人站了起来,头上的头发,约有五寸来长,脚上的皮鞋,至少有一只是破的,身上的大衣,最多也就剩了
一个扣子,······ 


原:静庵,那恰恰是你。 

太太:他说:“我不承认中国的旧戏有男女合演的必要。反对的人,无非是说,男人表演男性,女人表演女性,总要比男人表演女性,女人表演男性,
\newpage
格外的合情合理。这种见解是非常的高明。可惜的是他们那话,缺少了点根据。他们先就承认中国旧戏里面,只有两种怪物,——是的,两张怪物——一种是张了口大喉咙嚷的,一种是逼着口尖喉咙叫的;一种是把头发卷在脑袋后面的,一种是把它挂在鼻子底下;一种走的是中国的“八”字,一种走的是阿拉伯的“8”字。事实既然是如此,我不知道男女合演的必要在那里?”他说完这几句话,赞成的,反对的,鼓掌喝彩,全场一致。因此现在一班走阿拉伯“8”字的人,都保全他们的饭碗。(少顿。)那时会场的一个基角儿里,坐着一个美丽无比的妇人,头上带了一顶帽子,身上穿的一件旗袍,就连她也不得不佩服他
的聪明。 


原:静庵,那就是她。 


任:现在你来报复我? 

太太:喔,不是,我不是来报复你的,我是来报答你的。你说我是一个怪物,你知道《雷峰塔》的故事,现在你就是许仙,我就是白娘娘。(走到任的
\newpage
身旁,扶其肩,很亲密的。)静庵,这两个月,我们过的是怎样一种生活!我从来不曾有这样的美丽过,也不曾有这样的丰韵。你,看看你!你的灵魂,从来不曾有这样的清醒过,你的心房,从来不曾有这样的颤动过,你的感觉,从来不曾有这样的锐敏过。两个月的工夫,你写了十万字,把我的手都抄麻木了,到现在我还觉得酸痛。(无意之间,用左手抚摩她右手的手腕。)你应当怎样的谢我才是?(任执其手腕吻
之,忘却一切。) 

刘:(由右门走进。)太太,车好了。(退出
。) 

任:(犹如由梦中惊觉。)甚么?(任太太将
手缩回,向们走去,任厉声的问。)那里去? 

太太:(极平常的。)到金华胡同汪宅里唱戏
去。 

任:(立起。)唱戏去。(半请求办命令的)

\newpage
不要去! 


太太:不要去?为甚么不要去? 


任:那不是你去的地方。 

太太:那没有法子,我们有行业的人,就不能由我们自己挑选主顾呀。况且我己经答应了他们。(由右门走出。任复坐下,神气颇丧。二人片刻无语。


原:静庵。 


任:(抬头向之。)甚么? 


原:她不回来了。 


任:怎么,不回来? 

原:一定不回来。——可惜得很!(任即刻奔出。原亦立起,在屋里走了一两转,脸上现出笑容,但是他脑里想到的事情,只有他自己知道。他忽然将叫人的电铃压了一下,自己戴上帽子,着了大衣。)
\newpage


刘:(由右门走进。进来之后,向四面看了一看,偷偷的走到原的面前。)喔,大少爷,老爷和—
—和太太闹翻了没有? 

原:(取了一支纸烟,燃了火。)教我的车夫
把灯点了。 

刘:是。(不敢再问。代原开了门,随原走出,将门关好,数秒钟后,任走进,神情如旧,进来之后,即在火炉旁边的一张大椅上坐了。)(少停,任太太由左门走进,手里拿了一个长形的首饰盒子,两本账簿,一根颈练,练上系了一个象牙的坠子,一进
门,即不停的讲话。) 

太太:静庵,我可以不可以把这个象牙蝴蝶带去?这是我最爱的一件东西,恰好配我的那件天青色的衣服。(说话时,任迎面走去,两人同坐在桌旁小
椅上。)让我带去做纪念,好不好? 


\newpage

任:素贞!(手抚其腕。) 


太太:可以不可以带去?(取练在手。) 


任:这里所有的东西,都是你的。 

太太:替我带上。(任代她把练子套在颈上。她把长方形的合资打开了,拿出一本银行存款的簿子来。)这本簿子交给你。(将簿子打开。)这上面的存款,原来是三千块,我来了之后,你取了两次钱,第一次是十二月十八,取了三百块,第二次是一月二十,取了二百,现在净存二千五百块。(将簿子送到任的面前,任将它向旁边一推,正要开口,她已经接续不停的讲了下去。)这是我们的另用帐。(翻开另用帐。)这一个月的三百块钱,现在只剩了八十块,不过房租已经付了,电话钱也已经拿了去,还有新叫
来的两吨红煤,还没有动用得到。······ 


任:素贞,······ 

太太:是的,这是厨子和老妈子上工的日期。厨子的工钱,是八块一月,上月十九上的工。老妈子
\newpage
是三块钱一月,做半月,预支一月的工钱。这个月应该到二十四号支工钱,现在还么有到。我走了要是她还愿意在这里伺候你,你可以照旧的给她工钱;如果她不愿,那末就把这一个月工钱付给她。·····
· 


太太:······可是这个厨子,我劝你换了他。每天开三十吊钱的账,还是没有菜吃。一个星期用了我们五斤酱油!我老早就想回了他。(将账簿送到任的面前,任照样的向旁边一推。任太太周围看了一看。)最要紧的,是这个屋子,不要让他们弄糟蹋了,不要忘记教他们每天照样的打扫,椅垫桌布,一个星期换一次,地板两天洗一次,星期五擦玻璃。
······ 


任:素贞!听我说······ 

太太:啊,现在甚么事都弄清楚了。(看了一看表。)我们只剩了一刻钟的工夫,让我们坐在一张舒服的椅子上去,亲亲蜜蜜地谈一谈。(拉了任的手
\newpage

,两人同坐到火炉旁边的沙发上。) 


任:(执了她的手。)素贞,——不要走。 


太太:不要走?我不懂你的意思。 

任:不懂我的意思?你不是讲过么?这两个月,我们过得非常的快乐,为甚么不让我们继续的过下
去?唱完了戏回来,我在这里等你。 

太太:你在这里等我?我要三点钟才出台,你
能够等我么? 

任:我可以看书,我可以写东西,我可以抽烟

太太:喔,这都是无意识的话!让我们谈一点
正经的事。你那本书,打算甚么时候出版? 


任:出版?你走了,我立刻把它烧了。 

太太:烧了!无意识!这本书是一件无价之宝
\newpage
,——一件双倍的无价之宝,——因为这本书,字里
是你的灵魂,纸上是我的墨迹。 


任:你走了,我一个字也写不出了。 

太太:不对,不对,我如果不走,你就快要一个字也写不出了。现在我走了你有了新的情感,新的悲哀,一个有天才的人,有了新的情感,新的悲哀,
不怕没有新的文章。 

任:从今天起,我就不能再看见你的面貌,听
见你的声音? 

太太:我住在狮子胡同九号,那是我的私宅,
你什么时候都可以到我家里看我去。 

任:到你的家里去,那不是你的家,你的家在
这里,这是你的家,这是我们的家。 

太太:不错,这是我们的家,应当时常的回来

\newpage
看看。 


任:最少一个星期两次。 


太太:可以,可以。可是······ 


任:怎么? 


太太:可是我不能穿这样的衣服。 

任:不能穿这样的衣服?(了解了她的意思,
如剑穿胸。)喔喔!不要来,不要来。 



任:从今天起我不认识你,你也不认识我。 

太太:怎么!不准我到自己的家里来?不准我来看一看我自己的睡房书房?不准我到我自己布置的客厅里坐一坐?不准我在我自己绣的腰枕上靠一靠?
你能够这样的无情无义么? 

任:我的妻子今晚去世,从今天起我是一个鳏
\newpage

夫。 


太太:不续弦么? 


任:续弦! 


太太:讨个填房? 

任:填房!啊,填房,填房,一个房空了,是要填的,是可以填的,但是谁能够填这个空了的心!

太太:喔,不要这样的伤心,我还没有死。虽说这是你的好意,但是一个人都是不愿意死的,你知道!(看了一看表。)啊,现在我只有十分钟的生命
。我还有一个要紧的遗嘱,没有吩咐你。 


任:什么事? 

太太:就是填房的一件事。你说你的心不容易填,我告诉你,我的房,也是不容易填的。喔,那是怎样的一个睡房!床铺,被褥,枕头,幔帐,衣柜,
\newpage
衣橱,梳装台,洗面架,肥皂盒子,香水瓶子,地上的地毯,壁上的字画,各样东西,配合得何等的完美!没有一件东西,没有我的个性刻在上面。现在凡是我所用过的东西,我都留给你,不过你要答应我一件事,你每天要教他们照旧的去拂拭灰土,不要移动它们的地位。最要紧的是将来——将来新太太进门的时候,你先把我所有的东西,一齐烧了,然后再让她进
来。 

任:喔,无意识!我再也不结婚。你走了之后,我每天亲自去打扫,亲自去收拾,包你件件东西都
和你在的时候一样。 

太太:亲爱的丈夫!(吻他的发,又看了一回
表。)还有五分钟。 

任:啊,让我在你怀里睡一睡。我从来没有在一个女人的怀里睡过,——除了小的时候睡在母亲怀里。(躺直了身子睡在她的怀里。)只有五分钟,是不是?喔,不要紧,只要三分钟我就睡着了,——睡着了,和睡在母亲的怀里一样,什么事都不知道了。
\newpage

(静睡不动。) 

太太:可怜的小孩子!(代他理了一回发,又看了一回表,从放在身旁的一个钱包里,拿出一面小镜,一张小梳,一手执镜照面,一手用梳自理其发。
(闭幕)

\end{document}
