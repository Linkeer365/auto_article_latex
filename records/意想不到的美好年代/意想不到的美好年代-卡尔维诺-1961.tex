\documentclass{article}
\usepackage[utf8]{inputenc}
\usepackage{ctex}

\title{意想不到的美好年代\footnote{Click to View:\url{https://web.archive.org/web/20220729152935/http://libgen.is/book/index.php?md5=8C7F488ACD514AE29C2450B88A76ECFC}}}
\author{卡尔维诺}
\date{1961-07}

% \setCJKmainfont[BoldFont = Noto Sans CJK SC]{Noto Serif CJK SC}
% \setCJKsansfont{Noto Sans CJK SC}
% \setCJKfamilyfont{zhsong}{Noto Serif CJK SC}
% \setCJKfamilyfont{zhhei}{Noto Sans CJK SC}
% \setlength\parindent{0pt}

\begin{document}
\CJKfamily{zhkai}

\maketitle


\Large

十五年前,我们预料到了一切,只有一件事情例外,那就是世界会进入一个叫作“美好年代”的阶段。现在,我们正值这个阶段。经济繁荣就像是一个安乐乡,每个人都在关注自身的利益。昨天使政府官员和知识分子的想法和行为非常活跃的(不论是好还是坏)那种决不妥协的理想,如今更多是被可能主义和功利主义的言谈和行为所取代。无论是公开还是隐藏,所有人都确信,这个安乐乡会持续很长时间,甚至(这就是每个“美好年代”的典型特点)永远不会结束。的确,冷战尚未结束,某些区域也还在发生血腥的战争。然而,那些躲藏起来的人却只是注视着那些战争,如同它们是一个晴朗的夏日突然下起的冰雹。的确,先进国家和落后国家之间的不平衡正在加剧,然而,庆祝活动门外饥寒交迫的人和群的画面,

\newpage
呈现出的正是“美好年代”的典型景象。 

这才是我们身上真正的改变,而不是想法或者价值,因为它们没有理由改变(生命已经如此短暂,假如说一个人要去改变自己的想法,就会打破他的生存所能拥有的些微延续性和意义;我们的思考最好是永远朝着同一个方向,假如这个方向是错误的,那么,早晚会有其他人以更加正确的方法思考,而且会使你的错误变得“有用”)。原因在于,从前我们把生命看作某种紧张的、战斗的、艰辛的东西,我们要在其中练习选择好与坏的能力,锻炼我们坚强的神经和理智,以及使用揭露性的讽刺;但是,现在我们把生命看作一场表演,它在大体上是可以预测和令人安心的,我们希望享受它的所有细节,享受某种舒适、富裕和稳定的东西,并且在其中发泄紧张、焦虑和怒火。从前,我们觉得时间如此飞快地流逝,而生活在其中,我们感到如此平静,从来不会想到个人的死亡,我们唯一关心的是自己生存的空间在世界历史中占据多大的比例。现在,在我们身体之外,时间脉搏跳动的速度好像更加缓慢且难以察觉。控制着我们的是个人忙碌的生活和因此产生的不满,是想到青年岁月突然流逝,以及我们以前能做却没有做,和以后也不会
\newpage

去做的事情。 

我所讲的是资本主义世界,但我相信,这番话对社会主义世界也同样适用,至少是接受了(或许也是为了我们而决定和确认了)历史“变速”(假如借用一下报纸上的套话,就是与“中国路线”背道而驰的“赫鲁晓夫”路线)的那个部分。统治我们这个社会的是实际消费的快感,而在社会主义社会中,与此相对应的是可能消费的快感,而且被认为是可能达到的、几乎关键性的目标,也就是终于能够梦想消费却不必有负罪感的那种快感(这是最好的状况,因为在道德上,它同时拯救了享乐主义和苦行主义;我们对强迫性的享乐主义感到恶心,而苦行主义仅仅被理解
为受虐狂式的激情)。 

我之前说,每个“美好年代”都加剧了工业国家和落后国家之间的矛盾,因为它们花了很大力气在落后国家开辟殖民地。第一个“美好年代”正是欧洲殖民扩张的顶峰时期,而如今这个“美好年代”却标志着一次相反的运动:欧洲国家以各种可能的方式从殖民国家退出,在旧日的殖民地上,建立起了新的民
\newpage
族国家。世界政治的范围好像突然增加了十倍之多。两个彼此对立阵营的形成使世界出现了两极化,政治的多样性也仿佛被最终压制。如今,这种多样性又在“第三世界国家”重新出现。这些国家继承了殖民主义破碎的边界,重又开始了外交游戏,谁拥有更多的
外交关系,谁就能够得到更快的发展。 

最靠近欧洲工业整体的落后国家,通过大量移民来逃离坏死病,而这种移民就好像是受到一种生物力量的推动。从中世纪开始,地中海沿岸从来没有出现过如此广泛和无法控制的人口迁徙:西班牙人瞄准了瑞士法语区,北非的阿拉伯人要移民到普罗旺斯和巴黎,卡拉布里亚人要去利古里亚,西西里人要到都灵,米兰和卢切尔纳湖地区构成了三角地带,土耳其人要去法兰克福和慕尼黑。当19世纪资产阶级的民族概念中,不断加入法国将军和意大利统一一百周年庆典的组织者们浮夸的言辞时,欧洲人种地图在过去的十年中也发生了深刻的变化。几乎所有大城市的劳动阶级(无论是形式上的,还是历史上的)都与之前完全不同:语言传统、反抗的方式,在一年之间发生了完全的改变。在诞生之初,欧洲工人运动设想工业
\newpage
无产阶级的发展是连续性的,而且会始终如一地成长。然而,在对话进行到一半的时候,却发现面前是一
群彼此各异而又无法理解的对话者。 

同样是在工人运动中,发生改变的更多不是想法或者价值,而是领导者和工人之间方向与自发性(由此才会经常因为意想不到的活力和战斗力的实例而产生惊喜)之间、愿望与自然之间,还有计划和预期之间的关系。在这里,同样发生了“变速”:首先是对未来的期待,也就意味着在政治和工会斗争中进行一种不同的能量投入。政治。工会或者激进主义的官员也加入生产。由于工厂的大门在面前关闭,他们就一马当先尝试各种经济的举措,通常可以获得成功,因为他们比其他人更加陪明、积极和明智,而且具有作为人的那种特别的态度,也就是在不同的场合表现出最好的自己:无论是牺牲,是折磨,还是在经济繁荣时期,情况都是一样。工人阶级远离政治活动与知识分子远离政治运动的情况是不同的:对于工人世界来说,假如变身为一个小业主,那是因为他自然而然地适应了周围环境,通常不意味着理想的危机;而知识分子却认为应该将一个简单的社会学或者工作领域
\newpage

的变化与思想危机对应起来。 

于是就产生了一种意识形态上的态度,它倾向于将现在欧洲工业的经济繁荣看作一种自然而稳定的条件,要以这种优先条件为衡量标准,去评价每件事情。(假如说在修正主义的思想当中,感党这种态度是不言而喻的,那么在任何尽管是理智。有意义和能够接受的建议中,却都会感觉到缺少了某种东西,缺少了那种折磨的感觉,也就是在一个令人心碎的世界中思考的感觉,然而,它又是我们在思考的产物当中
能够识别出来的唯一真理的印记。) 

与此同时,每一个“美好年代”所在的时期,也总是革命极端主义和意识形态上的虚无主义的时期:对于以幻想和不公正形式出现的繁荣的拒绝,导致了对于可能通过它得到的(哪怕是暂时和有限的)任何享受和利益的拒绝。革命的苦行主义不再是一种可能和功能性的需要,而是由放弃和纯洁带来的快感,所以是有私心的热情,是身心上的先决条件,它预先确定我们的所有选择,也使得评价变得不再清晰。(就像在我们这里一样,假如一个人像“中国人”那样
\newpage
思考,那么或许他在绝对历史逻辑的层面没有出错。然而,健康人的道德观念是不会以任何代价保留自己的纯洁,而是冒着失去道德的危险,在实践的腐蚀中获胜,是试图以最小的放弃和痛苦为代价,尽可能达到目标,向一个仍然充满了未知数的未来前进,享受最好的东西,同时每前进一步,又都要面对更精糕的
情况。) 

前面的一个“美好年代”从1870年持续到差不多1914年,几乎经历了五十年的时间。当时,人们并不知道等待着他们的是萨拉热窝事件,而是相信了芭蕾舞剧《细刨花》呈现在他们面前的景象。不过,现在一切都已经显而易见。我们已经知道了萨拉热窝事件。咱们来算一算,假如我们的“美好年代”也能持续这么久,而且鉴于人类的进步,或许还可以更久一点,假如能够将萨拉热窝事件推迟到我们可能的寿终正寝之后,而且但愿让它遥远到连我们的儿孙都不会看见;但愿我们无须找到一个持续性的解决方法,也可以不费一枪一弹,就从一个分离和异化的
世界,过渡到一个完整而普遍的社会主义世界。 

\newpage

然而,事情不会这样。总有可能发生更糟糕的状况。我们没有任何办法预测这种不确定的平衡,繁荣和乐观主义还能持续几个小时、几个月、几个五年,或者五十年,甚至更多。萨拉热窝事件在任何时刻都可能发生,甚至可能是明天。我们不知道那会是怎样一幅景象,会是原子战(不过,或许超出预期的可怕事情并不会发生),或者是另外一种。或者是从未灭绝的古老妖怪的模样,又或者是一种我们不认识的
新形式。 

我们所知道的是,我们应该把“美好年代”公民的身份当作一种暂时性的状态,尽管我们的行动是完全自由和自然的。在我们成长为成年男人的时期经历过的那个种族灭绝和充满威胁的世界,仍然有可能出现,而且可能在任何时刻重新开始。在任何时候,我们都可能重新扮演受害者或者剑子手的角色,而我们早就已经准备就绪。我们并没有发生变化。说到底,我们周围任何重要的东西都没有改变,不论是结构,思想,还是意识。当然,我们如今感觉自己与个人生活乐趣的外在标志之间的联系尤为紧密;但是,当周围的这些标志还是如此稀少的时候,我们把它们看
\newpage
作一种“价值”,而且拒绝将它们当作一种自负来蔑视。就像今天,在这个名不副实的安乐乡的快乐当中,我们明白,并没有任何东西是属于我们的。一切都像纸糊的城堡,吹一口气就会倒塌。唯独有些事情是我们不能被剥夺的,那就是一次次区分正确行为和错误行为,为世界的新景象感到惊奇,以及通过我们自
身来体现未来对于我们的怜悯与嘲笑。 



\end{document}
