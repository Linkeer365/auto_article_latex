\documentclass{article}
\usepackage[utf8]{inputenc}
\usepackage{ctex}

\title{撕裂星空\footnote{Click to View:\url{https://web.archive.org/web/20230505213238/http://www.zh61wx.com/Article/Class2/14765.html}}}
\author{郑重}
\date{}

% \setCJKmainfont[BoldFont = Noto Sans CJK SC]{Noto Serif CJK SC}
% \setCJKsansfont{Noto Sans CJK SC}
% \setCJKfamilyfont{zhsong}{Noto Serif CJK SC}
% \setCJKfamilyfont{zhhei}{Noto Sans CJK SC}
% \setlength\parindent{0pt}

\begin{document}
\CJKfamily{zhkai}

\maketitle


\Large

地点:银河系猎户座E星P城。时间:猎户
纪年10年。星球振动频率:C级。 

瞪着一双充满血丝的眼睛,E星球地震及气象监测中心主任老K,正目不转睛地看着电脑视屏墙显示的曲线变化。忽然坐标点上那道曲线又似幽灵般地出现了——端坐在沙发上的他一跃而起,把面前小茶几撞了个后滚翻,上面摆放着的一碗未吃完的方便面和半杯浓咖啡,以一个优美的自由落体曲线飞向地板
,瞬间得到了粉身碎骨的结局。 

“地表塌陷”!怎么又是“地表塌陷”……这真是一个晴天霹雳,差一点儿把他全身的细胞劈碎!刹那间,老K感到全身冰寒刺骨,仿佛生命的灵魂要

\newpage
离他远去了。 

这两天,老K的精神世界里隐隐觉得有一个不安的搅动,而且越来越强烈。他觉得要发生不同寻常的大事,当时他估计E星可能要有一次重大的火山爆发,会影响到星球表面。但是,他万万想不到的是再一次发生了人们最不愿意看到的现象——地面又塌陷
了! 

这几天以老K为首的科学家们简直成了一群热锅上的蚂蚁,被烤得焦头烂额。自从埃德菲火山持续喷发以来,媒体和各界人士对地震监测专家们的谴责和质疑浪潮汹涌而来,上级部门也再三告诫他们必须不惜一切代价,做好灾情监测和预报工作,巨大的精
神压力和生理压力使他们没有一天能睡个安稳觉。 

按说“地表塌陷”如果只是个案,还不至于让
人们心惊肉跳。但是情况越来越不妙了。 

最近一个月来,E星球已经在东西半球相继出现了“地表塌陷”现象,至今已经累计达50多个大大小小的坑洞,令人惊奇的是,这些坑洞都呈现出不
\newpage
规则的圆型或椭圆型,有的坑洞直径竟然达数百米以
上,而且深不见底。 

目击者描述的情景和实地拍摄的情景大都相似,而且惊心动魄。随着轰隆隆的巨响,农村地区的小山和丘陵莫名其妙就消失了,被一个个巨大的坑洞所吞噬;十几座城市的高架桥因而折断,几十辆小汽车坠入坑洞不见踪影;若干个居民小区的数十栋大楼出现倾斜,附近上百万居民赶紧疏散,离开家园……一
时间人心惶惶,老百姓称其为“天坑”。 

老K和志龙等几个首席科学家紧张地交换了意见,经过一番推理、计算与总结,很快分析出来一幅全息激光立体图——这是最近一个月来,E星地表出
现塌陷坑洞的情况分布图! 


“这是最新数据!” 

志龙把手里的一支电子笔“啪”地摔在桌上,环视周围的科技精英们,以一种庄重的口吻道:“各地有关部门送来的报告称,他们认为地陷的原因是和
\newpage

地下水开采过度有关,你们怎么看?” 

同事小D分析说,从地质理论上看,如果地下水开采过度,就会导致地层以下出现真空区,区内压力将会失衡,真空区就会引起顶部地表坍塌;再或者是地下水平衡被打破,水量猛增,导致地下岩层中泥沙大量流失,构成地下空洞,外加地表压力大,土质
松软,形成塌陷灾害。 

可是,难道最近一个时期E星坑洞的密集出现
,原因会这么简单吗? 

老K指着坑洞分布图,若有所思地沉声道:“显然,这不仅仅是一个地下水的问题,大家请看,近期出现的‘天坑’集中分布区域主要在E星北方,大体分为两条主要干线,一条是北纬30度,从西蜀沿母亲江流域从西向东一直前出到东海口,一条是北纬
45度从桂西到闽东。” 


“先生们,这说明了什么?” 

\newpage

老K神情严峻,他突然说不下去了,他示意自
己的副手志龙发表意见。 

志龙会意,他目光炯炯,一字一顿道:“众所周知,E星纪年09年7月22日,发生近百年来前所未有的大日食现象,猎户座太阳被E星的M卫星遮蔽,射线强度陡增,而日全食的区域是北纬30度到45度之间,正好覆盖了整个母亲江流域,而这一区
域恰恰就是今日天坑最密集的区域。” 

看到大家都在凝神静听,志龙的语调更加低沉:“这说明了什么?天坑与日食之间有无内在联系呢?这更增加了天坑的神秘,是不是这些巨坑预示着什
么?” 


“你说预示着什么?”同事小D撇撇嘴。 

志龙皱眉道:“大家想必知道,有一部轰动全球的电影叫《E星终结》,影片描绘了猎户预言的世界末日,灾难毁灭了整个人类,不少人觉得这个所谓的预言很可笑,当然这只是一部艺术作品,而我们科
\newpage
学家也是不信这些的!不过近一个时期以来世界各地的灾难开始猛增,除了一系列地震之外,火山、海啸、龙卷风发作周期开始越来越短,而气候也开始高密度异常,这不由让我联想到,会不会艺术家们笔下的猎户预言是真的,会不会一场更大的灾难正在悄悄逼
近整个人类呢?” 

“我们都是科学家,没功夫讨论预言和你的联想”,老K挥挥手,打断了志龙的遐思,“说说你的
科学判断!” 

志龙深吸了一口气,下定决心道:“其实我个人认为,最近一个时期E星的深坑密集出现,是与那次前所未有的大日食有关!强大的猎户太阳喷发出了超级太阳风,因为‘太阳风’是太阳表面喷发出的高速带电粒子流,其能量之大,行动之猛,对E星而言
都是全球级的规模,影响是不可预测的。” 

同事小D问道:“你的意思是说猎户太阳发出
的超级射线风暴,影响了E星地质运动及构造?” 

\newpage


“对!” 

志龙神情有些激动,他情不自禁地挥挥手,大声道:“E星地核为什么能运转?从而推动整个星球运转?当然与太阳的影响密不可分!这不仅仅是引力问题,还有更为深远和复杂的影响,这是我们单从科学的立场所无法理解的——因为猎户座太阳系就是一
个整体,E星本身就是猎户太阳的一部分!” 

“太阳是太阳,E星是E星,这是最基本的常识,我们E星怎么会是太阳的一部分呢?”小D冷笑了,他讥讽道:“我说志龙,你自从失去女儿后,你感情破碎了,就有些神智恍惚了——我感到你连最起
码的判断力都丧失了。” 

提起志龙的女儿,志龙的心猛然间似乎被利刃刺了一下,他的思绪就像洪水倒流,时间瞬间倒退到了两个星期之前——这件事使他陷入了一个逻辑世界
崩溃的深渊之中。 

那是两个星期之前的一个夜晚,志龙被噩梦惊
\newpage
醒。他两只充满血丝的眼睛死死盯住桌子上那幅油画,这是女儿芸的杰作。画作反映的是一个两手抱头,充满绝望的人正在尖叫,他的背后是黑色奔涌的河流和血红色的天空,尤其是那张因恐惧和痛苦而格外扭曲的脸,乍看像一个骷髅那样怵目惊心。更让他匪夷
所思的是,芸把这幅画命名为“尖叫”。 

两个多星期前,雷岛埃菲德火山持续喷发。作为一个地质与气象科学家,他还从来没有这样焦躁过——近来E星地质活动太反常了。雷岛埃菲德火山持续喷发的壮观场景,盘旋在他脑海中挥之不去。万丈火焰裹挟着厚厚的火山灰升腾到8千米高空,形成了蘑菇般大大小小的火山云,在星球大气环流的作用下以排山倒海之势,正高速飘向希尼罗大陆。顶头上司老 K的电话都要打爆了,让他无论如何都要结束休
假,马上返回气象中心工作。 

不知怎的,志龙那天晚上的心情很糟糕,他有些生硬地拒绝了老K让他提前结束休假立即返岗的指示。他感到心烦意乱。其实志龙并不是一个对工作缺乏责任心的人,正相反,这几年他经过实地考察,在
\newpage

大气环流研究领域撰写出的论文就有十多篇。 

志龙的家里出大事了——他唯一的女儿芸丢了,芸是十七岁的大姑娘了,一个活生生的人,说不见就不见了,警方认为是不明原因失踪。在他看来,这个结论实在荒唐之极。他是科学家,从来都讲究科学实证,没有证据的结论属于荒诞,物质是不灭的,再说刚失踪两天,难道就找不着了?活要见人死要见尸
,怎么能没有证据呢? 

自从他十年前离异之后,女儿是他唯一的精神寄托和心灵支柱。如今他的精神支柱突然凭空消失了
,他怎能不上火不着急? 

想起自己的宝贝女儿,志龙又心里感到一阵揪
心刺痛。 

芸是个美人胚子,天生聪颖,个子高高的,体质也极佳,酷爱绘画和摄影。但她对学习却不大上心,尤其不爱上学听课,经常利用上学时间独自一人去野外郊游。志龙作为家长,起码也是管教不严,为这
\newpage
事他没少挨学校老师的数落,有好几次芸的班主任还特意赶到气象中心找他,商谈怎样教育孩子的问题。
 

说实在的,志龙一开始还没怎么把老师的话当回事。教育工作者总爱小题大做,危言耸听。你老师说我的女儿不爱学习,那怎么解释芸每次考试时,每
门功课都是A的成绩呢? 

关于这一点,芸的班主任也直摇头,这位女士无法解释,更无法理解。是芸智商高吗?还是她每次
考试取得的好成绩都是命运之神对她的关爱呢? 

芸在桌子上给父亲留下了一张纸条,大意是要和几个朋友去看火山喷发,拍几张一辈子难得的大自然壮观图景,她会注意安全,让志龙不必挂念——并叮嘱他一定看看她前两天画的那幅画。志龙一直觉得芸从小在性格和思想上就有些与众不同,有些怪异,就拿她的绘画来说,反映的符号抽象而怪诞,让人觉得很模糊、很神秘,不大清楚要表现什么具体而明确

\newpage
的主题。 

现在,这个孩子又一去不归,她究竟是怎么了
? 

但是正应了那句古话“人有旦夕祸福”,从前天晚上开始,芸的手机就打不通了,芸的几个朋友的手机也打不通了。志龙忧心如焚,只好报警。但是在
火山猛烈持续爆发的情况下,警方又能做什么呢? 

“芸,我的女儿,你在哪里呢?”想起自己的
女儿,志龙不禁眼含热泪。 

为了找寻女儿,志龙横下一条心,冒着极大的生命危险,单人独车勇闯了火山区,可是就在他接近火山区时,他的亲身感受真像是经历了一场炼狱般的洗礼。他感到整个世界仿佛静止了,全身的血液突然
凝固了,他被眼前的火山景象彻底震慑了。 

志龙清晰地记得:埃菲德火山顶上,电光狰狞的爆炸闪耀,炫目的强大闪电自天空力劈而下,把黑漆漆的天幕撕裂为数十道大大小小的碎块,地面喷发
\newpage
的熔岩火柱喷泉般直冲九霄,竟然和天空的雷电融为一体,火舌四窜,游走不已,宛如在混沌夜幕中一条条张牙舞爪的金蛇,咧开血盆大口,要吞噬这星球的
一切生灵。 

地下之火和天空之火合二为一,势头更加凶猛,空前浩大,使得埃菲德火山完全成为了烈火与闪电的王国,周围的氧气也变成强大的助燃剂,空气被烧灼得滚烫滚烫,形成了一片巨大无匹的橘红色的不断翻滚的火烧云团——志龙在心灵中似乎听到了大自然
正在发出歇斯底里地尖叫。 

很明显,在这个巨大的死神炼狱之中,志龙热汗淋漓,他感到自己的呼吸非常艰难了,而且视力也越来越模糊了。毫无疑问,火山爆发的结果,就是制造了一个以火山为圆点的恐怖的“小真空地带”,没
有氧气即意味着窒息和死亡的降临。 

志龙的神智似乎也越来越恍惚起来,用尽全力,挣扎着拽开防毒面具,一股浓重而浑浊的热浪,夹杂着刺鼻的氧化硫味扑面而来,封堵了他的呼吸,使
\newpage
他差一点儿完全窒息,恍恍惚惚之间,志龙的眼前浮现出了女儿芸那张柔美的笑脸,耳畔似乎又响起了她那优美而绵长的吟唱:“蓝色的天幕上全是星星,诉说着无尽的思念。有一颗星呵,是我们的家乡E星,古老的文明是我成长的希望。亲爱的爸爸妈妈,我要
在星空独自飞翔,但是前方的路还太长太长……” 

志龙在心里一遍又一遍地唱着,同时。他希望自己千万不要睡过去,他知道,一旦沉睡过去就可能
永远醒不过来了。 

志龙感到自己的意识进入了一个“黑洞”,正
在高速旋转中晕眩。 

志龙恍惚看见自己正穿过一片火海,熊熊烈火烧灼着自己的肉体。他极力挣扎着想奔出这人间炼狱,突然一头浑身是火、长着巨大尖角的怪兽,朝着他吼叫,妄图吞噬他。他竭尽全力与怪兽搏斗,与它死拼,但是怪兽还是抓住了他……志龙似乎听到有人在说话,那声音像是从遥远的天际传来,渺茫又虚幻。他极力集中精神,想寻找那声音的来源,却感到自己
\newpage

仿佛置身在无限的虚无之中。 

也不知过了多久,昏迷之中的志龙感到有水滴在脸上。他想睁开眼睛,可是眼皮好似被粘住了一般;他想抬起手,可是手臂仿佛被千斤巨物压着,休想
移动分毫。 

志龙觉得自己的口中被人塞进了一颗小圆球,一股强烈的薄荷气味散发出来,直冲进他的大脑。他
逐渐清醒过来。 

志龙最先恢复的是听觉。一声声充满亲情的呼唤刺激着志龙的耳膜,他的大脑开始接收信号。它找到了眼晴。经过反复努力之后,眼皮慢慢地张开了,
一片明亮的光线射进脑际,虚无消失了。 

志龙想揉揉眼睛,但是无力移动胳膊。他试着
张了张嘴,僵硬的脸上闪过一丝微笑。 

志龙睁开双眼,看到一束光——这是一束蓝色的柔光。自己正在这束光中,他睁大惊奇的眼睛,原
\newpage
来女儿芸正焦急地坐在他的身旁, 泪如雨下,泪珠
滴在了他的脸上。 


“芸——我的女儿!” 

志龙想喊出来,却发不出声。他觉得这束蓝色的光正在越变越大,笼罩着全身,使他感觉自己的身体轻如鸿毛,如同丧失了实质一般。然而他的心里感到滋生了一股强大的暖流,意识也较平时格外的清醒
。 

志龙忽然看到自己的女儿,浑身沐浴着柔美的
蓝色霞光,笑容满面地站在她的眼前。 

志龙大喜,他突然伸出自己的手,去抓自己女儿的手,然而仿佛接触到的是一层蓝光——无色无味
,透明之极。 

志龙叫道:“芸你究竟怎么样?这是哪里?你
我为什么会在这儿遇见?” 

\newpage

看到自己的父亲清醒了,芸感到如释重负。面对来自志龙心灵内的一连串疑问,芸温柔地笑了:“爸爸!我很好!这是在雷岛八千米高空——埃菲德火
山山口上!” 


“什么?在埃菲德火山的山口上空?” 

志龙完全震惊了,他的大脑嗡地一声,觉得自己的每个细胞差一点爆裂,这怎么可能?他简直不敢相信女儿的话!惊愕之余,志龙仿佛感到这很可能是
一个梦——他正在梦中和女儿相见。 

志龙感到自己的视神经正在受到某种力量的牵引,他向下一看,蓝色的光忽然退去,四周瞬间变得透明。在自己身体的正下方,埃德菲火山正在汹涌澎
湃地喷出大量的熔岩和火山灰气体。 

忽然,志龙的内心如同被重锤猛烈敲击了一下——从八千米高空俯瞰,火山喷发的姿态和情景震撼人心,好似一张因恐惧和绝望而极度扭曲的人脸,怎

\newpage
么这样眼熟呀? 

对了!像是那幅画——正是芸离家前画的那幅
“尖叫”! 

芸完全明白父亲的惊愕之情,她的神态平静如止水,全身又沐浴在柔和的蓝光之中,轻声道:“爸爸!这是真实的画面,凭借E星当代的科技水准是根本无法看到的,怎么说呢,你无法想象,因为你现在处在一个和E星完全不同的世界——这是一个在各方
面都远远超越E星的世界。” 

听到自己女儿的话,志龙觉得他作为一个科学家的思维已经变得越来越迟钝和迷茫了:完全不同的世界?难道这是一种什么暗示?抑或是某种神秘力量
的展现吗? 

志龙感到女儿芸正目不转睛地看着他。芸长长的睫毛下,一双美目晶莹闪亮,透出一种专注,一种宁静;不骄不躁,却洋溢着平稳深邃的热烈;不怨不
怒,却流露出包容一切的博大。 

\newpage

志龙整个身体都颤动不已。他感到这就像是一
种辉煌的仪式,有一种发自生命体本能的挥洒。 

此时此刻,芸的身体被阳光透射,通体透明,脉络清晰如画,如同一个至高境界的生命展示她的内部世界,一尘不染,经络优美。他恍惚间,觉得自己
的女儿好像一个下凡人间的美丽的天使。 

志龙忽然想起了秋日阳光中的白杨树,她与朝阳和谐,与落日也和谐,站立的姿势高雅优秀,美轮美奂,你若细细端详,便可发现那是一种人类无法模
仿的高贵姿态,令人惊羡。 

芸——她丰富灿烂的恰到好处,浑身披满了光的色彩,这是一种庄静,一种高远,一种永恒,这绝非凡间之物,她展现的是一颗神性的灵魂,超越人类
的灵魂,高远而深邃的灵魂。 

“芸……”志龙感到一种前所未有的心灵的震撼,他睁大了吃惊的眼睛,挣扎着叫道,“你,你还

\newpage
活着吗?怎么变成这样了?” 

“怎么说呢,爸爸,你看我这不是好好的!”芸露出了天使般的微笑,“只不过,你看到的已经不
是原来的我了…” 


志龙疑惑道:“你不是原来的你——?” 

芸微笑着说:“从严格意义上说,我并不是E星人类之一员——我们生来就把整个银河系作为自己的家园,我们的最大成就和最高境界是通过对真理的探索,追求获得和宇宙对称的灵魂,由此,心胸变得辽阔而平和,从而对这个无限存在永恒包裹着我们的伟大宇宙,献上发自内心的由衷敬意,因此,我们将
来到E星球视为自己的责任……” 

“将来到E星视为自己的责任?你……”志龙
简直不敢相信自己的女儿的话。 

芸微笑道:“在这有形态的宇宙内,一切生命都依靠形体而生存,就算微若空气,也是有形体的,只是人类肉眼凡胎,无法看到而已,而真正的符合宇
\newpage
宙真理至高境界的生命,是无需依赖任何形态而存在的,它是一种伟大而庄严的精神能量体,当我按照伟大造物主的指引来到E星球时,我选择了你——做我的父亲,当然还有我的母亲。因为你们和常人不同,你们的情感和精神产生了一个特殊反应,导致了DNA突变,这种反应是我们所需要的发射出来的一种波状物质,新的救生信息实际上就藏在E星的潜意识中,换句话说,选择了你们,就等于选择了能指引 E
星球进化的精神力量。” 

芸睁着一双清澈、成熟、智慧的眼睛,轻声道:“近百年来,E星星球的震动力一直在稳步上升,这是E星生命体的自我蜕变,自我升级,也是E星人
类进化的必经阶段。” 

志龙在震惊之余,他逐渐复了一个科学家的理智,问道:“星球的自我升级?也就说你的那幅画意
味着——你早就知道火山的喷发?” 

芸沉吟似水,平静道:“是的!埃德菲火山的持续喷发,就是E星自我震动的结果,那幅画只是一
\newpage
个启示,我相信自己的直觉,你必将来到埃德菲火山!爸爸,你毕生致力研究的E星,她不是死的,她本身就是生命,是一种星球型宇宙生命体——你们人类太渺小了,生活在上面却全然不知!在不久的将来,E星这种生命形式将面临一次伟大的空间转换,从本维度空间将升级到高维度空间,这是宇宙发展的必然规律,当然转换的过程虽然有些痛苦,但是转换一旦结束,就是凤凰涅槃般的新生!E星人类将进化为新的种族,并真正成为文明复兴后猎户座太阳系的忠实
守护者。” 

升级——转换——新维度空间——新的星际种族……这一连串词语搞得志龙神智有些恍惚——他现在从来不敢相信自己的感觉是真实的,他觉得自己好像是身处梦境,但又强烈地感到自己的中枢神经从来就没有这样清晰过……后来,志龙又感到一阵晕厥,他是被救援队从灾区抬到医抢救过来的。医生说他被雷电击中后居然能够生还,简直就是一个天大的奇迹

志龙被告知,他在医院整整发了三天高烧,满嘴说胡话——不过,志龙醒来后,他自己坚决不相信
\newpage
自己遇见女儿的场景是发烧做恶梦,他认为这一切都是真实的,可信的……志龙把自己遇到的一切都告诉了老K和他的同事们,可是大家都笑着摇摇头——他
们根本不相信。 

自从火山生还后,志龙似乎总是感到女儿没有走远,就在他的身旁,他总是感到体内有一股澎湃的能量大潮,无时无刻不在激荡着,冲击着自己的每一
个细胞。 

志龙的思绪瞬间回到了现在,他摇头道:“不,小D你不明白!我的神智比任何时候都要清醒——我坚信如此高密度出现的坑洞,并不是开采地下水等人为因素造成的,而是受太阳的射线风暴影响,E星内部的地核被穿透并发生剧烈变化造成的,这才是主因!就好像一个人发烧了,身体的各个部位,包括表
面温度都会有反应。” 

“你是说太阳发烧了?别是你发烧了吧?你还是个顶级科学家呢,简直是一派胡言,毫无科学根据!”小D拍案而起,怒视志龙,“这是科学殿堂,不
\newpage

是布道场,更不是精神病院!” 

面对一连串的质疑,志龙眼睛闪着坚定地光彩,朗声道:“不!我的女儿在临走前清楚地告诉我,E星它不是死的,而是一种生命存在形式,是一种星
球型宇宙生命体……” 

“我提醒你,我们是科学家,不是科幻小说家!”小D声色俱厉,指着志龙道:“志龙先生,你的女儿已经失踪了,这是不争的事实!你不要成天沉迷
于臆想之中,疑神疑鬼,不能自拔!” 

“年轻人,你懂这什么?还差得远着呢!我说
的是千真万确——E星是……” 

“是什么星球型生命体,对吧?好了,这话你说得数不清了,也太玄虚了”,老K有些不满,他打断了志龙的话,“还是说说现实吧!我想听听小D你
的看法。” 

小D摆出一副雄心勃勃的样子,庄重道:“我
\newpage
以为地表塌陷的原因就是过度开采地下水!过度开采地下水后,剩余水中的酸性物质不断增多,常年累月分解腐蚀岩石,最终导致岩层变薄,无法承受泥土的
重量而形成塌陷。” 

志龙摆摆手,不以为然地笑道:“你的意思是E星东西半球都在同时大量开采地下水?而且都集中在北纬30度和北纬45度地区,这可能吗?我问你三个问题,请回答,一、E星水资源是怎么分布的?二、就南北半球而言,谁开采地下水的历史更长?三
、南北半球的地质构造是否相同?” 

小D怒气冲冲道:“这还用说,E星的地下水分布是不均衡的,其中北半球较多而南半球较少,南北半球地质构造相同,而且南半球开采地下水的年代
比北半球更加遥远,超采时间更长,量更大……” 


“那好!” 

志龙一拍桌子,朗声道:“既然南北半球地质构造相同,北半球地下水资源多而南半球少,并且南
\newpage
半球开采地下水的年代比北半球更加遥远,超采时间更长,量更大,那为什么水资源少的南半球的地表不塌陷,却偏偏水资源多的北半球的地表塌陷了呢?并且好像得到了号令,都集中在北纬30度和45度的
分布线上呢?” 


小D一时张口结舌,不知所措了。 

志龙看看小D那窘迫的神情,意味深长道:“那一个个的深坑让我想到了核弹发射井。当发射井要发射核弹时,会首先打开发射井盖,于是一个深洞就出现了,这就如同打开了潘多拉之匣一样,那一个个神秘的天坑似乎就是E星打开盖子的发射井——会不会在某一天,某一时刻,这些深洞中会大量而猛烈地释放出E星内部物质,比如不可知的岩浆或气体?历史上恐龙的灭绝是很神秘的,根据很多恐龙死亡的姿势表明,它们是在一瞬间死去的,而能造成生物大范围瞬间死亡的不是岩浆,不是气候,而是一种致命气
体……” 

“致命气体?”老K惊异道,“怎么可能呢?
\newpage

” 

志龙目无表情,喃喃自语道:“我现在认为,E星会不会每隔一个周期就向地表释放内部气体,而每次释放都造成物种大灭绝呢?那一个个深洞,似乎就是打开井盖预备发射核弹的发射井,这些深洞中随时会大量而猛烈地释放出E星内部物质,从而危及人
类的生存……” 


突然,异变又起。 


“最新情况!”小D惊叫起来。 

原来电视屏墙上传来新闻消息:随着一声巨响,P城市中心广场突然发生地陷,广场突然变成了一个直径一百米的圆形大深坑,深不见底。一座百年历史的教堂、两栋楼房及几十名观光客顷刻间坠入洞中,不知去向,人群如同末日来临,惊慌失措,呼号奔走不已……看到电视节目的航拍镜头,志龙的内心仿佛被尖刀猛地穿透了,他的心脏一瞬间差一点就爆裂了,他几乎窒息。老K也死死地盯着他,满脸通红,
\newpage

冷汗淋漓。 

“老天!这个时间!这个地点……”志龙暗自
呼叫道。 


有道是:心有灵犀一点通! 

志龙和老K两个人几乎同时抢步来到E星坑洞分布图前,他们首先标出P城的经纬度位置:东经9
5度,北纬45度。 

接着,他们的手指又顺着东经95度线,绕过E星北极点,最后停留到西经95.5度,北纬45
度的一个位置上——那是威迪迦拉城。 

威迪迦拉城是西半球著名的旅游城市,但是就在上个礼拜三,老天敲响了这座城市的丧钟——城市的三分之一部分已经不存在了。深夜时分,随着轰隆隆的巨响声,这三分之一的居民区和商业区掉进了天
坑之中,一去不复返了。 

\newpage


“真是惊心动魄!”“真是前所未见!” 

老K从电脑中调出威迪迦拉城坑洞的图片,他们都被惊呆了——这个大天坑坑口广达数百米,在阳光的反射下散发着黑魆魆的亮光,活像一个魔兽的巨
口,简直就是一个地狱之门。 

“两座城市一个在东半球,一个在西半球,但
是坐标点几乎完全对应。”老K低声道。 

志龙定了定神,稳住了呼吸,说了一句他们最
不想听到的话。 

“两点之间,就是线段,一条几乎完美的直线——如果这两个天坑联通了,那么E星会出现什么情
况?” 

不用分析什么,因为这是明摆着的事实。位于东西两个半球的天坑,如果贯穿了E星地幔抑或地核而连通,形成一个横贯东西半球的直线型坑道,那么地核内的物质将会发生怎样的运动,而这又是一个怎
\newpage

样的星球级的沧桑巨变呢? 

老K和志龙感到全身的血液似乎已经凝固了——E星球地震及气象监测中心的科学家们,几乎所以的人,都感到不寒而栗——此时此刻,整个E星球似
乎猛然抖动了一下。 

如同掀起万丈波涛一般,一瞬间,老K、志龙他们被强大的能量左右摇晃,失去重心,纷纷摔倒在
地,一时间呼叫声不绝于耳。 

这是一系列空前广阔的连串巨响,如同天边响起一阵惊心动魄的霹雷——这是来自E星球内部的爆裂声,似乎还有巨大的波涛汹涌撞击的声音和飓风的
呼啸声。 

老K、志龙等科技精英们抑制住狂跳的心,强自镇定,趴在地板上,透过落地轩窗,举目向远方看去,白昼陡然消失,黑夜突然降临。此次此刻,一幅
诡异无伦的夜空景象映入他们的瞳孔之中。 

\newpage

老K、志龙和他们看到了夜空中出现了比北极光还要明亮耀眼非同一般的光芒,一圈一圈呈螺旋状,起伏不定,闪烁着五彩斑斓的诡异之光,这个巨大无比的光圈瞬间横跨天际,照亮了整个苍穹,使所有
的星星都黯然失色了。 

监控卫星接连传回图像报告,从东经95度、北纬45度以及西经95.5度、北纬45度的两个地点,几乎同时从星球内部喷发出一股乳白色烟雾光柱,呈扇面形直达外空,如同E星生出两只翅膀,景象壮阔无比。接着,电子视频墙显示,E星的大西海的海水顷刻之间掀起高达百米的狂澜,无情地席卷了一些岛屿和海岸,不出十分钟忽又消失不见了,如同被拔掉了塞子的蓄水池一般,似乎一滴水也没有剩下
,不留半点残屑。 

志龙、老K和同事们不敢相信自己的眼睛——P城已经彻底消失了,这座宏伟的现代化大都市如同被扯裂的玩具般,四分五裂,葬身于巨大天坑之中。

但是,辽阔无垠的大西海海水到哪里去了?难
\newpage

道是瞬间就蒸发了? 

这时,电子视频墙上一片雪花。小D报告,卫星突然出现故障无法继续工作,无法继续扫描分析全星球的变化。而此时此刻东方的夜空中出现了一个科学家们从未见过的、硕大无匹的新星球,比E星的卫星——广寒宫星还要大上一倍,放射出橘红色的黯光,如同一张半明半暗的脸,很快又消失到浓厚的云层
后面了。 

广寒宫星哪儿去了?难道距离E星轨道如此之
近又出现了一颗新星? 

“难道我们遇到一颗从未发现的新星?这不是什么广寒宫星——这是猎户太阳……”志龙低呼一声
,浑身不禁颤抖起来。 

“猎户太阳?你胡说什么……”老K高叫道,
“我们的神智难道都浑沌了?” 

一时间,合乎逻辑的世界似乎已经完全消失了
\newpage

,大家全懵了:这究竟是怎么回事? 

夜色渐渐散去,天空开始放晴了——尽管有些
灰蒙蒙的,但总归是放晴了。 

虽然星球震动猛烈而空前,但是据科学家们观察,位于E星阿德罗山脉顶峰的地震与气象监测中心附近的地理环境并没有发生变化。尽管地震与气象中心大楼主体倾斜了20度,但是仍然屹立着没有倒下。从外表看去,阿德罗山脉到圣母玛丽亚海岸,无论是形形色色的丘陵平原、还是怪石嶙峋的海岸,都看
不出有明显的破坏性的地质变化。 

E星的震动似乎已经暂时消失了。在老K的带领下,全体科研人员在经过身体和心理的巨大的震撼和莫名的惊慌之后,马上恢复理智,各就各位,继续想操纵各种仪器,但是,由于E星磁场发生严重干扰
和扭曲,所有的仪器已经停止运转了。 

“老K,我们现在成了盲人瞎马了!”小D尖

\newpage
叫道,“我们无法判断下一步的情况了。” 

老K下令,释放并使用气象中心的储备电能,
但据估算也只够用四十八个小时。 

“我有一个初步判断——大西海的海水已经流
进时空隧道!各位请看……” 

志龙一边说着,一边操纵电脑,按照数据程序把一个锥型四面体放到一个圆球里,它的四个角刚好
抵住南北纬45度附近。 

“锥形四面体——这是什么呢?”小D又高叫
起来,“难道这像是……马首大金字塔?” 

“年轻人,这回你猜得不错,这是本星球的马首大金字塔。”志龙意味深长地看了他一眼,微微颌
首道。 

提起马首大金字塔, E星人可谓家喻户晓。据考证,这座具有古老历史的金字塔的塔身呈现出一匹骏马的头,是E星前纪年三千年建造的,当时是E
\newpage

星的古斯铁里王朝的全盛时期。 

志龙环顾四周,目光扫过每一个人的脸,沉声道:“我一直都在思考一个问题,那就是星球的能量中心聚集点在哪里?我初步判断,洞穴点位于E星北纬45度处,这说明了什么?南北纬45度,这是已知的星体能量中心点,猎户座太阳系每个行星的同一地点,例如本星球的马首大金字塔、战神星的巨大人面雕像、朱庇特星的大红斑、光环星和天皇星上的紊乱气流等,都位于南北纬45度附近,这些地方也都
是星球的能量中心聚集点。” 

“星球能量中心聚集点?”众人疑惑道,“能
量中心聚集点难道不是在行星的地核深处吗?” 

志龙用肯定的口吻道:“地核的能量是地壳运动的第一推动力吗?我个人认为不是!推动星球运转的主要推动力来自猎户太阳对行星的作用!说得更明确些,就是超级太阳风暴对E星地核的穿透性影响!地核是受制于太阳的,这就必然形成一个星球能量中心点——而猎户太阳系的所有行星都有这样一个相同
\newpage

的点,那就是南北纬45度!” 

“那么,找到这个星球能量中心点又能做什么
呢?”老K摇头道。 

“这是一个高频率时空通道,换句话说,它是
猎户太阳和系内行星的转化媒介……” 

“转化媒介?”老K连连摇头道,“你——你是说刚才贯通东西半球的天坑是时空隧道?何以见得
?” 

志龙若有所思地望着远方的海岸,问道:“你们谁能告诉我,圣母玛丽亚海岸距离阿德罗山主峰有
多远?” 


“大约15公里。”小D不假思索道。 

“你再仔细看看——就现在!”志龙一字一顿
道。 

\newpage

作为一个讲求实证的科学家,小D生平头一次发现,从他站的地方到海天一色的地平线之间的距离似乎大为缩短了,这使他惊愕之极。因为从他现在站的阿德罗山主峰上极目远眺,地平线应该在15公里左右的地方,但他眺望到的地平线距离他至多只有5
公里左右, E星的体积好像刹那间突然缩小了。 

“E星正在变小,估计整个星球都在扭曲之中。”志龙沉思片刻,把目光投向顶头上司,问道:“
老K,现在的时间是几点了?” 

老K揉揉红胀的眼睛,看看手腕上的表,肯定道:“是七点!志龙你瞧,这是一块石英表,两个世
纪前的传家宝——很古董了,但准确率很高。” 


“是晚上七点还是早上七点呢?” 


“可能是晚上七点?” 


“晚上七点?你再仔细瞧瞧!” 

\newpage

“对呀,你看太阳挂在西天,马上就要落下去
了。” 

“落下去?你再仔细看看!”志龙挥挥手指向远方,高叫起来,“太阳刚刚升起!不信你看,就在我们说话的功夫,它又升高了一些,而且速度较快。


“你——你瞎说!” 

老K感到自己受到了侮辱似的,也高叫道:“
你昏头了,这根本不可能!” 

然而接下来发生的事实改写了历史——这个挂在西天上的猎户太阳正从西边地平线上快速升起,它刚刚冲破黑夜的羁绊,光芒四射,似乎正在无情地嘲
弄老K的判断,展开白天的漫长旅行。 


猎户太阳正从西方升起! 

猎户太阳就像一个比平时还要巨大的天国火球,似乎占据了半个空间,如同君王般居高临下,无声
\newpage
无息,却又感到一股残酷的、愤怒的灼热,天地之间
似乎充盈着这股无形而巨大的热浪。 

科学家们感到万分惊愕。因为除了猎户太阳的运动发生异常之外,E星空气中也发生了一种难以置信的变化,就拿这批科学家来说吧,他们已经变得气喘吁吁,呼吸急速,好像登山队员攀登珠穆朗玛峰一般艰难,感到周围的氧气急剧稀少了,而且他们彼此说话的声音也微弱了。造成这种情况的因素不外乎两种:一是他们突然患了耳聋,听觉变得迟钝了;二是
空气的导音效应大大降低了。 

作为顶级科学家,老K立即意识到发生这种不祥之兆,并不是因为太阳改变了它在天空中的运转,而是E星改变了它的自转方向——由原来的自西向东,变成了自东向西,而且自转轴已经严重倾斜了。E
星的轨道离太阳是越来越近了。 

志龙斩钉截铁道:“毫无疑问,由于太阳的影响,在空前强大的震荡之下,E星地核改变了它的运

\newpage
转方向和公转轨迹……” 

老K惊异道:“你是说E星围绕太阳的椭圆形
切向轨道不存在了,我们将直接奔向太阳……” 

“这是一种急速的变迁——我初步计算,用不
了48小时,我们将直接和太阳融合!” 

谈起科学以及E星的这次大灾变,科学家们早将生死置之度外。志龙介绍说,众所周知,像猎户座太阳这样一颗伟大的恒星,由于它周围具有巨大的质量,因而周围存在着强大的引力场,比如说,一个在E星上重120斤的人,在太阳上将重达1万斤左右,这是由于太阳周围的时空比行星周围的时空弯曲更
严重所致。 

志龙进一步谈到大胆的设想,他决然说:“现在有一个设想,从这颗猎户太阳周围的四维时空中切割出来一片空间的二维叶,你就会发现,在远离太阳的地方,由于引力微弱,所以时空是平坦的,而贴近太阳表面处的引力最强,那里的曲率也是最显著的,因为E星是第一颗内行星,最靠近太阳表面,所以它
\newpage

具有时空扭曲的最大化特征——出现黑洞。” 

爱因斯坦和罗森在20世纪三十年代首先研究了黑洞产生的形状,他们发现黑洞的嵌入把我们的空
间(上叶)和另外一个空间(下叶)联通起来。 

所谓穿越时空的理论是这样的:一般认为,两点之间线段最短。但还有更短的方法,如果宇宙是一张纸,上面有a和b两个星球,互相距离很远,那么快速到达的方法就是直线飞行。但是,还有更快的方法。如果我们把纸弯曲,那么,两颗星球岂不是靠的更近?这个就是翘曲空间。那么怎样弯曲宇宙这张“
纸”呢?这就要靠时空穿梭。 

科学家们认为,在黑洞周围存在着时间的空洞,从这些空洞里,我们可以超越一切时间,空间,速
度,能量,物质,甚至生命。 

“我看不出这有什么新奇”,老K沉声道,“这能证明‘翘曲时空’已经被你制造出来了?你研究了这么多年,怎么能证明天坑就是穿越时空的隧道?
\newpage


“翘曲时空不是我制造出来的,而是E星球的星球能量中心点,是它自己制造出来的,因为它要使自己获得新生!”志龙叫道,“刚才我们看到的景象说明了什么?天坑就是隧道!因为E星球它是有生命的,它要进行一次重大的星球升级,这就好比电脑一样,在升级前必须先要格式化,否则就不能新生——
或者说,格式化是升级的必由之路。” 

志龙的话音刚落,大地又开始剧烈颤动起来,天空变得愈来愈昏红黯淡,大量的火雨从天而降,平静的阿德罗山脉开始变得狂躁不安,像一个拉面条一样被前后左右甩动和伸缩,气象中心的大楼像是被送上了波峰浪谷,大幅摇晃,钢筋嘎嘎作响,墙壁和屋顶出现了无数大大小小的裂缝,眼看马上就要解体了。一些人已经被碎裂的石块当场打中,发出凄惨的叫喊;另外一些人正像无头的苍蝇般四处乱窜,滑向未
知的深渊。 

此时此刻,E星如同一个临近沸点的高压锅,空气中似乎开始燃烧了,充满了愤怒、咆哮和凶猛的
\newpage

气浪,狂暴激射,流动不息。 

“看来,我们要被格式化了”,老K爬到志龙跟前,一边瞪着眼睛,一边粗重地喘息道,“我们将
随着E星一起灭亡,这就是我们的共同命运……” 

“老K你看,P城天坑正在快速扩大,马上要把阿德罗山吞掉——这不仅是一个时空隧道,更是一
个通向新世界的大门!” 

“你是说,这个隧道打开了通向另一维宇宙空
间的大门?” 

“是的!E星本身也将进入这个隧道,它正在急剧扭曲和变化,并在太阳的帮助下完成转型的历史
使命。” 

“好吧!我们姑且认同天坑就是时空隧道的假设,但是面临即将到来的残酷现实——E星被太阳无
情烧毁的现实,我们也难有回天之力!” 

\newpage

突然,一块巨大的水泥块劈头砸下,重重压在老K的后背上,一股撕心裂肺般的刺痛传到他的神经中枢,他似乎能够听到脊椎骨的断裂声,不免有些绝
望和苍凉。 

恍惚间,老K觉得自己的身体空大而虚弱地躺在这里,他的思想正与这个躯壳若即若离,他现在的精神已经无法达到躯壳的各个部位,也管不了自己的手和脚了——它只聚集在自己的大脑、额头、眼睛和面孔这样有限的部位。他想起笛卡尔说的“我思故我在”,此刻,他在意识到“自我”时,其实只是意识到自己的眼睛和脸上的表情以及在这个表情上聚集的思想。这个部位是明亮的,而整个身躯从脖颈以下都已黑暗虚无,与“自我”脱离。他从嘴里喷出一股血,恍恍惚惚、断断续续的低语道:“老朋友,我——
我要走了……” 

“老K——”志龙紧紧抱住他的头颅,不禁热
泪长流。 

“我将告诉你一个秘密,这个秘密已经困扰我
\newpage
很久了,我本不想说,想把它作为终生的秘密珍藏,但现在不能不说了——去年我从马首大金字塔里发现
了一个秘密——这是一张外星人名片……” 


“外星人名片?” 

“千真万确!”老K吃力道:“我试过多次,这是他们对我们一次警告!它只有绿豆那么大,我把它植入了假牙里”,老头说着,用手用力一掰,前门牙分裂为二,一颗黄橙橙、亮晶晶的椭圆物体掉落下
来。 

突然,那物体闪出一束光——一束耀眼的强光

这束光芒无尽地旋转扩大,变成了一团跳动的光子,组成了一个人头的形状。光子组成的人头形状微妙地变化着,不一会儿变为一张清晰的骏马的脸。

志龙吸了口冷气——原来这张马脸就是E星历史久远的马首金字塔的立体雕像。刹那间,志龙的心中犹如狂涛汹涌,翻滚不息,受到了难以用语言描述
\newpage

的强烈冲击。 

志龙感到有一种声音——这种声音并不是响在
他的耳畔,而是响在他的心里。 

这是外星人在用大脑、心灵和思想在和他进行精神层面的传感通话——这是一种委婉有力的女性声
音。 

“亲爱的朋友!你在听吗?一场史无前例的伟大变革性进化已经开始了——你们的猎户太阳正在受到银河系中心高频率能量波动的影响,它正在开始承接一项新的使命,就是接收来自银河中心的高频能量。这银河系中心的高频能量如此强大,足以毁灭你们在时空里感知到的一切。正是如此,有必要经由你们的太阳来协调、转化这些高频能量,以便保护在你们
世界维度里的较低能量,完成最后的转型。 

“亲爱的朋友!你们的世界正被高强度的银河光子冲击,这意味着所有生命形式的细胞正在被重新校准,基因结构正在改变,构成你们的骨肉的碳的蓝
\newpage
图正在转变。首先是转变成水晶结构,然后当水晶结构完备之后,将开始吸收银河光子粒。改变了的细胞结构和功能,使E星生命不再畏惧星球的震动与升级——换句话说你们的细胞结构开始在更高的星际频率
上振动。 

“亲爱的朋友!时空之门已经打开——水晶和光子是精锐的载具。就是说它们能发送接受精微自然的信息,你们将开始变得对精微能量更敏感。这是你们以前的碳基身体感受不到的。理解了这些将对你们感知宇宙有深远意义。换言之,当你们精神境界发展到足够成熟,你们的能力会更加确定和可靠。但是,不是每个人都能安全转化,因为精神世界的加速和提升是与自身的灵性觉醒、提升和成长有密切关联的。只有那些精神成长与宇宙意志相协调的人,才能改变他们的基因结构来接受高强光子能量的猛烈冲击而不
被损害。 

“亲爱的朋友!请跳进时空之门!我就在你的
前方——您的未来将迎接你……” 

\newpage

外星人的声音消失了,老K低垂着头,自语道:“原来我不知道什么是时空之门,现在知道了——
E星已经打开了它……”接着,就无声无息了。 

“轰隆隆”一声巨响,地震及气象中心大楼中裂为三,神坛终于形神俱灭了。志龙和小D被一股震波无情的抛洒出去,向一个无边无底的超级深渊加速
坠落。 

突然,志龙觉得自己的身体受到一股无形力量的托举,像一片树叶那样飘浮起来,和他并肩在一起的小D也没有逃脱命运的摆布,紧接着飘浮起来,他
们两人旋转在一起,缓慢地下降。 

志龙的头脑里一片空白,他觉得自己的身体有一种撕裂般的巨痛。小D本能地抱着他的腰,两人像一个陀螺般越转越快,最后化做了一团白雾,进入了
天坑之中,一切都消失了。 

志龙和小D并没有死。在飘进天坑的刹那间,他们不由得发出一声尖叫,看到了最不能想象的事物
\newpage

——他们看到了另一个宇宙。 

浮现在他们眼前的是一幅奇妙而深邃的图景:他们周围竟然是广阔无边的漆黑空间,星光灿烂,有美丽的星云、星团、星河,还有划过虚空的流星,比
他们平时在射电望远镜中看到的还要多、还要亮。 

正在惊魂不定之际,忽然一道蓝色强光把他们笼罩其中,在那奇异的空间以光速向前冲刺,无数的星云变成了一条条的银线,无尽的伸展着。他们感到温度在急剧上升,转瞬之间就到了难以想象的程度。小D紧紧拥抱着志龙,大睁着一双黑亮的眼睛,仿佛不相信他们马上就要形神俱灭。他们像是到了烈火熊熊燃烧的地狱内经受烧烤,又像是被千锤百炼的生铁般经受敲打,仿佛肉体已经完全不复存在,只有纯粹
的意识还活着。 

志龙感到,就像身处一场梦中,一切都不真实起来。在他昏迷前的最后一刻,志龙心中忽然升起一种明悟,他仿佛觉得自己的女儿芸正满怀喜悦之情,扇动一双美丽的翅膀,浑身披满霞光,前来欢迎自己
\newpage
父亲的到来。在埃德菲火山的最后一面,好像就是为
了等待这一天的最终到来。 

跳进天坑的一瞬间,仿佛时光已经流淌了千万
年。 

也不知过了多久,志龙恢复了知觉。他恍恍忽忽地觉得自己好像躺在冬天冰凉的石头上,一股不可阻挡的凉爽似洪流般涌进自己的身体里,舒坦极了。

志龙微微睁开眼睛,这才发现他的年轻同事小D正万分惊慌地站在他的面前,满脸泪花。自己半躺在一块方方正正的东西上面——这是一块地质残片,但是志龙发现,这块残片却是P城市中心何塞大教堂
钟楼的一块铜钟板。 

“啊!你——我,我们还活着呢?”小D不由
得欣喜若狂。 

一时间,志龙精神为之一震,他发现了一幅璀璨绚丽的图像,他示意年轻的伙伴注意四周的环境。
\newpage


他们惊奇地发现,他们好像置身一个童话世界:天空彩云流动,大地覆盖着奇形怪状的巨型植物,山峦、高原、谷地各种地貌从浓稠的粉色大气中露出它们的本来面目。一派奇异无比的景象映入了他们的
眼帘。 

这是一个粉黄色的天地:粉黄色的山坡、粉黄色的小溪、粉黄色的树林、粉黄色的小花……最奇妙的是那些树木,形状极像E星上的蘑菇,大小不一,高矮不等,形成一片片粉黄色的“蘑菇林”……刹那间,他们嗅到了扑面而来的泥土的芬芳,感受到了对
大地、对泥土、对生命的无限热爱和眷恋。 

小D惊叹道:“这里的景象真是太美了,我们
这是在哪里呀?” 

志龙笑道:“我们哪儿也没去,就在这里!”

“哪儿也没去?”小D感到十分疑惑,又惊叫道:“志龙!你的头上有一圈光,淡黄色的光!时隐
\newpage

时现的!” 

“你也有,小D!”志龙注视着他的头顶,朗
声道,“你的头上也闪着光圈呢。” 

此时此刻,天边滚过一阵惊雷,他们抬头远看,发现他们日日夜夜研究的E星——他们无限熟悉的家乡星正在深蓝色的天幕上扭曲、挣扎、融化,浑身
蠕动着赤金般的火舌。 

“快看!那是我们的E星,它好像变得通红,成了一片火海,成为了炼狱……”小D不禁泪花滚滚。“我要回家——”小D哭泣着,跪倒在地,大喊道
:“我想见爸爸妈妈!” 

“小伙子,你好糊涂!这就是我们的家呀,我们的亲人的灵魂将伴随着我们去开拓新的纪元!那也不是我们的E星,我们现在不正在E星上吗?那只是一个低维度空间的景象,是一个旧世界的结束而已,
而新的文明世界即将开始。”志龙平静道。 

\newpage

小D擦了把眼泪,颤抖着声音,问道:“你—
—你说,这真的是新的文明开始?” 

志龙严肃地看着这个年轻人,说:“刚才我们在时空隧道中经历的一切说明了什么?这就是在原来的物理世界产生的革命性效应——我的内心一直有一种声音告诉我,这是一场伟大的进化,那些被旧有物理形式包裹的人类转变了,从现在开始,我们的基因结构已经改变,构成我们骨肉的碳的蓝图也正在发生革命性进化,能够接受宇宙能量,我们的天命就是击碎旧世界的藩篱,建设一个更加接近宇宙真理的新世
界。” 

小D抑制不住内心的波澜,急切地问道:“我
们能接受宇宙能量?一切都将重新开始?” 

志龙微微点头,庄重道:“是的,这不是终点,而是伟大文明的新起点,这是一个崭新的空间,一个新世界即将诞生——而我们应该就是新种族文明的始者。”

\end{document}
