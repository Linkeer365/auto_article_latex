\documentclass{article}
\usepackage[utf8]{inputenc}
\usepackage{ctex}

\title{水、时间和生命:骆一禾的诗学想象\footnote{Click to View:\url{https://site.douban.com/187916/widget/articles/14392012/article/28008504/}}}
\author{韩王韦}
\date{2008}

% \setCJKmainfont[BoldFont = Noto Sans CJK SC]{Noto Serif CJK SC}
% \setCJKsansfont{Noto Sans CJK SC}
% \setCJKfamilyfont{zhsong}{Noto Serif CJK SC}
% \setCJKfamilyfont{zhhei}{Noto Sans CJK SC}
% \setlength\parindent{0pt}

\begin{document}
\CJKfamily{zhkai}

\maketitle


\Large

时光如梭,至今年5月31日止,骆一禾离开我们已十九周年了。在这十九年中,诗歌界发生了很大的变化,“中年写作”、“个人写作”、“知识分子写作”、“民间写作”、“第三条道路写作”,以及后来的“下半身写作”、“垃圾写作”,真可谓是你方唱罢我登场,燕燕莺莺一派繁华景象。然而,在这满耳弦瑟之中,我们发现诗歌的根基是不稳固的,这种不稳固不仅仅表现在新诗合法性的可疑或缺失,而且表现在对这种合法性深入追问并彻底反思的缺失,同时,它还表现在对新诗诗人历史坐标体系的构建以及全面清察的缺失。正是这种种缺失使得我们无法直面某个诗人的价值与意义,我们无法肯定我们对他所做的判断是否准确,从而,我们只能援引他的社会影响和其他诗人的相关评判,来小心翼翼地搭设我

\newpage
们的观点。 

可以说,骆一禾就是这种境况中的一个牺牲品,我们几乎就要把他彻底地淡忘了。偶尔有人想起要阅读他,不是被其纷繁的意象所迷惑,读了几首诗后以偏盖全;就是借助阅读海子的经验来理解他,将他的诗虚构成海子的诗的延伸或附庸。冷漠已经使得我们不愿或不屑深入地了解这个声名不彰的诗人。这不
单单是骆一禾的悲剧,更是整个诗学的悲剧。 

无论是从诗歌史的角度,还是从诗学诗艺的角度,骆一禾的诗都是值得我们重视的。海子曾说骆一禾的诗具有“本质的单一性”[“深渊里的翱翔者:骆一禾”,西川,《让蒙面人说话》,东方出版社,1997],这种看法极为贴切。骆一禾的单一性其实就是他的想象的一致性,亦即对水的想象的至死的忠诚。水对骆一禾来说是具有巫性的,它不但以空间的流动形态穿连起昼夜,彰显出时间;还以其无争之善滋养着春天、生命以及苦难和血。骆一禾正是从水的神秘性着手来构建他的诗学理念的,他说:“没有谁/能像水那样原生”(《水(二)》)[本文所引用诗文,如未注明,皆出自《骆一禾诗全编》,上海
\newpage
三联书店,1997]。正因为水最原生,因而它是最好的,它就是这世界的促生之力,是大地的血脉,也是生而为人的命运。作为个体,血、泪、汗水是我与水的联系的佐证,我将生命投射到诗中,亦即是将我的水投射到诗中,诗是我“生命律动的损耗,也是它的感情”(《诗论·春天》)。也正因如此,骆一禾才以情感为依凭,开始了他对生、对力、对爱的讴歌和对人类苦难的关注。“人们可以永远地爱/永远不会失去欢乐的面容”(《生日》)。骆一禾的爱以
及表达爱的语言,都是饱含着水的。 

下文,我们将从水的时间,水的生命,水的伦理等三个方面来进入骆一禾的诗歌理念,试图能在深入理解的基础上,为骆一禾寻得一个本应属于他的诗
歌地位。 


    水的时间 


    水不会让季节生锈 

    水满含着化冻的香味——《水(一)
\newpage

》 

水是直观的,鲜活的,不生锈的,而立足于水的想象也是如此。“水是大雨的双脚天空的房子/是陆地和海洋的屋顶/揭示着十足的柔软  潮湿与明净”(《水(三)》)。柔软的水召唤着诗人全身心地融入其中,使自已的暇思与水的脉搏合而为一。没有我思,只有水在思,或者说是我以水的方式在思,水外无我,我亦非在水之外。骆一禾的这种暇思方式决定他必然反对“以智力驾驭性灵,割舍时间而入空间”,并认为“其结果是使诗成为哲学的象征而非生命的象征”(《诗论·春天》)。“谁都不能把水盛入任何器皿/或任何地方/赋予它任何形体”(《水
的元素》),骆一禾如是说。 

在骆一禾眼中,水天然是要与春天联系在一起的。在春天,“那诱发我的/是青草/是新生时候的香味”(《青草》)。春天是生生的时节,也是不停涌动的时间的象征。春日里大河奔流,云朵、雨水和麦子一起长大,“地球吹响绿色的树叶/原野蔚蓝/春天洁白如玉”(《春之祭》)。春天是水做的,它
\newpage
如水如时间一样,是自我生成的,春天生出春天,水生水,时间生时间。春旱来临时,“所有的桦树林/都延展着根须/像水一样/滋滋地吸进土里去/被地下热流/烤得发白/终于炼出了水/就这样救活了河”(《春天(一)》)。因为有了水,春就是不死的。而把握住春天这种永远向生的状态,就是骆一禾的诗歌梦想,是以,他说:“春天,我的朋友,我的美
学和血中的水”(《诗论·春天》)。 

骆一禾通过对水的时间性的想象,认为我们的传统应该是不断关涉到当下的,是生生不息的,而不应该是以龙为标识的死的概念。“龙是一种罪恶,一种大而无当的谎言”(《滔滔北中国(北方抒情)》),“龙是一个漫长的  没有意思的故事”,“作古代历史的继承人是危险的”(《春天(二)》),基于这种对龙的否决和对古代历史的质疑,骆一禾渴望“使血液真正地成为血/使水真正地成为水”(《年终》)。在他的笔下,世界万物皆因水而备有魔力,光闪闪奔跑的河流,光闪闪奔跑的大地,光闪闪奔跑的驼鹿,以及在阳光下闪烁着的鱼、裸身渡河人、无鞍马等等一起构成了骆一禾的景观体系[骆一禾的
\newpage
景观体系远不止于此,它要更庞大、更复杂,但无一不富有水的魔力。]。这景观体系是当下的,诉之于你我视觉的,它不能离开场所与地点而存在。“地点是我们这个时代依旧庄严的东西,它原型的质地给思维带来了血浆,艺术实体才不仅仅是头脑的影子”(《诗论·美神》)。强调地点其实就是对在场的一种重视。对一首诗来说,诗人的在场与否事关成败。诗人不但要在他的写诗之所拿起笔来,还要在他诗里所构造的景观中若隐若显,他得寄身于物质的想象里等待交谈。骆一禾就存隐于他的诗性的语言中,他的物质的想象中,以及永恒的水流涌动中。他意欲借“不能代之而生、不能代之而死的生命个体”(《诗论·美神》)来激活濒死的文明。传统文明如水流经这些个体,这些个体承其所来,开其所往。在他们身上,集合着过去、现在、未来,集合着永恒流动的水的时间。“生命在体力及精神上的挥发、锻造,这便是我们的历史,便是我们真实地负载着的、享受着的、身处其中的历史”(《诗论·水上的弦子》)。因此,骆一禾在诗中宣称:“我不愿我的河流上/飘满墓碑/我的心是朴素的/我的心不想占用土地”。(《生

\newpage
为弱者》) 


    水的生命 


    我们仰首喝水 


    饮着大河的光泽——《大河》 

水与大地一样,都是富有母性的,因而在骆一禾笔下,它们是相通的。“大地在动着/大地有会说话的眼睛/大地太柔软了”(《大地》),大地作为万物之本原,它是好生而多产的,“这深厚的土是如此潮湿  沉重/没有太多的话语/带着母性阵痛时/那种全身的力量……”(《平原》)。水是大地的乳汁,是母体与胚胎之间的牵连,它隐含着生产时的疼痛与力的挣扎,我们喝水,就是对这原初之痛与力的追祭,“这一瞬间  河流明亮起来/我们的身躯轰然作响/一切都回荡在激动的心中”(《河的传说》)。生命每时每刻都受到这原初之生的召唤,种子是永久的,而我们则是永恒之生的留滞,是水的力量在奔驰之际的重负。然而,大河到底是要奔流的,我们都饮着河之水长大,也注定是要奔流的,“两千只
\newpage
眼睛同时醒来”,“我开始大块飞行/一千只倾斜的鸟儿平展地起飞/低沉地掠向江心”(《世界的血·飞行》)。河之乳汁永不会是冰冷的[《水与梦:论物质的想象》加斯东·巴什拉著,顾嘉琛译,岳麓书社,2005,133页],我体内的血亦将如是。所以,骆一禾宣称要“以我的惊涛/站立在大地上/并以惊涛思想”(《沉思》、《雪》),这即是他作为一个生者的热情。对骆一禾而言,血是浓缩的水,是水的精华,也是火的燃料。他说:“我在辽阔的中国燃烧”,“我是有所思而燃烧的”(《诗论·美神》)。骆一禾燃烧时的火焰是液态的,这即是他的水之焰,在这水的火焰中,世界于诗人面前如花怒放。
 

燃烧是青春的悲壮之举。“白马跃向长空/扑落崖底/还摔不成一条路吗”(《青春激荡》),燃烧需要的就是这种精神,在损耗中寻找生的出口,在无路之处开辟出新的道路。它要求生命用心脏来“捶打地面”,希冀于能“垂直击穿百代”,“彻底燃烧”(《灵魂》)。燃烧同时又是邀请式的,它注定不会只属于孤立的个体,所以骆一禾会说:“一个人绝
\newpage
不是只有一个灵魂”(《黄昏(二)》)。他明确地将自已的诗歌写作置于“世代合唱的伟大诗歌共时体之中”,意欲生长出“精神大势和辽阔胸怀”(《诗论·火光》)。骆一禾的燃烧是人类整体式的燃烧,是意欲唤起所有生命体普燃之势的燃烧,“我就是大地上的  炽热的火焰/焚烧着  自焚着/穿过一切又熔合一切  不同于一切”(《世界的血·女神》)。他大声向四方呼告:“太阳晒在你的背上/性命和奔放的马群长在你们的身体上”,“后来的能者们/极尽你们的能燃与火种”(《舞族》),他就是要与世界一起,大块燃烧。这种大块燃烧的欲望,使得骆一禾要附形于万物之上,成为这世界本身。他说:“一个人要把一切都吞下去”(《乌鸦(一)》),然后用石磨进行研磨,使之融入我血,融入我的肺腑。“故我在不问生死的烈火之畔/故我的血流穿了
世界”(《世界的血·世界的血》)。 

然而,这种把自我进行无限延扩的思想是有紧张的,这紧张是血的燃烧的紧张,是作为燃料之水的供应的紧张。“雨水闪闪发亮/雨水在某些地方变成洪水/雨水的四周是洪水,洪水的四周是海洋”,唯
\newpage
有海才能提供我无限的燃料,给我大块燃烧之可能。海是高密度的,如我的脑浆,如水银,它容含了我所有的想象,囊括了我整个的诗歌世界。然而,当想象将骆一禾引领至“大海”之际,他却进入到一个非人的空间,诗歌的未竟之地,“这上没有道路,也没有那种可以预期的、并不陌生的道路的崎岖感”(《诗论·火光》)。骆一禾将自我与世界合一,却又与此同时迷失了世界,也迷失了自我。我在哪里?诗人如此追问。在这生之紧张中,骆一禾说:“我愿有一个人的面容/关注世界,并自我恢复”(《世界的血·
梦幻》)。 

在骆一禾心中,诗歌就是我的生命,就是我生之感情,它是要不断去寻求探触那未竟之地的。借用幻想之力,骆一禾将引领我们抵达诗歌的深处,水的
深处,那里大海扎向地核,存有普世的善和幸福。 


    水的伦理 

    移向海洋  温暖的鸟儿们
    
\newpage

    并且在那里快乐——《鸟瞰:幸福的
祭祀》 

饱含着水分的泥土是要种庄稼的,它期待着一茬茬的收割。收割所带来的刺痛提醒着我与泥土之间的牵连,促使我“善待亲人”,言语卑微(《泥土》)。对骆一禾来说,我们生在这有着上千年农业文明的古国,故乡的麦地即是我们的朝圣之所,“我们不知道麦地的来去/因此种下庄稼”(《麦地(一)》)。这在雨中闪光的麦地,顶着大太阳的麦地,散发着香甜味的麦地和承载着深重苦难的麦地,它使得“劳动的人们劳累”(《麦地(四章)·伊则吉尔老婆婆》),然而,人们却深爱着它。劳累且心怀热爱就是我们朝向麦地时的膜拜之礼。“是麦地让泪水汇入泥土/尝到生活的滋味/世界各地的死亡”(《麦地(四章)·麦地》)。麦地呼唤着我们行进于修远之途,北、北、北,“那人与方向诞生/血就砍在了地上”(《修远》),北方,那永远的黄河,那亿万人精神上的故乡。然而,这是一条充满苦难的道路,一个不断遭遇血与火的洗礼的方向。“走向麦地之门/鲜血泼在捅破的谷仓”(《麦地之门》),跟随着本
\newpage
身,跟随着那种向往,骆一禾说:“我沿着生命的大路走向我的老家”(《乱:美的祭祀》。然而,在这个枯水季节,在这农业文明的尽头,何处是故乡?“这时候我就是远在他乡/车站上孤零零的一位歌王”(《青年歌手》),“我打开那一朵朵碗大的向日葵/盼望它们断断续续  仍然生长”(《贫穷的女王
:女神现象的祭祀》)。 

于是,世界开始流血,文明开始流血,诗人也开始流血,“这就是时间的血和空间的血”(《大海·超密态物质》),我们的屋宇就处在这一团血污之中。“真实的血/是时代死去的血”(《世界的血·世界的血》),在这新旧交替之际,诗人大声歌唱:“世代相失的农民们,你们的火把/来到我的门前/让我纪念你们,你们这些粗糙的鬼魂”(《世界的血·雪景:写给世代相失的农民和他们的女儿》)。
对骆一禾而言,农业文明的苦难即是水的苦难,是生之为人的苦难,因而他会说“大地上成活的人们灾难而美”(《白虎》)。我们生在一起,“繁育、格斗、流血、千方百计的思索”,(《新月》),我们是同一体,美丽如大海一样。大海是具有自净功
\newpage
能的,“我们饮水食盐/被大海彻底席卷,生活归于枉然”(《大海·海洋出现》),大海如大火大雨一般,焚毁、淹没万物,并视之如刍狗。海运中,所有的历史、屠杀、尸骨、腐败与僵化,都得到了大消解,大炼造。在海洋的深处,是美丽的海王村落,那里“海的女儿含羞成长”,“老海王修筑着水做的门框”,骆一禾笔下的老海王是“农牧文明最光荣的父亲和酋长”(《大海·海王村落》),他沉默无语,日日里修补着自已的村庄。但这里并不是骆一禾所欲停之处,他还要往下坠落,去探寻海的更深入,那更神秘的地方。那里是否会藏有一所海之城,一如屋宇守
护着大地那样守护着海洋? 

“水在大块地潮湿/永动者坐在世界的心里”(《世界的血·女神》),骆一禾就是要追随这永动者,投身于水的温暖与幸福当中。跟随着他,整个人类,整个文明都将沉浸在这温暖与幸福当中,获得新
生。 

骆一禾的诗歌写作主张“从沉思和体验开始”(诗论·为《十月》诗歌版的引言),意图寻求人和
\newpage
物相结合的共同深处,寻求“一切生长着的根”[见里尔克的“论山水”,骆一禾与里尔克的主张并不相同,但这句话对认识骆一禾的诗歌理念来说却相当贴切。《给一个青年诗人的十封信》,三联书店,冯至译,1994.3],他提倡立足情感体验来建构自我的诗歌心象,以探寻自已所独有的、无与伦比的写作可能性。这种立足于不断自我拷问的写作方式,促使骆一禾卷入一种寻求自我超越的密集写作实践之中。他通过对水的本质的想象来整合他的写作。从水的滋生功能走向容污纳垢时水的自净功能,骆一禾不但追问了生命个体向生的可能性,还追问了整个人类文明的向生可能性。整个人类文明的运动形态就是海运,而我们则沉浮在这海运之中。我们需要深入自我的
体验,追古溯源,因为那里隐匿着伦理的开端。 

骆一禾的写作不仅仅是八十年代诗歌界写作竞争的体现,更是一种借助语言来觅获意义的生存方式,一种生命的自我流溢和自我敞显。他的诗歌价值以及对后来诗歌写作的影响,是需要我们予以重视并彻底重估的。他在为亡友海子的长诗《土地》所作序时说:“在这个世界上我不是只有一个灵魂,因而生者
\newpage
的悲痛是大的”。这种对人类灵魂的整一性的想象,与骆一禾的诗学理念相一致。也正是依据于此,他走了生命的博爱与责任的承担。

\end{document}
