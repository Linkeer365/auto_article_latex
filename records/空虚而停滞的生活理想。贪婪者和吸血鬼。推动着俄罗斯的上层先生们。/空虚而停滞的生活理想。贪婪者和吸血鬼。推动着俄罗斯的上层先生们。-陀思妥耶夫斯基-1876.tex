\documentclass{article}
\usepackage[utf8]{inputenc}
\usepackage{ctex}

\title{空虚而停滞的生活理想。贪婪者和吸血鬼。推动着俄罗斯的上层先生们。\footnote{Click to View:\url{https://web.archive.org/web/20220630085526/http://libgen.is/book/index.php?md5=0A2665DC7B9E1E006DCDA377D37A9CA7}}}
\author{陀思妥耶夫斯基}
\date{1876-04}

% \setCJKmainfont[BoldFont = Noto Sans CJK SC]{Noto Serif CJK SC}
% \setCJKsansfont{Noto Sans CJK SC}
% \setCJKfamilyfont{zhsong}{Noto Serif CJK SC}
% \setCJKfamilyfont{zhhei}{Noto Sans CJK SC}
% \setlength\parindent{0pt}

\begin{document}
\CJKfamily{zhkai}

\maketitle


\Large

今年3月号的《俄国导报》上刊登了阿先生即阿夫谢延科先生“批评”我的文章。回答阿夫谢延科先生是没有任何益处的:很难想象一位不理解他自己所写的东西的作家。不过,他即使是理解了,其结果也会一样。他在文章中所写的涉及到我的问题是,不是我们有文化的人们应该拜倒在人民的面前,因为“人民的理想实质上主要是空虚的、停滞的生活理想”,恰恰相反,人民应该接受我们有文化的人们的教化,接受我们的思想和我们的面貌。总之,阿夫谢延科先生非常厌恶我在2月号的《日记》中写的那些关于人民的话。我认为,这完全是由于我自己的过失所造成的含混不清所致。含混不清就需要解释清楚,然而回答阿夫谢延科先生简直是不可能的。同一个突然用下面这些话来议论人民的人,你能找到什么共同点

\newpage
呢?例如,他说: 

“俄罗斯的独立、俄罗斯的力量及其肩负历史使命的能力建立在他的双肩之上(即人民的双肩之上),建立在他的容忍和自我牺牲、他的顽强力量、执著的信仰和不计私利的豁达胸杯之上。他为我们保留了基督教理想的纯洁高尚的和渊源于自己的伟大的温顺的英雄主义,还有斯拉夫天性中那些美好品格,这一切曾在普希金诗歌的高昂歌声中表现出来,此后将会成为经久灌澈着我们文学的长流不息的源泉……”
 

请看,刚刚写了这些话(是从斯拉夫派那里抄来的话),可是在另一页上阿夫谢延科先生关于俄罗
斯人民讲的则是完全相反的话: 

“问题就在于,我们的人民没有给予我们有积极个性的理想。我们在人民身上发现的美德,我们的文学教导我们热爱的人民美德(这是我们的文学值得赞杨之处),总之,我们人民的全部美德都还仅仅处于自发存在的水平上,处于封闭状态的和安闲的(?)生活或者是消极生活的阶段上。人民里面如果出现
\newpage
积极的、朝气蓬勃的个性,个性的大部分魅力也要丧失掉,因为这种个性常常是以吸血鬼、贪婪者和独断专行者这些令人厌恶的面目出现的。在人民里面至今还没有积极的理想,期待这种理想——就等于寄希望
于未知的、也可能是臆想的因素。” 

在前面的一页上刚刚说过那些话之后,立刻就说出这些话来,在前面那一页上说的是“俄罗斯的独立建立在人民的双肩之上,建立在他的容忍和自我牺牲、他的顽强力量、执著的信仰和不计私利的豁达胸怀之上!”不过要知道,为了表现出顽强的力量,只是消极是不成的!为了创建俄罗斯,是不能不表现出力量的!为了表现出不计私利的豁达胸怀,就必须为他人的利益,也就是为共同的、兄弟般的利益而表现出慷慨的积极的活动。为了“以自己的双肩支撑起”俄罗斯的独立,决不能消极地坐在原地不动,至少也一定要站立起来,向前走出一步;至少要稍稍做点事情,而阿夫谢延科在这里立刻就补充说,人民只要一着手做事,就会即刻“呈现出吸血鬼、贪婪者和独断专行的人的这些令人厌恶的面目”。这样说来,竟是贪婪者、吸血鬼和独断专行的人用双肩支撑了俄罗斯
\newpage
。这就是说,所有我们那些神圣的大主教们(人民的保护者和俄罗斯土地的建设者)、所有我们的虔诚的大公们、大贵族们和平民们,就是那些为俄罗斯效力,直至献出生命的人们,历史怀着深深的敬意记住了他们的名字的那些人,——原来全是一些吸血鬼、贪婪者和独断专行的人们!可能,人们会说,阿夫谢延科说的不是那个时候的人们,而是现在的人们,——历史就是历史,那些都是年代很久远的事情了。但是,按照这种说法,岂非等于说我们的人民蜕化了?那么阿夫谢延科所说的又是什么样的现代的人民呢?从什么时代算起呢?从彼得大帝改革开始?从文明时期算起?从彻底被奴役的时代开始?不过,这样的话,有教化的阿夫谢延科先生就暴露了自己,那时任何一个人都可以对他说:教化您真值得,得到的代价是腐蚀人民,把人民变成清一色的贪婪者和骗子。阿夫谢延科先生,您的“光看坏处的本领”竟如此高明吗?我们的人民正是为了让你们得到文化(至少是根据法捷耶夫将军的学说)而被奴役的,而他在受了二百年的奴役之后,从你们、从得到文化的人这里难道不仅得不到感谢,甚至也没有怜悯,而仅仅是居高临下的

\newpage
蔑视,说他是贪婪者和骗子。 

(至于您在前面说的那些好话,是不能算数的,因为在另一页上您就把那些话一笔勾销了。)为了你们,他在二百年间里被束缚住手脚,为的是让你们获得欧洲的智慧,而当你们获得欧洲的智慧时,你们却居然神气十足地站在被束缚的人民的面前,从文化的髙度观察人民,突然间说人民“不好,消极,很少表现出活力(这是说被束缚的人民),仅仅表现出某些消极的美德,这些美德虽然也以富有生命力的汁液滋补了文学,但实际上却是一文不值,因为人民稍一开始行动,他就立刻成为贪婪者和骗子。”不,本来不该回答阿夫谢延科先生,如果我要回答,那也只是为了承认自己的失策,对此,我在下面将要做出说明。尽管如此,由于已经发表了意见,我认为有必要让读者对阿夫谢延科先生有一些了解。他是那种稍具教养的作家类型,对他的观察是很有趣的,然而,糟糕是,这种类型具有某种普遍意义。

\end{document}
