\documentclass{article}
\usepackage[utf8]{inputenc}
\usepackage{ctex}

\title{胎儿之梦\footnote{Click to View:\url{https://web.archive.org/web/20230409052349/https://www.99csw.com/book/8909/316409.htm}}}
\author{梦野久作}
\date{1935}

% \setCJKmainfont[BoldFont = Noto Sans CJK SC]{Noto Serif CJK SC}
% \setCJKsansfont{Noto Sans CJK SC}
% \setCJKfamilyfont{zhsong}{Noto Serif CJK SC}
% \setCJKfamilyfont{zhhei}{Noto Sans CJK SC}
% \setlength\parindent{0pt}

\begin{document}
\CJKfamily{zhkai}

\maketitle


\Large

——藉人类胎儿代表其他动植物的全部胚胎
。 

——有关宗教、科学、艺术及其他无限广泛内容之考证、援例与文献的说明、注记予以省略,或是
仅止于极端概略述及。 

人类胎儿在母亲胎内十个月期间,都是在作一
场梦。 

这场梦是由胎儿自己担任主角演出,所以应该称为“物种进化的实录”,是有如数亿年、甚至是数
百亿年长度的连续电影。 

故事始于胎儿自身最古老祖先的原始单细胞微
\newpage
生物的生活状态,紧接著为主角的单细胞逐渐变成人类,亦即进化成胎儿形貌的无从想像之漫长岁月,所遭受的惊心骇目天灾地变,抑或自然淘汰、适者生存的窒息般灾难、迫害、艰辛等等体验,是一部由胎儿本身直接又主观描绘的超越想像的奇幻影片。其中当然有实际映现如今已成化石的史前怪异动植物,也有使这些动植物遭致惨死灭种、语言无法形容的壮观天灾地变。另外更描绘出在这样的天灾地变中,残存而进化的原始人类,演化成现在的胎儿之直接双亲为止历代祖先的过程中,所经历深刻、惨痛的生存竞争,以及在各种复杂的欲望驱使下所犯的无数罪孽,结果,这一切化为胎儿的现实罪孽,终至成为极端惊骇颤
栗的大恶梦。 

上述的恶梦,藉著以下所述关于“胎生学”和“梦”两大不可思议现象的解决,已经直接或间接获
得证明。 

首先,人类胎儿在母亲胎盘内之时,一开始显现的形貌与一切生物共同祖先的原始动物相同,只是

\newpage
一个圆细胞。 

这个圆细胞宿于母体胎盘后不久,就分裂增殖
为左右两个细胞,紧密结合成一个生物。 

这左右两个细胞很快又各自分裂增殖为四个,同样紧密结合,摄取来自母体的养分,具备一个生物的功能。像这样,四个、八个、十六个、三十二个、六十四个……呈倍数分裂增殖后紧密结合,逐渐增大,由人类最初祖先的单细胞微生物,在母亲胎盘里依
序反覆进化至人类为止历代祖先们的演变过程。 

最先是鱼的形貌。接下来是鱼的前后鳍变化为
四足,成了匍伏爬行的水陆两栖动物形貌。 

然后是四足更强壮,成为可以四处奔跑的兽类
形貌。 

最后终于尾巴缩人,前足举高化为双手形状,后足直立步行,也就是人的形貌……等到进化至一般
胎儿的形貌,才呱呱出生。 

\newpage

此一顺序所需要的时间每个人尽不相同,但通
常不会差异太大。 

这些在胎生学上已是完全确定的事实,属于无人能否定的现象。但若是如此,所有婴儿为何要在母体胎内反覆进行如此繁复的胎生顺序?为何不在成为人的形貌后直接长大出生呢?另外,最初一个细胞为什么会像事先商量好似的,正确反覆胎生的顺序?也
就是…… 


“是什么让胎儿这么做呢?” 

对此,没有任何一人能够适当加以解释,即使查遍现代的科学书籍也找不到任何答案,只能以“不
可思议”几个字说明。 

第二,一切胎儿像这样毫无差错在母体胎内反覆遂行自己历代祖先进化的过程形貌,但因其经过时间非常短暂,把人类历代祖先的动物历经几百万年、甚至几千万年,由鳍变手足、鳞变毛发……之类的顺序,一点一点进化而来的各时代形貌,在仅仅几秒钟
\newpage
或几分钟可数的时间内反覆经历,这点可以算是无法说明的不可思议了,然而更下可思议的是,如此被浓缩的时间与实际进化的时间比例,却并非毫无道理。
 

亦即,人类胎儿约莫十个月反覆遂行原始以来祖先们的进化历程,但事实上,其他动物通常进化程度越低,其胎生所需的时间也越短,所以进化程度最低的原始时代细菌和其他单细胞动物,大部分完全没有胎生时间,而是以分裂方式变成新的动物,理由何在?还有,进化程度最高的人类胎儿为什么需要最长的胎生时间?换句话说,“是什么让胎儿这么做?”

在想要对这问题加以适当解释时,我们发现以现代的科学知识绝对不可能,同样只能用不可思议来
形容。 

以上是关于胎儿不可思议现象的实例。接下来试著从解刦学来研究观察如上述所形成的人类“肉体
”,同样发现数不胜数的不可思议现象。 

\newpage

亦即,试著从表面观察人类肉体发现,其进化程度愈高,也就是其胎生过程愈慎重进行,外观就比其他动物高尚优美。柔和且带著威严的五官轮廓、美丽的肌肤、匀称的骨架和肌肉,足以被称为万物之灵。但是,如果剥掉其肉体表皮,拔掉肌肉,检查其内脏,解刦其脑髓和五官详细观察,将可明白其各部分的构造,每一样都是承袭自低等动物进化而来的鱼、爬虫、猿猴等历代祖先的生活器官。也就是说,即使一颗牙齿的形状、一根头发的组织,都忠实记录在惊人的漫长岁月中进化而来所受到自然淘汰的迫害,抑或生存竞争的艰难历史,因此,为了鲜明纪念这样的历史,胎儿才会如此反覆进化,将演变成人类形貌的
一切伟大、深刻的记忆注记于每一个细胞中。 

不必说,这种现象已经能够利用进化论、遗传学或解剖学等予以证明,没必要在此详述,问题是,
谁记忆这些事情,让胎儿反覆遂行这种历史演进。 

关于“是什么让胎儿这样做呢?”还是无法说
明,同样只能用不可思议形容。 

\newpage


而且,不仅如此。 

如果进一步观察人类的精神内容,则会更深刻
痛切的证明这样的事实。 

亦即,人类的精神如果也从表面观察,会发现其完美程度绝非其他动物所能比拟,是以自觉“人类为万物之灵”或“文化的骄傲”的一层“人皮”,包覆自己的精神生活内容,施以称之为常识或人格的巧妙化妆,超然而自得其乐。但一旦剥下其表皮——也就是“人皮” 一看,能彻底发现出现在底下的乃是从该人类远祖之微生物演变成现在的人类形貌为止,经历漫长岁月的自然淘汰、生存竞争迫害,所形成的警戒心理或生存竞争心理遗传下来的不同时代的动物
心理样态之事实。 

也就是说,剥掉所谓文化人的表皮——藉著博爱仁慈、正义公道、礼仪制度掩饰的人皮之后,底下
出现的乃是野蛮人或原始人的生活心理。 

最能证明这项事实的人乃是天真无邪的幼儿。
\newpage
尚不知披上文化外皮的幼儿,充分发挥同样不知道披上文化外皮的古代民族之个性。拾起木棒就想玩打仗游戏,是延续历经部落与部落、种族和种族之间战争行为之生存竞争,亦即好战的原始人个性之遗传。也就是说,潜藏在细胞里的野蛮人时代的本能记忆,被木棒这种类似武器之物的暗示刺激而苏醒;见到虫类会毫无意义的追逐,则是见到会动的东西就想追逐的狩猎时代心理暗示刺激诱发;至于把捉到的虫类弄断手脚、撕掉翅膀、挤破肚子、火烤等等,只是处置、玩弄、侮辱猎物,或俘虏以彻底满足胜利感、优越感的古代民族残忍个性记忆的重现。还有,将婴儿置于暗处,婴儿会嚎啕大哭,也只是借不会用火的原始人对满是毒蛇猛兽的黑暗世界的恐惧之复活:另外,随处便溺则为昔日睡在树根或草丛时代的习惯之重现。
这些都可以藉著现代进步的心理学研究加以说明。 

接著,如果继续剥掉野蛮人或原始人另一层皮
,会发现底下溢满畜生,亦即禽兽的个性。 

譬如,同性……也就是陌生的两个男人或两个女人初次见面,表面上会像个人类般互相打招呼,可
\newpage
是内心却显现互相翻白眼,观察对方反应的心理状态,彼此稍不注意,双方马上就会从些微小动作中发现令其不愉快之点,互相皱起鼻头,仿佛街头常见的猫狗互相叫阵般,咒骂对方“畜生”或“禽兽”。另外,在日常生活到处可见比自己弱小之人,忍不住就会想稍微欺负对方:对于妨碍自己行动的人,则希望能有人帮忙杀掉对方;四下无人时,产生想偷窃的念头;偶而想闻一闻他人小便的味道;想埋藏自己的遗物等等如畜生般的心理表现,都是来自于禽兽的个性。

接著,我们再切开此禽兽个性底下的横隔膜,
立刻发现蠕动的虫类心理。 

譬如,企图推落同伴独自爬上高处:绕至无人看见的地方独享美味:做了对自己有利的事隋,立刻想钻进认为最安全的洞穴里;发现营养不错的家伙,会想偷偷接近并且寄生;不管他人感觉,任性做出令人不愉快的动作,力求自我保护:想躲在硬壳里,让敌人无法接近:发现敌人,即使牺牲别人,也尽可能想让自己得救;到了最后关头,挥舞毒针、喷出墨汁、射出小便、放出恶臭,或者利用保护色,幻化为地
\newpage
形地物或比自己强壮者的形状等等,低级、懦弱的人所作之事,皆是这种虫类本能的反应。也就是说,俗谚所谓的“蛆虫”、“米虫”、“爱哭虫”、“吸血虫”、“放屁虫”、“粪虫”、“弱虫”乃是这种虫
类时代心理遗传显现的轻蔑言词。 

最后……是虫类心理的核心。亦即,如果切开人类本能最深处的动物心理核心,将会出现与霉菌及其他微生物共同的原始动物的心理。那是只会无意义生存、无意义行动的活动方式,大多是藉著所谓群众心理、流行心理或看热闹心理来表现。如果意义拆开其行动单独观察,会发现似乎完全无意义,可是一旦集合多数,却产生如同多数霉菌聚集同样恐怖的作用!也就是往发光之物、高明之物、大声之物、道理简单之物、刺激明显之物等崭新且易了解之物群聚,但是当然没有判断力,也无理解力,与置于显微镜下的微生物同样无自觉、无主见,恍恍惚惚聚成大群体,虽有无意义的感激、夸耀和安心,最后却毫无作为的突然浸身感激之中舍弃自己的生命……献身于暴动、革命等心理,不过是与这种集中于一滴苹果酸的微生

\newpage
物相同。 

人类的心理在这时候才首度接近物理或化学方式的运动变化法则,亦即,因为和无生物只有些微差异,因此从事政治或其他拉拢人心职业的人物,所利
用的就是这种属于人性本位的霉菌特性流露。 

我们人类的精神生活就是,在上述各种心理之中,以最低级、单纯者为中心逐一向外,藉由高级复杂的动物心理包裹,最外层再包裹所谓的人皮,用交际、制度、身分家世、面子人格等等蝴蝶结或标签装饰,施加化妆,喷洒香水,然后昂首阔步于马路上。但若是解剖其内容,绝大部分就如前面所述,只是重
现潜藏在人体细胞中历代祖先的动物心理记忆。 

但是,如同前述的肉体解剖观察,问题在于:胎儿如何能够将这样千万无数、复杂多样的心理记忆
,包容于细胞潜在意识或本能之中呢? 

还是没办法说明“是什么让胎儿这么做?”。不,甚至一个人的精神内容乃是过去数亿年间的万物进化遗迹的这项事实,都被“人类是万物主灵”或“
\newpage
我是最高等的人类”的浅薄自以为是之态度所掩蔽,
处于完全未被注意的状态。 

以上列举胎儿的胎生;以及因胎生而完成的成人肉体和精神上出现有关万物进化遗迹之不可思议现象。接下来则是观察人类所做的“梦”之不可思议现
象。 

所谓的梦,自古以来就被视为是不可思议的代表,因此如果碰上一点意外的事,马上会认为“这是不是在作梦?”。见到和实际事物有些差异的奇妙景象,或是出现无法想像的特异、不自然的风景或物品,这些不合于现实世界的心理或物理法则之景物,若是根据连神话或传说也没有的奇想法则,该景物立刻千变万化,因此有关梦的真相和梦中的心理、景象变化法则,困扰古今不知多少学者专家,在此列举以下
三项梦的特徽,当作解明梦的本质、真相的线索。 

(一)梦中所发生的事情在进行变换之间,经常出现非常不合情理的部分,不,甚至能说这样的情形实在太多,所以才会认为这种超自然景象、物体的
\newpage
不合理活跃、转变就是梦。虽然如此,可是在作梦之时,不仅对梦中发生之事不会怀疑其超自然、不合理
,反而严肃感受到更为现实的深刻痛切。 

(二)以与现实同样的感觉,表现出至今从未
见过的风景或天灾地变。 

(三)梦中出现的事情即使是感觉上有如几年或几十年漫长的连续事件,事实上,现代科学已证明
,作梦的时间仅仅只有几分钟或几秒钟的短暂。 

以上列举有关“胎儿”与“梦”之各种不可思议的现象,乃是无人能够否定的科学界大疑问。但是,这样不可思议的现象,为什么迄今未能解决?为什么迄今犹未找到解决的关键?其中有两个原因存在。

其一,以前的学者对于有关让人类胎生、而且令因胎生而完成的成人作梦的人体细胞之观念完全不同。另一则是,一般人类对于流动在宇宙间的“时间
”观念,有根本上的差异。 

\newpage

换句话说,组成人体的每一个细胞内容比一个人类的内容还伟大,不,甚至是拥有能够和整个宇宙相比较的完美伟大之内容和性能。所以利用显微镜从外观察一个细胞,以化学方式分析其成份,藉其型态、色彩的变化研究其分裂、繁殖的状况等等老旧的唯物科学方式,当然无法了解细胞之内容与性能的伟大。这就像漠视英雄、伟人生前的功绩,只观察其尸体的外貌、解剖内部,企图确定其伟大个性和性能似的
,根本就是缘木求鱼。 

另外,对于所谓的时间也相同。中央气象台、我们身上的腕表、地球与太阳的自转公转等显示的时间,并非真实的时间,只是唯物科学擅自制造的人造时间,属于错觉的时间、伪冒的时间。所谓的真实时间应该不是这种无聊的尺寸所能局限,而是变幻自如、玄奇不可思议的东西。如果人们能够认同此项事实,应该同时也能够认同“胎儿之梦”的存在,当然也
就掌握揭开生命之神秘、宇宙之谜团的关键。 

本来,细胞就是只有人体约莫几十兆分之一、小度数显微镜无法捕捉到的微小颗粒,所以其内容的
\newpage
复杂或表现力的程度,应该也是人类整体能力的几十兆分之一……不管如何,细咆是极端的单纯无力。这是至今为止,大部分科学家根深蒂固的观念!因此,当细胞不可思议的生存、繁殖、遗传等能力陆续被发现时,科学家们都为此惊异万分:可是,其研究依然仅止于用显微镜观察、藉化学方式分析的范围,亦即仅限于唯物科学能说明的范围。这样当然无法跨越细胞是人体几十兆分之一程度的单纯无力之概念。他们甚至觉得,若更进一步研究,就等于冒渎唯物科学,
是身为学者专家的罪恶。 

伹这却是拘泥于唯物科学理论的学者专家基于形状大小来判断细胞的内容和能力、认为“应该就这么多吧!”的极端不合理推论之先人为主观念所产生的错误。生命的神秘、梦的不可思议等科学界大谜团之所以长久无解而残存至今,就是因为拘泥于这种“井底之蛙”式的不自由、不合理唯物论……换一个方式说明,这就是因为藉著过度拘泥于科学的非科学方式研究方法,想要研究广大无边的生命主体细胞之结果。所以我们必须一扫这种旧式的学问常识和对于受拘泥之唯物理论的迷信,用更不受局限、更自由的态
\newpage
度观察宇宙万物,同时把这个问题与更适切明了的实际现象栢对照,如此将会发现一个细胞的内容,远比利用显微镜或在化学实验室里观测、计量所得的内容更加深刻伟大,甚至与全宇宙相比都毫不逊色之事实。也就是,藉著超越现代的真实科学知识,我们必须直接面对迷信唯物科学研究观察法的人们一心一意想
否定却又无法否定的事实。 

首先必须举出的就是,细胞具有创造人类的能力。亦即,身为生命种子、宿于母体胎内的唯二个细胞,依前述顺序分裂增殖,循著历代祖先的进化脚步成长,回想那边是那样、这里是这样的依照鱼、蜥蜴、猿猴、人类的顺序,正确无误的创造自己。虽不能一概而论,但仍旧尽可能综合双亲的优点或长处,努力希望能有多一点点进步,所以虽然每个人的眼耳口鼻位置尽皆栢同,却仍能具备“这是我的儿子”、“酷似他父亲”、“脾气和他父亲一模一样”、“记忆力和我一样好”等等微妙调和。另外,请看每一个细胞的惊人记忆力、柑互间的共鸣力、判断力、推理力、向上力、良心,甚至灵性艺术的批判力等等,是何等深刻!还有,这些细胞的大集团——人类——接触
\newpage
宇宙万物而予以理解,并产生共鸣,创立国家社会这个大群体,共同一致塑造人类文化,其创造力又是何等深远广大!这种几乎可谓全知全能的伟大作用,总归一句,只能认为是最初唯一一个细胞的灵能显现,换言之,现代人类如此广大无边的文化,若深究其根本,只下过是一个存在于显微镜底下的细胞所含有的
灵能反映于整个地球表面而已。 

◇备注:具有如此伟大内容的细胞大集团,透过脑髓的仲介,将其灵能在细胞共同的意识下统一而成的就是人类。所以显现出的知识、感情、意志等,照理必须比每一个细胞的知识、感情、意志等更完美,但事实上正好相反。所以自从有了世界以来,无论任何贤人或伟人,面对细胞的伟大灵能总是形同无力,恰似星星在太阳面前必须跪拜一样……亦即,统一成为人类形貌的细胞大集团的能力,呈现不到其几十兆分之一的细胞能力的几十兆分之一的怪异现象。这可以视作由于人类身体各部位细胞灵能之统一机构——也就是脑髓——的作用尚未充分进化的缘故,导致细胞灵能的充分活跃受到妨碍:同时,也能够认为是地球上最初出现生命种子的单细胞在地球上最初出现
\newpage
的唯三昙忌(?)和其无限灵能,历经将灵能具体反映于地面的种种过程,进化至最有利、最有能力的人类后,又会继续进化至更有利、更有能力的生物,现在的人类只不过是过渡期未完成的生物,因此才会出现这种矛盾、不合理的怪异现象。这些是非常重大的研究事项,不是一朝一夕得以说明完全,所以在此只
是作为参考。 

一旦明白人类肉体及精神与细胞的灵能关系,
则有关“梦”之本质的说明就容易了解了。 

近代医学已经证明,所有每一个细胞与我们一个人的生命同样拥有——甚至超越其上—的意识内容与灵能。因此,全部细胞只要从事某项工作,就会伴随吸收养分、发育、分裂、繁殖、疲劳、老死、分解、消灭。而且,每一个细胞本身在工作、发育、分裂、繁殖、疲劳、老死、分解、消灭之间与我们个人一样,甚至更强烈意识到对其工作的苦乐,同时对这样的苦乐与我们个人感受的相同,或有超乎其上的联想、想像与幻想,就好像一个国家从兴盛至衰亡之间会

\newpage
留下无数艺术作品一样。 


证明这项事实的就是我们所作的梦。 

所谓的梦,本来就是人类全身在睡眠时,体内某一部分细胞的灵能受到某种刺激而苏醒,开始活跃后,苏醒的细胞本身的意识状态反映于脑髓,留存于
记忆之中。 

譬如,人类吃下不消化的东西后睡觉,这时只有胃细胞苏醒,开始工作,同时不断表示不满,发牢骚:“啊,好难过,做也做下完,这到底是怎么一回事”,或“为什么只有我们必须受苦”等等,于是胃细胞的痛苦和不满情绪就会化为一种联想,反映在脑髓。亦即,恰似觉得受苦的主角无辜被送进牢狱,铐上锁链,又呻吟地扛著超过体力所能负荷的大石头之际,还碰上不可抗力的大地震,被压在房屋底下挣扎……不久,痛苦的消化工作转为轻松,总算松一口气时,梦中的情绪反映于脑髓,联想、幻想的内容随之转为轻松,成为在山顶观看日出或雪橇滑雪的欢乐心
情。 

\newpage

另外,如果睡前想著“真希望见到她”而闭上眼,那么因这一念的官能刺激而难以入睡,冲动的想去找她却怎么也没办法去的焦急心情就会化为梦境描绘出来。她的容貌藉著美丽的花或鸟或风景来象征,在他梦中灿笑,可是一旦想取得时,却出现各种阻碍而无法接近。这时,不是留存记忆中的远古时代之天灾地变突然出现眼前,就是看见猿人祖先所居住的高山断崖,其中有时会感受祖父落魄乞食的心境,有时则是父亲泳渡大河的情景,也可能变成猿猴攀山越岭,或化身为鱼横渡大海,费尽千辛万苦终于能够到她——花或鸟等美好事物……最后,因为最初刺激心理
消失,梦境结束,人随之清醒过来。 

此外,因为尿床而梦见远古时代大洪水;因为鼻塞而在梦中重新描绘少年时代差点溺死的痛苦等等,像这样不管是手、脚、内脏、或皮肤的一部分都无所谓,当全身熟睡之间,受到某种刺激而苏醒的细胞,一定会联想、幻想、妄想与该刺激栢对应的对象作著某种梦。也就是,呼应细胞每个时候的情绪,从细胞本身传自历代祖先的记忆、或细胞过去的记忆,随便唤起类似的场景或情景,描绘出最深刻且痛切的那
\newpage
种情绪。如果该情绪属于非常识或变态,无法找到表现呼应的联想材料时,马上以想像的物品、风景替代。为了表现人体内细胞独特的恐惧和不安,会联想到像蚯蚓或蛇之类弯曲的厨房器具:为了表现痛苦,会描绘滴著鲜血的大树或在火焰中盛开的花朵。这和不
知神秘内幕的人类会想出长著翅膀的天使一样! 

这与我们清醒时的心情会受周遭状况支配而变化正好相反。在梦中,心情会先转移变化,随著心情变化,适合该心情的景象、物品、场景会不断跟著千变万化,尽管变化如何突兀、不合隋理,其间也不会感到丝毫矛盾或不自然。不但如此,还觉得比现实印
象更自然深刻。 

换句话说,所谓的梦乃是细胞独特的艺术,毫无条理的组合起象徽著唯有梦之主角的细胞本身才能了解的所有影像、物体的记忆、幻觉、联想,然后极
端清晰地描绘心情的变化。 

◇备注:近代欧美国家的各种艺术倾向,常藉著无意义或是片段的色彩音响,抑或突兀的景象、物
\newpage
体的组合,企图表现比旧有的常识性表现方式更深刻
的心情,这与梦的表现方式逐渐接近。 

梦的真相如以上所说明,乃是伴随著细胞的发育、分裂、繁殖,将细胞本身的意识内容反映于脑髓。接下来则说明在梦中感受的时间和实际不一致的理由。亦即,一般人相信靠时钟或太阳显示的时间乃是真正的时间,才会产生非常严重的错觉,惊愕于真正
的科学判断。这样,应该足以解释这项疑问吧! 

依据现代医学,将普通人平静的呼吸十八次,或是脉搏跳动七十几下所经过的时间定为一分钟,规定其六十倍为一小时,一小时的二十四倍为一日,一日的三百六十几倍为一年。同时因为一年也相当于地球绕行太阳一周的时间,因此有信用的公司所制造的钟表,其显示的时间就成为具有公信力的时间。但是,这主要还是人造的时间,所谓真正时间并非这种东西,证据是,如果由不同人分别使用这种同样长度的
时间,将出现极大的差异。 

举一个手边随处可见的例子。即使用同一时钟
\newpage
计算一小时,阅读有趣小说的一小时与在车站茫然等待火车进站的一小时,长度上有著相当惊人的差异:用竹尺计量物品一尺的长度,并非所有人看起来都是一样长度;潜水闭气的一分钟,和闲话家常的一分钟比较,前者感觉漫长得令人无法忍受,后者却几乎不
到一瞬间……这些绝对都是事实。 

再进一步说明,假定这儿有一个死人,该死人在死后也能够藉著其无知觉的感觉感受到时间的流逝,则其一秒钟的长度应该会和一亿年的长度相同。另外,这样的感受必须足死后的真实感受,所以等于感受到一秒钟包含一亿年,同时也在一秒钟感受宇宙寿命的长度。流泄在无限宇宙的恒常时间之真面目,就是如此极端的错觉。在无限的真实背后,时而如箭矢
般静止,时而似飞石般疾觎。 

所谓的真实时间和一般人认为的人造时间完全不同!别说是和太阳、地球及其他天体的运行,或是时针的旋转等完全毫无关系,而是对于所有无边无量的生命之个别感觉,同时个别地以无限伸缩自如的静

\newpage
止或流动。 

接下来,试著比较存在地球上的生命长度。在几百年之间,从繁荣茂盛的植物、生存百年以上的大型动物,至仅仅生存几分钟或几秒钟的微生物为止,大体说来,形状愈小者其寿命愈短。细胞也是栢同,在人体个别的细胞中,平均取出寿命较长与寿命较短的细胞,试著比较人类整体生命的长度,能够发现有如国家的生命与个人的生命之差异。但是,这些或长或短的各种细胞生命,其主观感受的一生长度完全相同,不管其由生至死的时间以人造时间计算是一分钟
或是一百年,丝毫不受影响。 

历经出生、成长、生殖、衰老、死亡而感受到的实际时间长度,同样都是一生的长度。下知道此种道理,将朝生暮死的婴儿之悲哀,与同样朝生暮死的昆虫生命相比较而觉得绝望,未免显得愚奸、不自然、又不合理,毕竟,这只是将毫无通融的人造时间和
无限伸缩自如的天然时间混淆思考的悲喜剧。 

一切的自然……一切的生物把这种无限伸缩自如的天然时间依各自所需的长度占领,视之为一生的
\newpage
长度而呼吸、成长、繁殖、老死。同样地,形成人体的细胞寿命,即使以人造时间计算是何等短暂,其占领的天然时间也必然是无限,因此若细胞使用无限的记忆内容和无限的时间大幅描绘“梦”,很轻松就可以在一瞬间、一秒之间描绘出五十年或一百年之间所发生的事情。在中国古老传说、流传至日本的“邯郸一梦”中,卢生一梦五十年其实只是粟饭一炊的时间
,这样的事实一点不可思议之处也没有。 

根据以上所述,各位应该能理解仅只是一个细胞的灵能是何等无限,尤其是其中的“细胞的记忆力”是多么深刻无量吧!在认同让人类的肉体和精神同时胎生完成的“细胞的记忆力”大作用之时,相信有关于“胎儿之梦”当中“是什么让胎儿这么做?”的
疑问,应该能冰释大部分吧! 

胎儿因为在母亲胎内,对于外界的感觉完全绝缘,处于与沉睡同样的状态。在其间,胎儿的细胞旺盛分裂、繁殖、进化,竭尽全力只为了“迈向成为人类之路” ,反覆思索祖先进化当时的记忆,持续将当时情景反映于胎儿的意识。如前所述,藉著母体胎
\newpage
盘完全隔离外界刺激的同时,又极端平静地受到保护,可以下必考虑任何事情,一心一意守住“迈向成为人类之路”的梦即可,所以梦的内容也会极端顺利、正确、精细的转移,这点乃是与任性奔放又自在的成
人之梦下同之处。 

若将这种情形反过来说明,那么,创造胎儿的是胎儿之梦,支配胎儿之梦的则是“细胞的记忆力”。胎儿在母亲胎盘内进化的过程和所需的时间相同且固定,这是由于现在人类是由某共同的祖先进化而来,因此细胞的记忆,亦即“胎儿之梦”的长度相同且固定。另外,这种长达数亿甚至数十亿年的“胎儿之梦”能用仅仅十个月作梦梦见,若参考前述的细胞灵能,绝非奇怪之事,而进化程度较低的动物之胎生时间较短,主要是因为该动物的进化回忆比较简单的缘故。因此,自原始以来未完成任何进化的低等微生物完全没有“胎儿之梦”,其理由也是因为它们仍旧与
其祖先一样在一瞬间分裂、繁殖。 

◇备注:上述的事实,也就是“细胞的记忆力”和其他的细胞灵能是何等的深刻微妙?对于一切生
\newpage
物的子孙之轮回转世具有何等深远影响?如何支配万物的命运?这些从几千年前以埃及一神教为本源的各种经典上都已有所叙述,因此现在世界各地苟廷残喘而成形的所谓宗教,就是粉饰这种科学观察、为求方
便教导未开化民族而予以迷信化的残骸。 

所以胎儿之梦的存在绝对不是新学说,特此记
之。 

那么,如果具体说明并未留在我们记忆的“胎
儿之梦”内容,其大概内容又是如何呢? 

虽然对照前此所述的各项,应该能够充分推测,但为了参考起见,必须试著说明笔者自己的推测内
容。 

人类胎儿在母体胎盘内所作有关历代祖先进化
之梦中,最常做的必须是恶梦。 

原因何在呢?因为所谓人类这种动物,在进化至今日的程度为止,自身完全没有装置像牛那样的角
\newpage
、像虎那样的爪牙、鸟的翅膀、鱼类的保护色、虫类的毒液、贝类的壳等等天然护身或攻击的道具,与其他动物相比,肉体很明显的柔弱、无害、无毒、无特征,可是却还能够据此暴露于所有激烈生存竞争的场合,与各种天灾地变缠斗,终于进化至像今日这样的最高等动物。这中间,可以想像应该体验过其他动物所难以比拟的生存竞争之痛苦与自然淘汰之迫害,因此其艰难辛苦的回忆绝对无可计数,其中,胎儿二清楚作著属于自己过去和自己同姓的历代祖先长达几亿、几千万年的深刻回忆之梦,又感受到如实际时间更缓慢成长,其辛苦绝非其祖先们在这个世间所感受到
的辛苦那样的短暂、肤浅,不是吗? 

首先,人类种子的一个细胞是与一切生物共同祖先相同的微生物,在子宫内壁的某一点著床后下久,随即开始做著与几亿年前无生代的无数微生物同伴浮游于温暖水中的梦。这种无量数、无限数下胜数微生物群体的每一个,其透明的身体吸收反射天空的强光,有的散发七色彩虹,有的射出金银色光芒,享受地球上最初生命的自由,漫无目的浮游、旋转、摇曳,在每一瞬间分裂、生存、死亡,那是何等果敢、欢
\newpage
乐、美好?但是……不久,所居住的水域发生微妙变化,无法形容的莫大痛苦袭来,大多数同伴瞬间死亡,自己也想逃生,可是全身却被痛苦束缚而无法动弹。好下容易挣脱这种痛苦、折磨,却又受到原始太阳如烈火般追迫,苍白月光如寒冰般穿透,或被狂风吹散于无边无际的虚空,被暴雨打落无间地狱。它们饱受这种无法想像的恐怖与不知生死的苦恼所玩弄,苦闷于“啊,我希望让自己变得更强壮”、“我希望身体能够忍受寒暑”的挣扎,细胞开始逐渐分裂增大,终于变成紧接于人类祖先的鱼类形貌。也就是说,完全拥有能够抵抗寒暑的皮肤和鳞、善于游泳的鳍和尾巴、嘴和眼睛、能够判断事物的神经等等,产生非常
惊人进步的形貌。 

但是当它得意的在浪边散步,炫耀“啊,太好啦,这样没什么好抱怨了,没有生物可以比我更完美”时,却发现比自己身体不知巨大多少倍的章鱼,伸开足以遮天的手,袭向自己。“哇,救命”,它逃进海藻林中闭住呼吸,不敢出声,好不容易才得救,松了一口气,慢慢抬起头时,却发现比章鱼更巨大几十倍的海蝎,它的巨钳已逼近眼前。它慌忙转身想逃,
\newpage
这时,三叶虫像云一般向自己背上覆盖,海葵从一旁剠出毒枪,奸不容易即时脱身潜入小石头下,全身发抖的向一同进化的生物同伴大叫,说:“啊,太可怕
啦,这样还不能安心生存。” 

可是同伴却说它大惊小怪,于是它只好将自己的身体用硬壳包覆住,只将手脚从岩石间伸出。对于自己好不容易历经进化至此,却必须在这种黑暗沉闷的水底忍受煎熬,它觉得非常不甘心,就拚命祈祷“我希望拥有能尽早到陆地,在那样轻快、明亮的空气中自由畅快跳跃的身体”,终于变化为有如小小的三
眼蜥蜴那样的东西,跳上陆地。 

“啊,好高兴啊,真好”……它四处跑跳不久,却遇上几乎令世界消失的火山爆发、大地震、大海啸从四面八方袭来,海洋沸腾如开水无路可逃,只能在火烫的砂地上痛苦奔逃,好不容易勉强躲过灾难,这回却置身如山一般高大的巨龙脚趾下,被翼龙的翅膀挥开老远,几乎被始祖鸟像妖怪般的巨嘴啄到。“啊,实在受不了”它大叫。一同进化的同伴有的身体长刺,有的变化为与附近生物同样的颜色或形状,有
\newpage
的披上盔甲,有的喷出毒气,可是它却下愿同样的苟活……它躲在石缝间一心一意祈祷,终于头顶上的一只眼睛消失,进化成两只眼睛的猿猴形貌,跳跃于树
梢之间。 

“太好了,这样就没问题啦,应该没有比我更自由自在、进步的生物”,它在树梢上用小手遮眼往四处观望,想不到背后一条蟒蛇袭来,它吓了一跳逃走,头顶上一只大鹫鹰低空飞掠而来,它在千钧一发之际勉强沿著枝哑逃脱,想不到虱虫开始在全身乱窜叮咬,山蛭也过来吸血,下管醒或睡都无法安稳,马上又有覆盖天地的大雷雨、大飓风、大冰雪肆虐大地。“啊,好无奈,我又没做过什么坏事,为何要遭受这样的折磨”、“真希望变成更健壮、可以不担心这种灾难的身体”……它把头埋入树洞里祈祷,终于尾
巴掉落,进化成为人类形貌。 

“好高兴,这样真的能过著极乐生活了吧!”

可是,为什么?为什么呢?梦为什么犹未结束?进化为人类形貌后,不久,又开始作人类的恶梦。
\newpage


身为胎儿祖先的人们,由于彼此的生存竞争,以及想要完遂遗传自原始人的残忍卑劣畜兽心理和各种私利私欲,犯下直接间接折磨他人的各种大小罪虐,而这些血腥恐怖的记忆二化为胎儿现在的王观重现眼前。弑君夺城、饮酒欣赏忠臣切腹、毒杀夫人和储君让自己的孙子继承,或是毒害生病的丈夫、与仇敌上床、闷死刚出生的私生子等等难以忍受的喜悦:嫁罪给媳妇,让她上吊自杀的愉快;把可恨的继子推落井中的痛快:折磨多位亲生女儿的有趣;让有妇之夫失恋自杀的骄傲;聚集美少年和美少女虐待的乐趣:花掉重要金钱的愉快:同性之爱的深刻;人肉的美味;毒药实验;背叛行为;尝试杀人;欺负弱者……等等各种令人难以接受的景象,化为眼前的梦逐渐转变。另外,自己的祖先们——过去的胎儿——隐藏的犯罪行为和无法告诉别人的无数秘密,变成血肉馍糊的脸孔、无头的尸体、井中的毛发、天花板上的短刀、沼泽里的白骨等等,逐一出现梦中,这时胎儿会惊骇
、恐惧、苦闷,在母亲胎盘内舞动手脚。 

像这样,胎儿作梦至自己父母这一代,终于没
\newpage
有应该作的梦了,这才陷入深深熟睡,不久母亲开始阵痛,他被推出子宫外。空气进入胎儿的肺部瞬间,潜逃至胎儿潜意识深处、与先前截然下同的表面且强烈痛切的现实意识遂渗透至全身,胎儿惊惶,害怕得哭泣出声。似此,胎儿——婴儿——终于接受父母完全的慈爱,开始和人类的和平之梦连结,然后逐渐清醒过来,让“胎儿之梦”续集化为创作自己本身的现
实。 

应该没有任何记忆的婴儿在熟睡之间会突然害怕哭泣,或像想到什么般的微笑,可以认为他是梦见了在母亲胎盘内尚未做完的“胎儿之梦”。至于一出生就四肢不全或精神有缺陷,在其胎生时代应该存在著相当足以说明原因的梦。另外,在母亲胎盘内常常发现只留下胎骨,或是牢牢缠在一起的毛发和牙齿的所谓“鬼胎”,必须认为那是胎儿之梦不知何种原因
停顿,或是急遽发展,最后断绝所留下的残骸。 

以上。换的内容

\end{document}
