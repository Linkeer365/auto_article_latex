\documentclass{article}
\usepackage[utf8]{inputenc}
\usepackage{ctex}

\title{忧郁·四首\footnote{Click to View:\url{https://web.archive.org/web/20230509131721/https://www.douban.com/note/356235437/?_i=3637764d7R2Q8h}}}
\author{夏尔·皮埃尔·波德莱尔}
\date{1857}

% \setCJKmainfont[BoldFont = Noto Sans CJK SC]{Noto Serif CJK SC}
% \setCJKsansfont{Noto Sans CJK SC}
% \setCJKfamilyfont{zhsong}{Noto Serif CJK SC}
% \setCJKfamilyfont{zhhei}{Noto Sans CJK SC}
% \setlength\parindent{0pt}

\begin{document}
\CJKfamily{zhkai}

\maketitle

\setlength\parindent{0pt}


\Large


《忧郁》之一 \\ 


雨月对着全城大发雷霆,\\
向寒茔四周惨淡的亡灵\\
倾盆泼洒出凄冷的黑影,
\\
又将四郊笼罩死亡阴濛。 \\ 


猫在砖地上寻觅遮身草茎,\\
羸瘦生疮的身子瑟瑟抖动;\\
檐槽里游荡着诗叟的魂灵,
\\
发出幽灵一般寒颤的悲声。 \\ 


大钟在悲鸣,钟摆已患伤风,\\
冒烟的木炭伴着它发出假声,
\\
\newpage

可水肿老妇宿命的遗产之中 \\ 


仍旧有浊香的牌局如常进行,\\
黑桃皇后与英俊的红桃侍从
\\
正诡谲地倾诉着已逝的爱情。 \\ 


《忧郁》之二 \\ 


我似乎拥有超越千年的回忆。 \\ 


账单塞满大柜抽屉,夹杂着\\
诗稿、情书、诉状、浪漫曲,\\
厚重卷发缠裹的一摞摞单据\\
比不上我头脑中愁苦的秘密。\\
这是座金字塔,通阔的墓地,\\
尸骸比公共墓地里还要拥挤。\\
——我这坟地,月亮也鄙夷,\\
里面蠕动着悔恨一般的长蛆,\\
向我逝去的亲人们不断攻击。\\
我是破败的香闺,残花满地,\\
过时式样杂乱堆积,到处是\\
布歇的苍白和水粉画的怨艾,
\newpage

\\
开盖的香水瓶孤独弥漫香气。 \\ 


跛脚的日子,真是漫长无比,\\
冰封岁月,当雪花飘飞如絮,\\
厌倦,这忧郁和闲愁的果实\\
便如不灭的永生,轮回不息。\\
——行尸走肉的你呵!无非\\
顽石一块,独对朦胧的恐惧,\\
昏睡在莽莽撒哈拉沙漠腹地;\\
如古老斯芬克司被地图忘记,\\
遭无忧世界遗弃,愤世之心
\\
面对夕阳,也只能长歌太息。 \\ 


《忧郁》之三 \\ 


我好似多雨之国的国王,\\
富而无能,年少却老迈苍苍,\\
既蔑视师傅们卑微俯首,\\
又厌恶将狗和其他宠物豢养。\\
无论阳台前待毙子民或\\
架鹰巡狩,都不能令他欢畅。\\
\newpage

弄臣们滑稽可笑的演唱,\\
也难安抚这残暴病人的愁肠;\\
百合花纹龙床变为坟场,\\
使女们眼中君王都容貌俊朗,\\
再想不出如何妖冶扮妆,\\
博得这年轻的骷髅龙颜舒畅。\\
炼丹术士再也回天无力,\\
他体内的腐恶毒素无法排光,\\
流传自罗马的血浴秘方,\\
那可是王者暮年难圆的奢望,\\
再不能温暖这愚钝皮囊,
\\
无血之躯只有忘川绿汁流淌。 \\ 


《忧郁》之四 \\ 


当低沉的天空像一个大盖\\
罩住被无穷的烦扰折磨而幽咽的心灵,\\
当环抱万物的天际
\\
向我们喷出比夜还要凄冷的黑影; \\ 


当大地变成一间阴湿的牢房,\\
\newpage

那里,希望像蝙蝠在低翔,\\
双翅胆怯地拍打着四壁,
\\
脑袋与腐朽的房顶相撞; \\ 


当大雨倾泻如注,\\
就像大牢狱的铁栅栏一样,\\
一群无声、肮脏的蜘蛛爬过来,
\\
在我们的脑中结网; \\ 


突然间,几口大钟狂怒地跃起\\
向天穹发出可怕的轰鸣,\\
如同没有祖国的游魂
\\
在顽强、固执地哀吟。 \\ 


——几列长长的柩车,鼓乐全无,\\
缓慢地驶过我的心灵;希望\\
被击败,正在哭泣,而残忍暴戾的愁苦在我低垂的头上竖起了黑旌。

\end{document}
