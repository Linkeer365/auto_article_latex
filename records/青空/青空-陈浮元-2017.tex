\documentclass{article}
\usepackage[utf8]{inputenc}
\usepackage{ctex}

\title{青空\footnote{Click to View:\url{https://web.archive.org/web/20230217013906/https://www.pgsk.com/book/chapter/65478/2.html}}}
\author{陈浮元}
\date{2017-04}

% \setCJKmainfont[BoldFont = Noto Sans CJK SC]{Noto Serif CJK SC}
% \setCJKsansfont{Noto Sans CJK SC}
% \setCJKfamilyfont{zhsong}{Noto Serif CJK SC}
% \setCJKfamilyfont{zhhei}{Noto Sans CJK SC}
% \setlength\parindent{0pt}

\begin{document}
\CJKfamily{zhkai}

\maketitle


\Large


我畏惧压在我们头顶上的天青色。 

当我三岁的时候,我妈告诉我要去交一些朋友,要不然会孤单寂寞。所以,我才会站在这片林子最矮的意杨树上,垂头看着自己深色的脚爪在浅褐的枝干上,左左右右来回不安地挪。林子中央有一群乌泱泱的挤在一块儿聊着葛庄的八卦,所有滑稽不滑稽的事情都能令他们发出桀桀的笑声,我习惯性地撩起翅膀将脑袋埋在深灰色的羽翼上,真是一群聒噪的乌鸦
。 

“哟,那缩头缩脑的,不是摔坏脑子的灰毛斑
鸠么。” 

你看,他们总喜欢给异类起绰号,好像这就能
\newpage
给他们在群体中平添许多的见多识广。与众不同代表着不可知,不可知对于一帮觅食遛弯不超方圆三里地
的乌鸦来说,将是个巨大的灾难。 

我的故事开始于我那个非常美丽但是喜欢瞎絮叨的妈,当整片意杨林的鸟都在窃窃私语谁让她怀了蛋的时候,在我还囤在她屁股底下的那段时日里,她逢鸟就叨哔自己孩儿他爹,就是那只括苍山上昂旋飞翔的苍鹰。乌鸦是一种充满极致的八卦炸裂的小群体

我妈秃着半边脑袋孤零零地窝在那棵歪了脖子的意杨上,表里如一的歪脖子没办法坚固到足以抵御葛庄那帮狗厌人弃的熊孩子们的摧残,我被他们轻而易举地从窝里打了下来,那时候还小,小到站不稳脚跟用不了翅膀,风猎猎地灌满耳际,摔到地上后,从内里传来的钝痛,像是一只被囚禁着想要撕开屏障的巨兽。过了许久我破风箱一样的身体才开始呼哧呼哧大口喘气,我失去了声音,没办法将撕心裂肺的痛楚发泄出来。身边的蚂蚁发出窸窸窣窣的声音,搬运着各种树屑蜿蜒前行,我看到了其中一只小个子,慢慢地抬起了自己的椭圆的脑袋,看向莫测的青空。我是
\newpage
世间的渺小,但却是蚁前的巨兽。顺着他小脑袋的方向,我看到了葛庄背面的括苍山,郁郁葱葱如蛰伏的
巨人的剪影。 

冬去春又来,那是我第一次看到他,鹰隼振翅划过蔚蓝天际,他在人类的头顶盘旋。山顶的诗人抬手理了理身上落满的鹰灰,不顾灌满衣袖的肆意山风,向更远的远方前行。浮尘嚣世,有人来去匆忙,有人嬉笑怒骂,有人爱恨别离,所有的繁华,皆如前尘旧事般,装点着人间无人问津的角落。有些时候明明
当作笑话听的故事,偏偏被我们记在了心里。 

习惯是一种可怕的东西,我习惯了自己长短不一的瘸腿,习惯了鸟儿们时不时的奚落,也许有一天我真的会以为自己是一只乌鸦和苍鹰杂交的斑鸠。然而总有一些例外,我仍旧不习惯装模作样地拥有很多朋友,我情愿待在孤单中,哪怕要忍受因为寂寞无助空虚的折磨,那是我为数不多的自尊,我不想以耻辱为代价去换取表面的朋友。我无意鄙薄别处的品格和处世方式,但是我相信一个个体的低下与高贵能通过

\newpage
最微渺的细节呈现出来。 

我不愿扑入你们的娑婆世界,也不愿躺进他们的十丈软红。可是我也有理想,呵,你可别说我比扁桃仁还小的脑容量谈何理想,人是人他妈生的,鸟是鸟他妈生的。我就这么每天在时光的间隙里面打着盹儿,那些每天谴责我妈是精魅的母乌鸦们,鸟嘴日渐脱落,那些时不时扔些野果糜肉在我妈窝前的公乌鸦们,羽毛日渐稀疏。我相信,今年将会有一个最难挨
的冬天。 

为了寻找食物,我飞得越来越远。夏天的时候,塘里好多傻鱼会冒出来透气,我真真是喜欢他们致密紧实的肌肉,以至于我现在都能在脑海里勾勒出,撕开的肉质上那种别致的纹理,流畅的线条,还有用爪子按在他们躯体上,弓起躯骨生动的跳跃,连味道不怎么好的鳞片都在阳光下发出七彩的光。而今年的冬天,连死在外面的动物都很少。大雪覆盖了地面,我聚精会神地搜索着目光之内能下嘴的食物,赤色的夕阳慢慢隐入地平线,我站在一棵白皑皑的松树上,银装素裹的天地间,我朝着家的方向振翅起飞。我小心翼翼地将嘴里的松果放在我妈的面前,打开翅膀盖
\newpage
在她落满雪早就僵硬的身体上:“妈,妈你说天气冷了要冬眠,那我就陪着你等待春天好不好,只要你醒来,只要你能再睁开眼。”我觉得我变得越来越强壮,我能从乌鸦口中抢夺下更多的食物,窝里堆满的松果时不时地掉落在地上。当歪脖子开出第一朵绿芽,我闭着眼将头埋在她干枯的羽毛上,是不是灵魂走了
,身体就会干瘪,那灵魂去了哪里? 

阳光透过树叶的间隙闪闪发光,丛林间呦呦鹿鸣,那鹿甩了甩脑袋,在横生的树杈间来回跳跃,早上刚抬头的婆婆纳在她的蹄子下,重新低落到了尘埃里。或许灵魂飘进了鹿的眼睛里,或许灵魂附在了沙沙的林风间,我抬头看了看青空下的云絮,她一定在
那里。 

“云啊,是青空下最特别的存在,它们俯瞰万物生灵,吸收阳光和尘埃,悠游自在,也能承载所有的愤怒和委屈,但是不言不语,只是淅淅沥沥地下雨
。” 

当你害怕生活的折磨的时候,你就要可着劲儿
\newpage
地折磨自己,这样子就不会再害怕,这样子我们失去爱人的翻滚着的记忆就会慢慢趋于平静。所以我决定离开,离开葛庄这片意杨林,跟随着云朵飘去的方向,飞去我以为永远到不了的远方。我看着括苍山的方
向,振翅起飞。 

一种技能当我们会的时候真的是轻而易举,可是有一段时间我怎么都学不会飞。那时恰值农忙,葛庄的老少爷们们搭着汗巾顶着日帽脸朝黄土背朝天,快速地收割着金灿灿的稻谷,我被我妈驮着,胸腔里面充满了谷物的芳香,可是我的心情却如麦穗般沉甸甸的。我害怕青空,那种万里无云一望无际碧洗的蓝,只要我一抬头,就能被他吞噬心神,以至于我站在天底下,都要把自己缩在阴影里才能找到一点安全感。而此时,我看着郁郁葱葱的括苍山,回忆仿若在昨天,那些永恒不变的都是一直在变的。我知道那只苍鹰必定栖息在针叶林最高的那棵树上,我不止一次地梦到他尖锐嘹亮的叫声响彻整片括苍山林,他会扇动暗褐色横斑的飞羽,将鲜血淋漓的尸体带回栖息的树上撕裂后啄食。而我终于到了这里,我想告诉他我的过去我的现在以及我的将来。可是谁能告诉我,都春
\newpage
天了,他为什么还缩成一团窝在那里。我模糊的视线里,他灰暗的羽毛和正在被昆虫消化的糜肉黏在一起,隐隐约约的白骨像刀剑般刺向我的双眼。我以为我永远都看不见老死的苍鹰的尸体,我以为你会和暴风闪电搏击,你会顽强地奋飞,你会高亢地尖啸,死亡是天地对你最大的戕害,我以为你会用血与肉的燃烧来维护自己不屈自由的灵魂,原来所有的血肉之躯都没什么区别,那个终将来临的夜晚,在你眼中光亮熄
灭之前,你是否惊惶,是否不安,是否,会孤单? 

我想我会继续前行,因为我看到那朵长得特别温柔的云飘向了北方。赶路的当口没事儿干又开始胡思乱想,那是一个阳光明媚的午后,我叼着一朵朵龙胆花和贝壳草,装点着被我妈嫌弃乏味单调的窝。对面的葛庄突然传来了一阵难听的枭声,那种仿佛将喉管扯出来挡在风口上的声音,对于乌鸦来说,猫头鹰的声音以及他们这种族群的本身,就是奇葩的存在,他们吃他们的妈啊,你们知道吗?鸟是鸟他妈生的呀!我颠颠地扑棱着受过伤因而难以平衡的翅膀,葛庄那个会温柔地喂我吃火棘果的“书香世家”抱着她儿子的尸体发出尖锐的叫声。“书香世家”是葛庄的粗
\newpage
俗婆娘们给她起的外号,古国战火连天,城里的大户人家也免不了拖家带口地逃难到类似于葛庄这种偏僻地带以求安稳,而“书香世家”的儿子,参了军马革裹尸而还。我终于明白,那时候我落在她窗棂边上,
她望向远方的那种期盼和忧郁。 

我想哪怕再痛我永远都发不出那种碾碎了灵魂般的叫声。因为我是一只乌鸦,摔坏了脑子的瘸腿乌
鸦。可是为什么我的胸口有一团火。 

我吃过了沿路好吃的不好吃的野果,也尝遍了各个池塘湖泊颜色不同的傻鱼,然后我到了一片火红的枫叶林,一辆马车上躺着一个潦倒的文人,四下散落的酒瓶间坐着一个吴侬软语的俏美人。哟,那个嚷嚷着“楚腰纤细掌中轻”的“文绉绉”,光天化日朗朗乾坤你要停车坐爱枫林晚吗?不要问我“坐”这个像是通假字的动词为什么要堂而皇之地放在“爱”这个名词前,我只是个五谷不分的乌鸦我懂什么。我就姑且跟着这个“文绉绉”,我看着他潦倒落魄,如浮萍柳絮般四海漂泊,走到哪儿诗不离嘴,手不离酒。他不知是真爱还是假爱那些青楼楚馆的娇俏美人们,
\newpage
身材曼妙,舞姿袅袅。他好像沉浸在自己虚构的梦里,梦里烟花如海,梦里国民昌盛。他又被人赶出了酒馆,我从那些人的窃窃私语和异样的眼神中知道了他是相门之后,隐藏在邋遢胡须下的英俊倜傥,消失于求而不得中的文采风流,我不想再看他满目疮痍、破
碎支离的世界,无端端的让人情绪低落。 

我一路向北,春天肥墩墩的大雁从南方飞向北方,秋天那帮瘦嶙嶙的从北方飞向南方,他们都会诧异地看着我这只背道而驰的傻鸟。我不理会,我找了个被松叶掩盖的洞口,将自己屯在那里,抵抗即将来临的难挨的冬季。那片温柔的云朵,日升月起反复在我头顶来回了一百多个日日夜夜。水珠滴到我鸟毛的时候,我正在努力地给自己的身体回暖,我的心仍旧
空荡荡的。 

多年以后的月夜,夜黑风高,我来到一个村庄,荒无人烟,连只四处觅食的野狗都没有,四周充满了隐秘的雾气,正当我打开翅膀准备离开这个是非之地的时候,我突然听到了一个沙哑的声音,声音来源于一块吊在树桩上笔直的焦炭,他说:“小东西,快
\newpage
过来。”我挥舞着鸟毛撒颠儿地跑过去,活久见啊,会说话的焦炭。我抬起我水汪汪的鸟眼好奇地看着这个莫名物种,他舔了舔自己的嘴角,我看到他门前的两颗尖牙,在黑夜的映衬下闪着幽暗的光,动物的本能告诉我赶紧跑,可是我却蹲在了他的肩头上。他抬起头看着幽幽的白月光,这真是一个适合倾诉的夜晚
,月色如水,万籁俱寂,极好极好。 

“在我还活着的时候,总是有用不完的力气,那些时光啊,匆匆忙忙变成了记忆的烟灰。而人间最大的惩罚是什么,是时间都忘记了你。”他像一个智者一样向我阐述他的过去,可我觉得他像傻子一样地在过去的日子里折磨着自己。被血统与道义统治着的遥远国度,历史像曼陀罗花枝般繁复伸展,奢华着散发诱惑的迷香。为了追求长生,他将刚脱离母体的亲生子为基,以欲望和阴谋为引,捣在了那碗散发着迷迭香的臼里。他那个总爱蹭他的脸颊温柔地唤他阿郎,那个低头浅笑如西府海棠般的妻子,看着他咽下药臼里最后一滴汁水的时候,冲破胸腔的声音,比乌鸦还难听。我激动地举起了我的翅膀:“我知道我知道,那是猫头鹰的声音。”他扯开嘴笑了笑,他有一双
\newpage
很好看的眼睛,如浩瀚星辰般破碎明亮。他变成了一个偏执的疯子,想要的东西志在必得,杀伐决断狠辣至极,在他的时代,王权酷吏被他拔到了极致,然后他被火烧被水淹被人钉在棺材里几百年。在孤独空虚中活着,在孤独中不朽,在沉睡中希望岁月的斑驳痕迹能够平淡地过去。与天争斗的一生,争不过时间,执念于控制一切却无法控制命运。“当我偶然经过这个村庄,看到了我心里勾勒了无数遍的眉眼,我知道我所追求的其实在最开始就被我亲手埋葬,那真是一个可爱的女孩子,和我的温婉那么像,我渴望触碰她,我渴望她身上的温度,当我的嘴唇抵到她跳动的筋
脉的时候我能感受到她真实存活着的生命力。” 

我不大的脑袋很难跟得上他的思维,稀稀疏疏的灰开始从他的身上剥落,有点点的亮光拨开迷蒙的雾,鱼肚白的天际慢慢展露出夕阳,他就任凭自己犹如腊肉般挂着,淡定地充满诗意地等待死亡的来临。我听到他最后的喃喃细语声:“不应该喝她最后那滴
血啊。” 

我以为我会一直这么漫无目的地飞着,直到我
\newpage
再也没有力气了,从高空中笔直坠落,死在一个无名的腐叶堆里。然而生活告诉我们,幸福就像是一直到处拈花惹草的风骚蝴蝶,你追着她跑的时候人家不稀罕你,当你放弃了安静地坐在那里喘气的时候,她还以为你是个深沉的思想者,有文化有内涵,被你吸引落在你肩头。那天下午我一时找不到肥鱼就将就着用腐肉换换口味,然后我看到了荆棘丛上那只蹦蹦跳跳的蓝羽碧纹的鸟儿,你们明白那种感觉么,整个世界都刹那间安静下来,风静云止,只能听到我的小心脏如战鼓般响彻我的耳际,我感觉我的脑子开始放起了绚丽的天空之花,她挥翅旋转的身影慢动作在我眼前回放,我觉得一定是中午的腐肉变质了吃坏了我的脑
子。 

那个时候我不知道,不管是什么动物,只要有了执念,就会活得如此用力如此倔强,如此任性毫不妥协。我像一个刚出茅庐的窃贼,自以为隐秘地尾随在她身后,我跟着她飞过了一片又一片的荆棘林,我觉得我的心被什么东西充盈得满满当当,我又想起了括苍山顶的那个站在猎猎山风中的采诗官,那个时候还是春天呢,而现在却又是冬天了。那天晚上我蹲在
\newpage
离她三米远的地方狼吞虎咽地吃着死狍子,嘴角沾满了凝固的血液,而她回头看了我一眼。以前我在二千米的高空翱翔,也能准确地发现水里的是白鲢还是草鱼,而在她烟灰色的瞳孔里我看不到自己的倒影,只有一层又一层闪亮的光晕。夜晚下起了纷纷扬扬的大雪,灰白的雪覆在身上像是死亡的鹰灰,我做梦梦到了自己,梦里白雪无垠,我踉踉跄跄地飞不到半米,不断地落在地上,留下了深深浅浅的脚印,我妈安静地蹲在远处看着我,好像沉默地鼓励我要不停歇地去尝试。在最后一个冬天,我听到了锐利的枝节慢慢穿透肌肉、刺破胸腔的声音,空气中飘浮着隐秘的血腥味,我看到那只鸟儿,以一种蛮野的姿势将自己钉在一根至长至锐的荆棘上,高昂着头,流动的红火淹没她的娇小身躯,那种我以为只存在于传说的歌声直达云端,让我的灵魂都在战栗。“值得吗?”我问不出来,也许连她自己都不知道,她明白自己的死亡和疼痛,可是那些古老隐秘的规则不会轻易更改,荆棘横生间,她唱呀唱呀,唱到身体流出了最后一滴血,而头颅仍以一种高亢的姿态,投向莫测的苍穹。我连站稳都用尽了力气,不知道过了多久,我慢慢回缓即将凝固的血液,头也不回地飞离。我不敢停下来,我害
\newpage
怕站在枝丫上一歇息,就再也飞不起来,我真是太老
了。 

在天空中我不会那么无所适从,不会显得和这个天地间格格不入。我不再犹豫,振翅高飞,飞向青空,我像一个贸然闯进人家领域的不速之客,侧身倾听着亘古的风在窃窃私语,他们从我身上快速地划过。空气渐渐稀薄,翅膀越来越沉重,风也愈加凌厉,身体就像要从内里爆炸了一样,我听到了我内脏撕裂的声音,我只想再高一点,我妈正在看着吧。东方泛起了鱼肚白,我屏住呼到胸腔里面的最后一口气,冲破层层云霭后,我看到一道道闪耀的白光,从天空中倾泻而下。诞为生灵,世道横生了太多的条框,我终于不再轻易跌倒,我们终将毫无顾忌地走到我们的终点,因为我们的身心,每走一步都变得更加强壮,然后每走一步,都变得更加脆弱,也许有一天我重新睁开眼睛,仍旧会困惑得像个懵懂的孩子,但我依然会竭尽全力争取一切机会,飞向我畏惧半生却又热爱半
生的青空。 

当身体与地面接触,发出那样沉闷的响声之前
\newpage
,我想起了人类村庄的袅袅炊烟,夜晚星辰的破碎明
亮,同类之间的冷言冷语,母亲永远忙碌的身影。 

温暖的阳光洒在乌鸦的身体上,雪白的天地间,开出了一朵血色的“虞美人”,采诗官悠长的木屐声,敲打起了料峭的春寒,唱道:人生如烟火焚城,静无声。

\end{document}
