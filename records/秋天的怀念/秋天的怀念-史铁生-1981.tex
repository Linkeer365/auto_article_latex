\documentclass{article}
\usepackage[utf8]{inputenc}
\usepackage{ctex}

\title{秋天的怀念\footnote{Click to View:\url{https://web.archive.org/web/20221025023307/https://www.bilibili.com/read/cv6031845}}}
\author{史铁生}
\date{1981}

% \setCJKmainfont[BoldFont = Noto Sans CJK SC]{Noto Serif CJK SC}
% \setCJKsansfont{Noto Sans CJK SC}
% \setCJKfamilyfont{zhsong}{Noto Serif CJK SC}
% \setCJKfamilyfont{zhhei}{Noto Sans CJK SC}
% \setlength\parindent{0pt}

\begin{document}
\CJKfamily{zhkai}

\maketitle


\Large

双腿瘫痪后,我的脾气变得暴怒无常。望着望着天上北归的雁阵,我会突然把面前的玻璃砸碎;听着听着李谷一甜美的歌声,我会猛地把手边的东西摔向四周的墙壁。母亲就悄悄地躲出去,在我看不见的地方偷偷地听着我的动静。当一切恢复沉寂,她又悄悄地进来,眼边红红的,看着我。“听说北海的花儿都开了,我推着你去走走。”她总是这么说。母亲喜欢花,可自从我的腿瘫痪以后,她侍弄的那些花都死了。“不,我不去!”我狠命地捶打这两条可恨的腿,喊着,“我可活什么劲儿!”母亲扑过来抓住我的手,忍住哭声说:“咱娘儿俩在一块儿,好好儿活
,好好儿活……” 

可我却一直都不知道,她的病已经到了那步田地。后来妹妹告诉我,她常常肝疼得整宿整宿翻来覆
\newpage

去地睡不了觉。 

那天我又独自坐在屋里,看着窗外的树叶“唰唰啦啦”地飘落。母亲进来了,挡在窗前:“北海的菊花开了,我推着你去看看吧。”她憔悴的脸上现出央求般的神色。“什么时候?”“你要是愿意,就明天?”她说。我的回答已经让她喜出望外了。“好吧,就明天。”我说。她高兴得一会坐下,一会站起:“那就赶紧准备准备。”“哎呀,烦不烦?几步路,有什么好准备的!”她也笑了,坐在我身边,絮絮叨叨地说着:“看完菊花,咱们就去‘仿膳’,你小时候最爱吃那儿的豌豆黄儿。还记得那回我带你去北海吗?你偏说那杨树花是毛毛虫,跑着,一脚踩扁一个……”她忽然不说了。对于“跑”和“踩”一类的字
眼,她比我还敏感。她又悄悄地出去了。 


她出去了,就再也没回来。 

邻居们把她抬上车时,她还在大口大口地吐着鲜血。我没想到她已经病成那样。看着三轮车远去,

\newpage
也绝没有想到那竟是永远的诀别。 

邻居的小伙子背着我去看她的时候,她正艰难地呼吸着,像她那一生艰难的生活。别人告诉我,她昏迷前的最后一句话是:“我那个有病的儿子和我那
个还未成年的女儿……” 

又是秋天,妹妹推着我去北海看了菊花。黄色的花淡雅,白色的花高洁,紫红色的花热烈而深沉,泼泼洒洒,秋风中正开得烂漫。我懂得母亲没有说完话。妹妹也懂。我俩在一块儿,要好好儿活……

\end{document}
