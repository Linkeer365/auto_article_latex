\documentclass{article}
\usepackage[utf8]{inputenc}
\usepackage{ctex}

\title{Δ12,百利甜酒与寻找治愈的猫\footnote{Click to View:\url{https://web.archive.org/web/20201018082529/https://zhuanlan.zhihu.com/p/102428274}}}
\author{午时葵}
\date{2020-01-17}

% \setCJKmainfont[BoldFont = Noto Sans CJK SC]{Noto Serif CJK SC}
% \setCJKsansfont{Noto Sans CJK SC}
% \setCJKfamilyfont{zhsong}{Noto Serif CJK SC}
% \setCJKfamilyfont{zhhei}{Noto Sans CJK SC}
% \setlength\parindent{0pt}

\begin{document}
\CJKfamily{zhkai}

\maketitle


\Large

在结束了四十余个小时失眠的折磨之后,我在久违的冬晨中重新夺回了对身体的控制权。凌晨的空气寒冷又清爽,满是冬天的味道,让人熟悉而怀念。「那些属于冬天的孩子,要是能在冬天的气息里彻底融化掉就好了呢。」怀着这样的心情,没等天亮我
就匆忙地赶到了食堂。 

这是考试周的最后一个周末,人似乎少了一些。吃到了一些平日里起床时早已售罄的东西,也明白了平日常吃的东西在这个时间还没有被制作出来这一事实。以惯常的速度在五分钟内吃掉了所有的东西之后,坐在座位上百无聊赖地刷着手机,而后得知了 
@朱俸民
 刚刚起床的消息。在刷完了时间线上所有新
\newpage
产生的内容之后,我们决定在一如既往的常规地点——清芬园路口会合。可能是因为熬夜过度吧,朱老师
这一整天看起来都不是很有精神。 

七点四十五分,天已经完全亮了,空气还残留着一丝清晨的味道。13号线上人满为患, 几乎要上不去车。勉强挤了上去,结果一半的人都消失在了下一站漫长的换乘通道中,想来这个时节都是去机场回家过年的人吧。到了西直门,大家又都纷纷奔向北京北站,而我和朱老师却在这里绕了一大圈,差点没找到出去的路。在眼睁睁看着上一班公交车开走十余分钟之后,我们终于成功登上了开往交大的公交车,时间大概是八点三十。在车上,我问了问 
@爆炸铁锤
 江峰会不会参加这次活动,并在一个小时之
后收到了他刚刚起床的答复 

交大的校园不大,走起来像是走在小区里,有种莫名的亲切感。我们很快就找到了活动举办的地点——一个非常普通的小教室,普通到让我们以为差点走错了。还有十余分钟活动就要正式开始了,而教室
\newpage
里却只有一个人坐在最后排。在略显局促不安的询问过后,得知了这里确实就是传说中的第十二届Δ-Workshop的举办地点。在接下来的几分钟内,一大群人突然一齐涌了进来,其中有着一些曾经见过的熟悉面孔,而来的最早的那个人开始和他们随意地开起了玩笑。我很快就意识到一开始那个穿着绿色外套,有些显老的人其实就是大名鼎鼎的喻良先生,并
愈发感到有些激动和紧张。 

Δ-Workshop,和每年一度的全国数理逻辑年会一样,是我长久以来最大的执念。也没有什么特别的理由,只不过是想看看真实世界中的数理逻辑研究,以及那些如雷贯耳的名字到底是什么样子,仅此而已。而现在,我不得不告诉自己这看起来只是一个圈内熟人间的日常交流活动而已。虽然这听起来反而更难得一点,但对于像我一样纯粹爱好者水平的圈外人士,却是不仅没什么可能去套磁,连听懂报
告内容的可能性都少了很多。 

第一场报告是由中科院软件所的方楠主讲的On the converging speed 
\newpage
of d.c.e. approximations,内容大概是算法信息论里的一些结果。一开始是科普基本的定义,这个阶段我还能勉强记一记笔记,学一点诸如 "1-random" "Solovay reducibility" 之类基本的术语,而后看到如何给d.c.e. real做微分进行随机性判别的时候基本已经只能靠直观听下去了。最后真正lim满天飞去算数列的收敛性的时候,整个人已经彻底变成⑨了(在整个报告中,朱老师睡
的很安稳) 

接下来是由中南大学的李熙报告的「贝叶斯框架下通用先验的选择问题」,讲了很多归纳推理。由于我从没接触过相关内容,因此除了那些哲学论述什
么都没听懂,也什么都没记住…… 

上午的茶歇随便蹭了点吃的,到处乱走,尝试
偷听八卦,发现基本听不懂。 

第三场是北京交通大学的于剑介绍的「基于认知的机器学习公理化」。因为不炼丹也不打算炼丹,
\newpage
也只能看个热闹。印象最深刻的是的他的幻灯片里有
很多有趣的段子,做报告听起来和听课没两样。 

中午一起蹭了张集体合影,不过大概没地方去要。这次活动的真正参会者们一块聚餐去了,宋诗畅老师的研究生带我们和另外几个北大来的同学去食堂吃饭。其中有一个人已经见到不知多少次了,自从去年年初在俞珺华老师开的证明论课上见到他开始,基本每次和数理逻辑有关的活动都会看见他,我猜他其实还去过很多我不知道的活动/课程。有点想找他要
个微信,不过最后还是只敢打个招呼作罢。 

吃过饭之后回到教室,翻着那本天天背在包里的paper。过了一会,江峰来了,我猜可能是因为上午的主题不够有趣?我像往常一样随便胡扯了一点序数分析一点高阶递归论,诸如「二阶算术的强度源于图灵度的强度」云云。江峰指了指坐在我前面的人,告诉我他是这里最懂后者的人之一。我理所当然
的完全不敢过去乱问,因为,他是杨跃。 

下午的第一场是北京师范大学的施翔晖报告的
\newpage
An I0 -analogue of an AD theorem(现场的幻灯片上换了标题,不过我已经记不清了)。这一场我听懂的很少,基本也就是「听说(在 ZF+DC+AD 中) AD 的一个推论作用在模型 LR 上和(在 ZFC+I0 中) I0(λ) 中的类似物作用在 L(Vλ+1) 上的现象是类似的」这种程度,为自己
至今不懂集合论而羞愧。(朱老师依然在睡觉) 

第二场是新加坡国立大学的杨跃主讲的「实数上的递归集是 Δ1 的? Δ2 的?还是 Δ12 的?」。杨跃先生的报告条理清楚,平易近人,十分的引人入胜(也可能只是因为我递归论学的多一点点?),讲了一些effective Polish space上的递归论,实数的递归论刻画,以及对柯西列等价类的处理,诸如此类。由于听的太
入迷,忘记记笔记了,十分后悔。(朱老师醒了) 

// 还记得在讲到MS机(Master-Slave Machine)的时候,杨跃说这两个词有些歧义,叫我们不要想歪,而后下面就有人窃
\newpage

窃私语:为什么不叫SM机? 

下午的茶歇,抓紧机会和江峰闲聊,依然是日常话题,依然像往常一样引来了某些路过的实在论者奇怪的视线。又顺便打听了一下「现代的非直觉主义非实在论者从何处得到一致性信念?」这个问题,并得到了下限可能是某种希尔伯特纲领的变体或者realizability这样的回答,可能彻底的形式主义确实没救了吧。而后见到了 
@嬴无翳
 ,前一阵子和qlbf见面时认识的北交同学,并互相交换了知乎账号,继续闲聊。在茶歇快要结束的时候恰好谈到序数分析不能为证明系统进行一致性辩护的事情,一个看上去有点谐的人突然冒了出来告诉我「序数分析中元理论和对象理论的强度是不可比的」,我不太清楚这个说法的出处和正确性,只好用「这是关于一致性的数学哲学讨论」暂且搪塞过
去。几分钟后,我看见他走上了讲台。 

// 来自几天后的一点补充:这个说法在当时的语境下不能算错,证明 Con(PA) 只需
\newpage
要 PRA+TIε0 ,它们在后承意义下确实是不可比的。但是在证明论中理论 T1 强于 T2
 的定义就是 T1 Con(T2) 。 

Δ-Workshop的最后一场,Weakly Aggregative Modal Logic: Characterization, Interpolation and Model transformation, 由来自四川大学的刘佶鑫报告。这一场是对弱聚合模态逻辑的介绍,主要聚焦在模型论技术上,讲了一点如何用树展开做有限模型上的van Benthem刻画定理,怎么构造Carig interpolation的反例,以及这东西的关系语义和邻域语义。这一场听起来也很舒服,可能是因为比较偏好报告者这种活跃的语言风格,以及内容和例子都稍微有点熟悉。不过还是很想听听证明论相关的内容,不然估计看过的
模态证明论以后也就白看了。 

活动没有什么结束仪式,和最后一场的报告者随便聊了两句就发觉人已经彻底走光了。一边和朱、
\newpage
江二人闲聊着往外走一边看着活动的真正参与者在教室外商讨着什么,估计他们接下来还有什么别的活动吧。我和朱老师一边考虑着晚饭的着落一边往回走,最终在桃李地下和紫荆地下这两个仅有的候选项中选择了前者,并叫上了 
@dram

 。 

晚上的13号线比早上还要更挤一点,在始发站就站满了人,下车很是费了一番功夫。出了地铁站,发现停在角落里的自行车被人强行推倒到一边加塞进去,卡的动弹不得。于是不得不一边高声咒骂毫无道德的加塞者一边花了整整十分钟把它的车踹走,以此继续行程。等到了桃李园,dram早已在此等候
多时了。 

晚餐是平凡的,味道一般般的意面,但至少落在桃李黑暗食谱里可食用的那一半上面。把咖啡杯当成子弹杯干了一shot的espresso,轻微的手抖,头也有些疼,可能是因为睡眠不足。在此,讨论的主题是「广义的语义化版本号」,即面向最终
\newpage
用户的程序的兼容性问题,虽然(形式上)类似的讨论总是没有什么结果。而后,我们一同去了朱老师的实验室准备材料,准备进入今天的核心活动:调酒。
 

一些金巴利,一些百利甜,一些咖啡利口酒,一些樱桃白兰地,补上家中已有的材料,便足以支撑一个晚上的活动。我们一同来到我十分混乱的住处,用散乱的paper和蓝皮集合论盖住的R18琪露诺抱枕显得有些微妙。我同惯常一样熟练地从书架上取下酒壶酒杯柠檬夹之类乱七八糟的工具和一堆堆玻璃瓶子,而朱老师,作为主要的调酒师,则开始调配
今天的第一杯:B-52。 

// 由于这次没什么照片,只能假装这里有
照片 

百利甜酒与咖啡利口酒的味道在口中相互交融,柔顺而甘甜,尾调混合着咖啡微微的苦味和橙子的香气,令人深深着迷。也正是从此开始,我彻底爱上

\newpage
了百利甜酒,爱上了这柔顺的奶油香气。 

Negroni也好,Singapore Sling也好,各式五颜六色的饮品被不断地制造出来。我们一边调,一边喝,一边闲聊。dram的思维很跳跃,让我总是无法回忆谈话的主题,却总是想到ta不断翻着自己的channel一条接一条的展示的情景。一个讨论了有几天的主题是proof assistant的设计,是否可以将kernel设计的尽可能小,以及是否应该使用传统的集合论而非类型论。虽然结论是显然的,但讨论本身还是很有趣。另外一些总是在持续讨论的事情包括但不限于一些大家都听过但叫不出名字的曲子,dram最近玩的解谜游戏,一些其他人发表的东西,以及dram和一些人的关系——因为我本质上是个很八卦
的人。 

由于我前一阵子被金酒灌到大醉,至今依然会对杜松子的味道产生强烈的不适,而dram又不怎么喝酒,于是这次大量的金酒配方都被留给了朱老师一个人。因此,我和dram有幸观察到了一个人从完全清醒到彻底喝醉的全过程。在这整个过程中,我
\newpage
与dram就在一边照料朱老师一边聊天。时间大概是深夜两三点,话题转向了药物,无用的医生和人与
人之间的关系。 

dram是一个看起来没有丝毫异常的人,虽然到了现在我已经能一眼看穿这类假象。一个缺乏情感的人,本质上是脱离于这个世界存在的,也很容易真正的脱离这个世界。我试图说服ta,但并没有什么作用,说服这种手段本身就不会有什么作用。我不清楚到底有多少是药物的作用,但我鼓励ta尝试把自己的生活重新染上色彩,就像ta对自己的头像做的那样。我几乎对每一个人都会给出类似的建议,一方面我相信如果这可以被做到那么一定是有效果的,另一方面我不知道自己还能做些什么。dram相信这已经是最优局面了,并不想作出任何改变。一方面,ta怀疑「做出某种positive的改变」很可能并不会得到positive的反馈,因为有些人的行为较为不可预料。另一方面,一个确实有些严
重的问题是,这可能会引起非预期的剂量增加。 

当药物对一个人不再有作用,甚至对生活产生
\newpage
了更为严重的负面效果,应该怎么办?我不知道,我真的不知道。我无法回答这个问题。如果没有药物,ta们可能明天就会从这个世界上消失,但即使有了这些药物,又能怎么样?dram对此感到有些悲观,我虽然总是展现出没来由的乐观,但并没有令人信服的理由来说服ta。我只能稍稍地安慰ta,即使药物带来的debuff是去不掉的,但我们依然可以通过一些积极的改变来改善我们的生存质量,以此迎接可能好的未来。虽然这似乎并没有说服ta什么
,甚至连我自己都不太相信。 

我一直都在寻找治愈的方法,但我可能真的做不了什么。我只能尝试变得可靠,变得温柔,但这并不本质的解决ta们的问题。我尝试变成一个情绪稳定器,但我甚至在真的遇到需要帮助的人时不敢开口。我异常害怕作为陌生人的自己,与并不熟识的人的交流让我感到极度恐惧。因此,我想我的作用无限接近于零。关于dram,至少我无法说服ta,而我对情绪还算敏锐的直觉也无法在ta这里抓住任何东西。可能是ta一直是这样,也可能是我不是可以观测到ta的情绪波动的对象?无论如何,我努力尝试
\newpage
去治愈,努力拥抱每一个人,虽然结局往往不甚理想
。 

过后,我在屋里翻找到了qlbf留给我的发圈,又一次尝试为自己扎起马尾,并意外的成功了。望着镜中还算勉强看得过去的自己,我随口问了dram一句:「我看起来像是有gd的人吗?」得到的答复是ta觉得我一点也没有。一点也不意外,这与我一直以来的感觉是相符的。我向往成为女孩子的美好的那一部分,但我也不讨厌作为男孩子的自己。事实上,我都不知道自己现在究竟是哪一面多一些。无
论如何,我对现状似乎还算满意。 

在稍稍安抚了因醉酒而有些痛苦的朱老师后,我和dram回到我的房间,关上灯,以极为别扭的姿势躺在了床上。谈话还在继续,虽然随着我意识的涣散主题也飘忽不定。虽然我还是异常的八卦dram的人际关系,连我自己也不知道到底是为什么。以这样一种几乎快要掉下床去的姿势卡在床的边缘与墙壁之间睡着我还是第一次,可能是因为快要到了身体的极限了吧。dram的情况也不甚理想,因为一整
\newpage
条被子堆在了ta睡觉的地方,导致没有一个合适的角度用来睡觉。当然,根本原因自然是朱老师是不可
移动的。(以及,我发现我还挺喜欢摸头的) 

几个小时之后,保持着这样的姿势睁开了眼,有些惊讶的发现睡眠质量还算说的过去,可能真的是太累了?不过还是和往常一样十分头疼,已经是起床的日常了。于是类似的闲聊再次开始,这次dram终于想过来了那个昨天晚上就把我成功绕了进去的问题:为啥ω^ω是一棵有无穷分叉的无穷树?当然,问题是显然的,纯粹是因为我在犯傻……于是随手拿起床上的Soare16,进行一直以来我们都很喜欢的活动——翻到随机一页,看看自己能猜懂多少内容。这次翻到了low degree和minimal pair,前者不敢说懂,后者是一堆乱七八糟的构造,唯一学到的东西是dram考我TFAE
是啥的缩写,我不知道( 

朱老师似乎重新恢复了正常,虽然看起来还是有些虚弱。我们稍稍整理了一下东西,决定下午按照原计划去唱歌。朱老师需要先回去收拾一下自己,于
\newpage
是我们一起回到了清华。我和dram去了听涛的清青牛拉吃东西,朱老师则回了宿舍。我们不约而同地点了高度同质化的食物,由此却观察到了两个人的食物份量大不相同。朱老师来的很晚,晚到我们几乎以为他已经睡着了。在此之后,是一系列对于这里的食
物的味觉稳定性的讨论(因为dram不能吃辣) 

由于我一时脑抽,这次又跑到了五道口去坐13号线。人比昨天甚至还要多,这次是关于后进先出的队列的讨论和「命令式响应式编程」,以及几乎每一次坐地铁都有的对铁路本身的讨论。好不容易到了三里屯,却发现找的到的KTV全部爆满,很是不解。是因为考试周结束了?回家之前都出来唱歌?最后好不容易找到了一家,跑过去一看,结果是一家系统垃圾设施过时濒临倒闭的KTV。但是因为路途较远
,也不好直接回去,就只能这样了。 

由于几乎所有我常唱的歌都没有,隔音效果又不是很好,所以只好找出一些曾经熟悉的相对老一点的曲子来唱。这次唱了很多初音的歌,感觉仿佛又回

\newpage
到了五年前,那些还不知道什么是车万的日子( 

从KTV出来之后去了附近吃饭,结果发现旁边的银河SOHO和那家KTV一样诡异。商场里空荡荡的,几乎完全没有人,大片大片或关门或倒闭的奢侈品店,一些廉价的外卖挤在角落里,甚至还有一个破旧的五金商店存在于这幢全北京最具设计感(之一)的大厦里。随便找了一家看起来还像点样的日料
外卖,简单吃过之后我们便回去了。 

在回去的2号线上,依然是一如既往的闲聊。聊到离散期末考试,罗素的PM系统和如何教给刚上大学的小朋友“适当的数学素养”,我大声地嘲笑贵系的离散数学课程以及那本屎一样的紫皮书,引得旁边的人一阵阵侧目。我们一致认为哪怕是让哲学系的人来开一门正经的数理逻辑都比现在强上百倍,然而这在政策上就不允许,于是贵清没救。列车驶至北京站,又下去了三分之二的人,愈发感到一个人在鬼城
孤零零的过年还是会有种莫名的酸楚。 

回到五道口我们又吃了一些烧烤,点了一些酸梅汤,并发觉这水平比起食堂的差了老远。店的名字
\newpage
叫聚点串吧,并由此引发了一些点集拓扑学的讨论。朱老师吃完之后就回去休息了,我和dram本打算像往常一样在外面聊到深夜,但是ta却无意中把晚上的药提前吃掉了,于是也不得不回去,不然就会在药物的作用下很快就变得困倦到无法正常回家。于是
我们也一起回家了。 

和dram分别之后,回到家里,把永远装着一大堆书籍论文的重重的包扔在地上,瘫倒在床上,看着这两天来可能需要关心的消息。然而只有一个可爱的小朋友找我说话,然后没了。洗漱完毕回到床上,打开知乎,突然觉得自己也需要一篇日记记录那些最终会被遗忘的过往,和有些多变的心情,就像ta们一样。于是新建了一篇文章,在标题栏里输入了「Δ12,百利甜酒与寻找治愈的猫」。退出知乎,看了看时间,大概是一月十二日,星期日,二十三点五八分。

\end{document}
