\documentclass{article}
\usepackage[utf8]{inputenc}
\usepackage{ctex}

\title{我为什么逃离科研\footnote{Click to View:\url{https://web.archive.org/web/20221027020258/https://blog.sciencenet.cn/blog-3383728-1178664.html}}}
\author{某清华博士生}
\date{2012}

% \setCJKmainfont[BoldFont = Noto Sans CJK SC]{Noto Serif CJK SC}
% \setCJKsansfont{Noto Sans CJK SC}
% \setCJKfamilyfont{zhsong}{Noto Serif CJK SC}
% \setCJKfamilyfont{zhhei}{Noto Sans CJK SC}
% \setlength\parindent{0pt}

\begin{document}
\CJKfamily{zhkai}

\maketitle


\Large

在来美国的前一天晚上和程老师吃饭的时候,师兄告诉我们程老师下午发的博客已经上科学网首页了,但当时也只有40多个回复而已,我们也没有在意,毕竟程老师的博文经常上科学网首页。当晚我们并没有继续讨论我工作选择的事情,而是像往常一样随便侃了侃,甚至还讨论了点学术问题。回宿舍以后我还跟几个好友开玩笑的说“哥出名了,上科学网首页了”。但没想到的是,我真的出名了,第二天我下飞机开了手机以后,短信不断。这几天很多朋友、同学、实验室老师、甚至毕业后就一直没有联系过的本科辅导员都纷纷对我表示关心,或支持我的决定,或劝我重回科研道路,不管怎样,我都很感动,在这里先向他们的关心表示感谢。当然也还有几位记者发邮件过来要采访。我本来不想回应,想继续沉默下去的,但是程老师说他认为这个事情的讨论对许多年轻
\newpage
博士生是有好处的。仔细想一想,这也许是我这个科研逃兵在离开前能对科研界做的最后的,其实也是唯一的贡献了。反正也已经出名了,程老师把书名、奖项都列出来了,想肉我的早就肉到了,死猪不怕开水
烫了,就发在人人吧。 

前面两部分我要先帮程老师和我家里人说几句话,对这些不感兴趣的可以直接从第三部分开始看。
 


一、关于程老师 

首先要帮程老师说几句话,因为很多支持我的都说程老师太push了。其实我一直觉得程老师是国内科研界少有的非常nice的导师之一,不但不push,还经常告诫我要多休息,多出去玩玩儿。另外程老师也给我们之间创造很多学术之外的沟通机会,会隔三差五的带我们出去吃饭,和我们几乎是无所不谈。我研一的时候在程老师面前还是非常拘谨的,但没多久就能畅所欲言了。实验室秘书都说程老师的学生都跟他“没大没小”的。只是在和他的交流中
\newpage
我一直不敢说自己以后不想 搞科研了,因为我深知程老师对我寄予厚望,我说出来他肯定非常失望的,而又因为这几年和程老师培养出的感情,我不想让他失望。我甚至一直在想就这样坚持搞科研搞下去,但真正到了该抉择的时候,我还是选择了自私的按自己
的意愿。 


二、关于家里的意见 

还好我父母都不上科学网、水木这样的网站,不然看到那些说我是因为他们给的压力而放弃科研的猜测后,不知他们会不会鸭梨山大。我父母确实是没钱没权也没啥本事的,不过因为有单位分的房住,他们靠自己不高的工资在北京也是生活无忧的,所以他们也从来没有要求过我赚大钱养活他们,只要我过得开心就好,他们甚至还认为家里如果能出个科学家是件光宗耀祖的事情。我年轻的时候也是向往过赚大钱的,不过渐渐的觉得自己其实更喜欢稳定安逸的生活,钱够花就成,当然能保证稳定安逸的话钱还是多多益善哈。所以如果不是我彻底厌恶了科研的话,我觉得科研这工作挺符合我的要求的,社会地位不低,待
\newpage
遇也足够过比较体面的生活了,关键是极度自由。我光棍节那天回姥姥家(程老师听成了老家,差了个lao,不过这无所谓了)算是开了个会,并不是他们劝我赶紧去挣钱,而是我想问问他们对我选择中学这样一个地位不高,挣钱也不多(不算自己外面接活的话,挣钱真的多不到哪去,被it民工们秒杀,更别提金融界的温拿了。而以我的性格,除非真的缺钱,不然应该不会去接活的),还挺累的职业有没有什么意见。最后大家一致认为我真的厌恶科研的话,坚持干一辈子科研一定不会幸福的,而他们并不在意我的
名利地位什么的,中学老师也挺好。 


三、我为什么逃离科研 

其实很简单,唯一的原因就是没兴趣了。没兴趣还算个比较中性的词的话,我其实可以说我已经厌
恶科研了,主要原因有两个: 

1.累。但再次强调这不是程老师强迫的,程老师给我安排的大多数任务都没有给定deadline,只是因为我从小被教育成听话的“好”孩子,
\newpage
只要别人给了我任务并且应该是我做的任务,不管我喜不喜欢,都会尽力去完成,不只是科研问题,甚至是帮实验室干杂活,都是完成的既快又好。这样的结果就是导致了程老师以为我喜欢做科研,所以就忍不住不停的给我安排任务。如此恶性循环了下去。后来实验室秘书也说,如果当时我能更加变通的面对程老师安排的任务,给三件就做一 件,程老师也不会批评我什么的,而我也不会被自己给自己的压力压垮了。当然比体力累更重要的是心累,体力其实有时候根本就谈不上累,我甚至可以好几天在实验室坐着无所事事的刷着微博逛着人人,甚至干脆出去跟朋友打牌爬山什么的去了,但这时脑子里还一直装着那些想不出来的问题,还有一些该做但实在是很烦,不想去做的任务(比如审一些很水很水很水的文章...),半刻也不得安宁。当我决定退出科研的时候,心里是久违的无比的轻松,而这样的轻松,更加坚定了我的
决心; 

2.没能力。这真不是装13。我虽然是有几篇控制界顶级期刊的文章,但顶级期刊的文章不等于是顶级文章。说实话,我还真是觉得我这几篇大文章
\newpage
无论理论上还是应用上都不算真的有用,甚至技术难度上也没啥挑战性,只是相比当今大批的水文,这些算是矬子里拔将军,我也没有为这些文章以及由这些文章而带来的荣誉真正的兴奋过。然后发的那么多其他文章中还有一半以上是程老师被一些国内期刊、会议邀稿而又不好不给面子,临时凑的没啥营养的综述类文章,而且真的是程老师自己主笔的,我只是帮帮忙而已。反正我是觉得这些只能证明我比较勤奋,根本不能说明我有天赋有能力。如果我继续搞科研的话,我能想象出的结果只有两个,要么迫于学校要求发文章的压力沦为灌水机器,虽然还能混得不错,不过天天自己鄙视自己,要么就是坚持不发水文,但又因为能力不足以做出真正有价值的工作而混得很惨。我觉得程老师的博文下面有一条回复对于我的看法是相当正确的:“From what you described, especially "听话出活,对我的要求,从来不说:“No”", this student is clearly not a top student. If he is not even a top student, he will definitely 
\newpage
not be a top researcher. In this case, it is better to advise him to get into some other things. Unfortunately, many Chinese professors' definition of top students are different from other people. They usually promote those students similar to this student of yours. This is unfair to truly top stud
ents.” 

当然也有一些对现在科研界风气的不满,不过这个我了解不深,就不胡说了,说多了被人笑话,还
有推卸责任之嫌。 


四、我为什么选择中学 

\newpage

1.我觉得我有足够能力应对中学数学的知识。这与我觉得我完全没能力做有价值的科研工作形成了鲜明的对比。不过除了知识能力,教中学更重要的是授课能力。我很清楚我现在的授课能力和优秀教师还有很大差距,但通过了学校的试讲,也在试讲中pk掉了不少北师清华北大的硕士博士们,至少说明了我还是具备基本的授课能力,我也相信授课能力是通过我自己的努力可以提高的。当然除了授课以外,优秀的中学教师还需要很多其他素质,比如基本的师德,对孩子的关心,亲和力等等,但这些我觉得我都还
是不错的。 

2.我也很喜欢教会别人知识的那种成就感。我也做过家教,我觉得当几个小时的家教比搞几个小时的科研舒服多了。我今年寒假还帮一个微积分挂了的大一孩子补了两天的微积分,当她告诉我她补考得
了90多分的时候,那成就感啊,杠杠的。 

3.生活比较稳定。以后生活中比较麻烦的事情,比如住房、子女入学等都可以解决了(房子不给产权,只是在职就可以住),但是中学老师的工资对
\newpage

一个博士毕业生来说确实不算多。 

4.我真的是没时间找其他工作,找工作的黄金时间我在美国啊。其实我之前真的都准备听程老师话,毕业去做博后了,因为我本来是要10月底就来美国访问的。但签证意外的被check了,于是要晚走半个月。然后没事干,就投了投简历,其实我也只投了4所高中,没有投其他行业,甚至我投的时候我也觉得我一定是赶不上试讲了,其中我在投给人大附的简历中还写道“因为本人11月至3月在美国,如果有幸能有资格通过初选参加试讲,是否可能将试讲安排在3月?”。但没想到有 两所学校很快就通知试讲了,其中某个学校的效率意外的高,上午试讲下午群面第二天终面,终面后不让走,等都面完了直接出结果,于是赶上了我能在出国前签约,要不我觉得他们也不会把职位给我留到回国后。这种种意外也算是一种缘分吧,再加上该学校也是所很好的学校,他们的教育改革理念(至少是宣传片上的)我也很欣赏,并且他们的待遇在高中也是很好的,跟家里商量后我就同意了。另外我真的没有考CPA啊,程老师

\newpage
记错了,我怎么会有时间准备CPA..... 

当然我也知道当高中老师并不是很轻松的事。比如说很累,不过这点搞过科研的表示呵呵。比如遇到实在不听话的孩子和无理取闹的家长,这种事情比较棘手,我有心理准备,但现在还不知道要怎么处理,以后会从同事那里得到经验的吧。再比如我虽然觉得我通过努力能提高自己的授课能力,但万一再怎么
努力也真的不行呢?这个....到时再说吧。 

写了这么多废话,总之就是我确定我对搞科研没兴趣了,而我觉得我对教中学是有兴趣的。我也觉得中学需要引进优秀博士,前提是得保证他们的教学质量,他们会给学生带来更广阔的视野。当然科研界更是亟需人才的,其实哪里都需要优秀的人才的(这是一句废话)。只是我自己肯定不是科研界需要的人才,对科研没有兴趣的人是不可能做出真正有意义的成果的,我希望自己可以是教育界需要的人才吧。就
说这么多了。 

看了大家的评论,这里再多啰嗦几句,要说的

\newpage
不多,就不另开一贴了。 

1. 辟个谣,我去的不是人大附中。帖子里说的是我在给人大附中投的简历中写了什么什么,后
面说的签约的是“某个学校”。 

2. 我强调的对科研失去兴趣是累和没能力,也就是两者交互作用的结果,并不是单纯的累。如果只是累,而我很有能力,能够或者相信自己能够在长久的辛苦后可以得到令自己满意的成果,哪怕中间会经历很多次失败,也会是痛并快乐着,甚至很享受这样的过程。但我现在没有并且也不认为自己有能力可以取得令自己满意的成果,所以这一过程只是痛苦。至于我该不该认为自己没能力,该不该不满意我现在的成果,其他学生如何避免我这样的想法,都是另一层面的问题,也不在我有能力讨论的范围了。而我
现在已有的自我认知,是短期内无法改变的了。 

3. 有批评说我没有国家使命感没有献身科
学的精神,这些批评我都同意,但无法改正。 

4. 还有人说我这篇文章会使很多彷徨的年
\newpage
轻学者放弃科研,于是我成了中国科研界的千古罪人什么的。这我就真不敢苟同了。能被我影响而放弃科研的人,一定是跟我一样没兴趣没能力的人,我们的离开不会对真正的科研界造成任何损失。真正的有意义的科研成果也绝不是靠人海战术完成的。中国的国
际论文发表量已是世界第二了,但又如何了呢? 

5. 确实我投身教育并不是我对教育有多么的热爱,因为我还没从事过,热爱无从谈起。只是我现在相信我有能力,我也觉得我有兴趣。也许几年后,像部分网友们说的那样,我会像现在逃离科研界一样逃离教育界,哪里都是围城,在外面永远搞不清里面是什么样子的。以后的事情以后再说了,也许我会再次厌恶逃离,但也许会很热爱呢,也抑或不喜欢也不讨厌,就这样平淡的继续下去。现在我只是从一条确定性的不幸福的道路,转到了一条有可能幸福或者是未知的道路。

\end{document}
