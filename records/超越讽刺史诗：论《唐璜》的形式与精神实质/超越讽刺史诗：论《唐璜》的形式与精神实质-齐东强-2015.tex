\documentclass{article}
\usepackage[utf8]{inputenc}
\usepackage{ctex}

\title{超越讽刺史诗:论《唐璜》的形式与精神实质\footnote{Click to View:\url{https://web.archive.org/web/20150316221954/https://book.douban.com/review/7396458/}}}
\author{齐东强}
\date{2015-03-04}

% \setCJKmainfont[BoldFont = Noto Sans CJK SC]{Noto Serif CJK SC}
% \setCJKsansfont{Noto Sans CJK SC}
% \setCJKfamilyfont{zhsong}{Noto Serif CJK SC}
% \setCJKfamilyfont{zhhei}{Noto Sans CJK SC}
% \setlength\parindent{0pt}

\begin{document}
\CJKfamily{zhkai}

\maketitle


\Large

上帝不是几何学家,不是数学家,而是诗人
。——以赛亚·伯林《浪漫主义的根源》 


一、《唐璜》与唐璜们 

1786年,《费加罗的婚礼》在维也纳遭受冷落,莫扎特带着对权势的厌恶愤然旅居布拉格。次年,《唐璜》的首演将莫扎特的声誉推向新的顶峰,使这座小城在欧洲大陆启蒙运动的烽火中陡然闪耀。席勒写信给歌德:“我曾对歌剧抱有信心,希望从中涌现出一种更为高贵的形式,就好似从古代的酒神节合唱中诞生了悲剧。”歌德复信说:“你会发现,你对歌剧所持有的希望,己经在最近的《唐磺》中得到
很高程度的实现。” 

\newpage

在莫扎特之前,唐璜的传说至少已经存在了二百年。最初,它选择了戏剧作为其走出西班牙乡间的道路。从1630年蒂尔索出版《塞维利亚的浪子》起,这一故事开始向更多国家流传。在法国,几乎同一时期,莫里哀(《唐璜与石像的宴会》,1665)便使唐璜形象在戏剧领域的表现到达了一座高峰。在文学与音乐世界两位先辈巨匠高屋建瓴的阴影之下,同样题材的创作之困难不言而喻。对于此种情形,阐释而非重写才是省力的途径,选择前者也未必意味
着平庸。 

然而,在19世纪,唐璜的故事突然开始在传说的浓雾中千变万化,辗转于各个国家、各个领域的大师之手。一大批神形各异的唐璜风靡起来,杰作也屡屡显身:在戏剧领域,普希金(《石客》,1830)与大仲马(《唐璜·德·马拉尼亚》,1836)先后创作了以唐璜故事为题材的剧本;在小说领域,唐璜的形象在巴尔扎克(《长寿药水》,1830)、梅里美(《炼狱里的灵魂》,1834),甚至在更往后的萧伯纳(《人与超人》,1903)等伟大小说家的笔下频频出现;在音乐领域,经由李斯特
\newpage
(歌剧幻想曲,1841)、理查德·施特劳斯(第七首交响诗,1888)等人天才的演奏与重作,虽仍无法掩饰莫扎特的光芒,却使唐璜以系列形象的方式毫无悬念地烙进西方音乐史。马达里亚加说:“四个伟大的形象在欧洲文学史上英名远播:哈姆雷特和浮士德名列其中,还有西班牙人堂吉诃德和唐璜。”
 

这一时期涉及众多门类创作、近乎竞赛与挑衅的潮流,最初由诗歌领域而起。漩涡的正中,正是拜伦的长诗《唐璜》。梁实秋一针见血地指出了这部作品的主人公与传统唐璜的区别:拜伦的唐璜浪漫而忧
郁,不似传统的唐璜那样轻佻。 

在拜伦以前,无论是莫里哀还是莫扎特,都没有颠覆唐璜形象的传统内涵。莫里哀使唐璜题材脱离了原来的低俗状态,开始步入严肃的艺术殿堂,而在《石像的宴会》里,唐璜只是一个不折不扣的恶棍,如同蒂尔索所写的那样。在莫扎特那里,一些可贵的转机已经出现:唐璜仍然是个荒淫无度、挑战一切约束的恶魔,但他对自由的渴望,对陈腐社会的蔑视,
\newpage
已经足以令人重新考虑其特质。这个唐璜模糊了善恶的分野,他同时是启蒙精神的英雄和敌人。这正是莫扎特的伟大之处,正如托马西尼的评论:“唐乔瓦尼固然是一个恶魔,但他也是一个迷人的恶魔。”然而,最终唐璜仍然无可救药地走向毁灭,虽然这未必能够表征莫扎特的真正意图,但至少可以说这位唐璜的精神突围失败了,他为证明自己而作出的挣扎也失败
了。 

拜伦彻底改变了唐璜。在拜伦手中,唐璜大大削弱了恶魔的气质,拥有了几乎相反的性格。他的激情不再是毁灭性的,而具备了一种洁白的生命力。他并不将自己置身世界之上,哪怕是观念上的,而是游刃有余地深入社会生活。拜伦犀利地抓住了一个拖泥带水的人物最关键的符号,从传统的浪子原型中发掘出一个崭新的唐璜形象,将唐璜的精神实质从历史中解放出来,使得世人对唐璜的种种刻板印象纷纷剥落。这种清洗实际上开拓了一片新天地,鼓励着人们无所顾忌地去创造。不论后来的唐璜被作者们赋予了怎样的含义,这个角色的生命力和复杂性却是拜伦赋予

\newpage
的。 

这一创举固然已经堪称伟大,可说到底,《唐璜》在人类精神版图中刻下的众多痕迹里,这只是其
中较为纤细的一条。 

这部史诗般宏伟的长诗始作于1818年。其时,拜伦已经离家两年,除了更早些时候已负盛名的《恰尔德·哈罗尔德游记》、《海盗》等长诗之外,惊心动魄的诗剧《曼弗雷德》也已经写作完成。歌德称这部从《浮士德》中汲取灵感而创作的作品是令他本人都“深受感动的惊人作品”。后来完成的诗剧《该隐》更是将矛头直指上帝,弹出反抗的最强音。事实上,这些累累硕果已经足以使拜伦名垂千古,他已
经注定是浪漫主义弄潮儿中的翘楚。 

然而,一部未完成的《唐璜》,竟几乎掩盖了拜伦其他作品的光芒,成为了作者本人的标签与代名词。王佐良先生的《英国文学史》在论及拜伦时,甚至直接使用“拜伦和他的《唐璜》”作为论述标题。诚然,一直以来,《唐璜》被习惯地称作“讽刺史诗”,讽刺指涉内涵,史诗指涉形式,这一称号似乎简
\newpage
明扼要地指出《唐璜》的伟大,而拜伦的英年早逝更让这部作品身披“遗作”的神秘外衣。但稍加考量,读者就会发现,《唐璜》的讽刺性似乎并未超过拜伦的其他作品,而其体例似乎也不像严格意义上的史诗。那么《唐璜》究竟为何赢得了如此广泛长久的赞誉?它在何种层面、何种程度上值得人们关注?它是否具备代表时代甚至超越时代的核心内涵?或者更清晰地说,我的困惑在于:《唐璜》是否能够超越“讽刺史诗”这一常规评价?而这一切须从对文本的研究开
始。 


二、超越史诗 

在第一章的第200节,拜伦写道:“我的诗篇是史诗。”众所周知,现行的“史诗”(epic)一词来源于古希腊语词汇epos,意为“说话”、“故事”。“epic”一词本身就有“叙事诗”的意思。在《诗学》中,亚里士多德谈论史诗的五个方面(情节、分类、长短、格律、写法),后四个方面都是在论述史诗的形式规定性和写作技巧,只有“情节”涉及到史诗的内容规定性。这反映着亚氏诗学
\newpage
理论里叙事性在史诗各要素中的地位。20世纪后,随着大量活态口头史诗在世界各地陆续被发现和研究,学者们开始对以荷马史诗为范例的古典诗学的史诗观念和研究范式进行反思和批判,然而在大多数观点
中,“叙事”仍然始终是史诗最基本的特征。 

从叙事模式的角度来看,《唐璜》与荷马史诗确实存在类似。除了如拜伦自己所说的,它具备“爱情,战争,海上的一阵大的暴风” 等等史诗元素之外,这部未完成的长诗在情节的特质上也类似《伊利亚特》。“神样的阿喀琉斯”,这个最强大的战士,在最初的战事之后,便因阿伽门农的不敬而在他漫长纠结的思索以及荷马冗长的铺陈中偃旗息鼓。直到帕特罗克洛斯死亡,才将他推回无可避免的战斗。最终,他将会宿命般地战死沙场,这个结局却不在《伊利亚特》之中。而这种“宿命-偏离-宿命”的钟摆型叙事在《唐璜》中也出现了:唐璜,这个浪子,这个为爱而生的情种,却在一开始两次铭心的恋情之后,便陷入市侩、征战和政治活动,在冲动与时运的妥协中,再也没能经历一次真正的爱情,而他的死亡同样是未知的。虽然拜伦曾经提到过《唐璜》的结局——
\newpage
他会出现在法国大革命中的巴黎,但我们也看到,在漫长的生活与写作中,诗人的想法显然变动不居。最初,拜伦说他要写十二章,而到了第十二章,他又说“要慢慢地写它一百章才够数。” 即使是有限的文本,我们也确乎看到,每当唐璜把他多余的激情倾泻到爱情之外的领域,不用太久,他也总会重新被爱情吸引,不论那让他分心的是海难、流浪、战争、政治还是幻想。依此发展下去,唐璜真的会在法国大革命中战死吗?或许另一种可能是,唐璜最终会死于爱情,像阿喀琉斯死于战斗那样,死于自身的宿命。当然,唐璜如何死去将永远是一个迷了。结尾的缺位有时会让解读者迟疑,但另一些时候,它也会赋予解读者
以勇气,去发掘更多的可能。 

从写作形式上看,《唐璜》初看起来则不那么像一部史诗,而更像是拜伦前期纪游诗与东方故事诗的结合与推演。陈中梅在《伊利亚特》的译序中谈到“史诗的结构”:荷马史诗虽然立足于个人,但情节绝不是总围绕着主角展开。除了个人英雄主义色彩的英雄业绩,“也用了相当大的篇幅构建个人背后衬托他们行为的‘集体’。”《伊利亚特》的许多篇幅都
\newpage
与阿喀琉斯无关,第十卷的情节甚至因为缺乏足够的关联性几乎被独立出来。《奥德赛》虽然情节更加紧凑,却也不是严格意义上的“奥德修斯游记”。《唐璜》的情节则始终紧扣着主角,以他的遭遇为线索,夹杂着叙述者的议论。一般认为,这二者的来回滑动构成了《唐璜》的主要内容,这种滑动是在两种不同表现形式(叙事与议论)之间的滑动,性质与史诗在一种形式内部的滑动(具体的情节虽然纷乱,但同属叙事形式内部)有别。在《英国文学史》中,王佐良先生也称这部长诗具有“两重性”。然而,但对文本的仔细辨别会告诉我们,也许《唐璜》并不包含这两
种截然不同的材质。 

事实上,长诗的第一主人公并不是唐璜,而是叙述者,因而“谈话”先于“叙事”。这是一个典型的“故事中的故事”。虽然叙述者自称是唐璜的“老朋友”,但这理所当然地无人相信。从叙述者常常作出的自我表白与对周遭事物的观点来看,这位拥有上帝视角的叙述者就是拜伦。唐璜的经历并不是拜伦要讲的故事的全部,而只是故事中的一部分。同样的,议论并不在故事之外,它也是故事的一部分。一般而
\newpage
言,叙事诗的重点应在叙事,议论作为叙事的异质,最多被当成辅助。如王佐良先生在查译《唐璜》的序言中就认为,叙述者的议论是一种“闲谈”:“闲谈不仅是《唐璜》的十分重要的组成部分,而且因为有了它,作品对现实的挖掘也大为深入了。”然而细心的读者也许会抱怨,拜伦的“闲谈”未免过多。如果它们只是对叙事的辅助,很多时候就会有喧宾夺主的嫌疑。王佐良先生紧接着也谈到:“然而以闲谈入诗并不容易。拜伦的另一胜利,在于他为他那夹叙夹议的风格,找到了合适的诗体,即意大利八行体。”但这一说法也许不能体现《唐璜》在结构上真正的特点。它不是两种材质的交错盘结,而是一个柔软而渐变
的整体。 

首先,我们可以注意到长诗中很少有直接的心
理描写。唐璜的心理活动形式是描述性的: 


  唐璜是真急了:他已经豁出去 


  或受桩刑,或被剁成肉泥喂狗, 

\newpage


  或者细细折磨得他痛死也行, 


  不然也可往大海或狮笼里投; 


  因此他英勇地站在那儿等死, 


  绝不苟且——除非碰上他的兴头。 


  这倒真是丈夫气概!但一遇到 


  女人的眼泪,就不免瓦解冰消。 

唐璜几乎不曾使用“我”来思索和抒情,这些都由故事的第一主角——叙事者代劳了。这就是说,拜伦很少使唐璜自行运动,不论是心理地还是行为的。在拜伦手中,唐璜只是一个不由自主的傀儡,换而
言之,叙事从未真正深入下去,以致可以独立。 

其次,拜伦很少愿意使自己的叙事连贯。在讲述故事的时候,他总是要以一些方式刺破本来可以形成的连贯。有时是一些话外音,有时是直接的自我指
\newpage
涉,有时是添加表征叙述行为的标记(如“言归正传”),有时是插入对叙事的评论,有时则是反讽。当然,更明显的就是干脆抛开叙事的“闲谈”。甚至描写或叙述中就经常直接带有拜伦的议论,比如引文里的“这倒真是丈夫气概”。总之,拜伦总是不断彰显作者的在场。但这不连贯的叙事,读者读起来却并不觉得突兀。王佐良先生借此称赞拜伦为他的口语体诗觅得了可谓天作之合的意大利八行体,承接并发展前人如蒲伯、彭斯等人的口语体诗传统,既而“开创了以后维多利亚朝诗人勃朗宁要走的路,而勃朗宁的语言又影响了更后的英美现代派诗人”。意大利八行体固然已经铺好了铁轨,但也许真正使拜伦的火车能够风驰电掣的,乃是《唐璜》内部和谐的驱动系统——强大的作者将他的意志深深刻入每一行字,以自己富于议论性质的语言,一以贯之地统摄所有叙述、描写
和抒情。 

可资证明的证据有很多。在第十五章,拜伦罕
见地挑明自己的写作态度和方法: 


\newpage

  我从不搜索枯肠,作半日苦吟; 


  我的絮叨就好像是我在骑马 


  或散步时,和任何人的随意谈话。 


  我只凭意兴之所至,写出那 


  浮现在我脑中的旧事或新话 

“随意谈话”是这段诗的关键,这正是《唐璜》浩瀚诗行的唯一来源。按照王佐良先生的说法,有三条理由使《唐璜》成为杰作,且为多数学者所认可。其中第一条便是“唐璜的内容异常丰富,它对当时欧洲现实做了广阔的写照和评论,是一部出色的讽刺史诗”,这句话指出了《唐璜》除叙事之外的两大内容:写实与反叛。而这些内容却不单单由议论承担,拜伦的叙事不仅参与写实,也以明确的反讽与较为隐蔽的对词汇和意象的甄别而参与反叛的工程。这也反映出《唐璜》的叙事与议论统一于拜伦的“谈话”,即本质上的“议论”,或曰议论性印刻的叙事、描写

\newpage

最后,这种议论性印刻并不意味着枯燥。这也
正是拜伦语言的魔力所在: 


  他们彼此望着,他们的眼睛 


  在月光下闪亮; 


  …… 


  就这样,他们形成了一组雕塑, 


  带有古希腊风味,相爱而半裸。 

这段美妙的文字出现在唐璜和海黛相爱的篇章里。结尾的妙喻并不像看起来那样自然而随便,它是一个典型的议论性刻印的描写。拜伦在寻找比喻的时候,并不单单从审美的角度来考量,在这里,他选择希腊雕像作为喻体,不仅具备一般比喻的效果,更是暗示着他心向往之的理想境界。正如前文谈到的,拜伦在描写过程中常常依靠对词语和意象的甄别参与评价,希腊雕像的隐喻正是以这种方式参与了拜伦的“
\newpage
谈话”,即便它如此细致动人,远非铿锵的议论可比
。 

由此观之,《唐璜》不仅在叙事模式上类似一种史诗,即使是初看起来并不相同的写作形式上,也与史诗神似:《伊利亚特》在叙事内部的诸种情节之间来回滑动,《唐璜》在议论内部的诸种“谈话”之间来回滑动。亚里士多德称赞荷马高出其他史诗诗人,是因为他明白“史诗诗人应尽量少用自己的身份说话”,而拜伦铤而走险,尽量多以自己的身份说话,
用“议论性”制造了史诗“叙事性”的平行线。 

以往,学界往往将精力倾注在《唐璜》内涵的解读上,而对于它的史诗性却鲜有论及。但《唐璜》最重要的创新并不仅在“讽刺”,它伟大并不仅仅因它有某种详实的写实而具备历史价值,也不仅仅因它有对世界的评判而具备思想价值,更因为他勇敢而强力地用自己的意志统治了自己的经验、思想和语言,从而无愧于第一章所作出的宣言。按照维特根斯坦的说法,我们不是因为某些现象有一个共同点而用一个词来称谓这些现象,而是因为它们通过很多不同的方
\newpage
式具有亲缘关系。一个崭新的个体之所以能够迫使旧的类型定义做出改变,是因为它与群体的相似甚于差别,或者相似之处较区别之处更为核心。而《唐璜》正超越了史诗惯常的定义,它如果不是以一己之力拓宽了“史诗”一词的含义,便是发明了一种可与其并
驾齐驱的新文体。 


三、超越讽刺 

罗兰·巴特大呼“作者死了”,以鼓励读者挖掘文本实际展现出的而非作者愿意展现出的内涵。这对《唐璜》或拜伦同样适合。如果说前文主要从形式和结构的角度探讨拜伦本人的写作意愿及它的完成度,那么接下来我们再来看看一个天才诗人在精神内涵上是否超越了自己和他人的想象,他实际上究竟走了
多远。 

起初,拜伦依靠《恰尔德·哈罗尔德游记》和一系列东方叙事诗塑造了一批“拜伦式英雄”,以其狂放不羁的反叛、刚烈的人格力量而声名鹊起。诸如异教徒(《异教徒》)、康拉德(《海盗》)、塞里
\newpage
姆(《阿比多斯的新娘》)、拉腊(《拉腊》)、路德派工人(《路德分子》)、拿破仑(《拿破仑颂》)、扫罗王(《扫罗王》)、普罗米修斯(《普罗米修斯》),后期仍有诗剧曼弗雷德(《曼弗雷德》)、该隐(《该隐》)等。这些形象即便不依靠作者的
帮助(议论),也仍然会实至名归。 

简单地说,这些拜伦式英雄唯一不满的就是整个世界。他们怒斥这个因启蒙运动而失去灵性的世界和这个腐朽的、毫无生机的社会,既嘲弄权贵和上层人士的趋炎附势,又鄙夷底层的蝇营狗苟。在《恰尔德·哈罗尔德游记》的第三章,拜伦写道:“我没有爱过这个世界”。这就是为什么人们被拜伦吸引和鼓舞。他强烈的呼号没有耗散在空气中,而是成功抵达人们的耳畔,让那些经历着同样的痛苦、怀有有同样的愤怒与怀疑的个体,感受到一个强大的精神支柱及其背后模糊的同伴。
鲁迅在《摩罗诗力说》中提到他们是“恶魔诗派”。这个词汇是骚赛最先使用的。1821年,他在长诗《审判的幻景》的序言里辱骂拜伦是“诗歌中的恶魔派”,说他败坏国家的政治与道德两方面的基
\newpage
础,是“恐怖和讥嘲、淫秽和渎神的可憎的大杂烩”。拜伦随即写了同题诗,作为对骚赛的戏谑与答复,还笑称他是“多产的诗匠”。在这首长诗中,拜伦将恶魔大肆颂扬了一番,并将弥尔顿奉为自己的先驱。他称天使为“保皇党”,在撒旦面前他们震悚不已。临近结尾,恶魔阿斯摩狄亚捉来了骚赛,让他在撒旦和大天使米迦勒面前唱歌、朗诵,而被天使和魔鬼两
个阵营唾弃,最终被圣彼得击倒。 

鲁迅也提到弥尔顿,谈到他对《失乐园》的看法:“亚当之居伊甸,盖不殊于笼禽,不识不知,惟帝是悦,使无天魔之诱,人类将无由生。故世间人,当蔑弗秉有魔血,惠之及人世者,撒但其首矣。”只有恶魔敢于反抗因循上帝的世界,他们的反抗也是拜伦的反抗。除了《审判的幻景》,恶魔在《曼弗雷德
》和《该隐》两部诗剧中都扮演了重要角色。 

以赛亚·伯林认为滑稽角色的恶魔性转变意味着“同所谓理性主义或启蒙传统决裂”。他将启蒙传
统归结为三个命题: 

\newpage


  1、所有的真问题都能得到解答; 


  2、所有的答案都是可知的; 


  3、所有答案必须是兼容性的。 

启蒙时代的多数人对未来满怀期待,他们自以为正在摧毁古老的偏见、迷信、无知和残忍,正在建立某种科学,以使人们生活得幸福、自由、道德和正义。而反抗的浪漫主义,即拜伦一代,彻底粉碎了这种幻想,把人们从落入陷阱中的启蒙理性中解救出来

这一类角色的代表是曼弗雷德,他是真正的恶魔。“拜伦式英雄”在《曼弗雷德》那里完成,进入形而上的世界;在《该隐》那里则趋于更高的突破。同浮士德相似,痛苦而绝望的曼弗雷德在不寐的午夜唤来大地、海洋、空气、黑夜、山、风等精灵。不同的是,他向精灵们谋求忘却而无果。地狱之王阿里曼涅斯为他召来他的心病——因他而死的阿丝塔特,也得不到宽恕。最后他同时拒绝了天堂和地狱,成为了一个没有归属的灵魂。《曼弗雷德》已经濒临拜伦式
\newpage
英雄的顶点,他让自己成为自己的诱惑者和审判者,从而使自己一个人成为了独立于天堂、地狱、人间之
外的世界,彻彻底底的属于一个人的世界。 

这样一个反抗一切的英雄注定走上绝对否定的道路。而读者一定发现,年轻的唐璜没有跟从他的前辈们,而是改头换面,以一个懵懂而多情的少年形象再度出现在这个世界,不同的是,他眼中的反叛和仇恨都淡去了很多。李赋宁在《欧洲文学史》中论道:“唐璜的形象已不同于感染了许多青年读者、性情孤癖的哈罗尔德式的人物。他更加人世,更善嘲讽,思
绪更加舒展,性格更富有弹性。” 

如果仍说《唐璜》是一部“讽刺史诗”,那么它的“讽刺”只可能来源于叙述者。唐璜作为第二个主角是不善讽刺的,他更擅长行动和爱。曼弗雷德与唐璜都不温顺,然而前者拥有着否定的强大力量,后者则拥有肯定的激情。同时,我们也发现,叙述者与唐璜的行为是相反的,叙述者忙于反讽、批判,疏于赞美,而唐璜驱使自己去生活,去经验,偶尔反抗。这反映着拜伦心里两种观点的斗争,下文我们仍将谈
\newpage
及。而即便是反叛,《唐璜》的强度也远远弱于《曼弗雷德》。除了骚赛这个“头号叛徒”之外,共同被《唐璜》的叙事者讽刺的,还有华兹华斯、柯尔律治这些上一代浪漫主义者,威灵顿等维护腐朽统治者的将领,以及泛泛的君主、上流社会等,而这些在曼弗
雷德对天地的怒吼中都黯然失色。 

作为恶魔或曰反抗者的“拜伦式英雄”产生的影响实际上比《唐璜》要多得多。以赛亚·伯林把恶魔形象的拜伦看做是对启蒙运动的完美反击,他很大程度地击毁了自柏拉图以来世界的独一结构。“拜伦强调不可征服的意志,同时,整个唯意志论哲学,整个必须由天才人物征服和控制世界的哲学观点也由他而起。自雨果起,法国浪漫派都是拜伦的信徒”,拜伦将反抗的火“传给拉马丁,传给维克多·雨果,传给诺蒂埃,传给大部分法国浪漫主义者;再由他们进一步传给叔本华”,后者把脆弱的人抛掷在汹涌的大
海上,暴露出帷幕背后令人震悚的混沌。 

尽管拜伦式英雄们为他们的作者赢得了相当的荣誉,但拜伦已经走得更远了一些。他越来越明晰地
\newpage
发现了反抗的深渊。《唐璜》与《该隐》共同作于《曼弗雷德》之后,当曼弗雷德已经将反抗的道路走到尽头,拜伦终于开始认真地寻找出路,因为他意识到反抗的浪漫主义与毁灭一切的虚无主义仅仅隔着一条浅沟。恶魔起先解放了人们,下一步就是摧毁。浪漫主义的确产是虚无主义的一个源头。法国浪漫主义诗
人维尼也塑造过他的撒旦: 


  ……再感觉不到坏事与善行。 


  他制造不幸,自己也无欢欣。 

加缪认为是虚无主义接替了浪漫主义,拜伦的后人不是雨果,而是波德莱尔与拉斯奈。他更深刻地看到了虚无主义的可怕,历数了形而上的与历史上的反抗思想,认为尼采的绝对肯定等效于萨德侯爵的绝对否定,甚至马克思也未能逃脱,尼采与马克思都以“未来”代替了基督教的彼世。马克思认为人要控制自然以服从历史,而尼采则认为应服从自然以控制历史。他主张以“南方思想”对抗虚无主义。而伯林却指出,浪漫主义的真正继承人是存在主义:“存在主
\newpage
义的关键教义是浪漫主义的,就是说,世上没有任何东西能够依靠。”拜伦式的反抗产生的解放力量为尼采和萨特扫清了路障,只需更加自信,便能拥有尼采所谓的“生成之力量”;只需再回顾它的本源,便会发现萨特所说的“发明”:“你是自由的,所以你选择吧——这就是说,去发明吧。”曼弗雷德事实上已经使得反抗所具备的解放力量消耗殆尽,转而变成了毁灭的力量。因此唐璜不再摧毁,而是在废墟之上一
次又一次进行着发明的尝试。 

事实上,反抗并不是一件十分困难的事。在拜伦之前,席勒就已经写作了《强盗》等作品。而早在公元之前,从埃斯库罗斯开始,反抗者就已经凝聚成为一个具体的形象了。按照加缪的说法,普罗米修斯与该隐是两种反抗的根源。普罗米修斯的火引燃了雪莱手中的火炬,这是一种出于肯定、行于否定、终于肯定的反抗,它的有效在于能够节制恶的手段而达成善;该隐却引诱了拜伦,他的反抗使恶泛滥,因为善与他们的敌人上帝是一伙的。相比于反抗,肯定性的
建造才更加困难。 

\newpage

而拜伦注定不会一直甘受摆布。最终他几乎挣脱了该隐的诱惑,改造了作为否定性反抗之根源的该隐。《该隐》在《曼弗雷德》的基础上继续向前,而在中途出现了一次转向的尝试,令否定的反抗拥有了肯定的希望。曼弗雷德反抗的是一个希腊的世界,这个世界神秘而不强力。而该隐则反抗了一个被基督教袪魅后的世界,它平坦,但强大而神圣。他起先是怀疑,拒绝赞美上帝。在卢西弗的引导下他开始学习严肃地思考。献祭仪式中,亚伯的羔羊比该隐的果实更得着上帝,他便带着对嗜血的造物主的愤怒而杀死了亚伯。人类因具备了获取知识的能力而犯下了第一次罪恶,而第二次则因为使用了这种能力。该隐背负着诅咒和烙印孤独地走向流亡的余生。但在这个拜伦式英雄之外,拜伦的诗歌出现了新的转机:该隐不再是拒绝拯救的曼弗雷德,他的妻子阿达,在该隐沉溺于无尽的仇怨时,看到了万物的生命力。她给予该隐爱
情和幸福。 

阿达形象的塑成是拜伦潜意识的一次集中苏醒。反抗之路上的拜伦常常遗忘他曾经肯定的事物,但有时埋在他心底的冲动也会提醒他将眼光从丑恶之物
\newpage
移开,关照生命的光彩。他也曾发现了普罗米修斯的
伟大,虽然当时他也许没能重视: 


  “你神圣的罪恶是怀有仁心” 

异曲同工的是,尼采说:“我爱人类,而当我克制住这种欲望时,就最是如此”。浪漫主义的恶魔开始记起从前他们也是人类。因此,《该隐》的转机不是无来由的妥协,极致的反抗并没有毁灭拜伦,在他的心中永远存在着对另一种激情,也许他没有想到,这种激情与被他唾弃的湖畔诗人们似乎只有程度的
差别。 

伯林发觉浪漫主义者常常徘徊于两个极端之间:即神秘性的乐观主义和恐怖的悲观主义,而唐璜正是他们的孩子——也就是该隐和阿达的孩子。唐璜调和了这两种极端,这也是拜伦不断与自己斗争的结果。拜伦的困难在于他要同时对抗两个敌人:衰弱的启蒙理性与毁灭的虚无主义。之所以困难,是因为这二者之间几乎不存在空隙。真正的建造是稀缺的,人们

\newpage
总是在废墟上建起另一座废墟。 

《唐璜》的伟大就在于这种对“建造”的探索。它的可贵绝不在它对腐朽的政治与社会的反抗,也不只在对自由、生命的向往。也许拜伦自己也没能完全搞清楚他真正想要探讨的是什么,但他的天才已经将他真正有价值的思想表露了出来,尽管是在一些不
起眼的角落: 


  至于我,我只愿意面对这戏台, 


  对茅屋或宫室都不加以褒贬, 


  好似歌德的魔鬼,纯作壁上观。 


  无论爱或恨我都力求不过分。 

这段诗中体现一种拒绝评价的态度,或者说,是一种“无视”的魄力。这种态度也许是同时解决两个敌人的钥匙。这种无视意味着境界的超越,它与否定和否定之否定不在同一层面。在超越对手的情况下才可能做到无视,而这种无视使得真正的建造——不
\newpage
受约束的建造成为可能。早在塑造起反抗的英雄形象之前,早年的拜伦就写过一首广为传唱的《雅典的少
女》: 


  我要凭那无拘无束的鬈发, 


  每阵爱琴海的风都追逐着它; 


  我要凭那墨玉镶边的眼睛, 


  睫毛直吻着你颊上的嫣红。 

多数时刻,我都愿意一厢情愿地认为拜伦最好的作品在这初出茅庐的时刻就已经写就了。在这短短的几句中,拜伦最本源的追求已经显现,那便是希腊情结:自由(无拘无束)、自然(爱琴海)、快乐(追逐)、美(墨玉镶边)、生命力(眼睛)、激情(直吻)、爱情(嫣红)。这一切构成了拜伦心中的圣地。在其后的创作生涯中,类似的象征性词汇也都频频出现,而拜伦式英雄澎湃的反叛常常掩盖这些须臾

\newpage
的光明。 

这些遮掩到了写作《唐璜》时明显地有被摈弃的倾向。《该隐》体现了拜伦的回忆,这种回忆是痛苦的,因为拜伦不是哲学家而是诗人。《唐璜》也是如此,叙述者和唐璜在诗行中争夺着主题,叙述者试图继续他的讽刺,而唐璜却对他的冷嘲热讽不以为然,他的注意力总被眼前的事物夺走,还要不时回想起他的朱丽亚,他的海黛,他的古尔佩霞和杜杜,他的喀萨琳,他的阿德玲、奥罗拉和费兹甫尔克夫人。
唐璜遇到的所有女人中,海黛是他的挚爱,她简直就是十多年前拜伦在希腊遇到的“雅典的少女”。朱丽亚让唐璜成人,而见到海黛的一刻她就被忘记了。后来,古尔佩霞“用那包含权力和热情的蓝眼睛”引诱他:“基督教徒,你能爱吗?”而唐璜却还在怀念着“海黛的岛屿和那爱奥尼亚的温柔面孔”。在相隔近八百页的时间之后,美丽的奥罗拉也打动了唐
璜,可他想到的第一个人仍然是海黛: 


  她不凡,但不像他那失去的海黛, 


\newpage

  她们在各自的世界里闪着光辉; 


  那海岛的姑娘生于孤寂的大海, 


  她完全是自然之子,天生热情 


  甚于沸腾的海,却也赤城、可爱; 


  但奥罗拉的特点完全不是这些, 


  她们像鲜花和宝石那样有别。 

在海黛的岛屿上,拜伦彻底抛却曼弗雷德和该隐的废墟,建造他魂牵梦绕的希腊。他不再理会从前的焦虑,讽刺没有出现。他全神贯注于这不知疲倦的建造。从昏迷中相见到第二章结尾,拜伦几乎用了一百节来描写唐璜与海黛的爱情,这可算是拜伦最投入的一次,他一反常态地很少插话,因此语势和画面几乎不曾中断。此前和此后,现实中或诗中,再也没有
谁有幸得到拜伦如此滔滔不绝的赞美了。 

和海黛关联的一切都是善的。就连兰勃洛,海
\newpage
黛的海盗父亲,也有“南国的秀气”和“爱奥尼亚的
心灵的优美”: 


  但仿佛有一种古希腊的精神 


  把一缕英气注入他的灵魂中, 


  一如在古昔,这精神曾鼓舞了 


  他那些寻找金羊毛的老祖宗。 

拜伦自己也有这样一股“英气”,奥登曾称赞他是“潇洒风格的大师”,“诗如其人”在他这里得到了很好的体现。希腊就是拜伦的理想国,这里的一切都未经污染。爱奥尼亚,希腊哲学在这里发祥,米利都学派一度观望着凡人无法染指的星辰;而金羊毛不仅召唤着英勇的伊阿宋和敢爱敢恨的美狄亚,也召
唤着他们同样自由的子孙。 

走出希腊的唐璜立刻失去了这种无视一切的勇敢和自信,重新陷入两种黑暗的焦灼。他再也没能做
\newpage
出在海黛的岛屿上所做出的建造,而是重新回到无计可施的现实。即便如此,唐璜的身上仍然深深印刻了希腊的阳光,仍然有原始的激情能够从他的身上源源不断地涌现。刻板的世界里,仍有一些人和事物能够引他投入生活,每被打动一次,就似乎又看到了海黛
的岛屿。 

作为反面典型的是拜伦的祖国:爱恨交织的英格兰。那里的水土哺育了他也限制着他,那里有他爱的人也有辱骂他的人。他的祖国“把自由应许给人类,/ 而今却要捆住人,连思想在内”。唐璜出使英国,还没能稍稍感受她的美,便被突如其来的官僚和铜臭气息扫了兴致。在这里,“太阳落了,烟雾象从半灭的火山口腾起来,弥漫着天空”,有着发达的“报刊、诉讼和毁谤”,枯燥的理性遏制着“生殖的爱
好”: 


  少年唐璜只游历过浪漫的国土, 


  只知爱情涉及生死,而不是诉讼。 

\newpage

这里“浪漫的国土”显然在指涉希腊,在那里,富于原始激情的海黛为他而死。在拜伦的家乡,唐璜却惊讶地发现,背负着生命之重的爱情竟然可以拿来交易和交涉。伯林所说的“思乡情结”出现了,从
这一点来看,拜伦也是海德格尔的先驱。 

拜伦不曾明确地说出希腊在他心中的含义,他只用自己的诗篇和死亡给予人们一些暗示,生于英国而死于希腊竟是他一生的注脚。而加缪则花了很大的篇幅来阐释希腊情结。在他看来,希腊精神意味着“节制”与“生命”。他认为,长久以来,欧洲人为着各种各样的原因而不再热爱生命,这是它悲剧的原因

在拜伦的所有继承人中,加缪的气质并不是与之最相像的,但加缪完好地继承了拜伦“反叛”之外的基因。除了在众所周知的《西西弗神话》中提出了“唐璜主义”以示范荒谬人的秉性,加缪在《反抗者》中论述的该隐也极类拜伦塑造的该隐。论及浪漫主义者,他用“花花公子的反抗”来暗示他对《唐璜》的看重。当然,他也继承了拜伦最核心的密码:希腊,这块自由与生命的土地也牵绊着他的一生。在《南
\newpage

方思想》的最后,他深情地说道: 

这个世纪深深的冲突是德意志梦想与地中海传统的冲突。现在,卑鄙的欧洲失去了美与友情,正在死亡,而太阳思想,具有双重面孔的文明,等待着曙
光出现。 

从某种意义上说,加缪继续了拜伦未竟的事业之一,他阐明了自己的也是拜伦心中的理想。拜伦用手做出了希腊的素描,而加缪却画出了她的结构图。细心的读者会发觉,孕育一切的太阳悬在了他们的头顶。太阳意象贯穿了加缪的小说。在《局外人》中,默尔索是一个收敛的唐璜,对这个窒息的理性世界怀有同样的拒绝和无视,而怀念着地中海的海浪、海风,温和而毫无保留的阳光。默尔索正是在太阳耀目的晕眩中扣动了扳机。在《唐璜》中,太阳也反复地出
现: 


  和自然为伴,不懂那一切, 


\newpage

  海黛是热情所生,在她的故乡 


  太阳发出三倍光明炙烤着人。 

朱丽亚曾姐姐一样爱抚着已经长大的唐璜,那时拜伦说这是个“太阳灼热的国家”;在唐璜与海黛的晚宴上,一个诗人吟诵久负盛名的《哀希腊》:“永恒的夏天还把海岛镀成金,可是除了太阳,一切已经消沉”;女人们也让“北极”一般的伦敦迎来了“夏天”,“这儿全是阳光”。古尔佩霞更能体现太阳
非同一般的特性: 


  如果说太阳没有斑点,那么她 


  就比太阳更无可挑剔,更完美。 


  太阳每一年把北极的冰削减, 


  对于罪恶,它的效果却适得其反。 

太阳作为一切的初始,是一个中性词。它不意味着善或者恶,而是混沌。善恶的概念是它的产物,
\newpage
而它当然高居其外。引文中的“罪恶”按照惯例也是一种反讽,隐含着拜伦对世俗价值的蔑视。“太阳”与“希腊”在拜伦眼中是同义词,二者都代表着某种原始的力量,后来的一切都从中衍生,也终将回到那里。在那里,活着的人们是在生活,在行动,在爱,而不是在做价值判断。就是说,拜伦的希腊意味着绝对自由的行动,是无所顾忌地投入生活。不是刻意地反抗和追求,而是溯源而上,忘我地建造。就在他写就唐璜与海黛的片刻,他超越了反抗和追求的二元对立,从启蒙理性与虚无主义的裂隙中发掘出了更广阔
的世界。 


四、余论 

李赋宁在《欧洲文学史》中说,“评论家多认为拜伦虽有一流的讽刺与叙事才华,但总的来说享受了与其才华不配的名誉。说他伟大,似主要不是诗文本身的缘故,因在所谓六大诗人中,他的文思较缺乏深度。”他也提到了拜伦在很长一段时间被论者冷落的史诗。王佐良先生也说“他不是一个深刻的思想家

\newpage
。” 

的确,拜伦的叛逆有时会过头,忘记严谨和得体。他对华兹华斯等先辈不加区分的批评,常让人反感他的尖刻甚至鄙陋。但是这就是拜伦,他认为诗人和诗之间不应该存在割裂,华兹华斯等人的确做出了不甚高尚的事情,这一点足以让拜伦愤怒。他理解的
诗必须是一种真诚。 

另外,拜伦所追求的事物又常常太过浮泛,也经常被反叛的自己打断,让读者不得要领。口语体诗歌很容易遭受这类质疑:正如以貌取人是一种难以克制的习惯,人们也愿意相信,真正的思想总是与雕琢
的文字联系在一起的。这种习惯也是危险的。 

在《唐璜》中,拜伦也展示了自己的眼界。除了被他批判的华兹华斯、柯尔律治和骚赛,他还用过许多篇幅来提及或者纵谈诸如普罗米修斯、荷马、阿喀琉斯、伊壁鸠鲁、第欧根尼、柏拉图、亚里士多德、贺拉斯、朗基努斯、西塞罗、奥古斯丁、弥尔顿、屈莱顿、薄伽丘、莫里哀、蒲伯、伏尔泰、霍布斯、马基雅维利、贝克莱、济慈等一大批哲学家、文论家
\newpage
、诗人、神话和宗教人物,显示出深厚的阅读功底。从这个角度讲,说拜伦思想浅薄是值得慎重考虑的。拜伦的激情在世间是稀缺之物,常常吸引读者大部分的注意力,而往往也因此遮蔽其同样值得关注的精神
内涵。 

也许思想的明晰性是拜伦的弱点,但即便真的如此,这也正是他真正伟大之处。他向人们展示了天才诗人沿着他的敏感所创造出的可能性,并且,这也是在向着批评家们发问:“思想不深刻”是否可以构成对诗人(而非哲学家)的诟病?换而言之,思想的深浅是否是衡量一位诗人的恰适标准?且不说语言的运用是诗歌不可忽视的评判标准之一(王佐良先生在查译《唐璜》的序言中也提到,拜伦的手稿中存在大量对词语的修改,表明他用词的苦心和刻意求工。他又评价“它在口语体的运用上达到了英国史诗上的最高成就”),说到底,诗人是如果本质上不是经验者,那么同样的,他也不可能成为一个合格的思辨者。
没能完成的《唐璜》令文学史遗憾。但拜伦自己可能并不会如此认为,因为遗憾大概是一个他永远
\newpage
学不会的词。他是一个行动的诗人,从荒芜的世界中催发出生命的激情是他真正的事业,他根本不会被易朽的事物迷惑。对他来讲,未能活着带领希腊走向光明或许才是最值得伤感的。他明白,哪怕是青草微弱的生命,也会映射出作为人类创造物的诗歌之廉价。拜伦以对诗歌的真诚和对生活的真诚,而无愧于自己
诗人的称号: 


  任文坛上的贩夫走卒去争执吧, 



  反正我坟头的青草将悠久地 \par   对夜风叹息,而我的歌早已沉寂。

\end{document}
