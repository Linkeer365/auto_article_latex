\documentclass{article}
\usepackage[utf8]{inputenc}
\usepackage{ctex}

\title{枕书\footnote{Click to View:\url{https://web.archive.org/web/20221026062638/https://rentry.co/z7vvo}}}
\author{程艳丹}
\date{2005-09}

% \setCJKmainfont[BoldFont = Noto Sans CJK SC]{Noto Serif CJK SC}
% \setCJKsansfont{Noto Sans CJK SC}
% \setCJKfamilyfont{zhsong}{Noto Serif CJK SC}
% \setCJKfamilyfont{zhhei}{Noto Sans CJK SC}
% \setlength\parindent{0pt}

\begin{document}
\CJKfamily{zhkai}

\maketitle


\Large

枕书搬到姑姑家来住了。枕书姓苏,一看名学便能着出她出自书香门弟。不错,枕书的太祖是举人,祖父是大学生,父亲是L程师。姑姑苏眠月也是
一派小姐牌气。 

枕书住在姑姑书房里。那天下午有一张娃娃脸的姑父在书房里忙着架木床,见枕书提了只包走进来,便一脸天真地笑:“枕书,累了吧,来,吃梳子!
” 

橘f皮鲜效嫩的,桃书玉滑的手指中在橘皮上轻轻一划,便沁出很多清凉的汁水,酥酥地溅在枕书脸上,有点酸,香香的。枕书皱皱眉笑笑,剩开需皮,捏了一瓣送入口中。姑父又抬头笑笑。枕书有点脸红,侧过头去看书架。书架有两面,白地板直至天花
\newpage
板,一架小梯子倚着南墙,书架拐角处有一盆蔽苑的兰花。枕书一抬眼便镖到书架上一本《橘子不要哭》
,李碧华的。 

床架好了,姑父道:“枕书,去冲个澡,冰箱里有吃的,自己在微波炉里热一下。我报社还有点事
,出去一趟,你姑姑晚饭前回来。 

“哦。”枕书一边铺床一边点头。姑父出去了。枕书去浴室。浴室小小的,很温馨。枕书披着头发看着镜子里的自已,定定地看了很久。她有点头晕,虚虚地唤道:“枕书,苏枕书。”镜子里的人也虚虚地叫:“枕书,苏枕书。”枕书傻傻一笑,去开热水喷头。蒸气弥漫开来,镜子模糊了,一切都模糊了。
浴室里各种香味都活跃起来,她有些迷醉了。 

枕书吃了一只小苹果,使到书房里来。明天就去这座城市最好的中学了,枕书有一丝莫名的兴奋。有开门的声音。是姑姑。姑姑把坠有玉佩的钥匙串扔
到客厅的红木桌上:“枕书,什么时候来的? 

\newpage

枕书迎过去:“下午才来的。姑姑撩撩头发:“明天开学吧?来,姑姑有好几套衣服,只穿过一两
次就不想穿。可都是名牌,你来试试。 

枕书走过去,姑姑打开自己卧室的衣橱门,一口气找出五套衣服,扔在床上。姑姑拉着枕书一件件试过去。姑姑突然按着枕书不动了。枕书穿的正是一件淡绿绣碎花的真丝上衣,白色丝绸长裙。枕书真的很美。姑姑下意识地映了眼衣镜,心中一动,二十六
岁的人和十六岁的人真是不一样。 

姑姑为枕书腾出间衣橱来专门放衣服。姑姑说在大的学校里穿衣服一定不能含糊。姑姑着看枕书又把她拉到浴室里。姑姑把枕书两条垂耳小辫解开来,拿木梳梳顺她的头发,从一只镶玉小盒里仔细挑出一对浅绿绸制头饰很久,微微一笑;“好了,去吧,明
天就这么打扮。" 

晚上姑父回来,见了枕书,似乎有点儿吃惊。
他微笑着给姑姑夹菜,也给枕书夹菜。 

\newpage

晚上睡觉,枕书裹着毛巾被,愣愣地看天花板。姑姑眠月跟姑父江南是分开瞬的,眠月睡西房,江南睡东房。枕书很奇怪,姑姑都和江南谈了六七年的朋友,为什么苏家人还吃不到她的喜糖?她嫌江南不好么?枕书侧着身子看书架,很多很多书、溢出书香
来,枕书有些困了。 

新学校很大,枕书被分在一班,全校十六个高一班中惟一的奥赛实验班。枕书坐在教室里,突然觉
得自己有一种骄傲的感觉。 

教室里已来了好几个人,他们都有一种叫人难以接近的傲气。枕书靠窗坐着,翻一本新到的《书屋
》。 

“你好,我能和你坐吗?”走过来一个短发女孩子。枕书点点头,女孩子坐下来:“我叫水鱼,水里的小鱼”枕书笑了:“我叫苏枕书。”水鱼歪着头:“好诗意的名字。”枕书笑道:“水鱼是你的真名啊?”水鱼点点头:“当然,我爸姓水,我妈姓余,

\newpage
年年有余的余。 

水鱼很能说,她是本校初中部毕业的,对这里
的情况了解得很多。水鱼告诉枕书很多事儿。 

“他叫莫言,初中时成绩奇好,竞赛全是轻而易举拿全国第一。这次考奥班,他又是第一。”水鱼指着第一排第一桌一个穿深蓝T恤的男生吐吐舌头。
 

开学第一天,枕书和水鱼就成了好朋友。老师
调整位置,她俩又成了同桌。 

奥班很苦,竞争起来是无血的战场,刀光剑影都化在空气里。开学一个星期枕书就开始这样想。枕书在班上不出众,总是隐在靠窗的座位上,默默做作
业。枕书有点想念过去,也有点想哭品 

姑父待枕书很好。每周日下午都开了车带枕书去超市或书店。姑姑有很多亮衣服,高兴起来她会让
枕书一件一件试穿 

\newpage

枕书爸来了。那日下午枕书放假,正在看苏童的《妻妾成群》,身边是一摊未完成的物理习题。枕书爸愤怒地夺过书摔到书桌上,大叫昏头了枕书。姑父江南走过来,哥哥别生气,枕书才看了一会儿。枕书不做声。枕书爸生气地说枕书你太叫我失望了,你
给我好自为之。说罢转身便走。 

枕书跪在书房地板上仍不做声。江南走过去:“枕书。”枕书木木地:“哎。”江南俯下身:“枕书,别生气了,啊?"枕书浅浅一笑:“生什么气啊,跟他生气有结果吗?”“枕书。”江南笑了,脱口而出,“你倔得可爱,有点像《妻妾成群》里的颂莲。”枕书勃然色变:“颂莲?”江南慌道:“哎呀,对不起,枕书,姑父中午喝多了,不是颂莲,是《暮鼓晨钟》里尊贵的冰月小格格。”枕书不置可否地笑
一笑,转身去写物理作业。 

期末考试,枕书全班第十,全校第十六。她还是挺满意这成绩的,拿了通知单回到家,迎面来的便是父亲的斥骂;“第十?好意思说。以前怎么考第一的?”枕书觉得太不可理喻,冷冷地扫过去一眼,还
\newpage
什么工程师呢,一点头脑也没有。父亲似乎读出枕书眼里的不屑,使一拍桌子:“你给我滚,考了第一再
进我的门!” 

枕书离开家了。背着包一个人在街上走,人来人往都在准备过年。枕书有点想妈妈。枕书想妈妈离开爸爸这一选择是对的,跟这种男人在一起的女人能幸福吗?哦,姑父真好,姑姑真不懂得珍惜。枕书想,将来如果谁对自已这样,自己一定死心塌地对人家
。 

“枕书,枕书!”有人在唤。枕书回过头去:“水鱼!”水鱼穿着大红绒外衣,短发上缀了两只绒球,很俏皮:“怎么一个人在街上?”枕书笑笑:“闷呗。你在买东西啊?”“我也在闲逛呢。"水鱼给枕书一条“益达”。“中午到我那儿吃饭,好吗?”枕书间:“哪儿呀?”水鱼挤挤眼:“等会儿就知道
了。走,先跟我买衣服去。” 

水鱼眼光很挑剔,走遍了整条街才从一家精品时装店里买到了一身勉强满意的衣服。水鱼穿着这浅
\newpage
紫丝绒高领衫子,黑色皮制缀宝石长裤,有一种炫目的美。水鱼又买了一件磨砂玻璃的紫发卡,在镜子前歪着头想半天,终于满意了。枕书打趣:“打扮这么漂亮,相亲啊?”水鱼粉脸一红;“枕书,开什么玩
笑嘛!” 

水鱼把枕书拉到一家小餐厅一一蓝月亮美食坊内。枕书看到了靠窗一桌坐着一一莫言。枕书有点恍然大悟。水鱼笑着跑过去:“莫言!"莫言领首:“水鱼,你这样子好美。哦一这是一一苏枕书吧?”枕书很不自在:“水鱼,奠言,你们聊,我先走了。"水鱼一把拉过她:“枕书!你怎么这样子,是我好朋
友吗?来来,快坐下。”枕书无奈。 

水鱼很兴奋,点了许多菜。枕书默默搅动着面前的一蛊海鲜汤。奠言话不多,头总是微昂着,唇边
是不可一世的一丝微笑 

吃过饭,水鱼嚷着要去游乐场,枕书忙推说有
别的事,便逃也似的离开了他俩。 

\newpage

接着去哪儿?家是不能回了。盲目地在街上走着,竟疃到姑姑住的那幢楼下去。枕书被姑父叫住了
,接着枕书便上了楼。 

姑姑抱着只枕头在沙发里坐着,正认真地咬一只苹果,膝上是一本三毛文集,她看看枕书:“怎么,被你爸骂了?”枕书不做声。姑姑撇撤嘴:“我小时候不也叫他骂么。就这公子牌气,我还怕他?”枕书定定地看老式座钟的钟摆,摇来晃去叫人眩晕。姑姑把苹果核给了姑父:“江南,你出去买点菜来,枕
书今年寒假别回去了。 

枕书泡在书房里感觉太好了,她不分日夜地读
那些书,日子在书香里无声地飘走。 

水鱼告诉枕书自己有男朋友了,就是莫言。说心里话,枕书是看不起现在的女生有什么男朋友的。她想,即便有了梦里的人,也该矜持些,像古诗词里
讲的“倚门回首,却把青梅嗅”该有多好。 

英言真是太优秀了。他就坐水鱼后面。相处久
\newpage
了枕书发现莫言还不是那么让人难接近的。他也有不懂的东西。比方说,他有时也会问枕书,“梅子黄时雨”出自哪儿?枕书便会轻声把贺铸那首《青玉案》背出来。而枕书要有题做不出来,是绝对不会间莫言的,她宁可坐在那里想一道递推数列题想半天,也不
愿回头问一句。 

姑父对枕书很好。每周日他都会开着车带枕书去大超市买吃的。枕书爱吃徐福记的太妃糖,橘子味的。枕书也爱吃萨其马,听说当年是梳了两把头穿旗
装的宫妃、格格们最爱吃的小食儿 

那是暮春吧,从超市回来的路上姑姑就开始呼姑父。一共三迎。当姑父和提了一袋台尚萨其马的枕书一起站在那儿的时候,姑姑冷冷一扬手,小茶几上一只景德镇细瓷青花碗便碎在地上。姑姑冷笑:“逛得开心?"枕书榜了。姑父赔笑;“眠月,什么事儿,呼我三迎?"姑姑抚着自己修长的手指:“不敢了
江南,可打搅你了1” 

泪在枕书眼里润了一圈,枕书一狼心,咽回眼
\newpage
泪,直直地望着站站。姑姑死死盯着枕书。枕书穿着淡蓝棉布外衣,胸前有一弯水蓝月亮,两条细辫被两根湖蓝缎带束着,清水洗过一样水灵。姑姑穿着浅黄堆云绣的绸制旗袍,长发盘着,情懒,尊贵。枕书有
些麻木,侧过头去 

江南忙进屋给姑姑湖了碗碧螺春,把地上的碎瓷收拾了。姑姑合了书,起身头也不回地进了自已的
西房,重重摔了门 

枕书一步步走到书房里,一步步登上木梯,她要拿那本陈杯改写的电视剧本《青衣》。她静静地读那本书,筱燕秋的爱情,筱燕秋对戏曲的痴迷,一点点把枕书带到一个空灵遥远的地方。她突然很敬佩书中的筱燕秋,在她眼里筱燕秋成功了,她在冰天雪地里舞着多年前的那出戏,她成功了,她涅槃了,尽管她付出了太多,太多。尽管她仍然痛苦在冰天雪地里。枕书把《青衣》放回去,趴在床上想了许久。她有点想妈妈。妈妈是个作家,在枕书八岁时受不了枕书爸对文学的极不理解,愤然离开了。妈妈现在在哪儿呢?新西兰还是荷兰?妈妈又嫁人了。枕书想自己与
\newpage
妈妈的人生观是不同的。妈妈渴望无羁热烈,自己只
想拥有云淡风清。 

江南在度房里做饭,香味溢出来。枕书到阳台上去,伏在窗口,枕着手臂。枕书有点儿累了,枕书
要休息。 

有淡淡的奶香,一点点钻进枕书的鼻端。枕书抬抬眼皮,是一张娃娃脸,有清澈的眼眸,有浅浅的微笑。他手里托着一块萨其马:“小格格,吃吧。”他开玩笑。枕书没有推辞,抬起头轻轻咬了一口,含在口里,很香,很好吃,软软的,一点点酥暖暖地化
掉。江南笑了:“去吧,吃饭了。 

天下着小雨,把枕书的心情淋湿。枕书想把心
事翻出来晒晒太阳。可是,雨还在下。 

水鱼妈妈来学校了。水鱼说她妈妈刚从云南出差向来,带了很多好吃的。两个女孩子夜自修时躲在书堆下分吃一只奶黄色的芒果。水鱼略略笑着,抬起腕来,一件银质的首饰,丁丁东东。银色的光把水鱼
\newpage
笼得像深海里活泼的小鱼儿,很可爱,很可爱。“莫言送的?”枕书轻问。水鱼点点头,笑腾里盛满幸福

班上组织奥赛辅导。枕书报了数学,水鱼报了化学。枕书回家去,把申请奥数辅导的表格给父亲,要他签字。父亲间为什么不报物理。枕书一侧头回答,我爱数学,父亲火了,说:“这是爱不爱的问题吗?将来考名牌物理必考,你数学已经不错了、物理100分却拿不到95,你为什么不再钻研钻研?死懒!以前我16岁的时候在路灯下咬干馒头做题目看书!身在福中不知福!只知道看那些毫无意义的杂书古书,你要当什么作家么!告诉你,作家没饭吃!玩文字游戏有什么了不起?虚无!也像你妈那样嫁个老外风光去!”枕书一扬头:“不要你管!”父亲甩过来一巴掌,被枕书抬手挡住:“你没有这个权利。”父亲气极,撕碎了那张申请书。枕书冷冷地弯腰将碎片拣起,头也不回地走了。“回来,有本事别进苏家的门!”父亲大叫,枕书没有哭她也没有进苏家的门,
更没有去姑姑家。她就一直在街上走着 

苏枕书?"一辆山地车在她身边停下米。“莫
\newpage
言?”枕书很意外,“你一"莫言微笑着:“怎么,散步?可真有闲情逸致。枕书笑笑:“原来你也会笑的。莫言推着车:“俊话,我是木头人啊?枕书道:“哦,我以为你只会对水鱼笑呢。莫言微昂着头:“是呀,找也以为你只会对水鱼笑呢。莫言的眸子若寒星,若宝珠;“去我家坐坐吧。枕书想想,轻轻点头。莫言父母都在北京做生总,他们拥有两家公司。英再的房间里有很多奥赛书。莫育给枕书池了咖啡。“今晚你住我妈的房间吧。"英言道。“啊一不!”枕
书惊诚地,“我回去,回去。 

“我送你。”英言道。枕书突然抬头道:“不,不,谢谢,我自己走。”莫言楞了愣:“那好,冉
见。" 

风有点儿凉。枕书问自己,去哪儿,去哪儿?她想哭,自己是哪儿也去不得了。路上人来人往,都是回家去的吧?空气里有樱花香,这地方哪儿还有樱花?枕书抬头望望四周,有菟虹灯,有汽车,有商场,有道旁树,没有楼花。枕书抱紧了怀里的书包。一辆白色的轿车在她身边停了下来。“姑父?"枕书心
\newpage
里一暖。车窗一点点摇下来,姑父那娃娃脸一点点明晰:“枕书,进来吧。”枕书迟疑:“姑父。”江南打开车门:“叫我江南。”枕书脸有点儿红,没做声,也没动。江南微笑:“进来,我有好东西给你”枕书迟疑着进去,姑父侧身递过来一只深碧描漆的茶叶盒:“打开看看。”盒子很大,像古时的食盒。枕书拉开第一房,馥郁的芬芳扑面而来一清水里,养了十七朵洁白的柜子花!枕书忙拉开第二层,七只各色各样的中国结。再拉开第三层,十七只奶油萨其马。“生日快乐!”站父笑了,“真巧,原本要去你家送给你,没想到你在街上。”枕书不知该说什么:"我…
…"走吧,时候不早了。"姑父启动引擎。 

姑姑到周庄去了,单位组织旅游。书房里那盆素心兰已开了花,水蓝的小花瓣娇娇嫩嫩的。枕书坐在写字台前把那么长的中请表一点点拼起来,在家长签字栏里写下“苏枕书"。她想自己是自已的主人,
自己已经十七岁了。 

枕书挑了一朵还未全开的桅子花压在了自己最爱的一册词书里。那一页恰有晏殊的《蝶恋花》:“
\newpage

昨夜西风满碧树,独上高楼,望尽天涯路。” 

上竞赛课好苦。报数学的女生极少,枕书看化学班物理班一大群女生,再看看数学班,心想自己是不是太任性了,明知道自己上化学班可有获一等奖得20分的可能,上物理班可以提高高考成绩,上数学班呢?获奖是天方夜谭,要想听懂老师讲的内容也很困难。可枕书还是不做声,蹙着眉仔细望着一黑板密
密麻麻的三角函数等等。 

枕书发现,每次上竞赛课,莫言总是坐在自己右边。他那修长白皙的手指夹住一支修长光洁的圆珠笔,时而在纸上写东西,时而轻盈地转起来。渐渐地,枕书也开始轻声问莫离几道题。英言很简洁地答完
题,又微昂起头来。 

这天姑父去贵州采访,姑姑一个人在家。夜自修后枕书回到家,感觉屋里气氛实在不一样。姑姑刚洗过澡,湿流流的长发笼著一张白玉一般冷的脸,姑姑的睡衣长及脚面,雷白的底子,开着暗紫硕大的玉兰,散溢着梅雨时节潮湿的气息。姑姑侧着头,冷冷
\newpage

地望着枕书。 

“姑姑。”枕书低低地唤,想躲过她的目光,进浴室去。“站住。”姑姑沉沉地发话。枕书回过身来。姑姑冷笑着,丢给枕书一只大木盒:“真聪明啊,苏大小姐,连江南的生日都弄清了,了不起,了不起!”枕书蒙了,继而脸红到脖根,姑姑翻了她的东西,翻了她预备送给英言的一套古筝古琴曲,可是姑
姑弄错了! 

“你凭什么翻我的东西!”枕书涨红了脸“哼,凭你叫我姑姑,凭你住在我家,凭你爸爸过去藏人家备给我的东西,凭休爸爸破坏我的婚姻!”苏眠月的唇颤动着。枕书莫名其妙。苏眠月愤怒了:“你很奇怪找为什么不结婚是不是?我告诉你,因为找真爱的人在我二十岁生日那天被你爸气走了!我在赌气,你懂吗?因为江南不配我!你现在既然要同我竞争,好,我奉陪,你有年轻,我有魅力,看谁胜谁负!"

枕书博然:“姑姑…”“你送吧,把这东西送给江南。我告诉你,下个星期欢迎你光临我们的婚宴
\newpage
!”姑姑走了。枕书把木盒放好,走进浴室。喷头的热水把枕书的细皮肤烫得娇红。“竞争………婚宴……江南………”枕书笑了,姑姑好傻,竟有这样的念头!江南,江南,就算自己不想叫他姑父,也不过是
想叫他哥哥而已。姑姑,真是小姐脾气。 

枕书在书桌边坐下,看那木盒里的磁带,《阳春白雪》、《平沙落雁》、《十面埋伏》、《春江花月夜》……还送给英言吗?算了吧,还是独守这份心境,要别人理解做什么呢?就算理解了又有什么用呢

平日只知道莫言数学特棒,真不懂他计算机为什么也这么出色。没出声儿他就获得了南大保送资格,专攻计算机。他转学去南师大附中,水鱼迷慨地问
他;“你很行的,为什么要保送?" 

“因为我喜欢捷径。再见,小鱼,再见,枕书。”莫言微扬着头,留给水鱼一个茫然的背影,留给
枕书一个苍白的背影。 

水鱼失落地折着伤感的纸鹤。枕书释然地抚着
\newpage
那只木盒一一所幸当初不曾送出去!日子紧张而有序,一如编好的程序。枕书仍一课不落地听奥数辅导,
尽管功课繁重,课程艰深。 

只是有时枕书会莫名地伤感,枕书想,自己已走过了花季,快要走进雨季,十七岁的女孩子该是这
样伤感寂寞的。 

姑父成了真正的姑父,他一身大红绣金福的对襟唐装,姑姑一身大红旗袍,一脸富贵满足的笑。苏家来了很多人参加婚礼,枕书爸爸心情也很好,直夸妹妹妹夫是天生一对地造一双。苏江两家人都喜气洋洋,姑姑姑父也喜气洋洋。枕书端着酒盏为姑姑姑父敬酒,用语得当,措词华美,大家都笑着议论,这是眠月的侄女儿呀,下半年上高二了,模样标致,成绩
又好,难得难得,苏家尽出才女呀 

枕扰记经决定搬走,搬到水鱼家去。因为姑姑已经怀孕,在他们那里不方便。水鱼爸爸在外当军人
,家中只有水鱼和经常出差的水鱼妈妈 

\newpage

一周以后,水鱼莫名其妙地开始躲着枕书。枕书间水鱼:“你怎么了?"水鱼突然大吼道:“你装什么糊涂……昨天莫言喝醉了,一遍又一遍喊你的名
字……" 

枕书呆呆地蹲到地上。她不明H这个世界怎么了。姑父对她好,难道她不该对姑父微笑吗?莫言学习好,难道她不该和他交流吗?他是水鱼的朋友,难
道月己就不能和他作朋友… 

她只好又回到家里。面对爸爸的讥讽她不说一句话,除了课堂提问她必须回答以外,从此她就很少
说话 


她在心里总是默默地念着:忍 


忍,快点考上大学离开家吧 


忍,快点长大吧 

枕书真是个平凡的女孩子,她还有很多个春夏
\newpage
秋冬要走过,但要留神,不然日子都溜走了,自己都知道

\end{document}
