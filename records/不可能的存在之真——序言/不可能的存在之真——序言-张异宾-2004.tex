\documentclass{article}
\usepackage[utf8]{inputenc}
\usepackage{ctex}

\title{不可能的存在之真——序言\footnote{Click to View:\url{https://web.archive.org/web/20221027010240/https://rentry.co/xw5s3}}}
\author{张异宾}
\date{2004-01-09}

% \setCJKmainfont[BoldFont = Noto Sans CJK SC]{Noto Serif CJK SC}
% \setCJKsansfont{Noto Sans CJK SC}
% \setCJKfamilyfont{zhsong}{Noto Serif CJK SC}
% \setCJKfamilyfont{zhhei}{Noto Sans CJK SC}
% \setlength\parindent{0pt}

\begin{document}
\CJKfamily{zhkai}

\maketitle


\Large

序 《拉康选集》的中译本已经出版近三年,而中国哲学界对拉康这样一位在后现代语境中极为重要的思想大师却近乎哑口无言。少关于拉康及其思想,我们几乎没有几篇正式一些的系统性、原创性的哲学学术论文。雪上加霜的是,在一些文学式的解释性复述文本中,拉康却恰以颠倒的误认形式被严重遮蔽,从而使我们与这位在海德格尔之后进一步揭示了个人的存在之残破性与深层痛楚的精神巨人总是擦肩而过。可深具反讽意味的是,今天刚刚进入物化世俗王国的中国人正焦虑于拉康所揭露的这种灵与肉、光与影、有与无的挣扎和痛楚之中。拉康语境中那个个人赝主体和大写的魔鬼他者正大行其道,令这块黄土
地上的世人病入膏育而无诊。 

究其根源,拉康在中国学界之所以备受冷落主
\newpage
要还是因其哲学思想的艰深难解。比起他的先生弗洛伊德,拉康绝对可以称得上是一块无法吞噬的精神顽石。就我自己的阅读体验来看,读拉康是一种对正常理性的折磨(列宁曾经愤愤地说,读黑格尔是引发头痛的好办法。那么,读拉康则让人头大而脑裂)。清高的拉康自己就公开表示对精神“暗礁的喜爱”,他的真实用意就是想绊倒这个世界上一切自以为是的人。一路读下来,拉康的文字实在好比一堆极端复杂多变的功能性话语症候群,绝难正常阅读和直解。对翻开拉康文本的常人来说,那里面似乎只有无人知晓的神喻。回在一定的意义上,他似乎有意不给读者确定的路(入口),而只留下“无路可走”的出口。他直接说:“我倾向于艰涩。”在流行的意义上,他的东西将不是常规式的“写作”。所以,人们读拉康会“像描绘生硬的无意识一样的独特的文体强迫读者如同解梦般地苦战”。甚至也有人说,拉康理论本身就是不能被理解的,即便理解了必定也只能是误解。在拉康门前团团转却不得其门而入之后,终于有人恶意诽谤道:“拉康是不可理解的,而他的追随者们只不过是一些百依百顺的傀儡。”当然,也有打圆场的:“拉康建构的观念游移不定,因此,面对一个特定的术
\newpage
语,最好是问“它有什么用处?”或者‘它的思考路径是什么?’而不是直接问“它究竟有什么涵意?等等。虽然这大都是一些阅读失败后心有不甘的丧气话,但恐怕也的确是相当多的人拿起拉康的书,却又最终绝弃的主要原因。所以,面对拉康,我必须自省的只有四个字:怒犯天条。我解读了,并且用中文重新言说了不可能说清的拉康哲学。加上在中国大陆学术界固有的硬性专业边界,这一次,我完全可能会犯下比解读阿多诺时更大的错。这也是我将本书的副标题命名为“拉康哲学映像”的真实缘由(其英文译名中我选用了影像易碎的“mirror”一词)。不过,已经在思辨的十字架上倍受折磨的我绝不想让大家跟着急,我还是想斗胆承诺:力争替你们打开拉康哲学那根本没有开口的密封瓶。记得巴塔耶说过,倘若你真想理解一种思想,便要在那些概念中非知性地“深深地活过”。可依我现在的痛苦经验,要在拉康的概念中“活一次”都是极不容易的。不过,这一回我的解读策略之一是讲故事。我先说一些发生在自己身边非常真实的事情、然后,再来慢慢地一点一点地靠近这种梦幻般吃语般的学术魔殿。真的不知道,我们

\newpage
能不能在拉康的那些理论恐怖主义概念中去活。 

第一个故事:两年前的一天晚上,我刚吃过晚饭准备看书,突然发现有个男子坐在我家客厅的沙发上抽烟(他没有预约,在保姆打开门倒垃圾时只说了一句:“张老师在家吗?”就径直闯入,并且一坐下就点燃了香烟)。我走上前发现是自己以前教过的学生王二。“怎么有空?”我硬压住心里的不快问他。“嗅,我发达了,来看一看我一直敬重的老师”王二眼里闪着一种常人很少有的光芒。不得不说,我其实挺讨厌那种从底层不知通过什么途径突然暴富起来的人,但我还在敷衍他:“哦,做什么?”他压低声音伏在我耳边说道:“做基金,与美国很大的一家公司合作,赚了二十多个亿。我有钱了。”这次足我暗暗地吃了一惊。可也是在此时,我看到他手指间夹着的是很便宜的石林牌香烟。“那太好了!”面对发展得不错的学生,我通常会这么说。接下来,他把抽完的烟头掐灭在香烟盒内盖上,并塞进烟盒中。我一直没有拿烟灰缸给他,这其实是一个不让他抽烟的暗示,可他并不明白这个常人不难发觉的暗示(象征)。“我已经在汤山少花五千万买了一块地,想请老师去当校长。”他又点燃了一支烟“这下,我终于可以按自
\newpage
己的想法来办一所大学了,这是我一生中最想做的事情!”这个时候,我不由得开始仔细打量他,他今年应该是三十岁上下,还在攻读在职硕士研究生。回想起来,当年在课堂上他总摆出一副拥有绝对真理的架势,面对他无时不在的责备目光,我倒时常感到自己是歪嘴和尚的那种心慌。有一两次,在课堂上听讲的他还突然高声说出一两句反驳我的话。接着,我又看清了一个细节,就是这个声称自己发了大财的人却背着一个如今连城里中学生都不用的黑色牛筋包。我悄悄在想,他是不是脑子出了问题?他并未停止兴奋的述说:“我差点死过一次,可现在真的很厉害了,我买下了阅江楼回下的一幢二层楼,窗户都是防弹玻璃做的。”不等我搭上腔,他已经又伏到我耳旁轻声说“想不想动一动响?我大伯在中组部有人。”至此我已经十分清楚,坐在我面前的这个抽着石林香烟的人,已经不是过去课堂里的那个学生了,而是一个趁人精神分裂之际现身序的无意识的个人欲望。我那个可怜的学生平素在内心里压抑的欲望像附身的魔鬼一般无意识地在说他。第二天,我证实了他刚从医院出来
。 

\newpage

第二个故事:主人公是我一位朋友的女儿梅子。地从小聪慧漂亮,任教于北方的一所著名大学。在他人眼里,她是一个天才幸运儿,因为她上中学的时候就在国外一个国际史学文献竞赛中获大奖,现在她已经是一位在大学里成就显著的青年学者留校任教的她发表在报刊上的论文,常常被误认为是学界资深前辈的大作,人们在与二十多岁的她讨论问题时,,总是恭敬地尊称其先生。在同辈人中间,她真的总是很像个学者,说话举止无不处处透着浓浓的书卷气和高高在上的傲气。可是有一天,她突然找到我这位朋友,一件意想不到的事情居然发生了:她清楚地告诉父亲,她要退学并且辞去在大学的工作。“为什么?!”父亲(常人思维中的他)几近是生气地问道。二十多岁的她字正腔团地说“我不再想做那个学者了。我只想做一个平常的女孩子。”她说,她现在只想过一个正常人的生活,而不再愿意继续做那个过去二十年来他们要地成为的成功者和学者。我这位朋友(他们中的一个)当时就急了,因为所有人都不想失去一个已经成名并且十分有前途的青年学者。“能不能既做一个平常的女孩子,又做一个学者呢?”那个伤心地父亲(他们的代表之一)小心翼翼地问道。“不!”
\newpage
她说了自己的理由她突然发现,她始终不曾为自己活过,开始是为父母,然后是为中学老师和大学老师,其中也包含着同学和周遭一切认识她的人面孔上的肯定目光。在这些无时不在的形形色色的目光中,她是一个成功者,她将成为著名的学者,她每天除了学习还是学习,看书和写文草成了她每天做的唯一事情。在他们的眼里,她不应该是一个平常的姑娘、她不应该是她。大写的她是人们无形中期望她成为的人,是其他人用目光交织铸成并且也被她自己认同了的大她者。现在小写的她醒悟了,她想做回她自己,拒绝那个大写的她。最后,在人们(不是她的他者)惋惜和反对的目光中,她走了自己所选的道路。几年前,她
结了婚,做了一个她想做的非常幸福的平常的人。 

之所以叙说这两个故事,是想让读者预先融入一种新的思考语境,这也是长期以来国内传统哲学研究不常驻足的一种情境,,即我们平素鲜少注意的现当代心理学研究中重要的精神分析学园地。首先当然想让读者熟悉一下这一思潮的开山鼻祖弗洛伊德式的基本话语,由此,我们才可能真正进入当代西方精神分析学中更激进的批判话语:拉康哲学和将拉康与马
\newpage

克思嫁接起来的后马克思思潮中的齐泽克。 

以下,不妨先来看看在这两个其实随时都会发生在我们身边的情景中,用弗洛伊德和拉康的眼睛可
能会发现的一些我们从前看不到的东西。 

在第一个故事里,对那个不经预约而突然闯进我家的王二,我不能责备、不能生气,更不能将他赶出门去,原因很简单:他已经不是一个正常的人了。他发疯了。他并不知道自己做的和说的是什么。也就是说,那天晚上坐在我家客厅里的并不是王二的自我意识主体,而只是通常被他自己深深压抑在很深的黑暗牢狱中的无意识欲望。根据弗洛伊德的精神分析学,以往被人视为主体本质的意识(理性)背后,还存有一种更为基始的东西,即由本能冲动和真实欲求构成并被压抑在意识阈限之下的无意识。做学生的时候,王二的个性很强,他不大看得起一般的同学和老师,可是他自己在现实中的生存能力又十分有限。显然,王二的真实本我中有种种欲望,譬如像索罗斯一样做基金交易,轻轻松松拿个几十亿的进账(他不应该这样贫穷);他不喜欢现在的大学体制,总想自己办
\newpage
一所大学(他的才华受到压制);他也希望在“上面有人”能在政治上呼风唤雨(他恨那些“没有本事却成功的人”)等等。可是在平时,迫于现实环境的压力,王二的自我不得不将这些想法压抑下去,日日在社会公众的层面中出现的王二其实只是一个戴着人格面具的超我。在王二没有发疯的时候,这些欲望可能只会在平时的口误和玩笑中有意无意地流露出来,较多地也可能是在梦中以怪异虚幻的方式实现。记得我曾经咨询过一位研究精神病学的朋友,他告诉我,使人发疯的最重要的病因即是个人的理想和欲望与现实环境的激烈对抗。当主体不再能控制自己的欲望,即无法成功地将它们压抑到意识控a)是容格后来提出的一个重要观点。面具即一个人呈不可能的存在之真一一拉康哲学映像制的冰河之下,以至欲望无意识奔涌而出并试图与坐在王位上的主体自我意识一争高下时,主体便精神分裂了。病症的程度可能会由轻微的脑子不做主到分裂式的时有吃语,再到狂怒的歇斯底里症。王二心比天高,可是他在现实中没有一丁点成功,最终以他的婚姻离异并丢了工作后的精神分裂杀死了主体。心怀强力意志的他唯一的解脱是发疯。不过,刚从医院里出来的王二病情已经得到了一定的控
\newpage
制,因为当时他还知道把抽完的烟屁股熄灭在烟盒中,而不是直接按在桌上。这就是弗洛伊德眼中所看到
的心理症候之幕帘背后的东西。 

好。我们再来看第二个故事。这一次,我们请来的评点人是自诩为弗洛伊德遗产继承人的拉康。借用拉康的眼睛来看,女孩梅子在下决心回到她自己之前的生活过程中,从来没有真正拥有过小写的地自己(即弗洛伊德所说的本我,拉康基至根本不承认本我的初始存在)。开始是父母,后来是从幼儿园到学校的老师同学,还有能够接近她、影响她的一切人,所有人都用“你行”“你是最棒的”、“你天生就是一个学习的料”、“你怎么能像一般的女孩子那样平庸呢”的话语每时每刻建构着一种并不是地的大写的她。那些并非恶意的其他(other)人的言行场意象式建构出来的镜像就是他者。这些亲近的他人用他们每日的目光、表情和言行围绕和建构着梅子对自己的心理和观念认同。起初,地可能在镜子里看到过这个作为对象整体的她(依拉康的说法,在这个镜像中,人对自我的认同已经是异化,镜像一自我认同的伪心像是主体最早的篡位者,也是小写的他者a的阴险
\newpage
意象,小他者先行性的篡位而取代了她,而她自己却在一开始就死亡了);后来,她主要存活在他人们的有脸和无脸的反指式形象自居中。她是小写他者的想象式的镜像存在(这其实是弗洛伊德那个开始向现实低头的自我的反向蜕化物)。从梅子能够完整地接受文化语言教化开始,她的存活就发生了一个重要转换,即从想象域转换到象征域。现在是语言符号(能指链)那种无脸的大写他者(上帝、观念、主义、事业、成功)建构大写的她了(超我=主体S,可是拉康一定要说这个主体其实是被本体性的删除斜线划着的$)。然而今天,梅子要做回她本真的自己。她不想继续作为他人欲求中那个光亮的女学者而存在了,她试图穿过欲望的幻像,经营一个普通女孩子的生活。
她要打倒形形色色的他者。 

我不知道梅子有没有读过拉康,可她却成功地通过拉康式的解放从学者式的他者阴影中逃脱了。但是进一步的问题是,她又如何知道,“普通的女孩子”是不是另一种更为阴险的他者之幻像呢?因为,在拉康的真实域中,不可能性才是存在之真。人就是腹

\newpage
中空空的症候或症象人。 

这两个故事显然有不同的语境。我们先以弗洛伊德和拉康两种话语分别解读了这两个故事,由此已经有了一种各自的差异性。可是,如果我们进而让拉康再去分析第一个故事呢?这就会呈现真正的思想异质性。在拉康的眼里,那天坐在我家客厅中的王二,并不是弗洛伊德所说的无意识本真主体。拉康的质问是,王二每天被压抑下去的欲望(做基金发大财、呼风唤雨等)真是他自己想要的东西吗?他在自我崩溃后直抒出来的无意识真是他的本我吗?拉康的回答是“不”!拉康会认为,王二的欲望其实是今天作为象征话语体系那个大不可能的存在之真一一拉康哲学映像写的他者欲望的欲望,他只是想要(欲望着)今天成天在人们耳边轰鸣的市场意识形态制造出来各种幻像,于是,王二的灵魂深处涌动的无意识就不会是他本己的本能冲动,而是种种大他者无形强制下的奴性物。拉康因此断言“无意识是大写他者的话语。”更可悲的是,拉康还要指认疯掉的王二本来就是一个被种种象征性身份和反讽性关系建构的空无,没有发疯之前也是另一种更深的本体论意义上的精神分裂。王二和所有人一样,并没有本真的本我(弗洛伊德的本
\newpage
能被拉康作为生物学的动物性排除在人之存在以外了),他降生到这个世界上以后对“我”的最早体认便是镜射幻像(小他者Ⅰ),然后,以他身边最亲近关系的他人们(爸爸、妈妈、爷爷、奶奶、姥爷、姥姥等亲人一直到一起玩耍的小伙伴、幼儿园老师等等)的反射性面容之镜小他者!,强迫自己成为一个他人眼中“应该”成为的“我”(伪自我)。长大以后,不在场的语言象征逐渐替代了身边当下的面容,教化式的大他者(能指链)成了新的成年“我”(=主体)规划和一砖一瓦建构主体生存情境的真正原动。处于社会语言存在中的“我”,只可能追逐人们都想要的东西,无从挣脱和免俗。而这些东西,都只是形形色色大他者的他性欲望。我们永远只是在无意识地欲望着他者的欲望可我们却自以为是自己的本真欲望。拉康认为,主体不过是一具被斜线划着的空心人。可怜而可悲的人,从来不是他自己,也永远不可能是他自己。这种不可能真实存在的此在,就是人的本体论存在意义上最重要的大写的真实。王二疯了,可连他的疯话都是大他者强迫下吐露的“真言”。这就是拉
康令我们深深恐惧的地方。 

\newpage

我敢说,读者听了这两个故事及其分析,可能更加疑惑了,因为旧问题不见得清楚了,却多半会产生一大堆新的疑相拉语听下去的欲望序了说明这些问题。故事中全部的谜底都在本书的叙述和分析之中。真想知道结论性的东西,只有耐下性子读完此书。所以,下面让我们先转而介绍本书写作的一些基本背景
和内容概要。 

首先,第一个背景问题就是我们已经提出来的拉康与弗洛伊德的关系。拉康常常说自己是在“回归弗洛伊德”,可是我们却发现了拉康在这种“回归”中对弗洛伊德的否定式的超越。这种超越其实凭借了太多的复杂历史资源,如超现实主义、经柯热夫一伊波利特中介过的黑格尔、索绪尔一雅各布森和列维一斯特劳斯的语言学结构主义,还有胡塞尔一海德格尔的现象学传统、萨特、列维纳斯等人的他人一他者理论和巴塔耶的圣性事物观等等。况且,这一切又都同样被颠覆式地挪用。可能出现的情况是,看起来拉康也使用了一些弗洛伊德或其他人的关键词,然而这些词语的涵义却已经居有了新的甚至是相反的语境。这是我们在解读拉康时容易碰到的第一件难事。拉康的
\newpage
自我、无意识、症候、能指、真实等概念都是这样的怪异情境。因此,倘若你在拉康这里读到熟知的传统精神分析学或其他概念,千万当心,说不定倒过来理
解才是拉康的真意。 

下一个需要交待的地方,是拉康学术主题和讨论域的多变性。我指的不是拉康思想在总体逻辑上的非同一性,而是说拉康绝非那种建构一种原创学术平台之后几十年凝固不变的学人。从1936 年提出镜像阶段开始,到1953年在“回到弗洛伊德”的口号下凸显语言结构主义的能指话语,依拉康门徒齐泽克的定位,这算是拉康的“古典时期”。20世纪60年代后期是拉康下一个新的理论时期的开始,即探讨诸如对象a、症象人、没有大他者的大他者之类的不可能直接触及的形上规计大中g不可能的存在之真一一拉康哲学映像选择了拉康“古典时期”中最重要的学术思想,即以1966年出版的《拉康文集》Ecrits)为核心的解读文本群,只在最后概要
地讨论了拉康晚年的一些东西。 

第三,也因之于拉康哲学的艰涩,我在这本书
\newpage
有关《拉康文集》关键文本研究的主要行文中虽然也采用了自己独有的文本解读和写作方式,可是我也刻意在支援背承的讨论和引论中拼贴了一些具象的、戏剧性的平和外观。为的是让大多数头痛欲裂的读者能够寻着些感性的通道,这就像我们平常看恐怖片时也
需要通过暂时的逃离稍稍放松一下。 

本书的第一部分(剧场指南)对拉康生平和理论逻辑进行了一个概括性的介绍和讨论。我尽可能在其中街略清晰地线性再现拉康哲学的思想轨迹,同时尝试着以非文本学的方式使拉康哲学的问题式与当代最重要的形上意义场链接起来,并使拉康的基本哲学理论逻辑的轮廓得以初步浮现。第二部分(序幕第一章)主要介绍讨论了作为拉康基本逻辑前提的弗洛伊德。当然,我只是非专业化地描述和简单评论了弗洛伊德精神分析学某些重要理论范畴和最一般的理论原则。这一部分中,我已经开始穿插拉康和齐泽克的评点,也算是一个大的理论逻辑过渡。第三部分(第一幕中的两章)则讨论了拉康早期的镜像理论。拉原正式登场之前,我首先在第二幸中介绍了影响青年拉康哲学始发的特定学术语境,主要是超现实主义思潮(
\newpage
特别是达利)和经柯热夫重新诠释的新黑格尔主义某些倾向性的观念。前者的作用在于对现实存在的批判性解构,尤其是达利以疯狂性艺术存在对现实之茧的真实挣脱;而后者的意义则在于柯热夫、伊波利特对黑格尔以对象性关系认同为核心的主奴辩证法和欲望辨证法的过度诠释。第三章是对拉康镜像理论的集中研究,在那里,我第一次深入分析了9分算是我的原创。第四部分(第二幕中的三章)是对拉康象征性能指学说的讨论。第四章主要介绍了作为拉康能指学说理论重要背景的语言学结构主义的基本学术资源,尤其是索绪尔和列维一斯特劳斯的符号学理论和象征性观念。相对而言,我认为后者对拉康的影响更直接和更深刻一些。第五章则聚焦于拉康的证伪性的语言观,其中尤以“语言即对存在的杀戮”为最,并由此导引出拉康极端的伪个人主体论。第六章是拉康哲学中作为逻辑核心的能指理论。在这里,拉康的能指成为篡夺一切事物和人存在之位的隐性暴君,生存现象中招摇在世的个人主体不过是能指链互指轮回的空心木乃伊。第五部分(第三幕中的三章)是拉康他者理论和批判性欲望学说。第七草主要分析了他者理论的历史性发生逻辑,包括从神学语境中的圣性他者和魔鬼
\newpage
他者,到存在主义的他人逻辑,以及列维纳斯的他者之面孔说。由此,当代西方学术语境中炙手可热的他者理论之复杂历史语境终于得以彰显。第八章则集中探讨了拉康的大写他者理论。拉康大写他者理论的前提是他的存在疯狂说,大他者正是主体际关系存在中的麾化力量,无意识看似我们最本己的东西,可其实却是大他者的隐性绳索。第九章是拉康独特的欲望论。从界划具象的需要、言说的要求出发,欲望是以本体性的空无为对象的,更重要的是,欲望总以大写他者的欲望为欲望对象,个人的欲望永远只能在幻像中得到虚假满足。最后一部分(终曲一章)的研究对象是浓缩了的拉康晚年迷入的存在真实域。关于那一时段,我着重分析了不可能性的存在之真的实质,以及晚年拉康所热衷的大写的物、对象a、症候之类的古
怪概念的基本哲学含义。 

我真的以为,拉康哲学对今天的中国学界来说实在太重不可能的存在之真一拉康哲学映像以为是的人在我们这块土地上实在太多,他们真应该在拉康精
神分析的躺椅上接受一场心理一哲学治疗。 

\newpage

最后,我举一个例子来结束本篇的言说。1958年,一位自认为很了不起的学者克洛德·迪梅尔走进了拉康的研讨会,当他听懂拉康所讲的东西时感觉如下:“真是恐怖,一个超凡脱俗的君子突然像鲤鱼一样一言不发了,那个充满诱惑的人把你洗劫一空,使你一文不名。这不再是理论上的,我被宰得鲜血淋漓。”为什么?他听到了什么?是什么东西令他感到恐怖?你如果真想知道,就请坚强地不言放弃地读完本书。然后,你还得说:“我一定要真实地活下去。”哪怕是“自指着面具而前行”(巴特语)!这算
是我们的事先约定。 


在这里真心祝你好运! 

张一兵2004年1月9日于香港国际机场

\end{document}
