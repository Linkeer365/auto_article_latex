\documentclass{article}
\usepackage[utf8]{inputenc}
\usepackage{ctex}

\title{忆张友伦老先生\footnote{Click to View:\url{https://web.archive.org/web/20220629040729/https://www.douban.com/note/268958030/?_i=6475627SdBskc1}}}
\author{李剑鸣}
\date{2013-03-29}

% \setCJKmainfont[BoldFont = Noto Sans CJK SC]{Noto Serif CJK SC}
% \setCJKsansfont{Noto Sans CJK SC}
% \setCJKfamilyfont{zhsong}{Noto Serif CJK SC}
% \setCJKfamilyfont{zhhei}{Noto Sans CJK SC}
% \setlength\parindent{0pt}

\begin{document}
\CJKfamily{zhkai}

\maketitle


\Large

将近三十年前,我大学毕业后分到湘中北一所师专教书。学校座落在远离市区的山里,校门口有一个小小的商店,除了卖一些生活用品,还代销图书。一天,我到商店里买东西,无意中看到一本书,标题是《美国农业革命》,不觉喜出望外,便毫不犹豫地买下了。回到简陋的单身宿舍,我几乎是一口气读完了这本书。作者的见解和文笔,都给了我极深、极好的印象,心想,如果要学美国史,就要做这样的学
问,写这样的书。 

当时,我正好对美国史产生了兴趣,想考研究生。于是,我就给这本书的作者张友伦先生写信,也得到了他的简短的回信,表示欢迎我报考。不久,有个同事从外地参加美国史年会回来,说在会上见到了张先生,还给我描绘了他的相貌和风度。于是,我更
\newpage
加坚定了自己的想法:一定要报考南开的研究生,投到张先生门下学习美国史。我那时何曾想到,这个小小的决心,后来竟彻底改变了我的人生道路,给了我
一种完全不同的工作和生活。 

不久,我调到湘潭大学工作,教学之余,全力复习备考。1986年,我幸运地考取了南开大学的研究生。不过,由于校方只允许报考在职委培研究生,我尽管考分不错,但也只能跟学校签了一个委培协议,领着工资去上学,三年后回湘大工作。当时也没有太多的想法,只要能到南开学美国史,就“于愿足
矣”。 

到了南开,在历史研究所的开学仪式上,我第一次见到了张先生。那时,他刚五十五岁,头发雪白,面色红润,待人亲切平易。我请他做我的导师,他当即满口答应。不久,张先生担任了历史研究所所长,工作繁忙,一直没有给研究生开课。不过,他请了好几位美国学者给我们授课,有时还亲自主持美国学者的讲座。我从未听过张先生的课,固然是一大遗憾,但私下从他那里得到的教益,却远非听课所能比拟
\newpage

。 

我那时年轻气盛,不知学问的深浅,一味地乱写文章,写了就请张先生批改。他看过后,只在稿纸边上加一些批语,当面并不给我多说。他看过的稿子,有的在刊物登出来了,有的收入了他主编的论文集。其中有一篇投到《历史研究》的文章,修改过几次,二审仍未通过。责任编辑让我找张先生想想办法。张先生看过稿子和编辑部的意见,觉得再改也没有什
么意义,就建议我放到《南开史学》发表了。 

通过这些文章,张先生大约看出我是个喜欢读书的人,希望我在学业上有更好的发展,便动员我考杨生茂教授的博士生,而且是提前一年考。这对我是一件至关重要的事情。我念的是在职委培,要报考博士生,必须征得原单位的同意,其难度可能很大。杨先生在跟我谈话时明确表示,如果跟他念博士,只能做史学史或外交史,而我却对政治史更有兴趣,这也让我颇费踌躇。再则,与妻儿长久分离,早有难以忍受之感,一心只想早点毕业回家。寒假里,我与家人商量这事,他(她)们都表示尊重我的意愿。开学后
\newpage
,我一到南开就去看望张先生。他见面就问:“我最关心的是你能不能考博士?”我摇摇头,做了否定的表示。我没敢看他的脸色,不知他是不是有些失望。
 

转眼到了毕业之际。一天,张先生把我叫到他家里,跟我谈了留校的事。他说,所里决定留我,并已请示学校,从校长基金拿出一笔钱,退还我的委培费和在学期间领的工资。张先生让我先跟家里商量,并问湘大是否愿意放人。我一时十分犯难,不知如何作答。在此之前,我已为回湘大工作做了精心安排。那时,湘大历史系因学生分配困难,已经停招一年,今后这个系还能否存在下去,大家心里都没有把握,我也很是彷徨。于是,我便找熟人疏通,准备毕业后转到湘大法律系,从事英美法的教学。这时,张先生突然提出要我留在南开,跟我一直以来的想法颇不一样。我给家里和朋友写信商量,他(她)们都支持我留在天津。经过一番周折,湘大终于同意放人,南开也按约偿付了我的代培费和工资,共计两万两千元。据说,南开以这种方式留一个硕士生,并没有先例,后来也没有听说有第二例。张先生做成这件事,肯定
\newpage
费了很大的心力,听说所里对他的做法也有些议论。我当时也隐约感到一些压力,暗暗下定决心:一定要
努力工作,做出成绩,决不能让张先生为难。 

刚办完留校手续,张先生就开始为我张罗家属的调动,并托人为我寻找住房。同时,他更是为我创造一切条件,提供各种机会,以利于我在学术上尽快成长。他带着我参加一些他主持的项目,如编写《美国历史词典》,写作《美国历史上的社会运动和政府改革》。此外,我自己也在动手写《大转折的年代》一书。当时,学术著作出版难已是一个普遍问题,我作为一个无名小卒,要独立出一本专业书,不免有“蜀道”之叹。其实,我的困难早在张先生的考虑之中。正巧有一笔出版基金可供申请,但我的书稿尚未成形,赶不上申请的截止日期。张先生手头有一部书稿,出版社已答应出版。他便想了一个变通办法:拿这部书稿去申请一笔资助,并与出版社谈妥,申请来的经费用于出我的书。为了办好这件事,张先生带着我去出版社,找负责人和编辑面谈,并把我列为他的书稿的第二主编,以便于申请。事情顺利地办成了,我的第一本书很快就出版了。按照当时的风气,一般是
\newpage
学生的作品由导师挂名;因此,当我作为第二主编的那本书出版后,常有人问我,是不是我做的工作,张先生只是挂名。我说,恰恰相反,所有工作都是张先
生做的,而我只是挂名。 

对我的职称问题,张先生一直十分关心。我想,他可能是出于两方面的考虑。一方面,美国史研究室正处于青黄不接的状况,需要补充有高级职称的人员,以建设一个相对合理的学术梯队;另一方面,我当时生活极为困难,一家人挤在一间十几平米的房子里,如果评上了副教授,就可以问学校要住房,能有较为舒适的生活条件。1990年冬天,南开大学首次实行破格晋升制度。张先生和师资处交涉,并动员我申报破格副教授。听到这个消息,我一时难以置信。我并不理解张先生的苦心,觉得自己的学术水平离副教授的标准相去甚远,根本不抱评上的奢望,述职汇报也做得敷衍潦草。果然是名落孙山。不久,张先生赴美国做研究,有一整年不在学校。我自觉学业上略有长进,再次申报了破格晋升副教授。原以为这次不会有什么问题,可是中途却出了一个很大的意外。有一封匿名信对我提出指控,称我在研究生期间发表
\newpage
的一篇习作有思想倾向的问题。在当时那种特殊的政治形势下,这样的指控是极具“杀伤力”的。校方对这封匿名信高度重视,并在校职称评聘会上公布,结果我没有得到足够的票数。张先生回国后,了解到这一情况,对学校有关方面做了解释和说明,力图澄清我的问题。到了再次评职称时,他又做了细致周到的工作,避免有人继续揪我的“小辫子”。最终是有惊无险,好歹评上了副教授,张先生心里的石头也算落
了地。 

接下来,张先生便开始操心我的出国问题。我依稀记得,当时有关部门做了规定,年轻教师不满三十五岁不能出国。在我快达到这个年龄要求时,张先生就着手为我出国做铺垫工作。他跟学校师资处和外事处谈了我的情况,又利用参加富布赖特学者选拔面试的机会,向美国使馆有关官员大力推荐我。这些事我原本毫不知情,等我到使馆办手续时,负责此事的美方文化官员对我说:“我们可是早就知道你,张教授多次推荐过你。”在我准备申报富布赖特项目的材料,张先生提供了不少参考资料,还为我写了推荐信。那年,张先生仍是面试小组的成员。在我面试时,
\newpage
他一句话也没有说。但是,我见他坐在那里,心里便
格外踏实,面试也顺利过关。 

在我出国期间,学校向张先生提出了退休的问题。他表示,自己已过了退休年龄,退是理所当然的,但美国史的师资不能削弱,希望学校考虑李剑鸣的博导资格。在他的要求下,学校同意我参加当年增列博导的评审。我当时身在国外,对这件事全然不知。张先生悄悄代我填写了繁琐的表格,并且亲自到评审会上替我述职。据在场的人事后说,他们看到一个白发苍苍的老教授,认认真真地替学生做这样的事,不免十分感动。这可能给我争得了一些“同情分”,增
列博导的事便顺利通过了。 

这期间,还发生了一件事。教育部启动了首届“跨世纪人才”评审工作,历史研究所推荐我申报。可是,我人在国外,不能填表和提供材料。张先生再次代我完成了这一工作。可惜,这次让张先生的心血付诸东流,我没有评上。2000年我再次参加“跨世纪人才”评审。这次是我自己填表和准备材料。表格之复杂琐细,材料之具体繁多,几乎让我半路放弃
\newpage
。我不由想起,几年前张先生替我填表和准备材料,不知花费了多少精力,耽误了多少时间,一时真是感
慨系之,叹息良久。 

张先生为我做了这么多的事,可是我当时却一无所知。我在美国期间,张先生给我写过信,似乎只字未提为我申报博导和“跨世纪人才”的事。我回国以后,才从同事那里得悉实情,内心的感动自是无法言表的。2001年,南开大学美国历史与文化研究中心给张先生庆祝七十大寿。时任历史学院院长的李治安教授表示,他一定要出席祝寿会,向张先生表达祝贺和敬意。他对我说:“我一直记得那年评博导时张先生为你汇报的情景。一个满头白发的老学者,站在那里替自己学生述职,这一幕实在感人!就冲这一点,我也要当面向张先生致敬。他不光是为你个人,这是从学科发展着想啊!”听了他的话,我深受触动
,内心似有千言万语,只是不知从何说起。 

另外,还有一件很有意思的事。我在张先生身边学习和工作达二十年之久,可是从来没有听到他表扬我。我想,张先生为人严谨克制,对学生要求严格
\newpage
,不轻易称赞人,自是情理之中的事。可是,张先生的熟人见了我就说:“你还不知道吧,张先生经常夸你,总是说你如何出色呢!”有个朋友还对我说:“你还在读研究生时,张先生就老是在所里夸你,说你如何优秀,说得我们都有些忌妒,心想,这李剑鸣是个什么人物,连张先生这样轻易不夸奖人的人,也赞
不绝口!”  

这些年来,张先生一直教我治学和做人,为我创造条件,铺平道路,做了那么多琐碎繁难的事,使我在学业上和生活上一帆风顺。可是,我却几乎没有为他做过什么。我没有给他抄过一页稿子,没有给他换过一回煤气,没有给他送过一件像样的礼物,甚至
也很少说过感激的话。 

现在,我已步入中年,不免添了一些暮气,总喜欢回忆过去。我想,如果在求学路上没有遇到张先生,如果没有张先生的关怀和提携,我会成一个什么样的人呢?严肃的历史研究不主张做轻率的假设,所谓“反事实推理”也有很多严格的条件限制。可是,我还是禁不住想,如果不是张先生,尽管我也可能会
\newpage
成为一个学术工作者,也会在某一所高校教书,但我绝对不可能有现在这样好的状态,也不会从专业工作
中获得这样多的乐趣。 

旧时文人称颂自己老师,常有“恩同再造”的说法。我过去曾想,这当中多少包含着夸张和客套的成分。可是,每当念及张先生多年来对我的教诲和扶持,我觉得除了这四个字,似乎没有更好的文辞来表
达我对他的感戴。 

在这天气和暖、万物争荣的五月,借着庆贺张先生八十华诞的机会,我终于能把在心里积累多年的意,痛痛快快地向他表达出来。

\end{document}
