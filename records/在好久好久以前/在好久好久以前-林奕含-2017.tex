\documentclass{article}
\usepackage[utf8]{inputenc}
\usepackage{ctex}

\title{在好久好久以前}
\author{林奕含}
\date{2017-04-27}

% \setCJKmainfont[BoldFont = Noto Sans CJK SC]{Noto Serif CJK SC}
% \setCJKsansfont{Noto Sans CJK SC}
% \setCJKfamilyfont{zhsong}{Noto Serif CJK SC}
% \setCJKfamilyfont{zhhei}{Noto Sans CJK SC}
% \setlength\parindent{0pt}

\begin{document}
\CJKfamily{zhkai}

\maketitle


\Large

从前他和她是一中女中的王子公主。那是明星高中还明星而不是艺人的年代。没有太多人追求她,也不是太少,但听见有他便纷纷退却了,像开灯的房间看不见烛火。她甚喜欢他性格中坚持到别扭的部分,比如他学古典音乐,竟根本不听流行音乐,去KTV也只会唱一首《朋友》。

第一次接吻她早已从大学休学,他在美国念大学,圣诞期间返乡。她才从精神病房出来,才第一次吞安眠药,第一次上吊。远远地看见他只穿一件薄长袖,冷气团把白上衣吹馁在他的腰身上,衣衫的皱纹亦有一种笑意。那笑意与从前被装在过于宽大、僵硬的泥土色制服中的笑亦没有不同。榕树下他很自然吻了她。大冷天的,竟然还有鸟在啼,巢巢的叶子中找不到那鸟,仿佛是树木本身在啼叫。她开始哭,说不行,说他什么都不知道,说她已经不天真。说了你的事情。他问到那一步了。她想都没想就说接吻。他又吻了她说他没关系的。可是她看见他的眼睛里有个

\newpage 

小孩中蛊似地手舞足蹈在扒撕一棵千年白千层的树皮。

隔天他陪她上台北会诊,他们接了一个十站地铁之久的吻。有一束光,像一束舞台灯光像一支倒挂的紫色郁金香包裹住他们。

后来她不舍得分开,去美国住了数月。他念理工,拿了作品要参加国际科展,科展办在荷兰。那是她第一次去荷兰。在荷兰一星期,做尽游客该做的事。印象最深的是安妮·弗兰克之家。安妮一家躲藏的那书柜不可思议地矮小,不能想象要阻挡庞然四十二臂的历史仇恨。折腰踏进去,里面却意外敞亮。马上想到《安妮日记》里散了一地的豆子和淹在豆里的彼得。也许豆子海里裤裆里的小鸡,和安妮一路摸索过去的笑声。一路上,书里的句子在脑子里走马灯。她才发现他的手一直拦在她头上,怕她一头撞上低梁,或者怕梁一头撞上她。出来之后有个安妮的小青铜雕像,他摆正相机说去拍个照吧。她说不要,说她不要跟安妮·弗兰克合照。他敛起笑容说他懂了,向她道歉。她说不是他的错,不是他或她的错,手指深深穿进他的手指里面。

他在美东的大学城读书。他去上课,她就坐在咖啡厅里看翻译书写文章。她没有学历,他不像其他人介意,只一直鼓励她写。他用的苹果电脑,她不

\newpage 

善用苹果的中文输入法,他竟甘心听她口述,他誊打。又不是什么了不起的文章。她英文在台湾还好,丢到美国就显得破破烂烂。他教她念“中杯”,“Grande”,在露天咖啡座夸饰着嘴型。他的嘴唇粉红红,尾音像一个微笑,她无限地望进去,想要溺死在里面。有时候在大卖场,美国的大卖场出了结账区总有租DVD的自动机器,那是她第一回看阿莫多瓦《破碎拥抱》,看之前神经兮兮地问他,没有中文字幕,她一定看不懂。看完之后,女主角死了,男主角瞎了,她哭得眼睛像杏桃;对他说,其实没有那么难。他抚摸她的头,像是在说:是的,亲爱的,这一切其实没有那么难。但是他们都错了。

分手之后她也不再准备考美国大学,开始了游离在幻觉幻听的生活。离他七年?或是八年?不记得了。她去年结婚之前,写了长长的信给他;解释在一起的一年里为什么她那样混沌,向他道歉。河河说道歉本是自慰。不是的,高中班上三分之二的同学是医生,脸书上绿大褂绿口罩外的眼睛,憔悴中有生机。动态里医学名词的拉丁文如异国的蚯蚓。她会想,啊,那就是我素未谋面的故乡。她的人生被抢走了,被弄坏了,在某一刻就扭曲,歪斜了。

要如何解释:是的,你吻了我,但我并未吻你。是的,你做了我,但我没有做。是的,那时,我

\newpage 

与你在一起,但我并不在那里。这一切,要如何解释,又为什么要解释?

那天她跟他说她上台北补习SAT的时候去找了你。为什么?她听见他的声音里有钉子、壁癌,和一整栋废弃的鬼屋。她说因为他根本比不上你。为什么?她听见流沙开始吞噬那鬼屋,鬼的尾巴开始嘬束,脸孔在融化。没有为什么,她说她就是爱你胜过爱他。一面说她自己也哭了,拿头脸身体去撞墙。他拉拦着她,沉沉地呐喊,像身体反刍之后的回音,他说了:没关系,真的没关系。过几天洗澡的时候淤青从乌云褪成老茶的颜色,一块一块在身上足有手掌大,斑斓得像热带鱼。她心想她是个人人放养其中的鱼缸。

那天她上台北补习SAT的时候去找你。隔着一年?或是两年?忘记了。你一开头就问她有男朋友了吗?她答有。你又问男朋友是谁?她说以前说过的,对面高中那男生。你一脸满意。当然她后来明白那是要减轻罪恶感。后来的事她没有对任何人提过,对精神科医生和心理咨询师也是两三年后才讲上来。长裤撕掉纽扣,小裤撕出线头。虽然也不是第一次。结束之后你开始看电视,凭着耳朵可以知道你又在看新闻,国家有人贪污,有人串供,你正义凛然、法相庄严地说起大道理,她静静地穿起衣服,静静地在你

\newpage 

旁边睡着。她依然不知道小时候发生了什么。也许她潜意识想重新被污一次。

他第二次回美国开学那日,说了一句情话给她。她泪不能止,因为那竟是从前你说给她过的。怎么可能迷信语言的人能得到真爱?送机之后她去买了一百颗普拿疼,不多也不少。那时在台南,被推进奇美,插鼻胃管洗胃。活性炭黑得像沥青,她像是把一生的黑夜都吐了出来。从成大调来解毒剂,又被送上救护车,高速公路一路蹄鸣,从深夜吆喝到白天,直推进台大的重症监护室。她的背可以感到一路上医院的地板很流利,毫不疙瘩,像一首童诗。身上插满了管线,红的红,绿的绿。呕吐的时候,心电图会尖叫,她的上身弹起来,牵动一声管线,管线连缀的点滴、机器痴痴地动摇。
转到普通病房,楚楚医生来看她,她想说话,无关紧要的词却像棉花漏出破娃娃:“耳机……走路……铅笔……”她捏扯自己的脖子和嘴唇,眼泪代替语言流了满脸。而楚楚还是一如往常对她说:“好,好,很好。”病房外,爸爸大声重复楚楚的话:“从没看过她情况这样糟?”为什么这个世界的隔音这样差。
后来他们一个在美国,一个在台湾,时差将近一个昼夜。视频聊天贪馋讲到他的睡觉时间,道晚安后就开着放着。他道晚安的笑眼,像不善用餐具的小孩子眯着筷子去拣一颗豆,那筷子的

\newpage 

深情。她一面看书一面看他睡觉。他偶尔打呼噜一声传过来,她总像电器被插上电源。她是全世界最幸福的人。
至重一次,他妈妈说了:你配不上我儿子。好几年以后,听说他家族有人生病,她马上想道:伯母,谁家的孩子都会生病。想到这里马上觉得自己邪恶,马上哭出声。伯母——我可能不配当你们家媳妇,但我是真爱你儿子的。后来也明白爸妈当初是不要他父母知道,第一时间才没送去成大,她心里一直有点恨意。
分手后一阵子,他放假回台湾,送了她喜欢的流行歌手CD给她。那个只听巴赫莫扎特伊萨伊的大男孩,微笑捧着荧光浓妆大人头的CD。她才第一次惊觉自己造成了如此之大的伤害。

闹分手的时候也是王子样,盛大的红玫瑰一抱一抱送过来。他在美国,请台南的花店老板写了字条:失去你我会活不下去。陌生的字迹,嗡嗡浮出他的声音。她知道他脸皮薄,竟还要在电话里叮嘱这样的信息,加倍觉得自己恶。可是来不及了。

她当然记得高中时候他在公众场合寻找她的目光,四目相接的时刻对她来说就像是呜呜如泣的火车在隧道里找到那个渐强的光,那个出口。在小小的地下室补习,转头抽过面巾纸,她一定可以看见他的眼睛——回过头来,左手边的河河已经在当医生,而右手边的册册在美国念博士。她以为自己会跟她们一

\newpage 

样。
那年,那天,你像夏天的鹅绒被,不合时宜地盖在她身上,感情强烈到凶恶。你说她美,说她才华,对她说与一个美且才的女生“能发生的关系都要发生”。她当然知道那是胡兰成的句子。她从未觉得自己像张爱玲,好比基督徒不曾觉得自己像耶稣。你清澈的恶意,她顿时间感到加倍赤裸、无所措其手足。
也许她早该明白,就像托尔斯泰描写当年的俄法战争,军队弃守莫斯科,撤退时把整个莫斯科城都焚毁了——你也像个兵,在离开她的时候,把不能带走的西,全部焚毁了。

\end{document}
