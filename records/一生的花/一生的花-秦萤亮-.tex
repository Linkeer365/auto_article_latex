\documentclass{article}
\usepackage[utf8]{inputenc}
\usepackage{ctex}

\title{一生的花\footnote{Click to View:\url{https://web.archive.org/web/20221007153327/http://m.zuowenku.net/372272.shtml}}}
\author{秦萤亮}
\date{}

% \setCJKmainfont[BoldFont = Noto Sans CJK SC]{Noto Serif CJK SC}
% \setCJKsansfont{Noto Sans CJK SC}
% \setCJKfamilyfont{zhsong}{Noto Serif CJK SC}
% \setCJKfamilyfont{zhhei}{Noto Sans CJK SC}
% \setlength\parindent{0pt}

\begin{document}
\CJKfamily{zhkai}

\maketitle


\Large


桃子已经二十六岁了,可还没有结婚。 

“什么时候才能参加桃子的婚礼呢?”热心的邻居大婶们,在见到桃子的父母的时候,总是这样追
问。 

“桃子什么时候才结婚呢?”已经成了家做了父母的旧日同学们,见到桃子的时候,也总是这样问
着。 

没有结婚的桃子,有时候应邀到朋友家里去做
客。 

在布置得很漂亮的房间里,跟年轻的夫妻面对面地坐着,大家一起喝茶,吃点心,看照片,聊着有
\newpage

趣的事情,度过一个轻松的下午。 

眼睛可是悄悄地,一直留意着墙上的大钟——
啊,还有一个小时就该告辞了。 


还有四十分钟了。 


还有…… 

时间一到,桃子就站起来感谢主人:“我该走
啦。谢谢你们请我来,今天下午过得真高兴。” 

于是主人也热情地回答道:“哪里的话,桃子
可要常常到我家来玩,不要一个人呆在家里。” 

结束了拜访,正是黄昏降临的时候。城市的上空,堆着玫瑰色的云块,暮色正从天边一点点展开。
 

灯一盏一盏亮起来,就像落上了尘土的星星。

\newpage
在暗淡的灯连成的路上,走着独自回家的桃子。 

渐渐的,桃子不愿意出门,也不愿意去做客了
。 


父母都很担忧。 


“这孩子,变得孤僻了啊。” 


“这样下去,怎么行呢!” 

在自己的卧室里,桃子好像总是听见父母在门
外叹息的声音。 

“也许应该自己搬出去住吧……”桃子心里计
划起来了。 

桃子开始留心橱窗上、报纸上的分类广告了。

在城市里,出租的房间,样子都差不多。有四方的灰白色的墙,从窗口望出去,只看见成群的灰色高楼。走进去,就像被关进了一个很久不用的抽屉里
\newpage


“住在这样的房间里……”桃子一直犹豫着。

一个黄昏,报上登出了一则小小的启事:“花
地旁边的木屋出租。” 


短短的广告,被花朵图案的分隔线围绕着。 

桃子想开了:“这个城市的什么地方,才有花
地和木屋呢……” 


但是,广告千真万确地印在那里。 

桃子是着拨通了电话,很久都没有人来接。不知怎么,桃子仿佛听到了那一头响起的铃声。是在远离人烟的地方,空寂的木屋里,一声又一声响着的电
话铃。 


终于,一个老婆婆来接听了。 

“是位年轻的小姐啊!”老婆婆似乎很高兴,
\newpage
“不是有心的人,可看不到这个广告啊。这个星期日
,你就来看看房子吧。” 


老婆婆告诉了桃子去木屋的走法。 

星期天的清晨,桃子坐上了横穿半个城市的公
共汽车。 

春日的原野上,只有这一条狭长的公路,无穷无尽地伸展着。车窗外,吹来了五月的熏风。在有节
奏的轻轻颠簸中,桃子朦朦胧胧地睡着了。 

不知道过了多久,桃子睁开眼睛,车上的旅客
只剩下自己了。 


汽车停在了一片花海的中央。 

像被漫山遍野的花毯陷住了一样,长长的绿色
巴士,静静地停着,风里满是芬芳的花粉气息。 


\newpage

桃子扶着车门,慢慢地下了车。 

这是一望无际的花的原野。浅红和绯红的,是蔷薇和杜鹃;金黄的,是迎春花和油菜花;粉红的,是天竺葵和樱草;月白的,是百合花和铃兰……像是把失落的春天都集中在这里了,带着朝露的数不清的
花朵,在没有人知道的地方,灿烂地开着。 

究竟有多少年,没有来过这样的地方了?还是少女的时候,曾经骑着单车一个人来过的挂着彩虹的原野吗?还是更早更早以前,无忧无虑地奔跑过的夕阳下的花地?或者是许多年前,特别珍惜的圣诞卡上,曾经画着这样开满鲜花的原野?后来,那卡片遗失
了,画面却变成了沉在心底的影像。 

在花地中央,被花的海淹没了裙裾,桃子忘却了寂寞和伤心,觉得自己又变成了多年以前的少女。

在不远的地方,桃子看见了自己要找的木屋。

那是一座小小的木阁楼,紧紧地挨着花地建造起来。从台基到屋顶,每个缝隙,都有细细的藤蔓花
\newpage
朵向外伸展着。看似用木板潦草钉起来的盒子,里面
却满满地盛着鲜花。 

这座房子,让桃子想起自己小时候,用玩旧的彩色积木,小心地砌起来的屋子。桃子喜欢得不知该
怎么办才好。 

一位老婆婆,正精神饱满的坐在木门前那窄窄
的梯级上。 

奇怪的是,她的衣着由鲜艳又罕见。她穿着一条绣满了花朵,像草地一样绿的披肩,戴着好像蒲公英绒球一样洁白蓬松的帽子,让人真想“噗”地吹一
口气,看那帽子会不会一下子给吹散了。 

“你就是想租房的女孩吧?”老婆婆微笑着问

“嗯,是我。”虽然是初次见面,恳求有一点不容易说出口,桃子还是热切地说了:“我喜欢您的
房子,请租给我吧。什么条件都行。” 

\newpage

“姑娘,你还没结婚吧?”注视着桃子的眼睛
,老婆婆突如其来地问了这么一句。 

“啊……”吃了一惊的桃子,不好意思地摇摇
头。 

“待嫁的小姐住在这里,最合适不过了。”披着草色披肩,戴着蒲公英绒球帽子的老婆婆庄重的说道。她把一把亮闪闪的大钥匙放在了桃子的手心里。

住进阁楼的第一天晚上,桃子好久好久都睡不

这是个满月的夜晚,也许因为窗子特别大,整
个房间,都好像浸在了月光的深水里。 

月亮地里,白色的薄窗帘轻轻地飘拂着。老婆
婆的问话,又在耳边响起了。 

“从前,我也有一个要嫁的人啊。”桃子对窗
外夜风中的花说道。 

\newpage

像回答一样,风中传来了隐隐约约的花香。这真像好久好久以前梦见过的地方啊。这样迷迷糊糊的
想着,桃子终于沉入了梦乡 


午夜的时候,桃子惊醒了。 

月亮已经落下去了。但是,房间里却被斜射进来的奇异的银白光辉照亮了。而且,侧耳细听的话,像是同时摇响了几千个风铃,静夜里满是细碎的铃声

桃子轻轻地爬起来,伏在窗台上向外看去。窗外的花地,成了一片灯的海。好像天上所有的星,都落在了花地里,发出了凉沁沁的明亮的银白光芒。这不是白天看到的,普通的花的田野了。每一朵花都施了魔法似的亮起来了。风铃的声音,正是从花地里传
来的。 

像梦游一样,桃子悄悄地开了门。从满是冰凉夜露的木台阶上下来,眼前就是望不到边的花的灯。侧耳倾听,花丛中那种铃儿一样,又急又快的清响,

\newpage
就是花们摇摆着身子,此起彼伏地发出来的。 

“花儿在交谈,”桃子忽然心里起了一种这样
的感觉。“让我再靠近一点点……” 

一点也没有发出声音,桃子走到了花地的边缘。然而,霎时间,花齐刷刷地向她看了过来。桃子有种被几千道灯光照着的感觉,眼睛都有点睁不开了。数不清的灯,像万花筒一样地旋转了起来。那缤纷的
颜色,像是要把桃子吸进去一样。 

“哎呀……”清醒过来的时候,桃子已经站在花地深处了。单薄的衣衫沾满了露水,他不由自主地
打着哆嗦。 

小铃的响声,在桃子的耳朵深处渐渐地慢下来了,变得舒缓、清晰和悦耳了。她一点一点地听清了
花儿们的话。 

一个很娇气的声音,在身边问:“你是谁啊?
” 

\newpage

桃子弯下腰,认出来了,这是像小姑娘一样娇嫩的油菜花。“我,是桃子……觉得这样说不大合适,桃子又补充了一句:“我是刚刚搬到附近来住的…
…” 

“那么,是邻居了呀。”像玉做的铃铛一样的
铃兰,叮叮当当地说。 

“嗯。”桃子的心定下来了。她在花地里跪坐下来。顿时,许许多多的花儿向她依偎了过来。在柔
软的亮晶晶的花儿包围中,桃子身上暖和了起来。 

“你们今晚,为什么这样不寻常呢?”桃子小
心地问。 

“因为这是满月的夜晚啊。在满月下,花地有
特别的魔力。” 

“我们是被彼此的光芒照亮的。”许多小金钥
匙一样的蒲公英,一起回答道。 

\newpage

想把根平时不一样的蒲公英看得仔细些,桃子的衣角,被什么钩住了。是有刺的棘杜鹃,拉着桃子的睡衣,尖声尖气地说:“你的衣服,太暗淡啦。”

“哎,是的,这是我的睡衣……”看着棘杜鹃那洒着斑点又打着美丽皱褶的殷红花瓣,桃子真的惭
愧起来了。 

“我们啊,一生只穿一件衣服。”一个特别温和平静的声音在说。这是色泽浓郁的大丽花。“这件衣服,就是我们的嫁纱,即使凋残了也不换下。我们
的一生,都是穿着嫁纱度过的……” 

“最开始,我们也许是一段根,也许是一粒小小的种子……”这是纤纤的樱草在说话。“可是,我们心里都藏着颜色……是无论在泥土深处睡了多久,
都不会忘记的颜色……” 

望着盛开的花,桃子深深地难过了。“我,也像在泥土深处啊。但是,我的花在什么地方呢……”

\newpage
桃子悲哀地想。 

“桃子,别灰心。花期有早有晚,说不定,你就是那种开得特别迟的花。”远处,眼睛蓝莹莹的矢
车菊温柔地说。 

“每个人的心里,都有花。但是,有的人的花,早早地就谢了。于是,他就看不到我们,也听不到
我们……” 


“桃子,你的心里,也有一朵花啊。” 

桃子不觉用手按住了心口。在那砰砰地跳个不停地胸中,真的有一朵自己都从未见过的花吗?是什
么样的花呢? 

“桃子,你心里的花,是白色的。”身边响起了一个亲切的声音,是仰望着桃子的灯笼花。灯笼花的话,忽然像是在黑暗的房间里,打开了灯的开关一样,桃子心中浮现出了一朵纤细的白花的影像,看见了那满含着寂寞的,细小的花瓣。然后,灯啪地灭了

\newpage
。花的影像不见了。 

“那,就是你心里没有开出来的花……”灯笼
花说。 

“那……我的花,怎么才能开出来呢?”桃子
不觉问出了声。 


“要像我们一样,一心一意地……” 


“不怕寂寞……” 


“不怕冷,也不怕黑……” 

“最重要的是,相信自己一定能开出花来……

铃儿的声音,渐渐低下来了。花的灯,不知不觉中,也从远到近,一盏一盏地熄灭了。“满月的魔法快要消失了。”朦胧中,桃子想道。然后,她的眼
皮也沉沉地垂下来,睡着了。 

醒来的时候,桃子是在自己的床上。窗前,洒
\newpage

满了四月的朝曦。 

“咦,昨晚……”桃子拼命地回忆了起来。她恍惚记得,自己是在月沉之后的花地里睡去的。身上
,好像还留着月夜的露水和花海的芳香。 

从这个奇异的夜晚之后,桃子觉得自己发生变化了。好像心里的一个角落被擦亮了一样,从早到晚,总在期待着什么。也许因为正是春天,无论走在城市的什么地方,总有各种各样的花儿开放着。在桃子眼里,那些花都特别的魅力而可亲。在满月的夜晚,它们会不会也苏醒过来,被一朵一朵地点亮呢?这样一想,桃子就觉得和花儿们共同守着一个美好的秘密

桃子的心,奇怪的变得容易喜悦了。特别晴朗的天气,好听的歌,还有蓝天上的鸽哨,都能使她快乐起来了。已经忘却的事,也不断地在脑海中浮现了

遗忘了很久很久的,少年时代的心情,在桃子身上复苏了。像一把灵敏的小提琴,轻轻拨一下弓弦

\newpage
,就会响起音符。 

这些年来,是从什么时候起,把内心的门,紧
紧地关上,再也不会对任何人打开了呢? 

“真好像在又冷又黑的地方,一个人待了很久
很久啊……”她盼着第二个满月之夜快点到来。 

下一个月圆的夜晚,月亮沉落以后,桃子来到了花地里。花儿都已经醒来了,依然是闪烁的,海一样的花灯。与上次不一样的是:“迎春、连翘,还有许许多多别的花儿,花期已经结束了,悄悄地凋谢了。明亮的蝴蝶花、金针花,还有许多繁茂的小草花,
又纷纷地开了出来。 

“桃子,你来了。”这是即使在夜色中,也特别艳丽的玫瑰。桃子俯下身,抚摸了它丝绒般的花瓣

“桃子,你心里的花,开出来了吗?”许多花
儿在问 

“哎,哪有这样快呢……”桃子默默地笑了。
\newpage


“不,不快,对我们来说,一生的时间,就只有一个春天那么长,”玫瑰微笑着说。“不要忘记,你是受过月下花地祝福的女孩啊……”坐在花地里,桃子沉思者抱住了膝头。她想在心里,再次看见那朵纤细的小白花。现在,它在悄悄地破土发芽吗?还是
仍然沉睡在泥土深处呢? 

“桃子,你的心,被一种颜色占满了。”一直
没有说话的野百合,静静地说。 

“不把这种颜色忘记,你的花,永远也开不出
来。” 


“啊,真的吗?……” 

桃子闭上了眼睛,努力地回想起来……那不属于自己的,牢牢地关在心里头的颜色,会是什么呢?

“让我们来帮助你吧。”许多花儿叮叮地说。

\newpage

“桃子,在我们中间,挑一朵你最不喜欢的花
儿吧。” 


“哎呀!”桃子小声地说。 

在像铃儿一样响着,小灯笼一样亮着,看上一眼也会头晕目眩的花海中间,挑出一朵不喜欢的花儿……桃子还是蹲下身子,仔细地分辨起眼前的花来。


“叮叮叮叮叮。” 


被桃子看到的花儿们都轻轻摇摆起来,像是在
笑: 


“可不是我呀。” 


桃子的耳畔响彻了几千重的细碎铃声。不知过了多久,桃子的眼前一亮。一片熟悉的色彩,从花海的深处,一下子跳了出来。那是一丛红色的凤仙花,
\newpage
凝然立在月下,一点也没有笑意地团团开着火红的花朵。那种红,像是要夺去人的魂魄一样,越看就越觉得红到心里。已经淡忘了的痛苦,一下子回到了桃子
心上。 

那丛凤仙花,让桃子想起了那个手指上染着寇丹、头发像缎子一样黑的美丽女孩。桃子要嫁的人,
就是在认识了那个女孩之后,才取消了婚约的。 

后来,套子的未婚夫和女孩一起离开了他们的城市,到远方去安家了。那丛火红火红的凤仙花,让
桃子想起了从前的伤心日子。 

那已经是好几年前的事了。桃子的眼泪不知不
觉地涌到眼眶里。 

花儿挤挤挨挨地站在一起,叮叮地悄声说着:


“找到了。” 


\newpage

声音越来越大,在桃子耳边响起了这样的话:


“带走吧。” 



“把她带回家去吧。” 

把那像女孩子一样美丽的、连看一眼都觉得痛苦的花带回自己的家里?桃子的心像是被猛扎了一下


花儿们大声吵嚷起来。 


“带回去吧!” 



“听我们的话,带回去吧!” 

花的声音变得又嘈杂又震耳,像一片打击乐,越来越尖利,简直让人无法忍受了。洁白的小铃兰,
拼命摇起了自己所有的铃铛。 

\newpage


桃子捂住了耳朵:“好吧。” 

就这样,桃子采下了红色凤仙花,把她带回家
里了。 

她找出了空玻璃瓶,盛满了水,把花放在照得
到月光的窗台上。 

最初的几天里,桃子和凤仙花谁也不理睬对方

然而,每当桃子悄悄地注视着凤仙花的时候,好像真的有一种灼热、痛苦的神秘颜色,在汩汩地向外流去。桃子的心,也不可思议地,变得越来越平静


一个月夜。 

“桃子啊,用我染指甲吧。”凤仙花忽然说。


“啊?”桃子吓了一跳。 

“把我摘下来,做成蔻丹,染你的指甲吧。”
\newpage

凤仙花清清楚楚地说。 


“这……” 


“快,我就要开败了。”凤仙花催促道。 


“可是,为什么……” 

“这样,对你对我都会很好的。快找我的话做
吧……” 

桃子一瓣一瓣地,摘下了红红的凤仙花。像小时候用花花草草打扮自己一样,在月下的窗台上,她把花瓣仔细碾成了红色的汁液。这样做着,桃子心里渐渐有了一种温柔的感觉。她仿佛听见凤仙花在无声
地说:“这,也是我深藏着的颜色啊……” 

在捣好的花瓣里加进明矾,桃子用纱布蘸着凤仙花的汁液,染红了每个指甲,然后又在月光里晾干。鲜艳透明的指甲涂成之后,桃子的心,彻底安详下来了。心底那秘密的殷红色,终于一点点淡去,最后
\newpage

完全消失了。 

“好像告别了过去的自己一样啊……”桃子快
乐的想道。 

一个黄昏,有人咚咚地敲响了木屋的门。这样偏僻的地方,谁会来呢?桃子开了门。门外,是陌生
的年轻男子。 

“我迷路了,能告诉我,这是什么地方吗?”
男子有非常亲切的脸和声音。 

“我本来是想散散步,结果不知不觉地走到这
一带来了,这是我完全没来过的地方。” 


“啊……” 

桃子问了他的住址,认真地为他指点了方向。


男子并没有马上离开。 

\newpage

“今天不知是怎么了,像被路边的花指引着一样,一个劲儿地往这个方向走……等看到花地的时候
,才知道迷路了…… 

“然后,一直走到这里,才好像到了目的地似的。真是奇怪的事情啊。”男子注视着桃子说,“能
把你的名字告诉我吗?” 


桃子犹豫了一下,说了出来。 

“桃子小姐,涂着这样的蔻丹……真像是我小
时候邻家的女孩啊。”男子微笑着说。 

桃子不知怎样回答才好。她低下头,欢喜的感
觉,一点点升起来了。 


那以后,男子常常来拜访桃子了。 

“桃子小姐住在这样的地方,第一次看见你的
时候,觉得分明是花的精灵啊!”他这样说道。 

\newpage

“那么,你觉得我是一朵什么样的花呢……”
桃子轻轻地问。 

男子凝视了桃子很久:“像一朵小小的白花呢
……” 

桃子不好意思地笑了。她的眼睛有点湿润了。


不久,桃子就答应了他的求婚。 

在准备退租的时候,桃子才想起了一件事。报纸上留下来的联系电话,就是小木屋里的。那么,如
今,要到哪里去找那位老婆婆呢…… 

把放着租金的信封,留在了桌子上,桃子怀恋地锁上了小木屋的门。不久的将来,这里还会住进新
的女孩吧…… 

当桃子作为新娘,再次来到花地的时候,已经
是深秋时节了。 

\newpage

原野上,只有星星点点的天竺葵和大波斯菊寥落地开着。曾经漫山遍野盛开的花,像雪一样的消融了,连一点点痕迹也没有留下。但是,桃子知道,在看不见的泥土深处,那些根和种子正静静地沉睡着。它们的心里,都藏着秘密的颜色,等待着属于自己的
花期。 

站在空旷的花地里,飒飒的秋风从身边吹过,
桃子隐隐约约地听到了那些随风远去的花铃。 


“要永远永远,在心里开着花啊……” 


桃子悄声说:“谢谢了。” 

她仿佛又看见了在满月的光辉里,一直开到天
涯的花。 

每个人的心里都有一朵花,愿它,开到天涯。

\end{document}
