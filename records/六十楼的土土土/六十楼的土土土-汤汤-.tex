\documentclass{article}
\usepackage[utf8]{inputenc}
\usepackage{ctex}

\title{六十楼的土土土\footnote{Click to View:\url{https://web.archive.org/web/20221009031815/http://www.dfetwx.com/article.php?id=1539}}}
\author{汤汤}
\date{}

% \setCJKmainfont[BoldFont = Noto Sans CJK SC]{Noto Serif CJK SC}
% \setCJKsansfont{Noto Sans CJK SC}
% \setCJKfamilyfont{zhsong}{Noto Serif CJK SC}
% \setCJKfamilyfont{zhhei}{Noto Sans CJK SC}
% \setlength\parindent{0pt}

\begin{document}
\CJKfamily{zhkai}

\maketitle


\Large

我每天都在寻找,瞳孔里开着烟灰色花朵的
人。 

瞳孔里的花,小小的,灰灰的,有五个花瓣,
或者六个花瓣——只是大多数人都看不见哦。 


我是谁?你不可能猜得到。 

我假装成一个人,住在城市的六十楼。我姓“土”,名“土土”,所以我叫做“土土土”。我喜欢
这个名字。 

有的时候,黄昏会特别美丽,这个时候我比较容易忧伤,我站在六十层的露台上,扯着嗓子喊“土土土——”,然后又扯着嗓子答应“哎——”其实就
\newpage
算我扯着嗓子,声音也是又哑又轻的。而且,一天比
一天哑,一天比一天轻。 

但我会继续呼唤,因为我喜欢它,虽然它不是
我原来的名字。 

更美妙的事情是,呼唤之后,我的眼前会出现大片大片的土地,开着花儿,长着草儿。我知道那是
幻觉,我喜欢这种幻觉。 

我清楚地知道,我不适合住在城市。只要走在硬邦邦的水泥路上,我就不断地摔跟头。嘴啃泥式、四脚朝天式,屁股落地式,头部着陆式……称得上千
姿百态呢。 

我更知道,适合我居住的,是在很远很远的乡下,是到处能踩到泥土的地方。可是,我住这里的年头已经很不短了,我是个重感情的人,我舍不得走。
更何况,还有一件事情,等着我去做。 


\newpage

唉,这儿,这儿原来真不是这样的。 

它曾经像一片巨大的桃树叶子一样,憩在两条小小的河流之间。我用我的脚仔仔细细丈量过,长22007步,宽2207步,没错,一步不多,一步不少。我对这里熟稔得像对自己长长的胡须,庄稼地啊,村庄啊,花呀,草呀,树呀,爱扭屁股的蛇啊,热情洋溢的青蛙啊……不喝一口水我也能说上十天十
夜。 

当然有谁愿意听我说这些呢?人们压根儿没有
时间留恋过去。 


再说我的胡子,已经被我雪藏了十几年。 

就好像曾经到处都是的柔软芬芳的泥土,被这
个城市雪藏了十几年一样。 

高楼和水泥路似乎是一夜间侵吞了这里。我只记得当时的手足无措和目瞪口呆,很长一段时间里的
我则像一阵受惊的北风。 

\newpage

我是想过离开,可是舍不得。虽然它陌生得常
让我悲从中来。 

我有一个布口袋,以前用来装馒头,现在我用
它来装泥土。 

我有一个露台,直走68步,横走24步,丁
点儿大吧。我给它铺上泥土。 

我朝着东、西、南、北不同的方向,走很远很远的路,背回一袋一袋的泥土。不同地方的泥土有不同的颜色,乌油油的黑土,晚霞般的红土,雪一样皎洁的白土,还有茄子紫的,橙子黄的,咖啡色的,绿茶色的……不同时候的泥土有不同的味道,春天是蜂蜜味,夏天是薄荷味,秋天是甜橙味,冬天是糍粑的
味道。 

当露台上的泥土铺到十厘米厚时,我在上面种
了番薯。土太薄,番薯长不大,藤也瘦巴巴的。 


\newpage

做这点事儿,就花了我整整三年的时间。 

第四年,我开始寻找,瞳孔里开着烟灰色花朵
的人。 

从六十楼下来,马不停蹄也需要两天时间。我不会坐电梯,我害怕。别问我为什么害怕,害怕是不需要理由的。我的双脚永远也适应不了踩着水泥地走路的感觉,一不留神,就摔个跟头。所以每一步我都
走得很小心很小心。 

我遇到的第一个瞳孔里开着烟灰色花朵的人,是一个男子,他的皮鞋很亮很亮,亮得能映出100层之高的楼顶。我走上去,拦住他的路。他不满地瞪我一眼,侧侧身子想擦肩而过,我追上他,又把他拦
住。 


“我们认识吗?”他皱着眉头叫道。 

“不认识。”我微笑着说,右手插在口袋里,
紧紧抓着一把土。 

\newpage


“那你……” 

“我叫土土土。”与此同时,我的手从口袋里抽出,摊开手掌,对着他的脸,狠狠地吹了几口气。
“呼呼”飞起的土顿时迷住了他的眼睛。 

他没有办法说完要说的话,半秒之内,变成了一条蚯蚓,匍匐在冷冰冰的地上颤栗。我捡起他,放
进口袋。 

没错,我要做的事情就是找到瞳孔里开着烟灰色花朵的人,把他们变成蚯蚓,让他们和泥土在一起
。 


只是这种寻找不太令人愉快。 

因为我总是摔跟头。我把鼻子摔扁了,还摔飞了两颗门牙。我的屁股之所以一边高一边低,也是摔的。每次我以不同的姿势和水泥地面亲吻的时候,周围爆发出的笑声,就像大小不一的冰雹“叭叭”落在

\newpage
瓦片上。 

几百年前,不,就算是十几年前,我并不是这么笨拙的。我能走得风一样快。我的脚踩着泥土的时
候,比鸟儿都轻盈。 

我把这些蚯蚓带到我的露台上。爬到六十楼,马不停蹄地需要两天时间。我不坐电梯,据说那只是
几秒钟的事情,但是我害怕。 

我把蚯蚓们一条一条放到“番薯地”里。他们一碰到土,毫无例外地,弓着身子就往土里钻去,一会儿全看不见了。我拍拍手,久久地微笑。我知道他们或者不停地睡觉,或者不停地挖洞,这些对他们都
有好处。 


我离不开泥土。 


可是我已经离开了十几年。 

城市零零星星的花坛里有土,我的露台上有土,对于我来说,它们连杯水车薪都不是。我总是口渴
\newpage
,大口大口地喝水,还是渴得厉害。我知道我需要的
不是水,而是大片大片的泥土。 

当这里还是大片大片泥土的时候,我常常仰面躺在泥土上,一躺就是几天几夜,全身被温暖、湿润和芳香所包围……一想到这些,巨大的幸福和悲伤便
狠狠冲击着我的眼睛,大滴大滴地落泪。 

泥土是世界上最好的东西。只是很多人还不知
道,很多人已经忘却。 

我的身体,因为太久太久没有亲近泥土,正一天一天地衰弱和枯萎。曾经我有一把发亮的胡子,因为掉得太凶,被我剪了,用一块蓝印花布包着放在衣
柜的最底层。 

我打算,等找到所有瞳孔里开着烟灰色花朵的人后,就带着我的胡子离开这座城市。再舍不得也得
离开。 

几天之后,“番薯地”里的蚯蚓一条接着一条
\newpage

钻出来。 

他们从土里一探出脑袋,就变回了原来的样子。那个男子的皮鞋依旧很亮很亮,惟有瞳孔里烟灰色
的花朵消失了。 


“到底发生了什么事情?”他们互相询问。 

问不出什么结果,他们目光终于聚焦到我身上
,“我们,好像,哪里见过?” 


“我叫土土土。”我微笑着点头。 

“哦——”拖得长长的声音,不住点着的下巴
,也许是真想起来了吧。 


接着就会听到他们开心地大叫: 


“感觉真舒服啊,好像泡了个热水澡!” 


\newpage

“全身轻松啊,好像卸下了很多担子!” 

“呼吸也通畅多了,心情明媚得像春天的阳光
!” 


“……” 


呵呵,我做到了,我对自己笑了笑。 

接着,他们就抓住我的袖子,一个劲儿地问我
是怎么回事。 

我不知道他们是否听得懂,我说,太久没有亲
近泥土的人,瞳孔里会开出烟灰色的花朵。 


“真的吗?” 


“烟灰色的花朵?我们可从来没有见过。” 


我说:“你们看不到。” 


\newpage

“看不到?那只有你能看到?” 

我说:“也许吧。但是你们自己一定能感觉到
,是——那种很深的疲惫和迷茫。” 

“哦——”拖得长长的声音,不住点着的下巴
,也许是懂了吧。 

他们向我表示感谢,亲热地和我拥抱。然后满脸阳光,身体轻盈地离开,有的人还唱着歌儿。而我
又下楼去…… 

多少年来,我一直不停歇地寻找着瞳孔里盛开着烟灰色花朵的人,可是他们比我想象的要多得多了


而且,好像还越来越多。 


我不得不一年一年地推迟着我离开的时间。 

我不断地摔跟头,最多的一天,摔了八百七十二个,嘴啃泥式、四脚朝天式,屁股落地式,头部着陆式……用了两百十六种姿势,其中有六个姿势比舞
\newpage

蹈家还优美一百倍。 

有一件事情令人愉快,现在我只要一合上眼睛,哪怕是摔倒在地时一刹那的晕眩,我就会做起梦来,梦见大片大片的泥土啊,开着花的,长着草的。我的脸、我的身体、我的胳膊、我的腿、我的每一个手指头、每一个脚丫丫,都紧紧依偎着它们,慢慢地活
泛、舒展,充满活力。 

有一天,那个穿着很亮很亮皮鞋的男子竟然找
上门来。 

他焦虑地说:“土土土,我的瞳孔里是不是又
开出了烟灰色的花朵?” 


没错,是六个花瓣的。我点点头。 

他说:“难怪啊,总是累总是困总是不开心。
” 

我从口袋里掏出一把土,迷住他的眼睛,他又
\newpage
变成了一条蚯蚓。过了几日,他一身轻松地离去,临走之前,他对我说:“土土土,你的眼睛里好像也有
什么东西呢。” 

我的心“咯噔”一下,我打了盆水,对着它看我的眼睛——瞳孔里正怒放着一朵烟灰色的花。我摇
着头对自己笑。 


后来,我便疲惫得下不了楼了。 

我知道,我的日子到了。我已没有力气离开这
座城市。 

那天正是惊蛰,春雷隆隆地响个不停。我吃下一把泥土,变成了一条蚯蚓。正要往土里钻的时候,
听到“咚咚咚”的敲门声,听到很多人在喊: 


“土土土,就住在这里!” 


“土土土,开开门啊!” 

\newpage

“土土土,我的眼睛里一定开出烟灰色的花朵
啦。” 


“土土土,帮帮我!” 


我往土里钻去,钻去,多么温暖湿润的泥土啊。我的眼睛睁不开了,我的身体不听自己使唤了。作为一个上千岁的土地公公来说,我已经尽力了。没错,我原来的名字叫做土地公公,庇佑这方土地曾经是我的使命。可是自从柔软芬芳的“桃树叶子”变得冰冷坚硬,我就力不从心了。现在我太累了,我要睡了
。我不知道我要睡多久,也不知道能不能醒来。 


对不起哦! 

土土土要睡了。

\end{document}
