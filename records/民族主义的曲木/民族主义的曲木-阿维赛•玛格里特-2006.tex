\documentclass{article}
\usepackage[utf8]{inputenc}
\usepackage{ctex}

\title{民族主义的曲木\footnote{Click to View:\url{https://web.archive.org/web/20221128022923/https://www.douban.com/note/174995348/?_i=9602296d7R2Q8h}}}
\author{阿维赛•玛格里特}
\date{2006-05}

% \setCJKmainfont[BoldFont = Noto Sans CJK SC]{Noto Serif CJK SC}
% \setCJKsansfont{Noto Sans CJK SC}
% \setCJKfamilyfont{zhsong}{Noto Serif CJK SC}
% \setCJKfamilyfont{zhhei}{Noto Sans CJK SC}
% \setlength\parindent{0pt}

\begin{document}
\CJKfamily{zhkai}

\maketitle


\Large

以赛亚•伯林对著名的临终遗言抱有强烈的兴趣。当魏茨曼的医生将其临终遗言转告给伯林时,既使他发笑,又令他烦扰。使他发笑的是其机智和心灵的显现,烦扰他的是其强烈辛酸的表达。魏茨曼临终前在病榻上剧烈地咳嗽,感到窒息,他的医生关照他吐痰,魏茨曼喘息着用意第绪语说:“没剩下什么人值得唾弃了。”然后死去。
     对伯林来说,魏茨曼是非神经质的犹太教徒的典型。然而,在以色列建国的那些日子里,他感到孤独和辛酸。一位到他家造访的朋友问他,这些天在做些什么。魏茨曼回答说:“做?他们告诉我,作为以色列总统,我象征着这个国家,所以我就整天坐着,象征着。”使伯林感到冲击的是,魏茨曼在他的故乡如同无家可归的李尔王。但另一方面,犹太复国主义的整个要义是为犹太人提供一个家的感觉
\newpage
。
     本一古理安(Ben一Gurion)是魏茨曼的论辩对手,也是要对魏茨曼的辛酸负有最主要责任的人。他曾邀请伯林担任外事部门的领导,后来又邀请阿尔伯特•爱因斯坦在魏茨曼之后继任以色列总统,他们俩都谢绝了。他们两人都认识到对家园的需要,认识到犹太人对一个民族家园的需要。但他们都相信,移居以色列是一个个人选择的问题,他们两人都拒绝了整合性的犹太复国主义(integralist Zionism),它将个人选择变成一个事关命运的问题,也变成了对扰太纯正性的检验。
     伯林不像那些因为不能承受做犹太人而成为犹太复国主义者的人,他完全坦然于自己是一个犹太人和一个犹太复国主义者。伯林非常关切普通人的需求,有些知识分子倾向于把他们特权化的需求投射到普通人的身上,而全然不顾普通人与他们不相符合的需求,但伯林却没有这种知识分子的精英主义偏见的痕迹。就个人来说,伯林不需要在以色列有一个家,但他相信很多犹太人需要。
     经典的犹太复国主义将这种主义看
\newpage
做是对反犹主义的解决方案,但伯林不这么看。他心里重视的不是特定的反犹主义,而是它的一种后果,即犹太民族家园感的丧失。伯林致力于将反犹主义界定为对犹太人的过度仇恨,与此同时,他比别人更明了反犹主义可能采取的种种精致的形式,是那些既憎恨犹太人又憎恨反犹分子的人将此精致化的,因为(除了别的原因之外)他们发现两种憎恨方式都过于粗俗。经典的犹太复国主义念念不忘的是一种充满恶意的反犹主义形式。伯林懂得,甚至那些温和的反犹主义也足以有效地使犹太人感到不自在,剥夺了他们行为的自发与天然的形式,致使他们变得过度自我意识。出于某些可以理解的原因,犹太人成为所谓现代人的生存苦境的一种隐喻,但这种苦境是更为具体而更少形而上学的。伯林有一次问我:“你认为什么是所有犹太人的共同之处?我所指的是来自萨那(Sana)、来自马拉喀什(Marakesh)、来自里加(Riga)、来自戈尔德斯•格林(GoldersGreen)的所有犹太人?”,然后,他立即自己回答说:“一种社会性的不自在的感觉,没有一个地方能让犹太人感到全然在家。”
     伯林,像他的导师魏茨曼一样,是
\newpage
直觉本能的而不是意识形态化的犹太复国主义者。他不觉得需要为他的犹太复国主义作正当性辩护。犹太复国主义明显要求大量的正当性辩护,因为不管你如何划分,犹太人重获家园就意味着巴勒斯坦的阿拉伯人要失去他们的家园。这困扰着伯林,但还没有严重到使他严肃地质疑他的犹太复国主义。他将犹太复复国主义视为犹太历史中固有的内在动向。一旦东正教聚居区(Orthodox ghetto)不再是一种可能的生活选择,而同化也不再是一个可选的方案,那么犹太复国主义就是一个自然的解决。与此同时,伯林也从不将犹太复国主义看做是一个平白的常识。他援引魏茨曼的话说:“为成为一个犹太复国主义者,你不必疯狂,但疯狂是有帮助的。”犹太复国主义中的“疯狂”元素是其成功的赌注。相信一个离散的民族能够建立起一个拥有六十五万人口的以色列国,仅仅是在第一次犹太复国主义者大会后的五十五年;而五十五年后,它成为五百万犹太人的家园,这要求一种信念的伟大跃进。
     伯林不是那种犹太复国主义的同路人,那种人相信在以色列可以实现一个乌托邦,只不过它不能被一个具有敌意的外在世界所理解。伯林不
\newpage
是一个为乌托邦寻找托词的辩护者,因为他对犹太人和以色列国家的性质不存多少幻想,他并不幻想以色列国家是照耀这个民族的明灯。他将以色列看做是对犹太人具体苦境的明智的解决方案。我的主张,或者用更夸大的方式说,我的论题是:伯林的犹太复国主义不是源自诸如民族主义或自由主义之类基本原则的一种意识形态。对他而言,犹太复国主义更类似于家庭事务,而不是教条。然而,伯林的犹太复国主义形态与潜藏在他的民族主义形态之下的那种情感是一致的。
     对伯林来说,民族主义的情感支柱是民族主义最重要的元素,比滋养它的一套信念更为重要。总的来说,伯林关心的是那些激发了种种社会运动的情感、感觉和情绪,更胜于关心它们的理念。他的关怀不是对那种伴随理智的纯粹哲学的关怀,而是对伴随着感性,对理念与感觉之间系统性纽带的关切。像休谟一样,他相信激发人们的是感觉和情感,而不是理性。虽然感觉没有理念是盲目的,但理念没有感觉是僵死的,因此活生生的事物才是他的兴趣所在。最后这一点对理解伯林的思想事业至关重要。
     伯林是尼采所称的“Psycho
\newpage
logue”,是法国人所称的“Moraliste",不同于这些术语的现代用法,这既不是指“心理学家”也不是“道德家”。一个Psychologue是这样一个人,通常是作家,尤其是小说家,他具有一种能力,通过移情、穿透表象和语词,将自己置于人类灵魂的运思。一个在法语意义上的Moraliste不是某个在道德意义上心灵高尚的人,他关切的不是道德判断,而是识破陈规、习俗和社会表象的层面,把握真正使人们行动的事物。因此伯林是一个Psychologue,也是一个Moraliste。虽然他是一个自由主义道德的伟大信奉者,但他从不相信其心理学,特别不相信那种源自启蒙主义心理学的事物,他将此看做是全然幼稚可笑的。在关涉民族主义心理学的时候,他转向浪漫主义以求得洞见。
     按照伯林的说法,是什么样的情感激发了民族主义呢?欲求归属是一个强烈的动机:无论是归属于一个家庭、一个氏族、一个部落还是(在我们的时代里)一个民族。你欲求归属不是借助你的作为,而是由于你之所是。你不希望你的成就成为你归属的前提条件。你不可能,比如说,做不成一个爱
\newpage
尔兰人。现代生活,因其强调竞争和成就,造成了对归属的情感解药。现代生活侵蚀了家庭、氏族、部落到了这等地步,以至于民族成为了基本归属的主要替代物。
     在伯林的思想中,归属与家园感连在一起。实际上,伯林采用了罗伯特•弗罗斯特(Robert Frost)的定义:“家园是那样一个地方,当你到那里,他们必须接纳你。我本应该称其为某种你不必配得上它才能拥有的事物。”家是一个如果你属于它、它就不能拒绝你的地方,这样一种家的观念正是以色列作为一个犹太复国主义国家的一个法律的基础,这就是“回归法”,在我看来是有疑问的但却是其构成性的法律,它决定了在这个世界上的每个犹太人都有资格无条件地享有以色列的公民权
     感觉在家(feel at home)又有多重要?以赛亚•伯林看到了感觉在家与处于自由状态(being free)之间的内在关联。当我们的一个客人询问:“我可以用这个或那个吗?”我们时常回答“请随意(feel free)”或者“就像在自己家一样(feel at home)"——这两者是可以同等互换使用的。这
\newpage
一对家庭事务的观察捕捉了在伯林看来是非常深刻的关联,即在家的感觉与处于自由状态之间的关联,那种能够自然和自发地行为举止的能力。对伯林来说,这就是重要的。我们已经提到了有两种推进民族主义的情感,即归属感和家园感,除此之外,伯林还提到了第三种:羞辱,在民族蒙耻的意义上的羞辱。这是在德国民族主义的形成中最为显著的情感。这被表达在“弯枝”(bent twig)的隐喻中。“弯曲”(bent)影射了被公认为更先进的以及文化上更成熟的外国势力对这个民族的羞辱性征服。然而,因为这是一个弯曲的枝条,这个枝条终究会反弹,反过来鞭挞民族的欺辱者。
     在这三种情感——渴望归属、向往家园以及感受羞辱——之中,哪一种在塑造犹太复国主义中起到了作用?缺乏归属感和家园感必定起到了关键作用,但其作用与它们对正常民族——这些民族的人民世代相邻居住——所发挥的作用有所不同。归属对于犹太人民并不意味着归属一个由像你那样的人所构成的民族。因为犹太人在如此长的世代里被驱散,他们彼此之间非常不同。因此,就犹太人的情况而言,归属于你的亲属并不意味着归属于你的族类。论
\newpage
及家园感,如果在家意味着在家里(也就是在一个共享的领土之内),那么犹太人缺乏领土。但是伯林相信,归属于以色列领土上的犹太共同体将为创造或恢复犹太人的家园感构成条件。他认为,对于移居以色列的犹太人来说,以色列成功地创造了一种家园感。据说,当人们对伟大的新康德派行学家赫尔曼•科恩(Hermann Cohen)谈及犹太复国主义的时候,他带着嘲弄的口吻反问道:“那么这些人想要幸福吗?”科恩的这一反应困扰了弗朗兹•罗森兹瓦格(Franz Rosenzweig)和戈舒姆•肖勒姆(Gershom Scholem),使他们担心犹太复国主义的目标或许还不够崇高。伯林会有不同的反应,在他的想法中犹太人甚至不会要求幸福——对他来说这本身就是重大的事情——他们仅仅是想要感觉在家。
     至于第三种情感,民族羞辱的情感,很显然,犹太人在他们的整个历史上作为个人也作为群体曾严重地蒙受耻辱。但是,东欧犹太复国主义运动的基层群众并没有感受到现代形态的民族羞辱,这种蒙耻的形态是指被另一个比自己文化优越和更为先进的民族所征服。当然,某些西欧的犹太复国主义
\newpage
领袖的确感受到这种羞辱,现代犹太复国主义的奠基人赫茨尔(Herzl)也的确被一种尖锐的民族羞辱感所激发。但是,赫茨尔对于渴望家园感的犹太复国主义知之甚少,他愿意将乌干达作为犹太人问题的解决方案,这显示出他被激发的动机是如此不同于他的民众。在这方面,作为领袖的赫茨尔受到了他的追随者的教育。
     至此,我论及了以赛亚•伯林关于那些支撑现代民族主义的几种情感的思想。但什么是现代民族主义?按照伯林的看法,这是一种信念:人们归属于特定的人类群体,而且每一群体的生活形式不同于所有其他群体。正是根据这种信念,群体中个人的特征是由这个群体所塑造的,而且只有参照这个群体的习俗、语言、宗教,以及共同的记忆和社会体制才能被理解。对某种种族主义形态的民族主义而言,还要加上遗传和种族的特征。这套信念不同于那种建基于部落情感和对祖先的自豪感的古老民族感情,它是一种现代信条。对于这一信条,伯林增加了另一个特征,就民族或其生活形式而言,它是一种有机体
     民族作为一个有机体的隐喻具有多种多样的用法。一种用法强调民族存在的表面多样性
\newpage
的统一,另一种用法强调民族的目的属性,还有一种用法强调在相互依存的亚群体(“器官”)之问的有秩序的划分。然而,有机体隐喻的一种最危险的偏执狭隘的川法,至少对伯林来说,是将道德优先性统合于整体(民族),凌驾于个体之上。在这种看法中,民族是一棵大树,而个人只是树叶。那种优先性采用了这样一个形式,使得民族目标对个人而言成为一个至高无上的理由。伯林相信,民族主义运动更多地是由感情和想象而不是由理性和信条所引导的,他也就更重视民族主义的各种隐喻胜于其种种主张。他知道,整合性的犹太复国主义者令人困扰的隐喻既不是树和叶子,也不是弯枝,而是根的隐喻。
     传统右翼将犹太复国主义描述为一种扎根的计划,既是地理的又是心灵的,前者是要植根于以色列土地的神圣地理,这个民族曾在这里被连根拔起,而现在落脚在西岸领地;后者是心灵的扎根,也就是要将这个民族转变为宗教或至少是传统主义共同体的精神根源。颇为反讽的事实是,犹太人的诽谤者曾指控他们是无根的世界主义者,这一指控如今在以色列被反过来用于针对左派。以赛亚•伯林并不喜欢这种根的隐喻。他认为人类被天然赋予了头脑和
\newpage
心灵,而不是根。他担心整合主义者在他们扎根的努力中变得头脑迷失、心灵冷酷。
     然而,也许因为伯林是一个木材商人的孩子,他自己被树的隐喻所吸引。是他让弯枝隐喻为人所知,也使康德的曲木隐喻变得著名。“取自木材的东西是如此弯曲,”康德写道,“人也像由曲木制成,不可能雕刻出任何完全笔直的东西。”我想伯林非常清楚,对康德的“曲木”隐喻的释义是复杂的。在康德论及人性的曲木的那篇文章(《具有世界主义目的的普遍历史的观念》)中,他援用了另一个隐喻。他对处于孤立隔绝中的人与生活在适当的市民社会中的人做出了比较。“森林中的树木通过相互争夺空气与阳光,迫使彼此在向上生长中寻找空气与阳光,从而它们美丽而笔直地生长。相反,那终被自由与孤立隔绝中任意伸展枝干的树木,它们的生长反而迟缓,变得弯曲而扭结。”我认为康德在此对比了自然的树木与木材的反差。他所对比的是如下两者之间的区别,在社会中的自然生长,犹如丛林中的树木;而人为设计的木材制品,类似于根据一个人工计划来塑造社会。因此,“曲木”是这样一个隐喻,用来反对依照一个蓝图来塑造人类社会。
\newpage

     以赛亚•柏林激烈地反对为整个社会制作先验的推理性计划,那么一个问题就出现了,这是由斯图尔特•汉普希尔强有力地提出的,即何以协调伯林对先验蓝图的反对与他对犹太复国主义的支持,因为犹太复因主义是一种意识形态的蓝图,施于犹太人的曲木之上,以期望矫直它们。我以为,在柏林对整体性蓝图的意识形态的反对与他对犹太复国主义的接纳之间存在着真实的紧张。然而,我仍然坚持,伯林的犹太复国主义属于他的“基础”(base),不是他的“上层建筑”(superstructure),而自由主义和文化民族主义属于他的“上层建筑”。要在学理上将伯林的犹太复国主义与属于他上层建筑的各种”主义”之间做出调和是不容易的,因为它们明确地分属于他灵魂的不同层面。
     以赛亚•伯林如此多的言辞被追忆和引述,他自己却没有留下著名的遗言,这是一种命运的反讽。但就我们讨论的主题而言,他却留下了遗言。在一九九七年十月底,我收到了以赛亚的一封来信,这令我吃惊。这封信包括了一个声明,标题为《以色列与巴勒斯坦人》,声明如下:
          由于双方都开始声称,
\newpage
作为他们的历史权利,他们完全拥有巴勒斯坦,而且因为任何一方的主张都不可能在现实主义的范畴内被接受或者不造成严重的不公,因此很明显,妥协,也就是分割,是唯一正确的解决方案,这是遵循奥斯陆(Oslo)的路线——拉宾(Rabin)因支持这一路线而被一个犹太偏激分子暗杀。
     在理想中,我们呼吁的是一种友好睦邻关系,但考虑到在巴以双方,顽固、恐怖的沙文主义者都人数众多,这是不切实际的。
     这个解决方案必须多少遵循勉强的容忍,为了避免更恶化的状况——也就是可能对双方都造成不可弥补的伤害的野蛮战争。
     至于耶路撒冷,它必须留作以色列的首都,其中穆斯林的圣地享有治外法权,服从穆斯林当局的管辖,由联合国提供保障来维护这一地位,若有必要可以使用武力。
     就这份声明,伯林在另外一张纸上附加了一个私人性的注释:
        这是我的处方:对任何人有任何用处吗?如果没有——扔进废纸篓。
     我认为这个注释意味着这份声明应
\newpage
当在以色列的报纸上发表,但我需要他明确的许可。就在他被送进医院做手术(从此他再也没有康复)之前,伯林夫人曾征询他的许可。在十一月五日伯林夫人发来的传真中写道:“以赛亚说‘是’。”就我所知,这或许就是他的遗言。
     令人如此惊讶的并不是这封信的内容,而是他写下此信这一事实本身。伯林终其一生对表达政治意见都小心慎重。他相信,只有当存在一个合理的机遇能够有所作为的时候才应该采取政治行动。对伯林来说,富于表达属于艺术的领域,而行之有效属于政治的领域。政治所考虑的只是能够改变事件进程的东西。但是这里有某些不同之处:在他最后的声明中,他就是要站出来且要被有所考虑。
     伯林在讨论民族主义的过程中,因十九世纪的先知们能够如此准确地预言我们世纪的一些重要特征而倍感震动。但他们没有看到一个主要的特征:民族主义的作用。他们相信,民族主义不会发挥什么作用。在论及犹太复国主义时.曾有两种预言指向了两个相反的方向。一种是里廉布卢姆(Lillienblum)和其他几位重要的犹太复国主义者的预言,他们预测犹太人的国家将会变成一个中东
\newpage
的瑞士,在那里犹太人和阿拉伯人分别居住在分离的行政区域;另一种是“老狐狸”托尔斯泰所作的预言,据他所言,犹太复国主义将成为中东的塞尔维亚,带着侵略性与扩张性。以赛亚•伯林期望一个瑞士而惧怕一个塞尔维亚的以色列,他预计它将成为两者之间的某种形态。在奥斯陆协议之后,伯林是怀有希望的,但在他生命的最后一年,他担心“老孤狸”托尔斯泰察觉了某种重大、可怕而真实的事情,因此有了封信。

\end{document}
