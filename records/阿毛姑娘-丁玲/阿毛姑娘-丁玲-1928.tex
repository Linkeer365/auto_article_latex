\documentclass{article}
\usepackage[utf8]{inputenc}
\usepackage{ctex}

\title{阿毛姑娘\footnote{Click to View:\url{https://web.archive.org/web/20221012213453/https://www.kanunu8.com/book3/8372/186100.html}}}
\author{丁玲}
\date{1928-07}

% \setCJKmainfont[BoldFont = Noto Sans CJK SC]{Noto Serif CJK SC}
% \setCJKsansfont{Noto Sans CJK SC}
% \setCJKfamilyfont{zhsong}{Noto Serif CJK SC}
% \setCJKfamilyfont{zhhei}{Noto Sans CJK SC}
% \setlength\parindent{0pt}

\begin{document}
\CJKfamily{zhkai}

\maketitle


\Large


第一章 


一 

这是一个非常的日子,然而也只在阿毛自己眼中才如是。阿毛是已被决定在这天下午将嫁到她所不
能想象出的地方去了。 

初冬的太阳,很温暖的照到这荒凉的山谷,阿毛家的茅屋也在这和煦的阳光中灿烂着。一清早,父亲(阿毛老爹)照例就走到菜园去浇菜。但当他走回来时,看见在灶前正烧着饭的阿毛,于是便似乎在说笑话一样,而笑容里却更显露出比平日更凄凉,更黯
澹的脸:“哈,明天便归我自己来烧了。” 

\newpage

这声音在这颇空大的屋子里响着,是很沉重的
压住阿毛的心了。于是阿毛又哭泣起来。 

“嘿,傻子!有什么哭的?终久都得嫁人的,难道就真的挨着我一辈子吗?莫说养不起,就养得起
,我死了呢?” 

阿毛更是大声的哭着,只想能扑到父亲的怀里
去。 

阿毛老爹又笑着来宽慰她:“那边很好,过去后总不至象在家里这样吃苦。哈,你还哭,好容易才对着这样一户好人家呢。你怕丢下阿爸一人在这里不放心,所以哭?不要紧的,等下三姑会来替我作几天伴,阿宝哥还赖着要住在我这里呢。他也无家,愿意
来也好,就把你睡的床让给他吧。” 

然而阿毛更哭了,是所有的用来做宽慰的言语把她的心越送进悲凉里去:是觉得更不忍离开她父亲,是觉得更不敢亲近那陌生的生活去。她实在不能了解这嫁的意义,既是父亲,三姑,媒人赵三叔,和许
\newpage
多人都说这嫁是该的,想来总没有错。并且这疑问也只能放在心里,因为三姑早就示意她,说这是姑娘们所不当说的,这是属于害羞一类的事。虽说她从她所懂得的羞上面,似乎领略到所谓出嫁,不过她总觉得这事大约于她或她父亲有点不利,因为近来她在她父
亲的忙碌中,是常常得了些不安去。 

若是别人只告诉她:有那末一家人,很喜欢她,很需要她去,不久就来接她了,那末,她一定会高兴的穿起那特为她预备的衣裳,无论她是怎样爱她的老父,怎样对于这荒凉的山谷感到眷恋,但是那好奇的心,那更冀求着热闹和愉悦的心,是会使她不愿挂虑到一些纷扰的事,因为在她的意想里,对于嫁的观
念始终是模糊的,以为暂时做着一个长久的客。 

现在呢,她是被别人在无意中给与了她一些似乎恫吓的好意,把她那和平的意念揉成一种重重的,纷纷的担心,而她所最担心的日子,她的婚期,竟很快的大踏步就来了。吃过早饭,三姑就来了,还带来
一葫芦酒。 

\newpage

阿毛老爹说:“唉,这个年成,喝什么酒?我是越简便越好,所以在阿毛的好日子,我也没请客,想在后天回门时,一同吃个便饭就算了。等下只有阿
宝会来帮帮忙,其实是什么事也没有。” 

三姑是一个五十岁上下颇精明的妇人,虽说也正是从这茅屋嫁出去,然而嫁得颇好,家里总算过得去。只是未曾生下一个半个她所热盼的儿子,所以她很爱阿毛,又常常周济一下这终年都在辛勤中,还愁着难吃饱的父女。她固然很能够体贴她贫困的哥哥,不过她总觉得既然是阿毛的好日子,又只阿毛这一个
女,所以她表示了她的反抗: 

“我告你,年成是年成,事情是事情,马马虎
虎不得的。看你还有几个今天?” 

但是一想到今天,她就住了口,又自己圆转她的话:“本来,也难怪,昨天一箱衣,就够人累了。客不请,也算了,只是总得应个景。横竖是自家几个
人,小菜也现成的。橱里鸡蛋还有吧,阿毛?” 

\newpage

在她眼里看来,阿毛也很可怜,虽说她也曾很满意过阿毛的婆家,且预庆她将来的幸运,不过她总觉得连阿毛自己也感到这令人心冷的简陋。于是她拥
过阿毛来,细心的替她梳理发髻。 

其实阿毛并不如是。她是在很温柔的自己理着鬓前的短发,似乎已忘了这非常的事,在很平心的注
意听两个老人讲着许多年前的旧话。 

在吃酒的当儿,才又伤起心来,这是完全为了舍不得离开这十几年所生活的地方,舍不得父亲,舍不得三姑,舍不得莱园,茅屋,以及那黑母鸡,小黄
狗,…… 

然而总得走的,在阿宝哥来不许久,从很远很远便传来锣声,号筒声……。于是阿毛老爹就叹了一声气,走到屋外去,阿宝就忙着茶的事,三姑更一面陪着揩眼泪,又来替她换衣裳,阿毛是真真的感到凄凉在哽咽着。不久,轿子就来了。除了三个轿夫外,还跟来媒人赵三叔,和一个阿毛应该叫表舅的六十多岁的老人,他们都显着快乐的脸在恭贺着。三姑听说
\newpage
在路上还得住一夜店子,就不放心,才又商量好,让阿宝哥送一程,等黑五更轿子又动了身时再回来。于是阿毛才也又宽心些,因为那老头子;那不认识的表舅,又是那样一个忠厚的像,赵三叔也跟着去,想来
或者没有什么可怕的了。 

悄悄的又听了许多三姑叮咛的话,知道过两天还要回来的,所以只稍微又洒了几点泪,便由老父抱
上轿了。 

这走的凄凉,是只留给这两个对挥着泪的老人的,三姑便想到当日自己出嫁的事,父亲是很深的在忆念着死去多年的阿毛的娘了。阿毛的娘,也是正象阿毛一样,终年都是很快乐的操作着许多的事,不知为什么,在刚刚把阿毛的奶革掉时,就狠狠的害着疟疾了。头一次算挨过,第二回可完了。于是老人又把希望和祝福,向太阳落土的那方飘去,那是阿毛的轿
子走去了的那方。 

在轿子里的阿毛呢,只不耐烦的在想那不可知

\newpage
的一家人家的事。 


二 

其实一切她都想错了。她实在没有想出那热闹来,那麻烦来,她只被许多人拿来玩弄着,调笑着,象另外的一种人类。这时她真该来痛哭了,但她却强忍着,这是她第一次懂得在人面前所吃的亏。她只这
样想:“后天回去了,我总不会再来的!” 

这家,这才是阿毛真真的家,是姓陆,本也是阿毛同乡的人。但撒来这里,这有名的西湖边葛蛉,是快有四十年了。早先是由阿毛的阿翁划渡船来养活一家人,现在是变得很兴隆了。这个老头子,还是划着船,不过已是很漂亮的,有布篷,有钢栏,有靠背藤座的西湖游船了。两个儿子呢,就替别人家种了几亩地,其实单凭屋前的一百多株桑树,每年进款也就够可观的了。阿毛,这算来是第二的媳妇。那大的已进屋十来年了。从前是由于家计未曾很满足的热闹过,现在就大大的请客了。客大约总属于划船的人,旅馆里的茶房,账房先生,还有几个熟店铺,丝行里的,其外便是几个庙里面帮闲的朋友,以及邻居之类。
\newpage

 

客人既是如此混杂,早知道主人是不会厌烦嚣闹的,所以都豪饮着那不十分劣的绍兴酒,加以那新娘的菲薄的嫁奁,抬不起他们的敬意来,所以他们只是那样毫不以为意的来使人受窘。阿毛真觉得苦,但她知道还另外有一个人也正象她一样在受人调排,她不禁又同情着那与她同命运的人,只想把头昂起去看看,不过想起三姑的话,头是依旧垂着,垂着,不怕
已是很痛的了。 

实实在在,这使她同情过的另外那人,便是她还未曾十分领悟出的所谓丈夫,他更吓着她了。她只想能立即逃回家去,她是并未曾知道她是应该被这陌生男人来有力的拥抱住,并鲁莽的接吻。她只坚决的把身子扭在一边无声的饮泣着。那男人也就放了她,
翻身睡去了。 

一切的人都非常使她害怕,无论她走到什么举方,都带着恤怯的心,又厌恨着那每个来呆望着她的脸的人。直到又要预备回去的那天早上,她才在眉央
\newpage

上展开那蹙紧了的她的心来。 

事实自然不是象她所想出的那样简单,那样无拘无束,终于她又别了她开始才发见的福乐来。是有十多年了,自己就都是生长在那样恬静,那样自由的仙谷里吗?她好生伤感,好生哭泣(是一生所未曾有过的)的向将要离别的一切都投过去那深深的一瞥,才又随着她那很健壮的夫婿走向她所惧怕的那个家去
。 

这家的位置,是在从葛岭山门通到初阳台的路边的山坡上。屋前满植着桑树,在冬天是只剩枯枝了,因此把湖面却更看得大,白堤只是象一缕线样的横界在湖的中央。屋后是一个姓陈名不凡的“千古佳城”,后来又盖上许多类似洋式的房子,佳城便看不见了,却从周围的墙上,悬挂出许多花藤,在冬天也只显得是如丝一样的无次序。左首是通到另外几个深幽的山坳去,那里错错杂杂的在竹林中安置着几所不大的房子。右边,便是上山去的石板大路了,路旁遍植着松柏,路的那边,便又是一所为松柏遮掩不住的粉着淡湖色的房子。在界于屋与路之间,便是一条已将
\newpage
完全干涸了的小溪。这里是同样排着杭州乡下式的瓦屋三家,她的家便是最右临着溪,临着大路的一家,是既静,且美,又宜于游玩,又宜于生活的一个处所


三 

刚住下来,依然还是不安,仅仅从一种颇不熟习的口语中,都可以使她忽略去一切美处。然而时间一拖下来,也就很惯了。开始是囝囝的笑,抹去她所有对人的防御的心,这笑是如此天真,坦白,亲爱,竞好象从前家中那黑猫的亲呢的叫声了。她时时来找囝囝,囝囝又欢喜她。因为常同囝囝玩,囝囝的娘,她大嫂也就常同她来闲谈了。大嫂是一个已过三十的中年妇人,看阿毛自然只是把来当小孩看,无所用其
心计和嫉妒,所以阿毛便也感到她的可亲近。 

第二便是颇能爱怜她的夫婿了。这男子是比她大八岁,已长成一个很坚实的,二十四岁,微带红黑的少年,穿一件灰条纹布的棉袍,戴一顶半新的鸟打帽,出去时又加上一条黑绿的围巾,是又带点城市气的乡下人。冬天没有什么事,又为了新婚,得准许在
\newpage
家稍微滞留一下的,有时就整天的留在家里劈粗的树干。所以在阿毛梳头发的当儿,他也可以去替她擦一点油,在阿毛做鞋子的时候,他又去替她理线。只要是阿毛单独留在自己的小屋子中时,他总得溜进去试用他许多爱抚,起始阿毛是很怕他,不久就很柔顺的承受了,且不觉的便会很动心,很兴奋,有时竟很爱慕起这男人了。他又替她买了一些贱价的香粉香膏之类的东西,于是她在一种好报答盛情的谦虚中,很珍惜起她一双又红又壮的手来,发髻也变成一个圆形辫
式的饼。 

阿婆看见她很年轻,只令她做点零碎的小事,烧火,扫地,洗衣裳……自然是比起在家中又要锄地,又要捡柒,又要替父亲担粪等等吃力的事,是轻松得多了。所以每天她总有得空闲时候去同侄女们玩,大的侄女是在邻近的一个平民学校读书,是已在三年级的一个十岁的伶俐女孩。第二,便是不很能给她欢喜的一个顽皮孩子,小的,便是囝囝了,囝囝只两岁,时时总喜欢有人抱,一看见阿毛,便拍着手,学她
娘一样的叫着阿毛的名字,“阿毛……阿毛……” 

\newpage

邻家也是操着同样生涯的两家,阿毛在这里使得了两个很投洽的女伴。三姐便是住在她间壁的一个将嫁的十九岁的大姑娘。在阿毛的眼中,是一个除了头发太黄就没有缺憾的姑娘。人非常聪明,能绣许多样式的花,这令这新来的朋友很吃了惊的。阿招嫂是用她的和气,吸引得阿毛很心服的,年纪也才二十多一点,穿得很时款的一个小腰肢瘦的妇人,是住在那靠左边的一家。她一看见阿招嫂走往溪沟头去了,于是她也走下石级去,在用石块拦成的那小水洼中淘米,趁这时,她们就交换起关于天气,关于水,关于小菜的话来。或是一听见在屋前的坪坝上传来三姐的笑声,她也就又赶忙把要洗的衣服拿往坪坝上去洗。从三姐的口中,她是可以听到许多她未曾看见,也未曾听过的新鲜的事体。三姐说起城里来,上海来(三姐是在九岁上到过那里的),简直象一种神话中的奇境
,她揣拟都无从揣拟了。 

一到夜晚,从远远的湖上,那天与水交界的地方,便灿烂着很繁密的星星。很大的金色的光映到湖水里,在细小的波纹上拖下很长的一溜来,不住的闪耀着,象无数条有金鳞的蛇身在不动的蜿蜒着。湖面
\newpage
是静极了,天空也很黑。那明亮的一排繁星,就好象是一条钻石的宝带,轻轻拢住在一个披满黑发的女仙的头上。阿毛是神往到那地方去了,她知道那就是城里,三姐去过的,阿招嫂也去过的,陆小二,她夫婿也去过的,所有的人都去过。她不禁艳羡起所有的人来了。她悄悄的向陆小二吐露了这意思,是还带着怯
怯的心,怕所得来的是无穷的失望。 

陆小二一听到他幼小的妻的愿望,便笑着说:“没有什么可看的,尽是人,做生意的。你想去,等
两天吧,路远呢。” 

于是她小小心心的又来盼望着。到十一月尾的
一天,这希望终于达到了。 


四 

在这旅行之中阿毛所见的种种繁华,寓丽,给与她一种梦想的根据,海一个联想都是紧接在事物上的,而由联想所引伸的那生活,都一切,又都变成仙似的美境,能把人捆缚得非常之紧,使人迷醉的升沉
\newpage
到里面,不知感到的是幸福还是痛苦,阿毛就由于这旅行,把她那在操作中毫无所用的心思,从单纯的孩
提一变而为好用思虑的少女了。 

同去的人,连自己也算进去,四个人:三姐两母女,还和着大嫂的女儿玉英,因为这天是礼拜,学校放了假,也要陪伴着去玩的。阿毛遵依着夫婿的话,从衣箱中翻出一件最好看的大花格子布的套衫,罩在粗蓝布的棉袄上,在镜子里也很自诩的了。然而小二却摇着头,于是又交给三姐一块钱,是替阿毛做衣料用的,阿毛也就更高兴了。实实在在这虚荣确是小
二很鼓舞了她的。 

出去的时候,是早半天。她们迎着太阳在湖边的路上,迤迤逦逦向城里走去。三姐一路指点着她,她的眼光也就始终现着惊诧和贪馋随着四处转。玉英不时拿脚尖去蹴那路旁枯草中的石子,并慢声的唱那刚学会的《国民革命歌》。阿毛觉得那歌声非常单调,又不激扬,只是苦于不能说清那自己从歌声中得到的反感,于是就把脚步放慢了。一人落在后面,半眯着眼睛去审视那太阳。太阳正被薄云缠绕着,放出淡
\newpage
淡的射眼的白光。其外有许多地方,望去不知有多少远,不知有多少深的蓝色的天空。水也清澈如一面镜子,把堤上的树影,清清楚楚的影印在那里,而且一
动也不动。 

不怕天气已很冷,沿路上还是有不少烧香的客。那穿着老蓝布大衫,挂着大红,杏黄香袋的能走路的小脚妇人,都是那样显着乡憨的脸,大踏步的往前
赶路。 


于是三姐说:“这都是往天竺去的咧。” 

她忍不住又问天竺是什么地方,原来是几个香火非常之好的寺庙。而且到天竺去,还得走过一个更其堂皇的,甚是有名的庙,那里烧香的人更多,去玩的也多。为了香客们,游客们的需要,那儿又开了不少店铺。她还想再去问一问那庙的名字,然而已走上一道桥,桥旁矗立着一座大洋房,这是出她想象中所有的那样巍峨,那样美好。她注视的望到那悬在天空中飘扬的一树旗子,她心也象旗子一样,飘扬个不住

\newpage

她走拢那门去,是一个铁栏的门。从门隙中她想看清一切,慌张的把眼睛四处溜走,忽然,便从她脑背后响起剧烈的喇叭声,并和着重载的车轮轧轧声,把她竟吓昏了,掉过头来就想跑。但就在她前面,便冲来一辆长四方笼子样式的大车,黑压压的装满一车活的东西,擦她身前就冲上桥去了。路旁的眼光,全注到她身上,许多笑谈也投过来,她痴迷的站着在
找她的同行者。 

“啊一哟一哟—天哪,快来吧!”这声音非常熟,所以她不困难的就望见三姐她们已走到一条街市
上了。于是她走拢去,侄女玉英也嘲弄了她。 

似乎象受欺了一样,很含点悲愤,但瞬息又忘了。虽说这街市很破乱,阿毛也颇感到趣味,一手拖着三姐的娘的手,随着走,又来留心到街两旁的店铺。有些店铺中又坐满了人在喝着茶,阿毛觉得很有趣。但所有的人,又都是正如同她公公,她父亲舞着大手在谈天的一些穿老布的乡下人,所以她又忽略过去,只很艳羡那些偶尔摆在茶桌边的鸟笼,那里是关有

\newpage
不知什么名字的鸟儿,又好看,又机伶。 


阿毛想:“一定到了。” 

三姐只在唇上笑了一下,说:“才一半路呢,
就走不起了吗?不是为什么那样急于要到呢?” 

这城里好象一个神奇的,也许竞不能走到的地
方了,在阿毛是如此以为的。 

是的,在她那可怜的梦想中,不知道是怎样的把一切事物幻想得多么够人笑!只要有人去一注意那在湖滨马路出现了时候的阿毛的脸,就可知道这正是一个刚从另一世界来的胆小的旅客。什么事物也不能使她想出一个回答来!连那裹着皮大氅,露着肉红的小腿在街上游行的女太太们,她都不知这也正是属于她一样的女性。她以为那只是别人特意把来装饰起来好看的,象装饰店铺一样的东西,所以她总也把眼光追过去。实在那太好看了,那好象假装上去的如云的光泽的黑发,那弯眉,那黑眼,那小红嘴唇,那粉都都的嫩脸,一切都象经了神的手安放上去的,她并且看见所有街上人的眼光,也正在跟着那咯咯的高跟缎
\newpage
鞋走,她就越觉得城里的人聪明,在如此宽阔,热闹,阔气的马路上,会知道预备几个美丽的,活的,比鸟儿,比哈吧狗,比什么都动人的东西,来让人浏览,这图舒适的方法,不为不想得周到了。并且她疑心她自己怎么也会插足在这样的一个社会中,她欣赏这样,欣赏那样,在她是不是生来也就安排定这福气的
? 

一行人,弯弯拐拐走了几条热闹的街,她遇着许多男的女的,穿着一些她不知是什么东西做的衣服,又光华,又柔软,样子也是令人只想去亲近,又令人不敢去亲近。他们都是坐在洋车上,汽车上(这也是刚才学来的知识),在街上游行,在店铺的沉重的大门边进进出出的。阿毛这才领悟为什么城里要设着这许多店铺,许多穿粗布衣的人来服侍,自然是为的他们。这时阿毛还没有想出为什么那些人会不同,不
过立即便来了机会让她了解。 

不久,她们走进一个堆满布匹的店铺了,那些美丽得正如阿毛所艳羡,所景仰的人们身上的布匹,闪着光,一长条,一长条,竟是那样不爱惜的拖在玻
\newpage
璃窗的后面,阿毛问,阿毛知道了她也将要在这店铺中拣一段好看的布匹做衣服,为了过年穿。她是觉得什么都好,既然也可以进来由自己拣,无论在窗中拖着的,在架上堆积着的,在匣子里安放着的。三姐替她拣了一段绿色的自由布,夹着一缕缕的白条,象水的波纹一样,她欢喜得跳了,但是三姐自己拣的,却令她仿佛更喜欢。她希望也同三姐一样,然而三姐笑了。三姐说小二哥只给她一块钱,若是定要买三姐买
的假花哔叽,则要二块多了。 

阿毛本没有想到要做衣,而小二要去爱惜她,自由布本已太够她满足,但既懂得是因钱少了却得不到假花哔叽,自自然然她会忘记她夫婿的好意,并且似乎在刹那间,,她狠狠埋怨了一下那特省下别的钱为她做衣服的小二了。本来也是,引诱她去欲望,而又不能给她满足。她只是想:“为什么他不给三姐两
块多钱呢?” 

回来的时候,在第二码头,雇好了一只船。荡漾的湖水,轻轻把她们推了开去,是离这繁华的都市,一步一步的远了。她把眼睛避过一边来,大声的叹
\newpage
着气。不过快到家时,她又非常快乐了,那还是一种虚荣。当三姐和玉英教她辨识她们自己的家时候,她看见她们的家是深深藏在一个比左近都好的山洼里,且在这山洼里,隐现着许多精致的小屋。从湖上望去,好象她们的家,就正在一幢红色洋楼的屋上面。这是幸而她忘记了在这山洼里,就仅仅只她们几家是用旧的木板盖成的几家简陋的小瓦屋,而随处还须镶补着旧的,上锈的洋铁板,且满屋都堆着零星的东西,从作工,至吃饭,又到睡觉的什么破的,舍不得丢弃
的什物都在那里。 


五 

新的生活,总是惹人去再等待那更新的。阿毛生活在这里,算是非常快乐了,又忙着过年,阿毛整天帮着阿婆,大嫂,兴孜孜的做事。把父亲,三姑,一切都忘记了。一到晚上,阿婆便约了隔壁婶婶来打纸牌,她偷闲就来看,有时就躲在自己房中同小二玩。近来小二更爱她,她也更乐于接受那谑浪。有时间婆在外间里喊倒茶,而小二偏反把腿夹紧些,好看她着急。她虽说恨小二太同她开玩笑,但她越觉得要同
\newpage
小二相好了。小二的手虽粗,而放在她胸上,是一样的象有电,她就在发烧,只想把这手拿开,而身子反更贴紧小二了。什么人都觉出他们两家头很好。小二
自己也感到他的妻是一天一天更温柔了。 

过年很热闹,是她一生中所还未尝过的热闹。新年里,又由大嫂引着在庙里玩了几次。这庙就是在她们隔壁那洋房的前面,是一个很有名的玛瑙寺。寺的命名的意义,自然她是不懂得,不过那大殿的装潢,那屋宇的高朗,她是也会赏鉴的。并且那里面几个很会说笑话的和尚,几个帮阔朋友,都非常有趣。阿婆也来庙里打过牌,住在玛瑙山居(就是她家隔壁的洋房)看门的金婶婶也常往庙里去。庙里有个叫阿棠的后生,她从她的本能觉得这人也正在拿小二望她的眼光在望她。她很怕。阿棠生得又丑。不知为什么她还是欢喜往庙里去。实在庙里比家里好。仅仅就家里那瓦檐也就太矮了,好象把一个人的灵魂都紧紧的盖
住,让你的思想总跑不出屋。 

闲了时,依旧在三姐处学来许多故事,三姐又津津有味的愿意教她。不知还是三姐觉得谈讲这些有
\newpage
趣味,还是想从这不倦的言谈中暂时一慰自己对于许
多物质上的希求。 

总之,她总算是狠幸福了。而且她真的也曾觉得很快活来。不过一到春天后,不知为什么总有许多
事物把她极力牵引到完全堕入一种思想里去了。 


第二章 


阿毛从小就生长在那荒僻的山谷。父亲是那样辛勤的操作,所来往的人,也不过是象父亲一样忠悫的乡下老人,和象她自己一样几个痴傻,终日勤着做事的孩于。没有事物可以使她一想到宇宙是不止就限于在她谷中的,也没有时间让她一用她生来便如常人一样具有的脑力,所以她竞在那和平的谷中,优游的度了那许多时日。假使她父亲,她姑母不那样为她好,为她着想,嫁到这最容易沾染富贵的西湖来,在她不是顶好的事吗?在那还依旧保存原始时代的朴质的荒野,终身做一个作了工再吃饭的老实女人,也不见得就不是一种幸福。然而,现在,阿毛是已跳在一个
\newpage
大的,繁富的社会里。一切都使她惊诧,一切都使她不得不用其思想。而她又只是一个毫无知识刚从乡下来的年轻姑娘,环境呢,又竭力去拖着她望虚荣走,自然,一天,一天,她的欲望加增,而掉在苦恼的里
面,也就日甚一日了。 

在新年里面,本是很快乐的,所接触的一些人物,也使她感到趣味。当然,她是只看到那谦抑,那亲热,那滑稽,而笑脸里所藏住的虚伪和势利,她却无从去领解。所以她终日都在嘻笑中,而带着热诚去亲近所有的人,连从前曾一度很扰着她的那城里的繁
华都忘掉了。 

直到有一天,天气不很冷,温和的阳光正晒在屋前院坝里。她和大嫂在那阳光处黏鞋底,三姐,阿招嫂她们也各自搬着小椅在屋外作活。几人谈谈笑笑的,也很不寂寞。大嫂又时时把她黏好的鞋底拿给别人看,大家又来打笑她。她是非常愧惭,很悔从前不
学好这针线,现在是全亏了大嫂来教她。 

正在说话很有劲的三姐,忽的把话打住了,阿
\newpage
毛看见她在怔怔的望到外面。阿毛也就掉过头来,原来从山门外已走进两个人来。那穿皮领的,那阿毛从前所看见过的美人儿,正被夹在一个也穿有皮领的美男人臂膀间,两人并着头慢慢朝山上走。于是:阿毛又随着三姐走到挨溪沟的这头,等着他们。终于他们也来了,他们是那样华贵,连眼角也没有望到她那边,只是那样慢慢的,含着微笑的一步一步,两种皮鞋谐和着响声往山上踱。不知那男的说了一句什么话,于是女的就笑了,笑得是那样大方,那样清脆。柔嫩的声音,夹在鸟语中,夹在溪山的汩汩中,响彻了这山坳,于是连路旁枯黄的小草,都笼罩着一种春的光辉。笑完了,又把两手去互相抚弄那双玲珑的小手套。于是这手套,在阿毛看来,就成了一种类似敬神的无上的珍品。阿毛一直送着那后影登了山后,才怅怅的回转头来。阿毛看见三姐同样也显着那失意的脸,
并且三姐又出乎她意料的做了个非常鄙屑的样子。 

回到原位时,大嫂和阿招嫂正在谈讲那些时款的衣式。阿招嫂劝大嫂作一件长袍出门时穿,而大嫂称说她年纪已太大,不愿赶时兴。于是阿招搜又说阿毛顶好做一件。阿招嫂又夸说阿毛生得倒很体面,加
\newpage

意打扮起来,是顶不错的。大嫂也笑了她几句。 

从此,阿毛就希望得一件长袍。其实她对于长袍和短衣的美,都不能分明的看出,只觉得在别人身上穿起总是好看的,阿招嫂既说长袍是时兴,那自然
长袍比短衣好了。 

并且,那女人的影子,那笑声,总在她脑子中晃。她实在希望那女人再来一次,让她好看得更清白点。她实在想懂得那女人到底是做什么的,就是说她要知道那女人的生活。她常常想,既然那笑声是那样的不同,若煮着饭,坐在灶门前拿起火钳拨着火时,不知又是将如何的迷人了。但是她立即就否认了。别人那样标致,那样尊贵,怎么会象她一样终天坐在灶门前烧火呢?于是她又想起烧火的辛苦,常常为去折断那干树枝,把手划破,并且那矮凳的前前后后,铺满着的脏茅草,脏树叶,把自己的鞋袜都弄得不象样了。阿毛是简直忘掉从前赤着脚在山坡上耙茅草,而两寸来长的毛虫也常常掉在她的颈上,或肩上的往事
了。 

\newpage

不久,阿毛所希望的事,就慨然的来了,并且
还超乎她所希望的,实在她应从此得到快乐了! 


许多人都沸沸扬扬,金婶婶一早就跑过来报消
息。阿招嫂说:“看样子很有洋钿呢!” 


“上海来的吧?”三姐很迷乱的发着话。 

阿婆似乎降临了什么好事一样,眯着眼向金婶婶笑:“你们今年一定可以多赚几个酒钱了。去年住
的那和尚,很吝啬吧?” 

“是的,外面人手头大方多了呢。昨天看妥房子,知道我们是看门的,一出手就给了两块钱,说以后麻烦我们的时候多着呢,说话交关客气。转去时又坐了阿金的船,阿金晚上转来,喝得烂醉了,问他得了多少船钱,他只摇头,我总想至少也给了半块。早上我们还说,可恨上面住的黄家同老和尚又不搬,不然换几个年轻人来,好得多了。只有师宾师父还算比

\newpage
较好些。” 

金婶婶这一番话,把个个人脸上都加了一层艳羡的光,都想到那两块钱去了,心也发着热。于是阿婆和三姐的娘又都拜托金婶婶,以后有生意,请也照顾点。金婶婶是俨然贵客一样又在这里坐了一个钟头
,大家都不敢怠慢的陪着她。 

一吃过早粥,在玛瑙山居的大门前,陆陆续续就出现了许多人,扛着箱笼的,抬着桌椅的。阿毛快乐癫了,时时偷着跑到金婶婶家去瞧。直到下午二点多钟了,那穿蓝竹布袍的年轻听差的东家才坐了洋车来。阿毛认得她,那就是她所渴于欲一再见她的美人,那男子也正是那陪着她来玩山的一个。不过这次她的衣服又换了一件,依旧是皮领,高跟缎鞋,然而却非常和气,一进门就对金婶婶一笑,看见戴破毡帽的阿金叔,也点着头。阿毛觉得金婶婶是也可爱了,仰慕的去望她,而在这时,那和善的眼光,带着高兴的微笑的眼光,又落到她自己脸上。于是阿毛脸红了,心跳跳的反不敢再去望人。那女人呢,也就接过一根很玲珑的棍子,是她丈夫给她的,一步,一步的踱上那通到小洋房的曲径去。那步法的娉婷;腰肢微微摆
\newpage

动的姿态,还是象那天游山时一模一样。 

阿毛很想再随着走上去瞧瞧,又觉得非常气馁
,无语的便退回家来了。 

那久闭的窗,已打开了,露出沉沉垂着的粉红的窗帷,游廊上也抹拭得非常干净,放着油漆的光。

一到夜晚,刺眼的电灯光便射放过来,阿毛站在屋外,可以从窗帷里依稀看见悬在墙壁上的画,或偶尔一瞥的头影。阿毛想知道那里面的人在做些什么,常常一人屏息的站着听。可是都寂然。直到有一夜,是夜深的时候,阿毛被一种高亢的,悲凄的提琴声所惊醒。阿毛细细的听,识出这正是从那二对刚搬来不久的新邻居所发出的,阿毛听到那琴声直想哭了。她悄悄的踱到屋外来。然而那声音却又低沉下去,且戛然便停止了。瞬即灯光也熄了,一切又都寂静得可
怕。 

阿毛真想不出那声音是从什么东西上所发出,而那年轻夫妇为什么到夜深还不睡,并弹弄出那么使
\newpage

人听了欲哭的歌调来。阿毛更留意到间壁了。 

是有着明媚的阳光的一天,阿毛正在溪沟头清洗衣服,忽然听着一种声音,好象就从自己头上传来的一样,于是阿毛又跑上沟边的高岸。她看见那女人裹着一件大红的呢衣,把上身倾在栏杆上面,雪白的手腕就从红衣的短袖中伸出,向下面不住的挥着,口中不知在说些什么,又是那样的笑。而从玛瑙山居的门边,就转出几个同样的女人来,尖着声音在向上回报。这使阿毛恍然,原来那也并不是什么希奇的东西,也许有着成百成千在她们那社会里,就如同在阿毛的这社会,也就有着不少的正象阿毛,正象三姐的人
在。 

并且天气一暖和,山色也由枯黄而渐渐铺上一层嫩绿,所有的树都在抽着芽,游山的人一天多似一天了。而来玩的,多半总又属于正象她邻居一流的人,这使得阿毛非常烦闷。纵然她懂得是由于她的命生来就不能象那些人尊贵,然而为什么她们便该生来命就不同,并且她们整天到底在享受一些什么样的福乐,是阿毛日夜都不安,把整个心思放在这上面的来由
\newpage



去年的十月,是阿毛嫁到这里来,而现在才二月,这几家人家又忙着要吃第二场喜酒了。日子是选在清明那天把三姐嫁到城里去。三姐虽比阿毛嫁时更懂得离别的悲苦,时常牵着别人的手哭,然而在她脸上,却时时显着比她妈还焦急,默默的又隐藏不住那高兴的笑。三天,两天,母女俩又进城买衣料去,打首饰去,所有的人都看得出那两颗心也整天盘旋在热
闹的街市里,早就不安于这破乱的瓦屋了。 

三姐嫁得很阔气,在朋友中,邻居中很骄傲的就嫁到婆家去了。原来新郎是一个国民革命军中的军爷,新近发了点小财,而又似乎被神捉弄了一样,有一次逛湖,坐了三姐爸爸的船。凑巧那天三姐进城去转来,也一同坐着走了一程。那军爷本有老婆的,但却很看上了三姐,又欺着三姐爸爸的职业低,敢于开
口要,谁知三姐一家人就都非常高兴的答应了。 

等到三姐再回来,已变得不再是从前的三姐了。穿着一件闪光的肉红色花长袍,一双挖花皮鞋,虽
\newpage
然不是高跟,但走路时样式,也随着好看多了。特别是连髻子也剪去,光溜溜的短发,贴在头上,并垂在鬓旁,而且那意气,是比什么都变得使人惊诧。她不再同阿毛她们随意说笑了。走的时候,还同阿招嫂闹了点小气走的。三姐的娘也觉得阿招嫂竟敢开罪于她女儿,是可气的事,女儿走后,又数说了阿招嫂几句。大嫂则属于同情阿招嫂一边,借着毫不懂事的囝囝
笑着说: 

“好宝贝,你要安分些,你娘是不得靠你卖给
别人做小老婆来过活的。” 

阿招嫂也不时投出那带刺的话,不过在三姐第二次回来时,她们又都非常艳羡的同三姐很要好了。

只有阿毛是不能了解为什么别人要轻视她,同时又趋奉她。阿毛只觉得三姐已更可爱,而且是跑到比她自己很高的地方去了。她把三姐的骄矜,看得很自然。那比三姐穿着得更好的女人,不是更显得骄矜吗?她并且想,如若她得有三姐的那些好衣服穿,那她的气概,将也会变成三姐那样了。所以她始终都非
\newpage
常敬重三姐,还特别敬重那来曾见过面的三姐的丈夫。三姐又不倦的欢喜讲着他,那军爷的一些轶事,那轶事一到了三姐会说话的口中,就变成许多有趣味的事了。并且那主人翁似乎是一个神奇的人,一个十足
的英雄了。 

阿毛虽说很天真,但她却常常好用她的心思,又有三姐,阿招嫂等的教诲,所以也就早不是从前的阿毛了。这算是她唯一的损失。她已懂得了是什么东西来把同样的人分成许多阶级。本是一样的人,而竟有人肯在街上去拉着别人坐的车跑,而也竟有人肯让别人为自己流着汗来跑的。自然,这使他们不以为羞的,都是因了钱的缘故。譬如三姐近来很享福,不就是因为她丈夫有钱的缘故吗?再譬如那些来逛山的女太太们,不也是因为她们丈夫或者爸爸有钱,才能打扮得那么美吗?那末,自己之所以丑陋,之所以吃苦,自然是为的自己爸爸自己丈夫没有钱的缘故了。从前还能把这不平归之于天,觉得生来如此便该一生如此,在这把命运看为天定中,总还可以消极的压制住那欲望。然而现在阿毛不信命了。现在她把女人的一生,好和歹一概认为系之于丈夫。她想:若是阿招嫂
\newpage
不是嫁给阿招哥,而嫁给另外一个有钱的人,那她自然不必怀着妊还要终日操作许多事。假设三姐不给军爷去做小,而嫁到她生长的那山谷去,那三姐还能骄矜些什么呢?再譬如自己不是嫁给种田的小二,那总也该不至于象这样为逛山的女太太们所不睬,连三姐
也瞧不起的穷人了。 

当她一懂得都是为了钱时,她倒又非常辛勤的
做着事,只想替她丈夫多帮点忙才好。 


是养蚕的时候到了。阿毛从没有看见过,也没有作过这等事,不过她却比所有的人都高兴。阿婆本来只愿孵两张的皮纸就够了,但因了阿毛的劝说,也就孵了三张。从清早起来,到睡觉,都是阿毛在那里
换桑叶。公公还说:“这孩子倒不懒呢!” 

阿毛对小二是比以前更温柔了;总承着他的意思去做事。谁料得定小二将来不发财,不把他老婆打扮起来呢?阿毛总幻想到有那末一天,也许小二做了军爷,也许小二从别的方面发了财,那她就可以把这
\newpage
双常为小二亲着的手,来休憩着。或者也去做点别个有钱女人所做的一些事。想来那事体也一定各如其衣饰一样的恰合身分,那一定非常有趣。而小二呢,小二是做梦也不曾知道正有人把火样,无限大的希望来在他身上建筑,且越堆积得高起来。他是整天都和着大哥无思无虑的跑到十里路外的田地里工作,看到太阳下山了,便又扛着锄头走回来。回来后,吃完饭,洗了脚,就快是睡的时候了。他连同阿毛玩都没有时间,也振不起心情,那里得知他妻的耐苦的操作中,
会压制得,有极大的野心? 

其实阿毛真可伶!什么人——就是连她自己也决不会懂得,当她打起精神去喂蚕,去烧饭洗衣的那
种想从操作中得到自慰的苦味! 

阿毛已经消瘦了好多。大嫂总喊她歇一会儿吧,莫做出病来,她却总不愿住手,似乎手足一停止工作,那使她极感到焦躁的欲念,就会来苦恼她。她又认为这富贵之来,决不是突如其来,一定要经过长久
的忍耐的。 

\newpage

一到夜晚,小二倒头就睡熟了。于是阿毛在黑暗中张着两眼,许多美满的好梦,纷乱的便来挤着她的心。有时想得太完全了,太幸福了,忍不住便抱着小二的脸乱吻,或者还吻在他身上,觉得那身体是异常热,自己也就发起烧来,只希望小二会醒来同着她玩一下,就仅仅用力来抱她一下,她不也就更可以象真的已尝着那福乐了吗?有一次,她实在忍不住了,推了几下都不醒,她就去拨那眼睛皮。小二是醒了,
但立即在她光赤身上打了一下,并骂着说: 


“不要脸的东西,你这小淫妇!” 

这能怪小二吗?小二是整天走了那么多的路,做了那么多的事,是疲倦使他躺下来的。而在他自己,一个正在年盛力强的男人,他又是那么喜欢阿毛的,岂有不愿去讨好阿毛,而让阿毛感到不满?譬如有几个夜晚,他被阿毛转侧的声音所扰醒,而他就抱过阿毛来,阿毛温柔的身体又鼓舞了他,他不觉就在他
妻面前很放肆了。 

若是阿毛是真的感到需要这性的安慰,那阿毛
\newpage
自然会很有精神的来回报小二了。但阿毛却又觉得小二是欺了她,可是她又不反抗,因为太忍受了,反更觉得伤心,这是当小二醒时,也许她正又在想到失意
的事在很灰着心呢! 

小二看到她冷淡,也无趣,有时又要骂着她几
句。 

并且常常当她一向他说起种田不好时,他也要骂她癫。他问她到底要做什么事才好,她又答不出话
来。 

小二纵不必定要有那远大的志愿,而象他妻一样,是只企望在有那末一天也会被人看得起些,但总也该特为他妻生出一种超乎物质的爱来。这样,或者那正在苦咬着欲望的焦愁的心,会慢慢从另一方面得到另一种见地,又快快乐乐的来生活也可能的。然而小二是一个种田的人,除了从本能的冲动里生出的一种肉感的戏谑和鲁莽,便不能了解其余的事,连想使他能变得稍微细致点,去一看他妻的不好言笑了的脸,他都不会留心到与在新婚时有什么变异。自然,在
\newpage
这情形下,已成为一个有贪欲的他的妻,竞从此把他
推远了去,是可能的事。 


阿毛真的对于小二就起了剧烈的反感吗?不呵,无论她在她那种阶级中,那已是一个勇敢的英雄,不安于她那低微的地位,不认命运生来不如人,然而她却并不真真的认识了什么。她只有一缕单纯的思想,正如许多女人一样。她的环境告诉她不能恨丈夫,所以她依旧常常受人蹂躏,同时又因为她不了解人们定下的定义,背叛了丈夫去想到别的男人是罪恶,所以她又在不知不觉中落在那更其不幸的陷网里,而其
不幸是更苦恼了她。 

早先她把所有的希望都建筑在小二身上。这根据可以勉力使她去忍耐做她已有怨懑了的事。但是,慢慢的,她便觉得这希望是比梦还渺茫。而且小二一点也不能鼓起她再有此希望于他的心。这根据既失了凭藉,她自然是深受到那失望的苦绪,而对于一切,又都彻底的灰起心来。现在是鸡生了蛋,也没人管,蚕子正在上山的时候,而桑叶总换不及。阿婆和大嫂
\newpage
几乎整天都在竹箔边,饭又弄得潦草,屋子又脏,所有的事都失了次序。有天晚上阿婆实在生气了,大声
嚷着: 

“别人养了儿子享福,我就该命苦,还要服侍
媳妇!” 

公公也知道是骂给阿毛听的。公公又不知道阿毛真懒散得怕人,只看到许久都是很勤快的,而忽然又那样骂着人,反替年小的阿毛有点不平,所以他淡
淡的说: 


“阿毛!你假使有了什么病,你就说吧!” 


阿毛仍然懒于去回答。 

“哼!病!在我们家很有着人去娇宠的小娘子,怎么不会有病!既然是那样娇嫩,就躺着去吧,横坚有人来孝敬的!哼!到底是害了什么病——莫不是
懒病?”阿婆一口气说完了,又打着冷笑。 

\newpage

正在洗脚的小二,觉得母亲好象连自己也很着了恼似的,并且自己不来理这事,也决不会就停止的
了。他讨好的也大声的嚷着: 


“妈啦个B,不做事,就替我滚回去!” 

阿毛把眼张开来望了她丈夫一下,又把眼阖下
来。什么地方都于她一样,她想,回去也成的。 

不过阿毛并没有回去,也许这又是错。不久阿毛又犯着从前的老病了,而且更甚,一没有事,就忽忽忙忙的站在屋外,看在山路上上下下的人。她左边那高处的房子里。也搬来两家象她右邻的人。他们进出又得走过她院坝,她常常等在那路口边去仔细看。现在她只看那衣饰了,她已不甚注意那脸蛋,觉得倒是走路时的姿态,反惹人爱慕些。所以在晚上,在黑的院坝里,她常常踮着脚尖去学,觉得似乎很象了,她就更不安。为什么自己就永该如此?阿拇嫂曾告过她,那些女人都是在学校念过书的。但阿毛一想,横竖也一样,未必她们念过书,就会不同于自己。未必她们会欢喜穿粗布衣,烧茶煮饭,任人看不起?未必
\newpage
她们也不会只希望嫁的丈夫有钱而自己好加意来打扮?并且阿毛也不自量;阿毛不懂得所谓书是如何的难念,她以为如若她有钱,她自然也会念书,如同她也
会打扮一样。 

现在她把女人看得一点也不神奇,以为都象她一样,只有一个观念,一种为虚荣为图快乐生出的无止境的欲望,这是乡下无知的阿毛错了!阿毛真不知道也有能干的女人正在做着科员,或干事一流的小官,使从没有尝过官味的女人正在满足着那一二百元一月的薪水,而同时也有着自己烧饭,自己洗衣,自己呕心呕血去写文章,让别人算清了字给一点钱去生活,在许多高的压迫下还想读一点书的女人——而把自己在孤独中所见到的,无朋友可与言的一些话,写给世界,却得来是如死的冷淡,依旧又忍耐着去走运一条已在这纯物质的,趋图小利的时代所不屑理的文学
的路的女人。 

若果阿毛有机会来了解那些她所羡慕的女人的内部的生活,从那之中看出人类的浅薄,人类的可怜,也许阿毛又非常安于她那能忠实于她的生活的一切
\newpage

操作了。 

阿毛看轻女人,同时她就把一切女人的造化之功,加之于男子了。她似乎是这样以为;男子的好和歹,是男子自己去造成,或是生来就有一定。而女人只把一生的命运系之于男子,所以阿毛总是那样想:“假设他也正是属于那一流穿洋服,拿手棍的人,就
好了。” 

然而这希望是无望,阿毛也早就不再去希望了的,所以她现在只是对于每天逛山的男人,很细心的去辨认,看是属于那一类的男人,而对于那穿着阔气的,气概轩昂的,则加以无限的崇敬。至于女人呢,她已只存着一种嫉妒,或拿着来和自己比拟,看是否应不应有那两种太不相等的运命。慢慢的,她就更浸
在不可及的幻梦里了。 


六 

白天,她常常背着家人跑到山上游人多的地方去,不过从始至终永久都没人去理睬她。她总希望有
\newpage
那末一个可爱的男人,忽然在山上相遇着,而那男人就爱了她,把她从她丈夫那里,公婆那里抢走,于是她就重新做起人。她又把那所应享受的一切梦,继续的做下去。她又糊涂,又少见识,所想的又脱不了她所见的一些根据,有时竟想出许多极不相称的事。然而她依旧在山上走,希望凭空会掉下什么福乐来,或者不意拣到一个钱包,那里面正装得有成千成万的钱,拿这钱去买地位,去买衣饰,要怎样,便怎样,不也是可能的事吗?但那钱包似乎别人都抓得极紧,而葛岭上也决不会有金窖银窖等着阿毛去挖。因之,阿毛失意极了,也辛苦极了,反又兴奋着,夜晚长久不能睡,听到枕畔的鼾声,更使得她心焦。性子不觉的也变得很烦躁。譬如,阿婆骂了,就乘机来痛哭,怄了一小点气,总要跑到院坝里大柳树下去抹泪,连公公也看不过,常常叹息。侄女们看见她没有一点喜悦相,也不去惹她。大嫂总嫌她懒,跑到隔壁家去数说。三姐再也不转来了。就是三姐转来,不也只能更给阿毛一些不平吗?阿毛是除了那梦幻的实现,什么也
不能给与她的需要。 

那梦幻,终于来到了,但于阿毛是得的什么呢
\newpage


一天,阿毛正穿一件花布单褂在垸坝里迎风坐着,那黑儿就汪汪的吠了起来。转过身来,阿毛正看见间壁洋房的那一对还和另外一个颇高的男人,从溪沟那边越过她这边来。她于是就站起身来看。那女人,只穿一件长花坎肩的女人,举着那柔嫩的,粉红的手膀,就朝阿毛摇了起来。阿毛不知那另外又送过来的笑脸是什么意思,心悸怦的跳,脸就红了,也不知
怎样去回报才对。 

三个人很大方的就走上她坪坝了,并朝她走来,她起先非常怕,看着几个异常和气的脸,也就把持
住了。 

“你姓什么?我听见别人叫你做阿毛,阿毛是
你的名字,是不是呢?”女的那个更走近了她。 

两个男人在互相说着阿毛连一个宇也不懂的话


阿毛脸红红的点了几下头。 

\newpage

女的继续又来问着她的家里人,和她的年纪。

阿毛只觉得那两对正逼视到自己浑身的眼光的可怕。阿毛想躲回屋子里去。忽然她又想到莫非那男子,就是她所想象的那个,于是她心更跳了。她望了那人一眼,颇高,很黑,扁平的脸,穿着的却非常讲究。阿毛眼睛似乎正有着什么东西在烧着一样,焦痛得又垂下来了。她这时只想就随着那人跑去就好,假设那人肯递过一只手来的话。时间在她似乎非常走得慢了,她担忧着,深恐她会被什么人瞥见了会走不成。其实阿招嫂就在门边瞧,囝囝还在院坝那端玩。而
阿婆这时也看见了。走出屋来就喊她。 

她一听到喊声,就又朝那男人望了一下,好象含了无穷的怨怼一样。那女的呢。却反走在阿毛前边,在同阿婆招呼。阿婆也笑吟吟的走了拢来。阿婆又令她搬几张矮椅来给客坐。两个男人也同阿婆说得很
熟了。 

闲话说了半天,那女人的机伶丈夫望了阿毛一

\newpage
眼,才又向阿婆说 

“我们想拜托你一件事,希望你总要帮到这个
忙……” 

“总要竭力的,请说是什么事吧!”阿婆不等
别人说完,插着来说话,显然很有兴味的样子。 

那人又踌躇了一下才又接着说下去,其余两人
都含着微笑在听他说。 

“这位先生,”手拍了一下那黑高个儿,“是住在哈同花园,是国立艺术院的教授,是教学生画画的。现在他们学校想请一个姑娘给他们画,每月有五十几块钱。这事一点也不要紧的,没有什么难为情。我们觉得这位姑娘就很好,不知你们肯不肯答应?”

阿婆脸色变得很快,但又为了在阔人面前,依旧又装着笑,说是阿毛有丈夫的人,怎么能是他们又解释那做那样营生。于职业,且保证说那里的人都是
规矩不过的。 

\newpage

阿毛自己是什么也不懂,只以为那男人一定是爱她,才如此说,听说又有钱,更愿意。及看见阿婆总不肯,心就急了,并且那几人觉得既无望,站起身
也就预备走,阿毛忍不住就叫了起来: 

“我要去的!我要去的!为什么不准我去?”阿婆一掌就把她打在地下了。当她抬起头时,她还看
见那男人最后投给她一个抱歉的眼光。 

连夜小二也非常咆哮的打了她,公公也骂,所有的人又故意给她看一些轻视的眼色,阿毛哭也不哭
,好象很快乐的挨着打。 


七 

这能说她是一生来就是如此温柔吗?恐怕光靠性情不会撒赖,未必就能如是忍耐那接连落在身上的拳头。她实实在在咬着牙齿笑。有那末一种极蠢的思想正在鼓舞她去吃苦呢,她总觉得拳头越下来得重,她的心就跑去得越远,远到不可知的那男人的心的处所去了。并且这痛也好象是正为了那欢喜自己的男人
\newpage
才身受的,所以倒愿意能多挨几下也好。而在第二天,天还没亮的时候,她又唤起她的希望,朝山上跑去

一口气就跑上喜雨亭。山上一个人影也没有,鸟儿还很安静的睡在窠里。湖面被雾气笼罩着,似一个无边的海洋。侧面宝石山的山尖,也隐没在白的大气里。只山腰边的丛树间,还依稀辨出是正隐现着几所房屋。阿毛凝望着玛瑙山居的屋顶,她把所有的能希望的力,都从这眼光中拂去。她确确实实在夜深时候;还听出他们所传出户外的笑声,而她又断定那笑声中是正有一个声音是她所想慕的那高大男人。她等着他来。她在喜雨亭呆等了许久,而他竞不来。雾气已看看快消尽了。白堤已迷迷糊糊在风的波涛中显出残缺的影。于是她又向绝顶跑去。她似乎入了魔一样,总以为或者他是已先上去了。既至跑过抱朴庐,又到炼丹台,还不见人影。她已微带了失望的心情,慢慢又踱上初阳台。初阳台上是冷寂寂的,无声的下着雾水,把阿毛的头发都弄潮湿了。这里是除了十步以外都看不清,上,下,四周都团团围绕着象云一样的东西。风过处,从云的稀薄处可以隐约看出一块大地来,然而后面的那气体,又填实了这空处了。阿毛头
\newpage
昏昏的,说不出、那恐惧来,因为非常之象有几次的梦境,她看见那向她乱涌来的东西,她吓得无语的躲在石龛子里,动也不敢一动。正在这时,她仿佛又看见那路上,正走来二个人影,并且象极了她所想望的人,于是她又叫着跑下去,然而依然只有大气围绕着她。她苦恼极了,她疲惫极了,却还打着勇气从半山亭绕到赤壁庵。庵里蹿出两条大黄狗朝她乱吠,她才又转到喜雨亭。到喜雨亭时,白堤已显出在灰色的湖水里,而玛瑙山居的屋顶是更清晰的,又被许多大树所遮掩的矗立在那路旁的山嘴上。她看着那屋顶又伤
起心来,而且哭得很厉害,大声的抽咽着。 

她想起昨夜的挨打,她不知这打是找不到偿还的。她很恨,又不知恨谁,似乎那男人也不好。而阻碍她的是阿婆,是所有人,实实在在确是小二阻碍了她。如若她不嫁,那自然别人不能藉口她是有丈夫的人而拒绝别人,她真有点恨小二了。她又无理由的去恨那男人,她为他忍受了许多沉重的拳头,清脆的巴掌,并且在清晨,冒着夜来的寒气;满山满谷的乱跑,跑得头昏脚肿,而他,他却不知正在什么地方睡觉呢。既然他并不喜欢她,为什么他又要去捉弄她?现
\newpage
在她是不知怎样来处置自己了。当她趁着一点点曙光跑出家门来时,她是没有料到她还该带着失望和颓丧又跑转家门去的。但是无论如何她总不能便留在这山上而不回去。假使竟象她所想的,那男人便在这有着
浓雾的清晨而把她带走不是顶好的事吗? 

雾还没向山顶退完时,纷纷的细雨就和着她的泪一同无主的向四方飘。葛仙祠的老道士在这时趿着草鞋下山来了,是往昭庆寺去买豆腐的,看见阿毛坐
在石磴上不住的哭,就问: 

“一清早,什么事跑到这里来哭?小心受凉了
,要病的!” 


阿毛觉得有人正在可怜她,反更伤心了。 

道士等了她半天,不见她答应,而且哭得更有滋味一样的,便手套着竹篮,从石级上又走下去,口
里一边说: 


\newpage

“好,我去叫小二来。” 

“求你!不要说,我马上就回去。”她跳起了,一把抓住了那道士。看见他已点了头,自己才向山下蹿去,但立即又转过身来,加上一句叮咛:“青石
师父!求你呵,不要说起这回事吧” 

于是她一边拭着泪,一边连跑带跳的回到家里去。小二问她到什么地方去了,她说到厕所,砰的一下,小二又打了她:“你这娼妇,又扯谎!我就刚从
厕所来。” 

她不做声,转到厨房去煨早粥。打开厨房的侧门,她看见隔壁那粉红窗榷还没掀开,依旧静静的垂
在那儿。 


第三章 


自从这次挨了打后,阿毛就不再挨打了。虽说阿婆还是不快活她,却找不出她的错处来。小二有时觉得她近来更其沉默了,又瘦得可怜,想去问问她是
\newpage
否有病,而又为她的冷淡止住了。说恨她没有讲话,又说不出口,所以小二只好也默着。常常当两夫妇单独在一块,阿毛就装睡着。小二也知道,有时受不了那静默,就站起身走到院坝去。在阿毛自己看来,或是在什么人跟中看来,她都太够柔顺了。然而在家庭的空气中,总还保留着一种隔阂,如同在平地上的一道很深的沟。就是说无论阿毛怎么在耐心的操作,那耐心却只能表白出她的心的倔强,而阿婆,大嫂……一切人都看出那倔强的心,是跑得离这家非常之远了

其实在她自己呢,她是不愿再计较到这些事了。她也不再希望,她觉得一切都无望。她想:“也好,就如此过一生吧!象我一样的命运,未必会没有!
” 

然而她却并没有就不再继续她的梦幻。从前在这梦幻中是紧咬着一颗跳跃的心,极望她梦幻的实现,现在呢,现在却只图能在梦幻中味出一点快乐的甜意,作为在清醒时所感到的悲凉的慰藉就算了。但在夜静后,所现出的一丝笑意,能抵得从梦境里醒来后的一声叹息吗?那萦回流荡在黑暗的寂寂的小房中的
\newpage
叹息,使得她自己听来都感到心悸,而又流着泪,她
自己也不懂为什么那叹息会发出那样悲凄的音。 

无论什么人都是如此,在一种追求中去生活,不怕苦恼得使你发颠,然而这苦恼却在另一方面又含有别一种力去安慰你那一颗热中的心。只是象这种,象阿毛一样,只能在无人去扰搅她时,为自己愿意找点可以暂时麻醉那悲苦的心灵,便特意使自己浸沉在一种已认为不必希望的美满生活的梦境里,真是想不
出补救的可怜! 

阿毛偶尔也一望那对屋的人,常常穿一件大衫在游廊喂鸟食的女人,不过瞬间她就掉转眼光来,似
乎怕看见什么可以刺痛她心的事物。 

更其使阿毛不愿常见的,还是住在阿毛左边山坡上的一个苍白脸色的年轻姑娘,她常常斜倒在一个世界上最和善的美貌男人的臂膀里,趿着一双嫣红拖鞋,在碎石的曲折的小径里,铿铿锵锵的漫步到阿毛她们的院坝边,站一会,或者坐在路旁的岩石上。两人总是那样细细柔柔的谈谈讲讲,然后又拥着,更其
\newpage
悠悠闲闲的走回去。并且几乎每天她和他都要并坐在一张大藤椅里,同翻着一本书,或又谐和着高低音在共唱着一首诗歌。也许阿毛是由于觉得她是太幸福了,所以怕看见她,怕看见了她,会相形出自己的不幸来,又感到伤心,阿毛总也愿意自己能快乐点才好。其实,那女人却正感到比阿毛更其应该的难过,因为她的肺病是很重了。不过在阿毛眼中看来,即使那病
可以治死她,也是幸福,也可以非常满足的死去。 

阿毛不愿出去玩,怕看见一些足以引自己又陷在无望的希望的悲苦中去,阿毛也不愿和家里人以及阿招嫂等谈讲,怕让自己更深切的懂得她自己也正是确定属于她们那阶级的人,并且还常觉出她们的许多伧俗处。所以她终日埋着头做事,做完事,就呆坐着,或呆躺着,简直不象从前终日都徜徉在这里,或又
躲躲藏藏的在那里了。 


阿毛病了,她自己不知道,她不知道她发青的脸色比那趿着拖鞋的女人的苍白还来得可怕。她整夜的不能睡,慢慢的便成了习惯,等到灯一熄,神志反
\newpage
清醒了。于是又恣肆的做着梦去。天亮时,有点觉得疲倦了,但是事情又催促她起来。她不愿为了这些又去让阿婆骂她懒,她又并不觉得那些操作会有什么苦,有时又故意让柴去划破自己的手,看那红的鲜血一颗一颗的冒出皮肤来。又常常一天到晚都不吃一口饭。有天小二实在忍不住了,就问她,辞色之间是非常
现着怜惜的样子。 

没有人去理会她,她也并不知道有病,但一有人去体惜她,她就又觉得真的已病得很深了。因为太悲痛了自己的得病,便又似乎应该去怨恨许多人,这病总不是她自己欢喜它而寻找得来的'她看着小二那
忠厚的脸就怪声的笑起来: 

“放心!我不会马上就死去的!”她那直向小二射去的两道眼光,却明明是显出那怨毒的意思,而且话也是如此话:“放心!总有一天我就会死去的!

她自己毫不思量的把话乱投过去,小二自然正如她所愿的感出那话的锋芒了。而她自己就会好过些吗?当她未曾说话以前的心境,也许还平静点,为了
\newpage
那言语进出得那样伤心,又加上从空气中再传来那音调的抖颤,反把那种本不甚凄怆的情调,更加浓了。她好象真的又觉得没有一个人不乐意她死的。而这病就是所有一切人的对于她的好意,她忍不住又要哭,
垂下头去抚弄那短衫的边缘。 

小二本是一番好意问她,得来的却正是相反的恶笑,心也恨了,只想骂她,又看见她那低着头默坐着的样子,显得也很可怜,便制住他自己的怒气,大
踏步跑出去了。 

如果小二能懂得她的苦衷,跑过去抱起她来,吻遍她全身,拿眼泪去要求,单单为了他的爱,去山珍惜她的身体,并发出千百句誓言,愿为他们幸福的生活去努力,那阿毛又重新再温暖起那颗久伤的心,去再爱她的丈夫,去再为她丈夫的光明的将来而又快乐的来生活,也是不可知的事。无奈小二,他只是一个安分的粗心的种田的人,他知道妻是应该来同着过生活的,他不知道他却还应该去体会那隐秘着的女人的心思。也许这又是阿毛的幸福,因为在他那简单的,传统的见解上,认为更是他妻的不对,更去折磨她
\newpage
也有之的,那末阿毛就可以永远沉浸在她的梦幻中。

阿毛看见小二出去了,觉得他冷淡得很,简直是非常之狠心,因此她更大颗大颗让眼泪直抛下来。

后来阿婆也觉出她的病来,看见她茶不思,饭不想的,疑是有了喜,倒反快乐,也愿意宽待她些了。觑着在无语把一双手浸在凉水里洗衣服的阿毛,这
老婆子就大声喊着说: 

“放在那儿吧。今天你起得太早,去躺一会儿
吧!” 

家里人又都似乎对待她很和平了,不过她依然还是那样从不见一点笑容在脸上,让人放不进一点好
意去。 


是八月的一天了,阿毛病还没有好,她依然起得非常早,早得院坝里还没有人影来往。头是异常的晕眩,她近来最容易发晕,大约是由于太少睡眠,太
\newpage
多思虑的缘故。但她还是毫不知道危险的,任这情状拖长起去。譬如这早上,已有了很凉的风的早上,本不该穿着薄夹衣站在大柳树下,任那凉风去舞动那短发。而且她把眼睛就放在那清澈的湖水上,心更比湖水还荡漾在更远的地方去了。看见在天空中飞旋的鹰鸟,就希望自己也能生出两片强有力的翅,向上飞去,飞到不可知的地方去,那地方是充满着快乐和幸福。所以她又常常无主的望着天,跟随着那巨鹰去翱翔。鹰一飞得太远了,眼力已不能寻出那踪迹,于是又
把那疲倦的眼皮闽下来,大声的叹着气。 

她正凝望着那天际线出神的当儿,一只手却拍在她肩头,她骇了一大跳,原来是阿招嫂,也没有理
好发,衣裳还是歪歪的披在身上。 

她痴疑的望着阿招嫂,觉得她也瘦了些,她是
自从七月—里分娩后就不常见了的。 


“喂,你没听见吗,是那儿来的哭声呢?” 

阿毛还没答应出她有没有听见,阿招嫂又用力
\newpage

拍了她一下,“听!”并且现着一副紧张的脸。 

她觉得很可笑,什么事该值得那样去注意?然而同时她也听见了,那哭声真来得那样悲痛,那样动
人! 

慢慢她们都听出那哭声正是从她们左边那山坡上所传来,阿招嫂又拖着她向那哭声处走去。一直走到最后边的一所洋房了。她已不敢再继续去听那激昂的狂乱的痛哭,不过她又不知抵抗的随着阿招嫂走上那游廊。房里的听差巳看见她们,也没有来禁止,都木偶样的站着。从靠东边的纱窗望进去,她们看见那钢丝床上,平平的无声无息的躺着那苍白脸色的姑娘。她的脸色是比平常更苍白了,盖一床薄花毡,眼睛半闭着,眉毛和柔发,都显着怕人的浓黑。那美男人呢,就挣扎在两个年轻朋友的怀抱里痛哭,硬要扑到那死尸身上去。阿毛望了那女人半天,想不出什么来,只觉得那情景和哭声忽然变成了一种力,深深的痛
击了她的心一下,便摔脱阿招嫂的手,跑回去了。 

阿婆,大嫂听说那娇美的姑娘死了,都跑去瞧
\newpage
,都也带着叹息回来。整天,她们又都在谈讲到这事

到下午,由几个人抬来一口白木棺材,又听到那更其放纵的可骇的哭声。不久,又由几个朋友送着那棺材出去了。阿毛坐在门边看着那匠人在不平的石级上,很吃力的走下去,好象她自己的心也消失在一
个黑洞里面。 

那棺材中,不就是睡的阿毛所怕见的最以为幸福的人吗?那病,那肺病,就真的无情的致死了她,使她不能不弃了她的一切福乐而离了尘世?可是她是
不是象阿毛所想,她死是很满足了的呢? 

阿毛望着那慢慢隐灭去了的棺材,就是那女人最后的一点影,阿毛真想哭了,觉得一切都太可悲。一切的梦幻都可从此打碎去。宇宙间真真到底有个什么?什么也没有!到头来,终得死去!无论你再苦痛些也好,再幸福些也好。人一到了死,什么也一样了,都是毫无感受的冷寂寂躺在大地里。那女人不是阿毛所最以为幸福的吗?然而到现在,她还不是毫无所知的一任几个穿短衣的匠人把她抬着,远离了她爱人
\newpage

的怀抱,而抬到不可知的陌生地方去了? 

从此,阿毛不再嫉妒那死去的人了。她也没觉得那死有什么可怜,她只感到这个生是太无味。她想,假设她现在是处在一个很幸福的地位,她也不会不
因了这女人的死而想到一切事去悲伤。 

这一整天,什么人都该看出阿毛是完全浸沉在
深思里过去了。 


那可爱的苍白脸色姑娘的死,给与阿毛思想上一个转变,使她不再去梦想到许多不可能的怪事上去。不过她的病却由此更深了,而阿婆巳知道不是喜,好象很恼了她一样,时时要拿话来刺她。好在她自己并不在乎,也不把那些话放在心上。直到她实在不能起来的霉天,她为了不愿把那空气弄得太不安静,她
恳求的对小二说: 

“拜托你,帮我一点忙,请阿婆原谅这个吧:我今天实在起不来,好不好让我静静的躺一会几?”
\newpage


小二摸她的手,觉得异常烧热,又瘦。本来已起身了的他,又倒下去吻了她一下,并去摸她全身,身上也如手一样的热,微微的渍着冷扦。小二觉得她很可怜,又觉得自己很抱歉一样,好久都不很理会她了,只因她癖性怪,自己不好说话。小二抚慰的向她
说: 

“不要紧,你放心,多躺躺吧!我明天会替你
请个医生来看看。” 

她只凄然的一笑,又有声无力的回报了小二一
个“呒……” 

到第三天,她父亲,阿毛老爹也来了。老人家依然很健壮的走了来,同亲家还没交换上三句话就到阿毛床面前了。阿毛把手递给他的,两人都哭了,都说不出一句话。相别还不到一年,而他以为很可以放心嫁出去的活泼女儿,是变到他一眼已认识不清的一
个无生气的瘦弱女人了。他哽咽的说: 

\newpage

“唉!……我害了你!现在我来接你,你跟我
回去吧!呵,阿毛,同爸爸回去呵。” 

阿毛紧紧的抓着她父亲,眼泪乱流,想能同着父亲回去也好。然而最后她又摇头,说什么地力都一样,又说父亲难得来,她病还不知会好不会好,来了
就多住几天,让她多看看他也好的。 

父亲很伤心的依着她的话暂时留下,不过,只住到第三天,他便发誓他宁肯死,他不愿住在这儿了,他受不了她那种沉默!他看她无声的流着泪,又找不到她的苦痛,问也问不出。于是他苦恼的忍着心回
去了。 

医生来过一次,看不出什么病,开了一个药方
也就去了。 

阿婆总说不出对于她的不满来。又疑心她向她父亲说了什么歹话去,所以他去时是现着那样不痛快的脸,又疑心小二也偏护了她,接连两个晚上都睡得

\newpage
非常迟。 

其实,只过得两夭,小二仍然不很留心了。夜晚,黑寂寂的,她不由不再想起许多事,因之,只望天快亮,听到点外边的闹声,把心事混过去就好。但夜又长,等着等着,她说不出那苦恼来,她很希望那庵里的彻夜的木鱼声会传来,那单调的声音不是很可以催她暂时睡一下吗?或是有点别的什么响声也好,
好把她不定的心又引开一下去。 


有一夜,当她刚刚想到一个人死去的事,而伤心起来,而长长的叹了气后,那声响,那凄侧的声响,又传来了。那是她从前有一夜听过的,就是她右邻的人所弹奏出的提琴声,那歌调在那弦上是发出那样高亢的,激昂的,又非常委婉凄侧的声音,阿毛又想哭了。她从前懂不了那音节的动人处,为什么会抓着一个人的心,使你不期然的随着它的悲楚而留出泪来,现在呢,她觉得那音调是正谐和于她的曼声的长叹。那末,在那音调里面所颤栗着的,是不是也正同于
她的那颗无往而不伤的心呢? 

\newpage

她怀疑得厉害,到底那对无忧的美夫妇,为什么要在这夜深奏出如许动人的哀音?她拚命挣起来,走到屋外,从玻璃窗望去,在明亮的电灯光底下,她把那女人望得清清白白的!那女人,她披着一件红的大衫,蓬乱着一头短发,手抱着一件东西,狂乱的摇摆着她半身。那声音便从那不知名的东西上所发出。忽然,那女人猛的又掷了那东西,只听见砰的一声,连女人也倒了下去。许久,许久,又都寂然。灯光从
墙上反射出很明亮的光照到好远。 

阿毛很想跳到对面去,抱起那女人来哭。那女人曾和她谈过一次话的,是如何的和蔼近人呀!为什么她也会独自在夜深如此的悲苦?她不是也现得几多
幸福的吗? 

阿毛在露水很重的夜里站了许久,心就盘旋在那间精致的,倒有一个美女人在地毡上的房子里,直到阿婆咳嗽,才又惊醒了她。她只得又勉强一步一步慢移回房去。她本只以为幸福是不久的,终必被死所骗去,现在她仿佛又以为根本就无所谓幸福了。幸福只在别人看去或羡慕或嫉妒,而自身是始终也不能尝
\newpage
着这甘味。这又是她刚从这个女人身上所发现的一条定理。她辗转思量了一夜,她觉得倒不如早死了好。


这夜过后的第二个夜晚,小二刚睡熟,便被他妻的转侧所扰醒。她揪着被角把身子弯成一团,不住的喘着气。小二也骇倒了,一摸她,满头浑是汗,身上也是的。而且刚当小二的手一触着她时,她从咬紧的牙关放出一声尖锐的叫。但小二再问她,她又默然
了,且强制住那喘气。 

小二起身去把煤油灯点亮了。她两眼直瞪着,两手紧箍住肚子。小二再三的问是不是肚子痛,她才点了一下头,立即又大声的喊道:“放心!不要紧的
!” 

一阵已比一阵厉害,脸色惨白得怕人,于是小
二去敲前房的门: 

“大嫂,大嫂,请起来一下,阿毛病得很厉害

\newpage
了呢!” 

大嫂看见她时,直叫了起来,只喊:“怎么了
,怎么了,你,阿毛?” 

大哥也走了来看,阿毛把被角咬着,手扳着床
缘,直望着他们摇头,意思是说不要紧的样子。 

这时阿公阿婆都醒来了。阿毛也强制不住,时时大声的叫着。小二去替她抚摸,她猛然推开他的手
去,并且叫道:“不用!不用!水!拿点水来!” 

小二捧过水去,她一下就吸干了。但更呻吟了起来。大哥断定吃了什么东西,问她,她还是乱摇着
头。 

阿婆又嚷起来,说是好好的人,要吃什么东西
来骇人,反威逼她说出。 

不久,她又平静下去,弱得一点力也没有,小
二走拢去握着她,她又哭了,她嘶声的说: 

\newpage

“原谅我吧!迟早我总得死,现在死了,免得长年躺着来折磨你。我不好的地方,你就忘掉了吧…
…” 

她又把眼光望到大嫂去,微笑的点着头,说:


“谢谢你一切,阿毛死了,来生投报吧!” 

大嫂倒被她的样子弄得也哭泣起来,劝着她不
要焦急,病总有天会好的。 

但猛的她又剧痛起来,她在板床上打着滚,口
里叫着:“痛死我了!痛死我了!” 


小二用力的去抱她,扳着她问: 


“说呀!你吃了什么了?” 

她哑声的嘶喊着,又怪声的笑了起来,在垫被
下抓出一大把火柴杆来抛出: 

\newpage

“是的,我吃了!我吃了!我现在就会死去!
我现在就会死去!” 


大哥拔上鞋就朝昭庆寺跑去赶医生。 

但等不了医生来时,她已在狂乱的翻滚中,又把自己毫无声息的掼在床上了,大张着口,朝上面呆
望着。 

小二走上去:“阿毛!说,为什么你要寻短见
?” 

“不为什么,就是懒得活,觉得早死了也好。

小二还想再去问,她作了一个手势,小二就停止了。这时从右邻又传出那动人的哀音。她咕噜着:
“唉!什么事都从此完了!” 

小二再去看她,她已死了。在肚腹间还不住的
起伏着。 

\newpage

于是一片哭声号啕起来。同时,那提琴声就又慢慢低沉下去,且戛然便止住了。

\end{document}
