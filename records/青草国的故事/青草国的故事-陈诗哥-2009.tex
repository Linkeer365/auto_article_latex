\documentclass{article}
\usepackage[utf8]{inputenc}
\usepackage{ctex}

\title{青草国的故事\footnote{Click to View:\url{https://web.archive.org/web/20221010010716/https://www.poemlife.com/index.php?mod=showart&id=53182&str=1214}}}
\author{陈诗哥}
\date{2009-02}

% \setCJKmainfont[BoldFont = Noto Sans CJK SC]{Noto Serif CJK SC}
% \setCJKsansfont{Noto Sans CJK SC}
% \setCJKfamilyfont{zhsong}{Noto Serif CJK SC}
% \setCJKfamilyfont{zhhei}{Noto Sans CJK SC}
% \setlength\parindent{0pt}

\begin{document}
\CJKfamily{zhkai}

\maketitle


\Large


一 

古语有云:有鸟飞过的地方就是鸟国,有青草生长的地方就是青草国,有蚂蚁爬过的地方就是蚂蚁国,非常清楚,这是自有天地以来就有的法律。尽管有时两个甚至几个国家的国土会重叠,但绝不会含糊
。 

据有关部门证实,这青草国竟是一个超级大国:国土遍及大半个地球,人口多得无法统计。呼伦贝尔大草原,科尔沁大草原、锡林郭勒大草原、非洲大草原、美洲大草原、各大山头就不说了,就是一条无名的小路,只要有一点泥土,都有可能住着数以千计的青草国居民。而单是非洲大草原上的居民数量,就多得让你十万个日夜也数不完,一亿个日夜也数不完
\newpage

,永远也数不完。 

我是一百零八世青草国王的朋友,常到他的王宫赴宴,与王公大臣们谈论时事、艺术、打猎以及魔法,每次都是这样。因为我的窗外长满青草,我学会了青草的语言。久而久之,对青草国的故事我也略知一二了,要知道,他们的故事比他们的人口数量还多一万倍。青草国的王宫当然是在草丛的深处,若从天空往下看,人们看到的只是一片普通的草丛,外国人就是想破脑袋也不可能找到入口。但我可以告诉你,入口就在某一片大草丛的下面。起初,那条入口处的小路只有蚂蚁般大小,毫不起眼,越往里走,路则越大,草则越壮,到了王宫附近,草都像参天大树一样高大了,那是来回巡逻的士兵草。这是一座朴素的宫殿,地上铺着一层细密的青苔,踩上去比地毯还要柔软舒服,宫殿四壁则是名贵的中草药,冬暖夏凉,散发出薄荷的气味。在宫殿的大门两侧,有一幅对联,上联为“梦想治国”,下联为“四海为家”。 青草国王说,这是老祖宗传下来的法典,是青草国立国的
基石。 

\newpage

确实,青草国的居民以梦想为生,甚至,因梦想而死。为了实现自己的梦想,他们经常四处漂泊,四海为家。古时候,一个青草国的长老发誓说,要把青草种到海里去,结果,这种奇思妙想变成现实,无论在海里,还是在河里,如今人们都可以看到美丽的水草。而不久前,有一株兔子草,有一天竟挑着一个包袱,周游了整个世界,他装扮成游方僧人、商人和学者,混迹于三教九流之中,学会了各种语言和习俗。他去过皑皑的雪山,也在大海边伫立了一个晚上,天亮时他说:“我狭小的胸膛里将永远回荡着大海的涛声。”有一回他在北边的沙漠上沉思了很久,天上的秃鹰听见他在喃喃自语:“一千年前这里可能是青草国的国土,那一千年后呢?”据说他独自一人居住
在沙漠里,研究某种神秘的东西,最后音讯全无。 

总而言之,青草国就是一个充满神奇梦想并付诸行动的超级大国。这个超级大国却又是一个十分平
和的国家,处处莺歌燕舞,谈论艺术和生活。 


二 

\newpage

这一天,邮差送来了青草国王的信,邀请我去王宫一聚。我当即启程,一路上欣赏优美的草原,青草国的农民正在田里辛勤劳动。这是一种真正的美啊!我感到一阵奇怪的风拂面而过,仿佛包含着一种呐喊,一种忧伤的情绪,便停下车,向正在劳动的青草
老农请教,因为他们的知识是十分渊博的。 

青草老农叹了一口气说:“先生啊,这股风不是别的,正是我们最尊贵的国王在王宫里悲伤地叹气,这口气飘出王宫,吹过街道,轻轻拂过青草国子民
的头发,就变成了延绵的风。” 

这位值得尊敬的老农还跟我细说事情的缘由。
 

原来,这个世界上最庞大的国家有一个奇怪的传统:历代国王自小都有做侠客的梦想。国王们可以实现任何梦想,唯独这个梦想实现不了,因为练武在王族里头是有失身份的,所以法律严令禁止。每逢日落时分,在处理完国家大事后,历代国王就会走到城堡最高的窗口旁边,眺望远方,想起这个无法实现的
\newpage

梦想,心中发出一次比一次强烈的深沉的叹息。 

我见到一百零八世青草国王的时候,他还在窗边,保持着眺望远方的姿势,脸上露出忧伤的表情。


他说:“你知道,国王总是身不由己的。” 

为了让他不再忧伤,我说:“不如明天一起带
上猪笼草去打猎,去捉昆虫?” 


他摇摇头。 

我说:“不如明天去海边一起和水草们捕鱼?
” 


他还是摇摇头。 

我说:“不如明天乘着蒲公英,一起去天国逛
逛?” 

他还是摇摇头。过了片刻,他又叹了一口气,
\newpage
这口气真的好长好长,他准备向我讲述一个古老的故事。他说:“你知道,当世界还是很早很早的时候,青草国只是一个小国,人口不过几千,不过那时候的青草国子民都十分高大,像树一样,除了喜欢做木匠、酿酒、讲故事、研究魔术等外,他们还喜欢谈论武艺,还自创了许多武功,而天性却是喜好和平,所以他们切磋武艺都是点到即止,决不会有人受伤。那时
候,人民安居乐业,安分守己,其乐融融…… 

“那时候,国王与民众一样喜欢武艺,但自从二世青草国王时代长老们用法律的形式规定国王不准练武,你知道,那是有失身份的,而且,如果国王也热衷于练武的话,那么,这个国家注定要灭亡的。之后的历代国王都只能欣赏武艺表演,而不能亲身参与。久而久之,青草国王就变成了谦谦君子,青草国也
变成了艺术之都,充满了梦想和光荣。 


“几百年后,六世青草国王继位了。” 


三 

\newpage

这个年轻的国王似乎与他的前任有些不同,他似乎比较豪迈,精力充沛,富于想像,上朝的时候他几乎从不打哈欠,甚至与朝中大臣称兄道弟。为此长老们曾数次表示不满。不过总的来说,他还是比较循规蹈矩,遵守国王的一切礼仪,像一切国王那样擅长打猎,每当黄昏来临的时候,他也会像其他国王一样来到窗边,眺望远方,发出深沉的叹息,一次比一次
强烈。 


有一天,王宫外面贴了一张国王的圣谕: 

“每人都有一个梦想,每人都有实现梦想的权利。吾国以梦想治国,天下皆知,莫不称颂。吾国子民性好习武,为实现这一梦想,发扬光荣的传统,丰富人民群众的生活,朕立刻宣布:从下月初八开始在京城举行一次全国性的武林大会!地方各省、各县、各镇、各村速速办去,不得延误。违者撤职查办。钦
此。” 

这一消息迅速传遍了大江南北,青草国的子民是多么激动、兴奋,甚至热泪盈眶,莫不称颂国王陛
\newpage

下的英明,因为他们终于有机会一展抱负了。 

长老们简直傻了眼,他们摸不清国王究竟在搞什么花招。他们说:“只要国王不参赛就行了。”于是他们也很快喜欢上这个消息了,私底下还说陛下果然神机妙算,因为他们也是十分喜好欣赏武术表演的

人们纷纷聚集京城,京城的客栈住不下了,人们就在京城四周的平原、山谷、河边、路边安营扎寨,迎接那个伟大的日子来临。于是,在京城的周围,形成了好几个繁荣的市镇。据统计,除了老弱病残的草之外,青草国的居民几乎全报了名参赛,连一向隐居在深山老林里的各路大侠也纷纷出山,赶往这一盛
宴。 

于是,在那些日子,人们彻夜不眠,谈论武艺与梦想,什么南拳北腿、龙爪手螳螂拳、如来神掌玉女剑法,包括江湖上传闻的武功秘籍《九阴真经》,也有人见证说在某某地方见过,而且的确有人练成,千真万确,不过那位侠客平时只喜欢行侠仗义,做好

\newpage
事不留姓名。 

就在人们废寝忘餐、潜心钻研武艺的时候,另一个问题产生了,全国人民都在练武,生产怎么办?粮食怎么办?这是一个非常重要的问题,连长老们都蓄手无策。还是非常聪明的六世青草国王想到了一个绝妙的养生办法:保持愉快的心情,只要晒晒太阳,把脚伸进泥土里,就可以吸收养分了。这一方法极大方便了青草国的子民们,当他们切磋武艺累了的时候,就蹲下来把脚伸进泥土里,片刻就恢复了精力。这一神奇的方法在青草国里得到广泛的推广(后来连树国也模仿这一做法了),现在,我们都能看到青草国的居民喜欢把脚伸进泥土里晒太阳,一副轻松愉快的表情。只有王宫里的贵族们还保持着那套复杂的饮食
习惯。 

武林大会的日子一天一天接近了,人们的热情也一天比一天高涨。就在人们的热情最高涨的时候,那天就到了。对着下面绿压压的,不,黑压压的青草国子民,六世青草国王发表了重要的演说,他说:“
废话少说,马上开始!” 

\newpage

比武一天进行十场,每场六世青草国王都会到场观看,人们都看到,他们的国王看得十分认真,这更加鼓舞了参赛选手和观众。比赛完后,国王会在王宫里亲自接见得胜者,他会屏退左右随从,和这些得胜者谈得十分投机。他仔细询问这些得胜者的门派、武功的口诀、心法,并请得胜者就他在会场上看得不清楚的招式重新示范一次。他还会问:“江湖上真的有《九阴真经》和《九阳神功》这样的功夫吗?”这些功夫大师说:“当然没有。都是小说家们瞎编的。

过了一段时间后,这些功夫大师们很惊奇地发现,国王不但能指出他们招式存在的缺陷,可以怎样改进,他还告诉这些大师怎样不费吹灰之力就可以破
解这些招式。这真是十分神奇的事情。 


这次武林大会一比就是三十年。 

这三十年来,国王每天都在王宫里与得胜者们畅谈、比划。民间有传言说,每当夜深人静的时候,有人听见王宫里传出呼呼的风声,风声凌厉,苍劲有力,无坚不摧。有人说是国王独自在王宫里练武。后
\newpage
来,这风声变得平和、温厚,但不可抗拒,仿佛涵盖
一切,人们说:“国王的绝世武功就要练成了。” 

终于,只剩下两名得胜者了,他们都是当今赫赫有名的大侠,分别是东山君子兰大侠和西山的鬼针草大侠。而“天下第一”的称号将会在两人当中产生

真不愧是当今的绝顶高手。两人从早打到晚,打得天昏地暗,飞沙走石,什么 “游龙戏凤”、“天下无贼”、“ 满城尽带黄金甲”、“ 十面埋伏”,这些武林中难得一见的秘技让人看得眼花缭乱、目瞪口呆,深深陶醉在武术的最高境界中。太阳下山的时候,比赛终于结束了。结果是东山的君子兰大侠获胜,西山的鬼针草大侠拉起君子兰大师的手,佩服
地说:“君子兰大侠就是当今的天下第一。” 


有观众说:“这话应该由国王说才对的。” 

但国王说:“管他呢。”他和君子兰大侠在王宫里谈了三天三夜,据说两人还秘密举行了一次比武,但谁胜谁负无人知晓。有人说,是国王赢了,因为
\newpage
他已学会天下的武功。有人说得更详细一些,当时国王向君子兰大侠露了一手,当时苹果树上的一只蛀了虫的苹果往下掉,国王貌似漫不经心,懒洋洋地举起了手,那只苹果就在半空中停住了,仿佛掉在一只透明的碗里。也有人说,是君子兰大侠赢了,因为他还有一招绝技“一指禅”没有在之前的比武大会使用过,所以国王没有学会,就是那一招击败了六世青草国
王的。 

第二天,一件更令人惊奇的事情发生了。王宫外面贴了一张国王的圣谕,圣谕上说:“君子兰大师德艺双馨,为世间难得之俊杰,朕应当退位让贤。”

当天晚上,六世青草国王潜出了王宫,无人知
道他的下落。 

长老们因之前国王暗中练武的传闻已让他们烦
恼不已,因此也不反对这一决定。 

君子兰大侠也欣然接受了这个让贤,七世青草国王继位。让人意想不到的是,这位新任国王很快就
\newpage
忘记了自己一身的绝世武功,因为老祖宗的法律规定:国王练武是有失身份的。于是,像所有国王一样,武侠变成了新任国王的梦想,无法实现的梦想,每当黄昏来临,他都会走到窗边长叹。七世青草国王勤政
爱民,很快,就成为人人都爱戴的好国王了。 

不久后,人们却发现江湖上出现了一个大侠,他武艺非凡,甚至没有遇到敌手,他到处行侠仗义,锄强扶弱,飞檐走壁。有人说,那位大侠的容貌酷似前任国王六世青草国王。那人说得一点不错,正是六世青草国王,他现在把自己称为“青眉”,而人们则
称他为“青眉大侠”。 


四 

六世青草国王成为青眉大侠后,梦想使他的生活发生了根本变化,某些习惯慢慢改变了,原来高贵的举止和言谈也变得直接、粗犷甚至粗野。因为他不再需要住在隐秘的王宫里,整天对着之乎者也的王公大臣。他享受着侠客的自由和乐趣。他周游世界,睡在大树上或者山洞里,饿了就蹲下来,伸进泥土里饱
\newpage
餐一顿,渴了就痛饮天上的雨水。有时他会拍拍某个行人的肩膀,等他转过身来,他已经飞走了。他甚至混迹于市井之中,用普通民众的言语方式和他们交谈,并倾听他们的故事。当他经过市场街,他会充当小贩们纠纷的仲裁。有时,他是孩子们的朋友,跳上树顶帮他们取下气球,并表演魔术,其实都是他用功夫
做了手脚的把戏。 

有时,他一个人涉江过海,施展水上漂的轻功,在海上行走整整一年。穿越大海的雁群累了,就在
他肩膀上歇会。 

他还去过鲸鱼国,和鲸鱼国王谈了七天七夜,创造出一套神奇的武功,叫“鲸波功”,是观赏鲸鱼舞队表演喷水得来的灵感。它的原理是:先把气聚集在丹田和胸腔,然后通过喉咙喷薄而出,威力之强就像十二级飓风,席卷一切,最后这股飓风会变成一场
大雨潇潇落下,空气清新。 

他喜好攀岩。有一次,他攀上珠穆朗玛峰顶,认识了一朵玉洁冰清美貌无双的小花,并深深地爱上
\newpage
了他。当然,她也爱他。但这种花只有一天的寿命,早上开,晚上谢,为此他们拥有一段刻骨铭心的爱情

有人说,他在南方的沼泽地里杀死了一条喷火
的恶龙。 

还有人说,他同时与七十大盗作战,最后把他
们赶回海上的小岛,从此不敢出来作恶。 

甚至,还有人说,他有许多私生子,留下一堆
著名的风流韵事。 

就这样,这位青眉大侠成为青草国的人们热烈谈论的对象。他的名声甚至传到国王。接下来很长一段时间里,青眉大侠隐姓埋名,淡出公众的视野,但人们还是热烈地谈论他,又有一些人作见证说,在某某地方看见有一个人极像他,两肩站着四只青蛙,好像在研究什么武功。结果,青草国很多人真的跑去与青蛙为伍,研究出什么“青蛙剑法”、“青蛙掌法”

话又说回来,青草国并不是永远太平的,因为
\newpage
青草国并不是所有人都喜好和平的。在青草国的北部,住着魔鬼草部落,他们凶残成性,研究邪恶的武功,并纠结青草国的一大批败类、土匪、恶霸、山贼、小偷、强盗、流氓、傻瓜、笨蛋、废物、孬种、白痴、瘟神、无赖,兴风作浪,坏事做尽,他们还有一个共同的白日梦,那就是攻打京城,夺取王位,一统天
下。 


有一天他们真的起来造反了。 

他们一路南下,见草就杀,见屋就烧,一时间火光满天,恐惧与惊叫弥漫在青草国的每一个地方,多少梦想遭到了蹂躏。这件事过了很久以后,如青草国的小孩啼哭,母亲们就会说:“如果你再哭,魔鬼
草就会来把你带走的。”他们就真的不哭了。 

必须有人起来抵抗。但是七世青草国王,昔日大名鼎鼎的东山君子兰大侠,早已忘记了自己一身绝世武功,他甚至害怕杀戮和打斗。他大力发展文艺,结果连将军们都深深爱上了音乐和舞蹈,每天哭湿几条手帕。这艺术的梦想甚至在军队和监狱也得到了广
\newpage
阔的土壤。士兵们观赏完高品味的艺术表演后说:“战争是多么的残忍啊!”罪犯们观赏完后说:“再不愿犯罪了,这多么可耻!”看着魔鬼大军就要逼近京城,他们简直不知所措,眼神有些古怪,七世青草国王甚至准备独自一人去劝说魔鬼草首领投降,他会说:“唱歌和跳舞也很好的呀。”然而,随从们好歹把
他拉住了。 

捶胸顿足是没有用的,撕心裂肺也没有用。这时候,一个消失已久的人出现了,他简直就像从天而降,那就是手执青龙宝剑的青眉大侠。在他隐世的那段时间,他跑进山里潜心研究,把“鲸波功”改进为“青草神功”,威力比“鲸波功”威力一百倍,因为喷出来的不再是风,而是火,但是那样本人也会因真
气耗尽而死。 

青眉大侠的出现振奋了青草国的人心,民间纷纷组织起有效的抵抗。他们打起“为了梦想”的旗帜,团结了一大群有为青年。于是一场浩大的战争开始了。但正所谓“道高一尺,魔高一丈”,梦想军人数虽多,但魔鬼军凶残成性,这暂时成了一场势均力敌
\newpage

的战争。 

这场战争我如果如实把它描写出来,那简直太残忍了,会吓坏小朋友的。我只能简单地说,他们从地上打到天上,因此天昏地暗,日月无光,残肢遍野
,血流成河,怪叫声此起彼伏。 

眼看这场战争就要无休无止地进行下去,倒下的草也越来越多。青眉大侠心想:“如果再持续下去
,会有更多的梦想遭受毁灭的。只能用那招了。” 

于是,他凌空而起,一个人和魔鬼草对阵。他十分生气地说:“你们这些批败类、土匪、恶霸、山贼、小偷、强盗、流氓、傻瓜、笨蛋、废物、孬种、白痴、瘟神、无赖,天堂有路你不走,地狱无门你闯进来。受死吧!”他聚集丹田之气,天上的白云、地上的枯叶、河里的流水似乎都进了他的口中,这样过了片刻,天地间变得很宁静,突然,一股熊熊大火从他的嘴里喷出,直向魔鬼草卷起。刚还在偷笑的魔鬼草们只能怪叫几声,就被烧为灰烬,落在海洋上,不

\newpage
知被海水带到何处。 

然而,在青草国的战士们欢呼胜利的同时,他
们也找不到青眉大侠的身影了。 

魔鬼战役过后,青草国又天下太平了,国王和将军们又重新爱上了音乐,七世青草国王又重新成为英明的君主。而青眉大侠则永远活在青草国人民的心中,时至今日,青草国的人们还用深情的音乐和舞蹈
缅怀青眉大侠。 


 


                    
            五 

一百零八世青草国王又长叹了一口气,说:自从那次大战后,青草国发生了巨大的变化,总的来说
可以归纳以下几点: 

1.原来许多高大的青草战士在战争中英勇负
\newpage
伤,失去一只手或一只脚,或者两只手两只脚,这些手啊脚啊鼻子啊耳朵啊等残肢落在地上,漫山遍野都是,它们并没有死去,而是立即伸进泥土,就地生长起来。有些甚至走到自己喜欢的地方才定居下来,结婚,工作,生儿育女。就是这样,青草国的居民开始遍布天下。不过也从那以后,由残肢繁殖下来的后代就像一只手或一只耳朵那么高,他们就是人类现在看到的漫山遍野的小草了。只有国王与王公大臣、军队以及没有在那次战争中负伤的人,还像以前那样高大

2.原来魔鬼草居住的地方,由于青草国的居民不愿移民到那里去,所以就变成了现在的撒哈拉沙
漠、塔克拉玛干沙漠、阿拉伯沙漠。 

3.经历过那次大战后,有些青草青年立志医
治世人,于是化身为草药。 

4.有些草因对七世国王和将军们的失望,则落草为寇,沦为草莽英雄,据说人类中一些比较憨厚
、有趣的强盗就是从他们的后代演变而来的。 

\newpage

5.有的则成为草根阶层,有人把他们称为“
草民”。 

6.为了纪念青眉大侠舞剑的英姿,一青草国
公民创造了草书,由人类流传后世。 

7.时至今日,青草国人民还在怀念那场惊天动地的战争,闲居的时候谈论青眉大侠与兵法,偶尔一阵风吹来,漫山遍野都是兵气,所以多少年后人类有个大将军看到满山青草舞动,就以为是满山的兵士,大惊失色,因此后世便有了“草木皆兵”的成语。

现任一百零八世青草国王坚持说,要不是他国事繁忙,深感责任重大,他也会像他的祖先青眉大侠一样放弃世俗的一切,退位让贤,去少林寺或者某个
深山老林练成绝世武功。 


我表示同意。 

讲完这个故事后,一百零八世青草国王如释重负,他又快活起来了。他挥挥手:“来,舞蹈,音乐
\newpage

!我们开始宴会。” 

于是,著名的青衣乐队和舞蹈队徐徐而出,滴
答滴地演奏起来。 

我回到家后,发现墙角长出了一棵小草,我想,一定是我离开家太久了,以至于外面这棵年轻的草以为这个房子荒芜了,没有人住,就冒失地移居到这间堆满书籍的房子来。我热情地接待了这个客人,只是叫他的兄弟姐妹别再来了,不然我的房子就会变成
草屋。 

就这样,我和这棵小草一起生活,把书上的故事念给他听,过了一天又一天,一年又一年,我们都
过得很幸福。 



2009年2月  安静居 



\end{document}
