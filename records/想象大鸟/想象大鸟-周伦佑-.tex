\documentclass{article}
\usepackage[utf8]{inputenc}
\usepackage{ctex}

\title{想象大鸟}
\author{周伦佑}
\date{}

% \setCJKmainfont[BoldFont = Noto Sans CJK SC]{Noto Serif CJK SC}
% \setCJKsansfont{Noto Sans CJK SC}
% \setCJKfamilyfont{zhsong}{Noto Serif CJK SC}
% \setCJKfamilyfont{zhhei}{Noto Sans CJK SC}
% \setlength\parindent{0pt}

\begin{document}
\CJKfamily{zhkai}

\maketitle

\setlength\parindent{0pt}


\Large

鸟是一种会飞的东西\\
不是青鸟和蓝鸟。是大鸟\\
重如泰山的羽毛\\
在想象中清晰的逼近\\
这是我虚构出来的\\
另一种性质的翅膀\\
另一种性质的水和天空 \\ 


大鸟就这样想起来了\\
很温柔的行动使人一阵心跳\\
大鸟根深蒂固,还让我想到莲花\\
想到更古老的什么水银\\
在众多物象之外尖锐的存在\\
三百年过了,大鸟依然不鸣不飞 \\ 


大鸟有时是鸟,有时是鱼\\
\newpage

有时是庄周似的蝴蝶和处子\\
有时什么也不是\\
只知道大鸟以火焰为食\\
所以很美,很灿烂\\
其实所谓的火焰也是想象的\\
大鸟无翅,根本没有鸟的影子 \\ 


鸟是一个比喻。大鸟是大的比喻\\
飞与不飞都同样占据着天空 \\ 


从鸟到大鸟是一种变化\\
从语言到语言只是一种声音\\
大鸟铺天盖地,但不能把握\\
突如其来的光芒使意识空虚\\
用手指敲击天空,很蓝的宁静\\
任无中生有的琴键落满蜻蜓\\
直截了当的深入或者退出\\
离开中心越远和大鸟更为接近 \\ 


想象大鸟就是呼吸大鸟\\
使事物远大的有时只是一种气息\\
生命被某种晶体所充满和壮大\\
推动青铜与时间背道而驰\\
大鸟硕大如同海天之间包孕的珍珠\\
\newpage

我们包含于其中\\
成为光明的核心部分\\
跃跃之心先于肉体鼓动起来\\
现在大鸟已在我的想象之外了\\
我触摸不到,也不知它的去向\\
但我确实被击中过,那种扫荡的意义\\
使我铭心刻骨的疼痛,并且冥想\\
大鸟翱翔或静止在别一个天空里\\
那是与我们息息相关的天空\\
只要我们偶尔想到它\\
便有某种感觉使我们广大无边 \\ 


当有一天大鸟突然朝我们飞来\\
我们所有的眼睛都会变成瞎子

\end{document}
