\documentclass{article}
\usepackage[utf8]{inputenc}
\usepackage{ctex}

\title{知乎q39952242a1980365254}
\author{匿名用户}
\date{2021-07-25}

% \setCJKmainfont[BoldFont = Noto Sans CJK SC]{Noto Serif CJK SC}
% \setCJKsansfont{Noto Sans CJK SC}
% \setCJKfamilyfont{zhsong}{Noto Serif CJK SC}
% \setCJKfamilyfont{zhhei}{Noto Sans CJK SC}
% \setlength\parindent{0pt}

\begin{document}
\CJKfamily{zhkai}

\maketitle


\Large

我母亲离世前几天,一直心心念念要去看海。

我答应带她去,但忙起来就忘了。

那天中午,她突然打电话给我,用很委屈的语气说:“你来接一下我可以吗?我找不到路了,轮椅没电了。”

我说:“你现在在哪里?”

她说:“我不知道。”

我说:“周围有没有醒目的建筑,大房子什么的?”

她说:“没有。”

我说:“怎么会没有呢?酒店,加油站,什么店子都可以,你随便找一个我就能导航过来。”

她说:“没有,我不知道。”

我说:“那你随便形容一下你周围是什么样

\newpage 

子。”

她说:“我在一座桥上。”

我打开地图看了下,旁边的确有座大桥,我连忙跑过去找她。

两百米长的大桥,我从桥头跑到桥尾,没找到她,赶紧又给她打电话。

我说:“你不要乱走,回到桥这里来。”

她说:“我就在桥这里啊,你到了吗?”

我说:“我也在大桥这里,我找不到你。”

她说:“那你估计跑对面去了,等我过来找你。”

我说:“你就在原地等着就行了!”

打完电话,我一边往桥对面跑,一边仔细打量两边,从头跑到尾,还是没找到她。

她一个病到路都走不稳的人,坐着轮椅把自己弄丢了。

我彻底烦躁了,我给她发微信视频。

我想看看她在什么地方,可她手机拿的很近,我只能看到她的脸。

她一脸憔悴,脸色灰沉,嘴唇是紫青色的,

\newpage 

额头上全是汗水。

我说:“你把手机拿远一点,照一下你后面让我看看。”

她就盯着手机,也不说话,不知道在摆弄什么。

我说:“你只要把手机拿远一点,让我看看你后面有什么就行。”

她说:“等一下,等一下。”

我等了几秒钟,她还是盯着手机不知道在搞什么。

我说:“你把手机拿远一点!”

她说:“我不会!”

我说:“怎么可能不会呢!你只要把手机拿远一点啊!动一下手,把手机往前放一点!”

她哭丧着脸,好像这很难理解,还是焦急地盯着手机,不知道在摆弄什么。

我说:“我求求你了,你让我看看你在哪里,你这样我找不到你。”

她说:“我知道了。”

然后还是没有把手机拿远。

我又催促了两次,她莫名其妙居然把视频挂

\newpage 

断了!

我彻底崩溃了,我骂了几句粗口,骂我自己,然后给她打电话。

电话打不通,我又急又气,头都晕了,打不通,我继续打,一直打,疯狂打,打了一百多个。

终于通了。

我说:“好了,什么都别说了,路边有人没有,你找个人问一下你在哪里。”

她说:“他们不理我……”

她话说了一半就不说了,我也明白了,坐着轮椅的她一看就是病入膏肓的人,没人愿意担风险做好人。

我要疯了。

我想不通,曾经那么果敢精干的一个女人,如今怎么会变成这样。

连将手机拿远这种事都没办法办到,我那时恨不得从桥上跳下去。

后来,她遇到一个小伙子,小伙子把准确的地址告诉了我,我终于找到了她。

她在一公里外的另一座桥上。

她坐在轮椅上,以为我要骂她,她低着头不

\newpage 

敢看我,强装镇定。

小伙子说:“太危险了,她一直在逆行,轮椅都跑到马路中央去了。”

我谢过小伙子,勉强冷静了下来。

小伙子走后,我说:“你现在是怎么回事,连手机拿远一点都不会了吗?”

她不说话。

我说:“你回答我啊,这是为什么啊?你到底有什么想不通的。”

我拿出手机,模拟了视频的样子。

我说:“你看,这是视频的时候,我叫你拿远一点,你只要用手往前一点就行,你不会吗?”

她还是不说话。

我没辙了,慢慢推着她回家。

那天太阳很大,下午闷热的不行,可我寒冷彻骨,满心绝望。

……

天将要黑,我和她才堪堪回到家,从中午折腾到晚上,两个人都累的够呛。

我随便炒了两个小菜,又从外面买了几盒米饭,叫她吃饭。

\newpage 



她尝了两口,说:“真好吃,今天饿了,我要多吃点!”

我说:“你吃吧,没人跟你抢。”

母亲的厨艺其实很好。

虽然我在酒店掌厨多年,自认厨艺是及不上她的,她也从没这么夸奖过我。

其中一道菜是鸡翅,一共有五个,我和她各吃了两个,碗里还剩一个。

我说:“我吃饱了,还有一个鸡翅你吃吧。”

她点点头,一声不吭地夹进了碗里。

若是以前,这种情况她一定不会自己吃,就算留过夜,也一定要留着给我。

她的病越来越重,食量也越来越少,在我印象中,这是她唯一一次,吃了足足两碗饭。

我原本对她有怨气。

她病的那样重,却只身一人跑到这么远的城市来。

虽然这是我曾经给她的建议。

她的病,是风湿性心脏病,因为老家多山多雨,每到冷天下雨,她的腿就会开始渗水,她的肚子

\newpage 

也有积液,还伴随着颈部血管的猛跳。

她总是睡不好。

在她病的没那么严重的时候,我告诉她,南方有一座小岛,四季如夏。那里空气非常好,并且距离大海很近,她去住一段时间或许会好一些。

她反对我,说那么远,光是路费都得多少钱?她心疼钱。

直到两年后,她的病情越来越重,不光身体越来越差,精神也越来越恍惚,乃至几次入院,医生都下了病危通知书。

她有一天打电话给我,说她要去海市。

我说:“你病成这样,就别胡思乱想了,老老实实在家养病吧。”

她说:“老家太冷太潮湿,身体受不了了。

我说:“你去那么远,谁来照顾你?”

她说:“我自己能照顾自己,我的病已经好多了!”

一个病人说的话,我本不该相信,但我确实信了,电话里她很精神,我以为她真的好了一些。

我说:“好吧,到那边有什么事你给我打电话。”

\newpage 



没过一个月,我接到了她的电话。

准确的说,是她的护士打电话给我,说她摔倒了,伤到了头,正在住院。

我得知这个消息,整颗心都吊了起来。

她在电话里委屈地说:“我摔倒了,我也不知道为什么,这次摔的这么厉害。”

我说:“你为什么不小心一点?”

“你明知道你身边没有人,为什么不小心一点?”

她说:“你过来好不好,你爸爸逼着我回去,我不想回去,这里环境很好,对我的病有好处。”

“只要你过来了,他们就不会逼着我回去了。”

我纠结又烦躁。

我有稳定的工作,如果去她身边,一切都得从头再来,虽然可以靠自由职业赚钱,那一定也比不上这份工作。

可是,一想到她孤独一人回去老家,那么寒冷,那么多雨的山城,我的心就软了。

我说:“我这两天辞职,尽快过来。”

她高兴的像个孩子,在电话里说:“那太好

\newpage 

了,那太好了。”

当我见到她时,她坐在轮椅上,已经病的不成人样。

她的脸,她的手,浮肿而紫青的皮肤紧贴着骨骼,浑身看不到一丝好肉,她病到了这个地步。

她说她的摔倒,是因为轮椅的两个踏板是分开的,她下轮椅时,脚不小心踩进了踏板间的空隙,于是被套倒。

我强忍心痛,严厉地告诫她,下轮椅时一定要找一个能抓扶的地方,先抓扶,稳住身体,再下轮椅,我不可能整天盯着她,我必须工作。

她默不作声。

她后面没给我添过麻烦,直到这次走丢。

我原本对她有怨气,怨她骗了我,让我答应病的这样重的她,走的这样远。

对自己更有怨气,她是一个病人,粗心大意可以归结为身体机能下降。我作为一个正常人,来到她的身边,还是没能照料好她,害她担惊受怕了一个下午。

我想,既然她愿意吃我做的饭,接下来我就认真给她做几天饭好了。

\newpage 



我告诉她,我会尽快找个时间,带她去海边转转,前提是她老老实实在家待几天,不要一个人出去乱走。

吃过饭,她有了精神,她解释说:“今天轮椅电没充够,所以才叫的你。”

我说:“就算轮椅有电,你病的这么重,也不能一个人走太远。”

她假装不高兴:“什么病,说不定我能活到八十岁呢?”

我说:“又没人不让你活。”

她就笑。

此前的万种怨愁,经过这次晚餐,消散了大半。

可我怎么也想不到,这竟是我和她的最后一顿晚餐。

她年轻时爱吃辣,重咸口,很少吃肥肉。

她那时却吃不得辣了,盐也不敢多吃,开始爱吃肥肉,她瘦的皮包骨,坐着都被硌疼,认多吃肥肉有利于长肉。

隔天中午,我还在考虑,晚上要给她准备怎样的饭菜。

\newpage 



突然又接到她的电话。

电话那头是个女声,说她出事了,现在在医院,让我赶紧过去。

我坐在出租车上,气的牙痒,心想她一定又一个人出去了,昨天才刚刚走丢,今天又进医院了。

为什么总不叫我省心。

明明那么叮嘱她了。

我在车上想了很多,想发火,想接着该怎么办,是带着她回老家,还是辞去工作,昼夜看护她。

不论哪一种选择,都非常麻烦,我想不管怎么样,得对她下最后通牒了。

我去到医院,找到她,她躺在病床上,一直唉唉地喊着。

我说:“怎么回事?”

她听到我的声音,努力地想要抬头,可病床是平的,怎么也抬不起来。

我赶紧走过去,发现她眼睛大大地睁着,一直在流泪。

她的头上缠了些绷带,半张脸都是干掉的血,我摸了摸她的额头,皮肤紧紧地绷着,烫的吓人。

她看到我,像是想要说话,可支支吾吾,什

\newpage 

么也说不出来,还是不断唉着,像是疼的。

她是很要强的一个女人。

从小到大,我从没见她这样过。

她得是伤的多重多疼,才会喊出声来。

我大脑一片空白,路上的所有想法和怒火都化为了乌有。

护士告诉我,她是被人送到医院来的,已经半个多小时了,刚刚做完检查。

我问她:“是谁送你来的,你是怎么受的伤?”

她看着我,“是……是……”龇牙咧嘴一会儿,什么都没说出来。

我说:“你别急,慢慢说,你是怎么受的伤

她痛苦地挤眉咧嘴,好一会儿才憋出几个完整的字眼:“不知道。”

我想她一定是伤到了某些神经组织,没办法正常说话和思考了。

我只好问她:“是不是轮椅,你是不是又被轮椅套倒了?”

她的头,费力地左晃右晃,像是在回忆,过了一会儿,终于点了点头,说:“嗯!”

\newpage 



我说:“为什么啊,我叫你不要一个人出去,叫你下轮椅得先找个抓扶的地方。”

她断断续续地说:“我也不知道为什么,我去买菜,莫名其妙就摔倒了,今年运气不好。”

她的外套浸透了血,我摸了一把,沾了一手

我一阵阵揪心。

我说:“怎么会莫名其妙呢,还运气不运气,你就是不听劝!”

她不再说话了,像做错了事,连呼痛的声音都小了些,这令我更是痛心。

医生把我叫去谈话,说她需要住院观察,因为发现脑中有轻微出血。

我想她伤这么重,住院最起码也得十天半个月。我缴了费,索性打电话把工作直接辞了,做好了持久战的准备。

矿泉水,卷纸,护理垫,尿不湿,马扎,我百度了一下医院陪护需要的物品,买了一大堆。

回去病房,她又盯着我支支吾吾地说话。

我问她怎么了。

她憋了半天,说她忘了。

我很怀疑她不是因为轮椅受的伤,而是被人

\newpage 

撞倒的。

因为有一次她被撞伤了小腿,也没有告诉我,直到伤口感染被我发现,在我的逼问下才说出口。

我说:“怎么能忘了呢,你再仔细想一想。

她怎么也想不起来。

她痛苦而气恼地说:“怪了,怪了,我明明记得的啊!”

我注意到她嘴巴很干,问她:“你是不是想喝水?”

她愣了一会儿,重重点头:“嗯!”

连喝水这种事都说不清楚,都能忘了。

我赌气地说:“等你出院,我带你回老家。

她闭着眼睛,脸上痛苦带着些委屈。

我狠心地想,不管她愿不愿意,一定得带她回去了,一边打开矿泉水,倒在瓶盖里喂给她。

她哆哆嗦嗦地喝,可没喝多少,就翻了白眼

她四肢古怪地扭曲着,浑身剧烈的颤抖。

她被一路推进重症监护室。

我没想到她会突然这样,整个人都蒙了。

家属不能陪同,我只能在楼道里或坐或立。

每隔几个小时,医生就打电话叫我一次,讨

\newpage 

论她的病史等情况。

我熬了一天一夜,恍恍惚惚,坐在楼道里睡着了。

我做了一个梦。

梦到我躺在床上,她忽然推门进来,还是那副瘦的皮包骨头,病入膏肓的样子。

她头上缠着纱布,手里提着一袋子菜。

我问她:“你不是在住院吗,怎么回来了?

她说:“我已经好了,住院要不少钱呢。”

我说:“就知道钱,你以后绝对不能一个人出去,买菜也不行,我可以去买。”

她提了提手里的菜:“我知道了,今天我给你做菜吃。”

我说:“买的什么菜啊?我可不吃肥肉。”

她得意地笑着:“都是瘦肉,便宜的很!”

我是被医生的电话吵醒的。

医生告诉我,所有的抢救药都用完了,病人已经不行了,要我赶紧进去。

我一边换隔离衣,一边麻木地签了很多文件

走到她身边时,她一脸憔悴,眼睛紧紧地闭着,她快要走了,却无法醒过来跟我道别。

\newpage 



我平静地站在那里,看着她,不知过了多久,直到护工告诉我,她走了。

我脑子里嗡的一声,一片黑暗从头顶压下来像是有什么东西塌了。

\end{document}
