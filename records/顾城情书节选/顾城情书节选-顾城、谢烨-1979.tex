\documentclass{article}
\usepackage[utf8]{inputenc}
\usepackage{ctex}

\title{顾城情书节选\footnote{Click to View:\url{https://web.archive.org/web/20221018022840/http://www.gucheng.net/gc/gcgs/gcsx/200502/1100.html}}}
\author{顾城、谢烨}
\date{1979}

% \setCJKmainfont[BoldFont = Noto Sans CJK SC]{Noto Serif CJK SC}
% \setCJKsansfont{Noto Sans CJK SC}
% \setCJKfamilyfont{zhsong}{Noto Serif CJK SC}
% \setCJKfamilyfont{zhhei}{Noto Sans CJK SC}
% \setlength\parindent{0pt}

\begin{document}
\CJKfamily{zhkai}

\maketitle


\Large


一、 


顾城致谢烨 

那是件多么偶然的事。我刚走出屋子,风就把门关上了。门是撞锁,我没带钥匙进不去。我忽然生起气来,对整个上海都愤怒。我去找父亲对他说:“我要走,马上就走,回北京。”父亲气也不小,说:
“你走吧。” 

买票的时候,我并没有看见你,按理说我们应该离得很近,因为我们的座位紧挨着。火车开动的时候,我看见你了吗?我和别人说话,好像在回避一个空间、一片清凉的树。到南京站时,别人占了你的座位,你没有说话,就站在我身边。我忽然变得奇怪起
\newpage
来,也许是想站起来,但站了站却又坐下了。我开始感到你、你颈后飘动的细微的头发。我拿出画画的笔,画了老人和孩子、一对夫妇、坐在我对面满脸晦气的化工厂青年。我画了你身边每一个人,但却没有画你。我觉得你亮得耀眼,使我的目光无法停留。你对人笑,说上海话。我感到你身边的人全是你的亲人,
你的妹妹、你的姥姥或者哥哥,我弄不清楚。 

晚上,所有的人都睡了,你在我旁边没有睡,我们是怎么开始谈话的,我已经记不得了,只记得你用清楚的北京话回答,眼睛又大又美,深深的像是梦幻的鱼群,鼻线和嘴角有一种金属的光辉。我不知道该说些什么就给你念起诗来,又说起电影又说起遥远的小时候的事情。你看着我,回答我,每走一步都有回声。我完全忘记了刚刚几个小时之前我们还很陌生,甚至连一个礼貌的招呼都不能打。现在却能听着你的声音,穿过薄薄的世界走进你的声音,你的目光……走着却又不断回到此刻,我还在看你颈后最淡的头
发。 

火车走着,进入早晨,太阳在海河上明晃晃升
\newpage
起来。我好像惊醒了,我站着,我知道此刻正在失去,再过一会儿你将成为永生的幻觉。你还在笑,我对你愤怒起来,我知道世界上有一个你生活着,生长着比我更真实。我掏出纸片写下我的住址。车到站了你慢慢收拾行李,人向两边走去,我把地址给你就下了
火车。 


 顾城 1979年7月 


谢烨致顾城 

你是个怪人,照我爸爸的说法也许是个骗子。你把地址塞在我手里,样子礼貌又满含怒气。为了能去找你,我想了好多理由,我沿着长长的长着白杨树的道路走,轻轻敲了你的门。开门的是你母亲,她好像已经知道了我,就那么很注意地看我。你走出来,好像还没睡醒,黑钢笔直接放在口袋里。你不该同我谈哲学,因为衣服上的墨迹惹人发笑,我想提醒你,又发现别的口袋同样有许多墨水的颜色,才知道这是你的习惯。我给你留下地址,还挺傻地告诉你我走的日子。离开那天你去送我,我们什么都没说,我们知
\newpage

道这是开始而不是告别。 

“你会给我写信么?”你说“会的”。“写多少呢?”你用手比了比,那厚度至少等于两部长篇小
说。 


 小烨 1979年7月 


二、 



小烨: 

收到你寄的“避暑山庄”的照片了,真高兴,高兴极了,又有点后悔,我为什么没跟你去承德呢?斑驳的古塔夕阳孕含着多少哲理,又萌发出多少生命。无穷无尽白昼的鸟没入黄昏,好像纷乱的世界从此结束,只有大自然、沉寂的历史、自由的灵魂……太阳落山的时候,你的眼睛充满了光明,像你的名字,像辉煌的天穹,我将默默注视你,让一生都沐浴着光

\newpage
辉。 

我站在天国门口,多少感到一点恐惧,这是第
一次,生活教我谨慎,而热血却使我勇敢。 

我们在火车上相识,你妈妈会说我是坏人吗?
 顾城 1979年8月 



顾城: 

今天我觉得精神特别好,现在可以告诉你,我病了,发高烧昏昏沉沉好几天,今天我真的觉得我已
经好了。 

这几天躺在床上,天天看或者说是听你的信,也许我真从你那带走了灵魂,它不时聚成你的样子,把你的诗送到我耳边,我好像一个住在海边的姑娘,
听小石子在海水中唱歌。 

你的信让我看见了将来,多好,为什么我不能和你一起看看将来呢。我感到云从松树上升起来,你
\newpage
一步步上台阶,你就走在我身边,我相信,这是命运
。我们在一起的时间很短,而命运是漫长的。 

这会儿,起风了,风吹起我的头发,好像把我的灵魂也吹得飞升起来,我太高兴了,真累……我闭上眼睛就能看见你,像兄长那样站在我面前。你礼貌地带着我走路,给我讲安徒生、讲法布尔的故事,讲路边的草怎么结出果子,瓢虫有多少斑点,你神气地走在路上,好像整个北方都属于你。也许,你还要回到你少年时放猪的地方,走被雨水冲坏的路,白石头美丽地显示出来,你的目光注视着它、穿过巨大的天空、向东方伸去,苦咸的泪洒遍荒凉的土地,到处是白蒙蒙的,就像雪,像冬天,你就在这上面走,越来越远,你还是相信有一个河岸,那里的土地被晨光照亮,曲曲折折的。有许多鸟、许多大雁在那栖息,它们把头放在翅膀下面睡觉。你是属于它们的,你会飞、眼睛里映着我和世界。而我只能躺着,躺在热砂子
上生病。 

真不想让你走得太远,我曾想过用手遮住你的眼睛。现在不了,真的那么做,会使我不得安宁的。
\newpage

 

没人说你是坏人,火车开来开去上边装满了人,有好有坏,你都不是,你是一种个别的人。 小烨
 1979年8月 


三、 



我手一触到你的信就失去了控制,我被温暖的雾的音响包围,世界像大教堂一样在远处发出回声。
你漂浮着,有些近了…… 

我醒来的时候,充满憎恨,对自己的憎恨,恨自己小小的可怜的躯壳,它被吸在地上,被牢牢地粘在蜘蛛网上,挣扎。现实不管你怎样憎恨,都挨着你、吸着你,使你离梦想有千里之遥。 顾城 197
9年8月中 


\newpage


我总要把你的名字写错,写错了还挺高兴,不
知为什么。 

你开始讲生活了,语气沉重,我知道生活不受我们意志的支配,可我并不害怕,因为有一种在痛苦中孕育的力量,使我能拒绝它,能把门“砰”地关上
。当然,我希望你不在门外。 

我不太敢相信现实,我相信你,甚至觉得了解你比了解我自己还多些,你了解我吗?我了解我吗?那天在北京站,我们告别的时候,我曾慌乱地闪过这
些念头。 

现在我伸出我的手。 小烨 1979年8月
24日 


四、 



\newpage

你把我想得很好,这使我很高兴,也很紧张,
因为我毕竟是个渺小的人。 

我想做一个好人,甚至还想有价值,这二者是统一的。我说的价值首先是内心的价值。小时候我这么写过:“向着光明走去,擦洗着自己的灵魂,用决心和毅力,抛去身后的暗影。”“负载着罪恶活着比死亡更可怕。”在痛苦、疑惑、内疚面前,我最不能忍受的是内疚。由于自身的叛卖行为,你看不起自己,不管你在尘世获得什么,这种蔑视都要伴随你终身。我深深地知道世界上只有一种快乐,那就是问心无愧的快乐,做一个好人的快乐。做一个艺术家,他要受到惩罚,因为他要穿过现实的罪恶,把这种信念带给人世,他要告诉人们在那个河岸上(就是你说的被晨光照亮的河岸)有这种快乐,这里没有、商店里没有、彩车里没有、高高的检阅台上也没有,他做了一个轻微的手势,他获得了价值,他也为此受到惩罚。

我不知道我能做些什么,但我知道我要做,在我失败的时候,在世界的门都对我“砰、砰”关上的

\newpage
时候,你还会把你的手给我吗? 

我不怕世界,可是怕你,我的理智和自制力一点都没用。阿喀硫斯是希腊神话中的英雄,他不会受伤,因为生下来时,被母亲握住脚在冥河中浸过。他不会受伤,但被母亲握过的脚跟却是他唯一的致命之
处。顾城 



刚才看电影,看见什么都想到你。我终于受不了了,我跑出来,脚踏着宽宽的台阶,我跑到了桥上,念你的名字。河水在巨大的黑暗中流去,最沉重的只是一刻,这一刻却伴随着我,河水在远处变成了轻轻的声音,而我却活在涌流之中。我看见我的手在黑暗中移动,遮住一粒粒星、一盏盏灯、一粒粒小虫的
歌唱。 

今天没收到你的信,我失望极了。 顾城 1
979年8月29日 


\newpage


信在路上呢,像我们坐火车一个往南、一个往
北,轰隆隆那么近,之后又错过了。 

你的手放在夜的水里干吗?那样你会累的,放得太深就要受苦,而你有许多事要做,我们来到这个世界,相遇还不到两个月,你还不知道我呢,你还不知道自己。自己是不容易了解的。小的时候,我喜欢长头发,总想留上小辫子,不愿再剪短发,可我并不会梳头,妈妈每天到点就得上班,也没有时间把我刚刚长得够握成一小把的头发耐心地梳成好看的小辫子。每天要做这件事将成为她生活中的一大负担。终于有一天,她不顾我的反对,硬是把我的头发又剪成了短发。我觉得自己像个男孩子一样,那么沮丧地站在院子里,心里恨透了那把剪子,恨透了我妈妈,决心再不跟她说话了。她是军人,在部队的医院工作,那时候我倒不觉得军人都像她那么厉害,因为亚如(我小学的同学)的妈妈就给她留了辫子,还有粱娟的妈妈就常常笑,她经常笑得老远都听得见,她还给我吃过自己做的泡菜田茭。我直傻得开始想象换一个妈妈了,我要挑一个最好的,在我认识的所有小朋友的妈
\newpage
妈中间,我一个一个地想过去,找了一遍,结果却全都被我自己否定了,这时我已经忘记了头发,可我还在无名地恨着我妈妈,不过我又不得不承认:我没有发现一个人能够换过来当我的妈妈。没有人能做我的妈妈,只因为我是她的女儿,这是我后来才知道的。这道理太简单了,没有原因,尽管当时我想出了好些非常可笑的理由,但却都不是唯一的。从妈妈那,我知道了一点自己这是件早就被注定的事,我要的一切都天经地义地在我心中。一切远离自身的挣扎、渴望
和要求都是徒劳的。 

也许我们此刻经历的河水和星星,就是我们走向自身的台阶。当你成为真的你的时候,你才知道了自己、知道我,才能成为我,那时,我就是你。我们再不知道黑夜是什么,我们走上台阶、走近我们相见
的日子。 小烨 1979年9月2日 

五、
 


\newpage

天一亮我就醒了,醒了就想到你,都成习惯了。我一边轻轻说话,一边想象你的回答,你真在回答
,今天会有你的信么? 

我给你写信的时候,心里总是挺奇怪:这些字再过几天就要看见你了,他们多幸福呵,我要是也能
变成一个字就好了(即使是一个白字)。 

我要做事了,我要见到你,重病、牢墙、死亡什么也不能阻挡我。我要把世界轻轻推开。见到你—
—那真实的我正在安静地梳理头发。 

快三点了,快来信了,我感到今天有你的信,
再过一会儿就能知道了。 

我很蠢,不能自己,我知道我在走一条古老的路,我为什么非要走那条路呢?渐渐重合又消失的路。我试图去想现实中的你、想我们在火车和广场上度过的那些短短的时光。那时刻真有光,你看我的时候,我的生命是怎样的亮起来,又安静、又辉煌,你的眼睛是琥珀色,你看我的时候车走了,车走了好几辆
\newpage

。 

在这条古老的路上,我有愿望,我总希望时间过去,快过、快过,最好取消算了,可是我又害怕,我还什么都没做呢。我就穿着这件世人的衣裳去见你、睁着茫然的眼睛去见你么?这眼睛不会看见你的,它只能看见一张图画。 顾城 1979年9月5日



我很喜欢你的信、你说话的样子,但我看着看着,忽然觉得要长癌了,我们就不能歇会儿,干点别的?比如说想想我什么时候去北京。要是冬天,我一
定学滑冰,请你姐姐教我(她会,我这么想)。 

小时候,我住在承德,那离北京不远和北京一样的冷。早晨,我去室外刷牙,回来时一拉门把,手就被铁粘住了。第一次被粘的时候,我吓得要命。可惜那时我不会滑冰,也许是因为我还太小。家里门前有块小空地,几张桌子大,四周用木条栅栏围成一个小院,再做上一圈田梗,就能种地了。冬天地里什么
\newpage
也不长,那地方就成了我的露天滑冰场。傍晚担上几担水,要不了一会儿就全冻成冰了,一夜过去,冰硬极了,平坦、透着水晶的光。不管你白天怎么玩,把冰上划出多少痕迹,只要晚上倒了水,过一夜便平整如初。我不会滑冰,但我有一个小冰车,爸爸给我做的,我就坐在上面,在我的小冰场上滑来滑去。你过去见过这么小的冰场么?可在我住的大院里几乎家家都有。这是过去的我的冬天,将来我要学滑冰,穿上冰鞋,像那种带冰刀的非常利害(我不喜欢滑旱冰)
。我要在冬天去北京。 

我们还能一起去别的地方,要是小时候的那个冰场还没化,你还能去看看,也许有一个我,你没见
过。 小烨 1979年9月8日 



我是有毛病,老咬文嚼字地活着,好像替谁活着似的。我不会说话,从小就不会。我刚开始以为话可以随便说,像鸟那样叫着说,可后来人们说“不对

\newpage
”,我就只好不说了。 

以后我离开城市到荒凉的地方去了,在那里放猪,远远地看见一个人在大地尽头走,会感到很奇怪,因为地那么大就托着这么两个人,我从不说话。风在我耳边一直吹,在风停止的时候,草就吐出了香气。每种草都用自己的气味和我说话,那种话不用翻译,就能一直留在你的肺腑里,沿着血液流遍全身。我有一次割草时把自己的手割破了,草茎也流出洁白的血来,我看见了自己和青草的血液,我便不觉得痛,我看见每一滴血都像红宝石那样好,一粒粒那么新鲜。这时候我觉得我要说话了,对我的血,对绿色如茵的草,我说:“我要赞美世界,用蜜蜂的歌,蝴蝶的舞和花朵的诗……月亮遗失在夜空中像是枚卵石,星星散落在河床上像是细小的金砂,用夏夜的风来淘洗吧,你会得到宇宙的光华。”我说:“我要唱一支人
类的歌曲,千百年后在宇宙间共鸣。” 

我对自然说、对鸟说、对沉寂的秋天的大地说,可我并不会对人说。我记得有一回我从桥上走过,一些收工的女孩坐在那,我于是看着远处,步子庄严极了,惹得她们笑了半天,那笑声使我快乐而耻辱。
\newpage


回到城里以后我一直看《辞海》,学习对人说话。一个客人坐在我家里,我对他说:“您好”;一个人在路上,我也对他说:“您好!”我总这样开始,直到结束,重复说这句合乎礼仪的话。有一次,我一激动忽然对人说:“中国人不关心灵魂,见面就问‘吃了么?’从来不问‘你悲哀么?’”第二天我走
近人的时候,他们就依次问我:“你悲哀么?” 

是的,我挺悲哀的,我不会说话,一点都不会。我也真想从这种倒霉的语调中跑出去,去干点别的
。 顾城 1979年9月中 



你真有意思,只会说“您好,”可你却教会了我说话,让我从教室的窗户里跳出来,落在蒿子里。
我对你说:“您好,你真好。” 

我们不要那么老,也不要长大、不要书包,我们可以光着脚丫,一直跑下去,“噼噼叭叭”地跑。
\newpage



跑吧。 小烨 1979年9月 


六、 



我把椅子推开,腿一弯就想,没有跑。我想还是应该由你在前边,我跟着,跟着挺好,我从来是远
远地跟着别人。 

那些男孩在夏天吃完晚饭后就出去了,他们越走越黑,好像是去掏知了,还是干什么?对了,是掏知了,我想起来了。他们从这颗树走到那棵树,忽然又蹲下来聚成一撮,这么着、那么着,乱争吵建议,有的说用水去灌,有的说用棍子去捅一捅,用变了声音的哑嗓子低低地骂人,呆了一小会儿他们又移动了,我才能跟过去。在我远远等着他们走开的时候,我总是用手去抠刷了白石灰的树皮。我对他们又讨厌又妒忌,所以总是暗暗地移过去,伸手在他们掏过的地方再掏一掏。我总希望最好能剩下一只没被发现的知
\newpage
了,好像一个披着盔甲的小鬼怪一样,我把手伸下去,又想碰到又怕碰到,直到现在我还能想起那种感觉,我记不起究竟我是否在那个夜里摸到过一个死知了

知了是个奇怪的东西,它从地下爬出来,用假眼睛看你,总有些棺材的味道。有一次看《辞海》我见过古代有一种玉制的琀,就是死人含在嘴里用的,样子极其简单、淳美,我甚至感到货币应该是这种样子,我一次次走近自己害怕的事情,我喜欢那个地底下的知了和琀。我溶化了铅,用泥巴做了模子,想把它铸造出来,我喜欢这种古老、光华像蛹一样的东西。它在桃树上爬,紫红紫红的桃树吐着透明的胶液,我看着它向前走了七步就停住了,背一点点儿裂开,眼睛空了,像一个泡被阳光照着透明,我离开一会儿,回来时它已经出来了,它从自己的壳里走出来。那个新鲜的淡绿色的知了美极了,比一片叶子还要新鲜
,我不敢呼气。在空了的壳里有纯白的经络。 

生命一次次离开死亡、离开包裹着你的硬壳,变得美丽。我也想离开自己获得再生,我跟着你好吗,在一个早晨,直到我落在桃树上的壳被别人捡走。
\newpage

顾城 1979年9月12日 



你说的是挺好的事:跟着,跟车子、跟人、跟奇怪的声音、冰糖葫芦、卖豆腐的,什么都跟,到冬天下大雪就出去跟脚印,挺害怕也挺高兴。我跟过一种带花的脚印,一溜儿轻轻转弯,绕过荆棘到山上去了,我总和别人争论那是什么,是黄鼠狼,还是狐狸,当然不是院里明婶家的老黑猫。最好是一种比较可
怕的东西——鬼装的或者索性是老灰狼站起来了。 

你跟着我当然不坏,可你知道我在跟什么呢。
 小烨 1979年9月中 


七、 



月亮升起来了,多亮呵,没一丝浮尘,没风,夜是灰蓝色的,冷冷的空间,月亮是圆的,你那么远
\newpage

,我却仍然能把手伸向你,看见你。 

小烨你离我很近吧,在这无法触及的无际的虚空中,千里万里也是微不足道的,你在笑在看、祝福……我好像在你明亮的呼吸中溶化了,不再是一个笨拙的人,我是一阵又一阵风吹着风铃,你会着凉的。
12点了,梦是一个美丽宫殿。 


12点 

人永远在看、在想,总有忧愁。我从来没有像现在这样充满了活下去的渴望,我好像在虚伪肮脏的海中漂了好久,终于看见月亮一样干净的海岸,我要到那去,要见到你。我的手被沉甸甸的海藻缠绕着,
暗暗地计划着,我知道微微退一下,海就会消失。 


1点 

中秋是我最喜欢的节日,因为离我的生日很近,它能使我想起最初的日子。我好像是从月亮的圆窗里跳出来,踏着积水来到村里、来到这个世界上。这
\newpage
个世界上有许多东西,城堡和道路,还有个小烨刚刚把头发盘起,她在好多田野上跑过,现在她丢下的那
些田野让月亮照着。 


2点 

我说“咱们走吧”。你说“怎么走呀?”我摘下一根草茎,在你手心写一个谜,一个永远猜不到的谜——没有谜底。你还在问“怎么走呢?”一本正经的庄稼已经移动了,我们已经在走了,你还想问呢?前边是大地的尽头,风吹起你的头发,像海燕一样飞舞,你的眼睛比大海还深。我回答了,我回答的时候,潮水总在遥远的地方,一次次描单调的花纹。 顾
城3点 



我开始过生日,一边过生日,一边长牙,牙一痛我就倒在床上,高兴极了,因为这样就不能算虚度光阴。痛呀,痛呀,痛得我心底坦然,以至于我生怕不痛了。我在想怎么还没有你的信呢,你微微一笑,
\newpage
肯定是不告诉我的意思,你一笑就把我挡住了,让我没法到那后边去。我总以为我使劲一想,就能弄清楚那是怎么回事。好多事情瞪着眼睛看它发生,可一到那就没有了,周围是蓝蓝的空气,什么缘故也没有,
多奇怪。 

一边过生日、一边牙痛,一边看了看窗外。我的窗外竟有三片树叶,我好像一夏天只看见这三片树叶。我写信给江河,我说我整个夏天只看见三片树叶,他就感动了,放下手头的伟大工程急急地跑来看我

他是个很有趣的家伙,看他的诗老容易把他想象成青铜像。看他开会抽烟的侧影,脸微微往下拉着,也令人肃然起敬,可是在家里就不一样了。他的家像一个洞穴,灯就像会发光的虫,他非常合适地坐在里边,和众多的朋友嘻嘻笑笑,因为没有一样的椅子,那些朋友坐得高矮不一,然后每天早晨他都带着好脾气扫地。他挺爱扫地,作为他爱清洁的标志,还有什么可干的,他就搞不清了,所以除了地上干净,别
处都很乱。 

\newpage

他来了,非常自然地吓唬我,让我别活得太高兴,说要对自己有所设计,要负责任:“你拒绝长大并不是一个办法,等到心劲一消你就傻了,谁都得老
。”他说着露一根白头发,又偏过头去看树叶。 

我不管,我有一个秘密,一个法宝,那就是你。一想你,这个世界就没辙了,三片树叶呀、白头发呀都没办法!一块块摞起来的理论、文学史也没办法。我们早就从课堂里偷偷跑出去过了,明天还要去,明天是你的生日吗?我把你的生日忘了,一只手伸在
蓝空气里,怎么也想不起来。 

一个最重要的事…… 顾城 1979年10
月 



这回你吹牛了,你正式23岁了,祝贺你。可你说,你忘了我的生日。我没告诉你,你就“忘了”?真能耐呀!当然现在我不会让你想起我生日的,以后再告诉你。能想起来的事都会忘,就像树叶会掉一
\newpage

样,因为在身外,一松手就没了。 

江河能看见几片树叶呢? 小烨 1979年
10月 


八、 



我不知道现实是什么,有的时候,它就像小毽子跳来跳去,在尘土中消失,可铃一响,我们又坐在它下面了。现实巨大的屋顶笼罩在我们头上,我们甚至在走过时相互看看都不可能。日光灯“嗡嗡”响着,使人变得迟钝。生存,“老师”举起手指说。生存成了存在本身。生存都是以不生存为前提的,你要变成工具、文字、齿轮,你要为将来牺牲现在,将来成为现在你还要牺牲下去。这道题非常奇怪:当人们在生存的过程中寻求的时候,他们把答案推给目的,而当人们在目的中寻求的时候,答案又回到过程之中,于是存在只剩下了令人沮丧的三个字:“活下去”。

\newpage

为了避免无聊,人们又想出要活得好些,要一级级升上去,要积攒,要在各种莫名其妙兴起的潮流
间奔跑,而且得相信从来如此,别无它路。 

我们叫“天”的时候,我们就是它遗弃的滚滚
泥沙。 

我也会渴,也会饿,可我仍然一直怀疑:这个生存是否确有其事,是神经的错觉,还是哪本书里编出来的。一本本书摞在那让人相信。那些老先生把现实和真理混在一起,把诗和红烧肉混在一起,好像想躲开什么,他们一定是想躲开什么。我还不懂,但我知道我一定会知道,一定会从这个布置好的会场中间走出来,就像过去,我忽然从几百人整齐的队列中走出来一样,一直走,走出门。 顾城 1979年深
秋 



你的信永远出乎我的想象,我希望你有的,你从来没有。(不过我自己也弄不清我希望些什么。)
\newpage


哲学是一种折磨人的东西,听你说说也许还能算是一种享受,可变成了文学,对我来说简直就成了溶化不了的一滩墨迹。我相信将来除了我有弄明白这些话的可能以外,不会再有人懂得你说的是什么了。

晚上星星都死了,只有一个月亮挺不好看。 烨 1979年10月

\end{document}
