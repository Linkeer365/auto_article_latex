\documentclass{article}
\usepackage[utf8]{inputenc}
\usepackage{ctex}

\title{绿螳螂\footnote{Click to View:\url{https://web.archive.org/web/20210715052304/https://www.zhihu.com/question/276193759/answer/491224580}}}
\author{余雷}
\date{2007-07}

% \setCJKmainfont[BoldFont = Noto Sans CJK SC]{Noto Serif CJK SC}
% \setCJKsansfont{Noto Sans CJK SC}
% \setCJKfamilyfont{zhsong}{Noto Serif CJK SC}
% \setCJKfamilyfont{zhhei}{Noto Sans CJK SC}
% \setlength\parindent{0pt}

\begin{document}
\CJKfamily{zhkai}

\maketitle


\Large

青峡镇的所有店铺中,有一个店从不关门。
这个店就是刚开张一个月的老朱粥店。 

青峡镇的治安一直不错。这是因为青峡镇的民风向来很好,而且刘知县对镇上的治安工作特别重视。但各家各户睡觉前还是会关闭好门窗。因为刘知县提醒大家,即使没有小偷来偷东西,野猫野狗跑进去
也是不好的。 

刘知县专门让打更的钟爷去提醒过老朱粥店的朱老板,每晚都要关门关窗户,小心火烛,防火防盗
。 


但朱老板还是没有在晚上关门。 

\newpage

刘知县就亲自到老朱粥店去了。老朱粥店现在是青峡镇最热闹的地方,那块写着“老朱八宝粥,喝过还想喝”的红绸布前,等待喝粥的人排成了长队。
 

看到刘知县矮小的身材出现在店门口,朱老板连忙跑过来:“对不起,对不起,小的该死。小的让知县老爷操心了。”他指了指店门口排队等待喝粥的队伍,“您看,不是我不关门,是我整天忙得连开门关门的时间都没有。他们喝完粥回去了,我还要接着
熬粥。没有时间嘛。” 

青峡镇的店铺都是老式木门。这种门的特点是:墙就是门,门就是墙。商铺临街的一面墙壁是用一块块门板连接起来的。开门和关门都要把门板一块块紧挨着放好。老朱粥店的门板少说也有三四十块,开
门关门确实很费时间。 

老朱粥店虽然只开了一个月,但粥店的生意好得出奇。粥店只卖八宝粥,正像红绸布上写的那样,每个到过老朱粥店的人,喝了“老朱八宝粥”,就“
\newpage

喝过还想喝”。 

刘知县觉得“老朱八宝粥”是他喝过的最好的八宝粥。自从老朱粥店开张以来,刘知县常常会闻着八宝粥的香味不自觉地走到粥店去。青峡镇上没人知道,这个身材矮小,做事勤勉的刘知县最喜欢的食物就是八宝粥。八宝粥那糯软醇香的滋味能让他忘掉一
天的疲累,模模糊糊地想起一些快乐的往事。 

但粥店的门口总是排着长长的队伍,衙门里总是有那么多的事情要等着他处理。所以不是每次去老
朱粥店刘知县都能喝到“老朱八宝粥”。 

没有喝到“老朱八宝粥”的那天,刘知县会难受得睡不着觉。第二天办事的时候就无精打采。好在刘知县带着儿子阿华去过一次后,阿华每天睁开眼睛就嚷嚷着要喝“老朱八宝粥”。刘知县同意阿华每天去喝“老朱八宝粥”,唯一的条件是,阿华每次吃完后,要给刘知县端一碗回来。阿华虽然觉得麻烦,但
见父亲同意他每天去喝粥,也就答应了。 

\newpage

这天上午,阿华和往常一样,自己喝完一碗“老朱八宝粥”,又让朱老板盛了一碗,放在准备好的
食盒里带回去。 

早上天刚下过雨,路上还有一些积水。街上没几个人,阿华提着食盒,小心翼翼地走在青石板路的中间。忽然,一阵风把阿华掀了一个趔趄,差点儿摔
倒。阿华站稳身子,但食盒却摔到了地上。 

“老朱八宝粥”流了一地。湿漉漉的空气中弥漫着“老朱八宝粥”的香味。过路的人都不由得吸了
吸鼻子。 

阿华还没来得及捡起食盒,又一阵旋风卷了过
来。 

这回,阿华看清楚了,根本不是一阵风,而是
一个人,一个走路像风一样快的灰衣人。 

这个人扔给阿华几个铜板:“不好意思。刚才

\newpage
把你撞倒了。” 

听口音,灰衣人不是本地人。阿华看不到他的脸,他的脸被一顶帽檐很宽的紫色毡帽遮住了大半。阿华只能看到他的嘴在动。说完话,那张嘴的嘴角竟
然流下了一绺口水。 

“这就是‘老朱八宝粥’吧?”灰衣人蹲下身子,不顾他那身做工考究的灰衣服是否沾上了泥水。他像一条大狗一样用鼻子“呼哧、呼哧”地闻了闻地上的八宝粥,起身一把抓住阿华:“我终于找到它了
!告诉我,哪里能买到这样的八宝粥?” 

阿华刚抬手指了指老朱粥店的方向,那人就像
一阵风一样,马上从阿华面前消失了。 

刘知县一直坐在前厅等阿华回来。看到阿华手里的空食盒,他连忙问:“今天的八宝粥全卖光了吗
?” 

阿华垂头丧气地说:“别提了。今天遇到一个奇怪的人,先是把我给您买的八宝粥弄洒了。然后,
\newpage
他又把老朱粥店的八宝粥全买了。弄得大家都没有粥
喝。许多人还在粥店门口等着朱老板熬粥呢。” 

“他怎么可能全买走呢?朱老板是用五口大锅
熬粥啊。” 

“说了您也不信,那个人把他身上的灰衣服脱下来,像一只大口袋一样,把所有粥倒在里面,背着
就走了。” 

“有这么奇怪的事,我倒要去看看。”刘知县
说着就往老朱粥店走去。 

老朱粥店非常热闹,门口等着喝粥的人比过年看戏的人还多。朱老板满头大汗地跑来跑去,劝大家明天再来,但谁也不愿意。有人说:“我好不容易排到了,现在回去明天还要重新排。我不走,我就等在
这里。你什么时候熬好了我就等到什么时候。” 

朱老板擦了擦额头上的汗水,无奈地说:“你

\newpage
们在这里等,我做不出来的。” 

“不会吧,朱老板,你是不是在粥里放了什么药,让我们吃了还想吃,要不,为什么大家看着你做,你就做不出来?”等着喝粥的人提出了自己的疑问
。人群乱成一团,把朱老板包围在中间。 

朱老板头上的汗水更多了,他使劲摇晃着两只手:“没有,没有,绝对没有。我怎么可能在粥里放药,那是伤天害理的事。我老朱敢对天发誓,我的粥
里要放了药,我就天打雷劈,不得好死。” 

刘知县的个子实在太矮小了,他站在人群外面,几次想走上前去,都被别人的背脊挡住了。他看到路边有一个大石墩子,连忙爬上去,对喧闹的人群喊道:“好了,好了。我来说句公道话,大家说朱老板粥里放了药,有没有证据?”大家都不吭声了。刘知县又转向朱老板:“朱老板,你与其发誓没有放药,不如当众煮一回粥,让大家心服口服。不就一锅粥吗
?最多几个时辰就可以了。” 

朱老板的头摇得像波浪鼓一样,他跑进店里,
\newpage
把五口大锅的锅盖一一打开,哭丧着脸说:“不是我不当着大家的面煮,是我没有办法煮。我们‘老朱八宝粥’历来的规矩是,五口大锅的粥分批煮,第一锅卖得差不多,第二锅就好了,第二锅卖完,第三锅正好……谁知刚才那个人把五锅粥一次都倒走了,现在
我不知道先煮哪锅,我实在没有办法煮啊。” 

刘知县跳下石墩子,人群自动分开,给他让出一条路。刘知县走进粥店,指着面前的一口锅说:“你就把这口锅当第一,第一锅煮得差不多你再准备第二锅,第二锅差不多再准备第三锅……这样不就好了
吗?” 

朱老板还是不愿意:“您是知县,管着青峡镇不假。但这锅的事您说了不算,我不能坏了规矩。”

刘知县问:“那你找不出第一口锅,你的粥店
不就开不下去了吗?” 

朱老板说:“那倒不是。每口锅有不同的声音,第一口锅因为煮粥的时间最长,敲起来的声音不一
\newpage
样。”大家着急地说:“那你就赶快把它找出来啊。
” 

朱老板说:“现在找不出来,要等到夜深人静的时候,没有一丝声音,我才听得出来。所以让你们
明天再来嘛。” 

原来如此,大家明白了。但现在离天黑还早,许多人都不愿意回去,就说:“朱老板,我们帮你干
点儿什么吧。” 

朱老板笑着说:“不用了,准备煮粥的材料我一个人就够了。”只见朱老板端出几大箩筐糯米、红豆、花生……放在五口大锅前,深深地吸了一口气,抓起一把花生,一粒一粒往锅里扔。朱老板的速度实在太快,扔出的花生就像一条线从他的手里一直连接到锅里。更神奇的是,如果有霉变或者干瘪的花生,
就会自己从这条线上掉下来。 


大家齐声喝彩。 

\newpage

有人问:“朱老板,糯米你也这样一粒一粒拣

朱老板没有回答,他用两只手一边抓起一把雪白的糯米,一只手放到锅上,轻轻一吹,雪白的糯米像雪花一样飘洒下去,几粒稗子和砂子留在手心里。这只手翻转下去扔掉砂子,又抓了一把米的同时,他已经在吹第二只手上的糯米了。朱老板轮换着吹两只手上的糯米,只见他的手臂上下翻飞,糯米就成了一
片白色的雪雾。 

所有人都看呆了。想不到,“老朱八宝粥”的
材料竟是这样挑选出来的。 

刘知县没有和任何人打招呼,悄悄回到了县衙。刚才出门时倒好的茶水还有些温热,他喝了一口,
自言自语地说:“没事就好啊。” 

“你怎么知道没事呢?”一个低沉的声音在门
外响起。 

刘知县并不感到意外,站起身说:“不愧是神
\newpage

捕,我就知道你迟早会来。请进来说话吧。” 

一个灰衣人走了进来,一顶宽檐的紫色毡帽遮
住了他的大半张脸。 

“刚才小儿说起遇到一个奇怪的人,我就想到会是阁下。再听说有人一次带走五锅八宝粥,我就确定是你。天下恐怕只有我们这个门派的人会那么喜欢八宝粥。” 刘知县给灰衣人倒了一碗茶水,“师弟请用茶,我们最后一次见面到现在已经有三年了吧?

“师兄还是那么敏锐。以你的资质,应该可以做到四大名捕。”灰衣人摘下帽子,恭敬地接过茶水

“过奖了。那次练功受伤后,我不仅再也不能长高,功夫也大不如前。一个没有功夫的捕快怎么去
抓贼?”刘知县笑眯眯地看着灰衣人。 

“但是你的胆识和谋略在我们师兄弟中无人能比。我们原来一起练武,一起喝粥的日子多好……”

\newpage

“我现在这样也挺好。只是辛苦你代替我为师傅到处打听好吃的八宝粥。”刘知县打断了灰衣人的话,“说正事吧,你上午把粥全部倒走,我就猜想你肯定还会回来。师傅他老人家只要一喝就知道这粥的
来历。” 

灰衣人钦佩地说:“什么都瞒不过师兄的眼睛。确实如此,师傅一尝之后,就觉得煮粥的人不是一般的厨师。这八宝粥的材料是用‘寒门吹雪’和‘拈花玉指’的功夫处理过的,所以,八宝粥煮熟以后,粮食的香味更浓,也更加粘稠。难怪会喝了还想喝。

刘知县点点头:“如果我没有猜错的话,今天晚上朱老板选锅会用手臂而不是手指去敲锅。到时候
你再动手也不迟。” 

灰衣人说:“小弟正有此意。师兄到时候会去

刘知县若有所思地说:“会去,我有几句话要
和他说。” 

\newpage

阴天的夜晚没有月亮。三更敲过之后,青峡镇笼罩在一片寂静和黑暗中。这黑暗中还有一点点烛光,那是老朱粥店灶台上的一支蜡烛。老朱粥店开业以来第一次关上了门,但蜡烛光执拗地透过门板照到了外面。朱老板盘膝坐在灶前,双眼微闭,两只手臂交错在一起,举过头顶。蜡烛光把他的影子反射到墙壁
上,就像一只正欲捕食的螳螂。 

朱老板的手臂突然伸长了一般,飞快地敲向一口大锅,铁铸的大锅发出一阵“嗡嗡嗡”的声音。他连续在五口大锅上敲了一遍,五口大锅发出钟鸣般的
声响,在寂静的夜里显得格外雄浑。 

朱老板凝神听了听,待声音停息后,又敲了一次。这一次,没等声音完全停止,朱老板就把五口锅调换了一下顺序。把它们放在灶台上,生起了柴火。

天快亮了,天边出现了一线淡淡的鱼肚白。五口大锅里的八宝粥开始“咕嘟、咕嘟”地冒着热气,
那香味让人恨不得马上喝一大口。 

\newpage

朱老板把桌子擦干净,盛了两碗粥放到桌上。然后一块一块取下门板,把门打开,对着门外大声说:“两位出来喝碗粥吧,冻了一晚上,挺不容易的。

刘知县和灰衣人从藏身的地方走了出来,灰衣
人说:“绿螳螂,听力功夫又有长进了。” 

朱老板笑着说:“哪里,是刚才听见两位咽口水的声音才知道你们在附近。要不早就请你们进来坐
了。” 

刘知县有些不好意思,但禁不住八宝粥香味的
诱惑,抬起碗就喝了一大口。 

朱老板问:“你们什么时候发现我是绿螳螂的

刘知县咽下一口粥,“从喝你的第一碗粥开始,我就怀疑了。如果你在青峡镇有什么动作,我们就
不会今天才来找你。” 

灰衣人说:“我以为你做了那件大案之后会逃
\newpage
之夭夭。所以并没有留心在附近找你。直到今天早晨把你的粥带回去让师傅喝,才想到你就是大盗绿螳螂
。你又是如何知道我们的身份的?” 

朱老板说:“昨天你不由分说把我的粥全部倒走,我就知道有麻烦了。因为道上的人都知道,名捕教头唯一的嗜好就是喝粥。而且能一次带走五锅粥的
人,除了他的徒弟不会有别人。” 


刘知县问:“你为什么不逃呢?” 

朱老板摇摇头:“我已经逃了这么多年了。不是还没有逃出你们的手掌心吗?”他搅了搅锅里冒着热气的八宝粥接着说:“‘老朱八宝粥’是我家祖传的秘方,但我一直不愿意开粥店,一心想干一番大事业。谁知事业无成还误入歧途,成了一个被通缉的盗贼。我之所以在这里开这个粥店,就是打算再也不逃
,踏踏实实重新做人。” 

“虽然你现在有悔改之意,但是你原来犯下的罪行依然不能饶恕。你还是要跟我回大牢去。”灰衣
\newpage

人说着,取出一副镣铐向朱老板走去。 

“慢!”刘知县拦住灰衣人,“你可以带他走,但是不要铐住他。天亮了,镇上的人看到不好。他在这里的身份是朱老板,我还想等他刑满释放后再回来煮粥呢。”刘知县又转向朱老板,“任何人只要把自己的功夫用在正确的地方,就是一个好人。放心,
你不在的时候,粥店我会帮你看管。” 

天色完全亮了,老朱粥店里八宝粥的香味把刚从睡梦中苏醒过来的人们都吸引到店里来了。“老朱八宝粥”依然是让人“喝了还想喝”。不同的是,今天在粥店里忙碌的是刘知县,他一会儿给大家盛粥,
一会儿给锅里添水,一会儿又张罗着买柴火。 


有人问:“朱老板呢?” 

刘知县指了指墙上,大家发现上面贴了几张奇
怪的标语: 


\newpage

“老板有事出门,把店交给大家。” 


“照应粥店,人人有责。” 


“自己动手,才有粥喝。” 

刘知县说:“明白了吧,以后来喝粥就要干活。在朱老板回来之前,我们不能让这五口大锅空了。
大家会干什么,就干什么吧。” 

老朱粥店的门从此再没有关过。

\end{document}
