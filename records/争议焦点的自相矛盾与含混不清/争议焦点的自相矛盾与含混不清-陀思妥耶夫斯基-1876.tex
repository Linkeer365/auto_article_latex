\documentclass{article}
\usepackage[utf8]{inputenc}
\usepackage{ctex}

\title{争议焦点的自相矛盾与含混不清\footnote{Click to View:\url{https://web.archive.org/web/20220630085526/http://libgen.is/book/index.php?md5=0A2665DC7B9E1E006DCDA377D37A9CA7}}}
\author{陀思妥耶夫斯基}
\date{1876-04}

% \setCJKmainfont[BoldFont = Noto Sans CJK SC]{Noto Serif CJK SC}
% \setCJKsansfont{Noto Sans CJK SC}
% \setCJKfamilyfont{zhsong}{Noto Serif CJK SC}
% \setCJKfamilyfont{zhhei}{Noto Sans CJK SC}
% \setlength\parindent{0pt}

\begin{document}
\CJKfamily{zhkai}

\maketitle


\Large

人们直截了当地向我们声明说,人民没有任何真理,真理只存在于文明之中,保留于有教养的人们组成的上层社会之中。为了充分做到严肃认真,我从文明的最高意义上理解我们这种可贵的欧洲文明,而不是仅仅理解为马车与奴仆,正是在这个意义上,我们比人民在精神上、道德上更有教养,更富有人性,更富有人道精神,因而也就完全有别于人民,这是我们的荣幸。在做过上述不偏不倚的声明之后,我就可以开门见山地向自己提出这样一个问题:我们自己确实是那样好,那样尽善尽美地受到文明熏陶,直至可以把人民的文明拋在一边,而向我们的文明俯首致
敬吗? 

在回答这个问题之前,为了合情合理,我仍要排除例如一切有关科学、工业及其他的议论,这都是
\newpage
欧洲可以在我们祖国面前无愧地感到自豪的。排除这一切是完全正确的,因为现在谈的根本不是这些问题;何况这个科学是在欧洲那边,而我们自己,也就是俄罗斯有教养的上层社会的人们,虽然上了两百年的学,可是我们在科学上还不足以大肆炫耀,为了科学而向我们、有教养的阶层膜拜,无论如何还为时过早。由此可见,科学完全不能构成俄罗斯人的两个阶级,即平民与有教养的上层的任何本质的、不可协调的差异,我再重申一次,要是认为我们与人民的主要的、本质的差异就在于科学,完全是不正确的,而且将是一个错误,应该从全然不同的方面去寻找这个差异。何况科学是人类共同的事业,科学并不是欧洲某一个民族发明的,而是包括古代世界在内的所有民族发明的,这是继承性的事业。从自己这方面来说,俄罗斯人民从来也不是科学的敌人,不仅如此,早在彼得大帝之前科学就已经进入我们的生活。早在彼得大帝一百三十年之前,沙皇伊万·瓦西里耶维奇就竭尽全力争夺波罗的海沿岸地带。他如果把这个地区争夺到手,掌握了它的海湾和港口,他必定会像彼得大帝那样建造自己的舰船,而离开科学是不可能建造舰船的,所以必然会像彼得大帝时期那样,从欧洲引进科学
\newpage
。我们的波图金们嘲笑、侮辱我们的人民,说俄罗斯人只有一种发明,那就是茶炊,然而欧洲人未必会加入波图金们的合唱。事情是非常显而易见的,一切事情的产生都有一定的自然规律和历史规律,我们在科学和工业方面的发明创造之所以少的原因并不是因为俄罗斯人民弱智低能和可耻的懒惰。某一种树木若干年就可成材,另外一种树木则要加倍的时间才能成材。一切都取决于人民所处的自然条件和社会条件,取决于人民首先需要什么。这里有地理的、民族的、政治的原因,千百种原因,原因全都是显而易见的和确切无疑的。任何一个头脑健全的人都不会责备和羞辱一名少年,只是因为他才十三岁,而不是二十五岁。有人说,“欧洲比消极的俄罗斯人更富有进取心和更聪明机智,因此,欧洲发明了科学,而俄罗斯人却没有。”但是,在欧洲那里发明科学的时候,消极的俄罗斯人却做出了毫不逊色的令人惊讶的行动:他们在创造帝国,并且自觉地完成了帝国的统一。为了推翻凶残的敌人,他们花了整整一千年的时间,假如没有他们,这些敌人就蹿进欧洲去了。俄罗斯人开拓了自己辽阔的边疆,俄罗斯人捍卫并巩固了自己的边疆,其巩固的程度是我们、有教养的人们即使在今天也是
\newpage
达不到的,刚好相反,我们大概还在动摇其巩固。结果是,在经过千年之后——在我们这里形成了帝国,在世界上无与伦比的政治上的统一,以致英国与合众国这两个现在仅存的有着牢固的、独特的政治统一的国家,大概在政治上的统一上也较我们大为逊色。而与此相反,在欧洲,在另一种政治和地理的情况下,科学却发展起来了。然而,随着科学的成长与巩固,欧洲的道德和政治状况则几乎普遍动摇了。可见,每个民族都有自己的特点,谁将羡慕谁尚未可知。我们无论如何都要掌握科学的,而欧洲的政治统一将会如何,尚难预料。可能,在十五年之前德国还宁愿付出自己一半的科学荣誉换取在我们这里早已实现的那种政治统一的力量。现在德国人只是从自己的理解来看,也达到了牢固的政治统一,但在那个时候,他们还没有日耳曼帝国,当然,无论他们如何蔑视我们,他们在心中还是羡慕我们的。可见,应该提出的不是关于科学,也不是关于工业的问题,问题实际上是关于我们、有教养的人们,从欧洲回到俄罗斯的时候,在道德上,实质上有什么比人民高尚之处,我们给人民带回来哪些东西作为我们的欧洲文明不可企及的宝贵财富呢?为什么说我们是纯洁的人们,而人民依然还
\newpage
是肮脏的人呢,为什么说我们就是一切,而人民则一文不值呢?我认为,在这个问题上,在我们、有教养的人们里面存在着极大的混乱,在“有教养的”人们里面未必有谁能够正确回答这个问题。相反的是,各有各的说法,有关松树成材的年限为什么不是七年,而是比这多七倍的年限之类的嘲笑更是司空见惯,这类话不仅常常出自波图金们之口,而且常常出自在教养上远远高于波图金之流的人们。关于阿夫谢延科先生我就不想提了。下面我要直接谈谈上文提出的问题:我们是否真的那样好,我们受到的教养是否至善至美到可以拋弃人民的文明,而向我们的文明俯首致敬呢?如果说我们真的带来了什么,那么,带来的是什么呢?对此,我要直接回答说,我们远远不如人民,
而且是几乎在一切方面都不如人民。 

人们说,在人民里面只要出现积极人物,立即就变成吸血鬼和骗子。(不只是阿夫谢延科先生一个人这样说,其实,总的说来,阿夫谢延科先生任何时候都未说过别人未说过的话。)第一,这话不对。第二,在有教养的俄罗斯人里面难道不是也时时刻刻都在出现吸血鬼和骗子吗?其实大概还更多,由于他们
\newpage
是受过教养的人,这也就更为可耻。人民则是没有受过教养的人,不过,主要的是,关于人民根本不能说什么在人民里面只要一出现积极的人物,大多数都会变成吸血鬼和骗子。我不知道,讲这种话的人是在什么地方长大的,我从童年时代起直到整个一生所看到的则完全是另外一回事。我记得,我还只有九岁的时候,有一次,在复活节的第三天的傍晚,大约五点钟,我们全家,父母和兄弟姐妹们都围坐在圆桌旁喝茶,谈论的正是关于农村的事以及我们要怎样到农村去度过夏天。房门突然开了,在门口出现了我们家的仆人,刚刚从农村来的葛利高里·瓦西利耶夫。主人不在的时候有时就委托他管理农庄,现在突然间出现的不是那个总是身穿德国式常礼服、仪表庄重的“管事”,而是身着旧粗呢短上衣、脚穿草鞋的人。他是从农庄步行走来的,一进屋就站立在屋中间,一句话也
不说。 

“出什么事啦?”父亲惊恐地问道,“你们看
,这是怎么回事?” 

“庄园烧了!”葛利高里·瓦西利耶夫高声低
\newpage

沉地说。 

此后的情景我就不叙述了;父亲和母亲都不是富有的人,从事劳动的人——复活节竟得到这样的礼物!原来是全烧光了,片瓦无存:住房、粮仓、牲畜棚,还有春播的种子、一部分牲畜、农民阿尔希波。头脑里浮现的第一个可怕的思想就是:彻底破产了。我们一下子全都跪到地上祈祷,母亲哭了起来。这时,我们雇用的奶妈、属于莫斯科小市民的自由人阿莲娜·弗罗洛芙娜突然走到母亲跟前。她是把我们这些孩子抚养大之后走的。那时她四十五岁上下,性格开朗、活泼,总是给我们讲那些优美的故事!她已经很多年不拿我们的钱,“我不需要”,她的工钱积蓄了五百卢布左右,全都存在当铺里,“等到老了再用”
。现在她忽然对母亲嘟哝说: 

“您要是用钱,就把我的钱拿去吧,我要钱做
什么,我不用钱……” 

我们没有用她的钱,日子也过去了。不过,这里有这样一个问题:这位普通的女人属于哪一类人呢
\newpage
,现在她已经死了,死在养老院,她的钱在那里有了大用场。我认为,不能把这样的人划进吸血鬼和骗子里面去,如果不能,那么应该怎样解释她的行为呢:她和她的行为是否仅仅是“处于自发存在的水平上,处于封闭状态的、安闲的生活和消极生活的阶段上”,或者说,她表现的是某种比消极性更坚毅的东西?听一听阿夫谢延科先生如何说明这个问题倒是令人非常感兴趣的。人们会轻蔑地回答我说,那是个别现象;但是我一个人就能够从我自己的生活中举出我们普通人民中间的千百例这类现象,同时我确切地知道,也有另外一些不蔑视人民的观察者。你们还记得吧,在阿克萨科夫的《家庭纪事》中,母亲流着眼泪恳求农民从春天薄冰上把她送过宽阔的伏尔加河,到对面的城市喀山去看望生病的孩子。已经有几天没有人敢踏上冰层了,仅仅在几十小时之前在渡口地方冰还破裂了。你们还记得关于这次过河的那段精彩描写吧,过了河之后农民们不愿意要钱,因为他们明白,这是为了流泪的母亲,为了我们的上帝才做的事。这件事发生在我们农奴制的最黑暗的时代!难道这都是个别事件吗?如果这也值得赞美的话,——那也仅仅是“处于自发存在的水平上的东西,都是处于封闭的、安
\newpage
闲的生活和消极生活的阶段上的东西”吗?由于同情母亲的悲痛而不顾自己生命的果敢冒险这件事情的出现,而且是在农奴制最黑暗的时期的行为,难道不是由于人民的真理,难道不是由于仁慈和宽宏大度以及人民的豁达观点吗?你们会说,可是人民是没有信念的,他连祷词都不会念,他向木板膜拜,还嘟哝什么关于神圣的礼拜五、关于弗洛尔和拉弗尔的无稽之谈。我对你们的回答是,所有这些思想之所以出现在你们中间是出于对俄罗斯人民的长期蔑视,在俄罗斯有教养的人物中固执地保持着这种蔑视。关于人民的信仰,关于人民的东正教,我们有二十来种自由主义的、荒诞的笑话,我们以神父如何要老太婆忏悔或者农夫如何祷告礼拜五这类冷嘲热讽的笑话来寻开心。阿夫谢延科先生如果真正明白了他所写的关于拯救俄罗斯的人民信仰,而不是从斯拉夫派那里抄袭来的,那时他立即就不会侮辱人民了,不再把人民几乎全都说成是“吸血鬼和恶霸”了。然而问题就在于这些人对东正教一无所知,因而对我们的人民也就从来都毫无了解。人民虽然没有上过学,他们关于自己的上帝基督的知识,可能还超过我们。他们知道得多,这是因为他们世世代代经受过很多苦难,在自己受苦受难的
\newpage
时刻,从一开始到现在他们时时都从自己的圣徒们那里听到关于自己这个上帝基督的事,这些圣徒为人民而活着,为捍卫俄罗斯的土地而斗争,直至献出生命,人民直到今天还崇敬自己那些圣徒,铭记着他们的名字,在他们的墓前祈祷。请相信,在这个意义上我们人民的最落后的部分所受的教养远远超出你们所想象的,这是由于你们在教养上对他们一无所知,而且有可能,他们比你们自己更有教养,尽管你们学习过义手册。

\end{document}
