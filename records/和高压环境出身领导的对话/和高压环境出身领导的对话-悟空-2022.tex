\documentclass{article}
\usepackage[utf8]{inputenc}
\usepackage{ctex}

\title{和高压环境出身领导的对话\footnote{Click to View:\url{https://web.archive.org/web/20220621111356/https://zhuanlan.zhihu.com/p/524236049}}}
\author{悟空}
\date{2022-06-04}

% \setCJKmainfont[BoldFont = Noto Sans CJK SC]{Noto Serif CJK SC}
% \setCJKsansfont{Noto Sans CJK SC}
% \setCJKfamilyfont{zhsong}{Noto Serif CJK SC}
% \setCJKfamilyfont{zhhei}{Noto Sans CJK SC}
% \setlength\parindent{0pt}

\begin{document}
\CJKfamily{zhkai}

\maketitle


\Large


A:“L总好,办公室的椅子坏了。” 


L:“嗯?~你这话什么意思???” 

A:“哦~没什么,就是办公室的椅子坏了,
要处理掉了……” 


L:“那你的意思是说这个椅子有问题。” 


A:“对,椅子坏了,轮子掉了。” 

L:“哦~那我问你,椅子坏了你为什么要告
诉我?” 


\newpage

A:“您是领导,就是给您汇报一声~” 

L:“你给我说一声、给我提交问题??……你让我知道这个是想说明什么呢??嗯???是想说
买椅子的有问题、还是说造椅子的有问题??” 

A:“不是有问题,就是椅子坏了,给您说一
声就要处理了...” 

L:“谁让你处理了?我怎么不知道有人让处理了???现在是问你想说明什么问题,怎么又谈到
处理上去了??想岔开话题???” 

(鼻子哼了一声怪腔,作语重心长状)“我告诉你啊小A,不要耍这种小聪明,你还年轻,正是学正经本事的时候,这么小就想耍心眼可不行,别的不说,咱们单位这些老同志为组织工作这么多年,什么样的事儿、什么样的人没见过?人家一眼就看穿了你~~~较起真儿来,这以后都会影响你进步的,知道
吗……” 

A:“好的L总,那我以后注意,就是制度规
\newpage
定,办公用品坏了给您汇报一声,然后就要正常处理
了。” 


L:“什么制度?” 


A:“就是管理办公用品的制度……” 

L:“我怎么不知道有这么个制度?谁定的这个制度?你是想说有人定了这么个制度需要我执行,
想把谁抬出来压我???” 


A:“不是这个意思……” 

L:(打断A接话)“不是这个意思?那是哪个意思??嗯?~还是说你觉得你的话就可以当作制度?那你A面子大得很呐,我在组织里服务这么多年
,都没敢说定什么制度……呵呵~~~” 

(点烟,眯眼)“那我得叫你A总啊,A总指示我处理椅子,那我得听命令签字,是不是?嗯??

\newpage
” 

A:“哎呦~L总,千万别开玩笑,我就是过来请示您的,这个椅子应该怎么处理您点拨一声,我
马上去执行,这样可以吧?” 

L:“别~~什么叫我点拨?我没点拨。我告诉你啊,这个事儿和我一分钱关系都没有,是你过来说要按制度处理这把椅子,是吧,你要对自己的话负责,不要动不动就往别人头上扣帽子,那还了得……

(往后靠到椅背上,抽一口烟)“更何况我还是你名义上的上级,对吧~哪能好事儿都是你的,坏事儿都是人家的,人不是这样做的,知道吧?~老抱着这种占小便宜的想法,走不远~~~(后续各种案
例和爹味说教)” 

A:“好的L总,我知道了,那我这先回去?

L:“着急什么,坐会儿,还没聊完呢……”
 

\newpage

(喝口水)“这个小A啊,咱们在这里虽然名义是上下级,就是说在这个单位~~~但是呢,我不认为有了这层上下级关系,两个人就应该那么严肃、那么保持距离,对吧,没那个必要,出了这个单位,大家在社会上都是平等的,你说是吧?谁会那么上赶着去拿人当领导啊,我又不给你钱花,你说是不是?
哈哈~~” 

A:(恭维一下)“那不会的L总,您是领导
,在哪儿都一样尊重您。” 

L:“你不用这么说,实事求是。当然,我这属于返聘,奋斗了大半辈子了,社会成就可能比你稍微高一些,在一些方面的关系更成熟一些,以后呢,如果有可能、机会合适的话,我还是会尽量的帮一帮身边的人,对吧,尤其是像你这样,还年轻、还需要进步,现在的社会压力又大,机会能争取一点是一点
……” 

“其实我们当年环境好得多,我刚大学毕业的时候……(进入单位的过程略)……那个时候也年轻
\newpage
,想法比较简单,领导让干什么就干什么,也吃了很多亏……有好领导就有坏领导嘛,这个也很正常,所以人有时候也不能太实在,一些事情上呢,不能光一股脑儿的听安排,就比如你今天说这个事儿,对吧,是你自己要说这些?还是有谁让你过来告诉我的?…
…” 

A(L在这个地方故意停顿,于是赶紧表态)
:“L总,这件事绝对……” 

L(赶紧打断):“你不要那么着急表态,这个事儿我不需要知道,因为我目的不是要调查你什么,我肯定是相信你的,要不然也不会给你讲这么多,对吧~~~我是在给你讲一些道理,能听懂听不懂的
呢,还得看自己……” 

A:“那感谢您指点,以后还是得跟您多学习
~~~” 

L:“哎~别~~~我没指点你,你也别跟我学习,要是每个人都让我说两句,那还不把我自己给
\newpage

累死。” 


A:“……” 

L:“这个事儿确定就是告诉我,不需要我再
去和谁说一声是吧?” 


A:“真不用,这个之前处理挺多的” 


L:“那行,你回去吧。” 


—————————————— 


还是自我解读一下吧,要不看不出味儿来。 

这篇文章是为了另一个回答准备的素材。内容来源,是一位体制内领导L(放到地方上相当于副省
级),发生过的对话(经修饰、主要意思还在)。 

机关单位“非升即走”,也就是一百个人坐办公室,假设说三年一考核,到期按照比例,一部分人
\newpage
升一级、另一部分走人,“升”和“走”两种结局天
壤之别。 

岗位没有日常工作,就看领导是不是给你表现和立功的机会,这种环境决定了,他们每个人每天就琢磨两件事儿:一,怎么赢取领导信任;二,怎么整走身边同事。无论其中哪件事成功了,自己留下来的
概率都会大大增加。(单位里的人自己描述如此) 

能留到最后的,其他方面且不讲,斗争绝对能
力一流。 

在较长一段时间的共事中,通过观察他们遇到问题的处理方式,加上他们互相拆台、彼此揭露,还有一点点的学习和请教,我将他们的斗争策略总结成
几个意识: 

1.斗争意识:所有行为都是一种斗争的试探

比如说,我告诉你,这杯奶茶不好喝,你可能

\newpage
觉得我就是单纯的在说奶茶不好喝。 

但是他们会迅速延伸到:你为什么说这杯奶茶不好喝?你是想借机贬低买奶茶的同事?为了拉低我对他的印象、增加你的上位几率?还是你想激起我不满的情绪?你激起我的情绪是为了什么?为了让我和买奶茶的同时发生矛盾、你坐收渔利?还是你想诱导
我表达不满情绪好拿我把柄? 

你想通过这些话来让我相信什么?诱导我去做什么?是你自己有的这些想法?还是有人指使你这么
说?谁会指使你这么说?…… 

是谁要通过这个方式斗我?或者是谁要利用我
斗别人?一切都是斗争! 


2.自保意识:遇事先撇清自身 

这个单位的所有人,无论发生什么事情,他们第一反应就是先把自己撇出去,撇的清清白白、干净
利落,简直形成了甩锅反射。 

\newpage

比如我说,这杯奶茶不好喝。他们第一反不会是讨论奶茶好不好喝,他们没有勇气去思考这个话题为什么会发生,或者说他们那个环境中,不允许他们去试探一个话题背后的分量,那些具有质疑和讨论天
分的早已被淘汰…… 

他们绝对是先撇清自己和奶茶的关系,“我早就说过不要买奶茶”“为什么不是买咖啡”……这都是低级的;“有人喝了奶茶情绪非常不好”通过起哄模糊自己的立场,“你们为什么会决定买奶茶”感觉像是询问实则把自己轻描淡写的撇开……总之是通过
各种胡搅蛮缠来模糊自己和这件事的关系。 

即便后面事情翻转,比如更高级别领导说,奶茶买的不错,他们也会毫无心理负担的迅速换个角度陈述自己的观点“有人喝了奶茶情绪非常不好,怎么就他个别”“你们为什么会决定买奶茶,还允许个别
人有意见?”…… 

3.拆台意识:努力往身边所有人身上泼脏水

\newpage


比如我说,这杯奶茶不好喝。 

他们会把这种正常的感官表达,进行极度的曲
解。 

“买奶茶的时候不说话,现在说不好喝,其实就是对买奶茶的同事有意见,事后报复真小人”;或者更好是同时打压两个人,“这俩人就是有私下矛盾,不会处理同事关系,都不成熟”;“这个人说奶茶不好喝,就是端起碗吃饭、放下筷子骂娘,忘恩负义”;“这个人说奶茶不好喝,就是不服从集体决定,
净一堆小九九”…… 

表面里大家有说有笑,背地里都想尽一切可能从最卑鄙、最恶劣、最下作的角度,来解读其他人的行为、语言,必须要让所有人的形象都是负面的,无
论是对平级还是下级。 

如果没有背景的话,只有最高领导可以有一些正面形象,其他人哪怕想着我只做一个正直、善良的人,什么机会都不要,这样都不行,必须要给你的身
\newpage
份画上一个严重的缺陷,并且要通过一切手段施压让你承认这个缺陷、巩固你有问题、但又不致死的形象,这样的你对领导来讲才没有威胁、才安全。否则在你有可能列入候选之前就要出局,理由是自视甚高、
不懂得深入群众。 


4.奴役意识:经常测试服从性 

比如什么“出了这个单位大家都一样”、“在XXX方面还是你比我厉害...”“Y领导是比我强...”如果你没接住这一茬,及时的装孙子、舔臭脚、表姿态……那未来的几天你会发现自己的工作
开展的无比艰难。 

类似于上面对话,L的思路就是:无论椅子这件事儿处不处理,和我都没有关系,哪种情况出了问题,我都能撇清干系,这样我才能保证主动权。及时的撇清和你这个人的关系,否定你从我这儿学习过什么、我指点你什么,这样无论你工作好与不好,都不占任何因果。即便我要通过你问一些事情,我看到你态度就知道你答案,就是不能让你说,你说了,我问
\newpage
你这个事儿就落实了,你没说,那就是我一种善意的
提醒而已。 

另外通过一次简单的对话,给A安上几个诸如小聪明、心机重、不服管……这样的帽子,这样才能拿捏你,如果你听话,这些都不是问题,如果你不听话,所有人都会知道你人性恶劣、不堪大用,并且讲
的有理有据。 


还间或穿插了一些服从性测试。 

此篇文章不附加态度,只是分享一下有趣的经。

\end{document}
