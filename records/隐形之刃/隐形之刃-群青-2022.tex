\documentclass{article}
\usepackage[utf8]{inputenc}
\usepackage{ctex}

\title{隐形之刃\footnote{Click to View:\url{https://web.archive.org/web/20221113135553/https://linkeer365.github.io/Linkeer365ColorfulLife3/990170083/}}}
\author{群青}
\date{2022-08-09}

% \setCJKmainfont[BoldFont = Noto Sans CJK SC]{Noto Serif CJK SC}
% \setCJKsansfont{Noto Sans CJK SC}
% \setCJKfamilyfont{zhsong}{Noto Serif CJK SC}
% \setCJKfamilyfont{zhhei}{Noto Sans CJK SC}
% \setlength\parindent{0pt}

\begin{document}
\CJKfamily{zhkai}

\maketitle


\Large

\url{https://www.zhihu.com/question/304924
099}  


“但真正努力之后,运气只是笑谈。”   

现在已经懒得对这些好事者说什么了,一天到晚不知道要干什么,工作就已经很辛苦了还要整一堆让自己更加辛苦的运动去做,然后又因为运动和工作带给自己的双重压力需要排解,然后去暴食或纵欲。
   

你们就不能对自己好一点,对自己身边的人好一点,对网线那头到处碰壁的陌生网友好一点吗,有事没事多跟爸妈视频一下让他们好好地看看你,或者
\newpage
是多跟一些好朋友去聊聊天吃吃东西旅旅游,再或者跟我一样打游戏看小说。你那么拼命,是有什么大仇
未报吗,还是说你生来的使命就是精忠报国。   

\url{https://twitter.com/ultramarine471/status/1421957453747531782
} 

有人问,你说的很有道理,但是像比如说锻炼就是一个很好的习惯,which我目前没有,但是确实很想去养成这个习惯,那么我应该采用何种动机去开展呢,因为身体健康在我的认识里确实是很重要
的。   

怎么说,其实我个人觉得最好的锻炼就是不锻炼,或者说你喜欢哪种运动就去做哪种运动就好了,喜欢打篮球就打去篮球,喜欢踢足球就去踢足球,有钱的话打打高尔夫也可以,有人能承受得起铁人三项,偶尔跑跑也没什么。 当然如果跟我一样不喜欢运动的话,那就每天少吃点多睡点,维持一个正常状态
\newpage

就可以。   

锻炼之所以会被认为是健康主义的重要举措,主要还是因为人们现在多多少少都有一点健康焦虑,每天伏于案牍办公室里空气又不好,吃的外卖又不知道是怎么做出来的,起早贪黑晚上精神过于兴奋睡不着,睡不着就想着看一些色情影片或者美食节目,压力再大一点就借酒浇愁,昏迷到第二天头痛欲裂地醒
来。   

问题根本不在于你有没有锻炼,问题在于你这种 daily routine本身就很有问题,或者说你工作学习生活总是处在一种诡异的气氛之下,并且目前不管是哪一样你都无法改变,能改变也会
被社会压力压到草草放弃。   

基于这种焦虑而无法改变的现状,各式各样的健身机构就开始下场鼓吹健身的好处了,因为健身其实是很迅速的,不会占用很多人所谓的宝贵时间。并且健身之后产生的一点激素也确实会让他们觉得,诶

\newpage
身体确实感觉不错,健身起到了效果等等。   

引用一句(被各种党建文章)用烂了的话就是说,“主旋律如果唱不响,杂音噪音就会有市场”,归根结底还是因为你根本没有办法在行动上去改变你的生活,甚至连改变生活的决心都没有,你都知道目前的生活很toxic,都知道不应该总是吃外卖,都知道深夜睡不着看片是没有用的,只会预支明天的
能量。   

你们连烂都不敢摆,连最温和的反抗都不去做,觉得自己每天虽然过得憋憋屈屈的,但这就一定就是蛰伏,就是潜龙勿用,等我积攒得够久一定可以积气成星。这种希望是可以理解的,只是实现的几率不
大。   

就像比如说乌龟闭气是可以活得很久的,河豚看到了也希望活得很久,它自己也开始学着憋气,憋到整个身子都鼓起来,可见其努力到了什么地步。 你们如果真的待不下去,为什么不能一走了之呢,有人说他刚刚贷了400万买了北上广深的房子,每月都有3万的房贷要还,一旦失业就意味着断供。  
\newpage

 

“你不知道断供延期一旦法拍,我的这间房子又是那种啥都不挨着的,估计连300万都没有,我家孩子还刚上小学,我不想让北京变成他触手可及却
最终幻灭的一个梦。”   

“前几天我跟最心疼的她吵架了,也不知道怎么就吵起来的,我知道她最终会原谅我,也正是因为她最终会一次又一次无条件地原谅我,我才决定要来这边的。我不知道为什么我人生中的每一步都那么的艰难,那么的错误和不由分说。 要是她没有遇见我,而是跟北京土著结婚了,现在会不会在摇椅上逍遥
地乘凉呢。”   

“没有,我以前确实很想把老是刁难、排挤我的那些人亲手做了,但后来想想他们也只是更早地更深地扭曲了本性,真正的坏人一定会首先把自己藏起来,没藏好的查出来一贪就是几十个亿。那是多少个我多少辈子才能挣来的钱,但我又偏偏杀不了他,他

\newpage
在我面前完全可能是一个和蔼慈祥的老人。”   

“强者抽刃向弱者,刺向弱者的刃必定是隐形的,就算不是隐形的也一定是经由他人之手来刺向弱者,这样就算弱者暴起而杀,也不过是换一只手套;而弱者之所以抽刃向弱者,因为他的目力所限只有同为弱者的四周,他根本看不见强者,看不见刺向他的刃从何而来,他只能生造出一个强者来排解心中的苦
恨。”   

“强者的抽刃也有分别,许多弱者挨的刃较轻,他们便自以为不痛甚至很舒服,恨不得多挨几刀才好,于是他们就簇拥到强者面前,帮助他擦拭他的刀具,顺便征求能不能代其行事;也有人是被强者察觉有剥削天分,被强者以各种手段骗上了梁山。但不管他们是被收买还是被欺骗,他们最终都成了权力的直
观形象。”   

“活得足够清醒且长久的人到最后都得像我一样从这跳下去,没有例外的,除非是在清醒过来之前就已经死了。凡是流着人血的都是诅咒的载体,我当然爱着家里的那些可爱可怜的宝贝,但我不能因为爱
\newpage
就不去斩断不死。”  

\end{document}
