\documentclass{article}
\usepackage[utf8]{inputenc}
\usepackage{ctex}

\title{最后的话\footnote{Click to View:\url{https://web.archive.org/web/20221017084212/https://www.facebook.com/chinadigitaltimeschinese/posts/1220145621346987}}}
\author{林嘉文}
\date{2016-02-23}

% \setCJKmainfont[BoldFont = Noto Sans CJK SC]{Noto Serif CJK SC}
% \setCJKsansfont{Noto Sans CJK SC}
% \setCJKfamilyfont{zhsong}{Noto Serif CJK SC}
% \setCJKfamilyfont{zhhei}{Noto Sans CJK SC}
% \setlength\parindent{0pt}

\begin{document}
\CJKfamily{zhkai}

\maketitle


\Large

终于还是要离开。一走了之的念头曾在脑海里萌发过太多次,两年多来每一次对压抑、恐惧的感受都推动着我在脑海里沉淀下今日对生死的深思熟虑,让我自己不再会觉得自己的离开只是草率的轻生,让我可以以为我最终的离去不仅是感性地对抑郁、孤独的排解,也是种变相地对我理性思考之成果的表达
。 

未来对我太没有吸引力了。仅就世俗的生活而言,我能想象到我能努力到的一切,也早早认清了我永远不能超越的界限。太没意思了。更何况我精神上生活在别处,现实里就找不到能耐的下脚的地方。活着太苍白了,活着的言行让人感到厌烦,包括我自己的言行,我不屑活着。离世前唯一的担忧是我的遗体大概会很难看且任人摆布,周围的环境决定了人很难
\newpage
有个体面的活法,连小小的中学里也处处是浓厚的政治气息(举一小小例子,西中教学楼内教师办公室靠走廊的的门窗无不是人为地被用纸贴上或用柜子挡住,或者干脆办公室靠走廊一侧就没建窗户,而学生教室却可随时被人从窗户向里一览无余,这就是种显而易见的对等级氛围和身份权利差异的暗示,套用周振鹤先生的概念,可谓之校园政治地理学。可叹很多老师从没意识到过他们这种不自重,用寡鲜廉耻评价毫不过分,因为他们一面对自己享有的这种特权安之若素,另一方面却大量抱怨着中学老师社会地位、收入、学校里面领导的官僚化作风,却不反思自己),这样的社会风气里,容不下安乐死这样很个人主义的事的,因为总有人想榨取别人,自然不能放别人自由地
生死。 

烦请所有得知我去世消息的人,如果你们觉得不能理解我,请给予我基本的尊重,不要拿我借题发挥,像对江绪林一样,那种行为挺卑劣、愚昧的。我实在不想虚伪地以令人作呕的谦虚把自己“留与后人评说”——以我自己的解释为准就好了。更何况我相信那些芸芸大众里的旁观者,只会给出那种为我所不
\newpage

屑的轻薄、庸俗的解释。 

你们知道吗,在这最后的时刻,在我给除刘雅雯外的每个人——包括我的亲人与学友——写下这些话的时候,内心竟然有种施舍般的悲悯。我想我应该坦白地告诉你们这一点,好让你们以对我的狂傲和自
以为是的嘲笑,来减少点你们心里的恐惧。 


遗嘱见下: 

1、本人去世后,我所有著述的著作权都转赠给刘雅雯。这是我对刘雅雯的心意,两年多来我一直对她有爱恋。另外,我觉得她是最能确保不让我的任
何著述在我去世后被出版、再版的人。   

2、我的藏书,凡是摆在书房书架和卧室书架上的,全都转赠刘雅雯处理。余下书籍,由我父母处理。转赠刘雅雯的藏书中有一些与西夏学、黑水城研究、民族史相关的书籍以及一些古籍,如果刘雅雯觉得用不上,烦劳刘雅雯挑出来转给王荣飞和胡耀飞处

\newpage
理。 

3、希望我的父亲能知足,珍惜我的母亲,同时改掉自己家长制的脾气以及极差的饮食追求,认清自己实际的生活能力和状况。太爱出去跟别人骑自行车,其实是不够挂念妻子和家庭。不要再保持那种单身宅男才会有的饮食习惯了,不健康,且这种饮食习惯是对性格和责任心的投射,说明人活得浑浑噩噩。
 

4、希望我的母亲能振起精神来多抓抓工作,多去挣钱。这样若我父亲先离开,至少还可以维持生活。一个志在过小日子的人,精神也会很脆弱,要学会找些东西依靠。金钱是可以依靠的,另外还有志业
也可以支撑人。 

5、剩下两次心理咨询,建议我父母分别去找郑皓鹏谈一次。我的离开不需要、不应该追责任何人,尤其是郑皓鹏,否则就是在侮辱我。我连我对刘雅雯的爱恋都没对郑皓鹏坦白过,而且我的心理问题太形而上了,郑皓鹏似乎比较适合解决诱因比较具有现

\newpage
实性的心理问题。 

6、感谢西北大学招生办刘春雷主任邀请我报考西北大学,很抱歉辜负他一片诚意。谢谢北京大学历史学系将我评为夏令营的优秀营员。谢谢邓小南老
师的关照。 

7、每次去李裕民老师家都能感受到平日很少能体会到的温馨和安稳感。我对不起李老师夫妇对我
的关爱。 

8、谢谢李范文老师一年多来对我的提携,答应给李老师整理《同音研究》的事也做不到了。恩情
难报。 

9、向我的“朋友”们致歉,抱歉我给过你们一些错觉,我曾自私地想让我尝试去适应与世界相处,努力过放下我自以为是的精神洁癖。但我天性敏感,总是善于从在貌似愉快的氛围中的发生的小小分歧里窥探出自己与别人的殊途,让你们为我这么一个于
你们活下去无意义的人耽误了些许时光。 

\newpage

10、我要承认我对历史研究的日久生情。之前在媒体上抑或私下里,总冷冰冰地说历史研究只是渐渐随年岁长进而被我习惯的工作而已。但活到最后,对之还是曾有过牵挂。人活在世上,实在不该太把自己当回事,但只要人要赖活着,总得靠某种虚荣来营造出自我存在的价值感,无用的历史研究曾让我底气十足。虽然我的两本著作烂到算作草稿都不配,但我对我的学问有信心。我对古人的历史没什么兴趣,但每当我为活着感到疲惫、无趣时,对比之下,我总会自然地想去缩进历史研究的世界。但是即便是做研究,也并非能让我拥有尽善的生活感觉,因为有太多虚假的“研究”,还因为本质上少有其他人会对研究爱得纯粹。一个人喜欢追索,哪怕是对任意领域的,都会受到现实的阻挠和精神的压迫。问太多、想太多是种折磨,因为这样的情况下人会很难活得简单肤浅起来。好像说远了,其实仅就对做历史研究的想法而言,我只是想明白了心有天游,拘泥在一门学问之中
,那样活着也是很庸碌的。 


说放下也就放下了。 

\newpage


林嘉文 

2016年2月23日 于西安

\end{document}
