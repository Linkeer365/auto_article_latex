\documentclass{article}
\usepackage[utf8]{inputenc}
\usepackage{ctex}

\title{人到中年是种怎样的体验\footnote{Click to View:\url{https://web.archive.org/web/20230418032652/https://paste.ubuntu.com/p/Z4xKtsqPj9/}}}
\author{匿名用户}
\date{2022-04-30}

% \setCJKmainfont[BoldFont = Noto Sans CJK SC]{Noto Serif CJK SC}
% \setCJKsansfont{Noto Sans CJK SC}
% \setCJKfamilyfont{zhsong}{Noto Serif CJK SC}
% \setCJKfamilyfont{zhhei}{Noto Sans CJK SC}
% \setlength\parindent{0pt}

\begin{document}
\CJKfamily{zhkai}

\maketitle


\Large

七三的,土埋半截,人生一片灰暗,了无生趣。就当这里是个树洞吧,没什么 条理,想到哪写
到哪。 

工作还算体面,收入足够温饱,在行业内稍有名气,但是我的未来丢了。唯一的闺女,半年前因为
不堪忍受抑郁症的折磨走了。 

她走了,走的很决绝,没有一点征兆,带走了
我的心,我的一切。 

孩子发现抑郁症三年,也陪了她三年,辞了老家的工作,扔掉了原有的一切,陪着她来到一个陌生的地方。新工作单位距离孩子的大学200公里,每

\newpage
周六早晨开车去她学校的城市,周日晚上回来。 

后来干脆在孩子学校附近租了套房,软磨硬泡把岳父岳母接过来照顾孩子,一切似乎都在朝好的方向发展。领她去北京名院看了五次心理科,都说没有
太大问题,按时吃药就好。 

不间断治疗三年,她的幻视幻听消失了,精神状态也好了很多,学业也慢慢走上正轨。就在所有人
都松了口气的时候,她忽然走了,没有一点犹豫。 

走之前和我联系,说是和同学去湿地公园玩,我还嘱咐她注意安全,她给我发了个笑脸,说你放心
,没事的。 

所以在接到她失踪的消息时候,我傻了,连夜开车跑到那个湿地公园,联系当地警方和蓝天救援队。一连找了两天,后来在一片水杉林里找到了她。就那么静静地躺在草地上,面容平静,穿着她最喜欢的衣服,带着白色的小兔帽子,背着刚买的白色绒毛挎
包。 

\newpage

直到看见她的时候,我都不肯相信这是事实。
 

我一直是一个把工作放在第一位的人,孩子小时候没给她太多关注,那时候自己也是事业上升期,光忙着自己的事,疏忽了她。闺女是个非常敏感的孩子,心思细腻,爱好古典音乐和诗词,会拉大提琴,
会吹埙。 

在高三的时候就体现出一些抑郁症的表现,并不明显,也没太当回事,以为就是压力太大。后来高中毕业在家的时候,发现有些异常,咨询过专家,用了点药,有些好转。等到大学开学,上了3个月的课,症状越发严重,我慌了,但是没有给她休学,决定
过去陪她。 

那时候就觉得什么事业都是假的,孩子是最重要的,所以匆匆忙忙辞职,跟她过去。陪伴她三年,我也开始抑郁了,和一个抑郁症病人在一起没办法。
我们就这么相互支持,苦苦煎熬着。 

\newpage

她会在我生日时候给我买花,会在我的书里夹个小纸条,说爸爸我爱你,会在微信下留言说我们一起勇敢面对。事到如今我还是觉得是在做梦,不敢相
信她就这么走了,扔下我在痛苦中沉沦。 

人生已经了无生趣,要不是还有白发高堂在,有同样悲痛欲绝的妻子在。我提醒自己,人不能太自私,还有家人需要我,但是一想起还有二三十年要在
痛苦中度过就不寒而栗。 

半年过去了,心底的伤口历久弥新,余生恐怕都无法摆脱。我想要给她补偿,弥补从前的过错,可
是上天不给我机会。 

前不久单位领导找我谈话,想给我一个管理岗位,毫不犹豫拒绝了。现在活下去就耗尽了我所有力
气,实在没心思干什么事业。 

孩子走了以后开始自闭,每天两点一线,不与任何人来往,在单位也可以一天不与任何人说话,常

\newpage
常对着孩子留下的手机和电脑发呆。 

闺女临走之前在QQ上留下一句话:不需怀念,活着时候能尽量表达爱意就是幸福,某些人总是要
走的,就像河流回归大海。 


看来她比我活得通透。 

怀念闺女时候,心情激荡,自己写的两首词。


江城子 

死生别路久彷徨,永怀伤,叹无常。一夜白头,谁与共凄凉。辗转枕席徒有泪,难回首,断肝肠。

身隔千里怯还乡,月如霜,满空房。忍睹旧颜,堪伴夜茫茫。残梦相逢游故地,林尽处,水杉旁。


 


江城子 辛丑冬月阴雨 

\newpage

忽来冷雨入窗棂,暮风轻,水生莹。独坐黄昏,萧瑟意难平。鬓染霜白人未老,千般苦,尽飘零。

乌云漫卷黛峰青,几时晴,月盈盈。料是无缘,唯梦会长亭。寒夜不眠埙伴弄,空呜咽,与谁听。


sepsepsep 


2022.05.01 


今天上知乎时候吓一跳,这么多的回复。 

昨天晚上胸中郁垒难舒,找个树洞倾述一下,没想能得到这么多陌生人的鼓励和安慰,非常感动。

这里澄清一个误会,那两首词不是闺女的作品,是我写的。想起孩子的离去痛不可耐,于是写词以
长歌当哭。 

其实闺女写过好多东西,但是从来不让我看,临走之前自己删了个干净,手机、电脑里一片空白。
\newpage


她用的是华为手机,上大学那年舅舅给买的,走之前特意清除了所有信息,打开锁屏,始终开机。

我懂她的想法,是想让我通过定位来找到她,因为她离开的那个位置实在是太偏僻,就是有手机定位,30多人也找了两天才找到,公园实在是太大了
。 

走之前孩子详细安排了自己的事情,经常玩的游戏账号送给了朋友,手机留给了我,网购的东西全部退掉,最后清空了微信、支付宝、银行卡里面的钱

闺女是服药离开的,不是安眠药之类,恕我不能说药物名称。是能够从网上买到的东西,短时间内足以致命,很难想象她能懂得这么生僻的知识并用在
自己身上。 

再说我夫人。我和夫人性格截然不同,我是外强中干型,表面刚强内心软弱;夫人却是外柔内刚,

\newpage
所以她比我先从痛苦中解脱出来。 

日常生活中夫人是个没主意的,什么事都要来问我的意见,我也就顺势假充大男子主义,在家做主。但是我们凡事都是商量着,结婚二十多年从没因为钱的事打架,因为两个人都不是乱花钱的人,彼此放
心。 

我在外人面前假装强势,不肯表现出软弱的样子,回家自己崩溃。又不喜欢四处诉苦,所以活得很
累。 

闺女的性格有点像我。一开始发现抑郁症就准备休学,但是她不同意,商量了几次未果,只好我去陪她。她大学上得磕磕绊绊,抑郁症很多的时候是身不由己,但是她又不肯放弃,自己不能与自己和解,
所以愈加痛苦。 

那时候知道从前忽视了孩子,在小时候没给她足够的关注,这是她得抑郁症的原因之一。我知道以前错了,想补偿,所以抛弃了老家的所有,陪在她身

\newpage
边。 

三年里我逐渐放下父亲的架子,努力接近她,了解她,陪她玩游戏,陪她看动画片,讨论她喜爱作家的作品,从《人间失格》到《1984》。渐渐地她接受了我这个朋友,向我倾诉,告诉我她有三个幻想中的伙伴,一直陪着她长大,其中一个叫白哉的是
她男朋友。 

对闺女了解越多我越愧疚,我从没想到小时候的疏于陪伴会造成这么大的伤害。我愿意为自己的错
误付出代价,只要能挽回,怎么样都可以。 

可是为什么在我幡然悔悟的时候上天却不给我
机会? 

我愿意用自己的命换回她的命,我的错误惩罚
我好了,为什么要带走我的天使? 

在闺女已经敞开心扉,一点点好转的时候却突
然带走了她,这是为什么? 

\newpage


上天不公,我心不甘。 

自己知道,痛苦的来源就在于此,我不肯放过自己,通过伤害自己来弥补愧疚。所以一遍遍撕裂自
己的伤口,躲在黑暗中品味痛苦来纪念闺女。 

就这样吧,人世间苟活着,假装是一个好儿子
好丈夫好父亲,我心已死,夫复何言。 



2022.5.4 

时隔三天再次登录,看到了更多人的善意和共
鸣,再次感谢善良的陌生人。 

我是个外强中干的性格,外人之前极其好面子,用他们的话来说,总是端着架子。平生朋友不多,性格有些孤僻,可以做个好老师,做不好哥们兄弟。

只有在这个树洞里匿名倾诉一下自己的软弱无

\newpage
能。 

原来写了一点对抑郁症的看法,想想又删掉了

我是一个失败者,在和抑郁症的对抗中没能保
住闺女,那点反思也都是反面教材,就别出丑了。 



2022.05.05 

闺女,半年前的那一个5号你离开的,走的时候天气还很冷,如今春暖花开了,你在远方还好吗?


给你捎去的衣服收到了吗? 


有人在身边照顾你吗? 


你的心情好一点没有? 


你的三个朋友和你在一起吗? 


\newpage

你们过得快乐吗? 

每个月的今天我都买你最喜欢的杨枝甘露和菠萝蜜放在你床头,我们一起分享,你一口,我一口。

还有你喜欢的向日葵和小菊花,能开许久,我
每天都换水。 

现在我也喜欢上了奶茶,只是不敢多喝,医生
说我的十二指肠溃疡越发严重了,让我控制饮食。 

你在远方好好的,自己照顾好自己,缺什么告
诉我。 

家里挺好的,我们都好,爸爸会照顾好妈妈和
奶奶。 

我们都爱你,永远爱你,不管你在哪里,变成
什么样子。 

一直想当面和你说声对不起,可是一直没有说

\newpage
,等下次见面时候我要亲口说给你听。 

你会哈哈一笑,拉着我的手说,老爸别逗了,
我早忘了。 


那时候才是我真正的解脱。 

我在期待我们重逢的日子,也许很久,也许很
快,上天弄人,谁知道呢? 


我不求来世,我怕忘了你。 


我只求今世,闺女,爸爸爱你。 



2022.05.16 

闺女喜欢日本文学,不是看了日本文学致郁,恰恰相反,抑郁了以后才看日本文学,那种极丧、自
我毁灭的文风说到抑郁症病人心里去了。 

闺女对津岛修治的评语:绝望却又怕死,颓丧
\newpage
而又无赖,丑的像鬼却又自诩为翩翩美少年,除了文
字功夫一流,做人就是垃圾。 


我们一起讨论过的书目: 

《人间失格》《我是猫》《河童》《雪国》《无影灯》《发条橙》《1984》《美丽新世界》《华氏451度》《神经漫游者》《安德的游戏》《官场现形记》《儒林外史》《镜花缘》《红楼梦》《警
世恒言》《马克吐温全集》 

闺女从小看书听音乐就与同龄人不同,上小学看《读者》《科幻世界》,听古典交响乐;上中学以后看《神曲》《三国志》《马克吐温全集》《百年孤独》,听巴赫、贝多芬、帕格尼尼,尤其对帕格尼尼
最为崇拜,称之为魔鬼的手指。 

她听的那些古典音乐我不喜欢,我只是个俗人,听谭咏麟、张国荣、孟庭苇长大的,有时候在车上她兴致勃勃要与我共享,我大多是告诉她带好耳机。不过我尽量满足她的爱好,走的时候她手里的原版光
\newpage

碟已经有一百六十多张。 

书我们倒是能谈到一起。我有个大书架,有三百多本,除了专业书籍外其他的各种各样,有《时间
简史》这样的,也有《银河英雄传说》这类的。 

闺女走了以后我把她的大提琴烧了,光碟、音响都烧了,她喜欢的东西给她带走吧。我的书架清空了,所有的书都当废纸卖了,书架送人当货架用了,以后不再有人和我讨论津岛修治就是太宰治就是个流
氓无赖的胆小鬼。 


我的世界定格在她走的那一天。 

现在的我就是飘在河里的一段朽木,随波逐流
到彻底腐烂。 



2022.06.01 


\newpage

闺女,节日快乐! 

昨天看了你小时候的照片,发现你小时候笑得
很好看,可是从什么时候开始你不会笑了呢? 


我不知道。 

有些遗憾永远不能弥补,有些愧疚永远不能释
怀,离开的人永远不能再回来。 


我无法理解你,但是尊重你的选择。 

以前也和你说过类似的话,还被你嘲笑了好久
,到现在都不能理解为什么嘲笑我。 


那是我的真心话。 

每天都在解脱的渴望与人性的道义之间摇摆,
情绪不稳,会瞬间转化。 


我在深渊中慢慢下沉 

\newpage


抬头望去 


却见你渐渐飘远 


飘向阳光灿烂的天空 


你的身影越来越模糊 


我的世界越来越昏暗 


我伸出双手 


入怀的只有冰寒彻骨 


和永恒的黑色 



2022.06.05 

闺女,又是一个5号,照例来分享你喜欢的菠

\newpage
萝蜜和杨枝甘露。 

今天和妈妈去了以前我们常去的公园,走在熟
悉的小路上,景色依旧。 

看见了你喜欢的那只黑猫,已经长大,不再有
其他的猫敢欺负它了。 

毛色发亮,胖墩墩的,今天给它猫条,只吃了
几口,看来最近吃得很好。 

那仿佛紫竹削成的耳朵很灵活,很好,很有精
神。 


你买的猫条还有不少,今天只发出去5个。 

我和妈妈谈到了你,说了好多你小时候的事。

说到你,忽然感觉胸口空荡荡的,不用低头,
我知道那里有个洞。 

我能感觉到风从洞中穿过,带起一阵微寒,就
\newpage

像一只活动在人间的基利安。 

你的房间还是原样,手机和电脑摆在书桌上,
案头放着摊开的笔记,字迹依旧清晰。 


床单,被子,枕巾,干干净净的。 

推开房门的一瞬间,仿佛你刚刚离开,房间里
还残留着你的气息。 


闺女,爸爸妈妈爱你,我们永远在一起。 



我于世界无所求 


人间于我亦如是 


缘起缘灭须臾尽 


花自飘零水自流 

\newpage



2022.06.28 



酌茶 


暮色松风啸月明, 


氤氲蟹眼破空宁。 


茶香暗转通幽路, 


瑞脑轻摇过玉庭。 


齿坠颜颓身未死, 


流连半世尽营营。 


安归碧落登金阙, 


\newpage

散作流云向岳行。 



2022.07.04 


闺女,昨晚梦见你了。 

一栋很大的房子,三个或者四个人,你穿着长
裙在拉大提琴,那些人说你演绎的真好。 


虽然记不得容貌,但是知道就是你。 

你说曲子的名字叫弦-弦-弦-弦,有四个小
节。 

旋律不记得了,只觉得让人心潮澎湃,很激动
人心的曲子。 


梦境很模糊,很扭曲,但是细节很清晰。 

你坐在那里,左手抚琴右手执弓的样子,即深

\newpage
刻又飘忽。 

虽然君子不言怪力乱神,但是这一刻我诚心祈
愿一切都是真的。 



明月如霜 


铺洒在床头 


温暖着时光永逝 


明月如歌 


萦绕在窗下 


吟唱着尘埃无声 


明月如雾 


弥漫在眼眸 

\newpage


拥抱着万物生灭 


明月如你 


高悬在心中 


映照着轮回永恒 



2022.07.05 


闺女,今天照例,杨枝甘露、水果、花。 


不过,今天的杨枝甘露不如原来买的好。 

因为换了一个商家,这次果太少,糖度太高,
给的量还少,下回还去原来那家。 


菠萝蜜没买,不是季节,没有好的。 

不过我买了芒果,切碎了放到杨枝甘露里面,
\newpage

正好弥补了杨枝甘露果肉太少的问题。 

没写给你时候,总觉得有千言万语,等落笔却
发现只有三言两语。 

昨天走楼梯下楼时候,妈妈走在后面,双手搭在我的肩上,一瞬间想起了你,以前你也喜欢这样。

以前闺女也这样/喜欢/来过/想要,每天都
要说几次,我忍不住。 

老家在你名下的房子卖了,爸爸妈妈不打算回
去了,空着也是空着。 


今年雨水少,天气很热。 


妈妈很想你。 


爸爸也想你。 


\newpage

我们都好,勿念。 


你也好好的。 

我把想说的话记录在这里,每天过来看一遍。


这样就不会忘了你。 


就这样吧,永远爱你,闺女。 



2022.07.10 


闺女,生日快乐! 

今天做了你爱吃的菜,小龙虾、三黄鸡、大虾
、还有青菜沙拉、烤牛排。 

二十几年前的今天,你来到人间,在产房门口
抱起你,那时候你是个小小的包裹。 

二十多年后的一天,在另一扇铁门前接过你,
\newpage

也是一个小小的包裹,那是你在人间仅有的残念。 


本来应该是你为我做的事,我为你做了。 


以往的点点滴滴,是美好也是刺痛。 


此生相遇,是我的荣幸,却是你的不幸。 


我们把你带到人间,却没有照顾好你。 


你已经不在,一切思考毫无意义。 


闺女,我累了,真的累了。 


我已经精疲力竭,不知道还能坚持多久。 



祭女文 

壬寅年丁未月甲子日,时值爱女之诞辰,阿父

\newpage
衔哀致诚,具清酌庶馐之奠,告女之灵: 


匆忘永伤,久思汝名 


复堪锥痛,怀之以情 


立于浊世,其光莹莹 


知天守命,哀哉运刑 


来又何欢,去时难平 


天地不仁,稚子先行 


三山渺远,青鸟伏翎 


悲号四方,余生无宁 


魂兮梦归,再见娉婷 


泪尽无言,皎月圆明 

\newpage


此恨绵绵,玄渊亦盈 


长歌当哭,祭吾英灵 


呜呼!尚飨! 



2022.08.05 

闺女,今天买了杨枝甘露,原来那家的,还是
原来的味道。 


天气太热,加了一点碎冰,低糖。 

没买菠萝蜜,买的榴莲,也是你爱吃的东西。

把一个酒瓶改成了花瓶,装满水,晶莹剔透,
好看。 

最近放弃了反思,也不想再和你共情,不再寻

\newpage
找你离开的原因。 

忽然想明白了一件事,找到又如何,已经永远
失去了你,不会再有弥补的机会。 


有些事一旦错过就不可挽回。 

你是洒脱的,也是自由的,爸爸却被自己的枷
锁困住了。 


坐井观天,望断愁云。 

每每想到你,心里就是撕裂般的痛,大半年了
,从未减轻过。 


妈妈也是。 


原以为妈妈能解脱出来,其实都是假象。 

虽然她性格坚毅,没有爸爸这么多愁善感,但
是伤痛一模一样。 

\newpage

每天爸爸妈妈在一起,互相说着宽慰的废话,
夜里一起撵转反侧,苦熬到天明。 

纵然如此,爸爸妈妈会坚强的熬下去,带着你
的一切继续品味这个世界。 


我们永远在一起,死亡也不能让我们分离。 


爱你,闺女。 



2022.09.05 


闺女,开学了。 

喝着微甜的杨枝甘露,吃着菠萝蜜,把花枝剪
短,插进花瓶里。 


时间过得真快,又是一个5号,秋天到了。 

昨天晚饭后出去遛弯,看见一个小小的身影,
\newpage

颇有你小时候的样子。 


盯着看了好久,然后被妈妈拖走。 

两个人都没说话,默默走了一段距离,妈妈忽然说,如果将来她老了,走不动了,就找点药来吃。


听了悚然一惊,因为想法是一致的。 


两个人可以一起走。 

如果上天明鉴,半生的好人没白做,给一个痛
快而且体面的结局,那最好。 


要是贼老天不开眼,就自己寻一个体面。 

绝不想受尽苦楚,蜷缩在某个养老院肮脏角落
里,凄凉地离开。 

一颗心已经碎成无数瓣,每一瓣都映射出你的

\newpage
身影。 

色即是空,空即是色,受想行识,亦复如是。


堪不破,也不想勘破。 


思念你是余生的执念。 


闺女,自己孤身在远方,保重。 



2022.10.05 


闺女,秋天来了。 

天气忽冷忽热,你那里一定是风和日丽,四季
如春。 

今天小雏菊配向日葵还有满天星,杨枝甘露低
糖不加冰,菠萝蜜破开去核。 

去年的十一在金山寺,在玄武湖,在茱萸湾。
\newpage


今年的十一在家独饮,然后坐在公园的长椅上
傻笑。 

去年的十一坐在月光下聊到深夜,说你的朋友
,你的老师,你的同学,还有人生感悟。 

那时候很震惊,忽然觉得旁边的人既陌生又熟
悉。 


今年的十一和妈妈相对无言。 

你抽走了爸爸的人生基石,那些虚幻的表象瞬
间碎了一地,像一堆玻璃渣。 

爸爸一向后知后觉,在执拗追求眼前浮华的时
候,却没注意背后的你活在阴影中。 

曾经表达过这样的歉意,你说没关系,只要不
离就不弃。 

\newpage


那时候你不过是在安慰,爸爸却当真了。 

随着时间的推移,越来越能理解你的痛苦,因
为那些黑暗正在复制。 


一切感同身受。 

许多该做的没有做,一步错步步错,再想回头
,已经不及。 

也许当时做了,并不能改变结局,至少能减轻
一点点现在的痛苦。 


整个人碎成玻璃渣,再也粘不到一起。 

看到你手机里发送的最后一条短信,说自己病了,要好好休息一段时间,把喜爱的游戏账号托付给
朋友。 


爸爸也病了,却不敢休息。 

\newpage


不能理解你的想法,但是尊重你的选择。 

虽然爸爸总是羞于公开表达自己的情感,连倾
述痛苦都要躲在角落里匿名,虚伪而且懦弱。 

我们还是要在阳光里拥抱一下,祝福我吧,闺
女! 



2022.11.05 


闺女,冬天到了,虽然气温并不低。 

杨枝甘露还是常喝的那一家,菠萝蜜和芒果,
在桌子上铺了一层桂花干花瓣。 


你过得怎么样? 


爸爸这个月过得很不好。 

妈妈高速追尾,幸好人没事;奶奶住院;爸爸
\newpage

自己生病。 

一直在失眠,最近越发烦躁,大闹公证处,宛
若泼妇。 

虽然事后羞愧得无地自容,但当时那一刻,却
感觉到肆意妄为的痛快。 


不说这些不快乐的事。 

最近做梦两次,都是我们一起去旅行,你情绪
不错,能感觉到内心淡淡的喜悦。 


想必你还是爱爸爸的,是不是? 

已经收起你的手机和电脑,不再给手机缴费,不再一遍遍看你的QQ留言,不再捧着你的笔记发呆

爸爸已经在自己的痛苦中沉得太深,几乎被黑
暗淹没。 

\newpage


你留下的各种痕迹,如刀锥,如冰火。 

每次看到都会痛苦万端,随之而来就是一股窒
息感。 

收起你的痕迹,是爸爸一种本能的自救,不过
用处不大。 

还是莫名的想哭,想嘶吼,想把自己撞得粉碎


岁月苦长 


日受万刑 


刀割锥刺 


如堕无间 


你看,你看,又说这种不愉快的事。 


\newpage

是爸爸的错。 

闺女,你已经走出黑暗,摆脱苦痛,愿你永远
沐浴在阳光里。 


闺女,爸爸爱你。 



2022.12.05 


闺女,冬天真的来了。 


还是杨枝甘露,榴莲果和向日葵。 


时间过得真快,已经一年。 


时间过得真慢,这才一年。 


你念念不忘的漫画终于制作成了动漫。 

下载到电脑上,一边喝着杨枝甘露一边看,有

\newpage
你男朋友。 


这些日子第一次打开你的电脑。 

爸爸现在是一枚石化的牡蛎,在灰黑色外壳图
上油彩,画上僵硬的笑脸。 

和妈妈说话都小心翼翼地避开你,因为只要提
起,她必沉默。 


人生的勇气不是离去,是留下。 


某些时候,艰难的不是死去,是活着。 

总想给你写点什么,但是坐在电脑前,思绪翻
涌,一片混乱。 


把你的房间收拾很干净,尽量不去那待着。 


不想、不看、不提起,这是本能的自救。 

自己欺骗自己,似乎从前的几十年都是这么过
\newpage

的。 

但这种欺骗如此脆弱,看见你电脑背包上的小
豆泥玩偶就瞬间崩碎。 

每天都在破碎与重组之间循环,世界变成一个
光怪陆离的泡沫。 

那种冲动已经被压制了大部分,为了你为了妈
妈为了奶奶。 

在理智与愿望之间反复撕扯,支离破碎,日渐
腐朽。 

每个月的5号爸爸都在这和你说会话,让这些
文字留在互联网的记忆中。 

只要有一个人能看到这些文字,你就没有真正
离开。 

而你的一切,早已经刻印在爸爸的墓碑上,虽
\newpage

不念起,永不消逝。 

那天是你离开的日子,也是爸爸离开的日子,
如今暂借这个躯壳结束还要做完的事。 


期待着与你重逢。 


闺女,宝贝,爸爸爱你。 

————————————————————
———————— 


起风了 


沉沉的绿色在风中摇曳 


仿佛熟识的影子在眼前穿行 




\newpage

淡淡的香味在风中传送 


仿佛温柔的气息在身边留空 




低低的细语在风中飘散 


仿佛遥远的呢喃在耳畔回声 




蓝蓝的天空在风中蔓延 


仿佛昨天的喜悦在梦里悸动 




阳光浓烈 


冰冷而刺目 

\newpage



2023.01.05 


闺女,新年快乐! 

这一个月,家里人都病了,好在不重,没人住
院。 

愿你永远快乐,永远不要再回到这个痛苦的世
界。 


给你买了杨枝甘露,菠萝蜜果和鲜花。 

昨天又看了微信上的对话,发现从前并没有走
进你的世界,对你的痛苦后知后觉。 

到现在为止,爸爸仍然无法充分理解你的绝望
,这让人愈加痛苦和绝望。 

知道你是爱爸爸的,你也知道我们的爱,但仍

\newpage
然不足以挽留。 

你说人世间的寄托在三个朋友身上,可是用药
之后他们消失了,锚不见了。 


那边倾诉着绝望,这里却在说着废话。 

不知道努力做的一切是在帮助你,还是害了你


时间不能倒流,一切不能重来。 


出现的所有,在意料之外,情理之中。 

一生的骄傲和自负被打成齑粉,自作自受,不
能埋怨。 

忽然想起祥林嫂,忽然能理解她丧子之后的癫
狂和愚痴。 


其实真的想做一次祥林嫂,可是爸爸不能。 

还要用虚伪的平静来维护最后一点可怜的颜面
\newpage



愚痴、虚伪而又无能的我。 


满眼黑暗,在泥沼中越陷越深。 

人生八苦,一应俱全,剩下的日子里可以细细
品味。 



不管做的好不好,那颗心,永远爱你。 

————————————————————
——————— 


2023.02.05 


闺女,正月十五,过年好。 


这次准备杨枝甘露,芒果还有冰糖雪梨。 


\newpage

冰糖雪梨是爸爸煮的,以前你能吃一大碗。 


没买花,因为附近的几家花店都人去屋空。 

最近在关注失踪100多天孩子的事,很庆幸

庆幸你的缜密,庆幸手机坚持了两天,庆幸当
地警方和救援队的全力搜索。 


所以,两天就找到你。 

这一年多,看书,找材料,剖析自己的心理和
你对照。 


逐渐理解了一些事。 


总觉得你足够坚强,不知道那是伪装。 


走之前,你很隐晦的表达过。 


可惜,爸爸没看懂。 

\newpage


错过最后一次机会。 


越了解,越痛,痛不可当。 

悔恨,是能撕裂人生的利刃,是命运肆意的嘲
弄。 

这个年没在家过,腊月三十早晨和妈妈开车离
开。 


提前规划好路线,先去苏州。 

拙政园,狮子林,博物馆,平江路,虎丘塔,
周庄。 


然后去南京。 


玄武湖,中山陵,牛首山,栖霞山。 


一路走,一路讲给你听。 

\newpage


吃饭的时候要了三副碗筷,你坐在对面。 

酒店要了家庭房,有三张床,你睡在妈妈身边


从未远离,音容还在。 


梦中相见,你走出人群,脸上淡淡的神情。 


转瞬梦醒,只留下一场虚幻。 


也许现实只是一场醒不过来的噩梦。 


闺女,爸爸想你。 

————————————————————


2023.03.05 


闺女,开学了,本来今年毕业的。 

永远看不到你戴上学士帽的样子,看不到走出
\newpage

校门的身影。 


今天的杨枝甘露爸爸自己做的,不好喝。 


芒果是朋友从海南邮过来的,没熟透。 


榴莲在超市买的,既不甜也不糯。 


所以,倒了杯酒。 


下酒菜是酸芒果,毕竟邮费比芒果贵。 

高度白酒,你喝不得,有一瓶卡布奇纳,偷偷
窥见你买过。 


买一盆水仙,上次那盆死了。 


点一支烟,看红火在阳光下闪烁。 


以前不抽,你走以后才开始。 

\newpage


医生警告烟酒勿动,按时吃药,没听。 


爸爸现在很平静,平静得死寂。 


烈焰燃尽必是冷灰,爆发过后只有余烬。 


日子一天天过,麻木、机械。 


还能怎样。 

每个月的五号找个清静地方呆一会,然后收拾
好,装作什么都没发生。 


不知还要熬多久。 


来,干一杯。 


随便说一句,不忙的时候托个梦。 



\newpage

sepsepsep2023.03.29 

闺女,爸爸错了,不应该再翻看你微信朋友圈

看到那年你在父亲节,发在朋友圈的“感谢你
给予我生命,我爱你,爸爸”。 


悲哀汇聚成海,再次席卷而来。 

原以为泪已流尽,心已麻木,却被几个字戳穿


无话可说,无以表达。 


sepsepsep2023.04.05 


闺女,真巧,今天清明节。 


很应景,阴天,下小雨,不耽误去踏青。 


买了小雏菊和向日葵,杨枝甘露,菠萝蜜。 


\newpage

今天不喝酒,喝药。 


这一个月在和胃病战斗,不分胜负。 

记忆力越来越差,丢了钥匙、运动鞋和雨伞。

还丢了一颗牙,没办法,影响吃饭,只能舍弃
几十年的老伙伴。 

有意找点事做,比如去公园喂猫,洗刷家里的
茶杯,买木手工。 

看到失能老人被护工虐待的报道,越发坚定从
前想法。 

爸爸没能守护好你,也失去自己未来的守护。


忽然想立遗嘱。 

前天梦见你,又是兴奋又是悲伤,醒来胸口空
荡荡。 

\newpage


习惯了,每次都这样。 


没有未来,没有希望,全是痛苦和恐惧。 

心经上说,无挂碍故无有恐怖,远离颠倒梦想
,究竟涅槃。 

爸爸是个俗人,没有大境界大解脱,只能淹没
在火狱中。 



爸爸想你。 

爸爸失去了你。

\end{document}
