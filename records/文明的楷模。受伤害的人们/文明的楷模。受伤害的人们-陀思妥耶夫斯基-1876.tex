\documentclass{article}
\usepackage[utf8]{inputenc}
\usepackage{ctex}

\title{文明的楷模。受伤害的人们\footnote{Click to View:\url{https://web.archive.org/web/20220630085526/http://libgen.is/book/index.php?md5=0A2665DC7B9E1E006DCDA377D37A9CA7}}}
\author{陀思妥耶夫斯基}
\date{1876-04}

% \setCJKmainfont[BoldFont = Noto Sans CJK SC]{Noto Serif CJK SC}
% \setCJKsansfont{Noto Sans CJK SC}
% \setCJKfamilyfont{zhsong}{Noto Serif CJK SC}
% \setCJKfamilyfont{zhhei}{Noto Sans CJK SC}
% \setlength\parindent{0pt}

\begin{document}
\CJKfamily{zhkai}

\maketitle


\Large

阿夫谢延科先生很早就在写评论,已经有几年了,我的过错是始终对他抱有某种期望:“会写出来的,我在想,迟早要说出有分量的话来。”其实,我对他了解得太少了。我的迷误一直持续到1874年10月号的《俄国导报》杂志出刊的时候,在这一期的杂志上,阿夫谢延科先生关于皮谢姆斯基的戏剧突然说了下面这样一段话:“……果戈理强使我们的作家过分轻率地对待作品里面的内容,又过多地仅仅仰仗一个艺术性。在我们19世纪40年代的文学中有很多的人对小说的任务持有这种见解,这种文学的中心内容之所以贫乏,部分原因也就在这里(!)”
。 

这是说19世纪40年代的文学的中心内容贫乏!我这一生中还是头一次听到这种奇谈怪论。这里
\newpage
说的是这样的一种文学,它给了我们果戈理全集和他的喜剧《婚事》(唉!这是内容贫乏的喜剧),后来又给了我们他的《死魂灵》(内容贫乏的作品,这是他要说的第一句话,这个人要是说点别的什么话,那就一切都好了)。继之又献出屠格涅夫及其《猎人笔记》(这也是中心内容贫乏之作吗?),还有冈察洛夫这位早在19世纪40年代就写出《奥勃洛莫夫》的作家,当时发表的此书的优秀章节《奥勃洛莫夫之梦》博得了全俄罗斯的赞叹!这就是后来又给了我们奥斯特洛夫斯基的文学,然而阿夫谢延科在自己的文章中正是极其轻蔑地对奥斯特洛夫斯基笔下的典型大
放厥词: 

“由于外在的原因,看来官吏们的社会不完全适宜于戏剧讽刺:于是我们的喜剧就更加执著地、全神贯注地把目光投向莫斯科河南岸和阿波拉克欣区商人阶层的社会,投向朝圣者、媒婆、酗酒小官吏、地主庄园管事、下层教士以及在彼得堡混生活者等入的社会。喜剧的任务就莫名其妙地缩小为摹似醉汉或者文盲的行话,再现野蛮的斗殴和粗暴的、有损人的情感的典型和性格。占据了舞台的是清一色的粗鲁、龌
\newpage
龊、令人厌恶的戏剧,而不是法国舞台上那种有时令人陶醉的、热情、欢快的资产者(?)的戏剧。(这就是所谓的轻松喜剧:一个人钻到桌子底下,另一个人则抓住他的脚把他拖出来?)有一些像奥斯特洛夫斯基先生这样的作家们向这种文学奉献了很多才能、心血和幽默,然而总的说来,我们的戏剧的内在水平却急剧下降了,很快就会出现这样的局面,我们的戏剧对社会中有教养的那部分人无话可说,而戏剧与社
会中这一部分人就毫不相干了。” 

这就是说,奥斯特洛夫斯基降低了戏剧的水平,奥斯特洛夫斯基对社会中那部分“有教养的人”什么话也没有说!因此,原来是没有教养的社会在戏院里赞赏奥斯特洛夫斯基的戏,并且对他的作品读得入迷?请看,从前有教养的社会是到米哈依洛夫斯基剧院去看戏,那里有“法国舞台上那种有时令人陶醉的、热情的、欢快的资产者的戏剧”。而柳比姆•托尔佐夫则是“粗鲁、龌龊”的。令人感兴趣而想要知道的是,阿夫谢延科先生所说的有教养的社会指的是什么人?污浊不在柳比姆•托尔佐夫的身上,“他的心灵是纯洁的”,污浊可能正是在“法国舞台上那种有
\newpage
时令人陶醉的、热情的资产者的戏剧”占统治的地方。艺术性排挤掉中心内容,这是什么意思?恰恰相反,正是艺术性使内容臻于高度完美:果戈理在《与友人书简》中虽然有其特色,但却软弱无力。在《死魂灵》里面,当果戈理不是作为艺术家,而是以自己的身份直接发表议论的那些地方,他写得真是苍白无力,甚至是毫无特色,而他的《婚事》、《死魂灵》和其他作品——都是最深刻、内容最丰富的作品,这一切恰恰是通过作品的艺术形象达到的。可以说,这些描写以极其深刻、难以承受的问题使心灵感到沉重难受,唤起俄罗斯的才智之士极其焦虑不安的思想,这些思想显然远非目前所能解决的;不过,在将来某个时候就能够解决吗?阿夫谢延科先生却大声疾呼说,《死魂灵》没有中心的内容!再看看《智慧的痛苦》吧,它之所以成为一部力作,完全是靠了自己的鲜明的艺术典型与性格,只有艺术劳动才赋予了这部作品全部的中心内容;只要格里鲍耶陀夫稍一离开艺术家的角色,开始以自己的身份发表自己的见解(通过剧中最苍白的人物恰茨基之口),他立刻就降落到十分平庸的水平,远远低于我们的知识分子的当时代表人物的水平。恰茨基的道德说教的水平比喜剧本身要低
\newpage
得多,有的说教纯系废话。艺术作品的全部深度及其全部内容,看来完全在于它的人物典型与性格。可以
说,几乎永远是这样的。 

这样一来,读者就会明白,他是在同什么样的批评家打交道,我已经到处都听到这样的质问:那么您为什么还在跟他纠缠?我再重说一次,我仅仅是想要说明自己的过失,在这个时刻我之所以注意阿夫谢延科先生,正如上面已经说过的,不是把他看做批评家,而是把他看做一种个别的、有意思的文学现象,这是对我有益的一种人。我很长时间都不理解阿夫谢延科先生,我指的是他这个人,而不是他的文章,他的文章我从来也没有理解过,其实对他的文章没有什么理解或不理解的问题,从1874年10月号的《俄国导报》上的那篇文章起,我就干脆不想去理解它了,虽然我一直深感莫名其妙:这样一位自相矛盾的作家的文章怎么竟能够出现在《俄国导报》这种严肃的杂志上呢?然而,突然发生了一件可笑的事情使我突然理解了阿夫谢延科先生:在初冬时节他忽然开始发表自己的小说《银河》。(可是不知为什么这篇小说又停止刊登了!)这篇小说使我看清了阿夫谢延科
\newpage
这位作家的整个面貌。其实,由我来谈论这篇小说是不合适的:因为我自己是小说家,我不便评论同行。因此,我不打算对小说做任何评论,何况小说还给了我几分钟真正的愉快。例如,年轻主人公公爵在包厢里观看歌剧时,被音乐感动得在大庭广众面前啜泣不止,而上层社会一位贵妇人则深表同情地喋嗓不休:“您在哭泣吗?您在哭泣吗?”不过,问题并不完全在这里,而在于我从小说中认清了这位作家的本质。阿夫谢延科先生作为作家,他表明自己是一位对上层社会崇拜到失去理智的活动家。简单地说就是他崇拜得五体投地,崇拜上层社会的手套、马车、香水、香脂、丝綢衣裙(特别是当贵妇人往椅子上落座时,衣裙在腿上和腰间发出的窸窣声),最后还崇拜那些迎接看完意大利歌剧回家的太太的奴仆。他无休无止,毕恭毕敬,像作祷告似的写一切,总之,似乎在举行什么宗教仪式。我听说(我不知道,这可能是在嘲笑他),这部小说之所以被发表出来,是为了纠正托尔斯泰在自己的小说《安娜•卡列尼娜》中对上层社会那种过分客观的态度,本来对上层社会是应该奉若神明,顶礼膜拜的。我还要再说一遍,如果说不是清晰地呈现出了一种全新的文化类型,当然,那也就根本
\newpage
用不着谈论这一切。原来,在评论家阿夫谢延科看来,马车、口红之类的东西,特别是奴仆如何迎接太太这类事情,这就是文明的全部使命,文明要达到的全部目的,这就是我们二百年间堕落和苦难的历程的全部结果,阿夫谢延科对此毫无嘲讽之意,而是加以欣赏。这种严肃、真挚的赞赏是十分有趣的现象之一。主要的问题是,阿夫谢延科先生作为作家,并非绝无仅有;在他之前就有“穿着细白布胸衣的毫不留情的尤维纳利斯们”,但他们从未达到如此顶礼膜拜的程度。即使说,他们并非全是那种人,然而我的不幸也正在这里,我终于渐渐地明白了,在文学中和在生活中这样的文明代表人物实在太多了,虽然他们不是严格的和纯粹的典型人物。我承认,我似乎豁然开朗了:从此之后,诋毁奥斯特洛夫斯基的言论和“法国舞台上那种有时令人陶醉的、热情的、欢快的资产者的戏剧”当然就都可以理解了。哎,这里根本不是奥斯特洛夫斯基,不是果戈理,也不是19世纪40年代(太需要他们了!),这里需要的就是一个彼得堡米哈依洛夫斯基剧院,上层社会乘坐轿式马车光顾的戏院,这就是全部原因,就是这一切以不可抗拒的力量吸引了、俘获了作家,使他眼花缭乱,神魂颠倒。我
\newpage
再说一遍,不应该只是从可笑的角度看待这件事,一切都不可等闲视之。在这里,总的说来,很多事情都出自一种特殊的癖好,可以说,几乎是一种病态的、但却可以原谅的弱点。比如说,上层社会的马车向剧院驶去:您只要看一看,马车是在怎样行走,穿过车窗射进路边的灯光如何使坐在车里的贵妇人兴高采烈:这已经不是在用笔写,这是在祈祷,对此应该给予同情他们中间有很多人在人民的面前似乎以手套的贵重为荣,在他们里面甚至有大量的人是自由主义者,几乎是共和派分子,然而有时也突然显得像个手套商人。这种弱点,对上层社会及其牡蛎和舞会上价值一百卢布的西瓜等种种雍容华贵的这种痴心向往,这种心神向往,无论它怎样天真无邪,它都要在我们这里,比如说,在那些从未拥有过自己的农奴的人们之间造就独特的农奴主;他们既然认为马车和米哈依洛夫斯基剧院是俄国历史上的文明时期的成就,他们就立刻成为不折不扣的信念上的农奴主,纵然他们根本没有打算重新奴役农民,但他们起码是毫无掩饰地蔑视人民,并且摆出一副握有最充分的文明权利的架势。于是他们对人民投过来种种耸人听闻的责难:嘲弄在二百年间被束缚着手脚的人民过于消极,指责遭受租
\newpage
役掠夺的穷苦人十分肮脏,责备没有受过任何教育的人不懂科学,申斥在棍棒下成长起来的人性情粗暴,有时竟然责怪他不到大海洋街理发店去涂脂抹粉,梳妆打扮。这样说,一点儿也没有夸张,确实是这样,全部问题也正在于这不是夸张。他们对人民的厌恶是根深蒂固的,如果说他们有时也赞美人民,那是出于政治需要,他们只不过是编织一些漂亮的词句,仅仅是为了装装门面,他们自己并不明白那些话的意义,因而在写了几行之后自己竟跟自己前后矛盾起来。
说到这里,我顺便想起了两年半前我遇到的一件事情。我在乘火车去莫斯科时,夜间同坐在我旁边的一位地主闲聊起来。在黑暗中,我影影绰绰看出来,这个人身材消痩,五十岁上下,鼻子红肿,双腿似乎有病。从气派、言谈和议论看,他是一位举止十分端庄的人,言谈有条不紊。他谈到贵族的既严峻又不明朗的处境,谈到全俄罗斯经济的惊人混乱,他语气平和,但对事情的观点却是严峻的,他引起了我极大的兴趣。没有料到:突然间,他似乎毫不在意脱口而出地说,他认为自己在体力上也远远胜过庄稼汉,当
然他以为这是无可争辩的。 

\newpage

“这就是说,您说的是,您属于在道德方面高
尚的有教养的人?”我试着解释说。 

“不,完全不是,绝不仅仅是一个道德方面,我的天生体质就比庄稼汉强;我的肉体就比庄稼汉又壮又健,这是由于历年来我们祖祖辈辈都要把自己培
养成为上等人。” 

无须争辩:这个瘦弱的、长着一个患瘰疠病似的红鼻子、拖着两条病腿(可能是患了贵族病——痛风)的人,实实在在地认为自己的体质、肉体比庄稼汉更壮,更好!我再说一遍,他说的话里没有一丁点儿愤恨情绪,但是您看得出来,这个没有恶意的人,即使在心平气和的时候,在某些情况下,也会突然间讲出关于人民的极不公正的话来,是毫无恶意地、平心静气地、严肃认真地讲出来的,这纯粹是出于蔑视人民的观点,这几乎是意识不到的观点,几乎是不取
决于他的观点。 

尽管如此,我也必须更正我自己的过失。我曾经写过关于人民的理想的话,写过我们应该像回头的
\newpage
浪子,在人民的真理面前低下头来,我们应该接受的只是人民真理的思想和形象。但是,另一方面,人民也要采纳我们带来的某些东西,这里所说的某些东西确实是存在的,它不是虚无缥渺的东西,它有形象,有形式,有重量,反之,如果我们不同意这样做,那就只好各走各的路,我们都将各自毁灭。现在我看到的是,似乎所有的人对这些话都觉得不明确。首先,人们要问:应该向其低头的那些人民的理想是什么;其次,我们所说的那种我们自己带来的、人民应该Sine qua non(必须)从我们这里接受的珍贵的东西指的是什么?最后,只凭我们是欧洲,是有教养的人们,而人民只不过是俄罗斯,是消极的,只凭这一点就说,不是我们,而是人民应该向我们低下头来,这样岂不是更简单些吗?阿夫谢延科先生当然是从这个意义上解决问题的,然而我现在不想仅仅回答阿夫谢延科一个人,而是想回答所有不理解我的“有教养的”人们,从“穿细白布胸衣的尤维纳利斯们”到不久之前声明说我们这里“根本没有什么值得保留的”那些先生们。现在,我们回到正题上来;我当时如果不追求简短,而是解说得更详细一些,虽然可以不赞成我的观点,但却不能曲解我,不能指责我
\newpage
述不清楚了。

\end{document}
