\documentclass{article}
\usepackage[utf8]{inputenc}
\usepackage{ctex}

\title{永远的蝴蝶\footnote{Click to View:\url{https://rentry.co/uwte9}}}
\author{陈启佑}
\date{}

% \setCJKmainfont[BoldFont = Noto Sans CJK SC]{Noto Serif CJK SC}
% \setCJKsansfont{Noto Sans CJK SC}
% \setCJKfamilyfont{zhsong}{Noto Serif CJK SC}
% \setCJKfamilyfont{zhhei}{Noto Sans CJK SC}
% \setlength\parindent{0pt}

\begin{document}
\CJKfamily{zhkai}

\maketitle


\Large

那时候刚好下着雨,柏油路面湿冷冷的,还闪烁着青、黄、红颜色的灯火。我们就在骑楼下躲雨,看绿色的邮筒孤独地站在街的对面。我白色风衣的大口袋里有一封要寄给南部的母亲的信。樱子说她可
以撑伞过去帮我寄信。我默默点头。 

“谁叫我们只带来一把小伞哪。”她微笑着说,一面撑起伞,准备过马路帮我寄信。从她伞骨渗下
来的小雨点,溅在我的眼镜玻璃上。 

随着一阵拔尖的刹车声,樱子的一生轻轻地飞了起来。缓缓地,飘落在湿冷的街面上,好像一只夜
晚的蝴蝶。 


\newpage

虽然是春天,好像已是秋深了。 

她只是过马路去帮我寄信。这简单的行动,却要叫我终身难忘了。我缓缓睁开眼,茫然站在骑楼下,眼里裹着滚烫的泪水。世上所有的车子都停了下来,人潮涌向马路中央。没有人知道那躺在街面的,就是我的,蝴蝶。这时她只离我五公尺,竟是那么遥远。更大的雨点溅在我的眼镜上,溅到我的生命里来。
 


为什么呢?只带一把雨伞? 

然而我又看到樱子穿着白色的风衣,撑着伞,静静地过马路了。她是要帮我寄信的。那,那是一封写给南部母亲的信。我茫然站在骑楼下,我又看到永远的樱子走到街心。其实雨下得并不大,却是一生一世中最大的一场雨。而那封信是这样写的,年轻的樱
子知不知道呢? 

妈:我打算在下个月和樱子结婚。

\end{document}
