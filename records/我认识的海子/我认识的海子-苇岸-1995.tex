\documentclass{article}
\usepackage[utf8]{inputenc}
\usepackage{ctex}

\title{我认识的海子\footnote{Click to View:\url{https://web.archive.org/web/20221120133946/http://m.zuojiawang.com/html/sanwen/45721.html}}}
\author{苇岸}
\date{1995-05-05}

% \setCJKmainfont[BoldFont = Noto Sans CJK SC]{Noto Serif CJK SC}
% \setCJKsansfont{Noto Sans CJK SC}
% \setCJKfamilyfont{zhsong}{Noto Serif CJK SC}
% \setCJKfamilyfont{zhhei}{Noto Sans CJK SC}
% \setlength\parindent{0pt}

\begin{document}
\CJKfamily{zhkai}

\maketitle


\Large

我和海子第一次见面,是在一个冬天,时间约在一九八五年底或一九八六年初。那天晚上,他是随一个写小说的朋友,一起到我家来的。当时他所执教的中国政法大学,正准备由市区迁往昌平,部分教师的宿舍先行搬到这里,临时住在城西北角西环里小区中央政法管理干部学院租用的楼里。我记得当朋友向我介绍说:这是海子,政法大学的教师,写诗的。我感到很惊异,因为看上去他还完全像个孩子。他身体瘦小,着装随便,戴一副旧色眼镜,童子般的圆脸,满目稚气。虽然他此时已二十出头,但在他身上,依然是一种少年的和早慧的气息。
 
海子一九六四年生,一九七九年十五岁时即从安徽怀宁家乡考入北京大学法律系,他毕业时的年龄恰是我们一般入学时的年龄。毕业后,他被分配至中
\newpage
国政法大学,初编校刊,后走上了讲坛。我尚未读过他的诗,也未听说过海子这个名字,但他的神童历程,已令我肃然起敬。一生远离巴黎,居住在比利牛斯山区的故乡小镇,写出“把我们得不到的幸福给予所有的人吧!”(《祈祷》) 的法国诗人雅姆(一个我非常喜爱和崇敬的诗人),被里尔克敬重地称为“外省的诗人”。此时我将我眼里的海子,看作“一个外省的少年形象的诗人”。他实际已在自己的诗中,写下了“第一个牺牲的/应该是我自己”(《一九八五年诗抄之二:种子》)这同样震撼心灵的诗句。
这之后,我们好像见面并不多。真正密切交往,是后来的事了。不久我从家里搬出,恰好也住到了西环里。我住的六号楼距他们的十五号楼很近,一楼之隔,几分钟的路。由于我们都是单身居住,因此来往没有任何顾忌。谁想到谁那去,完全不必考虑此时是什么时间,直到一九八八年他们搬进位于城东的政法大学新校。我们在西环里做了近两年的邻居。
海子在我所结识的朋友中,是我感到交往上最无障碍、最自然、轻松、愉快的一个人。他胸无城府,世事观念很淡。平日的海子,既有着农家子弟温和与纯朴的本色,又表露着因心远而对世事的不谙与笨
\newpage
拙。
海子比我小几岁。但无论是在文化视野,还是在诗学修养上,他都是一个先行者和远行者。他对诗歌更为专注和深入,他是一个洋溢着献身精神的纯粹的诗人。
海子广读博览,涉猎宽泛。他看书的速度很快,每次我到他那去,发现他正在读的必定是一本新的书。有时他从我这拿走一本书,第二天便会将读完的书送还。我有一些书是经他谈论、推荐,才买来或首读的。这里我首先要说的,是美国十九世纪作家梭罗的那部光辉著作:《瓦尔登湖》。由于海子的传播,我读到了这本有生以来对我影响最大的书(海子讲,他一九八六年读到的最好的书是梭罗的《瓦尔登湖》,一九八七年读的最好的书是海雅达尔的《孤筏重洋》)。《孤筏重洋》是一本小书,译本为一九八一年版,定价很低,海子碰上时大概买了好几本,分送给朋友。海子卧轨时,身边带了四本书,其中即有我们上述谈到的两书(另两本为《新旧约全书》和《康拉德小说选》)。一九八八年春,海子去了一趟四川,回来后,有这样三个细节使我至今记忆犹新。一是他说的一句近乎戏谑的话:四川常年阴天,所以当地人
\newpage
看起来就像每天都在搞阴谋似的;二是他送给我一张他在沐川与诗人宋渠、宋炜兄弟合影的照片;三是他向我推荐他在当地书店买的一本有着“文明人从未能在一个地区内持续文明进步长达三十至六十代人以上”论断的书:《表土与人类文明》,海子曾到我这里找过关于大地的书,他说至今尚未看到一本这样的书,梭罗的《瓦尔登湖》沾点边。那次他并未如愿,只拿走了汉姆生的小说《大地的成长》和一本《爱鸟知识手册》。
我们常一起进城。主要是去书店、看展览或见见朋友。我现在能够记起的有这么几次。一次在新街口书店,我们每人买了一本奥维德的《变形记》。这是一本深受历代作家喜爱的书,《神曲》中,它的作者被但丁列为荷马、贺拉斯之后的人类第三大诗人。我曾有在买来的书上即兴写下一两句话的习惯,类似“有助于文明社会丧失了的想象力复苏” (《希腊的神话和传说》)等。在这本书上我写了“热爱人类的童年”,时间是一九八六年十一月二十九日。另一次是海子随我去顾城处,那天我们被主人诚恳留住了。都谈了什么,我完全想不起来了。印象仅存的一个细节是,晚上我们一起看一个有关西藏的电视片,当
\newpage
时美术馆刚刚举办了一个“西藏民间艺术展”。我问顾城去看没有,顾城说了这么几句:听江河讲不太好,就没有去。后来我忽然醒悟了,江河是只看书、看画片、听音乐,而不看实物和自然的,我被他骗了。还有一次,我们一起到美术馆看一个国外画家的画展。是一人的,还是一个画派或国家的画展,我搞不清了。只记得这个展览出售许多印象派以来的绘画大师的画册,印制精美,都是原版进口的,很贵,但机会难得,我们每人买了一册。海子选的是塞尚,我选了马蒂斯。塞尚,一个崇尚体积和结构,注重造型的革命性画家,被世人公称为“现代绘画之父”。除此,我仍想更深地理解海子这一选择。写作本文的时候,恰好我的一位熟知海子作品的朋友,上海的青年画家丁乙,自沪来京观看意大利当代画家米莫·巴拉第诺画展。
我请他谈了他的看法。他认为,海子的作品虽然有着理性的框架,但本质上仍是抒情的,直觉上他需要补纳“理性”,故他选择了塞尚。
布莱认为,年轻一代的美国诗人在成长中,正在被学校生活的稳定、富裕所软化。他主张诗人应自觉接近自然和底层普通大众,过艰苦的目子。和梭罗
\newpage
一样,他身体力行。布莱毕业于哈佛大学,在纽约生活几年后,便迁到了明尼苏达州马迪森市附近的一个农场。在美国,我还知道诗人弗洛斯特和散文大师怀特等,亦在农场(美国的乡村)定居。在僻远的地方生活久了的诗人,唯一感到不利的是什么呢?布莱说:“最近我认识到住在一个不需要你,不敬重艺术的城镇,就一定会产生自我怀疑。是的,叶芝有时和自己争辩:不知多少次好奇地想到自己,原可以在一些人人能理解和分享的事物中证实自己的价值。”住到昌平这座距市区三十公里,毫无文化和精神可言的北方小城的海子,是否具有与布莱相同的感受呢?海子曾有一首关于“孤独”的诗,发表时,我注意到他换了这样一个标题:《在昌平的孤独》。
在骆一禾致友人的书信和西川的纪念文章里,都对海子的居室有所提及。一禾写道:“海子是个生命力很强,热爱生命的人。”“他的屋子里非常干净,一向如此。”的确这样。在他的楼道门上,贴着一幅优美的摄影作品,内容为风景中的欧洲城堡。
从楼下上来,你会觉得这幅画在向你说:这是一个诗人的居所。
一张床,几个书架,一张书桌,大体构成了我
\newpage
们这位热爱生活的诗人居所全部内容。墙上饰有一块醒目的富于民间色彩的大花布,一张梳着无数条小辫子的西藏女童照片,凡·高的《向日葵》,还有一幅海子很喜爱的俄国画家弗鲁贝尔的作品:画面是一个坐着的男孩或小伙子,英俊、漂亮,神情略显忧郁,面题为《坐着的魔鬼》。一禾说,海子的屋子里有一股非常浓郁的印度香的气味,并曾警告他“不要多点这种迷香”。这与他的写作有关。海子喜欢夜里写作,每晚他还要喝咖啡。
到了一九八八年上半年,他们搬进了新校。这使我们的来往骤然减少。新校在城东,由西环里骑车,至少需二十分钟。
海子不会骑车。我到他那去,又时常扑空。此后直到出事,近一年时间,我们见面的次数很有限。
关于海子的死因,当时有各种说法。虚妄的,铁心的,听到别人的灾难便兴奋的。如:海子练气功走火入魔,想试试火车的力量;海子写完《太阳》之后,感到难以为继了;海子想以死来提高他的诗等。一禾的“有过‘"天才生活’的人,大都死于脑子”的说法,是善意的,研究的,负责任的。他的角度,
\newpage
是依据海子留在校内的遗书中说他出现了思维混乱、头痛、幻听、耳鸣等症兆,伴有间或的吐血和肺烂了的幻觉等来确定的。
他认为,“这是脑力使用过度以后脑损伤的症候”。西川认为,加缪讲,任何诗人的自杀都是有其直接原因的,一禾的说法,提供了一个“背景”,它还是另有导火索的。我觉得这样的判断是全面的、客观的,接近真实的。
上海诗人陈东东,在他的悼文《丧失了歌唱和倾听》中,生动地把海子看作嗓子,把一禾看成一个倾听者,一只为诗歌(或海子的嗓子)存在的耳朵。这是个极为恰当和出色的比方。当我们读了一禾关于海子的文章和书信,我们会说,没有什么人比一禾更知海子及他的诗,一禾认为:“海子是我们祖国给世界文学贡献的有世界眼光的诗人,他的诗歌质量之高,是不下于许多世界性诗人的,他的价值会随着时间而得到证明。”我赞同他的结论和信心。海子,一个祖国难得的“点石成金”的诗人,在他的短暂的写作生涯(成熟期:一九八四年至一九八九年)中,为祖国和人类留下了五百首抒情诗,七部长诗(诗剧),计二百多万字的诗文作品。这里,我想再提两位早逝
\newpage
的俄罗斯诗人。一位是谢苗·雅可夫列维奇·纳德松,一个终年与海子相同的诗人,他死后,俄罗斯为他制作了金属棺材,举行了隆重葬礼,随后出版了他的诗歌全集。另一位是我们熟知的谢尔盖·叶赛宁,在他的故乡梁赞,每逢九月二十一日(叶赛宁的诞辰日),都要举行俄罗斯文学日;他出生的康斯坦丁诺沃村,也以叶赛宁的名字命名,并与周围地区一起,被宣布为国家保护区。海子离世已经十年了,而他身后的一切,还仅限于朋友们私下的种种努力。
我幻想,期盼,并满怀信心地相信:终有一天,海子会从他的祖国那里,得到像俄罗斯给予她的叶宁那样的荣誉。

\end{document}
