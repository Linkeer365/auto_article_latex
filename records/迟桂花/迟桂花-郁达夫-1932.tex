\documentclass{article}
\usepackage[utf8]{inputenc}
\usepackage{ctex}

\title{迟桂花\footnote{Click to View:\url{https://web.archive.org/web/20221109010910/https://rentry.co/3xt9u}}}
\author{郁达夫}
\date{1932-10}

% \setCJKmainfont[BoldFont = Noto Sans CJK SC]{Noto Serif CJK SC}
% \setCJKsansfont{Noto Sans CJK SC}
% \setCJKfamilyfont{zhsong}{Noto Serif CJK SC}
% \setCJKfamilyfont{zhhei}{Noto Sans CJK SC}
% \setlength\parindent{0pt}

\begin{document}
\CJKfamily{zhkai}

\maketitle


\Large


××兄: 

突然间接着我这一封信,你或者会惊异起来,或者你简直会想不出这发信的翁某是什么人。但仔细一想,你也不在做官,而你的境遇,也未见得比我的好几多倍,所以将我忘了的这一回事,或者是还不至于的。因为这除非是要贵人或境遇很好的人才做得出来的事情。前两礼拜为了采办结婚的衣服家具之类,才下山去。有好久不上城里去了,偶尔去城里一看,真是象丁令威的化鹤归来,触眼新奇,宛如隔世重生的人。在一家书铺门口走过,一抬头就看见了几册关于你的传记评论之类的书。再踏进去一问,才知道你的著作竟积成了八九册之多了。将所有的你的和关于你的书全买将回来一读,仿佛是又接见了十余年不见的你那副音容笑语的样子。我忍不住了,一遍两遍的
\newpage
尽在翻读,愈读愈想和你通一次信,见一次面。但因这许多年数的不看报,不识世务,不亲笔砚的缘故,终于下了好几次决心,而仍不敢把这心愿来实现。现在好了,关于我的一切结婚的事情的准备,也已经料理到了十之七八,而我那年老的娘,又在打算着于明天一侵早就进城去,早就上床去躺下了。我那可怜的寡妹,也因为白天操劳过了度,这时候似乎也已经坠入了梦乡,所以我可以静静儿的来练这久未写作的笔
,实现我这已经怀念了有半个多月的心愿了。 

提笔写将下来,到了这里,我真不知将如何的从头写起。和你相别以后,不通闻问的年数,隔得这么的多,读了你的著作以后,心里头触起的感觉情绪,又这么的复杂;现在当这一刻的中间,汹涌盘旋在我脑里想和你谈谈的话,的确,不止象一部二十四史那么的繁而且乱,简直是同将要爆发的火山内层那么
的热而且烈,急遽寻不出一个头来。 

我们自从房州海岸别来,到现在总也约莫有十多年光景了罢!我还记得那一天晴冬的早晨,你一个人立在寒风里送我上车回东京去的情形。你那篇《南
\newpage
迁》的主人公,写的是不是我?我自从那一年后,竟为这胸腔的恶病所压倒,与你再见一次面和通一封信的机会也没有,就此回国了。学校当然是中途退了学,连生存的希望都没有了的时候,哪里还顾得到将来的立身出世?哪里还顾得到身外的学艺修能?到这时候为止的我的少年豪气,我的绝大雄心,是你所晓得的。同级同乡的同学,只有你和我往来得最亲密。在同一公寓里同住得最长久的,也只有你一个人;时常劝我少用些功,多保养身体,预备将来为国家为人类致大用的,也就是你。每于风和日朗的晴天,拉我上多摩川上井之头公园及武藏野等近郊去散步闲游的,除你以外,更没有别的人了。那几年高等学校时代的愉快的生活,我现在只教一闭上眼,还历历透视得出来。看了你的许多初期的作品,这记忆更加新鲜了。我的所以愈读你的作品,愈想和你通一次信者,原因也就在这些过去的往事的追怀。这些都是你和我两人所共有的过去,我写也没有写得你那么好,就是不写你总也还记得的,所以我不想再说。我打算详详细细向你来作一个报告的,就是从那年冬天回故乡以后的
十几年光景的山居养病的生活情形。 

\newpage

那一年冬天咯了血,和你一道上房州去避寒,在不意之中,又遇见了那个肺病少女——是真砂子罢?连她的名字我都忘了——无端惹起了那一场害人害己的恋爱事件。你送我回东京之后,住了一个多礼拜,我就回国来了。我们的老家在离城市有二十来里地的翁家山上,你是晓得的。回家住下,我自己对我的病,倒也没什么惊奇骇异的地方,可是我痰里的血丝,脸上的苍白,和身体的瘦削,却把我那已经守了好几年寡的老母急坏了,因为我那短命的父亲,也是患这同样的病而死去的。于是她就四处的去求神拜佛,采药求医,急得连粗茶淡饭都无心食用,头上的白发,也似乎一天一天的加多起来了。我哩!恋爱已经失败了,学业也已辍了,对于此生,原已没有多大的野心,所以就落得去由她摆布,积极地虽尽不得孝,便消极地尽了我的顺。初回家的一年中间,我简直门外也不出一步,各色各样的奇形的草药,和各色各样的异味的单方,差不多都尝了一个遍。但是怪得很,连我自己都满以为没有希望的这致命的病症,一到了回国后所经过的第二个春天,竟似乎有神助似地,忽然减轻了,夜热也不再发,盗汗也居然止住,痰里的血丝早就没有了。我的娘的喜欢,当然是不必说,就是
\newpage
在家里替我煮药缝衣,代我操作一切的我那位妹妹,也同春天的天气一样,时时展开了她的愁眉,露出了她那副特有的真真是讨人欢喜的笑容。到了初夏,我药也已经不服,有兴致的时候,居然也能够和她们一道上山前山后去采采茶,摘摘菜,帮她们去服一点小小的劳役了。是在这一年的——回家后第三年的——秋天,在我们家里,同时候发生了两件似喜而又可悲,说悲却也可喜的悲喜剧。第一,就是我那妹妹的出嫁,第二,就是我定在城里的那家婚约的解除。妹妹那年十九岁了,男家是只隔一支山岭的一家乡下的富家。他们来说亲的时候,原是因为我们祖上是世代读书的,总算是和诗礼人家攀婚的意思。定亲已经定过了四五年了,起初我娘却嫌妹妹年纪太小。不肯马上准他们来迎娶,后来就因为我的病,一搁就又搁起了两三年。到了这一回,我的病总算已经恢复,而妹妹却早到了该结婚的年龄了。男家来一说,我娘也就应允了他们,也算完了她自己的一件心事。至于我的这家亲事呢,却是我父亲在死的前一年为我定下的,女家是城里的一家相当有名的旧家。那时候我的年纪虽还很小,而我们家里的不动产却着实还有一点可观。并且我又是一个长子,将来家里要培植我读书处世是
\newpage
无疑的,所以那一家旧家居然也应允了我的婚事。以现在的眼光看来,这门亲事,当然是我们去竭力高攀的,因为杭州人家的习俗,是吃粥的人家的女儿,非要去嫁吃饭的人家不可的。还有乡下姑娘,嫁往城里,倒是常事,城里的千金小姐,却不大会下嫁到乡下来的,所以当时的这个婚约,起初在根本上就有点儿不对。后来经我父亲的一死,我们家里,丧葬费用,就用去了不少。嗣后年复一年,母子三人,只吃着家里的死饭。亲族戚属,少不得又要对我们孤儿寡妇,时时加以一点剥削。母亲又忠厚无用,在出卖田地山场的时候,也不晓得市价的高低,大抵是任凭族人在从中勾搭。就因这种种关系的结果,到我考取了官费,上日本去留学的那一年,我们这一家世代读书的翁家山上的旧家,已经只剩得一点仅能维持衣食的住屋山场和几块荒田了。当我初次出国的时候,承蒙他们不弃,我那未来的亲家,还送了我些赆仪路肴。后来于寒假暑假回国的期间,也曾央原媒来催过完姻。可是接着就是我那致命的病症的发生,与我的学校的中辍,于是两三年中,他们和我们的中间,便自然而然的断绝了交往。到了这一年的晚秋,当我那妹妹嫁后不久的时候,女家忽而又央了原媒来对母亲说:“你
\newpage
们的大少爷,有病在身,婚娶的事情,当然是不大相宜的,而他家的小姐,也已经下了绝大的决心,立志终身不嫁了,所以这一个婚约还是解除了的好。”说着就打开包裹,将我们传红时候交去的金玉如意,红绿帖子等,拿了出来,退还了母亲。我那忠厚老实的娘,人虽则无用,但面子却是死要的,一听了媒人的这一番说话,目瞪口僵,立时就滚下了几颗眼泪来。幸亏我在旁边,做好做歹的对娘劝慰了好久,她才含着眼泪,将女家的回礼及八字全帖等检出,交还了原媒。媒人去后,她又上山后我父亲的坟边去大哭了一场。直到傍晚,我和同族邻人等一道去拉她回来,她在路上,还流着满脸的眼泪鼻涕,在很伤心地呜咽。这一出赖婚的怪剧,在我只有高兴,本来是并没有什么大不了的,可是由头脑很旧的她看来,却似乎是翁家世代的颜面家声都被他们剥尽了。自此以后,一直下来,将近十年,我和她母子二人,就日日的寡言少笑,相对茕茕,直到前年的冬天,我那妹夫死去,寡妹回来为止,两人所过的,都是些在炼狱里似的沉闷
的日子。 

说起我那寡妹,她真也是前世不修。人虽则很
\newpage
长大,身体虽则很强壮,但她的天性,却永远是一个天真活泼的小孩子。嫁过去那一年,来回郎的时候,她还是笑嘻嘻地如同上城里去了一趟回来了的样子,但双满月之后,到年下边回来的时候,从来不晓得悲泣的她,竟对我母亲掉起眼泪来了。她们夫家的公公虽则还好,但婆婆的繁言吝啬,小姑的刻薄尖酸和男人的放荡凶暴,使她一天到晚过不到一刻安闲自在的生活。工作操劳本系是她在家里的时候所惯习的,倒并不以为苦,所最难受的,却是多用一枝火柴,也要受婆婆责备的那一种俭约到不可思议的生活状态。还有两位小姑,左一句尖话,右一句毒语,仿佛从前我娘的不准他们早来迎娶,致使她们的哥哥染上了游荡的恶习,在外面养起了女人这一件事情,完全是我妹妹的罪恶。结婚之后,新郎的恶习,仍旧改不过来,反而是在城里他那旧情人家里过的日子多,在新房里过的日子少。这一笔账,当然又要写在我妹妹的身上。婆婆说她不会侍奉男人,小姑们说她不会劝,不会骗。有时候公公看得难受,替她申辩一声,婆婆就尖着喉咙,要骂上公公的脸去:“你这老东西!脸要不要,脸要不要,你这扒灰老!”因我那妹夫,过的是这一种不自然的生活,所以前年夏天,就染了急病死
\newpage
掉了,于是我那妹妹又多了一个克夫的罪名。妹妹年轻守寡,公公少不得总要对她客气一点,婆婆在这里就算抓住了扒灰的证据,三日一场吵,五日一场闹,还是小事,有几次在半夜里,两老夫妇还会大哭大骂的喧闹起来。我妹妹于有一回被骂被逼得特别厉害的争吵之后,就很坚决地搬回到了家里来住了。自从她回来之后,我娘非但得到了一个很大的帮手,就是我
们家里的沉闷的空气,也缓和了许多。 

这就是和你别后,十几年来,我在家里所过的生活的大概。平时非但不上城里去走走,当风雪盈途的冬季,我和我娘简直有好几个月不出门外的时候。我妹妹回来之后,生活又约略变过了。多年不做的焙茶事业,去年也竟出产了一二百斤。我的身体,经了十几年的静养,似乎也有一点把握了。从今年起,我并且在山上的晏公祠里参加入了一个训蒙的小学,居然也做了一位小学教师。但人生是动不得的,稍稍一动,就如滚石下山,变化便要连接不断的簇生出来。我因为在教教书,而家里头又勉强地干起了一点事业,今年夏季居然又有人来同我议婚了。新娘是近邻乡村里的一位老处女,今年二十七岁,家里虽称不得富
\newpage
有,可也是小康之家。这位新娘,因为从小就读了些书,曾在城里进过学堂,相貌也还过得去——好几年前,我曾经在一处市场上看见过她一眼的——故而高不凑,低不就,等闲便度过了她的锦样的青春。我在教书的学校里的那位名誉校长——也是我们的同族——本来和她是旧亲,所以这位校长,就在中间做了个传红线的冰人。我独居已经惯了,并且身体也不见得分外强健,若一结婚,难保得旧病的不会复发,故而对这门亲事当初是断然拒绝了的。可是我那年老的母亲,却仍是雄心未死,还在想我结一头亲,生下几个玉树芝兰来,好重振重振我们的这已经坠落了很久的家声,于是这亲事就又同当年生病的时候服草药一样,勉强地被压上我的身上来了。我哩,本来也已经入了中年了,百事原都看得很穿,又加以这十几年的疏散和无为,觉得在这世上任你什么也没甚大不了的事情,落得随随便便的过去,横竖是来日也无多了,只教我母亲喜欢的话,那就是我稍稍牺牲一点意见也使得。于是这婚议,就在很短的时间里,成熟得妥妥帖帖,现在连迎娶的日期也已经拣好了,是旧历九月十
二。 

\newpage

是因为这一次的结婚,我才进城里去买东西,才发见了多年不见的你这老友的存在,所以结婚之日,我想请你来我这里吃喜酒,大家来谈谈过去的事情。你的生活,从你的日记和著作中看来,本来也是同云游的僧道一样的。让出一点工夫来,上这一区僻静的乡间来住几日,或者也是你所喜欢的事情。你来,你一定来,我们又可以回顾一去而不复返的少年时代
。 

我娘的房间里,有起响动来了,大约天总就快亮了罢。这一封信,整整地费了我一夜的时间和心血。通宵不睡,是我回国以后十几年来不曾有过的经验,你单只看取了我的这一点热忱,我想你也不好意思
不来。 

啊,鸡在叫了,我不想再写下去了,还是让我
们见面之后再来谈罢! 


一九三二年九月 翁则生上 

刚在北平住了个把月,重回到上海的翌日,和
\newpage
我进出的一家书铺里,就送了这一封挂号加邮托转交的厚信来。我接到了这信,捏在手里,起初还以为是一位我认识的作家,寄了稿子来托我代售的。但翻转信背一看,却是杭州翁家山的翁某某所发,我立时就想起了那位好学不倦,面容妩媚,多年不相闻问的旧同学老翁。他的名字叫翁矩,则生是他的小名。人生得短小娟秀,皮色也很白净,因而看起来总觉得比他的实际年龄要小五六岁。在我们的一班里,算他的年纪最小,操体操的时候,总是他立在最后的,但实际上他也只不过比我小了两岁。那一年寒假之后,和他同去房州避寒,他的左肺尖,已经被结核菌损蚀得很厉害了。住不上几天,一位也住在那近边养肺病的日本少女,很热烈地和他要好了起来,结果是那位肺病少女的因兴奋而病剧,他也就同失了舵的野船似地迁回到了中国。以后一直十多年,我虽则在大学里毕了业,但关于他的消息,却一向还不曾听见有人说起过。拆开了这封长信,上书室去坐下,从头至尾细细读完之后,我呆视着远处,茫茫然如失了神的样子,脑子里也触起了许多感慨与回思。我远远的看出了他的那种柔和的笑容,听见了他的沉静而又清澈的声气。直到天将暗下去的时候,我一动也不动,还坐在那里
\newpage
呆想,而楼下的家人却来催吃晚饭了。在吃晚饭的中间,我就和家里的人谈起了这位老同学,将那封长信的内容约略说了一遍。家里的人,就劝我落得上杭州去旅行一趟,象这样的秋高气爽的时节,白白地消磨在煤烟灰土很深的上海,实在有点可惜,有此机会,
落得去吃吃他的喜酒。 

第二天仍旧是一天晴和爽朗的好天气,午后二点钟的时候,我已经到了杭州城站,在雇车上翁家山去了。但这一天,似乎是上海各洋行与机关的放假的日子,从上海来杭州旅行的人,特别的多。城站前面停在那里候客的黄包车,都被火车上下来的旅客雇走了,不得已,我就只好上一家附近的酒店去吃午饭。在吃酒的当中,问了问堂倌以去翁家山的路径,他便
很详细地指示我说: 

“你只教坐黄包车到旗下的陈列所,搭公共汽车到四眼井下来走上去好了。你又没有行李,天气又
这么的好,坐黄包车直去是不上算的。” 

得到了这一个指教,我就从容起来了,慢慢的
\newpage
喝完了半斤酒,吃了两大碗饭,从酒店出来,便坐车到了旗下。恰好是三点前后的光景,湖六段的汽车刚载满了客人,要开出去。我到了四眼井下车,从山下稻田中间的一条石板路走进满觉陇去的时候,太阳已经平西到了三五十度斜角度的样子,是牛羊下来,行人归舍的时刻了。在满觉陇的狭路中间,果然遇见了许多中学校的远足归来的男女学生的队伍。上水乐洞口去坐下喝了一碗清茶,又拉住了一位农夫,问了声
翁则生的名字,他就晓得得很详细似地告诉我说: 

“是山上第二排的朝南的一家,他们那间楼房顶高,你一上去就可以看得见的。则生要讨新娘子了,这几天他们正在忙着收拾。这时候则生怕还在晏公
祠的学堂里哩。” 

谢过了他的好意,付过了茶钱,我就顺着上烟霞洞去的石级,一步一步的走上了山去。渐走渐高,人声人影是没有了,在将暮的晴天之下,我只看见了许多树影。在半山亭里立住歇了一歇,回头向东南一望,看得见的,只是些青葱的山,和如云的树,在这些绿树丛中又是些这儿几点,那儿一簇的屋瓦与白墙
\newpage


“啊啊,怪不得他的病会得好起来了,原来翁
家山是在这样的一个好地方。” 

烟霞洞我儿时也曾来过的,但当这样晴爽的秋天,于这一个西下夕阳东上月的时刻,独立在山中的空亭里,来仔细赏玩景色的机会,却还不曾有过。我看见了东天的已经满过半弓的月亮,心里正在羡慕翁则生他们老家的处地的幽深,而从背后又吹来了一阵微风,里面竟含满着一种说不出的撩人的桂花香气。
 


“啊……” 


我又惊异了起来: 

“原来这儿到这时候还有桂花?我在以桂花著名的满觉陇里,倒不曾看到,反而在这一块冷僻的山
里面来闻吸浓香,这可真也是奇事了。” 

这样的一个人独自在心中惊异着,闻吸着,赏
\newpage
玩着,我不知在那空亭里立了多少时候,突然从脚下树丛深处,却幽幽的有晚钟声传过来了,东嗡,东嗡地这钟声实在真来得缓慢而凄清。我听得耐不住了,拔起脚跟.一口气就走上了山顶,走到了那个山下农夫曾经教过我的烟霞洞西面翁则生家的近旁。约莫离他家还有半箭路远的时候,我一面喘着气,一面就放
大了喉咙向门里面叫了起来: 


“喂,老翁!老翁!则生!翁则生!” 

听见了我的呼声,从两扇关在那里的腰门里开出来答应的却不是被我所唤的翁则生自己,而是我从来也没有见过面的,比翁则生略高三五分的样子,身体强健,两颊微红,看起来约莫有二十四五的一位女
性。 

她开出了门,一眼看见了我,就立住脚惊疑似地略呆了一呆。同时我看见她脸上却涨起了一层红晕,一双大眼睛眨了几眨,深深地吞了一口气。她似乎已经镇静下去了,便很腼腆地对我一笑。在这一脸柔和的笑容里,我立时就看到了翁则生的面相与神气,
\newpage
当然她是则生的妹妹无疑了,走上了一步,我就也笑
着问她说: 


“则生不在家么?你是他的妹妹不是?” 

听了我这一句问话,她脸上又红了一红,柔和
地笑着,半俯了头,她方才轻轻地回答我说: 

“是的,大哥还没有回家,你大约是上海来的
客人罢?吃中饭的时候,大哥还在说哩!” 

这沉静清澈的声气,也和翁则生的一色而没有
两样。 


“是的,我是从上海来的。” 


我接着说: 

“我因为想使则生惊骇一下,所以电报也不打一个来通知,接到他的信后,马上就动身来了。不过你们大哥的好日也太逼近了,实在可也没有写一封信
\newpage

来通知的时间余裕。” 

“你请进来罢,坐坐吃碗茶,我马上去叫了他来,怕他听到了你来,真要惊喜得象疯了一样哩。”

走上台阶,我还没有进门,从客堂后面的侧门里,却走出了一位头发雪白,面貌清癯,大约有六十内外的老太太来。她的柔和的笑容,也是和她的女儿儿子的笑容一色一样的。似乎已经听见了我们在门口
所交换过的谈话了,她一开口就对我说: 

“是郁先生么?为什么不写一封快信来通知?则生中上还在说,说你若要来,他打算进城上车站去接你去的。请坐,请坐,晏公祠只有十几步路,让我去叫他来罢,怕他真要高兴得象什么似的哩。”说完了,她就朝向了女儿,吩咐她上厨下去烧碗茶来。她自己却踏着很平稳的脚步,走出大门,下台阶去通知
则生去了。 


“你们老太太倒还轻健得很。” 

\newpage

“是的,她老人家倒还好。你请坐罢,我马上
起了茶来。” 

她上厨下去起茶的中间,我一个人,在客堂里倒得了一个细细观察周围的机会。则生他们的住屋,是一间三开间而有后轩后厢房的楼房。前面阶沿外走落台阶,是一块可以造厅造厢楼的大空地。走过这块数丈见方的空地,再下两级台阶,便是村道了。越村道而下,再低数尺,又是一排人家的房子。但这一排房子,因为都是平屋,所以挡不杀翁则生他们家里的眺望。立在翁则生家的空地里,前山后山的山景,是依旧历历可见的。屋前屋后,一段一段的山坡上,都长着些不大知名的杂树,三株两株夹在这些杂树中间,树叶短狭,叶与细枝之间,满撒着锯末似的黄点的,却是木犀花树。前一刻在半山空亭里闻到的香气,源头原来就系出在这一块地方的。太阳似乎已下了山,澄明的光里,已经看不见日轮的金箭,而山脚下的树梢头,也早有一带晚烟笼上了。山上的空气,真静得可怜,老远老远的山脚下的村里,小儿在呼唤的声音,也清晰地听得出来。我在空地里立了一会,背着手又踱回到了翁家的客厅,向四壁挂在那里的书画一
\newpage
看,却使我想起了翁则生信里所说的事实。琳琅满目,挂在那里的东西,果然是件件精致,不象是乡下人家的俗恶的客厅。尤其使我看得有趣的,是陈豪写的一堂《归去来辞》的屏条,墨色的鲜艳,字迹的秀腴,有点象董香光而更觉得柔媚。翁家的世代书香,只须上这客厅里来一看就可以知道了。我立在那里看字画还没有看得周全,忽而背后门外老远的就飞来了几
声叫声: 


“老郁!老郁!你来得真快!” 

翁则生从小学校里跑回来了,平时总很沉静的他,这时候似乎也感到了一点兴奋。一走进客堂,他握住了我的两手,尽在喘气,有好几秒钟说不出话来。等落在后面的他娘走到的时候,三人才各放声大笑了起来。这时候他妹妹也已经将茶烧好,在一个朱漆
盘里放着三碗搬出来摆上桌子来了。 

“你看,则生这小孩,他一听见我说你到了,
就同猴子似的跳回来了。”他娘笑着对我说。 

\newpage

“老翁!说你生病生病,我看你倒仍旧不见得衰老得怎么样,两人比较起来,怕还是我老得多哩?
” 

我笑说着,将脸朝向了他的妹妹,去征她的同意。她笑着不说话,只在守视着我们的欢喜笑乐的样子。则生把头一扭,向他娘指了一指,就接着对我说
: 

“因为我们的娘在这里,所以我不敢老下去吓。并且媳妇儿也还不曾娶到,一老就得做老光棍了,
那还了得!” 

经他这么一说,四个人重又大笑起来了,他娘的老眼里几乎笑出了眼泪。则生笑了一会,就重新想
起了似的替他妹妹介绍说: 

“这是我的妹妹,她的事情,你大约是晓得的
罢?我在那信里是写得很详细的。” 

“我们可不必你来介绍了,我上这儿来,头一
\newpage

个见到的就是她。” 

“噢,你们倒是有缘啊!莲,你猜这位郁先生
的年纪,比我大呢,还是比我小?” 

他妹妹听了这一句话,面色又涨红了,正在嗫
嚅困惑的中间,她娘却止住了笑,问我说: 


“郁先生,大约是和则生上下年纪罢?” 


“那里的话,我要比他大得多哩。” 


“娘,你看还是我老呢,还是他老?” 

则生又把这问题转向了他的母亲。他娘仔细看
了我一眼,就对他笑骂般的说: 

“自然是郁先生来得老成稳重,谁更象你那样
的不脱小孩子脾气呢!” 

说着,她就走近了桌边,举起茶碗来请我喝茶
\newpage
。我接过来喝了一口,在茶里又闻到了一种实在是令人欲醉的桂花香气。掀开了茶碗盖,我俯首向碗里一看,果然在绿莹莹的茶水里散点着有一粒一粒的金黄的花瓣。则生以为我在看茶叶,自己拿起了一碗喝了
一口,他就对我说: 


“这茶叶是我们自己制的,你说怎么样?” 

“我并不在看茶叶,我只觉这触鼻的桂花香气
,实在可爱得很。” 

“桂花吗?这茶叶里的还是第一次开的早桂,现在在开的迟桂花,才有味哩!因为开得迟,所以日
子也经得久。” 

“是的是的,我一路上走来,在以桂花著名的满觉陇里,倒闻不着桂花的香气。看看两旁的树上,都只剩了一簇一簇的淡绿的桂花托子了,可是到了这里,却同做梦似地,所闻吸的尽是这种浓艳的气味。老翁,你大约是已经闻惯了,不觉得什么罢?我……

\newpage
我……” 

说到了这里,我自家也忍不住笑了起来。则生尽管在追问我,“你怎么样?你怎么样?”到了最后
,我也只好说了: 

“我,我闻了,似乎要起性欲冲动的样子。”

则生听了,马上就大笑了起来,他的娘和妹妹虽则并没有明确地了解我们的说话的内容,但也晓得我们是在说笑话,母女俩便含着微笑,上厨下去预备
晚饭去了。 

我们两人在客厅上谈谈笑笑,竟忘记了点灯,一道银样的月光,从门里晒进来。则生看见了月亮,
就站起来想去拿煤油灯,我却止住了他,说: 

“在月光底下清谈,岂不是很好么?你还记不
记得起,那一年在井之头公园里的一夜游行?” 

所谓那一年者,就是翁则生患肺病的那一年秋天。他因为用功过度,变成了神经衰弱症。有一天他
\newpage
课也不去上,竟独自一个在公寓里发了一天的疯。到了傍晚,他饭也不吃。从公寓里跑出去了。我接到了公寓主人的注意,下学回来,就远远的在守视着他,看他走出了公寓,就也追踪着他,远远地跟他一道到了井之头公园。从东京到井之头公园去的高架电车,本来是有前后的两乘,所以在电车上,我和他并不遇着。直到下车出车站之后,我假装无意中和他冲见了似的同他招呼了。他红着双颊,问我这时候上这野外来干什么,我说是来看月亮的,记得那一晚正是和这天一样地有月亮的晚上。两人笑了一笑,就一道的在井之头公园的树林里走到了夜半方才回来。后来听他的自白,他是在那一天晚上想到井之头公园去自杀的,但因为遇见了我,谈了半夜,胸中的烦闷,有一半消散了,所以就同我一道又转了回来。“无限胸中烦闷事,一宵清话又成空!”他自白的时候,还念出了这两句诗来,借作解嘲。以后他就因伤风而发生了肺
炎,肺炎愈后,就一直的为结核菌所压倒了。 

谈了许多怀旧话后,话头一转,我就提到了他
的这一回的喜事。 

\newpage

“这一回的喜事么?我在那信里也曾和你说过
。” 

谈话的内容,一从空想追怀转向了现实,他的
声气就低了下去,又回复了他旧日的沉静的态度。 

“在我是无可无不可的,对这事情最起劲的,倒是我的那位年老的娘。这一回的一切准备麻烦,都是她老人家在替我忙的。这半个月中间,她差不多日日跑城里。现在是已经弄得完完全全,什么都预备好了,明朝一日,就要来搭灯彩,下午是女家送嫁妆来,后天就是正日。可是老郁,有一件事情,我觉得很难受,就是莲儿——这是我妹妹的小名——近来,似乎是很不高兴的样子,她话虽则不说,但因为她是很天真的缘故,所以在态度上表情上处处我都看得出来。你是初同她见面,所以并不觉得什么,平时她着实要活泼哩,简直活泼得同现代的那些时髦女郎一样,不过她的活泼是天性的纯真,而那些现代女郎,却是学来的时髦。……按说哩,这心绪的恶劣,也是应该的,她虽则是一个纯真的小孩子,但人非木石,究竟总有一点感情,看到了我们这里的婚事热闹,无论如
\newpage
何,总免不得要想起她自己的身世凄凉的。并且还有一个最重要的动机,仿佛是她在觉得自己今后的寄身无处。这儿虽是娘家,但她却是已经出过嫁的女儿了,哥哥讨了嫂嫂,她还有什么权利再寄食在娘家呢?所以我当这婚事在谈起的当初,就一次两次的对她说过了,不管它怎样,她总是我的妹妹,除非她要再嫁,则没有话说,要是不然的话,那她是一辈子有和我同居,和我对分财产的权利的,请她千万不要自己感到难过。这一层意思,她原也明白,我的性情,她是晓得的,可是不晓得怎么,她近来似乎总有点不大安闲的样子。你来得正好,顺便也可以劝劝她。并且明天发嫁妆结灯彩之类的事情,怕她看了又要想到自己的身世,我想明朝一早就叫她陪你出去玩去,省得她
在家虽一个人在暗中受苦。” 

“那好极了,我明天就陪她出去玩一天回来。

“那可不对,假使是你陪她出去玩的话,那是形迹更露,愈加要使她难堪了。非要装作是你要她去作陪不行。仿佛是你想出去玩,但我却没有工夫陪你,所以只好勉强请她和你一道出去。要这样,她才安
\newpage

逸。” 

“好,好,就这么办,明天我要她陪我去逛五
云山去。” 

正谈到了这里,他的那位老母从客室后面的那扇侧门里走出来了,看到了我们坐在微明灰暗的客室
里谈天,她又笑了起来说: 

“十几年不见的一段总账,你们难道想在这几刻工夫里算它清来么?有什么话谈得那么起劲,连灯都忘了点一点?则生,你这孩子真象是疯了,快立起
来,把那盏保险灯点上。” 

说着她又跑回到了厨下,去拿了一盒火柴出来。则生爬上桌子,在点那盏悬在客室正中的保险灯的时候,她就问我吃晚饭之先,要不要喝酒。则生一边
在点灯,一边就从肩背上叫他娘说: 

“娘,你以为他也是肺痨病鬼么?郁先生是以

\newpage
喝酒出名的。” 

“那么你快下来去开坛去罢,今天挑来的那两
坛酒,不晓得好不好,请郁先生尝尝看。” 

他娘听了他的话后,就也昂起了头,一面在看
他点灯,一面在催他下来去开酒去。 

“幸而是酒,请郁先生先尝一尝新,倒还不要
紧,要是新娘子,那可使不得。” 

他笑说着从桌子上跳了下来,他娘眼睛望着了
我,嘴唇却朝着了他啐了一声说: 


“你看这孩子,说话老是这样不正经的!” 


“因为他要做新郎官了,所以在高兴。” 

我也笑着对他娘说了一声,旋转身就一个人踱出了门外,想看一看这翁家山的秋夜的月明,屋内且
让他们母子俩去开酒去。 

\newpage

月光下的翁家山,又不相同了。从树枝里筛下来的千条万条的银线,象是电影里的白天的外景。不知躲在什么地方的许多秋虫的鸣唱,骤听之下,满以为在下急雨。白天的热度,日落之后,忽然收敛了,于是草木很多的这深山顶上,就也起了一层白茫茫的透明雾障。山上电灯线似乎还没有接上,远近一家一家看得见的几点煤油灯光,仿佛是大海湾里的渔灯野火。一种空山秋夜的沉默的感觉,处处在高压着人,使人肃然会起一种畏敬之思。我独立在庭前的月光亮里看不上几分钟,心里就有点寒竦竦的怕了起来,回身再走回客室,酒菜杯筷,都已热气蒸腾的摆好在那
里候客了。 

四个人当吃晚饭的中间,则生又说了许多笑话。因为在前回听取了一番他所告诉我的衷情之后,我于举酒杯的瞬间,偷眼向他妹妹望望,觉得在她的柔和的笑脸上,的确似乎是有一种说不出的悲寂的表情流露在那里的样子。这一餐晚饭,吃尽了许多时间,我因为白天走路走得不少,而谈话之后又感到了一点兴奋,肚子有点饿了,所以酒和菜,竟吃得比平时要多一倍。到了最后将快吃完的当儿,我就向则生提出
\newpage

说: 

“老翁,五云山我倒还没有去玩过,明天你可
不可以陪我一道去玩一趟?” 


则生仍复以他的那种滑稽的口吻回答我说: 

“到了结婚的前一日,新郎官哪里走得开呢,还是改天再去罢。等新娘子来了之后,让新郎新妇抬
了你去烧香,也还不迟。” 

我却仍复主张着说,明天非去不行。则生就说

“那么替你去叫一顶轿子来,你坐了轿子去,
横竖是明天轿夫会来的。” 


“不行不行,游山玩水,我是喜欢走的。” 


“你认得路么?” 

“你们这一种乡下的僻路,我哪里会认得呢?
\newpage



“那就怎么办呢?……” 

则生抓着头皮,脸上露出了一脸为难的神气。
停了一二分钟,他就举目向他的妹妹说: 

“莲!你怎么样!你是一位女豪杰,走路又能
走,地理又熟悉,你替我陪了郁先生去怎么样?” 

他妹妹也笑了起来,举起眼睛来向她娘看了一
眼。接着她娘就说: 

“好的,莲,还是你陪了郁先生去罢,明天你
大哥是走不开的。” 

我一看她脸上的表情,似乎已经有了答应的意
思了,所以又追问了她一声说: 

“五云山可着实不近哩,你走得动的么?回头
走到半路,要我来背,那可办不到。” 

\newpage

她听了这话,就真同从心坎里笑出来的一样笑
着说: 

“别说是五云山,就是老东岳,我们也一天要
往返两次哩。” 

从她的红红的双颊,挺突的胸脯,和肥圆的肩臂看来,这句话也决不是她夸的大口。吃完晚饭,又谈了一阵闲天,我们因为明天各有忙碌的操作在前,
所以一早就分头到房里去睡了。 

山中的清晓,又是一种特别的情景。我因为昨天夜里多喝了一点酒,上床去一睡,就同大石头掉下海里似的,一直就酣睡到了天明。窗外面吱吱唧唧的鸟声喧噪得厉害,我满以为还是夜半,月明将野鸟惊醒了,但睁开眼掀开帐子来一望,窗内窗外已饱浸着晴天爽朗的清晨光线,窗子上面的一角,却已经有一缕朝阳的红箭射到了。急忙滚出了被窝,穿起衣服,跑下楼去一看,他们母子三人,也已梳洗得妥妥服服,说是已经在做了个把钟头的事情之后,平常他们总是于五点钟前后起床的。这一种日出而作,日入而息
\newpage
的山中住民的生活秩序,又使我对他们感到了无穷的敬意。四人一道吃过了早餐,我和则生的妹妹,就整了一整行装,预备出发。临行之际,他娘又叫我等一下子,她很迅速地跑上楼上去取了一枝黑漆手杖下来,说,这是则生生病的时候用过的,走山路的时候,用它来撑扶撑扶,气力要省得多。我谢过了她的好意
,就让则生的妹妹上前带路,走出了他们的大门。 

早晨的空气,实在澄鲜得可爱。太阳已经升高了,但它的领域,还只限于屋檐,树梢,山顶等突出的地方。山路两旁的细草上,露水还没有干,而一味清凉触鼻的绿色草气,和入在桂花香味之中,闻了好象是宿梦也能摇醒的样子。起初还在翁家山村内走着,则生的妹妹,对村中的同性,三步一招呼,五步一立谈的应接得忙不暇给。走尽了这村子的最后一家,沿了入谷的一条石板路走上下山面的时候,遇见的人也没有了,前面的眺望,也转换了一个样子。朝我们去的方向看去,原又是冈峦的起伏和别墅的纵横,但稍一住脚,掉头向东面一望,一片同呵了一口气的镜子似的湖光,却躺在眼下了。远远从两山之间的谷顶望去,并且还看得出一角城里的人家,隐约藏躲在尚
\newpage

未消尽的湖雾当中。 

我们的路先朝西北,后又向西南,先下了山坡,后又上了山背,因为今天有一天的时间,可以供我们消磨,所以一离了村境,我就走得特别的慢。每这里看看,那里看看的看个不住。若看见了一件稍可注意的东西,那不管它是风景里的一点一堆,一山一水,或植物界的一草一木与动物界的一鸟一虫,我总要拉住了她,寻根究底的问得它仔仔细细。说也奇怪,小时候只在村里的小学校里念过四年书的她——这是她自己对我说的——对于我所问的东西,却没有一样不晓得的。关于湖上的山水古迹,庙宇楼台哩,那还不要去管它,大约是生长在西湖附近的人,个个都能够说出一个大概来的,所以她的知道得那么详细,倒还在情理之中,但我觉得最奇怪的,却是她的关于这西湖附近的区域之内的种种动植物的知识。无论是如何小的一只鸟、一个虫、一株草、一棵树,她非但各能把它们的名字叫出来,并且连几时孵化,几时他迁,几时呜叫,几时脱壳,或几时开花,几时结实,花的颜色如何,果的味道如何等,都说得非常有趣而详尽,使我觉得仿佛是在读一部活的桦候脱的《赛儿鹏
\newpage
自然史》(G.White's Natural History and Antiquities of Selborne)。而桦候脱的书,却决没有叙述得她那么朴质自然而富于刺激,因为听听她那种舒徐清澈的语气,看看她那一双天生成象饱使过耐吻胭脂棒般的红唇,更加上以她所特有的那一脸微笑,在知识分子之外还不得不添一种情的成分上去,于书的趣味之上更要兼一层人的风韵在里头。我们慢慢的谈着天,走着路,不上一个钟头的光景,我竟恍恍惚惚,象又回复了青春时代似的完全为她迷倒了。

她的身体,也真发育得太完全,穿的虽是一件乡下裁缝做的不大合式的大绸夹袍,但在我的前面一步一步的走去,非但她的肥突的后部,紧密的腰部,和斜圆的胫部的曲线,看得要簇生异想,就是她的两只圆而且软的肩膊,多看一歇,也要使我贪鄙起来。立在她的前面和她讲话哩,则那一双水汪汪的大眼,那一个隆正的尖鼻,那一张红白相间的椭圆嫩脸,和因走路走得气急,一呼一吸涨落得特别快的那个高突的胸脯,又要使我恼杀。还有她那一头不曾剪去的黑发哩,梳的虽然是一个自在的懒髻,但一映到了她那
\newpage
个圆而且白的额上,和短而且腴的颈际,看起来,又格外的动人。总之,我在昨天晚上,不曾在她身上发见的康健和自然的美点,今天因这一回的游山,完全被我观察到了。此外我又在她的谈话之中,证实了翁则生也和我曾经讲到过的她的生性的活泼与天真。譬如我问她今年几岁了?她说,二十八岁。我说这真看不出,我起初还以为你只有二十三四岁,她说,女人不生产是不大会老的。我又问她,对于则生这一回的结婚,你有点什么感触?她说,另外也没有什么,不过以后长住在娘家,似乎有点对不起大哥和大嫂。象这一类的纯粹真率的谈话,我另外还听取了许多许多,她的朴素的天性,真真如翁则生之所说,是一个永
久的小孩子的天性。 

爬上了龙井狮子峰下的一处平坦的山顶,我于听了一段她所讲的如何栽培茶叶,如何摘取焙烘,与那时候的山家生活的如何紧张而有趣的故事之后,便在路旁的一块大岩石上坐下了。遥对着在晴天下太阳光里躺着的杭州城市,和近水遥山,我的双眼只凝视着苍空的一角,有半晌不曾说话。一边在我的脑里,却只在回想着德国的一位名延生(Jensen)的
\newpage
作家所著的一部小说《野紫薇爱立喀》(Die Braune Erika)。这小说后来又有一位英国的作家哈特生(Hudson)摹仿了,写了一部《绿阴》(Green Mansions)。两部小说里所描写的,都是一个极可爱的生长在原野里的天真的女性,而女主人公的结果,后来都是不大好的。我沉默着痴想了好久,她却从我背后用了她那只肥
软的右手很自然地搭上了我的肩膀。 


“你一声也不响的在那里想什么?” 

我就伸上手去把她的那只肥手捏住了,一边就扭转了头微笑着看入了她的那双大眼,因为她是坐在我的背后的。我捏住了她的手又默默对她注视了一分钟,但她的眼里脸上却丝毫也没有羞惧兴奋的痕迹出现,她的微笑,还依旧同平时一点儿也没有什么的笑容一样。看了我这一种奇怪的形状,她过了一歇,反
又很自然的问我说: 


“你究竟在那里想什么?” 

\newpage

倒是我被她问得难为情起来了,立时觉得两颊就潮热了起来。先放开了那只被我捏住在那儿的她的手,然后干咳了两声,最后我就鼓动了勇气,发了一
声同被绞出来似的答语: 


“我……我在这儿想你!” 


“是在想我的将来如何的和他们同住么?” 

她的这句反问,又是非常的率真而自然,满以为我是在为她设想的样子。我只好沉默着把头点了几
点,而眼睛里却酸溜溜的觉得有点热起来了。 

“啊,我自己倒并没有想得什么伤心,为什么
,你,你却反而为我流起眼泪来了呢?” 

她象吃了一惊似的立了起来问我,同时我也立起来了,且在将身体起立的行动当中,乘机拭去了我的眼泪。我的心地开朗了,欲情也净化了,重复向南慢慢走上岭去的时候,我就把刚才我所想的心事,尽情告诉了她。我将那两部小说的内容讲给了她听,我
\newpage
将我自己的邪心说了出来,我对于我刚才所触动的那一种自己的心情,更下了一个严正的批判,末后,便
这样的对她说: 

“对于一个洁白得同白纸似的天真小孩,而加以玷污,是不可赦免的罪恶。我刚才的一念邪心,几乎要使我犯下这个大罪了。幸亏是你的那颗纯洁的心,那颗同高山上的深雪似的心,却救我出了这一个险。不过我虽则犯罪的形迹没有,但我的心,却是已经犯过罪的。所以你要罚我的话,就是处我以死刑,我也毫无悔恨。你若以为我是那样卑鄙,而将来永没有改善的希望的话,那今天晚上回去之后,向你大哥母亲,将我的这一种行为宣布了也可以。不过你若以为这是我的一时糊涂,将来是永也不会再犯的话,那请你相信我的誓言,以后请你当我作你大哥一样那么的看待,你若有急有难,有不了的事情,我总情愿以死
来代替着你。” 

当我在对她作这些忏悔的时候,两人起初是慢慢在走的,后来又在路旁坐下了。说到了最后的一节,倒是她反同小孩子似的发着抖,捏住了我的两手,
\newpage
倒入了我的怀里,呜呜咽咽的哭了起来。我等她哭了一阵之后,就拿出了一块手帕来替她揩干了眼泪,将我的嘴唇轻轻地搁到了她的头上。两人偎抱着沉默了好久,我又把头俯了下去,问她,我所说的这段话的意思,究竟明白了没有。她眼看着了地上,把头点了
几点。我又追问了她一声: 


“那么你承认我以后做你的哥哥了不是?” 

她又俯视着把头点了几点,我撒开了双手,又伸出去把她的头捧了起来,使她的脸正对着了我。对我凝视了一会,她的那双泪珠还没有收尽的水汪汪的眼睛,却笑起来了。我乘势把她一拉,就同她搀着手
并立了起来。 

“好,我们是已经决定了,我们将永久地结作最亲爱最纯洁的兄妹。时候已经不早了,让我们快一
点走,赶上五云山去吃午饭去。” 

我这样说着,搀着她向前一走,她也恢复了早

\newpage
晨刚出发的时候的元气,和我并排着走向了前面。 

两人沉默着向前走了几十步之后,我侧眼向她一看,同奇迹似地忽而在她的脸上看出了一层一点儿忧虑也没有的满含着未来的希望和信任的圣洁的光耀来。这一种光耀,却是我在这一刻以前的她的脸上从没有看见过的。我愈看愈觉得对她生起敬爱的心思来了,所以不知不觉,在走路的当中竟接连着看了她好几眼。本来只是笑嘻嘻地在注视着前面太阳光里的五云山的白墙头的她,因为我的脚步的迟乱,似乎也感觉到了我的注意力的分散了,将头一侧,她的双眼,却和我的视线接成了两条轨道。她又笑起来了,同时也放慢了脚步。再向我看了一眼,她才腼腆地开始问
我说: 


“那我以后叫你什么呢?” 

“你叫则生叫什么,就叫我也叫什么好了。”


“那么——大哥!” 

大哥的两字,是很急速的紧连着叫出来的,听
\newpage
到了我的一声高声的“啊!”的应声之后,她就涨红了脸,撒开了手,大笑着跑上前面去了。一面跑,一面她又回转头来,“大哥!”“大哥!”的接连叫了我好几声。等我一面叫她别跑,一面我自己也跑着追上了她背后的时候,我们的去路已经变成了一条很窄的石岭,而五云山的山顶,看过去也似乎是很近了。仍复回复了平时的脚步,两人分着前后,在那条窄岭上缓步的当中,我才觉得真真是成了她的哥哥的样子
,满含着了慈爱,很正经地吩咐她说: 


“走得小心,这一条岭多么险啊!” 

走到了五云山的财神殿里,太阳刚当正午,庙里的人已经在那里吃中饭了。我们因为在太阳底下的半天行路,口已经干渴得象旱天的树木一样,所以一进客堂去坐下,就教他们先起茶来,然后再开饭给我们吃。洗了一个手脸,喝了两三碗清茶,静坐了十几分钟,两人的疲劳兴奋,都已平复了过去,这时候饥饿却抬起头来了,于是就又催他们快点开饭。这一餐只我和她两人对食的五云山上的中餐,对于我正敌得过英国诗人所幻想着的亚力山大王的高宴。若讲到心
\newpage
境的满足.和谐,与食欲的高潮亢进,那恐怕亚力山
大王还远不及当时的我。 

吃过午饭,管庙的和尚又领我们上前后左右去走了一圈。这五云山,实在是高,立在庙中阁上,开窗向东北一望,湖上的群山,都象是青色的土堆了。本来西湖的山水的妙处,就在于它的比舞台上的布景又真实伟大一点,而比各处的名山大川又同盆景似地整齐渺小一点这地方。而五云山的气概,却又完全不同了。以其山之高与境的僻,一般脚力不健的游人是不会到的,就在这一点上,五云山已略备着名山的资格了,更何况前面远处,蜿蜒盘曲在青山绿野之间的,是一条历史上也着实有名的钱塘江水呢?所以若把西湖的山水,比作一只锁在铁笼子里的白熊来看,那这五云山峰与钱塘江水,便是一只深山的野鹿。笼里的白熊,是只能满足满足胆怯无力者的冒险雄心的;至于深山的野鹿,虽没有高原的狮虎那么雄壮,但一
股自由奔放之情,却可以从它那里摄取得来。 

我们在五云山的南面又看了一会钱塘江上的帆影与青山,就想动身上我们的归路了,可是举起头来
\newpage
一望,太阳还在中天,只西偏了没有几分。从此地回去,路上若没有耽搁,是不消两个钟头就能到翁家山上的;本来是打算出来把一天光阴消磨过去的我们,回去得这样的早,岂不是辜负了这大好的时间了么?所以走到了五云山西南角的一条狭路边上的时候,我就又立了下来,拉着了她的手亲亲热热地问了她一声


“莲,你还走得动走不动?” 


“起码三十里路总还可以走的。” 

她说这句话的神气,是富有着自信和决断,一点也不带些夸张卖弄的风情,真真是自然到了极点,所以使我看了不得不伸上手去,向她的下巴底下拨了一拨。她怕痒;缩着头颈笑起来了,我也笑开了大口
,对她说: 

“让我们索性上云栖去罢!这一条是去云栖的便道,大约走下去,总也没有多少路的,你若是走不
动的话,我可以背你。” 

\newpage

两人笑着说着,似乎只转瞬之间,已经把那条狭窄的下山便道走尽了大半了。山下面尽是些绿玻璃似的翠竹,西斜的太阳晒到了这条坞里,一种又清新又寂静的淡绿色的光同清水一样,满浸在这附近的空气里在流动。我们到了云栖寺里坐下,刚喝完了一碗茶,忽而前面的大殿上,有嘈杂的人声起来了,接着就走进了两位穿着分外宽大的黑布和尚衣的老僧来。
知客僧便指着他们夸耀似地对我们说: 

“这两位高僧,是我们方丈的师兄,年纪都快
八十岁了,是从城里某公馆里回来的。” 

城里的某巨公,的确是一位佞佛的先锋,他的名字,我本系也听见过的,但我以为同和尚来谈这些俗天,也不大相称,所以就把话头扯了开去,问和尚大殿上的嘈杂的人声,是为什么而起的。知客僧轻鄙
似地笑了一笑说: 

“还不是城里的轿夫在敲酒钱?轿钱是公馆里
付了来的,这些穷人心实在太凶。” 

\newpage

这一个伶俐世俗的知客僧的说话,我实在听得
有点厌起来了,所以就要求他说: 


“你领我们上寺前寺后去走走罢?” 

我们看过了“御碑”及许多石刻之后,穿出大殿,那几个轿夫还在咕噜着没有起身。我一半也觉得走路走得太多了,一半也想给那个知客僧以一点颜色
看看,所以就走了上去对轿夫说: 

“我给你们两块钱一个人,你们抬我们两人回
翁家山去好不好?” 

轿夫们喜欢极了,同打过吗啡针后的鸦片嗜好
者一样,立时将态度一变,变得有说有笑了。 

知客僧又陪我们到了寺外的修竹丛中,我看了竹上的或刻或写在那里的名字诗句之类,心里倒有点奇怪起来,就问他这是什么意思。于是他也同轿夫他们一样,笑迷迷地对我说了一大串话。我听了他的解释,倒也觉得非常有趣,所以也就拿出了五圆纸币,
\newpage

递给了他,说: 


“我们也来买两枝竹放放生罢!” 

说着我就向立在我旁边的她看了一眼,她却正同小孩子得到了新玩意儿还不敢去抚摸的一样,微笑
着靠近了我的身边轻轻地问我: 


“两枝竹上,写什么名字好?” 


“当然是一枝上写你的,一枝上写我的。” 


她笑着摇摇头说: 

“不好,不好,写名字也不好,两个人分开了
写也不好。” 


“那么写什么呢?” 


“只教把今天的事情写上去就对。” 

\newpage

我静立着想了一会,恰好那知客僧向寺里去拿的油墨和笔也已经拿到了。我拣取了两株并排着的大竹,提起笔来,就各写上了“郁翁兄妹放生之竹”的八个字。将年月日写完之后,我搁下了笔,回头来问她这八个字怎么样,她真象是心花怒放似的笑着,不说话而尽在点头。在绿竹之下的这一种她的无邪的憨
态,又使我深深地,深深地受到了一个感动。 

坐上轿子,向西向南的在竹荫之下走了六七里坂道,出梵村,到闸口西首,从九溪口折入九溪十八涧的山坳,登杨梅岭,到南高峰下的翁家山的时候,太阳已经悬在北高峰与天竺山的两峰之间了。他们的屋里,早已挂上了满堂的灯彩,上面的一对红灯,也已经点尽了一半的样子。嫁妆似乎已经在新房里摆好,客厅上看热闹的人,也早已散了。我们轿子一到,则生和他的娘,就笑着迎了出来,我付过轿钱,一踱
进门槛,他娘就问我说: 


“早晨拿出去的那枝手杖呢?” 

我被她一问,方才想起,便只笑着摇摇头对她
\newpage

慢声的说: 


“那一枝手杖么——做了我的祭礼了。” 

“做了你的祭礼?什么祭礼?”则生惊疑似地
问我。 

“我们在狮子峰下,拜过天地,我已经和你妹妹结成了兄妹了。那一枝手杖,大约是忘记在那块大
岩石的旁边的。” 

正在这个时候,先下轿而上楼去换了衣服下来的他的妹妹,也嬉笑着,走到了我们的旁边。则生听
了我的话后,就也笑着对他的妹妹说: 

“莲,你们真好!我们倒还没有拜堂,而你和老郁,却已经在狮子峰拜过天地了,并且还把我的一枝手杖忘掉,作了你们的祭礼。娘!你说这事情应怎
么罚罚他们?” 

经他这一说,说得大家都笑了起来,我也情愿
\newpage
自己认罚,就认定后日耎房,算作是我一个人的东道

这一晚翁家请了媒人,及四五个近族的人来吃酒,我和新郎官,在下面奉陪。做媒人的那位中老乡绅,身体虽则并不十分肥胖,但相貌态度,却也是很富裕的样子。我和他两人干杯,竟干满了十八九杯。因酒有点微醉,而日里的路,也走得很多,所以这一
晚睡得比前一晚还要沉熟。 

九月十二的那一天结婚正日,大家整整忙了一天。婚礼虽系新旧合参的仪式,但因两家都不喜欢铺张,所以百事也还比较简单。午后五时,新娘轿到,行过礼后,那位好好先生的媒人硬要拖我出来,代表来宾,说几句话。我推辞不得,就先把我和则生在日本念书时候的交情说了一说,末了我就想起了则生同我说的迟桂花的好处,因而就抄了他的一段话来恭祝
他们: 

“则生前天对我说,桂花开得愈迟愈好,因为开得迟,所以经得日子久。现在两位的结婚,比较起平常的结婚年龄来,似乎是觉得大一点了,但结婚结
\newpage
得迟,日子也一定经得久。明年迟桂花开的时候,我一定还要上翁家山来。我预先在这儿计算,大约明年来的时候,在这两株迟桂花的中间,总已经有一株早桂花发出来了。我们大家且等着,等到明年这个时候
,再一同来吃他们的早桂的喜酒。” 

说完之后,大家就坐拢来吃喜酒。猜猜拳,闹闹房,一直闹到了半夜,各人方才散去。当这一日的中间,我时时刻刻在注意着偷看则生的妹妹的脸色,可是则生所说而我也曾看到过的那一种悲寂的表情,在这一日当中却终日没有在她的脸上流露过一丝痕迹。这一日,她笑的时候,真是乐得难耐似的完全是很自然的样子。因了她的这一种心情的反射的结果,我当然可以不必说,就是则生和他的母亲,在这一日里
,也似乎是愉快到了极点。 

因为两家都喜欢简单成事的缘故,所以三朝回郎等繁缛的礼节,都在十三那一天白天行完了,晚上耎房,总算是我的东道。则生虽则很希望我在他家里多住几日,可以和他及他的妹妹谈谈笑笑,但我一则因为还有一篇稿子没有做成,想另外上一个更僻静点
\newpage
的地方去做文章,二则我觉得我这一次吃喜酒的目的也已经达到了,所以在耎房的翌日,就离开翁家山去
乘早上的特别快车赶回上海。 

送我到车站的,是翁则生和他的妹妹两个人。等开车的信号钟将打,而火车的机关头上在吐白烟的时候,我又从车窗里伸出了两手,一只捏着了则生,一只捏着了他的妹妹,很重很重的捏了一回。汽笛鸣后,火车微动了,他们兄妹俩又随车前走了许多步,
我也俯出了头,叫他们说: 

“则生!莲!再见,再见!但愿得我们都是迟
桂花!” 

火车开出了老远老远,月台上送客的人都回去了,我还看见他们兄妹俩直立在东面月台篷外的太阳
光里,在向我挥手。 


一九三二年十月在杭州写 

读者注意!这部小说中的人物事迹,当然都是
\newpage
拟的,请大家不要误会。——作者附注

\end{document}
