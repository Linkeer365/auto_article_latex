\documentclass{article}
\usepackage[utf8]{inputenc}
\usepackage{ctex}

\title{记忆麦子离走2005\footnote{Click to View:\url{https://web.archive.org/web/20221025140709/https://rentry.co/47es2}}}
\author{孟琳}
\date{2008-10}

% \setCJKmainfont[BoldFont = Noto Sans CJK SC]{Noto Serif CJK SC}
% \setCJKsansfont{Noto Sans CJK SC}
% \setCJKfamilyfont{zhsong}{Noto Serif CJK SC}
% \setCJKfamilyfont{zhhei}{Noto Sans CJK SC}
% \setlength\parindent{0pt}

\begin{document}
\CJKfamily{zhkai}

\maketitle


\Large

麦子终于还是走了。在她想离开后就再也没有出现。蔷薇花开了又谢,谢了又开,只有两个轮回,就已蹿得很高了。冬日的蔷薇是安静的,什么也不说,什么也不做,只望着对面高大的梧桐没什么表情。于是,我便什么也不想,关于麦子的离走或是已发生的更多。可是,现在春天快要来了,我不知道蔷薇会何时开花何时吐芽,会不会还是那般灿烂欢喜,甚至于寂寞了一冬的植物还会不会像从前那般生机勃勃。未知的安静会让人突然记起。恍然间我发觉麦子已经走了两年了。不是离开,是消失。玩笑般的辞别竟是不解的不辞而别和香然无音。可我是不会伤感的,生活就像是一栋房子,总得有人进进出出可麦子是我的朋友,生死与共的朋友。麦子是有信仰的。我喜欢有信仰的人。即使她只是指着某块石头或某片影子说,那是信仰。我信任有信仰的人,所以安静的麦子是
\newpage
我的朋友。我真的记不清我们相识是多大,几岁或十几岁,也许已老得有几百岁了也说不定。我总是疏于理清年龄的问题。几岁和几百岁间的虚虚实实总是被记忆磨砺得布满掌纹而凌乱模糊。就像我辨别不清恍然和恍惚之间的区别一样,同是俯仰间顿悟或回神所以,我这样的人才会和麦子成为朋友。我讲不出麦子确切的模样,不知那记忆是在蔷薇的开谢中逝去了,还是我竟终究从未在乎过她的声音和模样。麦子的灵魂是躲在牦牛的头盖骨里的,所以很多人都觉得她是个怪异而神秘的孩子但我不是这样认为的,信仰总会让人变得很简单。就像麦子信仰藏文化而我信仰摇滚一样。虽然我很少听摇滚乐,麦子也不懂藏文。可是我们信仰。信仰是真实而有力量的。麦子的脸又是很真实的,就是能够看到温度感受到皮肤和五官存在的那种存在即真实。麦子和我真的是朋友。我们似乎干了很多应该在那个儿岁或几百岁的时候干的有趣的事,比如在某棵蔷薇枝头剪下铅笔长的一段枝,培育生根,然后栽下。比如救一只中毒的大黑鼠,喂它青草。泥土的芳香和青草的鲜绿在雨后的清新中永远地留在了那段渐渐远行的记忆中,不干枯也不忧伤。我想麦子也一样。麦子拍了拍掌心的泥,抹掉面颊上的污
\newpage
痕,说她希望蔷薇花能够永远欣欣向荣。平静欢喜而带着涩涩的伤感。可是又哪里有什么永远呢,那些说好不离开的人终都以各种方式离开了。缓缓地,麦子对我说,生命与其他东西不同的地方就是同样长短的枝条与铅笔之间的不同。掌心的泥把掌纹勾勒得更加清晰,好像时光的蜿蜓。可我却并没有告诉她我害怕大灰鼠,尤其是中了毒奄奄一息的大灰鼠。那个生命总是让我敬畏。可终究还是没说。在那个连敬畏都不懂得如何表达的年龄里。恍惚间,麦子说你怎么不敢看它的眼,然后从自己的脖子上摘下那块牦牛骨,放在掌心递给我,眉头舒展地告诉我有了它就什么都不用怕。当我终于在很多年后告诉她我只是害怕那生命临行前的眼神时,长大后的麦子只低着头,微笑不语。那天天空很干净,我们坐在简陋校园东南角的双杠上,黑色绳子悬着的牦牛骨在麦子胸前荡,安静异常。如同当日我们栽在我窗前的蔷薇,安静,怒放。麦子总是说她喜欢蔷薇。因为枝头有很多花,不孤独也不张扬。我说我也喜欢,可我不知道为什么,也许喜欢就是喜欢,不为什么麦子是个很适合做朋友的人,安静而有温度。我们很安静地喝茶,麦子总是坚持用冷水泡茶,一日一日慢慢啜着。不着急,也不用等待
\newpage
什么。可麦子喝茶时也并不快乐,她想着她自己的事情麦子住在-一条很老很深的弄堂里,潮湿的木屋,很高的门槛。她妈妈给她留下很多怪异图案的牦牛骨后就不见了。那时她在襁褓中,所以麦子说她一生注定与那些图腾分不开。麦子和李婆婆相依为命,她说
如果哪天李婆婆不在了她就离开。 

也是天气很好的一天,我们坐在学校里放风筝。这次我看不出麦子在风筝上悬的藏饰是什么图案,好像是太阳花或是太阳还是很老很老的人头像,留着长长的胡须后来风筝线与别人的纠缠在一起,在不知多远的高空。记不清那风筝主人的面容了。记忆总是把一切棱角抹除。那男人和麦子争执起来。讶异。麦子从不会和别人争,安静地坚持就是她性格里除了信仰的全部刹那间,麦子变得很剧烈。一如嘴角残留下的血迹,在风中硬生生地变干拉起麦子的袖口,我说我们去买只新风筝吧,那只旧的已经太老了。我说你不要打了,你打不过他的。麦子居然咆哮起来,从未有过的可记忆只允许我记得那天风咆哮的声音,不住地在耳边回响。麦子的确又终再也记不清,虽然只在花开花谢两年中。后来的事便更加淡忘了。原本也没
\newpage
想要记起些什么,只是看到了蔷薇,看到了春天,又想到了那个在我们同时期盼蔷薇花开时突然转身离去的人,和她那生命与其他枝叶与铅笔的论断。不断想起,不断流淌。那些让我觉得自己坐在窗边孤独得宛如埋在蔷薇花边的牦牛骨终于再记起。那日我与麦子并未离去,在操场边一遍一遍地寻找,在夕阳沙砾中找寻,也在墙边长草的土坡上找寻。那是我记忆中最有色彩的片段。我们低着头,我看到的只是那个背影,执拗而坚决。夕阳是个很美丽的东西,经历了一日,因而慈祥温存。那时我从未想过这个背影会以如此这般的方式离去,一如往昔的安静和执著。天黑了,
我们一同离去,以为会一直扶持、生死与共 

至于风筝上的牦牛骨找没找到,我真的不清楚。只是当麦子平静地对我说“我们走吧!”我们就真的走了。麦子究竟是什么样的人呢?是两年的时间让我淡忘,还是相处时并未要记起呢?……记忆再次纠缠。后来麦子和我竟又一次演绎了惨烈。时间和空间可以由记忆任意改动。记忆让我不断想起又不断混淆很多东西。所以忆起的,又是发生在小学校园的事。麦子和他们谈起了信仰,这是所有争端的原由。瘦弱
\newpage
的麦子是如此拼尽全力,而又如此不堪一击。她如此失望。这难道和她的离开有什么关系吗?未可知晓麦子说有信仰的人是栽植了的蔷薇枝,日后会在生命中
生根发芽开花繁茂欢欣向荣,灿烂,存在 


可她没说没有信仰的人是什么,从未说起 

可我终也想不起麦子的更多信息。就像STEFANIE说的,总有一段旋律经常回响在耳边,可却不知道音乐的名字。记忆就是这么一回事,也许只是生命里的一段旋律,可它却不断浮起。我为什么叫她麦子呢?我为什么混乱了她的一切甚至她的性别呢…太长时间的忘却让我的世界一直云淡风轻而又混乱至如此地步。讨厌记忆的捉弄,混乱而不明晰模糊把清晰搞乱,现实的清晰再把模糊衬托得更加模糊。这就是记忆。麦子。忆起。麦子说没有信仰的人是未栽种的断枝,还带着刺。不论是什么植物,要么干枯,要么腐烂。没有定处也没有力量。麦子那日穿着干净的格子衬衫,不愤怒也没有任何不屑一顾。倚在河栏边,平静地微笑。“没有力量的存在即是腐烂。那就不是存在。只有存在才真实,不存在也就无所谓真不
\newpage
真实。总要有不一样的人存在,但这里不是我想要的世界。”长大后的麦子依然倔强。麦子,真实,千枯,存在。麦子。麦子拉着我,她手里攘着那些被她视为信仰的图腾,奔跑。像小时候一样她用力挥着锹,留给我一个倔强的背影。草汁的味道,铁锹与泥土碰撞的声音麦子埋下了她的牦牛骨,她所有的信仰。牦牛骨。好像我们当初埋大灰鼠一样,也沾染了些许草汁的味道,如此真实而当我独自坐在窗口看蔷薇时,竟不再确定那牦牛骨是否埋下在风中遥想,竟觉可怕。这是我的感觉,也是麦子想她妈妈时说的一句话。熟悉的语言,陌生了的时空。不知麦子现在何方麦子拍了拍掌心的泥,抹掉面颊的痕。“你知道我的信仰,所以我终将归来。我像守望那片金色的麦田一样守候麦子的归来。起风。有些冷。风声。麦子像麦子一样,安静,饱满,真实,金黄。有风拂过时才有浪花,否则就一如既往地平静而低垂。自顾自地守望着。………麦子挽了挽衣袖,说她很喜欢麦田。她弯着腰
,专注地寻找她从旧风筝上丢失的那个信仰。 

“旧风筝如果没丢,可以拿到麦田……”风声

\newpage
把一切淹没。 

而麦子终究还是走了。留下落寞的蔷薇孤单地开了两次又败了两次。留我按部就班地生活,信仰摇
滚却不再听任何摇滚乐。 

锹边的泥土同蔷薇一同安静地哭泣。隐约中,
孤单的烂叶在泥土的叹息中沉默。 


腐烂的绳,黑色,粗糙,裂痛。 


骨未烂,却被蚂蚁啃食得乱落。 


麦子的信仰呢? 


可能离走就是向背离的方向走去。 


可能乱落就是在凌乱中错落有致却无人能辨 


麦子说,音乐不只有摇滚才真实而有力量。 


\newpage

麦子说,她要去寻找她的信仰她的藏文化 


麦子说,她是不是背叛了牦牛骨。 


麦子说,她要远离母亲留下的悲。 


麦子说,她要到另一个世界 


麦子说,她一直在坚持 


麦子说…… 

她说她爱蔷薇花,爱风筝,她说安静是因为信
仰的不平静。 


麦子说她像守望麦田那样守望信仰。 


… 


麦子什么都没说。 

终只是不辞而别。我却在一片乱落中,在蔷薇
\newpage

和藏饰的安然叹息中,听到听到平静的声音。 


或许记忆也不是那般无趣。 


闭眼,落寞或遗忘。 


抬头,等待并守望 

当时间不断在我们身上打下印记的时候,有人选择了离走,有人选择在原地,妥协或随时间逝去。
 


可终将离开。 


麦子选择了离走,我选择在原地。 



…… 


轮回。 

\newpage

满世界的空空荡荡再想起。“有信仰的人或是朋友,是不用强调性别的。”记忆像空气,充满你的世界,可你永远都只摸得到却抓不着。存在即真实。感受不到存在,便终将无法相信记忆中的美好惨烈究竟是不是真的了。轻轻放下笔,看身边所有的来去。我的世界人来人往。生活还是一栋房子,总得有人进出出。

\end{document}
