\documentclass{article}
\usepackage[utf8]{inputenc}
\usepackage{ctex}

\title{高中时代}
\author{dhew}
\date{2000-01}

% \setCJKmainfont[BoldFont = Noto Sans CJK SC]{Noto Serif CJK SC}
% \setCJKsansfont{Noto Sans CJK SC}
% \setCJKfamilyfont{zhsong}{Noto Serif CJK SC}
% \setCJKfamilyfont{zhhei}{Noto Sans CJK SC}
% \setlength\parindent{0pt}

\begin{document}
\CJKfamily{zhkai}

\maketitle


\Large

我无法仅以时间衡量成长的历程。因为它们是一个时代紧随着另一个时代。以至于成长本身,变成了不断的跳跃。而在每一次跳跃之间,是漫长的停顿。停顿中,我无比茫然,不知道会是什么把自己狠狠推向前。而我所知道的是,高中时代结束了。我在等待另一个时代的来临。 
——题记 

那个周五,我托词第二天学校补课而没回家,和吕跑出去玩电脑。那地方大概算郊区,我们坐了很久的车,到的时候天已经黑了。还要在黑暗里面钻来钻去的找那个电脑房。 
老板是吕的朋友。金庸群侠传刚出来的那段日子里,吕在他的机房里连泡了五个通宵,就跟老板混熟了。熟到老板出去的时候让他收钱。后来,那间电脑房被查封,老板和机器一同消失。吕还惋惜了很久。我不知道老板是怎么辗转到这种地方来的,也不知道吕是怎么找到这里的。我们和老板聊了几句,但也没说什么有意义的话,一群人就在电脑前面坐了下来。 
我玩了很久。但也只是

\newpage 

第二天中午就动身回家。吕据说又玩了一整个通宵。然后在星期天下午回学校睡觉。 
我一直记得那个夜晚。虽然已经忘了是玩什么,却一直记得那个在黑暗里若隐若现的小房子,那台靠墙的电脑和老板点烟时眼镜被笼在烟雾里面的样子。甚至连一个人坐车回去时那种无聊都记得清清楚楚。诸如此类的琐碎细节挥之不去。现在想想,或许因为那是我高中时代的最后一个通宵。 

看多了漫画和动画,就不免觉得高中时代平淡无奇,没有和漂亮的女孩在街上偶遇,发现“对方于我是百分之百的女孩”,并终于擦肩而过的奇妙经历。也没有一觉醒来发现自己因为什么莫名其妙的理由而肩负拯救地球的命运和力量(尽管出于私人原因我很欢迎这样的事情发生,因为可以籍此把学校完全毁灭,然后再决定是否拯救地球)。 
不可否认,会期待这样的事情发生,是因为受到日式文化的侵蚀。我和我的朋友们几乎是不可避免的对诸如此类的幻想以及美伦美焕的二维人物产生好感,并因而忽略了身旁的少女们正以令人惊讶的速度迅速蜕变成长。直到很久之后,发现自己曾经是他人暗恋的对象时,才悔之晚矣。女生并不喜欢曾经暗恋过的人回到身边纠缠不休。这是我们的不幸。而等我们终于听到那句著名的otaku台词时,我们才惶然醒悟。原来自始

\newpage 

至终,自己不过是被虚幻的存在所诱惑罢了——“那毕竟是二维的存在。” 
记得是在一年前,一个朋友来信跟我说最近又看了什么漫画,并且用漫画少年不需要爱情这样冠冕堂皇的理由为自己的独身辩护。在进大学之后我便已相信现实生活和漫画分别位于一条路的两头。于是不得不沿着曾经走过的路拼命往回跑,以免陷入梦想主义的牢笼。而朋友的那封来信让我相信自己取得了一些进展。但我没有试着去说服他跟我走同样的道路,因为时间会代劳。青涩的半熟少年们总会变成为可以毫不犹豫吞下苦涩现实的成人。而那个家伙也确然在一年之后放弃了诸如此类的东西。仍然独身,但已将注意力转移到考研上,每天16小时的奋战不止。 
我想,这或许可算成长。 

高中时一个人在长沙上学,学校在河西,一边是湘江,一边是岳麓山。高三时,偶尔也会三五个人跑去爬山,气喘吁吁的爬到顶,再一路狂奔下来,踢的无数石头跟着我们一路滚下来。 
那时单纯的很。爱情像神话,被人递烟会觉得受了侮辱,桌球一直没学会怎么打。打马赛克的录像也要到进大学才看到。课外的娱乐除了爬山,就是玩电脑了。 
初中时学过一点basic,对这种能够回应指令的神奇机器极感兴趣。之后这兴趣便慢慢传移到游戏上。高二暑假家里买了电脑,还没来得及玩便开学了。十分

\newpage 

不爽的惦记着家里的那台,一边关注着身边的种种。后来听说学校附近开了电脑房,便拉了朋友去找。在路上碰到了去玩的学生。游戏生涯便由此开始。 
那天就订了周末的通宵。但上机时却碰上停电。同去的人和老板打了一个通宵的牌。我则占了床大睡特睡。直到早上才被推醒。神清气爽的回学校。朋友们则整个早自习一直打哈欠,并终于在第一堂课上睡倒。现在想想,那就是我的第一个通宵。没有任何值得称道的地方,只是普通的,让人无可奈何的事情而已。又或者生活本就是由类似的事情组成的。 

玩的人多,电脑房就慢慢遍地都是了。那时游戏虽不多,但也不少,又大多耐玩。人也有耐心,玩三国志可以认认真真从江夏打到全国。不像现在一开机就看能不能用fpe。我着了迷,每周都要通宵那么一两次。吕和我原本不是很熟,但经常结伴出去通宵,一来二去的,就成了朋友。 
吕家里很有钱,也是我们那群人中最早买电脑的。但仍然夜夜的跟我们出去通宵,嗯,应该说是,夜夜的带领我们出去通宵。当时不理解为什么,现在才隐约明白到,玩游戏这种事情,是要人陪的。这种心情与游戏无关,只是自然而然的不想一个人。 
高中之后,便不再有通宵的心情。每次想想,都不免惋惜,并且明白到,游戏什么时候都有,但是朋友不是。那种全心投入的

\newpage 

时候也已经过去了。 

我不知道友情的定义是什么。甚至不能肯定我和吕那种结伴出去玩电脑的关系,到底算是友情,还是在坠落时会抓住什么东西那样。但如果要用共处时间长短这样的标准来看,我们应该算是很好的朋友了。好到我可以把自己写的小说拿给他看,而在那之前,因为自知下笔如鸡行蟹爬,是从来不敢给人看的。 
而关系最好的时候,还是在玩电脑的时候。 
记得第一次玩魔兽争霸时,两个人对付一个电脑,结果兵都没出电脑就杀进来了。先被攻击的在那里大呼小叫说不行了不行了。另一个则在一米外喊就来就来,好不容易出两个兵赶去救时,对方已然被灭,而电脑的大军,正潮水般扑过来。打完那局,两个人对着屏幕发了半天愣。 
后来我们发狠打了一夜,终于练到可以独自对付一个电脑的水准。再练三个月,我们已然无敌。在那片学生如海,玩家如潮的地方只求一败,也算自豪。 
常去的电脑房藏在一家面店的地下室里。我们总是驾轻就熟的穿过只有苍蝇停靠的桌椅,直奔柜台后通向下层的楼梯。一个中午奋战下来,随便吃点什么东西就往学校跑,但大多数时候什么都不吃。因为午饭钱已经丢在地下室了。 
记得在那里我和吕一起吃了一次牛肉饺子。刚刚挑过魔兽争霸,他赢了,所以是我请客。在等饺子端上来的

\newpage 

时间里,我们聊了些热衷的话题,但我已记不得那时说些什么了。只记得饺子不好吃,肉馅有点发酸。 
想到那个中午,以及关于它的种种记忆,我总觉得有点无可奈何。好像自己抛了西瓜,捡了芝麻。以至于那段可以拿来验证友情的时光变成了关于一碗变质饺子的无聊回忆。但事情就是这样子。美好回忆和琐屑碎片一同填充了过往的时光。 
总对自己说,如果有机会的话,还要一起去那间饺子店吃东西。打打魔兽。可大概是没有机会了。就算我们还能聚在一起,魔兽也应该被淘汰了。 

有一次,班主任不知怎么心血来潮的关心起寄宿生生活,在下晚自习后溜到了寝室来。一群人正在激烈讨论中午C&C的战况来着,见到她进来,齐刷刷的闭了嘴。当时我就在想,完了,吕完了。 
班主任在吕的床上坐到熄灯也没见他回来。我们开始还支支吾吾说他在上厕所什么的,后来就都闭了嘴。班主任揪住我们一个个问吕去了什么地方。大家都说不知道。要是那天晚上找不到吕,也不会有什么问题,事后随便编点什么也死无对证。但班主任后来又带了她男朋友过来,那个显然是在大学里学计算机的家伙看到吕的床上有本《大众软件》,二话没说就拉了她出去,据说他们在那天晚上把附近的电脑房都翻了个遍。然后找到了他。 
第二天,我们起来时,吕

\newpage 

已经在床上躺着。中午回去,床上没人了。只剩下一团被子堆在床角。大家情绪低落。结伴出去吃饭,并徒然的猜测吕去了什么地方。可终于没有得出什么结论。而吕也始终没有出现。 
那天中午开始下雨。下午第二节课开始时,吕从教室后面进来。坐在他的靠窗的位子上。头发湿漉漉的,浅绿色的外套因为浸水而变成了墨绿色。他就那样坐在座位上,仿佛沉默的化身般,低头看着桌上的课本,偶尔抬起头茫然的看着黑板。 
后来,吕说他从家里一直走到学校。他在路上走了几个小时,虽然在下雨,而他一点感觉都没有。 
我默不作声,不知道该笑,还是该怎么的。 
后来,我们无一例外的被老师抓去教训。但也只是教训而已。吕在年级内通报批评。并且因为没有遵守寄宿生纪律而被警告处分。 
那之后,有整整一个月没人提起去玩电脑这回事。甚至是用一种沉痛的心情回顾着过去的点点滴滴。或许有那么一点惋惜,甚至有幡然醒悟的可能。可没有电脑玩的时间,确然无聊。生命像拉长了无数倍的下坠,人变成了耗尽的电池。从内到外逐渐干涸。我说不清那是怎么回事,但当时,的确是这样觉得来着。 
最后,不记得是谁牵头,我们又开始通宵。但之前的那种肆无忌惮已经消失,课业已经重了起来,老师也看的更紧。但这不是原因。只是有些东西不对劲了。 
现在想

\newpage 

想,其实在那时就已经隐隐看到了碰撞地面的一瞬间。未来通过这样的方式在我们面前展现了一个可能性。而之后,我们中的绝大多数人,也确然将这个可能变成了现实。没错,我们意志力薄弱。或者说,我们根本没想过要坚强。我们从未试着拒绝那些漫画,小说,游戏等等的非现实,因为没有理由。我们看不到自己的生活方式和未来有什么抵触。或者看到了,却根本不愿承认。就这样,我们选择了自己的未来。如果知道这个选择有多重的话,我们或许会重新考虑,但现在说这种话,就已经晚了。 

进大学后的很长一段时间里都没有和吕联络。因为觉得对不起那些曾经一同奋战的好友,内疚作崇,也确然不知道该用什么样的方式和他们说话。想起曾经那么频繁的结伴去玩电脑,总不免黯然。但生性不善交际,朋友们也是类似的人。大概是耽于梦境而不习世故的关系吧。大学期间曾经努力想要改正这个缺点来着。但高中时代的烙印过于深重,以至到现在仍不算是开朗的人。 
上网之后,也曾和几个老友取得过联络。开始很热闹的相互写信,以至于每天打开信箱都会有那么一两封来自过去的信件。但慢慢就觉得无话可说。仿佛维系我们的,只有高中时代一起玩电脑的经历。而那些经历,谈过几次之后,就变得苍白浅薄,以至于根本无法再谈。很久以后,才明

\newpage 

白我们应该聊点别的什么,例如学习,生活之类的。但当时却不敢触及这些话题。仿佛一旦如此,过去的什么就破碎了。就这样莫名其妙的,一群人又没了联络。 
吕其实是很好的人,高瘦如竹竿,仿佛风吹即折。说话到兴奋的时候脸上总是忍不住笑。眼睛也会眯起来。眼镜片在阳光下总是返着绿色的光。过生日的时候也请很多人出去吃饭。近来得到消息。说他在新西兰当牧场主。我不知道那个消息来源是不是在开玩笑。虽然我和他都很喜欢模拟农场。但我无法想象他要是真的去放羊了会是什么样子。 
记得闲谈时,曾经提到过要办游戏公司。当时说他负责程序,我负责剧情。两个人踌躇满志。但事后的种种证明我们并不是那种能将梦想和现实挂钩的人。那需要极强的决心,意志,和运气,而我们三者皆缺。那时老狼在慢慢的唱着“我的蓝色理想现在哪里?我曾幻想的未来又在哪里升起?”我想那便是我们的写照。 

很久之后,曾经和母亲说起那个时候。她当然毫无认同感,只说起有一次去长沙看我,到学校的时候,恰是午休,她看见我和几个男生一起往学校外面跑。她追在我后面跑,但还是追丢了。她不知道我是去什么地方。后来问班主任,才知道那个方向有很多电脑房。 
母亲把这当作一件有趣的事情在说。或许所有那些令人痛心的事情,回过头来看时,都是

\newpage 

有趣的。而我却不知该说什么,内里痛心不已。高中时代的种种美妙之处在那一瞬间统统折损。好像自始至终,那一切不过是阳光下的肥皂泡,当阳光消失时,就统统变成了灰色的。 
只是在那时,我才发现自己一直都是从一个方向考量生活中的种种,而这是一个多向度的世界。存在着彼此交错的不同方向,以及对同一件事物的不同观感。这个发现带来如此的巨大的意义,以至于我终于把对朋友们的愧疚抛到了脑后。 
或许这是成长的必然经历之一。然而在那之后,我却觉得有些什么东西失去了。仿佛一扇大门在身后关上,另一扇在眼前打开。或许会得到新的东西。但或许不会。而失去的,却是实实在在不可挽回的。 

此刻,我又想到了那个中午。当吕还在玩时,我在公共汽车上,抓着吊环,晃晃悠悠的,几乎就这样睡去。通宵后的疲倦让我觉得自己变成了什么空心玩偶。一会儿清醒,一会儿昏睡。仿佛被人抓在手上摇来摇去。那一次我站着睡着了。幸好在要下的站前就已醒来,下车,跑回独住的屋子。有那么一会还自觉清醒。但碰到枕头的一瞬就已睡着。醒来时,已经是黄昏。车辆的喧嚣从窗户钻进来。夕阳的光芒爬上了眼前的墙。 
接了家里来的一个电话。之后,便看着那光芒一点点的往上爬,一点点的溶入黑暗中

\newpage 

。莫名其妙的恐慌,仿佛正从高处坠落,却发不出声音,只看着整个世界向身后滑去。等待着碰撞地面的那一刻,却永远等不到。匆忙的打开灯,煮了点东西吃,才觉得好受点。那一夜剩下的时间,一直趴在床上看武侠小说。累了,就压着书睡着。再醒来的时候,天已经亮了。 
在那之后的几周里,我没有再玩过电脑。6月末,便上了去深圳的火车。再后来,就进了北方的一所大学。而朋友们却大多落马。以至于每每想起,都不免愧疚。仿佛自己是踩着他们爬上来的一样。之后再没回过长沙。只是在南下北上的旅途中,在站台上稍作停留。并因为呼吸到了熟悉的空气而激动不已。但是,也仅此而已了。 

很久之后,我在一篇小说中看到了这样的一句话:“我们怀有的理想迟早将这样烟消云散,犹如那原以为永远持续下去的无聊思春期在人生途中的某一点突然杳无踪影。”我想这描绘出了我的状况,那一切都已经消失了。而我还在寻找着它们的影子。 

我无法评价自己的高中时代,无法评价那种生活。我尝试着给它赋予某种意义,但无论怎么做都像是自欺欺人。我也不能用通用的价值观来衡量那段时光。因为如果这样的话,结论必然是整个高中时代失败透顶。而这是我决无法接受的。因为一旦如此,

\newpage 

那些狂热,那些投入,那些兴奋与激动都变得毫无意义。 
所以,我把这权力交诸于每个看到这篇文章的人。但无论你得到怎样的结论。不要告诉我。因为我想得到的,并不是结论。写这篇文章的目的,只是缅怀。 

正如此刻,寒意透窗而入,我坐在床前敲电脑,因为久未活动而膝盖发僵,却在想着那个夏天。我们嬉笑着从那条洒着波浪般阳光的林荫道上走过,谈论着那阵子刚刚推出的游戏。阳光懒散而灼热。我们为一些并不有趣的东西大笑,并相信生活永远是这样,轻松自由,永远有那么多抛洒不尽的笑。 
这样的回想让人心酸,因为知道那样的时间不会再回来了。因为知道生活大多数时候不过是平淡无奇的事务性操作。那些特立独行的日子则只是漫长道路上的小小一段罢了。 
将目光从屏幕上移开。窗外夜色深沉,就像藏着什么秘密般无声无息,让人捉摸不透。站起身,才能看见远处的灯光。一团一团的。仿佛就是被夜晚深深的掖在怀里的无数秘密。我突然感到了某种属于过去时代的毒素又慢慢的爬进了灵魂深处,不由得很想听听那首蓝色理想。可是翻了半天,却没有找到那张cd。几乎让我以为,它和那个时代一同消失不见。一同离我而去。我不得不深深的吸两口气,让自己不至于鼻子发酸。并且明白到,那段时光是

\newpage 

令我感动的。那消失已久的,不再回来的一切。那些被淘汰的游戏,那些如星散的老友。 
那是我的高时代。

\end{document}
