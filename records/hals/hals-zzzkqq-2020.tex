\documentclass{article}
\usepackage[utf8]{inputenc}
\usepackage{ctex}

\title{hals}
\author{zzzkqq}
\date{2020-03}

% \setCJKmainfont[BoldFont = Noto Sans CJK SC]{Noto Serif CJK SC}
% \setCJKsansfont{Noto Sans CJK SC}
% \setCJKfamilyfont{zhsong}{Noto Serif CJK SC}
% \setCJKfamilyfont{zhhei}{Noto Sans CJK SC}
% \setlength\parindent{0pt}

\begin{document}
\CJKfamily{zhkai}

\maketitle


\Large

*龙门少年杀人事件

*全文1.8w,是权重不均的废话子集

*本时间线设定全系捏造

有的人抓住一只蝉,以为抓住了整个夏天;就像有的课代表,抓住一个抄作业的同学,以为抓住了得宠的机会,死活也不肯撒手,非得闹到最后厮打一顿才算完。点动成线,线动成面,纷扰的话语交织起来,变成晚自习的骚乱。世界上的安静千篇一律,而这里的骚乱却各不相同:有人敲桌子砸板凳,有人跑调却浑然不知继续唱,有人明目张胆地放屁,有人比赛一口吞掉面包,有人围着他们下注,有人因此被噎着,有人学狗叫(一定是佩洛族同学),有人骂脏话,有人催促别人交作业,有人嫌他们太吵故而不得不用更大的嗓门盖过去,还有人不属于上面的任何一项,这个人就是槐琥。她合上生物书,引起了周围局部地区的短暂安静。人们安静的原因是她的动作具有警示性,书本的盖合标志着旧状态的结束和新状态的

\newpage 

开始。该状态可以是她扬起一把龙门币撒向空中,引发众人哄抢;也可以是她一记七武掠阵踢,把大伙都放倒在地上。鉴于她并不是近卫局那位家财万贯的大小姐,只是一个普通龙门市民兼学生,所以后一种状态的可能性还要更大一些。但她什么也没做,只是往上推了推她的平光镜片,让世界在眼中不会一分为二,然后像风离开沙丘一样离开教室。

她在天台找到了孑,他当时正点燃一支廉价香烟,随便一个路边小卖部都能买到的牌子,而且一旦只买一包,店主就会或多或少地投来嫌弃的目光。孑最开始没有注意到她,或许注意到了,但是没空理她。他把烟填进嘴里叼住,侧身背过风,用手掌护住擦亮的焰苗,凑到烟丝上耐心地点燃……之所以它造价低廉,是因为里面含有很多的杂质,燃点高,并不能碰一碰就着,而是需要片刻的等待。作为一个鱼贩,他最擅长的就是等待……等到火星像腹蛇快速吞吐的信子,在烟头上一眨一眨地翕动时,他就用两指夹住,从两片干涩的唇间取下它,抬起头来,这时就看见了槐琥。她已经离得他很近了,如果没有萦绕的烟味,她还能更近一些。孑呼出一口白气——只是在冷风中迅速液化的小水滴。他坚守着行业的卫生准则,而香烟则会影响他的味觉,附着到食材则会更麻烦,所以他从不往肺里吸。嘴唇之于香烟更多的是固定作

\newpage 

用,毕竟他长不出第三只手,其实他也考虑过这个问题,如果真的需要,那么第三只手应该装在哪里的事宜,最终的答案是介于右肋和腋下的那部分区域,这样他每一次挥刀,就可以多斩断一条鲜鱼……

他抬起白天忙碌了十几个小时的疲惫的眼皮,无精打采地瞥了槐琥一眼,作为招待顾客的最基本礼仪。如果她在清晨四点刚出摊时见到他,情况则将大为不同——他会一边擦手,一边向她介绍哪些是最新鲜的食材,哪些则因为隔夜所以打折。当然,说这些话时,他还是一双死鱼眼、垮着身子,像提着两大袋水产,声音也提不起顿挫,因为他本来就是这样。他很珍惜自己的身上的能量,节能是因为他本来就没有多少精力;就像他省钱,省钱是因为没钱。

他说:鱼丸给你放那儿了。手还夹着烟,指示一个方向,烟像钓线一样直着上升,让槐琥想到小学时健康绘本上的插画,死神正抓着钓竿,钓那些吸烟者的命。孑转身将烟放在天台的护栏上,让死神钓了个空饵。这个举动有明显的逃避嫌疑,很像是他被抓了吸烟的现行,所以才把烟放下以示毫无瓜葛。但槐琥见过他很多次这样做了,每次他来学校送餐,都要照例点一支放在天台。这烟是给死人抽的。他的一个老客户就是从这个天台上跳下去的。那个学生槐琥也见过,是个沉默内向的黎博利,跳楼那天,他耳朵

\newpage 

后面渗着血,族群引以为豪的美丽羽毛被人薅个精光。他的死是因为考试作弊挨了老师几句批评,顶嘴时无法忍受所以寻了短见;也是因为他在平时倍受霸凌,被班主任因为父亲有源石病而百般侮辱,所以活不下去了——这要看了解信息的渠道是校内的统一通报,还是自己的亲眼所见。如果他能说出来就好了,他说出来,槐琥一定会去把那些收保护费的人教训一顿,可惜他凡事都喜欢憋在心里,总从自己身上找原因,所以最后死掉了。

“你的是最上面那一层。”孑又提醒了一句。他在里面多加了几颗鱼丸,基本上那一盒不赚钱。槐琥揭开盖嗅了嗅,满意地又把它合上,想说些什么,天台的门突然被撞开,阿慌慌张张地跑上来,身上带着一股浓烈的榴莲味,让槐琥很容易联想到了他刚刚做过什么,同时心中猜测这次追上来的人会不会破上一次的记录。

“槐琥姐,你帮我拦着点,我先翘了,”阿把薄薄一层背包挎上,头也不回地说:“鱼丸你帮我捎回所里吧。”

他越上栏杆,一拽背包前面的绳头,身后便弹开一张巨大的滑翔伞。他在起飞前转头看了一眼孑,说:“哎,你也在啊。”然后蹬出天台,闪入林立的楼房间消失不见,只把狂妄的怪笑声留在了这里。

\newpage 

孑想:我见过他么?铁门慢悠悠地转着门枢,忽然又被一拳顶开,几个怒气勃勃的男子提着扫帚拖把挤在门框间,像一罐沙丁鱼罐头。为首的一个质问:那个菲林小混账呢?跳楼了?

“跳了,”孑敷衍他:“老板,要不要买点鱼丸,限时八五折,多买多送。”

“你是?啊,倒听人说过,一人一刀闯码头的分舵主……”来人的气焰很明显消了很多,尤其是看见了旁边还站着个学校里出了名的功夫菲林。

“别传谣了,老板,”孑叹口气说:“我真的是个鱼贩。买鱼么?”

“买?买个屁,大晚上的,不怕胆固醇病啊?”他揉着臀部,那里裤子破了一个大洞,应该是阿研究出了什么加强肠动力的药物。他对槐琥说:你告诉那小混账,别让我逮住了……槐琥朝他摆摆手,表示她知道他要说什么,阿要让他逮住没好果子吃,所以不必说下去了。那群怒汉一时也编不出什么新的威胁,只好铩羽而归,任由那个作恶多端的坏学生搭着滑翔伞跑掉了。孑一语不发地盯着被阿踩灭的烟头,他逃跑时当然注意不到这个小细节,一脚把它碾到了地上。槐琥提起蒸出水汽的塑料袋,说:我替他道个歉吧。

“不用,点烟只是个形式,跟烧纸钱一样,

\newpage 

都知道他是无意的,”孑拧一拧自己的鼻梁,皱着眉,好像直面着眩目的强光,说:“他也不是存心让人当成坏人。”

“那就明天见——说不定晚上就能再见。”槐琥拉开天台的门,走进楼梯。她知道孑也在说他自己。

她第一次遇见孑是在高一的夏天,孑刚刚转学过来,坐在班里靠墙的位置无所事事,像一封无人认领的信,一座公墓里的雕塑。如果有人对他笑,他就回一个更好的。他的脸因为长期日晒与种族特性,皴出黑一片白一片的瘢痕,而且不谙笑容的要领,所以看起来很像罪犯,笑起来就更可怕,俨然一个变态连环凶手。在他的笑眼中,每个人看自己的倒影都像盘刺身。他没有校服,自己的上衣也不体面,是件花点钱就能从劳保市场批发来的白T恤,上面印着生鲜大卖场,特别的掉价。这是因为母语的麻痹性,如果用维多利亚文印花,改成“Fresh?food?supermarket?”,虽然意思一样,格调可明显上去了。但如果真印上去,龙门的大爷大妈就认不得了,不仅如此,还要说他是个假洋鬼子,不买他的鱼,这样一来就得不偿失了。除此之外,他的牛仔裤磨得破绽层出,手指贴满创可贴。他穿球鞋,他戴铁项链,基本上就是这样。于是有几个同学讥诮他

\newpage 

土气、邋遢。世界上确实从来不会缺乏这种人……无论什么时间、什么地点,看见矮子就骂,看见高个就损,新生入学时,他们看见槐琥留着短发,一拳砸烂墙板,追得收保护费的高年级生满操场跑,故而在背地里说她是个二尾子,中性、粗鲁。孑倒是觉得,一个人长得中性,是否粗鲁,并不取决于她的发型、身手或是嗓音,而是唯一取决于她是否有一颗粗鲁的心……但他当时还不认识槐琥。围着嘲笑他的人呢,他也不想认识,正如他不想认识冰箱里馊掉的海胆。

那群人的意思很明确:如果孑为此发怒,他们便说:“他急了”;如果孑忍气吞声,他们便说:“他怂了”。这一招屡试不爽,从未失手,所以当他回话时,他们根本就没有备选方案。孑说:“是啊……欢迎光顾,同学的份上,打个九折。”这个反应很明显不属于两种预期中的任何一个,所以他们愣在原地,面面相觑,谁也不知道怎么办了……不过他们后来还是很得意地吹嘘:那个转校生被我们吓到啦——孑来上学的时间越来越少,有时甚至一连整个星期都见不着面,好像真的受了这些人的惊吓,但真相是他的养父董阿伯摔伤了,没人去看摊,只能他顶着,否则家里要揭不开锅了。接着考了一场联考,他的成绩是惨不忍睹的一片C,只有物理和历史是B-,好像一群侏儒里最高的两个。这是因为孑平时除了做鱼丸

\newpage 

,还经常帮着邻里捣鼓捣鼓收音机,对电路熟得很,手上的几处划痕就是焊锡烫的;至于历史,是因为历史老师刚订了婚,还处于甜蜜时期,对每个学生都倍加关照,随便填填空就能拿个C……他被叫去办公室谈心,被问及为什么他每天都要算账,然而最后三道大题却错得没边时,他答道:因为龙门币用不着圆锥曲线与导数换算……令在场的众人瞠目结舌。因为这种情况下的标准答案只有两个,其一是低头盯着鞋尖不说话,被老师骂到滚蛋;其二是痛哭流涕地悔过,以示自己发奋图强的决心。但他非得来这么一句,好像错的不是自己而是龙门币一样,显得问话的人很蠢,让诸位老师都不好拿捏了……

考完试后,他去学校的次数更是愈发减少,最后索性办了休学。众所周知,在知识可贵的背景下,除了源石病的瘟星和家徒四壁的穷人,没人会办理休学,就算是办了也不会受理,砸锅卖铁也要上,因为泰拉世界乱着哪,没文凭则无处立足,只能去黑心医药公司被压榨,或者当一辈子修车工、一辈子服务员。班里捕风捉影的包打听说他不是第一次转学了,这大概是他最后一次,他本来就是想看看自己是不是读书的这块料。槐琥想,他应该是,没几个人生下来不是读书的料,但他能分到学业上的时间太少了,正如花盆里的土质再好,也长不出盘虬巨木。又谈到他

\newpage 

他辗转颠沛的原由,因为他总是不受待见,说来也是,长了一张凶神恶煞的脸。原来那所学校里的学生玩丢书包的游戏,等到孑出手时,被砸的人莫名就飞出好几米远,拉开拉链才知道,原来孑的书包里全都是砖头。真相大白后,他就被一路追到了旗杆上,愣是直到天黑也没下来。这种说法多少有杜撰的意味,而且更像是阿的作风——这时大家都记起来,哦,这就是阿前几天刚干的事,还是槐琥出面,才把他从旗杆上拎了回去,不然第二天升旗仪式会是个麻烦。包打听脑子里的事情很多、脉络很杂,所以搞混两个人的事迹也是在所难免的……但这时又有了一个新问题,那就是孑的过去究竟是怎么样的一种故事。暂时还没人在意这种问题,因为大家都开始骂起了阿,没空理这个若即若离的转校生了。

槐琥挑了一天去看望他。说是看望,不如说是不期而遇。因为虽然槐琥古道热肠,但还远远没到没话找话嘘寒问暖的地步。那天孑没出摊,憋住一口气扛着两个花圈从巷子里钻出来。花圈一大一小,受力不均,当大的那个快滑落时,就被槐琥从后面托住。孑没看见她的脸,但闻到了她身上的气息,于是便没有回头,耸耸肩膀把花圈的支架扶正,对身后的槐琥道:别沾手,不干净。

“不碍事,”她说,“论心不论迹。”

\newpage 



孑没再回话,算是认可她的观点。跟她一路拐进下城区歪七扭八的居民楼街道。槐琥在后面单手扶着,像公园里身手去够飘飞的气球或者树叶的孩子。一个穿着朴素、打补丁的札拉克中年男人坐在楼道大门前的马扎上,摇晃着脑袋,拿抹布擦着脖子。他看见孑像开屏的孔雀一样背着花圈走来后,撑着膝盖站起,双手伸过去搀他。然后他看见了走在孑身后的槐琥,脸色一下子窘迫起来,慌慌张张的,好像手是从别人包里偷来的一样,不知该插进兜里还是露在人前。他支支吾吾地问:“阿孑,我这身给你丢人,这、这是,女朋友?”

“同学。”孑将花圈从肩上卸下来,放在楼梯口。这时男人才稍稍松懈了神色,如鲠在喉:同学来了,不来家里坐坐,那、那怎么……

“我带她去我家坐坐,大叔,先忙好自己的吧。”

槐琥从善如流地跟着他离开了拥挤的巷道,下城区尤其是贫民区的布局大多如此,层叠、逼仄,墙壁上喷满了粗鄙的涂鸦,杂物在墙根堆砌得像是菜市场收摊时的动物内脏,只能看见一条线段,每年都有人因为被剥夺了视野而被路口突然冲出的泥头车撞死,大家愤怒地谴责司机、谴责杂物、谴责造物主,却没有一个人想到设置信号灯或者限速警示标志。正

\newpage 

午的阳光把路面刮得明一片暗一片,像孑的皮肤。许多女孩子因为一块青斑便要死要活,孑的脸皮杂成这样还敢抛头露面,令她们完全无法理解。槐琥说:无意冒犯,但我还以为……出事的是你家。

“啊。那大概率是不可能的。我们笃信好死不如赖活着。”孑因为扛了一会儿重物,走起路来有些驼背,配合他恹恹的目光,像个垂头丧气的丧尸。他说,那个扎拉克男人是这家的亲戚,这家三口人都没了。他们生前还是董阿伯摊上的常客,爱吃肠粉,总要加过量的葱花。后来慢慢不来了,因为夫妻双双下岗,没钱再买这些口体之奉,晚上还得去菜市场捡别人不要的菜叶。就在昨天,这对夫妻拿最后一点积蓄买了一块猪肉、一颗白菜,拌了满满一盆馅子。等父亲去学校接完女儿回来,一家三口就一起吃下了包着毒鼠强的水饺。近卫局派来的法医说,已经出现了尸僵,由此推得死亡时间是十二到二十四小时。孑对这些学术性的东西不上心,只知道自己从此少了一个邻居。穷邻居间搭建纯真的友谊比富邻居间要难得多,但是断裂却很容易。所以他中途回了一次身,从兜里掏出皱巴巴的烟盒,夹出一根点燃,放在最下面一级的楼梯上。烟比纸钱容易买到,在贫民区,烟是一种流通的货币,学抽烟是大多孩子长大的必经之道,就像萨尔贡人的割礼。槐琥拿出便携的单词本,在旁

\newpage 

边等他。岩烧店里飘出煎蛋的香气,绑紧头带的店主端出一盘嗞着热气的牛扒,由老板娘转呈给窗边的客人。她脸上敷了一层粉,穿着仿十二单的东国衣裳,隆重而累赘,只在每个月的开业纪念日如此打扮。经过落地窗边时,她的视线在孑身上停了几秒,似乎也在等他回头,交接一下熟人间的善意。但他没有。槐琥想到了那个无厘头的科学实验,证实了目光具有能量,但看来孑并不具备接收它的同频……他在被淋湿之前总是觉不出下雨,和董阿伯大相径庭。往往是乌云还没聚起来,董阿伯的膝盖就已经疼得他龇牙咧嘴,不得不找块毯子盖上了。

因为董阿伯的关节问题,家的选址就要颇费心思。不能在低洼的地区,湿气侵扰;住的楼层也不能太高,否则爬楼就要把腿累断。孑选择出摊的位置也遵循此道,得考虑城管的巡逻路线、客流量的多少以及商圈的租子,所以定在了龙门中央公园。他的愿望是把铺子开到日落大道去,赚更多钱,正如董阿伯想在哥伦比亚焕发职业第二春,这两件事都没有实现。董阿伯是因为得不到鼠王的首肯,而孑是因为不具备赚大钱的铁石心肠。通往家门的街道两旁杂乱地倒着一排自行车,受潮的硬纸板发出酸味,堆积的汽车轮胎则是橡胶老化的臭味,一切使得大地不像大地,而是意象的集合。槐琥就这样穿过气味,像穿越十一

\newpage 

月的雨林,到了孑和董阿伯赖以栖身的家中。

孑给她接了一碗凉好的豆浆,碗是老式花瓷碗,和囍被、机械表或缝纫机一样,有一种古早的灰影。虽然豁了一个小口,但清洗得很干净。孑说,豆浆比咖啡有营养。这样便掩过了家中没有咖啡机的事实。但他紧跟着说了一声:何况家里也没有咖啡机……这样又显得他先前的行为多此一举,只收获了一份不必要的诚实。槐琥听了之后,啜了一口醇厚的豆浆(比路边摊多了半倍的豆量),道:德在人先,利居人后,挺好的。孑撕开比戒指还多的、箍满十指的创可贴,其中一些已经被油料熏染成老年乳晕的颜色,然后贴上新的。槐琥看见了他年轮般盘旋的伤口,想起了父亲的拳茧。一个练家子总要被迫在身上留下些证明,如果没有成果,便只是伤痕;一旦出人头地,就是勋章。她看着墙上的老照片,照片里董阿伯胸前挂着一朵大红花,坐在师傅们中间,问:你爸爸呢?

“生父在监狱。……养父么,应该在澡堂子。”他坐进二手沙发里,皮套已经不再能发出崭新时的咯吱声,像是被岁月磨平了棱角的人。他伸手挪过茶几玻璃桌面上的印制电路板,翻来覆去地看:“要不是今天我得去给街坊帮忙,现在应该被他叫去搓澡了。”

\newpage 



那么传闻是真的了,槐琥心想。关于孑,她听过形形色色的传闻,一如他手上各式各样的割伤。仿佛孑的本体被淡化,变成一道填空题,什么答案都可以往里填一样。董阿伯据说早年为某个大人物挡过一发流弹,这辈子都直不起腿走路,每天都得去泡三个小时热水澡加桑拿,才能让腿的血脉活络。要不是孑,摊子都开不下去了。至于孑的生父,传闻中是这样的:他本来是个公交司机,因为弯腰去捡硌到踏板之下的水瓶没有看路,造成了伤亡,进了监狱。她没有仔细追问——孑是精于市井之道的小贩,钱给多了就一定要找零,她担心孑会反过来问她,她怕她会忍不住跑开。

[那?一天看到什么了/记起的是什么呢/你又梦见他了么/还在找寻着吗/那个丢失了它的人/在纸上写下?回答我吧?回答我吧/一遍一遍重复说着的话……]

“唉?我还没调试呢,”孑微微睁大眼眶,盯着眼前传出失真歌声的收音机喇叭,比看见它活了还惊讶:“今天走运,可以去买张彩票……”

“……多谢款待。我想我该走了。”槐琥匆忙从父亲的影子里抽身,放下喝了一半豆浆的碗——孑给她盛的量太足了,一时半会无法完全喝完。他问:这就走了?

\newpage 



“嗯,明天还有考试,先回去复习了。”她说:“豆浆很好喝,下次就从买鱼丸的钱里加吧。”

“这是送的。”他似乎有些难为情地搔一搔后脑勺。

槐琥是朋友了。他后来这样向她解释:男生之间,至少是在他这里,一起受罚、一起吃饭或者一起看黄色漫画都不算朋友,但是如果请人去家里玩,就是认可友谊的表现。虽然那次是为了不让札拉克男人尴尬,才不得已请她来家里做客,有种迫使的意味,但也不能否认这个事实。越卑微的人越对底线有着偏执。董阿伯也清楚孑的这个规矩,他对孑说:时代真是变了,在我们那个年代,男人的友谊是四大桩:一起同过窗,一起扛过枪,一起嫖过娼,以及一起分过赃……不用想也知道,以他的说话思维,他一定还会教授孑其他的事,比如甄别烟草的优劣,陈酒的真假,还有男人右手的第二种用法,尽管这些对于孑来说都是无师自通。槐琥听了孑的话,暗想,这人真是幼稚得可以。但其实事情还比她想得更复杂,因为她告辞后,孑便开始了自渎。每周的这时候,龙门广播的一个频道都会播放戒色演讲,他要用这种方式表达对演讲者们的蔑视。那天他不用出摊,不必在意手部清洁卫生,看起来像是天赐良机。他当然可以去游戏厅、舞池或是酒吧一类的地方消遣,但是那需要大量

\newpage 

的余钱;他还可以去泡书店、泡公立图书馆,他平时也喜欢这样,但他那天就是没做。在他的意淫中,槐琥始终没有出现,甚至因为她的原因,菲林这个种族都推迟到浪潮过后才浮现出踪影,只能归类于某种心理上的爱屋及乌吧……

还有另外一种光景:孑推着餐车在前面跑,近卫局的城管在后面撵他。汤汤水水漾得涌出锅炉,洒在铁皮桌面与地上,久而久之便在街道上沤出两道车辙一样的轨迹。当然,如果被逮到,那他就不是孑,近卫局也就不是近卫局了。早在他还能上学的时候,槐琥就领略过他的逃跑技术。因为他要等董阿伯到来后亲自完成鳞鱼丸的小摊的交接,而董阿伯有一条腿瘸了,赶起路来像一辆方轮胎的自行车,要耽误些许时间,导致孑总是迟到。当然,孑可以向他提出早一点来的要求,但他闭口不谈。即使是槐琥,也是很多次才勘破的这个秘密,它像是有传染性一样,令她也对此默不作声,仿佛和孑形成了一种默契。很难想象这两个人会有默契,在门卫眼中,槐琥的形象简洁、干净、锐利而鲜明,像一根消过毒的针;而孑邋遢得像一个露了馅的菜包子,每天踩着点狂奔而来,身上一股海鲜的腥味,大衣的袖口总沾着一圈铁黑色的油污,露出鞋口的袜子一只黄一只灰。因此她极端反感这个转学来的乌萨斯,同他展开了旷日持久的追逐

\newpage 

战。那个门卫自己也是个乌萨斯,也是转关系过来的,而且职位只是门卫。换言之,有她没她,大家都一样的上学、授课。但她觉不出来,因为她就是这个可有可无的人本身。她当时年近半百——这个年龄的雌性看谁都不会太顺眼,只能说是孑倒霉。她和孑的班主任商量好,不要给孑留教室门。班主任嘴上答应,转过头就把承诺忘掉,这是因为孑是班级全勤的最后一块拼图,而班级全勤与其年终奖息息相关……她喊孑停下,把名字登记了再走,以便向纪律委告状。但是孑不理她,因为铃声还没响起,她这是故意使绊。他在生鲜大卖场待久了,知道年终奖的重要性——只要它在,班主任便睁一只眼闭一只眼,要是它丢了,班主任也不会帮自己了……

在躲门卫之前,孑已经和近卫局城管斗智斗勇多年,而且离开学校后,他还要继续斗下去,故而练就一身绝技,进校门时七成功力都用不上。槐琥当时和他还不熟,只把那些追逐战当作消遣来看。她和孑真正熟络起来还是在他休学之后,就像一些作者死了,作品才在世上流传,不说生不逢时,也是阴差阳错。围观的人越来越多,门卫堵孑成了校园一景,很多人宁愿冒着迟到的风险也要准时收看每日清晨固定上演的一幕,导致很多班主任的年终奖都没了。孑在飞速跑动中定睛打量着门卫的动作与眼神,一如他削

\newpage 

鱼片时仔细观察鱼肉的纹理。如果她上半身前倾,作搂抱状,孑就稍微下蹲,像滑冰运动员过弯一样从她的侧肋绕过去;如果她扎个马步,作抵挡状,孑就提速助跑,蓄力撑住她的背,从她头顶蹦过去。他像个库兰塔,像个黎博利,就是不像乌萨斯。又因为他挑选的角度太刁钻,步伐太古怪,所以大家都觉得他是个坏蛋。在龙门,你要么是一个好人,但是乏善可陈;要么是一个有趣的坏蛋,每天面对形形色色的、比门卫还讨厌百倍的人,像个擦边球一样与他们周旋……

对付近卫局,孑已经驾轻就熟。关键的操作在于摆摊地到居民区的一段路程。只要在这里逮不住他,便再也不能逮住他。因为孑和这里的人互相认识。一旦城管孤军深入,很快会引发各类灵异事件。孑跑过去之前,路面还算平坦整洁,他跑过去之后,路面上立刻会出现各种地刺、图钉、黄油与强力胶,伴随着倾盆的自来水与纸飞机,完全看不清路况。那又是一段下坡路,城管们常常人仰马翻,这样对付一个小贩,就太得不偿失了。

至于群殴事件的来龙去脉则是这样的:孑早有预感,总有一天得帮着槐琥解围,就像他预感总有一天会因为阿的零食住进医院一样。所以他提前备好了家伙,和佐料、抹布、蛇皮袋与零钱盒子一起,放

\newpage 

在手推车的柜子里,以备不时之需。当看见槐琥领着一群人奔跑得沸反盈天时,就顺手取了出来。那是一截被换下来的锈水管,他用纱布缠住弯头,免得把人动脉打裂。孑本来还想让战线拉长一些,不要妨碍周边做生意,但是把人引过去,街坊邻居就不安宁了。“而且呢,”他说,“不能让家里上新闻。”所以就地动起了手。那是一帮初来乍到龙门的叙拉古人,当街打架略微有些束手束脚,不敢坏了当地人的规矩,毕竟强龙不压地头蛇嘛。而且正如那位炎国大人物所言,要在战斗中学习,这也是促进对龙门了解的好时机……很快他们发现,龙门人喜欢看热闹,不一会就围上来一堆人,像从地底下长出来的似的,欢呼、喝彩、叫好,这是第一条知识。第二条知识是,龙门人爱拉偏架。明明是两边都推开了,但槐琥和孑一点伤都没有,自己这边反倒是平白无故挨了好几脚。槐琥的腿法也不是吹的,快把骨头踢断了……

第一拨围上去的人都倒下了,在余下的战斗中,他们将作为路障继续存在。正是加班结束回家的时间,路灯黯淡,围堵的人却越来越多,孑抽出空来,扯下他备用的旗帜,上面印着:鳞鱼丸,无人售货;请自觉,一盒十元。这时他感觉臀部一疼,匆忙回身,及时躲过了第二发飞刀。偷袭者是个操纵着浮游刃的沃尔珀,他看见孑阴沉的脸,以为是疼痛所致,

\newpage 

得意扬扬地摇晃着手中的枪械,炫耀道:“这个多向弹丸发射器,是今年雷姆必拓黑市上的最新产品,价值十一万龙门币。”——随即感到一阵强风向他撞来,鼓起的青筋与肌肉,因愤怒而骤缩的虹膜——孑欺身而上,一记勾拳硬生生拍在他的脸上。沃尔珀人只听见一声沉闷的蜂鸣,好像层层叠叠的碗碟一并在他头颅中打碎,鼻腔里涌出酸辣而带着苦味的液体,翻个白眼,直挺挺仰在地上,不省人事。

“这只一击必杀超度神拳,是我老爸十八年前的产品。”孑迈过他在地上摊开的手臂,掏出被浮游刃捅穿的钱夹,揉着屁股:“谁发明的浮游刃,我要请他吃泔水。”

他抖擞几下钱夹,抽出一沓小面额的钞票,无一例外地全都破了洞。如果没有它们,可能坐骨神经都要烂掉。本来今天收摊后,他就可以拿攒的钱去买一双新的雪地靴——原来那个穿小了。现在不得不跑一趟银行把坏钞替换。孑大致点了一遍,惊喜地发现实际数额比自己预想的还要高。那么换完之后,还可以再买些猕猴桃,补充维生素;家里的鸡蛋也快吃完了,还有豆腐……

孑想入非非之时,槐琥一招过肩摔,巧妙地将一个大汉摔到了他的身边。她喊:“别发呆了,火烧眉毛啦!”孑一阵激灵,从经济蓝图的构建中抽身

\newpage 

,继续挥舞水管,打翻了几个人,嘴里小声嘀咕:“你浑身不都是毛……”很快,第二拨黑帮也都仆倒在地,哀叫连连,似乎槐琥才是施暴的一方。这一带的市民没少见过槐琥行侠仗义,对于这个结果不算意外,倒是孑,他出手是第一次,做生意要以和为贵嘛。槐琥问:你伤哪了,我看看。孑活动着膀子,听见这句话,有点脸红,说:还是别看了吧。

“我们的增援来了!”一个倚坐在墙根的黑帮有气无力地扯着嗓子喊了一声。槐琥和孑不约而同地往路口看去,那里已经聚起了一众机车暴走族,打着明晃晃的远光灯,引擎与排气管不遗余力地制造着噪音。通常来讲,孑喜欢机车,但不喜欢有人开着它们来跟自己打架。他说:你还能打么?

“忙复习来着,今天没吃晚饭,有点力不从心了。”

“怎么又不吃晚饭。”孑颇为头疼地叮嘱她,“人是铁饭是钢。”

“不是正想来你这里吃点,谁知道路上碰见这些人。”

“呃,”孑又望了一眼路口,那些人已经开始汹汹地发起了冲锋,“你也不看看他们的势力。”

“哎呀,”槐琥终于不好意思地挠了挠头:“闹得痛快就没考虑那么多。”

\newpage 



“坐上来吧。”孑收起餐车上巨大的遮阳伞,用力折叠,从车头位置给槐琥留个空。她开始还有些不解,试探着坐上去,问:“这里?”

“坐稳了。”孑双手搭住扶杆:“我们甩开他们。”

他一瘸一拐地推动鳞鱼丸小摊,围观的群众给他让开一条路,有小孩子高呼:“姐姐,你那几招红眉咏春真漂亮。”槐琥眼睛亮起来,向声音的发源地招手:“挺识货嘛。”

孑推着车长驱直入居民区,车轮越滚越快,他一个箭步,蹬上踏板,如雨燕般在缅长的下坡路上滑翔。很快,黑帮们也闯进了街道。这时每一个楼巷间都溜出来一排地刺,前路洒满了图钉,让所有轮胎都报废。同时天上纷纷扬扬飘飞着旧报纸,鸡蛋从窗户后面随着骂声掷出:他妈的,大晚上这么吵,不睡觉了啊?!暴走族们弃车代步,但不是被强力胶粘得寸步难行,就是在黄油上哧一声滑倒,像圆木一样滚下去……槐琥听见身后鸡飞狗跳的嘈杂,抬头望一眼明晰可见的银河,情不自禁地笑了出来,而且一笑就不可收拾。孑还在分心注意着身后是否还有追兵,渐渐被她的笑声吸引。槐琥盘腿坐在餐车的最前端,抓着两根竖立的围栏,迎风欢呼,让他联想到桅杆。仿佛她是海上最传奇的海盗,而孑的餐车就是她的航船

\newpage 

。槐琥伸出佩刀一指,命令他纵身漩涡中与她一同起帆张扬。孑也因为她的模样而变得愉快,他问:你笑什么?但槐琥没有回答他,也许是她也给不出答案。

孑把餐车停在了公园隐蔽的一角,给槐琥做了一份简单的快餐,坐在作为垣墙的断台边上看着她吃。下面是碎石铺成的浅滩,不远处是人工湖,自动投料机传出噪声,往水里播撒鱼食,然后长久地安静。

“好久没吃过那么好吃的东西了。”槐琥长舒一口气,双腿交叠,有节奏地用脚跟敲打墙面。孑略微侧身,避免受伤的那一边屁股接触硬物。他说:是因为你饿的。

“是你手艺好。”

“但愿吧。”孑托着腮,一条腿像变色龙的尾巴一样耷下去。

“你还回去上学吗?”

“我恐怕不行。”

“那如果我帮你向教委申请助学金呢?”槐琥放下餐盒。

“我倾向于拒绝,”孑拧一拧眉头,“别管我了。我又不像你,门门功课除了A就没见过别的。”

“但是你不觉得,只有完成学业才能拥有一

\newpage 

个正常的未来吗?”她认真地说。孑半睁着眼睛,有些困倦地斜睨她一眼,不知如何打发:“这个么……倒是有个老先生来找过我,宣传他的新式办学。”

“新式办学?是前不久也来学校里演讲的那个?”

“呃,你去听了?”

“没有,我在复习生物。”

“……他说,因为现在论文普遍造假,学分掺水严重,枪手层出不穷,所以他们入学没有任何门槛……他们会请最好的师资教人,教最务实的学问、技能,因材施教,不考试,不评级,如果觉得学不到东西随时可以走人……”说着说着,孑觉得自己也变成了一个推销员。

“一定有什么代价。”槐琥听完,一语道破其中秘密。孑说:“他们不会颁发任何学位证书。”

“我现在知道为什么去听的同学都说是在浪费时间了……”

“别再考虑我了,”孑摆摆手,“看在我今天替你解围的份上,给个面子,让这事翻篇吧。”

他弄不清怎么说出口的是这句话。明明他准备好的素材是,家里有十五年的房贷要还,董阿伯一年的疗养费也不是小数,自己又没有什么进取之心,考上了大学也是二流三流,出去了也是受制于人的宿

\newpage 

命。但他只是说,算了。算了。槐琥并非死脑筋,看他意不在此,于是罢休。她忽然觉得孑这种颓丧的样子也很可爱,怀着一种实验课上用电流刺激青蛙神经的心情,她对他说:

“我们私奔吧。”

她以为孑会从墙上摔下去,怎么想都是这样,所以她已经做好了拉他一手的准备。但孑像一尊石雕一样纹丝不动,似乎将她的话置若罔闻。恍惚间又回到初见的夏天,清风一样淡的少年走过,那个坐在墙边一派事不关己的男孩。她不知道的是,孑的冷汗已经浸湿了背心,似乎伤口的疼痛都不再跳动。他花了几个瞬间,头一次认真地考虑过未来——鳞鱼丸小摊的归属、董阿伯的照顾、不能断供的资金……他张口却发不出声音,仿佛声带冻结。短暂地,他又察觉到这样一种潜在现实,槐琥只是戏弄他罢了。他看着槐琥,她眼神的每一次流转都像是要道出这个谜底。为了不让她亲口说出来,孑选择了先发制人。他说:“欠妥当吧。”

“你在担心什么?”槐琥毕竟是侦探社来的,她也洞察到孑的恐慌。

“你是能正儿八经说出私奔这种话题的人吗?”

“试玉要烧三日满。”槐琥站起来,拉上校

\newpage 

服的拉链。她偷偷掀起餐盒的底,将一张纸钞压在下面,“好了,不跟你辩论了……我可得回去了,万一老鲤报失踪了就不好了。”

“免费的。”孑抽出那张钞票,弹一下对她说。

“给你的医药费啊,真是的……”

孑有些无语的望着消失在树影之间的槐琥,扯一扯领口,让焦虑的热气从脖颈处蒸腾而出。伤口又开始沙沙作痛,他在原地坐到后半夜,悄悄下定了决心,拖着餐车踽踽独行,最后走到龙门近卫局的门口,对盖着报纸打瞌睡的夜班员说:“我要自首。”

——————————————————

三个警员坐在他对面,两个黎博利,一个阿纳缇。阿纳缇警员无精打采,满面没睡醒的样子,一个劲给自己灌速溶咖啡。孑咽了咽口水,问:能给我一杯吗,您光顾我的摊子时我也会送您一份赠品的。阿纳缇盯着他看了一会儿,抽出一只一次性纸杯,走到热水机前给他冲了一杯。倒不是为了他那点赠品,单纯看他面相凶恶的同时态度挺好。平时审犯人还得给一根烟套话,现在给自首者一杯咖啡不算过分。左边的黎博利抹掉眵目糊,扫视写字板上的笔述,问:伤哪了?

“屁股。左边的。”孑如实回答。

\newpage 



“取个证。”

他从椅子上起身,斜着脱下裤子,血液凝固,伤口已经结痂。房间里的吊灯白得刺眼,黎博利警员给他拍了一张照片,说:“提上吧。”

另一个黎博利问:“你用什么打的?”孑说:“管子。在书包里。”

警员拎起他两边背带长短不一的背包,眼神询问了一下,然后拉开,倒出几本包着书皮的书,反倒是水管最后一个才筛出来。他拾起那几本书,浏览书名:《调酒师入门》、《观星图鉴》、《无线电技术手册》、《Art?of?Meatball?肉丸的艺术》,其中还夹着几张超市打折券。他有些纳闷,调侃道:怎么看的书那么杂?

孑说:看着玩的。从这一点上讲,他已经犯了抗拒从严的错。这些没有一本是看着玩的。他做肉丸、帮邻居修收音机,需要时刻将知识谙熟于心。至于前两本,是他在二十四小时书店新买的。之前在高台与槐琥的交谈中,他越发感到了自己的浅陋。槐琥说了一种他闻所未闻的鸡尾酒,一颗他见所未见的恒星。它们分别是干马天尼与弧矢增二十二。他想要补充这方面的知识,以期待可以追上她的步伐,继续与她谈论宇宙和天空——至于人生,它没有什么意义。孑不坦白,是因为他也知道自己的可笑。他花了十八

\newpage 

年尚未学会如何去爱,却妄想在一个晚上学会如何相配。警员没有在这上面太下功夫,而是没收了他的水管,问:他们怎么伤的你?

“浮游刃,警官,他说是雷姆必拓黑市上来的。”

“浮游刃?’小姐’不是刚订购了一批浮游刃?跟无人机一批的那个……”

“小子,你真的不是混黑道的?”黎博利警员仔细打量着他的脸,“你这条件,可惜了啊。”

“谢谢。”孑一时不知该回什么好。他喝光苦涩的咖啡,疲惫却丝毫没有减少,像一座快要老死的桥。警员留了个心眼:万一他真的混道上,只是来踩个点呢?他招呼着孑过去,让他站直,拍张正面相备案。

“别垮着身子,真是,本来就不高,这样一崴就更矮了……行了,走吧。都不知道你为什么来……本来这事都入不了档案,要说罚款呢,又觉得对不住老实人。赶紧回去上学吧,离考试还有几天哪……

犯错就该自首——孑默默地收拾好背包,将书整齐地砌好,退出了办公室。他莫名记起去年生日那天,她送了他一把刀。做生鲜要用到的出刃。刀背宽厚,刀刃被他磨得锃亮而锋利。那是一把很好的刀,就像一个很好的人。路过阴暗的走廊时,孑突然感

\newpage 

到一阵困倦,也许是因为失了血。他躺在墙边一排等候用的椅子上,展开外套当作被褥,沉沉地睡了过去。天行将拂晓时,他呢喃了此次睡眠中唯一的一句梦话。槐琥。抬起舌根,悬在口腔,先是扯动嘴角,似乎有了笑意,接着轻轻呼气,像牧民吹响口哨。两次,两声。他的梦话。槐琥。

———————————————————

我曾经想过当天灾信使,孑说,去泰拉的所有地方。有中意的地方,就在那里定居;如果还没找到就死掉了,那就死掉。说完这些,他又不着痕迹地笑笑,补充一句:但是,没能去成。他换了个地方做生意,避开是非,同新顾客攀谈。他手上没有停下动作,仍调试着旋钮,仔细聆听粗糙的播音。考试在即,他得赶快把收音机做出来,作为送给槐琥的礼物。一个眼角有疤、头顶半秃的鲁珀人探头探脑凑过来,小心翼翼地说:“铃铛?”

“嗯?”孑不知道他什么意思。而一旁本来还和他谈笑风生的客人,看见来者不善,识相地带着饭盒走开了。

“茄子双引号。”鲁珀又说。

“我不懂。”孑反应过来,这是在讲暗语。他说:“我不是黑手党。”

“别说笑了,你这张脸不是,谁是?”他鄙

\newpage 

弃地哼一声,又换一种说法:“酒起子放在家。”

“客人,鱼丸要吗,还剩一些。”孑索性做起生意。

“你真的不是?”鲁珀人又看他一眼,咕哝道:“白瞎一张凶脸……”

“那我直接问你吧,”他挡住孑递来的菜单,从内兜夹出一张照片,展示给孑看:“见过这个人吗?”

孑的职业性笑容凝在脸上,冻结,碎裂。照片上的人是槐琥。他说:没印象。

“我听人讲,她来你这买过粉肠。”

孑还是面无表情:抱歉啊,一天接待的客人太多,记不清有这号人。

“这都能忘?我看她还挺有姿色吧,你看这翘翘的屁股,”鲁珀人不由自主地舔一舔嘴角,下意识还是把孑当成道上的人,“要不是这件校服,啧啧,真想看看她那对奶子……”

孑拿起抹布擦一擦手,问:“客人,你叫什么名字?”

“我?叫我瓦西里。”

孑从餐车后面绕出来,说:“那,瓦西里,我要赏你两耳光。”

鲁珀人猛地抬头,迎上孑突然的一记直拳,

\newpage 

幸而他混黑道多年,身强体壮,没有立仆,而是很快反应过来,和孑扭打在一起。孑很想打断他的鼻梁,把他的牙打在地上,像蜗牛的牙一样多。正当不可开交之时,一个声音喝止了他们的厮打。花白胡须的沃尔珀教父撑着单手肘拐走来,饶有兴味地欣赏孑擦拭嘴角的血迹,说:“我听说过你,一人一刀闯码头的分舵主。龙门黑道的传奇人物,有兴趣聊聊么?”

“没有。”

“瓦西里先生口无遮拦,是他的不对。”教父用指头推着名片,移到他鳞鱼丸餐车的桌面上:“时间,地点,我写在反面了。如果来,一定不会令你失望的。”

他带着手下扬长而去,像溜进门缝的影子。孑脱力般坐到折叠椅上,手腕又酸又涨。他摸过名片,脑海中忽然闪过一个词条。那是他在近卫局里看见的,虽然只有几秒,但以他的记忆力,已经足够了。名字来自一个过时的传奇,一个外强中干的榜样,十年前尚且能在龙门暗面作威作福,但随着魏彦吾的翻云覆雨,这个帮派渐渐失去了立足之地。或许是惨败于槐琥,扯动了他们最后的遮羞布,令他们不计代价地要报复她。百足之虫,死而不僵,但也只是时机未到。

“头儿……那小子会上钩吗?毕竟咱们也不

\newpage 

是当年……”

萨卡兹跟班的话被老沃尔珀的凶狠目光逼回肚子。他最不能忍受的就是听人提及帮派气息奄奄的现状,他没想到自己竟然也有将宝押在别人身上的一天。但他又露出一抹胸有成竹的笑,说:“有件事,你知道么,你刚才在他前面,应该也注意到了,”他说,“北极熊的毛虽然纯白,但它的本来肤色其实是黑的。”

“那万一他,他动了杀心……”

“可能么,先生?他一个卖鱼的匹夫,没有这种骨气。”

“我听说他行事一直古怪,好像能沟通冥界,左兜的烟给活人,右兜的给死人……”

“说得我越来越想会会他了……绅士们,我们到时候,就知道了。”

—————————————————————

好像看到了什么新奇的场面,学生们纷纷驻足,向男厕门口看去,状若窥阴癖集会。孑戴着口罩,怕被人认出似的,和阿一道走进同一隔间,但他那张脸如果不完全遮住,其实是很好认的。这个场面不可置信之处有三:首先,孑出现在了校园之中;其次,阿敢在光天化日之下招摇过市;最后,这两个人走

\newpage 

到一起去了。孑关死厢门后,外面就聊开了,几个学心理侧写的同学嚷嚷:我早就说过,那个菲林成天阴阳怪气的,肯定是同性恋——孑怎么也是啊,我可吃过他的外卖,我不会得梅毒吧?

“唷,孑……你不会是喜欢闻这里新换的熏香吧。”阿举起双臂,像推着一块透明玻璃,做投降的姿势,对外面的议论置若罔闻。在遇到真正喜欢的人之前,他也不知道自己是什么性恋。孑递给他一张造影相片和报告单,问:“你能治吗?”——董阿伯的伤势随着季节的到来不可避免地恶化。阿接过来一看,嘴角咧出一丝讥笑,说:这些医院,要价比我爸年轻时候更离谱了啊,他们干嘛不去抢喔。

“你能要多少?”孑飞快瞥一眼手表。

“哎,别急,铃响了我也不会去上课的,你,也不是为了来念书的吧?”阿拍拍他的肩膀,比想象中的要宽,他萌生一种想法,孑的耐受性应该比一般的实验品要强……可惜现在他是客户,所以还是不要打这个主意。他张开五指,孑问:“五十万?”

“不不,只要十五万,”阿抬高声音,眼里满是兴奋,“你也可以用身体支付一部分喔。”

“先这样吧。”孑说完,铃声便像长鞭一样抽响。走廊一阵东南西北的践踏,他压低声音:“你清楚槐琥惹到的那个黑帮吗?”

\newpage 



“槐琥姐?啊,你放心吧,只要我们侦探所敢惹的,都是稀松平常的人啦。不是软柿子,就是马上要被近卫局取缔的。她现在应该在联系大学咨询事宜吧——”

“你有能增强身体机能的药吗……我是说,还在试验中的,当志愿者就不用付钱吧。”

“你是我见过最好的乌萨斯。”阿的眼睛弯成月牙,从挎包里掏索一阵,取出一瓶淡色液体:“荔枝味的,副作用嘛,大概是发炎嗓子哑,细胞失水吧。而且屁量会增多……你最好多买点水。如果你能活着再见到我——开玩笑的——我就只收十四万。”

“最后一件事,”孑摘下书包,像分赃一样拿出松松垮垮的礼品盒。丝带已经因为挤压而变得扁平。他说,“帮我转交给槐琥。”

“慢慢慢——”阿挡住他的手,按回书包上,“我可不帮这种忙。孑,你得亲手交给她。”

孑离开学校后,坐在路边阻挡机动车的矮石墩上,狠狠地攥着他的礼物盒,几乎要将它捏瘪。如前所述,里面装着他制作的收音机。与市面上热销的那种不同,它的频带更宽,而且印刷版上,还画着他的画。除非把它拆开,否则槐琥永远看不见他的匠心。当天发生的另一件事更为重要,孑迟到了两小时,极大地损害了沃尔珀教父的等待。但他耐着性子等下

\newpage 

去。他知道内部人心涣散,甚至他自己,都暗中订好了潜逃的船票。若不是因为槐琥的仇一直横亘在他的胸口,他早回叙拉古了。不知为什么,在本来应该越发谨慎的晚年,他却一再犯错,为小事大动干戈,先惹了龙门近卫局,又动了商业大鳄的奶酪,哪怕槐琥这种微不足道的后生,都是心头的毒刺。不过再等下去,别说把门的,就连贴身保镖都会不耐烦地一走了之。

“孑先生,是吗?你不太准时。”

“龙门不是哥伦比亚、不是维多利亚,没有准时的习惯。”孑用手绢擦着脖子上的汗,顶回去。他大概也知道了此刻的处境,自己是被需求方,言语也就更加大胆。倒是沃尔珀教父,他十指交叉,彼此压迫得生疼,反复提醒自己不要迈进这该死的水产小贩的语言陷阱。他不想再纠缠下去,他已经老了。

“十万,那个女孩。”他推过去一张写好的支票。

孑略微起身,老人以为他介意条件,不紧不慢地跟一句:“你不做,也有别人会做。”

孑重新板正地坐回去,手插进右兜,惊得四周护卫纷纷掀开外套拔枪——但他只是掏出一瓶饮料,旋开盖子,环视众人喝了一口。偃旗息鼓时,他又伸进左兜,引动新一轮戒备——他夹出一根烟,递过

\newpage 

去,被老人拒绝了。

“我在想,如果你的烟是从右边拿的,会不会是警告。”教父挥挥手,叫那些保镖不要那么风声鹤唳,让别人的无心之举吓到,最终丢脸的是他。孑收回香烟:“你都是从哪知道的这些规矩。”

“这算是承认了么?”

“啊。是啊。”孑注视着老人,不用自己找话题,谈判倒没有想象中那么惊险,但他浑身还是渐渐燥热起来。他说:“我退出黑道好多年了。”

他编了很多版本,比如自己其实是黑道龙头、警局卧底,或者就顺着传闻,扮一回分舵主。斟酌一番后,还是选择了退隐的形象,更不易教人怀疑。教父眼中闪过难以察觉的得意,说:“我比你大了四五十岁,年轻人,你还是欠火候。”

“我喜欢龙门,但不喜欢它的监狱。办任何事需要风险,也需要代价。你是老资格,不用我提醒了吧。”

沃尔珀人眯起眼,仔细打量着这个只身赴会的少年。心想他是不怕死呢,还是穷疯了。自诩摸清了他的底细,但越是单纯的人,越容易把握不住。他戴满戒指的手敲击公园露天的石桌桌面,问:“你是卖鱼的,应该不会出错。”

“你想说什么?”

\newpage 



“你一个月税后能净挣多少钱?一千五顶多了。你是个不值一提的小骗子,比一粒烟灰还小。”

后半句孑甚至没有听见,他的身体中有火海在燃烧,血压一路飙升,几乎撬开他的大脑,耳朵里像几十架飞机一并起飞。他打开瓶盖,与之前只抿一口不同,他这次仰起脖子,全喝光了。对面的人笑着摇摇头:“这样的示威可不够格,怎么也得拿刀插大腿才有威慑力啊。”

“我直说了吧。你知道她为什么在我那里出现过?因为她是我的同学。”孑喘着粗气,说话断断续续的,将支票推回去,“虽然我不上学了,但她算是我处得最好的一个同学。”

“所以呢?”

“得加钱。”孑撑着大腿,嗓音沙哑,也许是做出了巨大的心理斗争。他还是觉得渴,手再次伸进右兜。

“那,你想加多少?”沃尔珀教父从支票簿里撕下一张新的,连笔一起再推给他。出乎意料地,孑没有接过去,而是一把攥住了他的手腕,力度大得几乎可以把骨头捏碎。他诧异地抬起头,看见孑满布血丝的眼睛,和因为缺乏营养而鲜红的牙龈。他拿出的不是饮料,而是一柄明晃晃的刀子:

“加你一命。”

\newpage 



———————————————————————

槐琥盯了很久的那班列车也开动了。她的视线由聚成的一束散成失焦,眼中掠过一张张陌生人的倒影,航行器摇晃着向远方奔赴,载着远走的希望或是孤独。移动城邦之间的交通极依赖于这种机器,其中做得最好的是企鹅物流与罗德岛。刚才哞带着阿来送行,开始还好,越到后面,阿越忐忑不安,一个劲瞄着表,教唆她换乘下一班,怪得槐琥都难免要怀疑,问:“你是不是给什么乘客下药了?”

侦探所的两人走后,她终于可以稍稍放松自己,脊背陷进柔软的靠垫,像瓶子陷入流沙。她这几天为了择校和学习,很少能睡个安稳觉。驾驶员还没有上车,她习惯性望着窗外,邻轨的列车走后,进站空荡的门户向她洞开,吞吐着游客与光影。孑踉踉跄跄地挤开人群跑进来,槐琥看见他,直起身站起来,怀着意料之中的惊讶,走到车门前等他。孑喷了香水,头发梳到同侧,穿一身显小的礼服,纽扣系到最顶一颗。他讪讪地解释:喷了点啫喱水。槐琥瞧见他发梢上的一点暗红,说:甚至还做了挑染。

“啊?挑染?呃,大概吧。”孑回头望了一眼,将收音机交给她:“我应该等不到你生日了,那时候应该是新学期。”

\newpage 



“你自己做的?”槐琥接过来,掂一掂份量。孑又回头看一眼,说:“本来我还准备了礼物盒,但是被弄湿了,就……”

“谢谢你,”槐琥推一推眼镜,笑着说,“不过,昨天晚上怎么没出摊?”

“嗯,我去找那个跟你说过的那个老先生了,就是新式学校的校长,”孑说,“跟他谈了谈对以后的规划,他说我很幼稚,很天真。”

或许是对爱情的见解,或许是对职场的分析,或许是对人心的参详,孑得到了幼稚的批评。真正的世界不是童话,老先生说,还有那个你说的菲林,她就比你成熟——成熟但不够成熟。孑要把这个消息带给槐琥,他是奔跑着来的,反而真到了她面前又胆怯了。偌大的车站井井有条,他却找不到一处地方摆放自己。身后的视线一直跟在自己身上——刚迈进大门,就有近卫局的人凑过来,问:“孑,是吗?星熊警官想见见你。而且我们曾经看见你出入过公园现场……”

孑打断他的话:“警官,我知道了。你们想让我做什么都行,我一定会配合的。但是能不能给我点时间,我来这里送个人。我怕我以后都见不到她了。”

航行器前端的红灯转成黄灯,督促乘客尽快

\newpage 

返回的提示音循环地响起。槐琥指着身后的庞然大物,说:那我回去了。谢谢你能来送我。

孑深吸一口气,终于挺直了身子,原来他也是可以很精神的,只要他想。而他的声音疲惫,一如既往。

“我想过最好的剧情,是一个人拥有足够的自由,去自己喜欢的城市,看喜欢的风景,学向往的知识,成为想要成为的人,”孑双手插着口袋说,“槐琥,这么好的剧本,你可别演砸了。”

“放心吧。”槐琥学着侠客的样子,抱拳拱手,转身离开。

槐琥回到车厢,坐在座位上,隔着窗户看孑。他身边冒出了两个穿深色制服的人,带着墨镜和贝雷帽,像爱护弟弟一样拍着他的肩。槐琥总觉得这身制服在哪见过。她想再确认一遍的时候,孑已经闪出了视野。伴随稍许的耳鸣,航行器从轨道上滑出平台,接近云层。收音机突然传出刺耳的杂音,像初秋将死的蝉鸣。她把声音调小,它却吱一声断了信号。龙门在她的眼中越发渺小,她回忆起自己在这座城市的成长,日光散落于她身上披靠的薄毯。在深深浅浅的回忆里,槐琥放匀呼吸,缓缓闭上眼睛。一阵轻微的颠簸,仿佛大脑皮层的精神波动,航行器驶出龙门边,她正梦见老虎。

\newpage 



\end{document}
