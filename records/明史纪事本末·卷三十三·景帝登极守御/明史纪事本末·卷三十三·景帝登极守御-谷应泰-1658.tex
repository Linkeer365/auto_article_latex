\documentclass{article}
\usepackage[utf8]{inputenc}
\usepackage{ctex}

\title{明史纪事本末·卷三十三·景帝登极守御\footnote{Click to View:\url{https://web.archive.org/web/20230409094236/https://www.zhonghuadiancang.com/lishizhuanji/mingshijishibenmo/18635.html}}}
\author{谷应泰}
\date{1658}

% \setCJKmainfont[BoldFont = Noto Sans CJK SC]{Noto Serif CJK SC}
% \setCJKsansfont{Noto Sans CJK SC}
% \setCJKfamilyfont{zhsong}{Noto Serif CJK SC}
% \setCJKfamilyfont{zhhei}{Noto Sans CJK SC}
% \setlength\parindent{0pt}

\begin{document}
\CJKfamily{zhkai}

\maketitle


\Large

英宗正统十四年秋八月,上北狩,太后召百官入集阙下,谕曰:“皇帝率六军亲征,已命郕王临百官。然庶务久旷,今特敕郕王总其事,群臣其悉启王听令。”辛未,太后诏立皇长子见深为皇太子,时年二岁,命郕王辅之。诏天下曰:“迩者寇贼肆虐,毒害生灵。皇帝惧忧宗社,不遑宁处,躬率六师问罪。师徒不戒,被留王庭。神器不可无主,兹于皇庶子三人,选贤与长,立见深为皇太子,正位东宫。仍命郕王为辅,代总国政,抚安万姓。布告天下,咸使闻知。”癸酉,郕王临午门,言官大臣次第宣读弹劾王振启章,言:“振倾危宗社,请灭族以安人心。若不奉诏,群臣死不敢退。”因哭,声彻中外。王起入,内使将阖门,众随拥入。有令旨籍没振,遣指挥马顺往。众曰:“顺,振党也。宜遣都御史陈镒。”时太监金英传旨,令百官退。众欲捽殴英,英脱身入。马
\newpage
顺从旁叱百官去,给事中王竑愤起捽顺首,曰:“马顺往时助振恶,今日至此,尚不知惧!”众争殴之,或就脱顺靴,捶击躧踏,立毙顺。众又索振党内使毛、王二人,英捽令出,亦击杀之,曳三尸陈东安门,军士犹争击不已。逾时,执振侄锦衣卫指挥王山,反接跪于廷,众唾骂之。于是众竞喧哗,班行杂乱,无复朝仪。百官既殴杀顺,益恟惧不自安。王亦屡起,欲退还宫。兵部侍郎于谦直前揽王衣,曰:“殿下止。振罪首,不籍无以泄众愤。且群臣心为社稷耳,无他。”王从之,降令旨奖谕百官归莅事,马顺罪应死,勿论。众拜谢出。是日,事起仓卒,赖谦镇定。谦排众翊王入,袍袖为裂。既出,吏部尚书王直者,笃老臣,执谦手而叹曰:“朝廷正藉公耳!今日虽百王直,何能为!”丙子,移王座入奉天门左受朝。陈镒奉令旨,籍振并其党彭德清等家。振第宅数处,壮丽拟宸居,器服珍玩,尚方不及,玉盘径尺者十面,珊瑚高者七八尺,金银十馀库,马万馀匹,皆没官。脔山于市,族属无少长皆斩。振暨山弟林等皆从驾,死
于兵。太后命以于谦为兵部尚书。 

二十三日,也先拥上至大同城下,索金币,约
\newpage
赂至即归上。都督郭登闭门不纳。上传旨曰:“朕与登有姻㜕,何外朕若此!”登遣人传奏曰:“臣奉命守城,不敢擅启闭。”随侍校尉袁彬以头触门大呼,于是广宁伯刘安、给事中孙祥、知府霍宣同出见,献蟒龙袍。上以赐伯颜帖木儿及也先弟大通汉英王。上曰:“秋稼未收,军士久饥,可令刈以入城。”又曰:“也先声言归我,情伪难测,且严为备。”从骑叩城下索犒军资,并内官郭敬等金银共万馀两来迎驾。
既献,复不应。 

初,也先来索赂,郭登曰:“此绐我耳!莫若以计伐其谋,劫营夺驾入城,此为上策。”乃谋以壮士七十馀人,饷之食,令奋前执其弓刀,因拥上还。召壮士与之盟,激以忠义,约事成高爵厚禄。士皆奋跃用命,已书券给之。会有沮者,既淹久,寇觉,惊扰而去。时登练兵振武,誓以死守大同。将士咸感奋
,屡出奇挫敌,故以孤城得全。 

也先拥上道宣府,总兵杨洪闭城门不出。事闻,逮洪系诏狱。上出塞,过猫儿庄、九十海子,历苏武庙、李陵碑。二十八日,至黑松林,也先营在焉。
\newpage
上始入也先营,也先拜稽首,侍坐设宴,令妻妾出上寿,歌舞为乐。仍奉上居伯颜帖木儿营,去也先营十馀里,伯颜帖木儿与其妻见上,亦如也先礼。也先屡欲谋害,会夜大雷雨,震死也先所乘马,谋乃沮,且加礼焉。袁彬侍左右,颇知书,性警敏。又有哈铭者,先随使臣吴良羁留在北,至是亦与彬同侍。又有卫
沙狐狸者,亦随上至漠北,供薪水,劳苦备至。 

二十九日,太后遣太监金英传旨:“皇太子幼冲,郕王宜早正大位,以安国家。”时议者以时方多故,人心危疑,思得长君以弭祸乱。于是文武群臣交章劝进,王再辞让。众请遵太后命,允之,遂择日行
礼。 

九月戊寅朔,上在迤北,也先遣使来言,欲送上还京师。使还,以金百两、银二百两、彩币二百匹
赐也先。 

癸未,郕王即皇帝位,遥尊上为太上皇,诏赦
天下,改明年为景泰元年。 

\newpage

也先复遣使致书,辞悖慢。兵部尚书于谦见帝泣言曰:“寇贼不道,势将长驱深入,不可不预为计。迩者各营精锐,尽遣随征,军资器械,十不存一。宜急遣官分设,召募官舍馀丁义勇,起集附近民夫,更替沿河漕运官军。令其悉隶神机等营,操练听用。仍令工部齐集物料,内外局厂昼夜并工,成造攻战器具。京师九门,宜用都督孙镗、卫颖等给领兵士,出城守护,列营操练,以振军威。选给事中御史如王竑等,分出巡视,勿致疏虞。徙郭外居民于城内,随地安插,毋为寇掠。通州坝上仓粮,不可捐弃以资寇,令在官者,悉诣关支准为月粮之数,庶几两得。”帝嘉纳之。以兵部郎中罗通、给事中孙祥并为副都御史,分守居庸、紫荆等关。以薛瑄为大理寺丞,分守北门。命侍讲徐珵、杨鼎,检讨王询等行监察御史事,分镇河南、山东等处要地,抚安军民。令各处招募民壮,就令本地官司率领操练,遇警调用。起杨洪、石亨于诏狱,命洪仍守宣府,亨总京师兵马。亨有威望,方面巨躯,须垂至膝。先协守万全,坐不救乘舆,械系诏狱。至是,以于谦言赦出之,使总京营兵马赎
罪。 

\newpage

十月,也先以送上皇还京为名,与其汗脱脱不
花寇紫荆关,京师戒严。 

先是,太监喜宁,故鞑靼也。土木之败,降于也先,尽以中国虚实告之,为彼向道,奉上皇入寇。七日,至大同城下,守臣郭登曰:“赖天地祖宗之灵,国有君矣。”也先知有备,不攻去。九日,至广昌,破紫荆关,杀指挥韩清等,都御史孙祥走死。朝野汹汹,人无固志。赦交阯败绩论死成山侯王通为都督,升鸿胪寺卿杨善为副都御史,协守京城。太监兴安问王通计将安出,通以挑筑京师外城濠为对,兴安鄙之。侍讲徐珵方有时名,亦锐意功业。太监金英召徐珵问计,珵曰:“验之星象历数,天命已去,请幸南京。”英叱之,令人扶出。明日,于谦上疏抗言:“京师天下根本,宗庙、社稷、陵寝、百官、万姓、帑藏、仓储咸在,若一动则大势尽去,宋南渡之事可鉴也。珵妄言当斩。”太监金英宣言于众曰:“死则君臣同死。有以迁都为言者,上命必诛之。”乃出榜告谕,固守之议始决。谦闻寇迫关,思各处刍粟数万计,恐为敌资,急遣使焚之,然后奏闻。或请姑待报,谦曰:“寇在目前,若少缓,彼将据之,适以赍盗粮
\newpage

耳!独不见宋牟驼岗事乎?”众皆是之。 

己卯,也先长驱至京城西北关外。命石亨等军于城北,兵部尚书于谦督其军;都督孙镗军于城西,刑部侍郎江渊参其军,皆背城而阵。以交阯旧将王通为都督,与御史杨善守城。时众论战守不一,主将石亨欲尽闭九门,坚壁以避贼锋。谦曰:“不可。贼张甚矣,而我又先弱,是愈张也。”乃率先士卒,躬擐甲胄,出营德胜门,以示必死。泣以忠义谕三军,人人感奋,勇气百倍。尚宝司丞夏瑄陈四策:“一谓寇多骑,长于野战,短于攻城,且坚壁勿战,使之气沮,然后出奇设伏,诸道奋击。一谓寇深入,宜令死士夜袭其营,设伏内地,以待追者。一谓寇既举国入犯,边无所御,宜分边兵内外夹攻,彼将自溃。一谓我军依城为营,退有所归,宜以三队为法,前队战退,令中队悉斩以徇,不斩者同罪,使士知畏法。”诏趋行之。喜宁嗾也先遣使来议和,索大臣出迎驾。众莫敢出,乃以通政参议王复为礼部侍郎,中书舍人赵荣为鸿胪寺卿,出朝上皇于土城庙。也先、伯颜帖木儿擐甲持弓矢侍上皇。复等见上皇,进书敕。上皇视汉字书,也先视番字敕。也先曰:“尔皆小官,急令王
\newpage
直、胡濙、于谦、石亨来。”上皇谕复、荣曰:“彼无善意,汝等宜急去。”二人辞归。寇益四出剽掠,焚三陵殿寝祭器,逼宣武门,南逾芦沟桥,散掠下邑,攻城益急。石亨折弓厉声曰:“宰臣不出计,莫能支矣。”大学士陈循等疏请敕宣府、辽东总兵杨洪、曹义各选劲骑与官军夹击。又请旨募斩也先者,赏万金,封国公。复伪作喜宁与太监兴安书云:约诱也先入寇,欲乘其孤军取之。书为也先逻卒所获,也先颇疑喜宁。既而宣府、辽东兵至,军大振。时诸军二十二万列城下,寇见大军盛而严,不敢轻犯。以数骑来尝,谦设伏空屋,遣骑诱之。遂以万骑来薄,伏发败之。石亨出安定门,与其从子彪持巨斧突入中坚,所向披靡,敌却而西。亨追战城西,复却而南。彪率精兵千人诱寇至彰义门,寇见彪兵少,逼之,亨率众乘之,寇败走。神机营都督范广以飞枪火箭杀伤甚众。都督孙镗御寇西直门失利,诸将不相援。镗急叩门求入,给事中程信监军西城,言镗小失利,即开门纳镗。贼益张,人心益危。乃闭城趋镗战,寇逼城,镗兵走死地,亦附城战。信与都督王通、都御史杨善城上鼓噪,枪炮佐镗。毛福寿、高礼往援,礼中流矢。石亨兵亦至,乃引退。于是也先知我有备,气稍沮。于
\newpage
谦使谍,谍知上皇移驾远,命石亨等夜举火,大炮击其营,死者万人。也先以上皇北遁,脱脱不花闻之,遂不敢入关,亦遁。也先出居庸关,伯颜帖木儿奉上皇出紫荆关。诸将分兵蹑其后,石亨与从子彪复破寇于清风店,孙镗、杨洪、范广逐寇至固安,又捷,夺回人口万馀。时寇骑散掠各郡,不过百馀骑,驱人畜以自卫,望之若万众,然犹杀官军数百人,洪子俊几为所获。上皇出紫荆关,连日雨雪,乘马踏雪而行,上下艰难,遇险则袁彬执控,哈铭亦随之。既入寇营,也先来见,宰马,拔刀割肉,燎以进,云:“勿忧
,终当送还。”食讫辞去。 

脱脱不花遣使来献马,议和,朝廷却之。胡濙、王直曰:“脱脱不花、也先君臣素不睦,宜受其献
以间之。”从其言,使人入见,赐衣服酒馔金帛。 

协守大同都督郭登议率所部,并纠集义勇,从雁门入援。先以蜡书驰奏,大略谓:“戎马南驱,三关失险,留连内地,为患非轻。欲悉起各处官军民壮,入护内廷。京兵击于内,臣兵击于外,使贼有腹背受敌之虞,首尾不救之患。”且曰:“忠臣切已,敢
\newpage
忘报国之心;成败在天,不负为臣之节。”以贼退,优诏褒答之。时我师屡㜕,边陲无完地。大同兵士战没之馀,城门昼闭,人心土崩。有爱登者,泣谓之曰:“事已至此,奈何?”登曰:“天若祚国,必无他忧。若敌势莫遏,吾与此城誓相存亡,当不使诸君独死也。”大同孤危,登气益壮。吊死问伤,亲为痛恤。昼夜筹虑,修城缮兵,以图后举。寻京师围解,登上疏言:“寇骑虽回,离边不远。传报有云,黄河已冻,且向延绥。青草复生,再侵京阙。事虽未信,备必先修。乞推诚待下,侧席求贤;明理克欲,以成圣学;亲贤远佞,以收人望。”既又传也先将复犯京师,登以京兵新选,不可轻战,又疏曰:“今日之计,可以养锐,不可浪战;可以用智,不可鬬勇。兵法知彼知己,可守则守。其涞水、易州、真定、保定一带,皆坚壁清野,京兵分据,犄角安营。以逸待劳,以主待客,勿求侥幸,务在万全。此谓不战而屈人兵,
善之善者也。” 

命都指挥董宽率兵督河间、沈阳等卫,缉捕盗
贼。时降人安置畿内者,乘时并起为盗。 

\newpage

十一月,以寇退,京城解严,降诏抚安天下。杨洪等班师还京。论功封杨洪昌平侯,石亨武清侯。加于谦少保,总督军务。谦固辞,不许。有诵谦功者,辄谢曰:“四郊多垒,卿大夫之耻。今但不城下盟,何功也!”学士陈循疏言:“守居庸副都御史罗通晓畅军事,宜召还。守宣府总兵杨洪及子俊皆善战,宜留之京师。”于谦曰:“宣府,京师之藩篱,居庸,京师之门户,边备既虚,万一也先乘虚据宣府为巢窟,京师能安枕乎!”兵科给事中叶盛亦上言:“今日之事,边关为急。往者马营、独石不弃,则六师何以陷土木;紫荆、白羊不破,则寇骑何以薄都城!即此而观,边关不固,则京城虽守,不过仅保九门,其如寝陵何?其如郊社坛壝何?其如四郊生灵荼毒何?
宜急令固守为便。” 

先是,土木既败,边城多陷,宣府孤危。既而复召宣府总兵入卫京师,人心益惧。或欲遂弃宣府,纷然就道。都御史罗亨信不可,仗剑坐当门拒之,下令曰:“敢有出城者必斩。”众始定。城中老稚欢呼曰:“吾属生矣!”因设策捍御,督将士誓死守。寇知有备,不敢攻。至是,上从于谦、叶盛言,乃以左
\newpage
都督朱谦佩印镇宣府,纪广、杨俊副之。佥都御史王
竑镇居庸。 

上皇北至小黄河苏武庙,伯颜帖木儿妻阿挞剌阿哈剌令侍女设帐迎驾,宰羊递杯进膳。寻值圣节,也先上寿,进蟒衣貂裘,筵宴。哈铭、袁彬常宿御寝傍,天寒甚,每夜上皇令彬以两胁温足。一日晨起,谓铭曰:“汝知乎?汝夜手压我胸,我俟汝醒乃下手。”因言光武与子陵共卧事。铭顿首。上皇夜出账房,仰观天象,指示二人曰:“天意有在,我终当归也。”上皇使哈铭致意伯颜妻,令劝伯颜送还朝。妻曰:“我妇人何能为!然官人洗濯,我侍巾帨,亦当进一言。”伯颜尝因猎得一雉,并酒一卣来献。铭时时
设喻慰上皇勿忧,或成疾。 

时也先声言欲送上皇还,众遂多主和。于谦独排众议曰:“社稷为重,君为轻。”遣人申戒各边将,毋堕贼计。命尚书石璞镇守宣府,都御史沈固镇守大同,都督王通守天寿山,佥都御史王竑城昌平,都御史邹来学提督京都军务,平江伯陈豫守临清,副都

\newpage
御史罗通守山西。 

景帝景泰元年春正月,上皇书至,索大臣来迎。命公卿集议,廷臣因奏请遣官使北,贺节进冬衣。上谓必能识太上皇帝者始可行。群臣惧,谢罪。事遂
寝。 

大同总兵郭登败寇于栲栳山。寇入大同境,登率兵蹑之。行七十里,至水头,日暮休兵。夜二鼓,有报云:“东西沙窝贼营十二,皆自朔州掠回。”登召诸将问计,或言:“贼众我寡,莫若全军而还。”登曰:“我军去城百里,一思退避,人马疲倦,贼以铁骑来追,即欲自全得乎?”按剑起曰:“敢言退者斩。”径薄贼营。天渐明,贼以数百骑迎战,登奋勇先登,诸军继进,呼声震山谷。登射中二人,手刃一人,遂大破其众。追奔四十馀里,至栲栳山,斩首二百馀,夺还人马器械万计。进封定襄伯,食禄千一百石,与世券。是役也,登以八百骑破寇数千,为一时战功第一。登为将智勇,善抚士卒,纪律严明,料敌制胜,动合机宜。在大同与贼相拒一年,大小数十战,未曾挫衄。常恨马少,步卒追贼不及。乃以己意设为夹地龙、飞天网,凿为深堑,覆以土木,人马通行
\newpage
,如履实地。贼入围中,令人发机,自相击撞,顷刻十馀里皆陷。又用炮石击贼,一发五十馀步,人马死者数十,贼传以为神云。时也先分调各部扰边,朱谦败之于宣府,杜忠败之于偏头关,王翺败之于辽东,马昻败之于甘州。修城堡,简精锐,各边皆有备。石亨佩大将军印巡边,石彪、杨俊亦间出,中国势遂振
。 

闰正月,叛人小田儿伏诛。小田儿为也先乡导,杂使中来瞷虚实,于谦授计侍郎王祎,就大同道诛
之。 

二月,叛臣喜宁伏诛。宁怀二心,教也先扰边。已不欲送上皇还,上皇深恶之。宁又忌袁彬,诱彬出营,将杀之,上皇急救之,乃免。彬与上皇谋,遣宁传命入京,令军士高磐与俱。密书系磐髀间,令至宣府,与总兵等官计擒之。既至宣府,参将杨俊出,与宁饮城下,磐抱宁大呼,俊纵兵,遂缚宁送京,诛
之。也先闻宁诛,与赛刋王等分道入犯。 

三月,也先、赛刋王寇大同、阳和,大同王寇
\newpage
偏头关,答儿不花王寇乱柴沟,铁哥不花王寇大同八里店,铁哥平章寇天城,脱脱不花王寇野狐岭并万全

夏四月甲戌,户部尚书金濂等议寇骑犯边,大军失利,遗有马营、独石、龙门、雕鹗等处刍粮,宜令督储侍郎刘琏、提督军务副都御史罗通及宣府总兵
朱谦、游击杨能会计徙运宣府。从之。 

都督杨俊请大举出塞,大同、宣府列营坚守为正兵,独石、偏头乘间设伏为奇兵,悉发京营与诸镇兵,出塞逐北,而犁其王庭,可以得志。于谦曰:“报仇雪耻,臣等职也。顾兴兵举事,系社稷安危。即如俊所言,万一我军出塞,贼以偏师缀我,而别遣部落间道乘虚入寇,是自撤藩篱,非万全计,臣愚未见
其可。”上从谦议。 

大同参将许贵请遣使腆币,以款寇兵,而徐为讨伐计。于谦曰:“前者固非不遣使。都指挥季铎、指挥岳谦遣,而寇骑已至关口。通政王复、少卿赵荣遣,而不获征太上一信。其狡焉侮我而龁我,何似而可言和?况也先不共戴天仇也,理固不可和。万一和
\newpage
而彼遂肆无厌之求,从之则坐弊,不从则生变,势亦不可和。贵介胄之臣,而委靡退怯,法当诛。”是时上任谦方专,疏既入,于是边将人人言战守。也先不
得挟重相恫喝,抱空名不义之质,始谋归太上矣。 

谍报也先逼总兵朱谦于关子口。明日,复报追石亨于雁门关。烽火连属,众皆恐,请大发兵援之。于谦策也先大队尚远塞,必张疑兵以胁我。乃上方略,授石亨,使皆坚壁,而令各营秣马厉士,若将大举者。仍遣延绥总兵帅骑渡河,于保德州设伏截杀。从
之。已而贼果不至。 

于谦以畿辅诸州郡兵力单甚,乃皆宿兵。奏遣都指挥陈旺、石端、王信、王竑等分屯涿鹿、真定、保定、易州诸处,而以右都督杨俊帅焉。久之,皆屹
然重镇。 

五月乙已,巡抚山西都御史朱鉴奏:“也先分道入寇,请令关隘守将画地救援。寇犯河曲、保德、岢岚,宜令偏头关策应;犯宁化、静乐、忻州、定襄、太原、清源、交城、文水,宜令山西策应;犯五台
\newpage
、繁峙、崞县,宜令雁门关策应。其石州、宁乡,宜令汾州守备分兵协守。”从之。武清侯石亨奏:“寇骑六万围代州,官军出战有斩获。又分营雁门关一路,恐侵京师。”下廷臣议:“黄花镇、鞍口,外卫西北边境,内护陵寝京师,宜益兵守备。”从之。仍令兵部稽在京军马数以闻。寇骑犯宣府,总兵都督朱谦等率兵力战,却之,官军阵亡者百四十人。都督江福
等兵应援不利,杀伤百馀人。 

兵部言:“通事马云、马青先奉使迤北,许也先细乐伎女,又许与中国结婚,皆出自指挥吴良,致
开边衅,请寘诸法。”诏下锦衣卫鞫之。 

立京团营操法。初,太宗以北伐故,宿重兵燕中。会承平久,不能无老弱,公侯中贵人往往役占。土木之难,精锐略尽,虽有五军、神机、三千诸营,然不相统一,每遇调遣,号令纷更,兵将不相识。于谦上言:“兵冗不练,遇敌辄败。额四十馀万,非尽可用者,徒费大家米。”于是即诸营选马步骁悍者十五万,分为十营。每营各以都督领之。五千人为一小营,营以都指挥领之。团操以备警急,是为团营,而
\newpage
以谦总督。列侯石亨、杨洪、柳溥为总兵,太监曹吉祥、刘永诚等监之。馀步骑仍归三大营,曰老营。自是兵将相识,每出征即令原管都督领之,故号令归一。洪、亨皆老将宿猾,而亨尤贪纵。谦威令严密,目视指屈口奏,悉合机宜。亨等虽为大帅,进止赏罚一
由谦,相顾𫖯首而已。 

戮左都督杨俊。俊,杨洪子也,恃勇桀骜不可驯。先备独石、马营等。土木之变,弃城逃归,马营、龙门等入城皆不守。既而命为参将,帅兵巡哨怀来等处,复辄调永宁守备官军于怀来,将永宁城西门砌塞。于谦劾其“方命专权,擅作威福”。诏宥不问。俊又以私怒都指挥陶忠,杖挞死。父洪惧祸,奏取俊还京,随营操练。既至,谦并劾其独石弃城,丧师辱国,及怀来私仇,捶死边将之罪,谓:“非诛俊,无以惩戒将来。”兵科给事中叶盛等亦劾之。于是逮系
法司,议罪,斩于市。 

阿剌遣使贡马请和,边臣留之怀来,以闻。是时,鞑靼政事,也先专之,兵最多。脱脱不花虽为汗,兵少。知院阿剌兵又少。君臣鼎立,外亲内忌。其
\newpage
合兵南侵,利多归也先,而弊则均受。及也先欲和,耻屈意,阴使阿剌等来言。于是礼部会议,请遣太常少卿许彬、锦衣都指挥同知马政译来使情伪。彬等言:“也先果欲议和罢兵,且奉还上皇。”奏至,帝问尚书学士陈循曰:“也先可和耶?”循曰:“遣而备之。”上曰:“然。”乃降玺书厚赐阿剌,数“也先挟诈,义不可从。即阿剌必欲和好,待瓦剌诸部落北
归,议和未晚。不然,朕不惜战也。” 

六月,吏部尚书王直等言:“也先遣使请上皇还京,盖上下神祇阴诱其衷,使之悔悟。伏望皇上许其自新,遣使臣前去审察诚伪。如果至诚,特赐俯纳,奉迎上皇以归,不复事天临民。陛下但当尽崇奉之礼,庶天伦厚而天眷益隆。”上曰:“卿言甚当。然此大位非我所欲,盖天地祖宗宗室文武群臣之所为也。自大兄蒙尘,朕累遣内外官员赍金帛迎请,也先挟诈不肯听。若又使人往,恐假以送驾为名,羁留我使,率众来犯京畿,愈加苍生之患。卿等更加详之,勿
遗后患。” 

上皇驾至大同。先是,也先入寇,声言选战马
\newpage
奉上皇南归。是日至大同,定襄伯郭登设计于城月门里,具朝服以候。潜令人伏城上,俟上皇入,即下城
闸板。既及门,寇觉之,遂拥上皇退去。 

武清侯石亨言:“雁门关一带山口,虽已筑塞,贼犹漫山径过,须断其半山可行之处。京城四面,宜筑墩台以便瞭望。”署都督佥事刘鉴言:“京师与怀来止隔一山,请自怀来筑烟墩,直至京师土城。遇
事,举火以报。”从之。 

秋七月,也先屡以和议不成,复俾其知枢密院阿剌为书,遣参政完者脱欢等五人至京师请和。礼部议。尚书胡濙等奏奉迎上皇,帝不允。次日,帝御文华殿,召文武群臣谕曰:“朝廷因通和坏事,欲与寇绝,而卿等屡以为言,何也?”吏部尚书王直对曰:“上皇蒙尘,理宜迎复。乞必遣使,勿使有他日悔。”帝不怿曰:“我非贪此位,而卿等强树焉,今复作纷纭何!”众不知所对。于谦从容曰:“(大)[天]〈据李贽《续藏书》卷十五《于谦传》改。〉位已定,孰敢他议!答使者,冀以舒边患,得为备耳!”帝意始释,曰:“从汝,从汝。”言已,即退。群臣
\newpage
出文华门,太监兴安传呼曰:“孰堪使者?有文天祥、富弼乎?”众未答,王直面赤,厉声曰:“是何言!臣等惟皇上使,谁敢勿行者!”安语塞,入复。时李实任礼科都给事中,帝命兴安传旨欲遣之,对曰:“实不才。然朝廷多事,安敢辞。”兴安入复命,遂以李实为礼部右侍郎,充正使,罗绮为大理寺少卿,充副使,马显授指挥使,为通事。上御左顺门召实等面谕曰:“尔等见脱脱不花、也先,立言有体。”上遗书脱脱不花可汗曰:“我国家与可汗,自祖宗来,和好往来,意甚厚。往年奸臣减使臣赏,遂失大义,遮留朕兄。今各边奏报,可汗尚留塞上,杀掠人民。朕欲命将出师,念彼此人民,上天赤子,可汗杀朕之,朕亦杀可汗人,与自杀何异?朕不敢恃中国之大,人民之众,轻于战鬬,恐逆天也。近得阿剌使奏言已将各路军马约束回营,是有畏天之意,深合朕心。特遣使赍书币达可汗,其益体朕意,副天心。”复降玺书谕也先及阿剌,遗可汗、也先、阿剌白金文绮。时阁臣及抚部诸臣承上意,止言息兵讲和,不及迎复上皇意。实等遂偕完者脱欢行。以十七日至也先营,地名失八秃儿。既见也先,读玺书毕,乃引见上皇。上皇居伯颜帖木儿营,所居毡毳帐服,食饮皆膻酪,牛
\newpage
车一乘,为移营之具。左右惟校尉袁彬暨哈铭侍。实等见上皇泣,上皇亦泣。上皇曰:“朕非为游畋而出,所以陷此者,王振也。”因问太后、皇上、皇后俱无恙,又问二三大臣。上皇曰:“曾将有衣服否?”实等对曰:“往使至,皆不得见天颜,故此行但拟通问,未将有也。”实等乃私以所有糗饵常服献。上皇曰:“此亦细故,但与我图大事。也先欲归我,卿归报朝廷,善图之。傥得归,愿为黔首,守祖宗陵墓足矣。”言已,俱泣下。实等因问:“上居此,亦思旧所享锦衣玉食否?”又问:“何以宠王振至此,致亡国?”上皇曰:“朕不能烛奸。然振未败时,群臣无肯言者。今日皆归罪于我。”日暮,实等归宿也先营,酌酒相待。也先、伯颜貂裘胡帽,其妻珠绯覆面垂肩。碗酪盂肉,更互弹琵琶,吹𥫩儿,按拍歌劝酒。也先曰:“南朝我之世仇。今天使皇帝入我国,我不敢慢。南朝若获我,肯留至今日乎?”又言:“皇上在此,吾辈无所用之。每遣使南朝令来迎,竟不至,何也?”实等反复譬晓,欲奉迎上皇意。也先曰:“南朝遣汝通问,非奉迎也。若归,亟遣大臣来。”实等遂辞归。上皇出三书授实,其一上皇太后,其一达于上,其一谕群臣。伯颜帖木儿约实速来成和好,且
\newpage
指也先幼子曰:“此与朝廷议姻者。”实不敢对。实未至京,会脱脱不花亦遣使皮儿马黑麻请和,右都御史杨善慨然请行。人皆危善,善曰:“上皇在沙漠,此为臣者效命之秋也。”中书舍人赵荣亦请往,乃遣善、荣及指挥王息、千户汤胤𪟝,同皮儿马黑麻往。道遇实,实告以故。善曰:“得之矣,即敕书所无,可权以集事也。”实既还朝,具述也先情,及上皇起居状。诸文武大臣合疏言:“李实出塞,道中行,北骑闻欲议和,皆举首加额,及见也先,殊喜,言迎使夕来,大驾朝发。”实又具道也先悔过,宜迎复。上曰:“也先诈。杨善已去。第以迎复意书敕付也先。”使还,大臣言:“也先非诈也,臣等询李实详矣。彼使来和,当遣使答。今请迎复,乃不与偕,是轻迎驾重讲和也。不迎驾归,何以和为?”帝令再议。李实言:“也先约臣迎驾,毋出八月五日。臣言须得旨,不敢擅为期。也先言期必不可失,遂令渠长偕罗绮往大同,调还扰边人马。臣还过怀来、宣府,见军民始敢出郊刍牧,诚非空言。伏望陛下俯从群请,脱有虞诈,亦可塞之。若过所期,更欲使臣,亦不敢往。”帝竟付迎复于敕书而已,不遣使,曰:“待杨善归。”监察御史毕銮复言:“群臣之情切矣。陛下必待
\newpage
善归。夫中国所恃者信义也,不迎不义,失词非信。就令彼诈,我备在也。”翰林邢让亦以为言。帝曰:“上皇朕兄,岂有不迎?彼情叵测,正欲探之。情诚而迎,又何暮焉。”杨善既出境,也先使所善田民者,为馆伴来迎,且有所探,饮帐中,谓善曰:“我亦中国人,被留于此。前者土木之役,六师抑何弱也?”善曰:“当是时,六师之劲悉南征,而中贵人振欲邀太上幸故里,止扈从,一不为备,故溃。虽然,彼幸而胜,未见为福。今者南征之士悉归,可二十万。又募中外材官技击,得三十万。悉教以神枪、火炮、药弩,射命中,百步之外洞人马,复穿七札。又用言者计,沿边要害,皆隐金椎三尺,所值蹄立穿。刺客林立,夜度营幕若猿猱。而皆已矣,置之无用矣。”问:“何以言无用?”曰:“和议成,方且欢饮若兄弟,而又何用也!”其人悉以语也先。二十九日,至也先营,值其出猎。八月初二日丁卯,与也先相见,也先问减马价故。善曰:“往时外使,不过三十人。今多至三千馀人,即稚子亡弗赉者,金帛器服络绎载道,而岂得言薄?”也先曰:“然则奈何留我使?予我帛,时剪裂幅不足者?”善曰:“帛有剪裂不足者,通事为之也,事露而诛矣。即所进马有劣弱,而貂
\newpage
皮敝,岂太师意耶?至使臣所从人,为奸盗他所,或遇害,中国留之何用!”也先又问市釜事,善言:“此小民市易,朝廷岂知?”善因历述累朝恩遇之厚不可忘。且言天道好生,今纵兵杀掠,上干天怒,反复辨论,数千百言。也先喜。也先问:“上皇还,更临御否?”善言:“天位已定,不得再易。”也先问:“古尧、舜事如何?”善言:“尧让位于舜,今日兄让位于弟。”也先悦服。平章昻克问善:“欲迎复,来何操?”善言:“若操贿来迎,后人以尔贪贿归上皇。今无所操而归,书之史册,后世皆称述。”也先然其言,曰:“史中好为书也。”伯颜帖木儿请留使臣,遣使欲南朝更请上皇临御。也先曰:“曩令遣大臣来迎,大臣至矣,不可无信。”引善见上皇。明日,也先设宴饯上皇于其营,善侍。也先与妻妾以次起为寿。酒中,令善坐。上皇亦曰:“从太师言,坐。”善曰:“虽草野,不敢失君臣礼。”也先顾羡曰:“中国有礼。”罢酒,送上皇出。明日,宴使臣。又明日,伯颜帖木儿设宴饯上皇。又明日,亦宴使臣。又明日,癸酉,上皇驾行,也先与渠帅送车驾可半日许,下马,解弓箭战裾以进,诸渠帅罗拜哭而去。伯颜帖木儿独送上皇至野狐岭,进酒账房。既毕,屏人
\newpage
语哈铭曰:“我也先顺天意,敬事皇帝一年矣。皇帝此来,为天下也,归时还当作皇帝,即我主人,有缓急我可得告诉。”众皆道傍送驾,进牛羊。善口呼:“皇帝行矣!”伯颜帖木儿再送驾出野狐岭口,上皇揽辔,慰藉而与之别,伯颜帖木儿大哭归,仍命渠帅率五百骑送至京师。既别去,行数里,复有追骑至,上皇失色。既至,乃其平章昻克出猎得一獐,驰使来献。受之,乃去。驾入关。丁丑,上皇至宣府南城。上遣太常少卿许彬奉迎。工部尚书高谷、给事中刘福等言:“奉迎上皇,礼不宜薄。”礼部连日会议未定。壬午,上皇至宣府。癸未,千户龚遂荣投书于高谷所。谷袖入,传示文武大臣。王直、胡濙谓:“礼失而求诸野。”欲以上闻,中止。给事中叶盛、程信、于太上疏言:“诸大臣持一帖,群立午门傍聚观,议论藉藉,乞宣问之。”书言上皇之出,以宗社故,非游猎也。都人闻上皇且还,无不喜跃,迎复礼宜厚,上亦当避位恳辞,然后复位,否则贻讥后世。上诘诸大臣,已而知书出谷所。上曰:“朕未尝塞言路,谷大臣,胡不告朕,为匿名书耶?”遂荣恐累谷,乃发愤自白。陈循、王文见之恚甚,请治其罪,下锦衣卫狱。然上不深罪也,寻释之。己卯,上皇至怀来。将
\newpage
抵居庸,礼部始得旨,群臣同礼部议迎复仪注,兵部总戎议防变方略,百官集会议所,都御史王文忽厉声曰:“孰以为来耶?黠寇不索金帛,必索土地耳!”众素畏文,相顾莫敢言。给事中叶盛等造礼部问,时胡濙已具仪注送内阁矣。略谓:“天宝之乱,玄宗幸蜀,肃宗即位灵武,尊玄宗为太上皇帝。肃宗收复两京,迎还上皇。至咸阳,备法驾望颜楼。上皇在宫南楼,肃宗著紫袍,望楼上,拜舞楼下。上皇降楼,抚肃宗而泣,辞黄袍,自为肃宗著之。肃宗伏地,顿首固辞。上皇曰:‘天下人心皆归于汝,使朕得保馀龄,汝之孝也。’肃宗乃受。今备法驾安定门外,诚为太简。”帝曰:“虑堕狡寇计,故简其礼。大兄入城,朕知尊亲。”遂备法驾候安定门外。庚辰,上皇至唐家岭,遣使回京,诏谕避位,免群臣迎。丙戌,百官迎上皇于安定门。上皇自东安门入,上迎拜,上皇答拜,各述授受意,逊让良久。乃送上皇至南宫,群
臣就见而退,大赦天下。 

命保定伯梁瑶征苗寇,以河间等降丁从征。先是,永乐间,塞北部落来降者,多安置河间、东昌等处,生养蕃息,强悍不可制。方也先入寇,乘机骚动
\newpage
。至是,大发兵征两广、湖、贵苗寇,兵部尚书于谦奏遣之。其有名号者厚赏犒,随军有功则官之。已而
更遣其妻子往,自是肘腋无他患。 

二年秋九月,也先遣使求通好,固邀我使往报
。上从言官议,诏绝之。 

三年夏四月,命都督同知孙安镇守独石、马营
,以兵科都给事中叶盛为山西右参政,协赞军务。 

先是,杨洪镇独石、马营等八城。已巳失守,残毁未复,议者欲弃之。于谦曰:“弃之则不但宣府、怀来难守,京师不免动摇。”乃荐安,授以方略,仍命盛赞其军务。盛至,列利害八条以进,次第行之。率兵度龙门关,且战且守,八城完复如旧。盛又请帑金五千两,买牛犊,简戍卒不任战者,俾事耕稼,岁课馀粮于官,凡军中买马、修器、劳功、恤孤诸费皆取之。盛在独石五年,军民赖之,边境得安。时土木北狩,浙、闽、三楚、贵、竹盗贼蜂起,前后命将将兵,皆出谦独运,号令明审,动合机宜。虽宿旧勋臣,少不中程律,即请旨切责不贷。片纸行万里,电
\newpage
耀霆击,靡不惴惴效力,毋敢饰虚辞以抵者。以故天
下咸服谦,而归上能用人。 

谷应泰曰:英宗北狩,战士兵甲死亡略尽,边关守隘望风奔溃,摇足之间,黄河以北非国家有矣。幸而迁都议格,锺簴不惊。然而君父叩关,臣子拒敌,彼出有名,我负不义。狐疑既生,上下瓦解,讲使亟行,责问无已。长安必不可守,英宗必不能归,徒使有贞之辈操星象而笑其后也。嗟乎!南迁不行,然后国存;和议不行,然后君存。两议俱息,君国皆存,而少保之祸不得旋踵矣。当夫北兵四合,守御单寒,虎穴故君,已置度外,围城新主,亦危孤注,身先矢石,义激三军,家置环寺之薪,人守州兵之哭。傲如石亨,怯如孙镗,懦如王通,无不斩将搴旗,缘城血战,追奔逐北,所向披靡。此一役也,军声复振,君臣固守,陵阙盘石矣。然而遣使入朝,动请迎驾,悬师剽掠,辄托回銮。彼直我曲,彼壮我老。也先者,方且挟此奇货,羁制中原。以战不败,以和可成,输币不还,进而割地,割地不归,诱之称臣,中原生灵,自此无安枕矣。而乃兄终弟及,父子之情既割;社稷为重,君臣之义亦轻。至则龙衣糗食,敬输橐𫗴
\newpage
之忱;归亦别院闲宫,不过汉家之老。然则挟天子者,挟一匹夫耳!邀利之心懈,而好义之心萌,郭登之
言决,而杨善之说行,英皇自此生入玉门矣。 

昔太公置鼎,汉祖分羹;徽、钦被执,宋高哀请。一则新丰鸡犬,还老阙庭;一则泪洒冰天,终于舆榇。盖相如碎璧而璧存,贾胡藏珠而珠去,拥空名者视同虚器,居必争者势难瓦全也。夫昭王沈汉,穆满难归;楚怀入秦,顷襄不反。彼此得失,危不间发。故汉高分羹之语,乃孝子之变声;郭登有君之谢,实忠臣之苦节。英宗不感生还,反疑予敌。谦死东曹,登贬南都,忠臣义士所以仰天椎心而泣血也。景帝外倚少保,内信兴安,狡寇危城,不动声色。当时朝右,岂乏汪、黄;建炎践祚,亦有宗、李。相提而论景诚英主。而乃恋恋神器,则又未闻乎大道者也。

\end{document}
