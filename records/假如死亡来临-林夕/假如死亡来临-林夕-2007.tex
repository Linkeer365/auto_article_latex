\documentclass{article}
\usepackage[utf8]{inputenc}
\usepackage{ctex}

\title{假如死亡来临\footnote{Click to View:\url{https://web.archive.org/web/20230617032727/https://www.sohu.com/a/245382733_304015}}}
\author{林夕}
\date{2007-02}

% \setCJKmainfont[BoldFont = Noto Sans CJK SC]{Noto Serif CJK SC}
% \setCJKsansfont{Noto Sans CJK SC}
% \setCJKfamilyfont{zhsong}{Noto Serif CJK SC}
% \setCJKfamilyfont{zhhei}{Noto Sans CJK SC}
% \setlength\parindent{0pt}

\begin{document}
\CJKfamily{zhkai}

\maketitle


\Large

朋友是做证券生意的,整天满世界跑,难得
见他一面。我们通常的联络方式是打电话。 

有一天晚上,他打电话来,我们东南西北地聊。他突然问我:“如果让你花一元钱,可以买到你哪
一天会死的信息,你买不买?” 


我想了想,摇摇头说:“不买。” 


“为什么?” 

“人生最大的痛苦莫过于知道自己哪天死。我认为,最好的死亡方式是:让死亡突然间来临,来不
及思考,生命突然终止。” 

\newpage

沉默片刻,电话那端,他轻声说:“可是,我
买。” 


“我怕死亡突然来临时,还有许多想做的事没有做。不过,我也不想知道得太早,提前10天让我
知道就行。” 


“你想用这10天做什么?” 

“5天的时间给我的家人,好好陪他们;5天
的时间给我自己,做我最喜欢做的事情。” 


“你最喜欢做的事情是什么?” 

“和我爱的人在一起,开着车带她穿过大森林
。” 

我笑了:“这并不难,你为什么不现在就做呢
?” 

\newpage

他叹了口气:“现在这么忙,哪有时间啊?”
 

我也在心里叹了口气,不禁想起另一位朋友。他是一家外贸公司的经理,也是满世界地飞,整天忙着谈判、签合同,一年难得回家几次。他觉得很欠妻子和女儿的,就说等公司业务发展好了,陪她们去欧洲度假。公司的业务一直在发展,可是他总觉得还不够好,结果一拖再拖,始终未能成行。后来,他赴日
本谈判时,心脏病发作死在途中。 

许多时候,我们总把最喜欢做的事情留在最后。可惜,死亡来临之前并不通知我们。尽管我们已经荣幸地迈入21世纪信息时代,信息高速公路已经架到我们家门口,却没有一家公司可以出售死亡的信息。所以绝大多数人留在最后、最喜欢做的事情,最后带进坟墓里去了。

\end{document}
