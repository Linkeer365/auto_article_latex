\documentclass{article}
\usepackage[utf8]{inputenc}
\usepackage{ctex}

\title{我的名字叫红·第29章·我是你们的姨父\footnote{Click to View:\url{https://web.archive.org/web/20230409140739/https://www.99csw.com/book/2362/71166.htm}}}
\author{奥尔罕·帕慕克}
\date{1998}

% \setCJKmainfont[BoldFont = Noto Sans CJK SC]{Noto Serif CJK SC}
% \setCJKsansfont{Noto Sans CJK SC}
% \setCJKfamilyfont{zhsong}{Noto Serif CJK SC}
% \setCJKfamilyfont{zhhei}{Noto Sans CJK SC}
% \setlength\parindent{0pt}

\begin{document}
\CJKfamily{zhkai}

\maketitle


\Large

他一说是他杀了高雅先生,屋内就出现了时间的死一样的沉寂。我想他也会杀了我。我的心怦怦跳了很久。他来这里是为了杀我吗,还是为了来自首并恐吓我?他知道自己究竟想要什么吗?我很害怕,明白了尽管自己多年来熟悉这位杰出画家所有的技巧和能力,但对他的内心世界却一无所知。我能感觉到他僵直地站在我身,面对我的颈背,拿着大的红墨水瓶,不过,我没有转身看的脸。因为知道我的沉默会
让他感到不舒服,所以: 


“野狗还在吠个不停。”我说。 

我们再度陷入沉默。这一次,我知道我的死亡,或者我是否能避免这场厄运,将取决于我,取决于我对他要说的话。除了他的作品,我只知道他是个极
\newpage
聪明的人,如果你们同意一位插画家绝对不可在作品中流露他的灵魂,那么这一点当然是值得骄傲的事情。他是如何趁着没人在家的时候来这里堵住我的呢?我衰老的心里一直在迅速地盘着这些,但脑子却一片
混乱,找不出头绪。谢库瑞在哪里呢? 

“你先前就知道是我杀了他,对不对?”他问
。 

我根本不知道,他向表白了我才知道。在我的内心深处,甚至在想着他杀死高雅先生或许未尝不是一件好事,那位已故的镀金大师可能真的慢慢地屈服
于自己的恐惧,会把我们大家都毁了的。 

面对这位我独自与他共处一室的凶手,我的心
底隐约升起了一股感激之情。 

“你杀了他,我并不感到惊讶。”我说,“我们这种活在书本中、做梦都梦见书页的人,只害怕这世上的一样东西。不但如此,我们挣扎着面对更大的禁忌与危险,在穆斯林城市中搞绘画。如同伊斯法罕
\newpage
的画家谢赫·穆罕默德一样,我们每一个细密画家都免不了内心感到罪恶与后悔,有一种强烈的刺激因素在刺激着我们最先责怪我们自己,使我们感到后悔而乞求真主和社会宽恕。我们总是像罪人一样,更多时候像是怀着歉疚,偷偷摸摸地制作书本。教长、传道士、法官和神秘主义者们总是指控我们犯有亵渎罪,对我们进行攻击。我十分清楚,对于他们无休止的攻击的屈服,以及我们自己的这种无穷尽的罪恶感,扼
杀同时也滋养了细密画家的想像力。” 

“也就是说,你不怪罪我清除了那个白痴高雅
先生吗?” 

“文章、插画、绘画中吸引我们的东西也就在这恐惧当中。我们之所以从早到晚,跪着在烛光下彻夜工作,直到双目失明,为绘画和书籍献自己,绝不只是为了金钱和赏识,而是为了逃离他人的嘈杂,逃离人群。然而相对于创作的热情,我们也想让那些我们所要逃离的人们,观看欣赏我们受启示创造出来的画。但要是他们说我们无信仰呢,这会给一位真正具备天赋才华的画家带来多大的痛苦!然而,真正的绘
\newpage
画也正隐藏在这无人能见、也无人能表现的痛苦之中,它就在那些最初人人都会说是坏的、没画好的、没有信仰的图画里。一位真正的细密画家明白他必须达到那个境界,与此同时,他也害怕到了那个境地后的孤独。又有谁会愿意一生都忍受这种可怕、焦虑的生活呢?在别人之前先责备自己,细密家以为这样就能摆脱多年来所承受的恐惧人们也只是在他坦陈其罪行时才会相信他,才会把他烧死。伊斯法罕的插画家则
是为自己点燃了这把炼狱之火。” 

“但你并不是细密画家。”他说,“我也不是
出于害怕才把他杀死的。” 

“你之所以杀他是因为你想要照你所想的那样
毫无恐惧地来绘画。” 

长久以来头一次,这位想要杀我的细密画家说出了颇有智慧的话:“我知道你说这些是了转移我的注意,愚弄我,好从这种处境中摆脱出来。”他接着又说:“但你最后所说的没错。我要你明白这一点。

\newpage
听我说。” 

我扭头看着他的眼睛。当他说话时,已经浑然忘记我们之间惯的礼仪。他被自己的思绪牵着走。然
而,是往哪儿去呢? 

“用不着担心,我不会侮辱你的尊严。”他说。他从我的身后绕到了我的前方,哈哈笑着,但却有着非常痛苦的一面。“就像现在这样,”他说,“我在做什么事情,但感觉做这种事的人不是我。仿佛体内有什么东西在扭动,让我干所有的坏事。不过我确
实需要它,对于绘画来说也是一样的。” 


“这些都是关于魔鬼的无稽之谈。” 


“也就是说我在撒谎吗?” 

我感到他没有足够的勇气杀死我,所以想要我激怒他。“不,你没有撒谎,但却不知道你内心所感
受到的东西。” 

“不,我清楚我内心的东西,我还没死就承受
\newpage
着死后的痛苦。我们不明就里地因为你而陷入了罪孽之渊。可是现在你居然对我说‘要再勇敢点’。因为你我成了凶手。努斯莱特教长的疯狗们会把我们都杀
光的。” 

他愈是没有自信,喊的声音就愈大,而且更用力地抓紧了手里的墨水瓶。会有人经积雪的街道,听
见他的叫喊而进屋里来吗? 

“你怎么会杀他的?”我问,更多的是想争取时间而非出于好奇,“你们是怎么在那口井边相遇的
?” 

“高雅先生离开你家的那天晚上,是他自己找的。”他说,出乎意料地想要自白,“他说见到了最后一幅双页图画。我费尽唇舌劝他别小题大做。我带他来到了被大火焚烧的地方,告诉他我在井边埋了钱。他听说有钱,就相信了我的话。还有什比这更能证明这位画家的动机其实源于贪婪?因此我不觉得遗憾。他是一个有才华但又平庸的画家。这贪婪的蠢蛋马上准备用指甲去挖冰冻的泥土。如果我真有金子埋在
\newpage
井边,就不用干掉他了。没错,你为自己挑选了一个卑鄙的家伙来替做镀金的工作。我们的往生者的确有技巧,但选色和用色却很低俗。我没有留下一丝痕迹。告诉我,什么是‘风格’的本质?今天,法兰克人和中国人都在谈论一位画家才华的特色,都在谈论所谓的‘风格’。究竟一位好画家该不该有风来区别于
他人?” 

“不用担心,新的风格并不一个细密画家想有就有的。”我说,“一位王子会死,一位君王会打败,一个似乎天长地久的时代会结束,一个画坊会被关闭,那里的画家们都会四散而去,会四处去为他们自己找寻其他爱好书籍的保护者。也许将来有一天,一位仁慈的苏丹会从不同的地方,比如说从赫拉特,从哈勒普召集起那些流亡在外、满腹困惑但华洋溢的细密画家和书法家,邀请他们来到自己的营帐或宫殿,建立起他自己的画坊。即使这些互不熟悉的艺术家们最开始仍用他们各自所知古老风格来进行绘画,但过了一段时间,就好像街上在一起打闹的小孩子们一样,他们之间也会发生同化、争执、互斗。在经过了多年的争执、嫉妒以及对排版、色彩与绘画的钻研之后
\newpage
,出现的就是一种新的风格。通常,创造出这种风格的人,是那个画坊里最优秀、最具天赋的细密画家,我们也可以说他是最幸运的。其余细密画家所能做的,便是通过无止境的模仿,不断修饰这一风格,使其
臻至美。” 

他无法再直视我的眼睛,带着一种出乎我意外的温和态度,恳求我的仁慈与诚实,几乎像个少女般
颤抖着问我: 


“我有自己风格吗?” 

一下子,我以为自己就要掉下泪来了。鼓起所有的温柔、同情和慈爱,我迫不及待地告诉了他我所
相信的事实: 

“在我六十多年的生命中,我所见到的最才华横溢、手最巧、眼光最细腻的细密画家就是你。如果在我面放一幅由一千个细密画家合作完成的绘画,我
也能够立刻辨认出你那真主所赐的笔触。” 

\newpage

“我也是这么想的,但我知道你并没有聪明到能够明白我技巧中的奥秘。”他说,“你在说谎,因为你怕我。尽管如此,你还是从头开始说说我的风格
。” 

“你的笔似乎脱离你的控制,依照自己的意志,选择正确的线条。你笔下的图画既不写实也不轻浮!当你画一个拥挤的场景时,通过人物的眼神和他们的位置,使得文字意义中的张力幻化成为一声优美永恒的呢喃。我一遍又一遍地看你的图画,就为了倾听那一声呢喃。每一次,我都愉地发现它的意义又改变了。该怎么说呢,我会重新细读你的图画,这样一来,就能把里面一层层的意义堆叠起来,显现出的深度
甚至远超越欧洲大师的透法。” 

“呣,说得很好。别管欧洲的大师。再往下说

“你的线条的确华丽又有力,观赏者反而宁可相信你所画的而不是真实的物品。这样,正如你能用你的才能使最虔诚的信徒放弃信仰一样,也能用一幅

\newpage
画来引导最不知悔改的不信教者走向安拉之道。” 

“确实,可是我不知道那算不算是赞美。接着
说。” 

“没有一个细密画家比你更懂得颜料的浓度和它们的秘诀。最光亮、最鲜活、最纯正的色彩都是你
调配的。” 


“好的。还有呢?” 

“你知道你是继毕萨德和密尔·赛依德·阿里
之后最伟大的画家。” 

“是的,我很清楚这点。既然你知道,却为什么还要和那庸才中的庸才黑先生一起合作书本,而不
是和我?” 

“首先,他的工作并不需要细密画家的技巧。
”我说,“其次,和你不同,他不是杀人凶手。” 

他对我甜甜地笑了笑,因为我也是马上就带着
\newpage
一种宽松的心情对他笑了。我感觉以这种态度,用风格这一话题或许能逃离这场噩梦。借着我所提起的这个主题,我们开始愉快地讨论起他手里的铜蒙古墨水瓶,不像父亲与儿子,而像两个阅历丰富的好奇老人。我们谈论着青铜的重量、墨水瓶的对称、瓶颈的深度、旧书法芦杆笔的长度,以红墨水的神秘,他还站在我面前轻轻摇晃墨水瓶,以感觉墨水的浓稠度……我们谈到,如果不是蒙古人从中国大师那儿学来了红颜料的秘密并把它引进呼罗珊、布哈拉和赫拉特,我们在伊斯坦布尔就绝对制作不出这种颜料。我们聊着,时间的浓度似乎也像颜料一样在变化着,时间在一点一点地过去。在我心底的一角,仍在疑惑着为什么
还没有人回来。真希望他放下那只沉重的墨水瓶。 

带着我们平常工作时的轻松态度,他问我:“等你的书完成后,那些见到我作品的人会赞赏我的技
巧吗?” 

“如果我们可以,真主保佑,没有阻碍地完成这本书,当然,苏丹陛下会这么拿起来看一看,首检查我们是否在适当的地方用了足够的金箔。接着,他
\newpage
会凝神观看自己的肖像,好像在阅读有关自己个性的故事。和所有的苏丹一样,他会崇拜于他自己,而不是我们精美的绘画。再者,如果他花时间欣赏我们辛勤劳苦、牺牲视力、融合了来自东方和西方的灵感创造出的丽景象,那就更好了。你也知道,如果没有奇迹现,他就会把书本锁进他的宝库,甚至不会问是谁画的边框,是谁镀的颜色,是谁画了这个人或那匹马。而我们也将如所有技艺精湛的工匠一样,继续回去
作画,只希望有一天会有奇迹降临。” 

我们静默了一会儿,仿佛都在耐心地等待着什
么。 

“这种奇迹什么时候才会出现?”他问“我们画了那么多的画,眼睛都快瞎了,但这些画什么时候才会真正得到赏识?人们什么时候才会给予我,给予
我们,应得的爱戴?” 


“永远也不会!” 


\newpage

“为什么?” 

“人们永远也不会给你所想要的,”我说,“
将来,人们对你的赏识还会更少。 

“书本会流芳百世。”他骄傲地说,但对自己
也是毫无信心。 

“相信我,没有一个意大利画家拥有你的诗意、你的执着、你的敏锐、你用色的纯粹与鲜艳,然而他们的绘画却更为令人信服,因为它们更像生命本身。他们不是从一叫拜楼的阳台上去看世界,也没有忽略所谓的远景画法。他们描绘在街上看见的景象,或是从一位贵族的房里看到的事物,包括他的床、棉被、书桌、镜子,他的老虎他的女儿以及他的钱币。他们画所有的东西,这你也知道,我并不全然信服他们的所有做法。对我而言,通过绘画来直接模拟世界是不敬的行为,我深感憎恶。然而他们用这新方法所画的图画,确实有不可否认的魅力。他们一五一十地描绘眼睛所见的事物。没错,他们画他们所见的,我们则画我们所想像的。一看他们的作品,你立刻就会明白,惟有通过法兰克风格才能让一个人的面孔永垂不
\newpage
朽。而且,不单单是威尼斯的居民迷上这个概念,整个法兰克地区所有的裁缝、屠夫、士兵、神父和杂货小贩都样……他们全都请人用这种方式画自己的肖像。只要看过那些图画一眼,你也会渴望这么看自己,你会想要相信自己与众不同,是一个独一无二的、特殊而又奇怪的有生命之物。要达到此种效果,画家不能以心灵所见的相貌来画人,而必须呈现出肉眼所见的形体,以新方法画。将来某一天,大家都会像他们那样画画。当提及‘绘画’时,全世界都会想到他们的作品!就算是一个对绘画一窍不通、愚蠢可怜的裁缝,也会想拥有这么一幅肖像,为借由看见自己独特的弯鼻,他会相信自己不是一个平凡的傻瓜,而是一
个特别的、独一无二的人。” 

“那我们也可以画那样的画。”爱开玩笑的凶
手说。 

这一次,就连我心中那不太灵光的部分也明白这不是错误,而很可是即将束我生命疯狂与愤怒。这种状况让我惊恐万分,我开始用尽力气痛苦地高声哀号。如果要画出我的号叫,那它就会是绿绿的颜色。
\newpage
然而我知道,晚的黑暗中,在空旷的街道上,没有人听得见它的嘶喊,也没有人看得见它的色彩我是孤零
零的一个人。 

他被我的哀号吓了一跳,迟疑了一会儿。刹那间我们四目相对。我可以从他的瞳孔里看出,尽管恐惧而怯懦,他仍决定听任自己的所作所为。他不再我认识的细密画大师,而是一个来自远方的、连我的话都听不明白的、坏透了的陌生人。这种感觉把我此刻的孤独延长成了几个世纪。我想抓住他的手,如同拥抱这个世界,但却没有用。我乞求,或者以为自己是开口说了:“我的孩子,我的孩子求你不要杀我。”
像是在梦中,他似乎没有听到我在说话。 


他再次拿墨水瓶砸向我的脑袋。 

我的思想,我面前的事物,我的记忆,我的眼睛,因为我的害怕而全都融合在了一起我分辨不出任何一种颜色,接着,我才明白,所有的色彩全变成了红色。我以为是血,其实是红色的墨水;我以为他手

\newpage
上的是墨水,但那才是我流个不停的鲜血。 

在这一刻死去,我而言是多么的不公平,是多么的残酷,又是多么的无情。然而,那正是我年老而血迹斑斑的脑袋慢慢带我前往的结论。接着我看见了。我的记如同外头的积雪般一片惨白。我的头在我的
口中痉挛发痛。 

现在我应该向你们描述一下我的死亡了。也许你们早就了解了这一点:死亡不是一切的结束,这是毋庸置疑的。不过,正如每本书上都提到的那样,死亡却疼痛得令人难以置信。感觉不只是我碎裂的脑壳和脑子,好像身体的各个部位都纠缠在了一起,全都融成一团,在痛苦中扭曲着。要忍受如此无止境的剧烈痛楚显得是那么的难,我内心的一部分选择了惟一
的方式——忘记疼痛,只想寻求一场甜甜的睡眠。 

临死前,我记起了自己年少时听过的一个叙利亚神话事。一个独居老人,一天半夜醒来,从床上起来倒了杯水喝。当他把杯子往茶几上放时,发现原本摆在那里的蜡烛不见了。去哪里了呢?一丝微弱的光线从房里透隙而出。他循着亮光,转身回到卧房,却
\newpage
发现有个人拿着蜡烛躺在他的床上。他问:“你是什么人?”“我是死亡。”陌生人说。老人一下子神秘地静了下来。“所以,你来了。”他接着说。“是的。”死亡满意回答。老人坚定地说:“不,你只不过是一场我没做完的梦罢了。”老人倏然吹熄陌生人手里的蜡烛,切都消失在了黑暗中。老人爬回自己的空
床,继续睡觉,然后又活了二十年。 

我知道这不会是我的命运。因为他再次拿墨水瓶狠砸了我的脑袋。剧痛难耐之中,我只是隐隐约约地感觉到了头部所受的击打。他、墨水瓶以及被烛光
微微照亮的房间现在就已经逐渐模糊远去了。 

尽管如此,我知道我还活着。因为我还想要攀附住这个世界,还想要远远地逃离,因为我的手臂膀为保护我的脸和血流如注的头还做了许多的动作,因为我好像曾一度咬住了他的手腕,因为墨水瓶还砸中
了我的脸。 

我们大概还缠斗了一会儿,如果算得上是缠斗的话。他既强壮又激动,把仰天打倒在地。他用膝盖
\newpage
压住了我的肩膀,把我紧紧地钉在了地上,一面用极为不敬的言语不停地对我这个濒死的老人说着些什么。也许因为我听不懂,也听不到他的话,也许因为我不喜欢看他那双血红的眼睛,他又狠击了我的头一次。他的脸、眼睛和身上一片艳红,沾满了墨水瓶中溅出的墨水,以及我猜想,沾满了我身上溅出的鲜血。
 

想到自己在世上最后见到的竟是这与我敌对男人,我悲伤万分地合上了眼睛。刹那间,我看见一道柔和温暖的光芒。光线舒适而诱人,如同睡眠一般,似乎可以马上化解我所有痛楚我看见光里有一个形体
,孩子气地问:“你是谁?” 

“是我,阿兹拉尔,死亡的天使。”他说,“我负责终止人们在尘世的生命旅程。我负责拆散孩子与母亲、妻子与丈夫、父亲与女儿,以及爱侣们。世
上没有一个人躲得了我。” 


当我明白死亡不可避免时,我哭了起来。 

\newpage

我的眼泪使我口渴万分。一边是我满是鲜血的面孔和眼睛感觉到的越来越剧烈的令人麻木的疼痛;另一边,是一个疯狂与残酷都将终结的地方,然而那个地方对我来说很陌生也很恐怖。我知道它是光亮之地,亡者的国度,是阿兹拉尔召唤我前往的地方,因而我很害怕。但另一方面,我也明白自己无法久留于这个让我痛苦得扭动哀号的世界,在这充满骇人痛楚与折磨的尘世,已没我的立足之地了。若要留下来,我必须忍受这可怕的痛楚,而这却不是我这老迈的身
躯可以做到的。 

因此,临死之前,我的确渴望死亡的到来。与此同时,我也立刻明白了自己一生在书里都没找到的答案,也明白了人们为什么无一例外地都能成功地死去,原来都只是由于这种简单的欲望。我也明白了死
亡将使我变得更有智慧。 

话虽这么说,但我满犹豫,就像一个即将远行的人,克制不了自己想再看一眼他的房、他的物品、他的家。惊惶中我渴望再见女儿最后一面。我真的好想好想,甚至知道只要咬紧牙关,忍受痛及愈来愈迫
\newpage
切的口渴,再撑久一点,就一定能等到谢库瑞回来。

于是,我面前致命而温和的光芒略微暗淡了些,我的心打开来,倾听我躺着死去的世界里的各种声响。我听见我的凶手在房游荡,开柜子、翻我的纸张,专心找寻最后一幅画,当他发现无所获后,我听见他掀开我的颜料箱,踢倒柜子、盒子、墨水瓶和作桌。我感觉到自己不时发出呻吟,苍老的手臂和疲倦的
双腿偶尔不自觉地抽搐。我等待着。 

我的疼痛丝毫没有减轻的迹象。我越来越渴,再也没有力气咬紧牙关。但是,我继续撑着,等待着

接着我突然想到如果谢库瑞回家,她可能会遇见卑鄙的凶手。这一点我本连想都不愿意去想。这时候,我感觉到杀我的凶手离开了房间。他大概找到了
最后一幅画。 

我剧渴难耐但仍然等待着。来吧,亲爱的女儿
,我美丽的谢库瑞,快来吧。 

\newpage


她没有出现。 

我再也没有力气承受折磨了。我知道死前将见不到我女儿最后一面了。这锥心刺骨的悲伤让我想哀痛而死。正在此时,一张我没见过的面孔出现在左侧
,微笑着,善意地递给了我一杯水。 


我忘记了一切,贪婪地伸手想取水。 

他缩手拿回水杯。“承认先知穆罕默德是个骗
子,”他说,“否定他说过的一切。” 

是撒旦。我没有回答,我甚至一点也怕他。既然从来不相信绘画等于被他愚弄,我满怀自信地等待
着。我梦想着前方的永恒旅程,以及我的未来。 

这时候,刚才看见的光亮天使朝我接近,撒旦消失了。我的一部分脑子明白这位赶跑撒旦的光亮天使是阿兹拉尔,但心中叛逆的一部分则想起《末日之书》中写道,阿兹拉是一位天使,他拥有一千只翅膀,覆盖着东方和西方,整个世界都在他的掌控之中。
\newpage


正当我愈来愈感到困惑时,沐浴在光芒中的天使朝我靠近,仿佛想帮助我是的,就如葛萨利在《壮
丽瑰宝》中写的那样,他和地说: 


“张开嘴,让你的灵魂得以离去。” 

“除了‘奉真主之名’这一祷文之外,我不会
让任何东西离开嘴巴。”我回答他。 

这不过是最后一个借口。我知道自己再也抗拒不了,我的时辰已到。有那么一刹那,我到相当难堪,想到不得不把死状凄惨、丑陋血污的尸体留给我再也见不着的女儿。但我只想离开这个世界,就像抛开
一件紧绷的外衣一样。 

我张开嘴,陡然间,就像描绘我们的先知拜访天堂的升天之旅的各种图画中所描绘的一样,所有的东西都变得色彩斑斓,一切都淹没于璀璨缤纷之中,好似奢侈地镀上了各种金亮的涂料痛苦的眼泪从我眼中滑落,艰难的最后一口气从肺部和口中溢出一切都
\newpage

沉浸在了神秘的寂静之中。 

现在我能看见自己的灵魂轻轻地脱离了躯体,被捧在阿兹拉尔的手心里。我蜜蜂般大小的灵魂沐浴在光芒之中,因为离开躯体时的颤动,它现在仍像水银般在阿兹拉尔的掌心中微微震动。然而我并不太注
意这点,思绪沉浸于我所来到的崭新的陌生世界。 

度的痛苦过后,我的内心充满了平静。死亡并没有像我所害怕的那样给我带来疼痛,相反,我变得舒服了,很快明了此刻的状态将恒久持续,而我活着的时候所感觉到的那种压迫束缚只是暂时的从今以后,都会是这样,百年复百年,直到世界末日。我既没有为此感到沮丧,也没有为此感到高兴。我过去短暂经历过的事件,如今一件接一件,同时展开呈现在了广袤无垠的空间。现在,所有的事情都同时在发生着,就好像一位爱开玩笑的细密画家在一幅巨大的双页图画中的各个角落里画上了各种互不相关的事物一样

\end{document}
