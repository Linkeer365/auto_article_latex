\documentclass{article}
\usepackage[utf8]{inputenc}
\usepackage{ctex}

\title{予我\footnote{Click to View:\url{https://web.archive.org/web/20220629010802/https://book.douban.com/subject/3064256/discussion/1202118/}}}
\author{薛卫民}
\date{1996}

% \setCJKmainfont[BoldFont = Noto Sans CJK SC]{Noto Serif CJK SC}
% \setCJKsansfont{Noto Sans CJK SC}
% \setCJKfamilyfont{zhsong}{Noto Serif CJK SC}
% \setCJKfamilyfont{zhhei}{Noto Sans CJK SC}
% \setlength\parindent{0pt}

\begin{document}
\CJKfamily{zhkai}

\maketitle

\setlength\parindent{0pt}


\Large

1、予我
\\
予我茫然四顾无所不在的天高地阔…… \\ 


掠世的浩浩长风\\
鼓荡滔滔的不羁之水汇成江河\\
予我撕裂旷野的奔涌淹向七月的流火
\\
予我五千年大雨中滂沱的雩歌 \\ 


广袤的黑土黄土红土\\
隆起宿命的原色\\
予我震动大地的鼓乐逐向天边的漂泊\\
予我五千年尘土飞扬中的情歌\\
予我无径抵达的新陆\\
予我彼岸椰林的婆娑\\
予我千重波涛下\\
\newpage

百劫依存的巨舸\\
予我嘹亮远方的雄鸡之展歌\\
予我隔世的梅朵\\
在伸手可及的溪中洗濯\\
予我伸出典籍的苤苢
\\
重泛一季灿灿的绿色 \\ 


2、永无宁日之日出\\
漆红凿满谶语的列列岩壁\\
永无宁夜之月落
\\
倾倒江海斟注高攀的盟榼 \\ 


予我历历可见\\
望穿石人之眼的逝者\\
逝者携长萧而去\\
长萧不作悲声
\\
长萧吹青青的竹林万旌齐扯 \\ 


予我劈面而遇\\
撞痛嫣红天宇的来者\\
来者挥雁而至\\
\newpage

雁行不栖洲渚
\\
雁行拍喑哑的涸谷为泽 \\ 


予我遥遥仙境偏不遥遥\\
仙境生寻常一株丝柳\\
予我匆匆一瞬即为永恒
\\
叠起的峰峦涌动为波 \\ 


予我渐行渐远的足音\\
予我赫然重现的车辙\\
予我萧萧马嘶逐逃匿的驿站
\\
予我夕阳被自身的光芒刺破 \\ 


3、予我惊惊\\
予我瑟瑟\\
予我赤裸\\
予我无遮\\
予我谶语
\\
予我盟榼 \\ 


予我望穿石人之眼的逝者
\newpage

\\
予我撞痛嫣红天宇的来者…… \\ 


4、予我森林\\
予我寂寞千年的森林
\\
予我躁动千年的森林 \\ 


承周天寒彻,有万千大雪飞来\\
万千雪片\\
都做不成请柬\\
天上的云走地上的风走\\
惟有森林不走\\
悬柄柄冰剑——那是固体的泪流\\
寸寸相逼
\\
逼之所向皆指独自的胸口 \\ 


融时已是橙黄的树汁\\
滴滴溢自皲裂——那是众蚁之路\\
飞不出九月的蜜蜂\\
自圆其内成为琥珀去酿\\
甘甜岁月之蜜\\
于是田地间始有
\newpage

\\
永久 \\ 


予我森林!\\
予我前年寂寞\\
将千年寂寞成厚土黝黝\\
予我千年躁动
\\
将千年躁动成云飞风吼 \\ 


积厚土黝黝落秋叶在地\\
扬云飞风吼生新叶在头\\
更更替替\\
均是苦难的历程均有快乐的节奏\\
生生息息
\\
均掩于无声之中均溢于蓬勃之后 \\ 


5、予我翰海\\
予我翰海浩浩汤汤围我无水
\\
予我潮汐刚刚逝去唤我有声 \\ 


涌至无涯处仍极目无涯\\
遥遥无始逐向遥遥无终……\\
\newpage

那不可遏止的汹涌\\
浮幽魂为岛屿\\
那涵日溶月的平静
\\
沉岁月为石钟 \\ 


总有拍岸之声拍得天动地动\\
总有撞陆之声撞起叠叠山峰\\
而天虽动地虽动\\
路径仍在向远行\\
而山虽起峰虽起
\\
花上的蝴蝶不惊 \\ 


让我在极乐时沉重\\
让我在炼狱里轻松\\
让我听见\\
砾石之痕声声呼痛
\\
黄昏却呈一片娩后的安宁…… \\ 


予我翰海!\\
我无阔以盛\\
翰海浩浩汤汤之水\\
\newpage

我无籁以应
\\
潮汐逝去唤我之声 \\ 


只在迷茫中\\
瞩望漂移之岛屿举月为灯\\
只有孤寂时
\\
倾听沉默之石钟旷世永鸣 \\ 


6、予我咸湖!\\
予我咸湖盈盈居于高原\\
与冰山雪原之水相呼\\
予我咸湖不逐轻浮而歌之水\\
四通八达时
\\
宁择无路—— \\ 


积多少众神之泪\\
积多少苍生之苦\\
积多少不泻之盐
\\
积多少不释之卤 \\ 


众神之泪\\
\newpage

灌溉着湖畔不知如何而哭之草\\
苍生之苦\\
浸润着远方不知如何而忧之木\\
不泻之盐\\
已溶于挥湖而去的过客之血\\
不释之卤
\\
已泛上逶迤它方的旅人之途 \\ 


予我咸湖!\\
予我咸湖浩渺中一片蔚蓝如初\\
予我至洁至纯之素\\
予我朗日下一群天鹅踏虹而舞
\\
予我至欢至乐之故 \\ 


7、予我天空\\
予我不敢直视的太阳
\\
和那些婴儿般高贵的面庞 \\ 


予我天空\\
予我夤夜沐浴灵魂的月光\\
予我幸福的宁静
\newpage

\\
荫庇一隅中的安详 \\ 


予我毫不矜持的高高在上! \\ 


予我复苏的羞耻\\
予我跪满忏悔者的天堂\\
予我莹莹的泪水\\
予我祈愿\\
予我蜕落于传说中的翅膀\\
予我不可或缺的虔诚\\
并为此让我学会
\\
仰望 \\ 


8、予我大地\\
予我鸡鸣犬吠环护着的家园
\\
予我随风飘曳的炊烟 \\ 


予我大地\\
予我生生息息回黄转绿的梦幻
\\
予我周而复始的辽远 \par \\
\newpage

一滴闪烁的露水\\
何以会悬垂千年\\
一穗陨落的籽粒
\\
何以会代代繁衍 \\ 


予我无法遗忘的童话\\
讲述童话的声音\\
使用着不断成长的语言——\\
一个人追山\\
山永远在他的前边\\
他撞上的山都不是山他撞上的总是他投宿的那家客店的门板……

\end{document}
