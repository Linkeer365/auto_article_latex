\documentclass{article}
\usepackage[utf8]{inputenc}
\usepackage{ctex}

\title{福昌婆婆\footnote{Click to View:\url{https://web.archive.org/web/20230516122624/http://old.zh61wx.com/rdtj/Class354/13349.html}}}
\author{马昇嘉}
\date{2009-03}

% \setCJKmainfont[BoldFont = Noto Sans CJK SC]{Noto Serif CJK SC}
% \setCJKsansfont{Noto Sans CJK SC}
% \setCJKfamilyfont{zhsong}{Noto Serif CJK SC}
% \setCJKfamilyfont{zhhei}{Noto Sans CJK SC}
% \setlength\parindent{0pt}

\begin{document}
\CJKfamily{zhkai}

\maketitle


\Large


引子 

听母亲说,福昌婆婆与我祖母年轻时就是要好
的姐妹。 

我出生那年,祖母得病过早离开了人世,福昌婆婆一如既往,依然是我们家的常客。福昌婆婆对我尤为钟爱,每次来家总要抱抱我亲亲我,她系在腰间的藏青色布围兜,仿佛是一只神奇的魔术袋,里边有
取不完的糖果、糕点、山芋、乌菱…… 

那时我不会开口说话,但只要福昌婆婆来了,一听到她的大嗓门,我就会兴奋异常,瞪大眼睛四处寻找。一旦见了她,我嘴里发出"啊、啊"的叫声,在木立桶里蹦啊跳的。福昌婆婆满脸的皱纹绽开成一
\newpage
朵好看的秋菊,嘴里说着:"来,抱抱,乖囝——"于是,我张开两臂向她扑去。福昌婆婆让我坐在她膝盖上,从围兜里摸出零食细心地喂我。吃着吃着,福昌婆婆突然觉得腿上一阵热辣辣的——哈,是我把她裤子尿湿了。母亲忙拿干净尿布为她擦拭,福昌婆婆褪褪裤管上的尿水,笑着说:"没事,没事,小囝水
,当酒水。" 

尽管我没了祖母,但自幼在我的心目中,福昌
婆婆就是我的祖母。 

这种美好的感觉,一直保持到我10岁那年—
— 


(一) 

春天来了,小草发芽,柳树报青,桃花灼灼地
盛开着…… 

那晚,我在睡梦中,楼下一阵急促的敲门声把

\newpage
我惊醒。 


"嘭、嘭——" 


"嘭嘭嘭……" 

深夜,幽静的石板街上,敲门声显得那样惊心
动魄,我家的小木楼仿佛也在微微的颤抖。 

敲门声一阵紧似一阵,伴随着一个女人凄厉、
粗哑的哭声: 


"呜……" 


"啊……" 

我的心一阵战栗——"天哪,出什么事了?……"身子不由得直往被窝里钻,耳朵却支棱着,屏息
敛气,听着楼下的动静。 

屋里一片漆黑,我听到妈妈轻轻叹了口气,说

\newpage
:"真可怜啊!……" 


爸爸担忧地说:"怎么又犯病了?" 

听口气,爸爸妈妈知道楼下的女人是谁。"是
敲我家的门?——"我紧张地问。 

"不是。小孩家不用管,快睡觉。"爸爸说。
 

听说与我家无关,我稍稍松了口气,不一会又迷迷糊糊地睡过去了。第二天早上醒来,竟然把昨晚
发生的事忘得一干二净,而大人们也没再提起过。 

整整一个星期,没见到福昌婆婆,我有点儿想
她。 


我问母亲:"福昌婆婆好久没来了?" 

母亲说:"福昌婆婆生病了。"我要求母亲带我去看望福昌婆婆。母亲说,"已经去看过了,病情

\newpage
好了许多,过几天就会来我家。" 

果然,那天放学回家,我一进门就听到福昌婆婆熟悉的说笑声,大大的嗓门,爽朗的笑语。见了我,福昌婆婆依旧满脸笑容,指着桌上一只白色搪瓷杯
子说:"伟伟,快来吃青团子!" 

阳春三月,是江南水乡吃青团子的时节。从田野里挖来浆麦草,洗净捣碎,滤出青汁,糅进糯粉,以枣泥、豆沙、白糖、猪油为馅,在笼屉里蒸熟后晶莹碧绿,十分诱人。福昌婆婆每年逢这时节,总会做了青团子给我送来,她做的青团子水分、火候把握得
好,又糯又软,清香可口。 

"伟伟,福昌婆婆今天特地去乡下挖浆麦草,磨粉做青团子,给你送来了。"妈妈说话的神情,非
常过意不去。 

福昌婆婆用筷子夹了一个青团子给我,还没送进嘴,我就闻到一股扑鼻的清香。我张嘴咬了一口,
一股糖水溢满了嘴巴,甜得直沁心脾。 

\newpage


"好吃吧?"福昌婆婆问。 

"好吃,好吃。"我大口咀嚼着,含糊不清地回答。这时我才注意到,福昌婆婆脸色苍白,确实比
先前瘦了许多,两只眼袋下垂得更厉害了。 


"福昌婆婆,前几天你生病了?"我问。 

福昌婆婆脸色似乎有点儿尴尬,她用手摸着满是褶皱的脸,笑了笑说:"没事,没事,伤风感冒—
—" 

妈妈打断了我的话,把搪瓷杯子向我挪了挪,
说:"伟伟,趁热,再吃一个吧。" 

这时,福昌婆婆站起身,说:"时间不早,我
回去了。" 

我和妈妈把福昌婆婆送到门口,看着她的身影
渐渐消失在石板街尽头。 

\newpage


(二) 

时隔不久的又一个夜晚,我再次被敲门声、哭号声惊醒。我听辨出,敲门声比上回猛烈,而且用的
是铁器。 

"你这杀千刀——"粗哑的嗓音拉着哭腔,夹杂着"咣、咣、咣——"的铁器撞击木板门的声音。我听出了,那是菜刀的声音,在敲对面梅婶家的门。


"杀人啊?——"我忍不住喊了起来。 


"不要声响!"爸爸发出低沉的声音。 

"我,我害怕……"由于惊慌,我说话有点儿
急巴。 

"小孩家别多管,快睡觉!"爸爸还是那句话
,声音闷闷的。 

我听着一阵阵的敲门声、哭号声,这回却再也
\newpage
不能入睡。我脑海里思索着,想象着,楼下的女人会是谁?听声音似乎有点儿熟悉,可是一时却想不起来。她为什么要敲梅婶家的门?与她家有什么刻骨仇恨
?…… 

这时,楼下响起一个男人瓮声瓮气的声音,是
在劝那女人回家。 

过了一阵,哭声、骂声渐渐远去,石板街恢复
了宁静。 

"唉,"妈妈叹了口气,"一辈子的痛啊……
" 

"这次毛病发得更厉害了!"爸爸忧心忡忡地
说。 

看样子爸爸和妈妈非但对这个女人熟悉,而且
充满了同情。 

"那个女人到底是谁呀?"我冷不丁冒出这句
\newpage

话来。 


"怎么,你还没睡着?"妈妈觉得奇怪。 

"我睡不着,挺吓人的!"说着,我又问,"
她为什么一直半夜三更来吵闹?" 

"唉……"黑暗中,妈妈又是一声长长的叹息
。 


(三) 

那天放学,我背着书包走在石板街上,稻香村糖果店里飘溢着诱人的香味,门口架着一只圆筒形的铁炉子,大铁锅里装着沙子和栗子,一个工人手持一柄大铁铲,正在不停地翻炒栗子,发出"嚓、嚓"的声响,吸引了不少路人观看。我不敢停留,怕禁不住
诱惑,而口袋里没有分文。 

路过福昌婆婆家的酱酒店时,只见福昌婆婆站在店门口,正在向我招手,大声喊道:"伟伟,过来
\newpage

,过来!" 

"福昌婆婆——"我亲热地叫了一声,走上前
去,"有事吗?" 

福昌婆婆从布围兜里拿出一包东西,用纸袋装着。我拿在手里,纸袋热呼呼的,还闻到一股熟悉的
香味。 


"糖炒栗子!"我高兴地说。 


"快拿回家去。"福昌婆婆叮嘱着。 


我谢过福昌婆婆,兴冲冲往家走。 

走出没几步,背后突然被人搡了一下,回头一看是同学阿海。他瞪大眼睛,像看着一个陌生人,说
:"你,你刚才叫她什么?" 

阿海是班上皮大王,平时我不跟他玩。我扭过

\newpage
头,只管走自己的路。 

"你,你怎么叫她福昌婆婆?"阿海大惊小怪


"关你什么事?"我没好气地说。 

"她是疯老太婆呀!"阿海夸张地叫了起来。

我几乎没加思考,立马回击道:"你才是疯子
呢!" 

"真的,我不骗你,她是疯老太婆……你,你竟然叫她福昌婆婆——"阿海边走,边嘀咕着,一脸
不屑的神情。 

我心头升起了一股怒火,一把抓住阿海的臂膀
:"不准你说!" 

"我偏说,你还吃疯老太婆的东西!"阿海两
眼盯着我手中的纸包,仿佛抓住了什么罪证。 


\newpage

"你再说一遍!"我下了最后通牒。 

"就说,就说,你吃疯老太婆的东西!"阿海
一脸蛮横。 

话未落音,我"啪"地当胸砸了他一拳。阿海没有料到我会真的下手,脚下没有站稳,踉跄着往后退了两步,随即举起拳头,扑了上来。我一侧身,灵巧地躲过了他的拳头,可是左手托着的一包糖炒栗子
,不幸被他击中,顿时哗啦啦全撒在了石板街上。 


"赔我的栗子!"我一把揪住阿海的衣服。 

石板街上的街坊邻居见了,纷纷前来劝架,帮
着把栗子一颗颗捡拾起来,装进我的书包。 


阿海自知理亏,悄悄地溜走了。 

回到家,我气呼呼地把书包里的栗子一股脑儿倒在桌子上,有几颗骨碌碌滚到地上。妈妈见了,问
是怎么回事,我把与阿海吵架的事说了。 

\newpage

妈妈听了,责怪道:"同学之间不应该动手打
架!" 


"谁叫他骂福昌婆婆!"我气愤地说。 

"骂人是他的错,可是……"妈妈欲言又止。

"可是什么?——"我急切地问道,预感到某
种不测。 


妈妈叹了口气,弯下腰,拾地上的栗子。 

"妈妈,告诉我,到底是怎么回事啊?"我苦
苦央求着。 


…… 


(四) 

我怎么也不敢相信,石板街上夜半闹事的女人

\newpage
竟然是福昌婆婆。 

福昌婆婆一家是徽州人,幼时母亲早亡,随父亲来到江南茜浦镇,开了一家酱酒店。我祖母去店里买油盐酱醋,福昌婆婆热心,常常会帮忙送至家中,一来二往俩人便成了要好姐妹。福昌婆婆长得并不漂亮,粗手大脚,干活倒是一把好手,店里脏活重活抢着干,深得她父亲喜爱。转眼到了婚嫁年龄,无人前来提亲,父亲四处托媒无果。我祖母见乡下亲戚阿奎年纪相仿,只是家境贫困,人倒老实能干,便说媒招为入赘女婿。一年后俩人生下一子,取名强强。强强自幼受到家庭宠爱,游手好闲,不求上进。成年后,福昌婆婆为儿子娶了媳妇,在我家对门置了酱酒店,交与强强管理,期望能去邪归正,自食其力。然而事与愿违,强强染上赌博恶习,父母、祖父规劝不听,媳妇说他,常遭拳脚相加,以致最后把一个好端端的
酱酒店抵了赌债,媳妇愤而离去…… 

人财两空,强强祖父年老体衰,急火攻心,一命呜呼。福昌婆婆经受不住打击,导致精神错乱,我祖母帮衬阿奎将福昌婆婆送进医院。半个月后福昌婆婆出院,强强已不知去向,有说他无颜见父母,去了
\newpage
徽州老家,也有说是与人结伴去了新疆谋生。阿奎劝
说媳妇,就当从未有过这个不孝之子。 

福昌婆婆身体恢复健康,但心头的痛却解除不了。每年春天来临,世间万物苏醒,这病魔仿佛也会准时醒来,一俟发作,福昌婆婆就会在夜深人静之时哭闹敲门,痛惜失去的家业……对门梅婶,从别人手里买下房屋,开了一家百货店。梅婶同情福昌婆婆,
事后奎公公上门道歉,她从不计较。 


(五) 

自当知道福昌婆婆犯有精神病,在我头脑里出现的总是披头散发、手持菜刀、龇牙咧嘴、面目狰狞的形象,从此便心生惧怕。街上见了福昌婆婆,我便远远避之,绕道行走;福昌婆婆来我家,我一定躲到楼上,不再照面;她带来好吃的东西,我不再嘴馋,
而且心里常常会生出一种厌恶。 


可是,命运会捉弄人。 

\newpage

那年冬天,我父母去县城探望生病的姑母,早上乘小火轮进城,傍晚才能到家,午饭让我去福昌婆婆家。想到要与福昌婆婆一起吃饭,我顿时头皮发麻
,心中一阵慌乱。 

"不,我不去,我要在家里吃!"我向妈妈提
出。 

"已经跟福昌婆婆说好了,听说你要去,她高
兴着呢!"妈妈早做了安排,无奈,我只得顺从。 

福昌婆婆家,前屋是店面,旁边一条长长的陪弄直通后屋,我跟妈妈来过多次,熟门熟路。可是,今天当我迈进陪弄时,心里竟然忐忑不安起来,脚步格外地沉重。我明白,福昌婆婆是喜欢我的,直至今日还是把我当作她的亲孙儿一般疼爱,可是,我再也
找不到往日的感觉了。 

陪弄幽深而又寂静,我听得见自己"怦、怦"
的心跳…… 

\newpage

福昌婆婆在厨房灶台上忙碌着,我倚在门口,陌生地盯着福昌婆婆有点儿驼起的背影。福昌婆婆转过身,见了我,惊喜地招呼:"伟伟,肚子饿了吧?
快进来,吃饭!" 

厨房里弥漫着一股诱人的香味,福昌婆婆从铁锅里盛出红烧肉,在稻草编结的焐窠里盛出热气腾腾的米饭,从煤球炉上端来一锅鸡蛋汤,还有其它三四盘菜,很是丰盛。大冷的天,饥肠辘辘,热饭热菜很合我胃口,我急不可耐,端起碗埋头吃了起来。福昌婆婆坐在我对面,慢慢咀嚼着,眯缝着眼睛看着笑着,不时往我碗里夹菜,说我正在长身体,应该多吃一
点。 

福昌婆婆的男人,我叫他奎公公,他多半时间一直守在店堂里做生意。此刻我与福昌婆婆面对面坐着,没有多余的话语,厨房里十分静谧,听得见咀嚼饭菜的声音。吃着吃着,我开始不自在起来,特别当我抬起头来,目光正好与福昌婆婆黯淡的目光对接时,我的心里就会引起一阵不安,后背上热辣辣的,仿

\newpage
佛有针刺的感觉。 

平时喜爱的菜肴,今天吃在嘴里却辨不出滋味,我胡乱往嘴里划拉着饭粒。见我快吃完了,福昌婆
婆放下饭碗,要为我添饭,我说吃不下了。 


不,即使能吃,我也绝对不想吃了。 

这时,福昌婆婆从刀架上取下一把菜刀——看她手持菜刀的模样,我心中突然涌起一种莫名的恐慌,我想起夜半石板街上的敲门声,福昌婆婆哭哭啼啼举着菜刀……我浑身一个激灵,一股恐惧紧紧攫住了我的心。我再也坐不住了,放下饭碗,顾不得扒完最后一口饭,说声"福昌婆婆再见",拔脚就往外溜。

身后响起福昌婆婆的喊声:"伟伟,我削水果
你吃啊——" 

我一口气冲出陪弄,走在石板街上。冬日的阳光当头照着,身上感到了一股暖意,我长长嘘了口气,心中顿觉舒畅了许多。我暗暗庆幸,明天不用再来

\newpage
受罪了。 

晚上,爸爸妈妈告诉我,明天姑母要动手术,他们还得去县城医院,让我继续去福昌婆婆家吃午饭

天哪,怎么会是这样呢?我的头皮一阵发麻。然而,我没有吱声,心里想好了,明天坚决不去福昌
婆婆家了。 


( 六 ) 

第二天早饭过后,上学路上我买了两个包子,
放在书包里,作为午餐。 

中午放学的时候,同学们都回家去了。教室窗外西北风在呼号,刮得老槐树光秃秃的枝丫不停地颤悠。教室里空落落的,剩下我一个人,真有一种饥寒交迫的感觉。我从书包里掏出包子,啃了一口,包子冰凉凉的,有点儿硬,很难下咽,跟刚出笼的包子完全是两种味儿。这时,我想起了福昌婆婆,想起了福昌婆婆家充蕴着饭菜香味的厨房,这会儿她肯定在等

\newpage
待我吃饭呢…… 

我好不容易嚼完了一个包子,只见传达室季老伯走进教室来。他手里拿着一个青颜色的棉套子,里面包裹着一只饭盒子,他告诉我是一位老太太叫送来
的。 


不用说,肯定是福昌婆婆给我送饭来了。 

棉套子包裹着的饭盒子暖暖的,沉甸甸的,揭开盖子,啊,喷香扑鼻,是我最爱吃的油煎带鱼,饭和菜都冒着热气。因为吃了冷包子,此刻我肚子里也
是冰凉凉的,不由得来了食欲。 

这时,阿海和几个男同学突然闯了进来,真搞不懂他们回家吃饭竟会这么神速。我赶紧盖上盒盖,
把饭盒塞进棉套子。 

"伟伟,疯老太婆给你送饭来了!"阿海大呼
小叫直嚷嚷。 


\newpage

"疯老太婆是你什么人?" 

"哈哈,吃了她的东西,当心也变成疯子!"


以阿海为首的这几个同学,在班上成事不足,败事有余,老师常常会觉得头疼。我听不惯他们对我的嘲弄,也不愿跟他们多费口舌。我拿定主意,捧起
饭盒走出教室。 

我来到教学楼旁的小河边,河水随风泛着一道
道波纹,由北往南潺潺地流淌着。 

阿海他们跟随着我,在不远处叽叽喳喳议论着
,观望着…… 

我拿出饭盒,揭开盖子,金黄的带鱼继续散发着诱人的香味,我咽了下口水,双眼漠然地望着天空
。随即,我毅然举起饭盒,往河里倒去。 

"嗵——"的一声,饭菜全都撂在了河里——看着水滩边小鱼儿在争抢啄食,心中生出一种如释重
\newpage
负的感觉。我明白,不是怕吃了福昌婆婆的东西"会
变成疯子",而是怕阿海他们日后的无事生非。 

然而,意想不到的是,就在我掉过头去的瞬间,我竟然看到了福昌婆婆!寒风中,福昌婆婆佝偻着
背,正站在传达室墙角边,朝我怔怔地望着。 

我一下惊呆了,手拿着空饭盒,木然地站着。

福昌婆婆很快转过身去,脚步蹒跚地走出了校
门…… 


( 七 ) 


福昌婆婆病了,再次发病了! 

这是妈妈当天晚上去送还饭盒时知道的。妈妈问了我没去福昌婆婆家吃饭的原因,我只得推说来不及完成作业。妈妈告诉我,奎公公说,福昌婆婆这两天起大早去水产公司排队买带鱼,第一天快轮到她时货卖完了,今天又去买,可能受寒了。妈妈说,为了
\newpage

我喜欢吃的带鱼,大冷天的,福昌婆婆受累了。 

听了妈妈的话,我心里久久不能平静。福昌婆婆为我买带鱼受寒犯病,可是,我知道福昌婆婆犯病的真正原因!……这话我不能对妈妈说,只能暗自藏
在心底,隐隐作痛。 

那一晚,格外寒冷,北风呼啸着从窗户、瓦楞里一股脑儿地往小木楼里钻,我和爸爸妈妈早早躺进
被窝睡觉了。 

夜半,我再次被惊心动魄的敲门声吵醒,听着一声声哀伤、凄厉的哭泣,眼前即刻闪现出福昌婆婆瘦削的肩头、佝偻的身影,我的心里浸染着一种伤痛
和惶惑,并且伴随着深深的自责…… 


"福昌婆婆!"黑暗中,我发出一声惊叫。 


"伟伟,怎么啦?"妈妈急忙问道。 

"你,你们为什么不去劝阻福昌婆婆,大冷的
\newpage
天,她要冻坏的啊!"我拉亮了灯,几乎哀求着说。

"唉,没用,发病的时候,她不认人的。"妈妈无奈地叹了口气,"等会儿奎公公会来领她回家。

"大冷天发病可是第一回啊!……"爸爸担心
地说。 

楼下,石板街上的呜咽声,在这寒冷的夜晚,
显得更加凄凄厉厉,揪人心肺。 

福昌婆婆的哭声变成了呻吟,时断时续,时隐
时现…… 


可是,不知怎的,奎公公没有来。 

"不行,福昌婆婆的病会加重的!"不知哪来的勇气,我一骨碌翻身起床,说,"我把她送回家去
!"我穿好衣服,"噔、噔"地冲下楼去。 

打开屋门,一股寒风袭来,我浑身禁不住一阵
\newpage
哆嗦。借着路灯的一丝光亮,只见上街沿台阶上蜷伏
着一个瘦削的身影,正发出哼哼唧唧的声音。 


哦,福昌婆婆!我赶紧迈出家门。 

这时,一束手电光照射过来。"慢,伟伟,我
去看看——"怕有什么意外,爸爸把我叫住了。 

爸爸来到福昌婆婆身边,弯下腰,轻声喊道:
"福昌婶,福昌婶——"接着伸手扶起福昌婆婆。 

福昌婆婆果然不认我爸,她一甩手,爸爸没提
防,一个趔趄,差点摔倒。 

"福昌婆婆、福昌婆婆——"我拉住她一条胳臂,一叠声喊着。福昌婆婆身上穿着单衣,寒风中,
身子在簌簌发抖。 

说也奇怪,福昌婆婆似乎辨出了我的声音,黑
暗中慢慢抬起头来。 

\newpage

我紧紧攥住了福昌婆婆冰冷的手。她不再哭泣
,不再吵闹,渐渐平静下来…… 

石板街上,灌满了凛厉的北风,摇曳着寒冷的
灯光…… 


爸爸脱下棉衣,披在福昌婆婆身上。 

"福昌婆婆,走吧,我们回家去——"我与爸
爸搀扶起福昌婆婆,可是福昌婆婆却迈不开步来。 

爸爸蹲下身,背起福昌婆婆,匆匆往北街走去
…… 


福昌婆婆昏迷不醒,连夜送进镇医院抢救。 


爸爸和奎公公一起守护在福昌婆婆身边。 

谁也没有想到的是,天快亮时,福昌婆婆竟然
咽了气。 

\newpage

听到噩耗,我和妈妈急急往医院赶去,泪水止
不住哗哗地往下淌…… 

福昌婆婆走了,走得太匆忙了!我多么希望再能听一听她的大嗓门,再能吃到她做的青团子,甚至
愿意听到她夜半来敲击我们家的木板门…… 

福昌婆婆走了,我知道,她是带着对亲情的渴
望,带着深深的遗憾离开这个世界的! 

我懊悔莫迭……

\end{document}
