\documentclass{article}
\usepackage[utf8]{inputenc}
\usepackage{ctex}

\title{云生的饭\footnote{Click to View:\url{https://web.archive.org/web/20230411134151/https://paste.ubuntu.com/p/Y9WQwbDP54/}}}
\author{田芬芳}
\date{}

% \setCJKmainfont[BoldFont = Noto Sans CJK SC]{Noto Serif CJK SC}
% \setCJKsansfont{Noto Sans CJK SC}
% \setCJKfamilyfont{zhsong}{Noto Serif CJK SC}
% \setCJKfamilyfont{zhhei}{Noto Sans CJK SC}
% \setlength\parindent{0pt}

\begin{document}
\CJKfamily{zhkai}

\maketitle


\Large

“云生,饭!”每天中午,云生总能在座位
上拿到同学们给他取回的饭盒。 

云生是个弱智生:大脑袋,细脖子,个子比我矮大半头。他搞不清自己的饭盒,我们在他的饭盒上做了记号,由我们几个班干部每天轮流给他取送,我
被排在星期一。 

云生拿到饭,总爱笑笑说:“谢谢大哥哥!”或“谢谢大姐姐!”不管年龄大小,一律这么叫。每
天中午,他总能吃到许多大家夹给他的小菜。 

云生上课很守纪律,从不吵,总是写着自己也
看不清的字。 

\newpage

一天上美术课,云生削铅笔,把手指划破了,那是把生了锈的小刀。他没有哭,用破布包住了伤口
。第二天,他照常来上学。 

不久,云生的手指发炎了,大家也不知道,只
看到他手指上还是包着布头。 

星期一,我又去给云生拿饭,发现饭盒很轻。他本来只有我一半的饭量,这次更少了。我说:“云生,你要成仙了,吃这么一点儿饭?”云生无精打采地说:“我不想吃。”我一摸他脑门儿,很烫,马上告诉了班主任老师。一会儿,云生的爸爸来了,把他
背了回去,匆忙中忘了拿饭盒。 

云生真的“成仙”了,他因得了破伤风,耽误了医治,死了。星期天,我们去云生家悼念他,老师
也去了。 

星期一,我去食堂拿饭盒,怎么也找不到云生的饭盒,猛然想到他已经不在了,就默默地回到了教室。我找出云生的饭盒,把一大团饭拨进他的饭盒,
\newpage
又夹了菜。同学们都围上来了,给他拨了满满一盒饭。我站在后面说:“云生,你吃吧……”许多同学都
哭了,我们失去了一个多听话的小弟弟呀! 

愿云生到有人给他拿饭的地方去吧!

\end{document}
