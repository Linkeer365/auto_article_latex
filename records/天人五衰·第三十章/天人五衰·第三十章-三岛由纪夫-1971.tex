\documentclass{article}
\usepackage[utf8]{inputenc}
\usepackage{ctex}

\title{天人五衰·第三十章\footnote{Click to View:\url{https://web.archive.org/web/20180331070522/https://www.kanunu8.com/files/world/201104/2648/63240.html}}}
\author{三岛由纪夫}
\date{1971-01}

% \setCJKmainfont[BoldFont = Noto Sans CJK SC]{Noto Serif CJK SC}
% \setCJKsansfont{Noto Sans CJK SC}
% \setCJKfamilyfont{zhsong}{Noto Serif CJK SC}
% \setCJKfamilyfont{zhhei}{Noto Sans CJK SC}
% \setlength\parindent{0pt}

\begin{document}
\CJKfamily{zhkai}

\maketitle


\Large

到寺门还有很远一段上坡路。车可以直达寺门。司机望着云絮所剩无几而日光变本加厉的天空,执意要开车送本多上去,说老人步行太勉强。但本多断然拒绝,让司机在山门前等候。他无论如何都想亲
自体验一下六十年前清显的辛劳。 

本多凭依拐杖,背对山门内富有诱惑性的树荫
、站在门前眺望来时方向。 

四下里,知了声蟋蟀声此起彼伏。如此闲寂之地,仍有田野远处天理市汽车的喧嚣编织进来。但眼前的公路,却是全无车辆可寻,路肩印着细碎的沙影
白光光伸展开去。 

大和平原的悠闲情调一如往昔。平坦得如人间
\newpage
本身。远方,排列着小贝壳般房顶的带解镇闪着亮光,如今大概也有了小工厂,见有淡淡的烟柱升起。六十年前清显病危下榻的旅馆,位于镇上现在也应当可以见到的石板坡路的旁边。但旅馆本身想必不至于原
样保留下来,前去探访也不可能有什么收获。 

带解镇、整个平原的上方,晴空朗朗,纤云渺渺,惟有远方迷濛群山腾起海市蜃楼般的云絮上端以
其雕塑般的工丽切去一条碧空。 

本多受不住炎热和疲劳的夹击,倏地蹲在地上。蹲下时,眼睛似乎被青草凶狠而尖锐的叶端上的光芒刺了一下。蓦然,鼻端掠过一只苍蝇,本多真怕它
是为嗅出了腐臭。 

司机再次下车,担心地朝本多走近。本多瞪了
他一眼,随即立起。 

其实本多也担心自己能否走到寺门。胃和背部的疼痛同时袭来。本多甩掉司机,走进山门。他自我鼓励着,装出一副至少自己看上去干劲十足的样子,
\newpage
沿着凸凹不平的砂石坡路向上登去。路左边的柿树干上如传染病似地贴满鲜黄色的青苔。右边路旁的蓟草则花瓣几乎落得精光,露出淡紫色的秃头。本多顾不得仔细看这些,兀自喘息着移动脚步。好在拐角处都
稍微平坦。 

投在前面路上的一道道树荫,给他以神秘莫测之感。这条下雨时想必变成河床的路不规则地时起时伏,阳光洒落处如裸露的矿床闪闪生辉;树荫遮掩处显然沁出凉意,仿佛有什么窃窃私语。原因在于树荫。至于原因是否果真来自树本身,本多则不得而知。
 

可以在第几个树荫下歇脚呢?本多问着自己,也问着手杖。第四个树荫位于汽车无法窥知的拐角,静静地招引本多。刚刚捱到,本多便瘫痪似地一屁股
坐在路旁栗树底下。 

本多以极度的现实感想道:开天之初,我便注
定于此日此时,在此树荫下休憩。 

\newpage

走路时忘了的汗水和蝉鸣,随着休息一齐涌来。他把额头触在手杖上,通过杖头银压迫额头的痛感
来缓冲胃和脊背阵阵吃紧的疼痛。 

医生说胰脏有肿瘤,并笑着说是良性肿瘤。笑着说,良性的。要是把希望寄托在这上面,这一生算是白活!回京后拒绝手术这点本多也并非没有想过。但他心里清楚,如果拒绝,医生会马上动员“近亲”施加压力。自己业已落入圈套。一旦落入生而为人这个圈套,前面便不可能有更可怕的圈套等待自己。本多改变主意:索性来者不拒,并要做出满怀希望的样子。就连印度作牺牲的小山羊,都能在脑袋落地后还
挣扎那么久! 

本多站起身。这回已没有监视者讨厌的目光。他倚着手杖,迈着踉跄得近乎放肆的脚步向上登去。登了一会儿,他觉得自己好像在拿踉跄开玩笑。霎时
间,疼痛不翼而飞,脚步也轻快起来。 

夏日青草的气息弥漫四周。路两旁松树渐渐多了。倚杖仰望天空,由于日光强烈,树梢无数松塔的
\newpage
鱼鳞影看上去如雕刻一般清晰。不久,左侧出现一片
荒芜的茶园,上面爬满蜘蛛网和牵牛花。 

路的前方,仍横亘着几道树荫。眼前那道如破损的竹帘.影透着光点,远处三、四道俨然丧服衣带
黑得化不开。 

他拾起路面一个硕大的松树塔,又乘势坐在巨松的浮根上。体内又痛丝丝沉甸甸热辣辣疲劳得不到释放,如生锈的尖针弯曲着。他掰了掰拾起的松塔,彻底干透开裂的焦茶色塔瓣一片片强悍地抗击着手指。周围有几丛鸭跖草,花已被烈日晒蔫了。雏雁一般跃动的叶片,护拥着已经枯萎的极小极小的紫蓝色花朵。无论背靠的巨松,还是仰面见到的青瓷色长空,抑或仿佛被扫帚划过的几片云絮,全部干涸得可怕。

本多不会分辨四下的虫鸣。有的虫鸣沉稳有致为所有的虫鸣打下基调,有的虫鸣近似做恶梦时的咬
牙切齿,有的唧唧不止一味让人胸闷。 

再次站起的本多怀疑自己是否真的有气力走到
\newpage
寺门。他一边走,眼睛一边数着前面的树荫。他要看看自己能在多大程度上忍耐这酷热,忍耐这登攀几乎令人窒息的痛苦,同时能越过几道树荫。不料从开始数时算起,竟也越过了三道。有一道阴影树梢部分只及路宽的一半,为此颇感犹豫,不知算做一道还是半
道。 


路稍稍左拐不多一会儿,左侧出现了竹丛。 

竹丛本身即如世人的集体,有的芦笋一样纤细柔嫩弱不经风,有的则绿得发黑粗粗大大似不怀好意
,互相紧依紧靠难解难分。 

在此他又歇了一回。擦汗时第一次看见了蝴蝶。离得远时,只见黑黑的剪影,及至飞近,才发现朽
叶色的翅膀镶有蔚蓝色艳丽的花边。 

有一方水沼。旁边一棵大栗树。本多在它黑绿色的阴影里再度歇息。一丝风也没有,只有豉母虫在青黄色的沼面曳出细细的水纹。一棵枯树贴着沼边横躺下来,如一座桥。惟独枯树那里闪着细密的涟漪。
\newpage
涟漪扰乱沼面天空的湛蓝。或许有树枝在沼底支撑的缘故,这棵连叶片都已枯萎发红的倒木的主干并未浸在水里。尽管在万绿丛中仿佛生满一身红锈,但仍保
持了站立时的英姿,而不容争辩地继续以松自居。 

本多像要捕捉从尚未抽穗的芒草和狗尾草中摇晃着飞出的小灰蝶似地爬起身。水沼对岸一片青白的
扁柏树林正朝这边铺展,路面阴影渐渐多了。 

汗水透过衬衫,似乎西装后背都已被沁湿。不知是热汗还是油汗。总之年老后如此大汗淋漓还是头
一遭。 

不一会儿,扁柏树让位给杉树林。其交界处孤零零地立着一棵合欢。那柔软的密叶如梦一般飘飘忽忽掺进杉树刚硬的针叶,给本多带来泰国的回忆。这
当儿,那里飞出一只白蝴蝶来,在前边引路。 

坡路突然变陡,估计寺门快到了。加之杉树深处透来一股凉风,本多脚步轻快了许多。原来横在路

\newpage
上的一道道阴影,现在成了一道道阳光。 

白蝴蝶在幽暗的杉树间忽上忽下地飞着。飞过因点滴泄下的阳光而闪烁其辉的凤尾草,朝深处黑门那边低回飞去。本多想,不知蝴蝶何以全都飞得如此
之低。 

穿过黑门,山门即在眼前。想到总算来到月修寺门前,本多不胜感慨:六十年——自己活六十年的
惟一目的就是为了重访此地! 

当面对可以望见里面车廊陆舟松的寺门站立时,本多几乎难以相信自己已真的身临其境。他甚至舍不得跨进寺门,心旷神怡地站在左右各有一扇耳门、上敷十六瓣菊花纹瓦的寺门立柱前。左门柱挂一门牌,用秀气的小字写着“月修寺门迹”①。左门柱为印
刷体,字迹已有些模糊: 


天下泰平 


奉转读大般若经全卷所收 

\newpage


皇基永固 

穿过寺门,便是一条铺着小粒黄沙的甬路,黄沙间对角嵌着方形板石,沿着饰有五条卵黄色横线的院墙一直通到内门。本多用手杖一块块数着板石。数到第九十九块时,便已置身于内大门跟前。只见木格拉门合得紧紧的,扣手很别致,白色剪纸上带有菊花
和卷云图案。 

一时间,往日的记忆从全身复苏过来,宛然在目,以致本多忘了叫门,只是呆呆站着。六十年前,正当年轻的自己便是站在同一拉门前,站在同一块地面。拉门上糊的纸想必已换了上百次,但和那个春寒料峭的日子同样白刷刷地在眼前关得整整齐齐。门前地板的纹路也只是比以前略略凸出,并未显出久经风
霜的老态。一切不过弹指之间。 

他恍惚觉得清显仍把所有希望押在自己这次月修寺之行上面,在带解那家旅馆发着高烧等待自己的归去。假如知道自己已在这弹指之间沦为举步维艰的

\newpage
八十一岁老翁,清显不知何等惊诧! 

出来开门的是一位穿对襟衫的六十来岁的执事。见本多很难跨上地板,便拉起本多的手,领他走过八叠、六叠等好几个房间,进入正殿。执事很客气地说来信内容已经领教了,请他坐在包有黑白相间布边儿的草席上面摆得方方正正的座垫上。记忆中,六十
年前不曾进过这里。 

壁龛挂有雪舟摹写的云龙画幅,淡雅地插一支石竹花。一位身穿白绉纱衫系白腰带的老僧用方木盘端来红白两色糕点和冷茶。敞开的拉窗,可以见到满目苍翠的庭院。院里密密麻麻地长满枫树和丝柏树。透过树的空隙,可以窥见游廊在书院的墙上的投影。

执事说着万无一失的闲话,时间很快过去。本多觉得,只消在这凉风习习的殿里端然一座,汗便消
退,痛便减轻,甚至有羽化升天之感。 

这便是原先以为不可造次来访的月修寺,自己现在就这样坐在它的一个房,间里。死的临近轻而易举地促成这次来访,解开了系于存在深处的秤砣。爬
\newpage
山路的千辛万苦突然给自己一种身轻气爽的安详。如此说来,抱病走到这里的清显说不定也因遭拒绝而获得飞翔的力量。本多一时浮想联翩,甚至浮想都使他
感到慰藉。 

四下蜂声盈耳。但在幽暗的室内听来,竟带有钟声余韵般的清凉。执事再没提起本多的来信,时间很快在闲聊中流逝。本多又不便主动催问能否面见住
持。 

蓦地,本多觉得如此泛泛空谈便见不到住持。说不定执事看到了那本周刊杂志,而建议住持以身染
微恙为借口拒绝见面。 

实际上,本多心里也很为难,不好意思背此恶名求见住持。不过,若非负此耻辱负此罪孽和死到临头,本多也不至于产生来这里的勇气。现在想来,去年九月那桩丑闻,倒是月修寺之行的第一个阴暗的推动力。再说确切一点,阿透的自杀未遂也好失明也好本多自身的发病也好绢江的怀孕也好,全都聚为一点凝为一团,敦促本多下定决心,使他沿着烈日下的山
\newpage
路奋勇冲到这里。否则,本多恐怕只能举头遥望山顶
上的月修寺之光。 

可是,倘若住持因此之故而拒不接见,便也只能认为是前世的报应。估计今生今世是不得相见了。但同时,本多心里又总觉得即使不能在这里——在此生此世的最后时刻最后场所见到,也还是迟早可以相
逢。 

正因如此,宽慰才代替了焦虑,达观才取代了悲戚,二者愈发清凉生津,使他得以承受时间的推移
。 

这当儿,重新露面的老僧在执事耳边低声说了
句什么,执事转向本多: 


“住持说马上面见,请,这边请!” 


本多一时怀疑自己的耳朵。 

对着小庭院的北客厅,拉窗大开,加之院里绿
\newpage
色过于鲜明耀眼,本多刚进来时竟没认出这便是六十
年前谒见上一代住持的房间。 

记得当时有一面色泽鲜艳的十二月风景屏风,现在则代之以芦苇风档。隔着檐廊,蝉鸣声声入耳,茶院苍翠欲燃。梅树、枫树、茶树等绿丛深处;闪出夹竹桃的红色蓓蕾。踏脚石之间落着白白尖尖的竹叶,闪闪反射着夏日的阳光,同后山杂木林上方白光光
的天空上下交辉。 

险些撞墙的小鸟的振翅声使本多转过头来。原
来一只飞入游廓的麻雀,扑打一下白墙飞远了。 

里面房间的纸糊拉门开了,本多不禁并拢双膝——老尼住持拉着弟子的手出现了。这位身着白衣紫袈裟、脑袋青光闪闪的老尼,便是应当八十三岁的聪
子。 


本多不由渗出泪水,不敢正面仰视。 

住持隔桌在眼前坐定,端庄秀丽的鼻子一如往
\newpage
日,漂亮的大眼睛顾盼依然。虽然今昔不同时,但本多一眼即看出是聪子。六十载光阴竟被他一步跨过,一般人从青春年少到风烛残年遍尝的俗世辛酸她都一一得以幸免。面部变化不过如庭院里过得小桥从树荫来到阳光下之人那脸上的光亮变化而已。如果说当年正值芳龄的娇美是树荫下的碧玉,今日老年的风采则是阳光下的花容。本多想起今天从宾馆出发时阳伞下脸色或明或暗的京都女子,那明暗正好反映出美的性
质。 

莫非本多经历的六十春秋,对于聪子无非过桥
走过明暗交替的庭院的片刻? 

在聪子身上,老并非趋向衰竭,而是直指净化。光洁的肌肤静静生辉,美丽的眸子更加澄澈,仿佛体内有历久弥光的瑰宝,使得年老结晶为浑然天就的玉石,隐隐透明而峻冷,硬骨铮铮而圆润。双唇依然娇嫩,尽管有无数细纹,但每一条纹都如清洗过一般洁净。略微低俯变小的身体,含有无可言喻的威光华
彩。 

\newpage


本多含泪低下头去。 


“欢迎光临!”住持以爽朗的声音应道。 

“贸然写信打扰,诸多包涵,又承慨然接见,不胜感激!”本多不敢随便,寒喧十分郑重。听得自己喉头发出的这带有痰音的老声老气,自觉狼狈不堪,不由得又强调一句:“那封事先奉上的信,想您已
经过目。” 

“嗯,拜读了。”话就此打住,陪同的弟子于
是抽身离去,剩下住持一人。 

“真叫人怀念啊!如您所见,我已成了今天不知明日的老朽之身!”听得住持已经看信,本多来了
精神,语气中带有几分轻佻。 


住持旋即略微晃动一下笑道: 

“信拜读了。见您如此热心,我想可能是佛缘

\newpage
,就决定见您一面。” 

听到这里,本多心里残存的一两滴活力原液顿时进发出来,恍惚回到六十年前雄姿英发地面对上一
代住持的那一天。他索性丢掉客气,这样说道: 

“为清显的事来这里最后一次相求时,前任住持没有让我见你。事后当然想通是出于迫不得已,但当时却是怨恨来着。不管怎么说,松枝清显是我最要
好的朋友。” 


“这位松枝清显,是什么人?” 


本多目瞪口呆。 

耳朵诚然有点失聪,但这句话分明没有听错。住持此言委实莫名其妙。除了幻听找不出第二种解释

“哦?”本多特意反问,想让住持重说一遍。

不料,重复同一句话的住持的脸上,既无炫耀之色又无韬晦之意,莫如说甚至可以从中窥见童女般
\newpage
天真无邪的好奇心,和下面淙淙前流的静静的微笑。


“这位松枝清显,是什么人呢?” 

本多这才察觉住持大概意在让从自己口中说出清显的事来,于是在注意不致失礼的同时,摇动唇舌依照惟恐消失的记忆述说了清显同自己的关系、清显
的恋爱过程及其悲剧性结局。 

本多滔滔不绝的时间里,住持始终面带微笑地端然正坐,随声附合了几次。不难看出,即使中间老僧送来冷饮她优雅地端起送往嘴边的时候也没有漏听
本多的话。 


听罢,住持以不带任何感慨的平淡语调说: 

“倒是满有意思,只是我不认识那位松枝。至
于他的那位对象,您恐怕记错人了吧?” 

“可您原名不是叫绫仓聪子吗?”本多一边咳

\newpage
嗽一边急切切地说。 


“是的,那是我的俗名。” 

“既然如此,不会不认识松枝清显吧?”本多
颇有些怒不可遏。 

所谓不认识松枝清显,只能是装糊涂,不可能是什么忘却。当然,住持方面或许有某种缘由使她咬定说不认识清显。问题是,若是俗世女子倒也罢了,而身为德高望重的老尼居然说此弥天大谎,不仅足以使人怀疑其信仰的虔诚,而且令人认为她压根儿就未曾皈依佛门。因为到这一境地都尚未摆脱尘世的伪善!本多寄托于此番会晤的长达六十载的迷梦,在这一
瞬间灰飞烟灭。 

面对本多超乎常规的追究,住持丝毫没有惊慌。尽管如此溽暑蒸人,那紫色袈裟却仿佛透丝丝凉意。那声音、那眼神全然不为所动,谈吐依然流畅而动
听: 

“不,本多先生,在俗时受到的恩惠我一件也
\newpage
没有忘记。只是,的确没有听说过这位松枝清显。恐怕根本就没有这个人吧?您倒像是觉得有,而实际上则莫须有——事情会不会是这样的呢?听了您的这些
话,您总有这么一种感觉。” 

“可你我是怎么相识的?再说,绫仓家和松枝
家的家谱也应该还有吧?户籍总还查得到吧?” 

“俗世上的来龙去脉,固然能以此理清。不过,本多先生,您真的在这世上见到过清显这个人吗?而且,我和您过去的的确确在这世上见过面吗?您现
在可以断言吗?” 


“的确记得六十年前来过这里。” 

“记忆这玩艺儿嘛,原本就和变形眼镜差不多,既可以看取远处不可能看到的东西,又可以把它拉
得近在眼前。” 

“可是,假如清显压根儿就不存在,”本身如坠云雾,就连今天这里面见住持也半像是做梦。他像
\newpage
是要唤醒自己——如同哈在漆盆边上的气晕一般急速消失的自己那样情不自禁地叫道:“那么,阿勋不存
在,金让也不存在……说不定,就连这个我……” 


住持的眼睛第一次略微用力地盯住本多: 


“那也是因心而异罢了。” 

一阵久久的默默然对坐。而后,住持肃穆地拍
了下手。随身弟子应声出现,在门口俯下身去。 

“来一次不容易,请观赏一下南园吧!我当向
导。” 

弟子再次拉起要当向导的住持的手。本多像被
操纵似地站起身,跟着两人穿过幽暗的书院。 

弟子拉开拉门,引本多进入檐廊。宽阔的南园
顿时展现在眼前。 

绿草如茵的庭院以后山为背景,在炎炎烈日下
\newpage

闪闪耀眼。 

“今天一早就有布谷鸟叫来着。”年纪尚轻的
弟子道。 

草坪边缘长着一些树,大多是枫树,从中可以窥见通往后山的柴扉。虽时值盛夏,枫树却已红了,从绿丛中燃起火焰。几块园石悠然点缀着绿地,石旁开花的红瞿麦一副楚楚可怜的情态。左面一角有一眼轱辘古井。草坪中间有一深绿色瓷凳,一看就知被晒得滚烫,怕是一坐上去就会灼焦。后山顶上的青空,
夏云耸起明晃晃的肩。 

这是一座别无奇巧的庭院,显得优雅、明快而
开阔,惟有数念珠般的蝉声在这里回响。 

此后再不闻任何声音,一派寂寥。园里一无所有。本多想,自己是来到既无记忆又别无他物的地方
庭院沐浴着夏日无尽的阳光,悄无声息……

\end{document}
