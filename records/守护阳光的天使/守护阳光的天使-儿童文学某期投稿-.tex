\documentclass{article}
\usepackage[utf8]{inputenc}
\usepackage{ctex}

\title{守护阳光的天使\footnote{Click to View:\url{https://web.archive.org/web/20220729100235/https://www.wenji8.com/home/118925.html}}}
\author{儿童文学某期投稿}
\date{}

% \setCJKmainfont[BoldFont = Noto Sans CJK SC]{Noto Serif CJK SC}
% \setCJKsansfont{Noto Sans CJK SC}
% \setCJKfamilyfont{zhsong}{Noto Serif CJK SC}
% \setCJKfamilyfont{zhhei}{Noto Sans CJK SC}
% \setlength\parindent{0pt}

\begin{document}
\CJKfamily{zhkai}

\maketitle


\Large


天使,你寂寞吗?我读书给你听。 


一 

终于,在数学老师发表了一大篇的演讲,说了无数个“因为”“所以”后,沐霏忍不住打了个哈欠,她低头看了看表,还有10分钟下课。这时,戴着老花镜的数学老师转过身在黑板上抄练习题,其他人都安静地拿笔在做,沐霏却转向窗外闲看,橡皮已经被手指甲挖得凹凸不平,她仍然使劲地挖。在看窗外有30秒钟后,沐霏转过脸,恰好与数学老太婆那双浑浊但具有穿透力的眼睛相遇,两个人相持了6秒种,沐霏拿出笔开始演算,老师仍旧回去抄题。很多时
后,沐霏总觉得数学老师是个神秘的女巫。 

\newpage

下课了,随便填了个答案交给课代表,沐霏便无事可干,一边喝水一边看小S那消瘦的背影。笔在他修长而又青筋突起的手上飞快地摆动着,小S还在埋头苦干。大概那道题真的有难度吧,沐霏想。环顾四周,班里大多数人都没离开座位,沐霏继续喝水。
 

这时,一道刺眼的阳光从外面射进来,沐霏把窗帘拉上。现在正是三伏天,如果不是学校强制性补课,沐霏准是第一个冲出教室,她真不知道着所学校有什么可留恋的。窗帘刚拉上,又有一只手把窗帘拉开,阳光照得人睁不开眼睛。沐霏好不容易才睁开眼,问道:“是哪个家伙拉的窗帘?”“本大爷!”小S写完了作业转过来挑衅的说。沐霏望着眼前这个又瘦又矮、好像营养不良的男孩子,咬牙切齿地说:“我说你这人是不是有毛病啊,这么热的天你还要晒太阳,你要是想把自己变成烤肉串,你干脆到楼下去晒,不要妨碍我!”小S也来劲了,他放开嗓门:“你知不知道晒太阳有多舒服!”“我没兴趣,现在请你拉上窗帘!”“我要是不拉呢?”“不拉我就……”正在他们舌战时,脸上长满青春痘的英语老师抱着一
\newpage
叠卷子匆匆走进来,她用一种居高临下的语气对全班说到:“把无关紧要的东西全收进去,把课本放在讲台上来,准备考试!”一本本教科书放到讲台上,一张张黄色的试卷传下来,让人感到一阵眩晕。沐霏已不记得这是这个星期第几次眩晕了,不过对她来说英语考试就等于睡觉课,因为她总在5到10分钟之内把试卷上的空填满,然后便倒头大睡。老师也懒得管
这种问题学生了,自生自灭吧。 

沐霏拿着笔填到一半时发现笔芯写完了,她不在乎,正好可以多睡会儿,于是把头埋进双臂,开始睡觉。过了有20分钟,沐霏觉得肘下痒痒的,醒来一看,肘下压着一支笔芯和一张纸条:快写,懒猪! 一看就是小S的字,狂草加鸡爪。沐霏装上笔芯,填完剩下的空,继续睡。其实她自己也弄不明白和小S到底是敌还是友,到目前,她还弄不清楚小S是怎
样的人,因为,她连自己是怎样的人都没弄清楚。 

下课铃响了,沐霏也醒了,阳光明晃晃地照着教室,她也忘了拉窗帘。在往前交试卷时沐霏清楚地看见,一束束阳光均匀地洒在小S头上,好象渗下去
\newpage
了。沐霏想,大概他会进行光合作用吧。难怪他那么
喜欢晒太阳,应该到沙漠里去。 


二 

沐霏这两天心情不爽,窝着的火不知道往哪里发,她级别很高的QQ号被盗了,怎么都找不回来,都怪自己太大意没有申请密码保护。沐霏每次去网吧只是聊天,她也奇怪像自己这种沉迷网络的学生居然不沉迷网络游戏。她尝试着玩过几回,觉得幼稚,想沉迷都不行,反倒对聊天很有兴趣,于是每个夜晚都不做作业溜出家去泡网吧,第二天红着眼睛去学校补
作业。 

夏夜的风拂过脸颊,微微发烫的脸颊顿时有了一丝凉意。沐霏理了理凌乱的头发走出网吧。街上的汽车已经很少了。走到一个岔路口时,沐霏迎面撞上了两个像混混的人,沐霏认得他们是学校里记了大过的学生,刚想转身就走不理他们,肩膀已经被一只手搭上了:“你撞疼了我们,总要补偿吧。”懒得跟他们理论,于是她狠狠地说:“老子心情不好,要打架
\newpage
随时奉陪!”刚要开打,她的手就被另外一只手抓住,一阵飞奔,沐霏只觉得道路两旁的树像百米冲刺一样从她眼前疾驰而过。最后终于停了下来,沐霏才看清拽着自己跑的人时小S。小S边喘气边得意地说:“要不是我你早就被打成肉病了!”“你别管闲事可不可以?”沐霏一阵胸闷,然后大步流星地朝前走。小S大叫道:“你就是个仙人掌,长大后没人娶你!”沐霏真想回去揍小S一顿,她一回头,小S早就走
了,父母已经很久没回来过了。 


三 

阳光懒惰得让人想睡觉。沐霏带着一字未动的作业来到学校,一个汉堡几口下肚,她拍了拍手,揉揉眼睛开始写。一股清新的洗衣粉味道钻入鼻腔,她用余光瞟见一套干净的校服,再向上时一张苍白且熟悉的脸,一双青筋突起的手轻巧地把凳子放下来,无声无息的坐下。沐霏顿了顿,继续写她的作业。突然那张熟悉无比的脸凑到她眼前。小S本来就白,在阳光素无忌惮地照耀下显得更加苍白一些。他看着面无表情低头狂写着的沐霏,轻描淡写地问:“哎,仙人
\newpage
掌,昨天还好吧?”没想到一句话就把沐霏惹火了,她一下子站起来,猛地推了一下桌子,把小S弹到了地上。小S捂着肚子慢慢站起来,沐霏依然面无表情地说:“我的事你最好少管,你不要和我这样的问题学生说话。还有,我叫沐霏,不叫仙人掌。”小S也火了,使劲踹了一下沐霏的桌子,她的膝盖撞上了桌腿,钻心地痛,忍忍,泪没有出来。于是两人开打,
作业本课本到处乱飞,一片狼藉。 


四 

老师办公室开着空调,凉快得很。沐霏有点不想走了。小S咬着下嘴唇,不服气的望着班主任。改完了一组作业本,班主任终于缓缓抬起头来,幽幽地说:“我第一次看见男生和女生打架。你们……还有没有一丝羞耻心?明天叫家长来一趟。”沐霏开始想好久没有回来过的父母,等了那么久,他们的样子都有点模糊了,爷爷奶奶也只有过年来一下,平时她都是一个人孤单度过一整天。家,已经不重要了吧。无
所谓。 

\newpage

可是,就真的无所谓了吗?沐霏心里的湖水被
微风吹过,泛起细小的涟漪。 

走出办公室的时候,阳光刺得必须闭上眼。沐霏想,外面还真是热呀。蝉鸣一声盖过一声,使天气变得燥热。小S双手插在口袋里,歪着头,阳光依然
均匀渗透进他直直的头发里。 

浴室的水哗哗地流着,沐霏脱下长裤,看看膝盖,已经紫了一大块,一按就会痛。抬起头,看着杯
水汽模糊了的镜子,照出模糊的自己。 

7岁的时候,父母吵得很厉害,而自己只能大声地哭,以为哭声可以让他们停止吵架,可却是火上浇油,妈妈把门一摔久再没回过家。于是从那时起她告诉自己哭最没用,眼泪最不值钱,于是就真的没再哭过。每次和别人打了架就一个人回家,洗个澡,换上干净的衣服,然后把泪水往肚子里咽……这次她同样也没让眼泪溢出眼眶。其实也没怎么生小S的气,只是想起了很多往事,沐霏感到鼻子酸酸的,这才发

\newpage
现浪费了很多水资源,赶快洗了睡。 

把眼泪咽下,是不是就会变得很坚强?是不是就不会再受伤?也许我真的像仙人掌一样,只是用带
刺的外表包住流泪的心罢了。 


五 

冬天似乎很快就来了。沐霏抬头看看日历,已经11月份了,可是阳光依然灿烂,给人的感觉还是在过夏天,唯一不同的是早上起来微微有一点寒意,不过一到中午,太阳又乐开了花。篮球场上依旧会有耍酷的男生穿单薄的衣裳独自练习投篮,也依然会有羞涩的女生靠在铝合金窗户旁,一边吃饭一边痴痴的看着。这一切都好像停留在那个荷花盛开的夏季,空气中似乎还有栀子花的味道。秋天呢,秋天是一片空白,好像被人们遗忘,聒噪的夏天过后,冬天接踵而来,而且还有一丝夏天遗留的气息。人们都说,今年
是个暖冬。 

月底过生日。沐霏想到这里,手中的笔停下来呆了一会儿。但生日还是一个人过,她的眉头不由得
\newpage
皱了一下,于是搓了搓手心的汗,提起笔继续写下去。这些细微的举动没有逃过小S的眼睛,他开口问道:“干嘛皱眉头啊,肚子痛啊?”还是那种不轻不重的语气,漫不经心的表情,让人猜不透是关心还是嘲笑。沐霏也不轻不重地回了一句:“嗯,还好。”“哦。”他的眼神黯淡下来,眼睫毛垂了下来,然后转过身去,于是世界一片沉寂。他那个细微的眼神,谁
都没看见。谁都没看见。 


六 

11月30日。“祝我生日快乐!”沐霏在心
里对自己说。 

今天一天还算过的顺利,早上语文连堂她好好补了一觉,老师什么也没说,估计也懒得说了。哎,不管它了。数学考得还不错,经过小S的帮助,自己
数学长进了不少,虽然上课还是不大认真听讲。 

走到家门口,灯居然是亮的。沐霏想:会不会是……妈妈!于是飞快地拿出钥匙,打开门,一个人
\newpage
也没有。只是桌子上多了一个生日蛋糕,水杯下有一
张纸条: 

不要问我到底是谁,告诉你我是天使。生日快乐!还有你QQ号被盗了,用我的吧:xxxxxx
x 密码 xxxxxxx 


Sunny 

虽然一头雾水,不过还是很开心,毕竟自己不是一个人过生日。沐霏对着纸条说了声谢谢,安静地
许了个愿,吹灭了蜡烛。 


15岁,悄然而过。 


抬起头,发现窗户开着。沐霏家住一楼。 


七 

“这道题嘛,用1式减去2式,得到3式,然

\newpage
后用1式减3式,就得到a+b 

+c值了。““哦,谢了。”自从那次数学考试之后,沐霏开始认真对待数学了,遇到不懂的就会问小S。每当这时小S就一副自我陶醉的表情,沐霏就用笔使劲敲小S的头,吼道:“少花痴了!”小S便会安静下来,用平和的语气讲题,缓缓地,如同小
溪。 

“哎,花痴。”沐霏拍拍小S。“嗯?”小S疑惑地转过身来,那表情好像在说:“叫我吗?”沐霏用“就是你”的眼神看着小S说:“你QQ号是多少啊?”小S神色紧张的说:“你要干吗?”沐霏哭笑不得:“我又不盗你号子,只想加你,那么紧张干什么!”“我,我号子已经被盗了。”“哦。”沐霏
低头喝水,不再说话。 


他是在撒谎吗? 

记忆中的小S无论再怎么讨厌也没说过谎话。

门口又有张sunny留下的纸条:“今天有
\newpage
进步,学习认真了。”沐霏忍不住了,用笔在纸条上留下了一句话:你到底是何方妖怪?然后拉开门进去
。 


一个号子罢了,何必那么在乎。 


也许,我只是习惯你对我说真话罢了。 


疑问句正在逐渐变成肯定句吗? 


八 

早上头有点晕,摸摸还有点烫,肯定是昨天晚上泡网吧着凉了。冰箱里的退烧药只剩下空瓶了,抬头看看钟,还差十分钟打铃,况且英语作业还一字未动,如果迟到了,一定会被班主任训上半天,像自己这种问题学生,他才不管你是不是生病了还是怎么了

感觉自己还能走,不至于倒下去,于是赶快赶到学校。喝了一大杯热水,还是感到很恶心,想吐。沐霏昏昏沉沉地伏在可捉上。朦胧中好像上课铃响了
\newpage
,揉了揉眼睛,然后又昏沉沉地睡去。小S转过身来拿笔记本,看了看沐霏,叹了口气,想:这丫头昨天晚上一定又是通宵泡网吧。于是轻轻抽出压在沐霏小
肘下的笔记本,又转过身去, 


这节数学课,上新课,很重要。 

下课了,沐霏感觉越来越难受,眼皮像灌了铅似的抬不起来。小S转过身来看见她还在睡,摇了摇她:“还睡呢,再睡你的脑袋都扁了!”没有回答。他发现沐霏耳朵红红的,于是用手贴了帖她的额头,
不出所料,她发烧了。 

沐霏迷迷糊糊中觉得好像是有人把自己的手搭在肩上,然后吃力地背起来,跌跌撞撞地走着,而自己明明全身乏力还逞强,神志不清地说:我能走,别背我。“别乱动,否则咱俩都得滚倒在地上了。”是得到的唯一回答,于是刚才还乱挥的双手立刻安静下
来,接着眼前一黑。 

自己躺在校外诊所的病床上,手背打着吊针。
\newpage



这是沐霏醒来以后的第一反应。 

刚从卫校毕业的护士打趣地说:“你发烧的度数可真有个性,39度8。”沐霏的嘴巴动了动,算
是笑笑。 

扭过头,她看到一张熟悉无比的苍白的侧脸,嘴唇紧闭,毫无血色。窗外射进一缕阳光,把他的脸照得有了阴影。沐霏舔了舔嘴巴,用沙哑的声音说道:“喂,花痴。”小S转过来,阴影从右边跑到了左边,脸上写着三个字:有事吗?“一看就知道你缺乏营养,要多吃点红枣鸡蛋,不然别人以为你妈虐待你。”小S笑笑,脸上有了点生气。随后他拿出身后的笔记本,扔给沐霏,说:“早上的笔记。”沐霏翻开自己的笔记本,布满了狂草加鸡爪,她挑剔地说:“你就不能把字写好点吗?”“帮你就不错了,再说,我字迹一直这样。”小S有点恼。“谁要你帮我写了,自做多情。”“你怎么不识好歹呢?”沐霏一副“懒得理你”的表情,然后转过头去。过了一会儿,一双手伸到自己面前,一只手上放着一片阿司匹林,一
\newpage

只手上托着一杯温水。 

于是心里千万只鸥鹭,声音由远及近,由近及
远。 

sunny:以后少去网吧,我可在天上看着
你. 

抬头望望天空,只有星星无声地相互守望着。
  


九 

期末考试临近,教师里火药味越来越浓。要窒息了,沐霏把笔一扔。小S弯腰捡起来,交给沐霏,
皱了皱眉:“发什么神经啊?” 

沐霏不管他,自顾自说道:“期末考试谁发明的?我一定找他算帐。”“你妈发明的,你爸申请专利……”还没说完,沐霏一拳过去,小S揉着肩膀:“你就不能斯文点?山顶洞人时期的仙人掌!”又是
\newpage

一拳,比刚才还用力。 


日子就在每天的打闹中过去。 


考试渐渐逼近,逼近。 


在门上留言:加油考试! 

班主任每天顶着个秃了顶的脑袋重复几次,考
试不要慌张,要沉着冷静。 

沐霏心里有个声音却在说:“我真想逃掉考试
!” 


十 

两个星期后的期末,小S的座位却空了,沐霏刚开始觉得挺清静,可是半天之后就不习惯了,没人斗嘴的日子真难熬。跑去问班主任,班主任依旧用敷衍的语气说:“他身体不好,在医院疗养。”沐霏知道班主任是不屑和自己这种人废话。于是决定自己去
\newpage

医院。 

眼前这张脸比平时还要苍白十倍,可窗户还是固执地开着。沐霏扔给他的第一句话是:“让你多吃点红枣鸡蛋,你不信,现在好了。”小S沉默半晌,说道:“你一女的跑来看一男的,成何体统啊。”沐霏胸闷:“你当初和我打架时怎么没想起来你是男的我是女的?还不是照样把我打得鼻青脸肿?然后我们一起站办公室里吹空调啊!”小S挠挠头发,抱歉地笑笑:“我们是不打不相识哦。”“相识你个头啊!”沐霏又一次胸闷,抄起一个白色枕头朝小S砸去,
然后头也不回地走出病房。 

身后传来一声接一声的“沐霏、沐霏、沐霏、沐霏”最后变成了“仙人掌、仙人掌……”可自己倔强的双脚仍不肯往回退一步。终于,单薄的声音消失在空旷的走廊上,于是松了一口气径直往前走。年轻护士慌张的声音扩散开来:“医生,前几天刚手的病人心脏病发了,医生……”医院就是这样,沐霏想。

徒步回家,的笔迹今天例外地没有出现在门上
\newpage

,沐霏望望天空,依然只有星星相互守着。 

人生的舞台上,时间正酝酿着一场阴谋,该上场的人上场了,该先下场的人下场了,悲伤正潜伏而来,快乐正一点点被吞噬,世界仍然一片欢腾。我们总是在一次又一次的固执中,一次又一次地擦肩而过

十一早上天气暖和,沐霏没有穿羽绒衣,抓了件外套就向医院跑去。是为昨天的事道歉吗?不知道。路上心里还在犹豫,到底要不要去呢?还差点撞到
一个早点摊子。 

抬起头,已经到医院大门口。广场上清洁工人正刷洗着被尘灰沾染已久的地面,清洁剂的味道混合着医院的消毒水气味飘荡在医院里。顾不得这些,硬
着头皮走了进去。 

小S的病床上空了,沐霏想,他是不是已经回家洗澡睡觉了?于是心里有一种被耍的感觉,自己被耍了。刚要,一个护士站在门口问:“请问你找谁?”沐霏说:“六病房三床的病人是不是出院了?”护
\newpage
士沉默了一下问:“你是不是叫沐霏?”沐霏点了点头,有点诧异:“你怎么知道我?”护士叹了口气,从口袋里掏出一张纸条塞到沐霏手里,说:“他临终
时要我给你的。” 

像是有个人提着大桶水朝自己泼来,从头发到
脚趾全是湿漉漉的,潮汛将至,却毫无征兆。 

呆了一会儿,才问:“什么时候?”护士面无表情:“昨天晚上先天性心脏病突发没有抢救过来。
” 

“医生,医生,前几天刚收的病人心脏并发作
。医生……” 


“沐霏、沐霏……仙人掌、仙人掌……” 

昨天的画面一幕一幕毫无顺序地在脑海里播放
,大脑像被人搅了一下。 


\newpage

缓缓打开小S的字条,狂草加鸡爪: 

仙人掌,实我就是那个sunny,一定会怪我向你撒谎吧.好了,说清楚了,我要去天堂了,上帝以前遗忘了我,所以现在我得去找他算帐啊.他怎么样也得让我做个真正的天使,那么我就做个爱护阳
光的天使吧,在离你最近的那片云上看着你. 

沐霏就这样捏着纸条一动不动,捏到指尖发白,指关节僵硬,突然,眼泪在毫无防备的情况下如洪水般夺眶而出,伴着急促的呼吸声。于是所有的酸甜苦辣被揉在一起,形成了眼泪,这一刻,泪水真的只
有咸味吗?八年来,第一次哭了。 

再也不会有人和自己打架了,再也不会有人上课时提醒自己要听讲不要睡觉了,再也不会有人抄了笔记逼着自己了,再有不会有人不拉窗帘任凭阳光射
得眼睛睁不开了。 


再也不会有人了。 

我为什么连你的笔迹都认不出来?你里我那么
\newpage

近,我却没有想起你, 

而现在,一切都需要加上一个“曾经”了。你
曾经离我那么近,我却曾经没有想起你。 

一切都似乎回归平静。依旧每天上学,老师依旧用老花眼等着她,窗帘依旧拉不上,阳光依旧射进来,只是多了一份孤独。前面的位子一直空着,也没
人来不上去。 

今年冬天,阳光灿烂,今年冬天,没有下雪,
人们说,今年是个暖冬。 

“沐霏,你又在发呆吗?站在外面去!”于是她竟温顺的出去,没有脸红,没有耳赤,也没有股指。走道走廊上,抬头看天,很高很高的树梢上顶着一朵小小的白云,被金色的阳光笼罩着,沐霏咧咧嘴,舒心的笑了。教室里开始读书,沐霏也打开课本,第
一次那么大声地朗读者。 


\newpage

书声琅琅,响彻云霄。 

天使,你寂寞吗?我读书给你听。

\end{document}
