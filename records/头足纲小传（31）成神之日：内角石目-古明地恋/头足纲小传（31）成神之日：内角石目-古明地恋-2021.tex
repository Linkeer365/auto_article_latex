\documentclass{article}
\usepackage[utf8]{inputenc}
\usepackage{ctex}

\title{头足纲小传(31)成神之日:内角石目\footnote{Click to View:\url{https://web.archive.org/web/20230708002350/https://paste.ubuntu.com/p/8wz6N8KjDn/}}}
\author{古明地恋}
\date{2021-01-09}

% \setCJKmainfont[BoldFont = Noto Sans CJK SC]{Noto Serif CJK SC}
% \setCJKsansfont{Noto Sans CJK SC}
% \setCJKfamilyfont{zhsong}{Noto Serif CJK SC}
% \setCJKfamilyfont{zhhei}{Noto Sans CJK SC}
% \setlength\parindent{0pt}

\begin{document}
\CJKfamily{zhkai}

\maketitle


\Large


私者一时,公者千古。 

奥陶纪是一个军阀混战的时代。奇虾已经被推翻,但业已形成的各门各纲仍马不停蹄地进行着生存竞争。愚昧落后的海洋生物裹挟在时代剧变的浪潮之中,使奥陶纪成为地球物种分类多样性增长最快的时期。广泛分布的陆缘浅海孕育了难以置信的生物多样性。这些生物在温暖、潮湿、稳定的温室中演化,诞生出各种各样不可思议的物种,并被地球历史上一些
最猛烈的灾荒与战争所打破。 


1、星降恩典 

在奇虾被推翻后的这一系列事件被称为奥陶纪生物大辐射事件(GOBE)。在地质历史上最高的
\newpage
海平面之下,寒武纪爆发产生的大部分动物门、纲,以它们形形色色的体型和结构,为它们在奥陶纪以多样化的方式填补生态位空间提供了条件。有史以来生
物多样性的最大增长发生了。 

在奥陶纪,生物的目数量增加了两倍,科增加了三倍,而属增加了近四倍。奥陶纪的动物不再仅仅以土地为生,水中有着更广阔的空间和更先进的浮游生产方式。一些动物游进水中,接触更大的世界。许多奥陶纪无脊椎动物在水层中开始了生命周期,它们的幼虫和幼体构成了浮游生物的重要组成部分,导致浮游植物和浮游动物的多样性也猛然增加,经济迅速发展。这种“爆炸性”的多样化过程被称为“奥陶纪
浮游生物革命”。 

在海底,三叶虫,节肢动物,腕足动物,棘皮动物和早期的脊索动物占据着一些领地,在水层中,浮游着许多小型的叶足动物、节肢动物、桡足类和鳃足类,浮游三叶虫及其幼虫,有孔虫,以及大量广泛分布的笔石和放射虫。它们充塞着水体,提供了不仅是近岸,还有远洋的栖息地。腹足类啃食着藻类和残
\newpage
骸,双壳类过滤着浮游植物和动物。但在与在水中食人民之肉的奇虾和在地上食人民之肉的海蝎的战争之后,恩德胜(内角石目Endoceratida)
成为了这个仍显落后时代的统治者。 

头足纲不一定是奥陶纪的,但奥陶纪一定是头足纲的。那是内角石统治公海的时代。巨型的内角石在水中巡游,所到之处尽皆披靡。海蝎奇虾收敛了利爪,无数形状各异,大小较小的头足动物在他周围游
动,沐浴在他的荣光之下,视他为人民的救世主。 

非常幸运地,内角石受到了天父的恩赐。落在
他周围的陨石没有砸死他,反而让他更加强大。 

奥陶纪生物大辐射可能是陨石撞击的结果。奥陶纪生物多样化的主要阶段在大约470万万年以前开始,与l型球粒陨石母体的小行星带破裂相一致——这是过去几十亿年里有记录的最大的小行星破裂事件。小行星的分裂导致了陨石撞击率的上升,并在最初的分裂后持续了1000 - 3000万年。中奥陶世的陨石撞击事件比其他时代要频繁5-10倍
\newpage
。小的撞击事件破坏了局部的生态系统,杀死了原本的居民,同时创造了一系列新的空白生态位,促进了
物种形成事件来填补这些机会之地。 

Schmitz和他的同事(2008)提出奥陶纪生物大辐射可能是陨石撞击的结果,至少在Baltoscandia地区,陨石撞击和腕足动物多样化这两个事件的发生“似乎是完全一致的”。然而,也有人声称,奥陶纪的多样化是在陨石撞击频率
提高之前就开始的,因此目前两者仍有争论。 


2、统治建立 


天父是公平的,而内角石抓住了这个机会。 

寒武纪末的灭绝,震旦月蚀”(晚寒武纪的灭绝事件,“late Trempealeauan Eclipse”)消灭了大部分的大型掠食者。三叶虫经历了三次独立的灭绝,而晚寒武世广泛分布的4目34属的角石只有一目两属存活到奥陶纪,那

\newpage
就是劫后余生的爱丽丝(爱丽斯木角石目)。 

人偶工厂被毁灭后,爱丽斯木角石目的幸存者和他们的后代重新繁衍。从晚寒武纪不到5 - 6厘米长,弯曲的环角石(cyrtocones)开始,内角石目(Endoceratida)从巴斯利尔角石科(Bassleroceratidae)分化而出。它们拥有直或弯的壳,隔璧颈为无颈式至长颈式,连颈环简单或复杂,体管后部具内锥及体
隙,气室沉积和内体管沉积发育。 

但与其他鹦鹉螺不同的是,内角石拥有一根非常粗大的体管,有时宽达壳直径的一半。体管多位于壳的腹边或近腹边,体管内有由锥体迭置而成的内锥,它们互相连接,在顶端形成尖圆锥形的内锥管。在生活时,体管内不仅有吸管状的肉质体管,还充满呈圆柱状的其他软体组织。换句话说,内角石的软体部分不限于体室内,还存在于体管内。这增大了它们软体占身体的比例,同时大量的软体组织可以更好的吸
水排水,提高浮力的调节效率。 

奥陶纪内角石的快速演化和增殖是浮力系统增
\newpage
强的结果(Crick, 1988)。各种充满气体和气室体管沉积(相当于压舱石)的壳体的进化,发达的中枢神经系统,可抓握的触手加上可切割咀嚼的颚,增强的、更有效的呼吸系统和扩大的鳃,以及漏斗喷水的快速推进。
作为游动的顶级海洋掠食者和食腐动物,几乎没有证据表明其他动物能够动摇内角石的统治,甚至是与其有竞争关系,直到它们灭绝。内角石也没有把
它们赶尽杀绝,而是与之联合,共同协商。 

最早的内角石有相对较小的直的或内内弯曲的壳,通常有环形的肋;随着时间推移,体管直径增加,壳体变直,内锥管开始发育。到了中奥陶纪,带有直锥壳的巨大个体便开始占据优势 (Balashov 1962, 1968;Teichert)
。 

最巨大,以及最繁盛的内角石出现在北方海域,劳伦大陆、波罗的大陆、西伯利亚大陆周遭的海洋中。从中奥陶世开始,若干属的内角石,即“房角石”,包括房角石属(Cameroceras),内
\newpage
角石属(Endoceras),成为古生代最大的软体动物,以6米长,数吨重的体型成为该亚纲历史上最大的头足类之一。在中奥陶纪,内角石的鼎盛之
年,他的威望和力量达到顶峰。 

房角石Cameroceras作为一个分类单元的历史要追溯到1842年,在古生物学的还处于起步阶段的时候。这就是为什么Cameroceras也被称为废纸篓分类单元的原因,很多其他大型角石化石在过去都被直接归于这个属。这些标本现在已被重新分属于其他属,然而房角石已经成为了巨
型角石的同义词。 


3、简朴生活 

天人不食五谷,但统治者却与凡人相同。古来统治者数千百万,但能做到身居高位还粗茶淡饭者,
古今少有。 

在很长一段时间里,所有已灭绝的鹦鹉螺亚纲头足动物都被认为与现代鹦鹉螺一样营底栖生活,但
\newpage
最近这种观点被否定了:在鹦鹉螺亚纲中,不仅有底栖的食肉动物,而且也有上层的浮游动物,它们的生
活方式和繁殖方式都与鹦鹉螺不同。 

然而,对于某些鹦鹉螺类的习性我们仍然知之甚少,而最神秘的头足类之一,便是最巨大的内角石目。巨大的内角石要吃下大量的食物,但到目前为止,还没有发现它们软组织的痕迹,也没有发现喙的残
骸,故难以直接推测它们的食性。 

具有圆锥形壳的鹦鹉螺类动物因为平衡时壳口向下,通常被认为是底栖动物(如Westermann 1998),在过去,内角石也一直被认为是底栖的捕食者,在海底或海底附近捕捉小动物。但是,内角石与其他角石最大的不同之处便是更宽的体管和锥形的气室、体管沉积物,即内锥管(endocone)。内锥管是在壳体顶端形成的,其重量无疑影响了内角石壳在水中的方位(Balashov 1962, 1968)。内角石的体管与其他鹦鹉螺类的体管如此不同,以至于这些动物的生活方式和

\newpage
解剖结构一定与其他鹦鹉螺类截然不同。 

Balashov(1968)指出,位于壳顶内锥管的重量会将壳顶下压,降低外壳的后部,壳口方向便会趋向水平。因此,有这种壳的动物似乎不太可能营底栖生活,而更趋向于生活在水体中。无论如何,一个又重又长的锥状壳(从几十厘米到几米)的机动性很差(Holland 1987),这对于底层的生活方式是没有用的,因为底部通常有很多障碍物。虽然一些化石显示,一些内角石的正圆锥壳在背腹方向稍微变平,但也有许多内角石壳在侧面扁平,因此不能用于推测它们的底栖生活习性。此外,在内角石中,从未出现过强烈的背侧扁平的,如珠角石目的兰姆角石Lambeoceras和棱角石Gonioceras的乌贼样形态,因此先前假设的
底栖底栖生活方式可能不适用于内角石。 

然而,在之后的研究中,科学家发现内角石矿化的内锥管分布在体管的上半86.9%部分,质量分布仍然相对靠近体腔,因此内锥管的存在只能略微改变开口朝向,而且降低了内角石在游泳时的稳定性。因此,如果内角石是悬滤食者(Mironenk
\newpage
o,2018),它们必须依靠严格的自我要求,主动运动才能以水平方式游泳;但角石的游动能力始终未知,因此,以底栖生物为食的底栖生活方式也有可
能出现(Frey,1989;Kröger)。 


4、恩及万民 

在自身成为最高统治者的时候,内角石却清心
寡欲,专注于让人民安居乐业。 

内角石将首都定在北方,而有史以来最繁盛的鹦鹉螺动物群也在那里。Platteville动物群从格陵兰岛延伸到新墨西哥,在这个时期,一个极其多样化的软体动物群进化出来,与正常的奥陶纪类群如腕足动物、三叶虫和棘皮动物相关联。对这些动物群的分析表明,鹦鹉螺的多样性激增,整个领地都得到开发,几乎填满了所有可用的捕食和腐食生态
位。 

Platteville动物群的软体动物占所有动物总数的68%,其中腹足类占43.7%,
\newpage
头足类占34.5%,双壳类占14.9%,其他软体动物占6.9%。其次最丰富的类群是腕足类,占总种类的13%多一点。无论是从形式的多样性、分类群的数量、个体的数量,特别是许多都超过2米的个体大小来看,Platteville动物群似乎
代表了鹦鹉螺多样性的顶峰。 

Platteville鹦鹉螺有九目,从直的到不同曲率度的,从开卷到闭卷,再到截断幼螺壳以减轻重量的形状,从轻盈的浮游者到沉积物发育的底栖种类,这些生物活跃在各种职业和层次上。活跃的捕食者、埋伏的捕食者、食腐动物、食草动物,滤食性动物,无数的升斗小民生活在内角石的庇佑之下。在那时,职业不分贵贱,没有人歧视食草者和食腐
者,肉食者也不会得到更多的权利。 

上奥陶统的Platteville群只存在了100 - 200万年。不久之后,Maquoketa组的头足类动物群便仅剩14种,且失去了多样性。14种贝壳中有12种是直壳,过去从事各行各业的鹦鹉螺已经消失。也许是气候变化导致一些
\newpage
物种灭绝了,也许是不断加深的海水吞没了浅海的栖息地,也许是一些食肉者怀了二心,也许是这些因素
的综合作用,或者是其他尚未考虑到的因素。 

但不管怎么,奥陶纪鹦鹉螺的辉煌在以食肉为荣的结局中结束了。在那之后,鹦鹉螺几乎都成为了肉食者,直到它们被外敌所胁迫,不得不转为腐食。
 

冰期和大灭绝结束了奥陶纪。冰河时期被称为赫南特冰期,持续了大约50 - 100万年。海洋中沉积的大量碳酸盐锁住了大气中的二氧化碳,结束了长期的温室效应,全球温度下降,结果在北非形成了一个冰区,海平面随之下降了约100米。85%的物种、61%的属和26%的科死亡。冰川作用降低了海平面,从而暴露出海底。和往常一样,热带物种受到的影响最大——当地球变冷时,温带和极地动物可以向南迁移,但没有热带动物可以避难的地方。紧接着,海平面迅速上升,缺氧的海水淹没了大陆架地区,并摧毁了在寒潮后出现的机会动物群,庞大

\newpage
的内角石的统治也至此宣告结束。 

然而,作为让头足人民挺直腰板,一心为民的第一代统治者,人民永远不会将他忘记。当被他打倒的海蝎卷土重来,当被他制服的鱼类兴风作浪,头足子民们便会不由得怀念这位伟人。终于,内角石被尊为“头足历史上最伟大的人”,以肉身成为头足类心中的神圣。内角石的成神之日,不是他开始统治万民的那天,而是他在人民的心中,成为信仰的那一刻。

在他在所有人心中成为神的那天——世界开始
走向终结。 



\end{document}
