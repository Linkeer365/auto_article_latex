\documentclass{article}
\usepackage[utf8]{inputenc}
\usepackage{ctex}

\title{一只马蜂\footnote{Click to View:\url{https://web.archive.org/web/20221012145951/https://rentry.co/m5ixy}}}
\author{丁西林}
\date{1923}

% \setCJKmainfont[BoldFont = Noto Sans CJK SC]{Noto Serif CJK SC}
% \setCJKsansfont{Noto Sans CJK SC}
% \setCJKfamilyfont{zhsong}{Noto Serif CJK SC}
% \setCJKfamilyfont{zhhei}{Noto Sans CJK SC}
% \setlength\parindent{0pt}

\begin{document}
\CJKfamily{zhkai}

\maketitle


\Large


剧中人: 

吉老太太:年约五十余岁,身材细小,体质强
健,淡素服装,非常的清洁。 

吉先生:吉老太太的儿子,年约二十六七,强
健,活泼,极平常极自然的服装。 

余小姐:年约二十五六,姿态美丽,面目富有
表情,服装精致。 


仆人 

布景:一间小小长方形的房子,后面墙壁中间

\newpage
,两扇宽门。门的左边置一衣架,靠墙一小桌, 

桌上置鲜花。右边靠墙立一书柜,内藏成套的中西文书籍。右壁的里边,开一独门,门前为短门大窗,窗边置写字桌,上置文具。房的右壁,后半亦开一门,前半靠壁置书架,架上置装饰品。壁上悬字画。房子中央略偏前与右,置一小圆桌,上置茶具,桌的右侧置大椅(即安乐椅),左侧置可坐两人的长椅
,两椅之间,置一小椅,椅上皆置腰枕。 

开幕时吉老太太睡卧在大椅上,脚下置高垫,
手中报纸,落地上。 

吉先生:(将左门徐徐推开,见老太太睡卧椅上。轻步走至衣架,取了一件薄大衣,走至椅前,轻轻盖在老太太身上。老太太醒觉。吉含笑问)睡着了
没有? 

吉老太太:我本想闭了眼睛歇一会,不想一不
留心,就睡着了。(坐起) 

吉先生:老人家的眼睛,同小孩子的眼睛一样
\newpage
,闭不得的。一闭了,就不由你做主。(将报纸拾起
,坐在小椅上) 


吉老太太:现在什么时候了? 

吉先生:(由包里取出一个表看了一看)三点
一刻。 


吉老太太:你在哪里一只到现在? 


吉先生:在书房里写了两封信。 

吉老太太:喔,不错,你替我把那封信写了吧
。 

吉先生:好,现在就写。(坐到写字桌,从抽屉里拿出信纸信封,砚里倒了水,磨墨取笔,预备写
字)怎样写法? 

吉老太太:随便的写几句好了。你把我们动身

\newpage
的日子告诉他们,叫他们雇一只船到港口接一接。 

吉先生:你一面说,我一面写吧,一定下星期
二动身么? 

吉老太太:喔,已经不是日子,还再不动身!
 

吉先生:(一面写,一面念,一面说)“……十九日起程回南。”(停笔用手指计算日期)十九,二十,二十一,(写)“二十一日到港。叫张宏同江
妈雇一只船到港口接一接。”(问)是不是? 

吉老太太:是,最好叫到李老四家的船,干净
。要是李老四的船出了门,叫邓祥发家的也可以。 

吉先生:(写)最好叫到李老四家的船。(一面写一面口中低声地念)……邓祥发家的也可以。(
问)还有什么? 

吉老太太:(自己想她的心思)这几天太阳已经很厉害,不如叫他们先把南房里的皮衣服拿出来晒
\newpage

一晒。 


吉先生:好,还有什么? 

吉老太太:没有什么。(自言自语)王妈回家
,说过了节,就回来,不知现在已经回来了没有? 


吉先生:(继续地写信) 


吉老太太:余小姐,应该送她点礼物才好。 

吉先生:(先写完了信,然后答话,再接着写信封)你不是说送她一件衣料的么?(写完了信封)
好了,写完了。 

吉老太太:(被吉打破她的深思)写完了么?

吉先生:(走至椅前,将这信送出)要不要看
一遍? 


\newpage

吉老太太:你念一念吧。 

吉先生:(念信)“二妹览:‘已经不是日子
,还再不动身!’母亲说,……” 


吉老太太:这是写的什么? 

吉先生:这是写信的一个帽子。(继续一句一句的念信)“母亲定于十九日动身。二十一日到港。叫张宏同江妈雇一只船到港口接一接。最好叫到李老四家的船,干净。要是李老四的船出了门,叫邓祥发家的也可以。这几天太阳已经很厉害,不如叫他们先把南房里的皮衣服拿出来晒一晒。王妈回家,说过了节,就回来,不知现在已经回来了没有?”没有写错
吧? 


 

吉老太太:(笑)喔,你们现在写信,都是这
样写么? 

吉先生:这是最时行的直写式的白话文,有一
\newpage

句,说一句。你没有旁的话要说么? 


吉老太太:没有。 

吉先生:这下边是我的事。(继续念信)“这次母亲在京,一切都好,惟有两件事,不大称心……
” 


吉老太太:我有什么事不称心? 

吉先生:(不答,继续念信)“第一,她这次来京的目的,本想劝她的儿子,赶紧讨个媳妇,她可早点抱个孙儿。方头大耳,既肥且皙。暧!不想来京两月,绝少成绩。媳妇,毫无影响,孙子,渺无消息;第二,她满心满意,想亲上加亲,把姊妹改做亲家,侄儿变做女婿。不想她那不肖之女,又刚愎自用,不顺母意。因此上,这几日来,口中不言,心中闷闷,不过那位表侄先生,现已广托亲友,多方物色。夫诚能动神,勤能移山,况在佳人才子聚会之首都,求一称心合意之老婆乎!故数月之内,定有良缘。将来一杯喜酒,或能稍慰老年人愿天下有情人无情人都成
\newpage

眷属之美意也。”说得对不对?不要生气啊。 

吉老太太:(稍有不快之意)我有这些闲工夫来同你们生气!你们的事,我老早就对你们讲过,由你们自己去,我一概不管。你们爱怎么说,就怎么说

吉先生:(将信封好,贴了邮票,走至椅旁,一手放椅背上,一手理她的头发)妈,你是一个特殊的女人,你什么事都是非常。你是一个非常的良妻,一个非常的贤母。惟有这一件,你没有逃出了个母亲
的公例。 

吉老太太:把这件大衣挂起来。(吉将衣挂原处。老太太追想到她以前的生活)“贤妻良母”,配不上这四个字!(吉坐到原处)你父亲死的时候,你只有八岁,云儿也只有五岁。那个时候,我就不相信那私塾先生的教书方法——也一半舍不得你们去受那野蛮的管束——所以我就拿定主意,自己教你们。一直把你教到十六岁。那时所有的产业,就是那分来五十亩坏田。现在你们可以不愁穿,不愁吃。不是说大话,要是你们不是每年上千块的学费用费,现在大约
\newpage

十倍那么多都不止了。 


吉先生:所以我说你是一个特殊的女人。 

吉老太太:是的,贤妻良母,有什么稀奇?现在的一般小姐们不是一天到晚所鄙薄不屑得做的么?

吉先生:你要原谅她们。她们因为有几千年没有说过话,现在可以拿起笔来,做文章,她们只要说
,说,说。连她们自己都不知道说的些什么。 

吉老太太:现在这班小姐,真教人看不上眼。不懂得做人,不懂得治家。我不知道她们的好处在什
么地方? 

吉先生:她们都是些白话诗,既无品格,又无风韵。旁人莫名其妙,然后她们的好处,就在这个上
边。 

吉老太太:我问你,这样的人也不好,那样的人也不好,旧的,你说她们是八股文,新的,你又说
\newpage

她们是白话诗…… 

吉先生:是的,同样的没有东西,没有味儿。

吉老太太:那末你到底要怎样的一个人,你就
愿意? 

吉先生:(耸肩)坏的就是连我自己都不知道。要是找老婆如同找数学的未知数一样,能够立出一
个代数方程式来,那倒容易办了。 

吉老太太:怎么你们表兄弟两个。这样的不同!那一个就请这个,托那个,差不多今天等不到明天
。你总是不把它当成一件正经事看。 

吉先生:不把它当成一件正经事看!因为我把它看得太正经了,所以到今天还没有结婚。要是我把它当做配眼镜一样,那么你的孙子,已经进了中学。

吉老太太:(觉得她没有办法)倒一杯茶给我。(吉倒了一杯茶送给老太太,自己亦倒了一杯,慢
\newpage
慢饮之。老太太沉思半响)你知道不知道,你的表兄
弟已经同我说了几次,要我替他做媒? 


吉先生:怎么不知道? 



吉老太太:你知道他要说的是谁么? 

吉先生:余小姐,是不是?你问过她了没有?


吉老太太:(很慢地答)没有。 


吉先生:为什么不问她? 

吉老太太:为什么不问?(少顷)我想今天问
她——好不好?(语时视吉) 

吉先生:很好,看护妇配医生,互助的原则,
合作的精神,结婚时最好的演说资料。 


\newpage

吉老太太:(微微地叹了一口气) 


仆人:(推开左门)老太太,余小姐来了。 

吉老太太:请她进来(仆人走出,吉放下茶杯
,忙走至写字桌,整理笔砚,折好了桌上报纸) 

(仆人由外面推开左门让余走进,自己随后收
去了桌上的茶具) 

余小姐:(带了帽子手套,一手提钱包,进来之后,一面与主人招呼,一面脱去手套,将钱包置于
门旁小桌上,解下帽子)老太太,吉先生。 


吉老太太:余小姐 


吉先生:余小姐(吉接过帽子,挂衣架上) 

余小姐:老太太,对不住得很,劳你们等了。

吉老太太:没有什么,请坐。(让余坐大椅)

\newpage

余小姐:喔,老太太坐,老太太不用客气,我这儿坐好。(扶老太太坐大椅,自坐小椅,吉自坐长椅上)两点半钟就想来,突然来了一个病人,要替他腾出一间房间来,忙了半天,还打算打电话,说不能来了,后来我想老太太就要回南,无论怎样忙,都要
来陪老太太玩半天。 

吉老太太:多谢你,我们也知道你医院里事情很忙。所以一向不常请你出来。今天是因为我们快要回南,想请你来,我们好当面向你道谢。这一次实在劳苦了你。起先是我们吉先生,住了两个星期,都是你招呼,后来又是我自己,我们实在感激你的了不得

余小姐:老太太太客气,那是我们的职务。老
太太这几天饮食可好一点? 

吉老太太:胃口不强,我一向就是这样,那一次到北京来,因为在路上略微受了一点辛苦,所以觉得不大舒服,实在没有什么病。我们吉先生一定要我到医院去,说医院里怎样的舒服,怎样的干净。我总是不想去。后来他又说我精神不好,一定是睡觉不好
\newpage
,非得到一个清静的地方去静养几天不可。我被他说不过了,方才住到医院去。我出来的时候,他还要我
再多住几天。 

吉先生:我的母亲是不相信医院,不相信看护
妇的。 

吉老太太:我并没有说我不相信看护妇,我是
因为常常听见讲医院里招呼不大周到。 

吉先生:没有什么,你现在不但相信她们,并
且喜欢她们。 

余小姐:我们也知道,外面有很多的人,说我们的坏话,现在不是我来替自己辩护,有时实在不是看护妇的疏忽,实在是这一班生病的太太小姐们的麻烦,我常时同其余的同事说了玩,说这些人什么事不
会做,连生病也不会生…… 

吉先生:要生病生得好,本来不是一件容易的

\newpage
事。 

余小姐:她们第一,就不肯听医生的话,要这样那样,一天要压几十次铃子。你对她们说,叫她们不要吃东西,她一回儿要到外边买些水果,一回儿想叫家里送点鸡汤。你想,要叫我们同平常人家的老妈子伺候太太小姐们一样,我们哪里有这么许多工夫?我们平均每人要招呼十个人。喔,说也是无用,她们
哪里肯讲理? 

吉先生:做看护妇本来是一种很苦的职业,因
为世界上最不讲理的是醉汉,其次就要算病人。 

余小姐:好笑得很,遇到一种奇怪的人,病快
好的时候,他还要你陪他谈天。(看了吉一眼) 

吉先生:那真是可想而知的讨厌。要是个男人,还没有什么,假若是个女人,那恐怕简直没有办法

吉老太太:不过我终是不相信,其余的人,能够同你一样。纵然有你这样的能干,也一定不会这样

\newpage
的和善,这样的体贴。 

(仆人由左门入,手里拿了一个盘,盘中置茶
壶、茶杯、糖罐等物) 


余小姐:(老太太欲倒茶)老太太请坐,让我
自己来倒。(倒了一杯茶送老太太) 

吉老体贴:喔,谢谢你,(吉倒了一杯茶送余
) 

余小姐:(受吉之茶)谢谢。(欲代吉倒茶)


吉先生:谢谢,我不喝茶。 

余小姐:(一面喝茶)老太太为什么不在北京
多住几天?有吉小姐在家,难道还不放心么? 

吉老太太:她倒什么都能够,不过我这次离家
已经很久。我本是因为吉先生病了,所以来看看。 

\newpage


余小姐:我想吉小姐一定也是很能干。 

吉老太太:什么叫能干?不过一个女孩子应该
知道的事,我不容她们不知道。 

余小姐:不过要想能向老太太一样的能干,恐
怕不容易。 

吉先生:做能干父母的子女,是一件很苦的事。暑假那么热的天气,回到家,只有两个星期,两个星期一过,就一个赶到乡里去种田,一个赶到厨房里
去烧饭。 

吉老太太:(笑)我是一个很顽固的人——我现在也有了年纪,也不怕人笑话——我以为一个人多知道一点事,一定不会有坏处。我不相信,一个女人
会做了饭,就不会做文章。 

吉先生:不错。不过困难的不是会做了饭的女
人不会做文章,是会做了文章的女人就不会做饭。 

\newpage

余小姐:吉小姐会到北京来么?我很想认识她
,我想她一定是同老太太一样的和气、可爱。 

吉老太太:她旁的没有什么好处,不过还直爽
。就是我嫌她有点新的习气。 

余小姐:(高兴)我想我们一定会变做好朋友
,她来的时候,老太太一定要叫她写信给我。 


吉老太太:(向吉)你有她的照片没有? 


吉先生:有一张的,不知到哪里去了。 

余小姐:(忆起)喔,吉先生信里,说老太太
要我一张照片,我今天带来了。(走向小桌) 

吉老太太:(不解)我没有说要照片。(向吉
)我几时…… 

吉先生:你怎么没有讲?真是有了年纪的人,

\newpage
说过去的话,不要几天就忘了。 

余小姐:(装不听见,由钱包里取出一张小照片)这一张不大好,不十分像,等以后有了好的时候
,再送老太太吧。(以照片送给老太太) 

吉老太太:(看照片)你已经长得很好看,这
张照片更加好。 

吉先生:(向老太太取了照片,取笑老太太)你平常最讲究会说话的,怎么今天自己把话说差了?你应该说,这张照片固然好看,但是总不及照片的主
人好看。(与余对看了一眼) 


吉老太太:我是说的老实话。 

吉先生:你们还坐一会儿才去吧?(向老太太)我送你一个好看的相片框子。(吉带照片由左门走出。两人不语者片刻。老太太对余注视,余不知所语
,取了一块糖来吃) 

吉老太太:余小姐,我有几句话,很久就想同
\newpage
你谈谈。(将椅移近,余忙将口里的糖吞下,理了一理裙子,坐直了身子,用心地听)我想你一定以为我是一个很爱舒服的人,你知道我年轻的时候,很过了些辛苦的日子。我们吉先生,从小就没了父亲,家里大大小小的事情,都全靠我一个人去问,连他们的书,都是我自己教他们。差不多吃了二十年的苦,才把他们带到这么大。现在他们什么事都用不着我去担心。不过还有一件,我放不了心,就是他们还都没有成家。(余的身子略微地颤动了一下)这一层,我也同吉先生说过好几次,他都不把它当一件事。——我也不知道他到底是什么意思。现在子女的婚姻,本来也用不着父母去管,所以我也只好由他们自己去。(叹了一口气,略顿)我有一个表侄。(余转了一转身子,恢复了自然和呼吸)你大概也认识他,他到医院看过我。他虽然只看见过你几次,但是因为他时常听见我说你怎样的好,所以他很敬重你。他向我说了好多次,托我说媒,我都没有提过。因为我自己儿子的事,我都不管,我哪里有工夫去管旁人家的事?不过他说,他一来不知道你的意思,所以不好向你开口,二来就是想对你说,也没有个好的机会。他,人是一个极好的人,他学的是医道,现在预备自己挂牌行医。
\newpage
他的脾气很好,也会死一点坏的嗜好都没有。——喔,我知道我是一个很腐败的老太婆,说媒的事,是你
们现在最不喜欢的。要是这样,我请你不要生气。 


余小姐:(如梦初觉)我很感谢老太太的好意
,哪有生气的道理? 

吉老太太:他还想在我回南之前,得一个回信。我想这也不是立刻就要怎样的一件事,你如要细细想一想,你回去写封信告诉我,我想也没有什么不可以。(略顿)你的意思怎么样?你有什么话,尽可对我说,你知道我差不多把你同自己的女儿一样的看待

余小姐:(思索了一会,打定了主意)我想我们年轻的人,一点经验没有,什么事都全靠年纪大一
点的人到处指点教导。老太太的意思怎么样? 

吉老太太:喔,这是你自己的事,总得你自己
做主。 

\newpage

余小姐:老太太的意思,如果觉得很好,那自
然不会有错。 


吉老太太:那我就说你很愿意? 

余小姐:不过我想总得写一封信回去,问问父
母的意思。 

吉老太太:不错,不错,自然应该这样。那你
就写封信回去,等你接到家里回信之后,再说吧。 

余小姐:我想单由我写信去,还不十分妥当。


吉老太太:那有什么不好? 

余小姐:可以不可以请吉先生写一封详细的信,把老太太的意思告诉我家里,我再另外写一封,一
齐寄去? 

吉老太太:不错,不错,应该这样。回来我对吉先生说一说,叫他写起一封信来。写好了,我叫一
\newpage

个人送给你,你说好不好? 


余小姐:老太太的主意很好。 

吉老太太:我们还是坐一会,还是就到公园去
? 


余小姐:老太太的意思怎么样? 

吉老太太:我们就去好不好?我叫他们去请吉
先生去。(走去压电铃) 


余小姐:我借你们的电话用一用。 

吉老太太:在那边的院子里,你知道。(余由右门出,仆人由左门入)你去请吉先生,就说我们现在到公园去了。(仆人由左门出。老太太坐回原处,
若有所思) 

吉先生:(由左门入,手里拿了照片,装好了框子。进来之后,将照片放在书架上,看了一看,移
\newpage

动一回)余小姐哪儿去了? 


吉老太太:(沉思中)打电话去了。 

吉先生:(坐到小椅上,取了一块牛奶糖,慢慢取其外皮,随便地问)你的媒做得怎么样,问了她


吉老太太:问过了。 


吉先生:她怎么样讲?(将糖送至嘴边) 


吉老太太:她很愿意。 

吉先生:(将糖由嘴边拿回)她很愿意?她说
很愿意么?她怎样说? 


吉老太太:她没有说什么。 

吉先生:她没有说什么,你怎样知道她很愿意


\newpage

吉老太太:这用不着说的。 

吉先生:喔,不错,这一类的事是用不着明说的,是不是?同天气一样,只要看看天色就知道了。
(老太太对他严厉地看了一看)那么,已经定了? 

吉老太太:她还要写封信回去,问问她的父母
,要等…… 

吉先生:问问她的父母!(解悟)喔!(把一
块糖投入口中) 

吉老太太:你笑什么?你笑她把她的父母太看
重了,是不是?我听了很欢喜。 

吉先生:没有的事!我听了也很欢喜!(又拿
了一块放进嘴去)她说了什么时候写信没有? 


吉老太太:她要请你替她写。 

吉先生:要我替她写!这真奇怪。我又不是她的亲兄弟,亲叔伯,她为什么要请我替她写信,这不
\newpage

是奇而又奇的事? 

吉老太太:你看了奇怪么?我看了一点也不奇
怪。 


吉先生:为什么不奇怪? 

吉老太太:因为——因为还没有认出她。她是一个大户人家出来的女孩子,知道什么是应该说的,
什么是不应说的。她知道害羞。 

吉先生:喔喔!女孩子!害羞!(又拿一块糖
放进嘴去) 

吉老太太:怎么你向来不吃糖的人,今天爱吃
起糖来了? 

吉先生:今天的糖特别有味儿!(高兴,即起
)你们现在就到公园去么? 


\newpage


吉老太太:等余小姐打完了电话。 


吉先生:(想了一想)你不换一件衣服? 

吉老太太:不过是到公园去坐一坐,谁再去换
衣服? 

吉先生:可是天气很凉,不换,也应该加一件
。——在哪里,我替你去拿,好不好? 

吉老太太:我自己去,你不知道。(吉开右门让老太太走出,将门关好,走到书架,取照片在手,细细地审看。将照片放回,在屋里走了两转。余由右
门入) 


吉先生:电话打通没有? 

余小姐:打通了。(注意老太太不在房内,两
人对看了一看) 

吉先生:(将长椅向前稍推)老太太到后面取
\newpage

换一换衣服,叫请你在这里等一会。请坐。 

余小姐:(由女人的直觉知将有有趣的谈判发生,为准备抵御起见,先摸了一摸头发,理了一理裙子,选了长椅离小椅远的一边坐了。吉坐小椅上)老太太真是一个可佩服的人,那么大年纪,穿的衣服,
比年轻的小姐们还要讲究。 

吉先生:一个人什么都可以不讲究,惟有衣服
不可以不讲究。 


余小姐:为什么? 

吉先生:因为人是一个社会动物。一个人生在世上,所有的一切物质上的幸福,精神上的愉快,都是社会给他的。所以一个人对于社会,应当尽量的报
答。 


余小姐:那与穿衣服有关系么? 

吉先生:关系大得很!因为报答社会,有种种
\newpage
不同的方法。有职业的,藉他的职业,有技能的,用他的技能。当兵的可以替我们杀人,做律师的可以替我们打官司,做医生的可以替我们治病。不过还有一种人,——就像我们——既无职业,又无技能,最少也应该有几件好看的衣服,才不至于走到人家面前,
叫人家看了难过。 

余小姐:(笑)哈,我明白了。愈无用的人,
愈应该穿好看的衣服,对不对? 

吉先生:对,不过有用的人,也不应该着不好看的衣服。社会上没有一种职业,我们可以承认他有不顾装束的权利。一个人,自生至死,也没有一个时期,我们可以承认他有无须掩饰的特权。假若一个女人,因为她已经结了婚,就不管她头发的高低,因为她生了儿子,就不管她袖子的长短,或是一个男人,因为他能够诌几句诗词歌赋,就不洗清他的面孔,因为能够画得几笔山水草虫,就不剃光他的下巴,拉直
了他的袜筒,那都是社会的罪人。 

余小姐:这样讲,恐怕我们都是社会的罪人。
\newpage



吉先生:你?喔!(欲言又止) 


余小姐:我怎么样? 

吉先生:你?两个月前,你冤枉说我发烧的时
候,我不是已经对你讲过么? 


余小姐:我冤枉说你发烧? 

吉先生:自然是冤枉。什么温度三十九,脉跳一百多,那都是你造的谣言,——是的。完全是谣言。——不过我很感激你,假使没有你的谣言,我如何能够住到两个星期?喔!那两个星期!那是我一生最快乐的两个星期!(叹)暧,无论怎么,不会再有的

余小姐:(同想到那时的景况)是的,也不知
说了多少话!从来没有看见过这样爱说话的病人。 

吉先生:是的,那都是些极真诚,极平常,极正当的话。为什么平常我们不能讲?为什么要男人装
\newpage
了病,方才可以讲?为什么女人听了一定要冤枉说他发烧?要是现在我说你的眼睛生得怎样的动人,嘴唇怎样的可爱,你会装做没有听见,把我的额角摸一摸,枕头拥一拥,说一声:“现在歇一会儿吧。你说话说得太多!“社会真是一个不自然的东西!这一类的
话有什么说不得?为什么现在不能说? 


余小姐:因为——因为你现在不发烧! 

吉先生:你怎么知道我不发烧?我一年到头,没有一天不发烧。你要不相信,你现在替我试一试。(伸手放在长椅边上,余从长椅那一边,移到这一边,先理了一理裙子,然后用右手把脉,同时看左手上的腕表。约数秒针无语)我病的时候,说了很多的话
,是不是?(余点头)说了些什么? 


余小姐:(余将手缩回)你说中国是一个可怜的社会,男人尤其可怜,除了赌钱,遇不到人家的小姐太太,除了生病,得不到女人的一点同情。所以你

\newpage
一星期要打一次牌,一个月要装一次病。 

吉先生:对呀!这像生病的人讲的话么?——
发烧不发烧? 


余小姐:(犹豫)七十七次。 


吉先生:可见得是说谎。 



吉先生:因为你就没有数! 


余小姐:喔,一个人可以随便说谎么? 

吉先生:自然不能“随便“。不过我们处在这个不自然的社会里面,不应该问的话,人家要问,可以讲的话,我们不能讲,所以只有说谎的一个方法,
可以把许多丑事遮盖起来。 

余小姐:我们从小就知道,说谎是不道德的。

吉先生:道德是没有标准的,随时代随个人而
\newpage
变的东西,平常“所谓“道德,不是多数人对于少数
人的迷信,就是这班人对于那班人的偏见。 

余小姐:这样说,世界上没有善恶好坏的标准
了? 

吉先生:世界上只有脏的习惯是坏习惯,丑的
行为是恶行为。 

余小姐:所以什么谎都可以说,只要说得好听
。做贼,赌钱,都可以做,只要做得好看? 

吉先生:一点都不错。不过世界上美神经发达的人很少。做贼同赌钱的时候,大半都是不大十分雅观。说谎,说得好的人很多,不过我最佩服的是你。

余小姐:我向来不说谎,你说我说谎,你有什
么证据? 

吉先生:对呀!所以佩服你的缘故,就是因为拿不出证据来。不过一个人说谎话说太多了,总有一
\newpage

天,转不过弯来,要露出马脚来。 


余小姐:我向来不欢喜说话。 

吉先生:好吧,白说是没有用的。我问你一件


余小姐:什么事? 


吉先生:老太太替你做媒没有? 


余小姐:(着急)你不应该问这句话。 


吉先生:为什么不应该? 

余小姐:因为这一类的话,连自己的父兄都不
应该问,朋友更加不应该。 

吉先生:喔,新文化!新文化!不过你知道不知道?一个人的婚事,从前,是父母专制,现在因为用不着父母去管,所以用不着父母去问。(吉先生的一件,以为婚姻的事如果不要人帮忙则已,如要帮忙
\newpage
,父母应该是最重要的人物,现在所以不要他们过问,一则因为他们专制,二则也因为他们不能帮忙。这一层似乎还没有人见到,所以附此说明)但是现在的婚姻是朋友专制,要想结婚,非靠朋友帮忙不可,所
以你说朋友不应该过问,是完全错误。 


余小姐:我取看看老太太去。(起立欲走) 

吉先生(起立阻之)不要走,不要走,我还有一件要紧的事,没有对你说。请坐。(两人同坐下)我不在这里的时候,老太太同你讲了很多的话,是不
是? 


余小姐:是的。 


吉先生:她说到我不想结婚的话没有? 


余小姐:说了很多。 


吉先生:你知道,我不想结婚。 

\newpage


余小姐:为什么不想结婚? 

吉先生:因为一个人最宝贵的是美神经,一个
人一结了婚,他的美神经就迟钝了。 


余小姐:这样说,还是不结婚的好。 


吉先生:是的,你可以不可以陪我? 


余小姐:陪你做什么? 

吉先生:陪我不结婚。(走至余前,伸出两手
)陪我不要结婚! 

余小姐:(为他两目的诚意与爱情所动)可以
。(以手与之) 


吉先生:给我一个证据。 


余小姐:你要什么证据? 

\newpage

吉先生:你让我抱一抱!(释其手,作欲抱状


余小姐:(走开)等你再生病的时候。 

吉先生:不过我母亲都告诉我,说你已经答应
了做她的侄媳妇,那怎么办? 


余小姐:(得意)那没有什么,我的父母不愿
意我嫁给医生! 

吉先生:对,我知道,我们是天生的说谎一对
!(趁其不备,双手抱之) 

余小姐:(失声大喊)喔!(老太太由右门,
仆人由左门,同时惊慌入,吉已释手) 

吉老太太:什么事,什么事?(余以一手掩面
,面红不知所言) 

吉先生:(走至余前,将余手取下,视其面)
\newpage

什么地方?刺了你没有? 


吉老太太:什么事?什么一回事? 

余小姐:(呼了一口深气)喔,一只马蜂!(
以目谢吉) 

闭幕

\end{document}
