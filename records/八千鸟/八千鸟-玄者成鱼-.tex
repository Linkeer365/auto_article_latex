\documentclass{article}
\usepackage[utf8]{inputenc}
\usepackage{ctex}

\title{八千鸟\footnote{Click to View:\url{https://web.archive.org/web/20221009024002/https://tieba.baidu.com/p/6344064717}}}
\author{玄者成鱼}
\date{}

% \setCJKmainfont[BoldFont = Noto Sans CJK SC]{Noto Serif CJK SC}
% \setCJKsansfont{Noto Sans CJK SC}
% \setCJKfamilyfont{zhsong}{Noto Serif CJK SC}
% \setCJKfamilyfont{zhhei}{Noto Sans CJK SC}
% \setlength\parindent{0pt}

\begin{document}
\CJKfamily{zhkai}

\maketitle


\Large


一 


遇见严池是在七月。 

具体哪一个年份的七月,我忘记了。年少的日子过得如蔓延无度的洪水,因为无法超生,所以逼迫
自己快速地丢弃。 

但我记得那是七月的某一天,因为七月的天空总是绽开如一朵肆无忌惮的玫瑰,充满着天真和残酷
的气息。 

不管是在哪一年,不管是十年前、十七年前,还是,我没有降临到这个世界之前,所有我看过的描

\newpage
写七月的书籍,都是这么说的。 

那是七月的某一天,具体是哪一天,我当然也忘掉了。因为那一天和那个暑假的任何一天没有任何
不同。 

我起床、刷牙洗脸、吃早饭、上街,然后一抬头,看到一个小小的招牌,上面歪歪斜斜地写着——
八千鸟。 

必须承认我是冲着这个名字走进去的。像我这样的未成年人,本就不应该进酒吧。但是“八千鸟”这个名字吸引了我,为什么不是九千鸟呢?我一下子
陷入这个问题里面。 

“八千鸟”是一个音乐酒吧。在那个七月的黄昏,酒吧里飘着紫罗兰般的香气,不知为什么,这种香气像一连串子弹轰然击中了我,让我无法用正常的
语言去描述那个酒吧的模样。 

因为在香气中,酒吧里的一切都是漂浮着的,仿佛蒙着一层晶蓝色的水雾,我开始恍惚,恍惚中竟
\newpage
然听到了飞鸟扑打翅膀的声音,我不由自主地仰头,在黑色岩石般的天花板上,果然飞着无数只蓝色的鸟
儿,随着酒吧里的音乐鼓点微微震抖。 


“是八千只吗?”我问。 

“是啊。”他笑着,从一个酒台后面走出来,
“不多不少,正好八千只。” 


“为什么不是九千只呢?” 

“因为折到第八千只的时候,这种纸张就卖光
了。” 

这真是一个好理由。他也是一个好老板。穿洒脱的深灰色休闲装,微卷的栗色头发,高大,瘦,脸色苍白。他是无数文艺作品里的男主角,有一个忧郁
的名字,但是我不记得了。 


所以,我干脆用“老板”来代替称呼他。 

\newpage

但是“严池”这个名字,却是代替不了的。严是严格的严,池是池塘的池,两个普普通通的字组合
在一起,竟有一种似曾相识的感觉。 

名字和人之间,也应该存在某种磁场,彼此吸
引,或者排斥,就如同人与人之间的缘分一样吧。 

严池走进来,并没有什么特 别,他推开那扇小小的木门走进来,穿蓝色的衬衣,斜背着一把深红色的木吉他,吉他上落满了那天黄昏的彩霞,可能因为这样的彩霞太过于灿烂,我不由得眯了眯眼睛,然
后看到命运的幕布缓缓拉开。 


二 

很长一段日子里,严池每天黄昏都在驻唱台上唱歌。他唱歌的时候,双手紧紧握住话筒,略略偏着
头,表情似有无限悲悯。 

在我看来,那些歌的曲调老了点,但他的声音是年轻的,有些音甚至唱得稚嫩,与他的表情如此不
\newpage

符,却仿佛有着穿透时空的力量。 

对我来说,这就足够了。哪怕两个小时之后,他就会背着那把深红色的木吉他,匆匆消失在酒吧的
后门,去赶下一场演出。 

每天,我都坐在酒吧的一角,像一个无家可归的孩子,静静地从黄昏听到夜幕渐沉。严池却从未注
意到我——这个坐在台下听他唱歌的女孩。 

这个女孩也和他同样的年轻,也和他同样喜欢喝菩提水,也和他同样喜欢穿水蓝色的衬衣,也和他
同样沉浸在每一个飘扬的音符之中。 

可是,他竟然从来没有注意到我。哪怕有几次,在酒吧里的客人寥寥无几的时候,他也没有把目光
投向我一次。 

我有些失望,也有些不甘心,终于有一天演出结束后,我忍不住向老板打听他的情况。没错,我不敢直接走上驻唱台,问那个沉醉在音乐里的年轻人—
\newpage
—哪怕只是一个粉丝,也有自己的羞涩和骄傲,不是
么? 

老板站在酒台后面,叮叮当当地调酒,脸上挂
着淡淡的微笑。 


我开门见山:“告诉我那个歌手的事吧!” 

他的笑倏地消失了,眼神却如一片轻柔的羽毛,缓慢地落到我的身上,我伸手将羽毛弹开,他轻轻地开口说:“你答应过只会安静地坐在那里听歌,我才让你进来。”又看了看表,“你也答应过我九点之
前一定会离开这儿。” 

“半个小时之后我一定到家,我只是想更了解
他一点。”我极力解释着,“我喜欢他的歌。” 

老板叹了口气,若有似无:“我让你进来真是
一个错误,你真是一个难缠的小姑娘。” 


\newpage

我毫不退却地笑了:“也许一直都是。” 

老板看着我,他的眼神淡得要化掉,因为淡,
所以捉摸不定,所以神秘又忧伤。 

良久,他终于说:“他叫严池,是附近大学的学生,家境不好,来我这里驻唱打工挣学费,我所知
道的就这么多。” 

严池!他真的叫做严池!严格的严,池塘的池。为什么我会在他推门进来的一瞬间,就知道是这两
个字构成了他的名字呢? 

就好像,知道杯子叫做杯子,窗户叫做窗户,
天空叫做天空一样简单自然。 

如果这世上所有的故事,都这样简单自然,该
有多好啊! 


三 

那天,我像往常一样,在九点半之前赶回了家
\newpage

。 

客厅里亮着灯,冷气开得很足,爸妈坐在沙发
上看电视,对我的夜归并不介意。 

似乎从来都是这样,我做任何事情,他们都不会过多干涉,还美其名曰“相信我”。的确,一个像
我这样的好学生是没有胆子走弯路的。 

最多,我只会在午夜万籁俱寂的时刻,打开小说本,将那些得了奖无人分享的喜悦,跌倒后独自爬起的疼痛,寂寞时无处投递的心情,遇到美好事物时
隐秘的激动,沉默地付诸于一个个虚构的故事。 

可是,就连这样的倾诉,也越来越少了。难道
长大,竟意味着与这个世界,越来越无话可说? 

望着空空如也的小说本,我苦笑一声,曾经还以为自己是个文学天才呢,渐渐才明白,期望自己不
平凡,这本身,就是一个再平凡不过的想法了。 

\newpage

卧室的房门轻微地动了一下,我惊警地将日记本“啪”地合上。回过头去,妈妈有点尴尬地望着我
:“我看看你睡了没有。” 

“马上睡!”我歉意地笑笑,为自己的失控感到一丝不安。乖孩子的面具戴得久了,连面对最亲的
人,也不知如何才能卸下。 

妈妈了然地点点头,退了出去,顺带帮我关上
了门:“那你早点休息。” 

一时间,我有些沮丧,难道她从不好奇我在写
些什么?莫非她以为我在练习作文? 

撇撇嘴,我将注意力转移到书桌上来,却发现刚刚合上小说本时动作过大,振动的气流将本子里夹
着的一张照片甩了出来。 

我好奇地翻过来,脑子里“轰”然一响——怎
么,怎么会是他?! 

\newpage

照片上的严池,穿着白色的长大衣,站在星光熠熠的舞台上,用我无比熟悉的姿势握着话筒,成就
一个张口欲唱的表情。 

真的是严池吗?我几乎要把那张照片看出一个洞:他的头发似乎比现在长,额发垂下来遮住眼睛,
看不清那其中深藏的奥秘。 

为什么我会有一张的照片?明明才认识他不久

第二天黄昏,满怀疑问的我,依旧去了“八千
鸟”,却在看到严池走上驻场台时,蓦地止步。 

再次胆怯了,我注定只能窝在小小的角落,手里捏着那张莫名的照片,听他像一架不知疲倦的点唱
机,久久唱下去。 

他今天唱了自己写的歌。那首歌的旋律十分特别,似银瀑四溅,又如海潮汹涌,从悬崖峭壁到寒潭
深渊,一路跌宕起伏,让人止不住地跟随。 

\newpage

就在这样的歌声里,我坐着,渐渐忘记了翩跹的时光,忘记了那些走远了的过往,忘记了曾经说过的话,做过的事,爱过的人,也忘记了那张照片带来
的惊惑。 

“他的声音,有一种魔力。”我对老板说。在这个酒吧里,我只同老板一人交谈。就像只听严池一人唱歌一样。其他来来去去的客人和歌手,如同浮云
,轻轻一吹,就散了。 

老板说:“这种魔力可能只对你一个人有效。严池并不是我们这里最好的驻唱歌手,他太年轻。”
 

老板的话也许是对的。客人对他的热情并不高
。他们在歌声中聊天、说笑。 

对他们来说,严池只是一段背景,是他们放松休闲的一幅无足轻重的背景。如果对碧海蓝天厌烦了
,可以随时换上另一张青山绿水的画面。 

\newpage

我想严池是不在乎的。我希望他不在乎。既然为了生活,来到这种地方唱歌,就不要计较别人的视
若无睹和嗤之以鼻。否则,会很痛苦、很矛盾。 

我听说,人生本来就有太多矛盾和痛苦的事情,但是,至少在这里,在“八千鸟”,我希望他快乐
些。 

如果他知道,有一个人这样喜欢和关注他的声
音、他的音乐,是否,他会快乐些? 

可惜,他从来没有正眼看过我,他并没有注意
到我。 


四 

那之后的一个晚上,“八千鸟”发生了一点小动荡。一个喝醉了酒的客人,一定要驻唱歌手唱一首
难度很高但曲调很俗的老歌。 

那时,在台上驻唱的是一个叫极雪儿的女歌手
\newpage
,她对这种俗气的歌曲鄙夷不屑,坚持不唱。客人便
打碎了酒瓶,叫嚣着开始闹事。 

老板连忙出面调解,却也束手无策。客人一定要点那首歌,极雪儿哭诉他是故意刁难,双方僵持不
下。 

正在老板束手无策之际,驻唱台边响起一个声音,淡淡的,慵懒的,无谓的,严池从备椅上站起来,拨弄了几下吉他,说:“我来唱吧。她今天唱了很多歌,嗓子有点哑了。”那是我第一次听见他说话的声音,原来他说话的声音这样散漫清淡,像一阵抓不
住的风,从我耳边飘过。 

老板趁机打圆场说:“是啊是啊,就让严池唱吧。他对这首歌很熟悉,极雪儿今天唱得太多了,您
就放她一马吧。” 

那个油光满面的客人从鼻子里哼了两声,坐下
来,算是默认了。 

\newpage

严池便上台唱,很高的音调,可笑的歌词,这首歌一点都不适合他。他唱得很吃力,脖子上的青筋
都暴出来。 

底下的客人却极尽喧哗之能事。极雪儿抱着双臂冷冷地站在一旁看着这一切,她有一双清高的眼睛
,可是严池的傲骨和才华,难道会比她少一分吗? 

我的血液慢慢沸腾起来,一股克制不住的冲动支配了我,就在那一瞬,我几乎要冲上去代替他唱!

但同时,一双沉稳有力的手稳稳地压在我的肩
上,又把我压回座位。 


是老板。他摇了摇头,说:“不要上去。” 

“为什么?”我争辩着,“那首歌我也会,我
也能唱好。” 

“我没有钱付你酬劳。”他笑了,但是他的眼

\newpage
睛里是一层迷朦的光,让人隐隐地害怕。 

我只好老老实实地坐在那里,不知道为什么,一向温和的老板,此时却仿佛换了一个人,他的身体里蕴藏着一种巨大的不容置疑的力量,我无法挣扎,
更无法违背他的意思。 


也许,他是对的。 

一曲终了,严池微微地颔首下台,我看到他的
额头闪着晶亮的汗珠。 

“他是一个性格很好的年轻人。”老板突然开口,“其他歌手不愿唱的歌,他愿意唱,替我解围。
” 

我没有接话,在我的心目中,严池就是这样子
的,虽然我也不知道这印象从何而来。 

“你也是一个很好的小姑娘。”老板话锋一转
,竟然提到我。 

\newpage

“所以,我才让你进我的酒吧里来。”他说。
但是我不明白这句话的意思。 

“这是缘分吧,或者说是命运的安排。”老板语气幽幽,“可是,我不知道对你来说,这样的安排
好不好。” 

心里突然涌上一种说不清的酸涩,我望了望老
板的脸,他微皱的眉头仿佛深锁着一个秘密。 


五 

后来的几个晚上,一如既往。我却敏锐地感觉到严池有哪里不一样了。他依旧是不动声色地低着头摆弄吉他,清冽的旋律流泻出来,他的声音里,却藏
着一种压抑。 

我偷偷地跑去问老板。他告诉我说,严池写的
又一首曲子,被唱片公司毙掉了。 

“谁不希望得到别人的认同呢?可是,既要迎
\newpage
合市场,又要坚持自我,实在是太难了。”老板在为严池抱不平,“严池有才华,可是,如果不愿妥协,
就只能一直等待伯乐。” 

看不到希望的前路,还能一直走下去吗?我的心收紧了,如果他从此不唱了,我也不应该有任何怨
言。 

他们说,在这个现实而残酷的世界里,信仰、梦想、感情,被抛弃、被放弃什么的,难道不是常有
的事吗? 


但严池并没有。 


他夜夜唱下去,不知疲倦。 

他的声音,一天比一天沉郁、深切、动人,仿佛一个漩涡,让人无法自拔,只能一路深陷,直至灭
顶。 

时间走得如此之快,我非常害怕,害怕这个夏
\newpage
天过去得如同从前所有的夏天一样那么快,害怕严池
会像从前我所有的偶像一样,一夕忽老。 

此时,他还那么年轻,这是他年轻的时光,我站在年轻的不染风尘的他的面前,听他年轻的声音充满激情的歌唱,这仿佛就是一场梦,我害怕梦醒得太
快,一切都来不及。 

七月,转眼就过去了,我和严池却依旧没有任
何交集,反而和老板的谈话越来越多。 

老板其实是一个神秘又有趣的人,知道很多稀奇古怪的事情,常常把人逗笑。他不喜欢我叫他“老
板”,常常自称为“摆渡人”。 

我问他为什么要这样称呼自己,他笑而不答。我又问他有关“八千鸟”的问题,第一次见面,他说
是纸张用光了的缘故,可这是太牵强的理由。 

“其实这与一个传说有关。”禁不住我的一再

\newpage
询问,他终于开口。 


“什么传说?”我很有兴趣。 

“很久很久以前,在人类没有出现之前,万物之间是没有距离的。因为宇宙之神亲手做成了八千只
蓝色的鸟儿。 

这些鸟儿实际上是时空鸟,能够用翅膀搭成通过时空之河的桥梁。有了八千鸟,所有生灵都能够随
心所欲地前往任何空间与任何时间的交叉点。 

那是一段非常快乐的史前期,没有了时空的阻
隔,任何生物都感受不到孤独。 

后来,人类诞生了。人类的欲望太多,而执念也太强,他们为了一点点小事就滥用八千鸟,让每只
鸟儿都不堪重负。 

终于有一天,八千鸟叛逃了,她们飞越迷雾森林,从此销声匿迹。失去了八千鸟,时空的大河再次

\newpage
阻隔于众生之间…… 


而世界,就变成了你所知道的这个样子。” 

“所以,你的酒吧叫‘八千鸟’,是想重新摆渡在时空中徘徊不定的相聚的愿望吗?”我笑着,心
中却升起隐隐的不安。 

“是这样的,让亲人不再分离,朋友能够聚首……不过……”他话锋一转,“这只是传说而已。你
相信吗?” 


“相信什么?”我明知故问。 

他望着我,眼睛里是一层迷蒙的光,那欲言又
止的表情,在变幻的灯光下,显得暧昧而可疑。 

我感觉到一种不平常的东西正在慢慢滋生,一时间害怕起来,连忙岔开了话题:“严池要驻唱到什
么时候?他开学之后还会来吗?” 

“说不准了。时候到了,该结束的必然会结束
\newpage

。” 


这句话真让人沉默。 

老板却突然说:“既然你这么想认识他,又不敢直接走到他的面前,为什么不采取迂回点的方式呢
?” 

“比如呢?”好像又有一点希望从沉默中升起

他指了指头顶,那里,八千只鸟儿熠熠生辉。


六 

得到了老板的指点,我开始给严池写信。将天花板上的鸟儿取下来,可以将支言片语写在她的翅膀
上,再重新挂上去。 

这种方式简直匪夷所思,可老板非常肯定地说
,严池会收到的。 

\newpage

我相信了他。没有缘由的,就好像我没有缘由
地走进了这间叫“八千鸟”的酒吧一样。 


可是写什么呢?拿笔的手都微微发抖。   

我写:“严池你好,我很喜欢你的歌。”太直
白了,擦掉重来。 

又写:“严池你好,你的歌唱得真好,我会永远支持你。”太俗气了,像一个傻乎乎的歌迷,擦掉
重来。 

再写:“严池你好,你注意过我吗?我一直在某个角落默默关注着你。”天啊,太恶心了。擦掉擦
掉。 

我不知道写什么了,任何话语都显得那么矫情多余。纸鸟温柔地立在我的手心,但她却无法替我承
载一言一语。 

踌躇太久,以至于最后落到翅膀上的诗句,已
\newpage
经沾满了历史的风霜——“不惜歌者苦,但伤知音稀
”。 

这是《古诗十九首》中《西北有高楼》中的一句。他会懂得的吧,从来都是千金易得,知己难求,
千百年前如此,千百年后亦然。 

古往今来的人们慷慨以歌,也许求的,并不是
出名,并不是得利,而只是一个相通的灵魂。 

我不知道严池什么时候才能看到这些字。他唱歌的时候总是略略偏着头,从不向头顶上望,我想他也从不知道“八千鸟”的传说,也许我要等待很久。

可出乎意料,当天晚上他就发现了。演出完毕之后,老板对他耳语一句,他便没有立刻离开,而是
将鸟儿取了下来,小心翼翼地展开来,低头去读。 


我傻傻地坐在那里,死死地盯着他的表情。 

但是我的眼前一片模糊,使劲揉了揉眼睛,却
\newpage

还是看不清楚。 


连中考都没有这么紧张过。 

他在另一只纸鸟上给我回信,老板转交给我,
他写道:“你是谁呢?我是否见过你?” 

严池是否见过我呢?我不知道,他的目光还是一次也没有落到我身上,我也没有勇气走上前去自我
介绍,但我想,总有一天他会见到我的。 

从那以后,我在纸鸟上的落款全部都是三个字:八千鸟。我希望自己可以像这些时空鸟一样,能让
我们之间没有阻隔。 

那些日子,是我所能记得的,最美好的一段时光。我们通过八千只纸鸟交流。这是唯一的方式,唯
一的交集,在我看来,也是最好的。 

我们渐渐地熟悉起来,纸翅上的熟悉,没有负

\newpage
担,没有顾虑,更能随心所欲地沟通。 

我们无话不说,甚至谈到“爱与死亡”这样的终极话题,我从来不会对任何人提起。因为同学朋友只会觉得我故作姿态,而老师家长则会认为我没事找
事想太多。 

但是他很认真地回答了我:“我觉得爱和死亡只有一个相通之处,那就是她们都是永恒而不可抗拒
的,我一直希望能用自己的音乐诠释她们。” 


他用“她们”这个字眼,和我的习惯一样。 

我们有太多相同的地方。比如都喜欢喝菩提水,都喜欢穿水蓝色衬衫,都喜欢沉浸在飘扬的音符中


但我们却是不相识的。 

终于有一天,他问我:“你在哪里呢?如果可
以,我们见面吧!” 

这是一个太难的问题。我不知道我在哪里,轻
\newpage

轻捧着那只纸鸟,我彻彻底底地迷失了方向。 


于是没有回答。 

第二天,他在纸鸟上写:“这个要求太唐突了是吗?对不起,只是我要离开这里了,所以想见你一
面。” 

他要离开了吗?这么快?我如遭雷击,这些日子恍若南柯一梦,梦醒后才惊觉夏天已经走到尽头。


七 

严池要离开“八千鸟”是不容置疑的事情了。

老板告诉我,虽然上次他的曲子被唱片公司高层毙掉了,但却有幸获得了一个有名的制作人的青睐,制作人签下了他,也许他很快就可以发第一支单曲


我由衷地为严池高兴,因为他值得。 

\newpage

但失落却不可避免,得知消息的那晚,回家后,我再次打开了小说本,望着淡蓝色的格子线,一种很迫切的渴望迅速地从脚底攀爬上来,我拿起笔,突
然想写一个新的故事。 

没有读者,没有关系;被退稿、被忽视、被嘲
笑异想天开,也没有关系。 

严池,不也这样一天天继续唱下去了么?如果,严池能读到这篇没有读者的小说,我内心的空洞是
否就能小一些? 

我奋笔疾书,那张被揉捏过度的严池的照片就放在一旁,他穿着白色的大衣,额发垂下来,遮住眼
睛,遮住不能言说的奥秘。 

我写得太入神,以至于连妈妈走进来给我送一杯牛奶都没发觉,直到,直到她轻描淡写地说了一句:“原来你还在喜欢他呀?我还以为你已经有了新偶
像了。” 

\newpage

我一惊,手中的笔滑落下来:“什么意思?”

妈妈指了指那张照片:“这不是严池吗?我记得你刚上初中就喜欢他的歌了,那时有同学笑你喜欢
这么老的歌手,你回家后还大发脾气呢!” 

“老?”不知为什么,我听见自己的声音在不
自觉地颤抖。 

“对啊,”妈妈奇怪地看着我,“这不是你说
的嘛,那会儿他不就已经快四十了吗?” 

可是,我明明刚刚还见过他——二十岁的他。

脑中一直被拧紧的水龙头突然爆开,哗哗涌流的疑问和猜想,堵得我不能呼吸:为什么严池从来没有注意到我,连看了纸条之后也没有发现我,我却第
一次见到他就知道他的名字,还拥有他的照片? 

为什么除了老板,从来没有人和我说过一句话?为什么老板要自称“摆渡人”?为什么老板要给我
\newpage

讲那个传说?难道,难道—— 

这个猜想的力量如此强大,明知很荒谬,我却
怎么也抵抗不了。 


我想只有一个人,能回答我的问题。 

再次来到“八千鸟”,我已经平静下来。无论
如何,我需要一个答案。 

老板见到我,什么也没说,先把一只纸鸟递给
了我。 

我打开来,是严池熟悉的字迹。他写道:“今天是我最后一次在这里驻唱,也许以后不再有机会见
到你了。 

但我会一直记得那一天,1992年7月16日,你的纸鸟奇迹般出现在我的手心,你写给我一句
话,用了一句美丽的古诗。” 

\newpage

1992年7月16日——我盯着这行数字看了很久,盯到眼睛发酸,有泪掉下来,然后我抬起头来,对老板说:“我记得今天是2011年7月16
日。” 

老板望着我,眼底似有怜惜:“事到如今,我也不能再瞒你了。没错,这个酒吧和你的世界不是同一个时空。我们只是穿梭停留在了这里。我们的时间
,还是十九年前。” 

“还记得那个传说吗?‘八千鸟’消失之后,为了平息万物恐慌,神便创造了这个酒吧,负责摆渡
时空之河两岸真挚的愿望。 

但不是每一个生灵都有机会穿梭时空,我或者酒吧本身,会选择有缘的人,吸引他们进入酒吧,为他们倒转时空,帮助他们实现心底的夙愿。你,就是
其中的一个。” 

“为什么是我?为什么让我进来?”我咬紧下

\newpage
唇,不让她发抖。 

“你是一个聪明但孤独的小姑娘,渴望有人理解,也想要了解别人。严池的歌,深深打动了你,你
觉得,在这样的音乐共鸣中,你们是平等的。 

他不再是一个高高在上的偶像,你也不是一个泯然于众人的小粉丝。你一直如此期望着,所以,‘八千鸟’给了你这个机会。” 老板停了停,似乎在
给我时间思考,“但是现在,你后悔了吗?” 

后悔吗?后悔走进这个酒吧?在这个七月的黄
昏?当然不! 

我一直想捕捉到严池年轻时的时光,我一直想与他成为灵魂中的朋友、知己,哪怕只是一瞬间,哪怕实现得如此不可思议,哪怕之后一切成空,我也不
会后悔。 

“小姑娘,‘八千鸟’的规则是,如果当事人要求,任务结束之后我可以帮你忘掉这一切,你需要吗?”老板还是那样看着我,眼睛里蒙着一层薄薄的
\newpage

雾气,我终于知道,那雾气里包裹着什么秘密。 

这秘密,让我终于不再平凡。我要带着这段时间的记忆,离开这里,离开他所在的时空,开始自己
的生活,但我并没有任何遗憾。 


八 

这是严池在“八千鸟”驻唱的最后一晚,也是
我们在这场时空游戏中的最后一晚。 

他站在驻唱台上唱了整夜,深澈如海水的声音淹没了“八千鸟”的每一个角落,俘虏了每一位客人

这是他唱得最好的一次。每唱完一首,掌声经久不息,这是感谢,也是鼓励——美好的东西,终有
一天会获得认同的吧! 

最后一首歌。酒吧的灯光突然全部暗下来,只有一束清蓝的光打在严池的身上,他的水蓝色衬衣泛起梦幻般的色彩,让整个人如在天空深处般圣洁冷峻
\newpage


他说,那么认真地一字一句地说:“下面这首
歌,送给一个名叫八千鸟的女孩。” 

他开始唱,双手紧紧地握住话筒,略略偏着头,额前的头发垂下来,遮住了眼睛。这个表情,我如
此熟悉,看过千遍万遍。 

我哭了,但是我知道,在他的世界里,我的眼泪是无声的,所以可以更加放肆地大哭,直到哭声淹
没那首他缱绻而唱着的,古老的诗歌: 


“…… 


清商随风发,中曲正徘徊。 


一弹再三叹,慷慨有馀哀。 


不惜歌者苦,但伤知音稀 


\newpage

……”              


十二点的钟声响起,童话结束了。 

酒吧里如此安静,客人都已经离开,包括老板
。我感激他,给我这个最后的机会。 

严池在收拾乐器。他的动作轻柔缓慢,如水流


我知道他看不见我。 

第一次也是最后一次,我这么勇敢地走到他的面前,去向年轻的他告别,去向那些年轻的时光告别

然后,我把最后一只纸鸟轻轻放在酒桌上,转
身飞跑出了那个酒吧,跑出了他的世界。 

我不敢停留,怕稍一停留,就忍不住热泪奔流

那只纸鸟上写着:“虽然我们的世界是不同的
,但我们的灵魂是相通的。我会一直在。” 

\newpage

我想,他一定能够懂得。         
      

从此,我再也没有见过那个叫做“八千鸟”的酒吧。我不知道她穿梭到了哪个时空,也不知道她又一次摆渡了哪些愿望。但是,我仍然能够看到严池。

在那些大大小小的电视屏幕上,在那些流光溢彩的舞台上。他唱歌的时候,双手仍旧紧紧地握住话
筒,略略偏着头,是无限悲悯的表情。 

他瘦了,成熟了,稳重了,那些年轻的充满梦幻与激情的日子已经远去,声音也慢慢脱离了稚嫩,
却仍旧有着穿透时空的力量。 

对我来说,这一切就足够了。

\end{document}
