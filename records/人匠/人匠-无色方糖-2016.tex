\documentclass{article}
\usepackage[utf8]{inputenc}
\usepackage{ctex}

\title{人匠\footnote{Click to View:\url{https://web.archive.org/web/20160406064853/http://www.zhihu.com/question/39314386/answer/88531424}}}
\author{无色方糖}
\date{2016-03-29}

% \setCJKmainfont[BoldFont = Noto Sans CJK SC]{Noto Serif CJK SC}
% \setCJKsansfont{Noto Sans CJK SC}
% \setCJKfamilyfont{zhsong}{Noto Serif CJK SC}
% \setCJKfamilyfont{zhhei}{Noto Sans CJK SC}
% \setlength\parindent{0pt}

\begin{document}
\CJKfamily{zhkai}

\maketitle


\Large


1. 

在我七岁的时候,父亲亲自斩下了我的左手。
 

他说,做我们这个行当的,得有保命的本钱。
那年我太小,哪里懂得这句话的意思。 


父亲说的行当,是人匠。 

世上有画匠,木匠,瓦匠,也有人匠。人匠的手艺,是罕有的手艺。不是精湛纯熟到极致,火候老
道的人,是万万不敢提起自己人匠的名号的。 


\newpage

这手艺的神妙,我亲眼见过。 

父亲的双手,像是有种魔力。他曾经单手拆下来一位老农的胳膊,断口处平滑如玉,没有一丝血迹。之所以用拆,是那个动作真的轻巧流畅,就像是摆弄木偶。他两指在胳膊上划过,被农具刺穿的伤口像是墨水一样散开,又消失不见。父亲反手轻轻一触,
那胳膊又接了回去,浑然天成。 

他曾经给一个脑满肠肥的大汉瘦身,父亲手一打过去,那一团耷拉的肥肉就像是软泥一样滑落下来
。 

他用指甲轻轻滑过,就能给你开添一个双眼皮
。他轻轻敲打,就能纠正你绞痛的肠胃。 


我曾经问父亲,到底什么是人匠。 


父亲只说了两个字。 


“修人。” 

\newpage


2. 

我十二岁的时候,父亲拿来厚厚的一本册子,
沉声问我 


“当不当人匠?” 


我当时的回答是,“当。” 


“好,这是祖师爷留下来的。好好读。” 

此后每日,我都会细细品读这本古书。书里记载的都是玄异的技法,我常常通读入迷,茶饭不思。

我读那古书读了数月,感觉已经烂熟于心。父
亲又叫我过来,一一问我。 


“那书有几章?” 


“十一章。” 

\newpage


“第六章讲了什么?” 


“《离骨》” 


“做给我看。” 

我低下头来,用食指在中指的一个指节轻轻划
过,一节指骨便呈在了手上。 

这样说来有几分诡异,甚至于恐怖。但没有丝毫痛感,也没有任何不适,指骨被完整的抽离出来,干净的像是一段玉玦。我中指轻轻一动,那指骨便又
回到身体。 


父亲点点头,他蹲下身,直视着我的眼睛说 

“人匠可以修人,也可以杀人。心术不正的人匠夺人器官,取人性命,自古有之。你将来离家的时
候,带上我那柄伞,以便与别的匠师相认。” 

说完,他让我闭上眼睛。用双手的大拇指划过
\newpage

我的双眼。 

我睁开眼睛,发现目力更加敏锐,甚至可以清
晰点数手上的汗毛。 


唯独看不见父亲。 


3. 

母亲是很温柔的人,跟父亲的严苛截然相反。
从我十二岁那年,我跟她相依为命。 

她对人匠事情绝口不提,她是个本本分分的妻
子,本本分分的母亲。 


但我是不安分的。 

十二岁的我,学会独立,学会家务,唯独没有学会怎么安稳。我在家闲不住,又是满脑子好奇心的年岁,总是问母亲各种问题。而母亲肯回答的甚少,

\newpage
只是反复念叨四字家规“心善,人善。” 

我闲的发慌,只好磨练玄妙的技法。偶然间,我突发奇想,自行构想了些需要双手并用的技式,然
后又心凉下来,想起自己其实只有右手。 


我有的只是遗憾,不是怨恨。 

自那后,又过了平淡的四年。在我十六岁生日的早晨,我发现母亲抱着黑色的长筒站在门口,脸上
满是泪痕。 

她哭的眼睛红肿,哽咽着问我说,你想知道我
为什么会跟着你爹么。 

我摇摇头。母亲虽然没有富贵的出身,却是真正的美人,眉眼如画。那不粘脂粉的秀美气质,也不是轻易可得的。父亲则相貌平平,过人之处,也就是
独到的手艺罢了。 

她说:“他当年背着这长筒,身上就两个铜钱,却也要买一个馒头给饿坏了的我吃。他舍了一切,
\newpage
把我从那里救出。你父亲修了一辈子人,唯独修不好自己。我知道你技法精湛更胜他人,但你最需要学是
父亲的善。” 

我点头,不知道回答些什么。而父母曾经经历
过什么,所说的“那里”又是什么,我全然不知。 

她抱着我,又要哭出来,她说:“你是程家的孩子,注定要游历四方。你十六岁了,我把这长筒交给你。里面有伞一柄,信一封,玦一块。我不懂这物件的用处,只知道那古训。‘遇危难,开伞。至境界
,阅信。见故人,持玦。’我能给你的就这些。” 

我不知道母亲在哭什么,却也想跟着哭。内心要离家的冲动和热血在一瞬间结冰,我什么感觉也没
有,什么也不愿意去想。只想跟着她一起站着。 

我呆呆傻傻的走出门去,母亲深深地鞠躬。我
第一次见她这样伤心欲绝,她别过头去说 


\newpage

“儿,娘很想你,但别回来。” 


4. 

父母为我起名为善。我叫程善,也许是寄希望
于可以万事成善。 

但我出门的第二天,便在山路见遇见了山贼。那是通往皇城的必经之路,没想到最近也是山贼肆虐。我想起了母亲说的“遇危难,开伞”,便从黑色的长筒里抽出那长伞,墨色的大伞上面满是繁复的雕文
,让我眼花缭乱。 

我从马车上跳下来,那一众山贼看了我的大伞,全都呆了。有几个胆识大的,气血盛的年轻人想要冲上前来,每当要靠近我这黑伞,都四肢僵硬,动弹
不得,更近的就浑身抽搐,痛苦不堪。 


“别动!” 


那山贼的头子呵道。 

\newpage

“是程家的黑伞,都不想活了?再近一点,就
要变一团烂泥喂给猪狗!” 

我看那几个山贼面色实在是苦不堪言,于心不忍就把伞合了起来。但即便如此,有几个气力弱的还是步履蹒跚。我又只好把黑伞收进长筒里,那几个人
才恢复如初。 

头子走了下来,满脸堆笑的看着我,让我满身
不自在。 

“程家的少爷,皇城里面据说有大恶作乱,去
那里做什么。” 


我回答说 

“听闻圣上寻找天下能人异士,聘金不菲。我
去那里,讨个生活。” 

“小少爷呦,程家人哪里还需要讨生活。”头

\newpage
子说完见我面有愠色,便识相的走上山区。 


只是那人,走前细细地打量了我的左袖。 

想必他已经发现了我没有左手,我也没有太过放在心上。只是我渐渐发现,只有一只手的情况下,的确有很多技式使用起来相当不便。如果那山贼想在
这上面做点文章,可能是个麻烦。 

等山贼都走后,车夫突然从马上翻下来,然后
开始放声大笑。 


是个身材娇小,面容俊秀的女孩。 

其实,自从父亲轻划过我的双眼之后,我的目力精锐,已经不能以常理考量。我早早透过她的面纱
看穿她的相貌,只是没有说穿。 


“小屁孩,没想到老娘我是个女的吧。” 


我微笑着点头说“没有。” 

\newpage

“你不出手,我就把那几十个人全都放倒啦。
” 

我又笑着点头,配合着说:“有女侠护佑,我
当然放心。” 

我这么配合,只是想看她什么时候能切入主题
,满足她的好奇心。 

“小子,你那伞挺有意思的,能给我看看么。


5. 

她叫明彩,自称是武功最好的画师,画工最好
的侠客。 

她乔装打扮,竟然只是为了能顺利上山征伐山贼。我很难想象这样一个满脑子江湖梦的丫头,会甘愿当一个宫廷画师。但事实就是如此,就好像曾经最
讨厌礼法的我,要进入皇家这种循规蹈矩的地方。 

\newpage

程家的名声不小,但大多都是民间的传说,已经与事实相去甚远。所以听说我是程家人,还以为我有什么夸张的威能。但我说道人匠的技法的时候,她
还是很是吃惊。 

而我把她的左臂像车轴一样轻松旋转了两圈后
,她差点吓得晕死过去。 

我说“这算什么,要是我想,都能把我胳膊接在你身上。只是一是我只有一只手,很不方便,二是
父亲当年明令禁止我这样做。” 

她对我的左手相当感兴趣,因为民间都说,程
家有着天赐的双手,但是到我这里只有一只。 


这个问题,我没法回答。 

十六岁的我涉世未深,阅历尚浅。有明彩这种同龄人相伴,是为数不多可以缓解心头焦虑的事情。

只是明彩不时提出的问题,常常让我哭笑不得
\newpage


“程善,你可以把我变美喽?”明彩很兴奋的
问我。 

我回答说“可以是可以。但是你挺美的啊。而且给人更易面貌的技法是最考验人匠经验的,像我这种毛头小子,当然是不敢做这种细致的活,而且……


而且,我只有一只手。 

“好啦,我是不会难为你这种小毛孩的。”明
彩摆摆手,满脸写着刻意的大度。 

“我是在想,程家人把另一个人塑成圣上的身
躯和模样,是不是可以偷梁换柱呀。那还得了?” 


6. 

我们在路上走了数日,又在皇城的客栈住了两
天。 

\newpage

她全然不怕我,不但不怕,还很泰然,甚至是
放肆。总是挑衅我让我开伞给她,我都拒绝了。 

我说,你画幅画给我吧。画的好了,我便给你
开伞。 


她笑了足足有一刻,止不住。 

明彩作画的时候问我,说:“你们程家人可以化男女老少,胖瘦美丑,这画像到时候也不尽然像你
啊。” 

我说:“我喜欢我这张脸和身体,是不会改的
。再说,又不是画我。” 


“这画像不是画你的么?”明彩有些疑惑。 

“当然不是,我要自己的像做什么。我要你的
画,我想看你。” 


\newpage

明彩的脸红透了。 


她沉默下来,安安静静的为自己画了一幅。 

那时我还没懂,人可以修成画,画却不能化作
人。 

“像,真的是太像了。”我看着那幅画不禁咋
舌惊叹。 


“我画自己,想不像也难啊。” 

我知道,明彩这谦辞是站不住脚的。对于画师来说,画他人像,抬头就能看见,那人若是好好配合,神态动作又不曾更易,当然容易。而明彩只是对着这张无暇白纸,凭空从脑海里画出自己。明彩端着那
画像时,就如同持着一面铜镜一般。 

可能是我见识太少,但在我眼中,这种画工说
是绝世无双也不为过。 

明彩作画时那种入迷痴醉,也是我之前见所未
\newpage
见的。我忍不住连连称赞她,她终于也有觉得害羞的
时候,连忙避过身去。 

我问道“明彩,你还有没有别的画,拿来给我
看看。” 

她点点头,从自己背着的木箱里抽出十几幅画卷。其中花鸟,草木,男女老少,鸡犬牛羊,无一不
活灵活现,细致入骨。 

只是这山水,楼宇,顽石,连云,晴空却显得单薄失色,空洞无味。与之前说的那些,画工相去甚
远。 

我仔细端详,不禁发问:“明彩,为何你画活
物妙不可言。但是画其他的却如此苍白?” 


明彩没有回答我,她只是莞尔一笑。 


7. 

\newpage

从客栈离开时,掌柜的特地来嘱托我们二人。
他说 

“听闻现在皇城不安定,弄得是人心惶惶。有
大恶人!” 


我问:“什么恶人?” 


“程家!” 


他说完这话,明彩忍不住瞥了我一眼。 


“程家?”我反问。 

“就是,就是程家”掌柜的说到这里,战战兢
兢,声音发虚,摆手让我靠近些。他低声说道 

“现在有个程家的大恶,在城里,找那身体健壮的小伙子,面容俊美的姑娘,拿去做‘人模子’。

明彩憋不住好奇,她问:“人模子是什么?”
\newpage


“小姑娘你不知道,那程家把人一掌打成烂泥,皮,肉,骨分的清清楚楚。好的心肝脾肺,全拿去给达官显贵用。貌美姑娘的皮囊,都留去换给宫里的
妃子。你生的俊俏,更要小心才是啊!” 

我们走出客栈后,我沉声说:“要是我找到这恶人,就拿程家的古刑伺候他。把他头颅拿下来,保他不死。再去他的舌头,让他求生不得,求死不能。

我看到明彩惨白的脸色,露出笑颜说:“我也只是听父亲说起的。这古刑曾经是处置违反家规的族人,但至今不知过了多少年月。程家人也渐渐不再过
问世事,那严苛的刑罚也就废弃了。” 

我们两个走了良久,一直相对无语。她欲言又止,让我心里不太安稳。我们一直走到一个僻静的路
口,再往下,就不同路了。 

明彩尝试着笑了下,笑的很浅,她说:“记得我说过什么吗。我怕的是,你技法太过神妙,若是进
\newpage
了皇宫,是宫中人身上的肉刺。他们要是不除了你,
也会利用你。” 


“你怕我作恶?” 

“你是白纸,我怕被染了色,在上面画了些妖
魔。” 

“女侠去哪了?你这时候又像个弱女子。”我
只好这样避开她的话锋。 

她别过头去,又转回来,那神色又像是曾经的
明彩。 

“小子,过了这个路口就没有本女侠罩着你了。你好自为之吧,哈哈。”眼看我转身就要走,她一
把按在我肩上,说 

“别忘了,那天我给你画像,你答应给我开伞
的,想反悔?” 

\newpage


我摇摇头说: 

“哪里哪里,明女侠的约,我哪敢反悔。只是这伞高大,在那屋里不便展开。你站远一点,我就开
伞。” 

明彩离了我有四丈远的时候,我喊道“别逞能
,要不要再离得远点?” 

“老娘我天不怕地不怕,区区一把破伞,不能
奈何得了我!” 

我便放心的把黑伞打开,古奥的花纹覆盖了我
的视线。 


“好了么?”我问。 


没有回答。 

我合上伞的时候,明彩已经跑远了。她是习武之人,我知道。在这小路上轻巧无比,如蜻蜓点水。
\newpage
但我还是一眼看见她在那路的尽头,一边飞奔,一边
哭。 


我心海里惊起涟漪,只在想,她哭什么呢。 


8. 

那年我十六岁,缺了些责任和担当。想的,都很浅。所以我不会太在意母亲为什么会哭会那样伤感,明彩为什么要跑要不辞而别。即便在意,也很快被
时间冲淡,在意几日罢了。 

明彩在那里跟我分道扬镳之后,我自己向着皇城的内城走了一日。路上的我突然惊觉,一时间差点
要叫出来。 


这丫头,该不会对我有点意思吧。 

我摇摇头,决定把这些念头抛在脑后。我当时一心想着入宫,只想着要找到那程家恶人:如果皇城里有恶,那宫中一定有大恶。就好像天下有恶,则居
\newpage

高位者中必有大恶。 

内城近在眼前,那里的小门是我进宫的入口,
远处只看见几个身披甲胄的护卫。 

我的确是不懂武艺,所以当他们看到身材纤瘦
,体质文弱的我相视讪笑也是理所当然。 

领头的护卫把佩刀按在桌上,上下打量我,又瞧瞧我左手的位置,摇摇头说,你,活脱脱一幅残废
样,能会点什么呀? 

我深深鞠躬说,兵爷,小弟武艺稀疏,只涉猎
了些旁门左道。 


说完,他们又是一阵哄笑。 


我只好右手轻轻一指点在那领头的额上,说 


“失目。” 

\newpage

那人的眼窝深深的陷了下去,空洞的双目像是
干涸的井口。 

众人惊慌大叫,有抽刀咆哮的,有瘫倒在地的
,有面色苍白的。 

我手一离开,那人又恢复正常,止不住的粗喘
。他大汗淋漓,言语颠倒,像是失了魂。 


我又一次鞠躬说 


“各位兵爷,麻烦行个方便。” 

领头颤颤巍巍的递给我一个黑铁腰牌,说:“进去之后…,找…,找王总管。他会好好安顿你。”他慌张的看向我,眼神却不觉间锁在我背后的长筒上

我道谢之后,走入城里。恰是秋风过境,我身形不稳,像要化在风里。一众护卫,只远远观望,无
人敢上前一步。 

\newpage


大概,恶人,以恶慑。 


9. 

我见王总管的时候,正听见他在训斥手下的侍
女。 

“你干活再这样毛手毛脚,小心被罚去‘废人
居’!” 

那侍女听罢大骇不已,吓得花容失色,连忙跪下要自扇耳光。王总管看见我来的时候,一手扶起那
侍女,轻声吩咐这般那般。 


那侍女抹去泪痕,小步走到我身前行礼。 

“大人请跟我来,‘异人居’就在不远处。”

我微笑点头,与那侍女走了稍许,见四下无人
就低声问: 

\newpage

“姐姐,我好奇那‘废人居’是什么去处?”

侍女满脸惊惧,她看着我退了半步,说:“大人,那‘废人居’里面可不单单是废人,尽是些妖魔
。” 


“我只是打听而已,并无他意。” 

侍女环顾了片刻,与我耳语道:“听闻里面有什么单眼的老头儿,四腿的妖婆,无嘴的异童。前几日有几个姐妹去里面清扫,活脱脱吓得昏迷了两三日

我面上不惊,心里却起了阵阵波澜。这些所谓的妖魔,听着都像是程家的手笔。人匠可以修人,自然也可以害人。跟我猜的别无二致,让皇城百姓人人
自危的大恶,应该就在这宫里。 


“那姐姐知不知道这‘废人居’怎么走?” 


侍女面露难色说:“奴婢不敢说。” 

\newpage

我语气和缓地说:“那我也不为难姐姐了。世上哪里有如此畸怪之人,估计只是相貌生的奇异丑陋
,以讹传讹罢了。姐姐也不必放在心上。” 


她点点头:“奴婢也希望是如此。” 

她将我送到异人居便离开去。我见她走了,食指在右眼上一扫,一个眼珠落到我手心里,温润如古
玉。我闭着右眼,将那眼珠向天上轻轻一抛。 

只见我的视野随着眼珠忽地上升。天地宽阔,万象大千,尽收眼底。这内城的宫苑,草木,行人都
在我惊人的目力之下。 

原来如此,这废人居的位置当下就被我摸个通
透。 

我一手要接那坠下的眼珠,那眼珠光滑通透,我险些没有接住。幸得周围无人,否则定要被这异景
吓得昏死过去。 

\newpage

说起这抛眼珠观广袤的技法,是我曾经脑子一热的产物。实际用起来,条件很是苛刻。一则是你的目力要足够敏锐,否则就算眼珠在高空也未必能看清。二则是偶尔会借不到眼珠,虽然人匠的眼珠的确是
不会被摔坏了,但没准也会找不到的。 


最后,我站在异人居门前许久,安眼珠。 


10. 

异人居,有一条规矩:不许与其他异人相见。每日从自己的房内走出,必须带上宫里配的斗笠和面纱。以我的目力,可以阅他人面容,但还是不许交谈
,不许递物。 

呆了三日,内心的疑虑尤甚。虽然说是用来招待各路能人异士,但是既不许相见,又不吩咐所谓事宜。日夜闲散,与其说是招待,更像是牢狱。终日焦躁后,一天夜里,我从异人居溜出,按照所记的路线
去见侍女口中的“妖魔”。 

\newpage

如果侍女所说不假,那可能真的有魔。最大的
魔,是人。 

我披斗笠,戴面纱,倒夹黑伞,穿行在夜色里。冷月孤照,四下无音,寂如坟墓,只有脚步声回响
。靠近那废人居的时候,面前朦胧有一个暗影。 

是活物。身形如同羊马,四足着地,步履迟缓。但我的确没见过那样的羊马,只得靠近细瞧。我却
没料想,那是人。 

是一位老者,双臂处被替换成了扭曲的两腿,嘴的地方变的平滑无物。他的身躯只能这样匍匐在地
上,脖颈僵硬到无法抬头,也看不见这月景。 

他终于发觉有人靠近,奈何发不出声音,只能在鼻腔里惊慌的哼哼,在浑浊的双目里透露骇意,身
躯止不住的战栗。 

我心中一颤,把黑伞向地上一点,说:“老人

\newpage
家,不用害怕。我没有恶意。” 

老者显然已经很难相信人,还是止不住的退去。我蹲下身来,把头深深的沉下去说:“人匠不善,
是我程家之过。” 

我把右手轻按在老者后颈,又抚过老者鼻下。


我说:“您现在已经可以抬头,讲话了。” 

老者又惊又喜,眼中含着泪光。他激动地发抖,想抬头看天。只是我为他新开的口很粗劣,而且他
已经许久没有讲话了,只能呜呜地说着:“谢…” 


只讲了一句,那老者便佝偻着身躯咳起来。 

我拍了拍老者的后背,右手顺着脊骨摸下去,说:“您不用太急着讲话。虽然我给您开了口,但是你喉嗓已经受损大半,加之体质虚弱,已经不方便讲话了。我只问您些问题,‘是’便点头,‘不是’便
摇头。” 

\newpage

刚刚摸了这老者的身骨,不单单是四肢和口做了手脚,全身多处器脏,静脉,筋骨都已经被折腾的混乱不堪。这老人必定痛苦万分,生不如死吧。这样折磨人的手段,不单单是人匠,还要够残忍,够熟练

这样的程度,我已经无能为力了,随意施技,只能徒增其痛苦。即便是父亲在此,也未必能修好这
位老者。人匠虽能修人,却不能修尽一切人。 

我问:“把您变成这样的,是宫里的人么?”


他点头。 


“您见过他的面貌么?” 


他摇头。 


“您变成这样有五年么?” 


他点头,然后微声说“七。” 

\newpage

我看他神情痛苦,看来是回忆起当年梦魇,也
不忍心再问,只好说: 


“老人家出来,是为了看月么?” 


我把黑伞抬起,问:“您还有什么心愿,讲给
我吧。” 

老者终于含笑,却又热泪两行,他支吾着说出
二字:“赐....死。” 

我已经猜到他的愿景,便站在老者身旁,将那大伞张开。雕文在月光下显得分外诡丽,黑伞下老者霎时间化为一滩肉泥,片刻后又散作血水,终成为腾
腾的红雾,如朱砂飘起,附在伞的纹路里。 


生而无乐,唯死求欢。 

我转过头,急忙把伞合起,那偷看了许久的侍

\newpage
女忍不住惊叫。 


11. 

这是给我带路的侍女。我问她,姐姐,看了多
久了。 

“奴婢知错,奴婢有过,求大人饶我……”她跪下身要给我磕头。我连忙扶她起来说:“这位姐姐,我想你不就寝,来这里游荡,也多少是对这废人居
放心不下。我只想问你,刚刚那老者是何人?” 

“奴婢不知。”她说完开始抽泣,哭的接不过
气来。 

“我不害你。”我说着一手搭在她肩上,轻轻发力,只觉得她肩骨有异,右臂虚软。她急忙从我手
中挣脱,又要给我磕头。 

她眼神飘忽在我那伞上,大概是畏我这黑伞。我把伞被背过身去,说:“姐姐,你身子有没有哪里

\newpage
不适?” 


她摇摇头,愈加是害怕的发抖。 

我眉头微皱,只得说:“罢了。我不强求,也
不难为你。我只问你姓名,能讲么?” 

她点头,终于肯站起身,说:“小女子有一贱
名温良。” 

温良不说,我却能猜个三分。她藏匿,她心虚,她欲言又止,她定然对着宫中的诸多怪事有所了解。只是她的确怕,又有难言之隐。我断定她不到处声
张所见之事。所以我再没问她,各自分别。 

被温良弄出了些声响,我恐生事端,又回到住
处。 

自那后,我门前的侍卫,又多了六七人。但我依然相信,这事与温良无关:否则,我早就不是这般下场。朝中人若是听闻有一把杀人不留痕迹的黑伞,

\newpage
即便不招惹奸恶之徒,我也活不长久。 

我这次彻底找不到这监察的疏漏,像软禁一般
被关了半月有余。 

夜里我躺在床上,思绪是惊涛怒海,搅的我寝食难安。我坐起身来准备开窗,却看见窗外有个蹲着
的人影。 


透过窗间的缝隙,我大致猜到了这来客。 


我说,你怎么跑来这里的?一边放她进来。 

明彩满身血迹,肩上还有一道极深的刀伤。她从台上跳下,打了打身上的尘土说:“有个侍女,秀
气模样,告诉我你待在这里。” 

我叹息,又摇头说,我问的是门前的侍卫,你
怎么过来的。 

她漫不经心地答:“我说我是御用画师,要进来逛逛。他们非不听。我只好跳上屋顶,没想到屋顶
\newpage

上还有三个带刀的,让我放倒了。” 

她说的轻描淡写,但我终究是放心不下。我右手各轻点了她锁骨,右肩,右肘说:“砍伤,刺伤两处。骨损一处,筋损两处,右臂差点断掉。再严重些,我也修不好你。即便现在这样,要修你也要一个时
辰。” 

明彩站的不稳,不由靠在墙上,从腰间抽出几
排画卷说“我没事,我是来给你带几幅画的。” 

我只轻瞥了两眼,有轿子,椅子,花瓶。都是
些宫中普通的物件。 


但细瞧才觉得有异。 


“等下,明彩。这都是你画的?” 


“当然。”她的声音有点干瘪。 


\newpage

“你什么时候把死物画的这么好了?” 

她没回答,我这才发觉明彩面色惨白,嘴唇青
紫,倒在了墙角。 


12. 


天色渐晚,日光昏黄。 

她的伤比我想的还重,甚至痛及筋骨,脏器也有轻微的淤血。我花了足有三个时辰才修好她。最后
实在太过疲倦,我直接在床头睡去。 

我梦见明彩,见到的是一片雪白,白色的柳叶从我面前像素湍一样飞过。我听见明彩在我身旁清唱,唱的是我没听过的曲调。那唱腔如泣语,却又带着
几分洒脱。她的声音简单真挚,一字一句唱道: 


自有智,自有惑,辨得物与我。 


百种阳,百种阴,化作天地和。 

\newpage


不见善,不见恶,唯留因和果。 


千般圣,千般魔,任由他人说。 


这曲是什么?词又是什么呢? 

到最后,我满脑子回荡的都是最后那句“千般圣,千般魔,任由他人说。”沉醉之间,却已醒来。

我醒了时,明彩就坐在床边。其实我是很想问那天分别之后为什么要哭的,更想追问那梦中的曲调
。但我终究没有问出口。 


她先开口问,你身子,还撑得住么。 

我说,我当然撑得住,这都是末事。我给你讲
件大事,希望你不要怪我。 


她说,你说说看,我也先听听看。 

我指着柜子说:“侍卫被打伤,宫里严加戒备
\newpage
,我这里也被搜查。为了把你藏柜子里,我当时把你
拆了。” 


“拆了?” 

“就是拆成若干块,成一摞。然后…,堆起来。虽然不告诉你,你也未必知,但我还是觉得不该瞒
你,况且…” 


她瞠目结舌,半响说不出话来。 

明彩摸了自己浑身上下,然后指着我,我连忙
示意她小些声响。 


“你摸了我全身!” 

我没想到她竟然着眼在这点上,哭笑不得说:“这倒是其次,只是我单单觉得把人四分五裂,有违
天理。而且不是隔着衣物么…” 


\newpage

“我倒觉得蛮有趣的…。” 

“这可不是什么趣事啊,明彩。”我摇头讲“父亲曾说人匠里有先人为了避难,自己拆分血肉筋骨藏匿起来。虽然最后被他人恢复,却受不得被拆解后
那种状态,终日恍惚,郁郁而终。” 


她显然没能听进去我的说辞。 

我拿起那画卷问:“那接着说点大事。这些画
,到底是什么来由?” 

“的确是我画的,是我当上宫廷画师后,所画
的一些宫中物件。” 


“但你根本不会画死物啊。” 

她跳下床,然后笑着讲:“所以那些都是活物


我不禁悚然。 


\newpage

“你是说,这些曾经都是人?”我问。 


“是人,而且他们现在还活着。” 

“这不太可能,如果把物件镂空,以人匠的技法把人切分软化,将之注入。或者为人蜕皮,置入某个物件里,让血脉经络和外物长在一起。这两种难度都很大,而且就算能成,这人也活不了多少时日。”

“那你看这张。”明彩从袖中抽出一张褶皱的宣纸,上面潦草的画着一个人形。是我那夜里化进伞
的老者。 


我问:“你也见过这老者?” 

她说:“在夜里曾见过一面。时间太短,只画了个大概。我拿这纸问过一个侍女,她说这老人要去当‘椅子’,只是体质太差,没当成,成了所说的‘
废人’。” 

我半响无语。到底是怎样的人,要将人抽成模子,做成椅子,弄得分崩析离,生不如死?要这样违
\newpage
天理,逆人伦?这宫里我见过的人事有多少,未能的认识又有多少?我触到的恶可能只是河川,未见的恶
也许是汪洋大泽。 


心口有一团火在灼着,烫得难受。 

我凝思了片刻问:“你一直在说的侍女,是不
是叫温良?” 

明彩摇头说:“不知。我当了画师后,是那侍女来给我送纸墨。我便问她见过一个身背长筒,略显纤弱的男子没有。她便说你在这里云云。我又给她看了一眼那老者的像,她告诉我这是废掉的‘人椅子’

现在我心中有了个大概,明彩见过的侍女定是温良。但温良不肯把她所知向我全盘托出,却肯一五一十的讲给明彩。要说信任明彩,她与明彩也不过一面之缘,萍水相逢,又难说有什么情分。若是她在明彩身上另有他求,比如一直想图一幅画,没准倒还说
得通。因为明彩画起活物来,倒是精妙的可怕… 

\newpage

想到这里,我扫了眼床上散落的画卷,问起早有的困惑:“明彩,你只会画活物,有什么缘由么?

“我要是问起你的伞为何如此神妙,你有缘由
么?” 

这是在讲她的笔不同寻常么?我还没理顺个中道理,却见到她有点失意地看向我,眼眸里藏了些落
寞,只是脸上强挂着笑言,还像是与我打趣。 

我这才发觉。明彩赌上性命来见我,又守了我
一日。但我却连半句关切也没给过她。 


13. 


今晚,要再去废人居一次。 

起码要弄个彻底,弄个明白,直到让我心安。

我提出这个决案的时候,明彩对我佩服非常,

\newpage
说我看起来弱不禁风,没想到依然心怀天下。 


我说,我的心哪里怀的住天下呢。 

我不自欺欺人,我明白。这天下是应家的天下。我只是一块瓦砾,一片泥壤,一颗棋子。我尽力翻搅这池底,充其量也只是死水微澜。天下里有多少恶事,我触之不及。但这宫中种种,放任不管,终有一
天要惹火烧身,把我和明彩焚为灰烬。 


丑时初,便起身。 

“丑时是侍卫更替, 屋顶上只有一人。见面之后,只要让我的血沾到侍卫肌肤,我能让他气血逆行数息,他经脉胀痛而不能动,你我就逃出。”我这
样讲。 

明彩是一个挺容易劝和被说服的人,起码我目前还这样想。我给她了讲了些小时候的趣闻,要不是
我捂住她嘴,她能笑得把大殿里的侍卫都召来。 

我心又放下来,回想起自己好久没有这样自在
\newpage
惬意的聊天。我都忘了,自己在忙什么,求什么。生而为人,成而为匠,又能代表什么。万千善恶,又有
多少瓜葛。我都不愿想。 

我想的是,能这样闲半个时辰,就闲半个时辰
。哪怕下一息,要见血光,动刀兵。 

她也给我讲了些她初入江湖的所见,说她骑着马跨了多少山岭,画了多少人家。说她被江洋大盗劫了银两,还不忘给人家画像。说她曾经饿过三日三夜
,看见客栈的美食差点把不住碗筷。 

她说,家传人匠,有祖传口诀什么的说来听听

“哪里有,只有天天念叨的‘心善,人善四’字家规。还有什么玄之又玄的古训,让我到什么境界
,见什么故人。”我答道。 

“古训,这种没灵气的东西。我编都能编个十几句呢,不过是什么道法自然,天地轮回,人心善恶

\newpage
的老话。” 

的确,明彩说的也确有道理。我没反驳,只顺
着她说 

“明女侠,你说的也在理。可惜你不是古人,所以你说的只能是‘今训’,又有多少闲人肯听?”我话音未落,已经听见屋顶上细碎的脚步声,那是侍
卫交接。 


丑时到,暗云蔽月。这是再也闲不得了。 

我以眼神示意明彩,她心领神会。我伸出右手,垂下几滴暗红的血让明彩用牛皮接着。明彩跃窗而出,身形矫健,只听见屋顶传来三声轻巧的踏步,又
归于沉寂。 

“上来吧!”她探下半个身子,向我兴奋的摆
手。 

我武艺不通,行动迟钝。在屋顶上翻上翻下也是温吞水,全然没有明彩那样得心应手。费了些功夫
\newpage

才从异人居离开。 

我说:“刚刚让你拿侍卫的刀了。如果这次去废人居有什么不测,你第一件事就是把我这黑伞砍断
,然后再把我右手戳穿。” 

明彩暗暗瞥了一眼我背着的长筒说:“程善啊,程善。你这黑伞的确是个宝贝,可天下的宝贝又不
是只有你这黑伞一件。” 

我笑问:“听明女侠这么说,应该是见过更加
珍奇之物了?不妨拿出来看看?” 

她却跑开来,说:“快走吧,一会就要天明了
。哪天穿给你看。” 

穿?是一件衣物,还是靴子?我本以为她那画笔有精妙之处,才致她善画活物。难道还另有原因?我反复回想明彩穿过的衣物,既没有太过华美的样貌,也没有什么不凡的功效。所以应该是我还没见过的

\newpage
衣物。 

我再没过问,与她一齐跑到废人居门前。我拉
着明彩侧身到门一旁。 


我在她耳旁道,里面有人要出来,很多人。 


晚秋风起。 

然后我们两人听见了里面凌乱的言语声,嘈杂
纷乱,弄不清次序。 


“活着的还有九十七人,都带到后殿。” 

“你怎么跟来了?这不是你这女人家该来的地
方,快回寝宫,老实睡觉!” 

“你们几个别搬那骨肉了,全都堆在那边便是


言语声只持续了片刻,又是沙沙的拖行响。 

然后我听见簌簌的颤响,像是万木成枝从地上
\newpage

攀过。 

我们两人一动不动,静着藏了些许时候。直到
死寂。 

大门依旧敞开,只是夜色太深,周遭的景致都
像蒙在墨里。 

是一个空荡荡的大院,房宇都被拆了去。只有

“这天色太暗了。里面的景物我能看见,你应
该看不太真切。”我拦住要上前去的明彩说。 

“你拦我做什么?我护着你还差不多。你看看
,这里面有东西么?” 


我说,只能看见石砖。 

“这不对,石砖上都是脚印,还有拖行物件的痕迹。这里的人和物都被移走了,就是刚刚的事情。 ”我眉头紧锁,在目力所及之处尽力去看,看每一
\newpage

个错过的细节。 

明彩的每种情感,都盛满到装不下,溢出来。所以我一眼就看破,她的不安。她快步走上前去说:
“这砖下面有东西,你要来看下。” 

我右手按在地上,一路沿着石砖的缝隙擦过。
到了明彩身旁,近乎惊的不能言语。 

“这地砖下有血肉,血肉下又有经脉。这地下有大东西,东西上有还有筋骨百千……”我一边摸着
,一边在心里估量着地下的东西。 

不可能,没可能的。这地下是血肉与土长在一起,人的脏器混作一团像是根茎深深埋下,筋骨如同
枝叶潜在土中。 


明彩走到大院中央,愣在那土堆之前。 

“程善!这土堆…”她还没说完,又听见簌簌的颤响。有什么东西在地下躁动不安,要破土而出。
\newpage


我终于警醒,然而步伐已经跟不上炙痛的心绪

“是手!地下有手臂!”话音未落,那些石砖一一被撬动,发出沉闷的碰响。无数只手臂相互接连,盘错着从地下窜出。它们肆意生长,从每一个石砖下面死死地抓住我和明彩。我和她转瞬间被拉出十步
之遥,那些手探上我的双腿,腰腹和肩膀。 

一股蛮力在狠狠地把我向后拉,接下来,就是我被更多的手抓住,像是被锢上无数的枷,然后被扯
到粉身碎骨。 

我右手成掌,依次斩过身上的手臂,被我斩过
的就像蜡一样断掉又缩回去。 

“明彩!不要用蛮力挣,这手里面有人匠的血,那些手都是化骨,脱血的技式!”我跑过去想要救明彩,却发现她右臂已经被几十只手死死锁住,她借
着腰腹的力,还在苦苦支撑。 

\newpage


如万蛇缠身。 

若是再迟一息,怕明彩要被化作一个空皮囊。所以我一掌从上至下斩了下去,掌锋切过那些残臂,她身后的长发,她的右臂,最后从她右脚的脚踝处离
开,她就这样被我斩成几段。 


像刀斩乱麻。 


14. 

明彩终于脱出,我把她背着,在我肩上轻的感觉不到分量。我狂奔着,探过她的身体,心中一阵凉

到底是用多少人的血肉铸成的那万千邪手?到底用了多少人匠的血才能达成那样的技式?我想不出

这里面,到底葬了多少性命,埋了多少冤骨,
腐了多少血肉,去了多少生灵。我不敢想。 


\newpage

我能想的,就是明彩到底被伤的多重。 

她估计已经损了三成的骨,四成的血。我予了
她一些我的血,只听见她在我背上说: 


“程善,你听过《云鬼词》吗。” 


我愣住了,不知道答她什么。 


只能摇摇头说“没有啊。” 


她的声音快要听不见,她说 


“总有一天,我要唱给你,让你说好听。” 

她骨已经酥了,精血也不稳。被那邪手抓过的地方,更是软的像泥偶。我感觉她就要像蜡一样融掉

我说,你听着啊,我会修好你的。我是程家唯
一传人,天下第一人匠。我什么人都修的好的。 

我说,我是持黑伞的程善。他们听了都怕我。
\newpage

唯独你不怕我,所以你也没什么可怕的。 


她只是笑,却连半句话也没力气答。 

我跑到再也提不起脚步,接不上呼吸。到了哪
个角落里,把明彩在地上放安稳。 

这也许是大殿后,也许是寝宫后。我完全顾不得这是哪里,明彩在我怀里瑟瑟发抖,蜷缩的像个婴
孩。 

把那信读了吧,我这样想。我留着这封信这么久,这么长时间都好奇里面撰写了什么。但里面无论是怎样的文字,都抵不过生死之隔。“至境界“,至得什么境界?明彩可能就活不过今晚,我没准哪日也
难逃一死。到时候那信还有谁人来读,谁人来阅? 


到那时,只是一张废纸。 

我把那长筒翻弄,果真找出一信封。开封之后,掉出一根发丝,一张信笺。信笺微微泛黄,细腻如
\newpage

羊脂,上面是密密麻麻的暗红字迹。 

手抖个不停,我怕连那字也辨不清认不得,心里突突的要跳出来。而又感觉明彩的呼吸渐渐弱下去
,我一手按在她两个胛骨间。 


果然,精血两亏,她的脉已经衰下去了。 

我突然感到胸口酸楚胀痛,有股戾气不得不发
。为人匠,生而修人,怎肯让人在自己面前死? 

我几乎要将牙根咬出血来,心意已决:五指按在她后背,贴上心房所对的位置。一息间,我感觉到
她全身的经脉和我联接。 

她的血不能再流,就让我的替她流。只要我程
善还有一息尚存,就没有明彩死去的道理。 

我一边用断臂拨弄着信笺,一边用我的心脉律
动明彩的血流。就这样直到东方微亮。 

\newpage

天明,上朝的鼓声和晨曦交杂着盈满内城。百
官来殿,国君起朝。 

周遭喧杂了起来,是侍女,太监和群臣的脚步声交叠在一起,恍若皇城这头凶兽揉弄惺忪的睡眼,打着哈欠。脚步越来越近,他们应该很快就能看见我
们。 

来的可能是当今圣上应如意,可能是司礼监的秉笔太监,也可能只是小少监和侍女,或者那个叫温
良的女子。但对我来说,都没几多差别了。 


那时的我像枯木一样呆坐着,满脸泪痕。 


15. 

我读完了那封信之后,倒释然了几分。我的那些恨,怒和恶意,全都被埋的极深。我压在心底里都没去想,只是想着将来的筹划。我把那些带刺的,险毒的念头都包裹的精致圆滑,用笑脸把自己裹起来。

\newpage

然而筹划到哪里,将来是怎样,也不尽明朗。要保全我,要救明彩,应该怎样走,都悬而未决。到我抉择的时候,只权当是赌,献上我有的所有筹码。

我抬眼,看见两个普通的侍女满脸惊疑的朝我
走来。我没见过她们,或者见过,也全然忘却了。 

因为我支撑了两个人的心脉足足一夜,现在已经是强弩之末。我连沉稳的站住都很勉强,更不要说
走动了。我靠着墙,半天才含糊出一句话: 


“两位姐姐,能帮忙指个路么?” 

两人打量了我,暗暗一笑,说道:“你这人满头银丝还叫我们姐姐,倒不如我们叫你一声‘叔伯’

我努力地含着笑说:“也好。那些倒是小事。
只是小的想知道怎么去见王总管。” 

其中一个见我身形不稳,要过来扶我。她说:“看你打扮和腰牌,应该是异人居来的吧。现在你见
\newpage
不到王总管的,他应该在陪皇上散步。异人按规矩是不得进寝宫的,你要是被旁人看见了,要吃苦头的。

我摇头说:“劳姐姐费心了。您只给我引条路
便是,至于走不走,我再权量。” 

另一位侍女拉了拉她的衣襟。她迟疑了片刻,然后指着一个方向说:“我与你面生。但看你的神情确有急事,便告诉你。向那边走到路口,再向右,便
能看见牌子…” 

她眼神停在我身后的明彩上,说道:“这位姑
娘,我见过的。” 

我抱起明彩说:“她有腰牌,是宫里的画师。
你们认得一位叫温良的姐姐么?” 

两人点头,那在前面的侍女说“认得。她虽然做事毛糙,却见识广博,能言会道,在我们之间很是
有名。” 

\newpage

我说:“那劳烦两位姐姐代我,将这位姑娘带去温良身旁。她刚得了大病,气血衰微,需要人来照
顾。温姐姐应该会照看她的。” 

那侍女看了看面色青白的明彩,半点没有犹豫就接过了,一到手里,她眉头微皱说:“这姑娘怎么
这般轻?连我一人都抱得动,像一团柳絮似的。” 

我说:“这姑娘天生身骨纤弱,又有恶疾,体
轻也是理所当然。” 

两人相识,又耳语一阵。我没去听,大概是些关于我来路不明,行踪可疑的话。但两人终归还是放
下心来,讲到: 

“我看你气色很差,步履蹒跚。应该也有些顽疾未愈。要是行走不便,大可不必勉强,随我两人先
去休息。” 

我转身离开,摆摆手说“谢两位好意了。我走

\newpage
一条路便是一条,没太多回头的道理。” 

两人已经走远,而我还在想刚刚那侍女的不寻常:她从我手中接过明彩的时候。我右手碰触她一根中指。她中指的三个指骨,应该都是中空的。如果有人攥住她的手猛里一捏,她的手应该会化成骨渣和肉
泥。 

这侍女应该还不知晓,但我却也不想透露。因为去骨易,入骨难。而且以我现在的身体状态,更是修不好她。如果这样贸然告之与她,恐怕只能让她惊
惧不安,惶惶不可终日。 

其实,从昨晚开始。我离家后的年少热血,有
一半已经凉了。 

我一边用右手尽力修着自己,一边想着要怎么见到王总管,见了又能讲些什么。我还想让那些欠了
债,欠了万千血债的人,能一并偿了。 


所以我还得活着。 

\newpage

不仅要活,为了信里说的那些事,还要努力活
着。 

我想,既然能见到王总管,怎么不见掌印太监,怎么不见首辅?既然我只有这些筹码,又没太多可
以输。想当一个赌徒,为何不添点彩头? 


最后,那就直接见当今皇上应如意好了。 

应如意,我只有小时候在画像上见过。他给我唯一的印象,就是他作的那句诗“江山成绣锦,天下
应如意。”据说有几年,连春联都是这两句。 

那时候,他离我太远,至于他到底嵌在天幕,还是深埋黄土,与我没有半点瓜葛。应如意残暴无道还是英明神武,对我来说没有任何意义。我不关心他的天下,他也定然不会关心是否有我这一介庶民。如果我说我有一天要见他,那显得不和道理,不符章法
,不切实际。 

我从没想过有一天,会持着黑伞,站在他面前
\newpage


但我依旧会去,因为我还有一半的血,余温尚
存。 


16. 

阻止我去见应如意的情况,有太多了。被侍卫发现,被其他不那么温和的侍女发现,甚至应如意已
经离去。 

我把伞开到两成,想到了所有最恶劣的情况。
但我都没有遇见。 

我遇见的只是一个小太监,挡在后花园的门口

我说,你去跟里面,随便哪个人说。就说程家
有人来了,持着一把黑伞,背着一个长筒。 

小太监很听话,他跑着进了院子里面。我看他
答应的这么爽快恳切,就像是他等了我许久一样。 

\newpage

过了些许时间,那小太监一摆手说“大人请进
吧。皇上就在里面等您。” 

我一时间没有反应过来,脑子还有点发蒙。实在是有点太顺利了,顺利的不真实,像是浮空幻影。

我走了十几步,看见一树桂花后面坐着一位衣
冠华美的男子。我便问:“你是应如意?” 

身后有人轻轻拍我说:“他只是个壳,我是应
如意。” 

我回头,看见一位面相很和善的男人,全然不
像画卷上那般冷峻。 

他坐下来,饶有兴趣的打量我,然后示意我就坐。他说:“你见到天子不下跪,不行礼,不谦逊,
你真的不懂礼法么?” 

我说,你等我来找你,就是为了听一句草民叩

\newpage
见皇上?说这话时,我的眼神轻轻扫过他的左手。 

应如意听后大笑,然后拍拍我肩膀,连说了几声好。他已是不惑之年,却依旧像个少年一样笑的没
有节制。 

应如意说:“你那天进城门,侍卫就注意到了你的黑筒。我想你在这宫中呆久了,总有一天要来找
我。” 


我说,我该夸一句皇上料事如神么? 

他摇摇头说:“这些话,我都懒得听了。我听闻你天资聪颖,十六岁就已可以单手让侍卫失目,已
是难得。我想让你在我身边做事。” 

我抬起头,凝视了片刻晚秋的桂花,然后说:“皇上贵为天子。让我一介草民做事,还要费这么大
周章?” 

他说:“你年轻气盛,有些事情你不愿意做,

\newpage
也不会懂。该让你经历一些。” 

我想问宫中的诸多恶事,他是否知晓。我还想问,那年,那天,他的所作所为。我什么都想知道,什么问题都想问。但我知道今天不合时宜。应如意对我近乎了如指掌,而我却对他一无所知。况且,他还
有整个天下。我只有一条命,一把伞罢了。 


我说:“草民知道了。我会尽心做事。” 

应如意说,你有什么要求,尽管提。宫里有的
都不会亏待你的。 

我慢慢的抬起眼帘,眼神里什么情感也没有,淡漠的就像逝者一样,我缓缓地说:“给我张床,让
我好好睡一觉。多谢陛下。” 


17. 

应如意说明日酉时末去他书房。我欣然应允。在离开后花园之后,我没有去应如意给我安排的新的住处,真的去酣睡一场。而是背着长筒去找了温良。
\newpage


自我见过应如意之后,我像是晋成了朝中权臣。三宫侍女,以礼相待。六院守卫,无不避让。我一言语说我想见一位叫温良的侍女。全都喜笑颜开,迎上来要介绍引路。我被拥的心烦意乱,费了些功夫才
见到温良。 

温良凝视着我,在茶桌旁特意留了一个空位。

大概是我眼花,她比往日显得年轻,也没当初见我那么胆怯。她对我行礼,然后说:“大人,见过
皇上了?” 

我点点头说道:“见过。皇上温文尔雅,不愧为国之贤君。我想问问,姐姐见过一位叫明彩的画师
没有。” 

她又问:“那位画师,是大人托我照顾的,我定当多加留心。只是这宫中如若泥沼,谁也不得抽身。我也未必保得住那姑娘,只可怜她生了副好皮囊。

\newpage


我的心猛地一缩,隐隐阵痛。 

我说:“连姐姐也救不得明彩么?前辈,那日我按过您肩膀。您肩骨是刚刚修过,手臂又是新的皮肉,加之经脉运行极缓,理应是极其老道的人匠才是
。人匠的技法,恐怕我比您还差得远呢。” 

她说:“哪里。你天资聪颖,自幼刻苦。要说这技法之精,我也不及你。我若是有所见长,也只是技法之广罢了。这姑娘,救是可以救。但人于人匠眼中,就如同木于木匠眼中。都是物件,是器具。什么生灵,活物,都是无谓的说辞。宫中总有人,要贪这
姑娘的皮肉。” 

我愣住,半响无语。感觉胸口被什么压着,喘
不过气来。 


一阵寒意。 

我攥着手里的茶杯,右手不觉的发抖,我转过头问:“前辈,宫中之恶事,你无所不知。你真的不
\newpage

插手么?” 


她先说了四个字。 


“年轻气盛。” 

她又说:“程善,你见过的恶是怎样?我见过人匠把人的头沉下肩膀,让他人的眼目被自己的肠胃消化。我见过把人的喉舌嵌进镯子,叫那人求死不能。我又见过人匠把人蜕皮去骨,放到秤上像猪牛一般称量。我活的太久,做男人,女人,老人,小孩,无
一不包。天下大恶,尽收眼底,你能一一去了?” 


我说,好,好。 

我说:“前辈成圣成魔,我不言语。前辈想当侍女便当侍女,想当权相便当权相,倒也乐得自在。
我只问你几个问题,望前辈如实回答。” 


她应允,脸上挂着几分失意。 

\newpage


我问:“请问,什么是‘铸人’?” 

温良神色古井不波,她伸出自己的右臂说:“
这条右臂,不是我自己的,你看的出来吧。” 


我点头。 

她说:“用人匠身体的一部分,混合他人之血肉,再加以特殊的技法。可以铸造一人。铸出来的人,有如真正的人。若是用人匠的部分多,就与人匠像些,甚至于心意相通。若是用人匠的部分少,就不太相仿,铸出来的人也活不长久。被铸的人若是寿命尽
时,就成一团气雾,散了。” 

我恍然间醒悟,脸上露出的不知是不是笑。我想笑又笑不出,只好把面容摆的狰狞,像是画像里的
罗刹。 

我说,前辈,今早来抱走明彩的侍女,是你铸
的人吧。 

\newpage

她说:“是。那日我救了一位废人居的女人。但是被折腾的不成人样,身体扭曲的像是一个箩筐。我一气之下把那身体打的稀烂,然后用我的一根头发
铸成了你见到的那个侍女。” 


我感觉自己快结冰了。 

我突然觉得自己知道温良为什么要救废人居的那个女人,那女人到底是谁。但我又痛恨自己知道,
像胸口被毒刃刺穿。 


哽咽。 

我快说不出话来,只能含糊讲道:“前辈,那
封信是你写的吧。” 


她点头。 

我说:“前辈。您救得女人是不是我母亲?”


\newpage

我说:“我那日用黑伞度化的老者,是不是我
父亲?” 


她又点头。 


我起身向温良跪谢。 


我说,前辈,多谢您养育之恩。 


泪流。 

温良摸着我的头发说,程善,别哭。你一定会
是天下第一人匠,一定会好好活着。 


然后,她给我讲了一个很长的故事。 


18. 

这故事我已经在信里看过一遍了,只是那时我
还不知道,是温良讲给我听的。 

\newpage

我母亲曾是宫中的一个侍女,父亲是异人居的
一位异人。 


他是人匠,技艺超群。 

他有位多年的至交,叫温良。温良潜心铸人之法,准备用自己毕生心血和右臂,铸成一个人。但是温良没有机会,他找不到合适的底子,他要把这门技
艺用在最合适的人身上。 


他等了蛮久,然后等到了机会。 

应家的寝宫要降生新皇子,先皇应天安等待着
自己的儿子和未来的国君的诞生。 


噩耗打击了应天安。皇子应如意夭折。 


温良说,我能救活皇子。 


先皇说,好,若成,赐你荣华。 

\newpage

温良斩下了自己的右臂,铸成了新的应如意。

新皇子生来便有二十多岁模样。先皇吓得惶惶不安,惊惧万分,大叫“怪胎!”,然后一病不起。


又过了两年,应如意登基。 


应如意说,天下应如意,我要求万人长生。 

人于人匠,如木于木匠。他有人匠一臂,可以施人匠之法。他要让人融于万物,求得万人不朽。要人成椅子,成桌子,成瓷瓶,成怪,成魔,生不如死

温良没有得到荣华,他活在悔恨和厌倦里。没了铸人的痴求,他什么也不剩。他没曾想,铸人失败,就会铸成魔。他找了位被应如意玩弄到求死的侍女
,杀了她。取了侍女的皮囊,他变成她。 


温良就想这样活着。 


\newpage

父母当时刚刚生下我。 

母亲被折磨不堪,父亲为了救母亲,像我一样
血脉相连,一夜白头,纹上眉梢。 


时间在父亲身上汹涌流逝。 

父亲一直反对温良铸人,但这时,他说:“我俩尚不能自保。但善儿不能没有父母。你取我双手,去铸成一男子。再用你杀的那侍女和你发丝一根,去
铸一位女子。去罢。” 


这二人,便是我父母。 

温良取了我父亲双手,在废人居找了位男子,铸成我记忆中的“父亲。”然后又取了自己几根骨和
发丝,铸成了我记忆中的“母亲。” 

应如意只有右手有人匠之能,他要我父亲献上左手,才是完整人匠。但我父亲已经没有左手可献,
他只剩两只残臂,手只是一阵幻痛。 

\newpage

应如意说,好,你没有手,那还当人干什么,不如当椅子。而且你没有,总有一天你有子嗣,子嗣
也会有手。 


温良说,要程善的左手,应如意才会罢休。 


于是我单手,成为人匠。 

温良算过,男子用双手铸成,至多活十一二载。女子用骨和发丝铸成,也不过二十载。所以必须吩
咐,让我十六岁前离家。 


然后我来到皇城。 


然后我来到宫中。 

然后我用黑伞杀了那位已经不成人形的老者,
那是我父亲,他被做成椅子,有七年。 

然后温良救出了废人居里,我那要被做成箩筐的母亲,把她打成血肉,铸成一位侍女。这位侍女,
\newpage
只靠这根发丝,只能铸成中空骨,空心肉。最多能活
三月。 

最后我来到已经是妙龄侍女的温良面前,听完
了这个故事。 


我说,谢谢你。 

我说,谢谢这天下,如此善待我。万谢应如意


我说,皇上万福金安。皇上天地同寿。 

我明白一切的始作俑者都是温良,但我恨不起来她。从某种意义上说,她就是我的父母。她养我育我,除了没有告诉我古书第十二章《铸人》外,传给
我一切。甚至不垂涎程家的黑伞。 


她成全我。 

我说:“温良。我懂了。你能不能再帮我一次

\newpage
。我愿意什么都给你。” 

温良说:“我缺一只作为人匠的右手,你也能给么?你给我之后,就是普通人了,连黑伞也没得资
格打开。” 

她别过头说:“老道的人匠天下不知几许,但
是持黑伞的程善只有一个。” 

我说:“能。在你帮我之后,我就给你。我没
有手,也无所谓。当普通人,也没所谓。” 

温良不悲不喜,她合上了紫砂壶的盖子。把那
一盏茶倒在地上。 


她说,好,我帮你。 

我这次笑了,难得笑的很开心。我说,那好,
让我看一眼明彩吧。明日酉时末,我们就动身。 

我穿过回廊,走了几间房,见到了面色苍白的

\newpage

明彩见我很兴奋,她跳起身来,给我舞动了拳脚,尽力打的生龙活虎。我一只手攥住了她要挥动的
手臂。 

我卸力说:“你看,要是以前的你,我哪里攥
得住。” 


她撇撇嘴说,切,那是我让你。 

我说:“好了,不用逞强了。你身体没大碍了
?” 

她说:“全好了。温良姐姐是位大善人,也比
你厉害多了。” 

我笑着点头说:“我也这么想。温良的确是位
善人,也比我厉害多了。” 

我看了看周围散落的画纸,都没能成画,只是
在纸上潦草几笔。倒像是孩子赌气的涂鸦。 

\newpage


我说,怎么不画了? 


她说,没得画,这些东西太丑了,不想画。 

我说:“行,随你心意。你要画便画,还要多
加休息,照看自己。” 

她佯装嗔怒道:什么时候明女侠的事情也要你
叮嘱了? 


我说,是小的的错,臣有错,臣悔过。 

她看我这幅滑稽的作态,要笑出声来,但是还
没笑,就开始咳,咳得站不稳,像柳叶随风。 

我连忙搀着她到床上躺着。她说:“你不用管
我。你怎么像老了几十岁一样?是我眼花了么?” 


我说,哪里,我本来相貌就老成。 

她说:“不对,我能看出来。你的身体比你的
\newpage

心老的快。发生什么事了么?” 

我能感觉到她冷汗在流,她像这样撑着大声说话,应该胸和肺都像刀挂一样痛。她是很勉强的吧。我的心一阵疼,连忙说:“明天再来看你吧,我去办
些事情。” 

我看了看地上的画,总觉得该说些什么。脑子里像是一片浆糊没了头绪,嘴上却笨拙的,把那锐的
话都说钝了。 


我说:“明彩,我…。挺喜欢你的画的。” 

她硬挤着全部的气力说:“明天等着我的画吧
!” 

出来时,温良在门口站着等我,应该是一直在
听我俩讲话。她只说了句。 

“睡吧。好好睡一觉。明天起来,就什么都有

\newpage
了。” 

那夜我进入梦乡,梦见一片雪白之中,明彩穿着一袭白衣来见我。嘴里唱着清澈的曲调,唱着“千
般魔,千般佛,任由他人说。” 


我听着那曲子,慢慢被大雪淹没。 


19. 


这日酉时,我准时到应如意的书房。 

书房里摆满了大大小小的瓷器,摆件,甚至脸
谱。 

应如意很高兴,他笑的开怀,连说:“来,程善老弟,我给你看我收藏的这些器物。个个都是宝贝

我说,哦?皇上尽拥整个天下,竟然还有皇上
所稀罕的宝贝,那我真得见上一见。 

他说:“哪里哪里。给我做事,将来不会亏待
\newpage

你。这些宝贝,你想要哪个,我都分给你。” 

我轻笑说:“皇上说笑了。这都是皇上千方百
计拿来的典藏,我哪敢奢求呢?” 

应如意拍拍我肩膀说:“不难不难。难得是这
颗心。” 

他问:“程善。你看,做人匠,单单是修人,
岂不是大材小用?” 


我问,皇上有何高见? 

他指着那堆瓷器说:“高见倒是谈不上。你看,那里面有窈窕的少女,有佝偻的老者,有车夫有店小二甚至有山贼,芸芸众生相都让我打作肉泥堆砌在里面,岂不是万世长存,这才是人之大匠,才是人匠
之本啊。” 

应如意啊,你只是人匠铸成的一个木偶,一个

\newpage
玩具。也不过活二十几年的光载,还能妄贪万世。 

我强挤出欣然的表情说:“皇上所言极是。看
来我之前所求人匠之道,反倒是窄了,小了。” 

他又指着那边摆着的脸谱说:“别这样妄自菲薄。你再看,那墙上挂的,都是人的面皮。这脸谱,
岂不是活灵活现?” 

我点点头说:“果然生动非常,真是绝世无双


我定睛一看,一眼扫到了墙上明彩的面庞。 


我指着明彩的脸说:“皇上,这面皮……” 

应如意神色一滞,他说:“老弟,你想要这个?这是我今早刚刚拿来的收藏,还新鲜。不过你若是
喜欢,我绝无吝啬的道理。” 

明彩就这样被做成了脸谱。她要被活剥,要被去骨,要刮下脸上的面皮。然后挂在墙上。我再也没

\newpage
机会看到明彩的画作了。 


我不敢想,一动这念头,就觉得残忍。 


我没有伤痛的力气。 


我父母,我明彩,我左手。我与谁问。 

我想起那日离家,前往皇城。我热着全身的血,背着长筒,觉得自己是天下第一人匠,觉得自己能
独步天下,举世无双。 

人匠可以修人,不能修心。可以修千万人,不
能修天下人。 


浮生幻影。 


热血尽凉,只剩这一腔还发烫。 


我抽出了长筒里的伞,举在我面前。 


\newpage

我问:“应如意,你知道善恶么?” 

应如意看见我那黑伞,面色淡然。他说:“程善,我之前就说你不懂礼法。你看看,天子面前,就要贸然动刀兵。你也年纪不小,怎么还信善恶那一套

我突然笑出声来,我把伞张开,伞上的黑色雕文绽放在书房里,周遭所有器物为之一颤。那些器具桌椅里面的人,尽皆被我毁做肉泥。万千血雾从周遭腾起,一一附到我那伞上。屋内像是爆开一团血莲,
一股血腥味浓郁后又消散不见。 


一伞开,杀生无数。 

应如意叹息道:“可怜我这些藏品,都被你这伞毁了。你杀这书房里这么多人,难道就能称之为善
了么?” 

我说:“谁说我是善?谁说我是恶?庸人才信善恶。善人有善报?恶人有恶报?都是虚妄之言。我只讲因果。你杀天下多少人,是你的能耐。但你杀我父母,杀我明彩,取我左手。是你种下的因,今天,
\newpage

才是果。” 

我听见外面侍卫腾腾的脚步,像海浪一般涌来

应如意说,我知道你要来,不会一点防备没有
的。你是程善,不是什么凡夫俗子。 


我说,皇上说笑了,我就是凡夫俗子。 

应如意说:“可惜,可惜,可惜啊。时至今日
,还要我亲自来,我来教你为臣的礼节。” 

我说:“不了,你若想听礼法,我讲给你。”

我放声大呵,声如洪钟大吕,回荡于三宫六院,久久未散:“我是程家唯一子嗣,天下第一人匠,程善!今我持黑伞求应如意一见,与你讨我父母债,
我明彩债,及千千万万血债,愿你一并偿!” 

我知道应如意有人匠双手,黑伞不能伤他分毫。但我开着伞只是为了戒备周遭赶来的侍卫,不让他
\newpage

们近身。 


这撑不了多久,外面是万箭齐发的破空声。 

我很快的被箭雨打的血肉模糊,倒在血泊里,
眼睛也被血染。 

朦胧中,应如意说:“程善。黑伞不能救你,
只有我才能救你。” 


他靠过身来,想要拿那把黑伞。 

我摇摇头说:“应如意。你也不能救我,因为
你救不了你自己。” 

我言罢,从右手袖口中又伸出一只手,像蛇一
样盘过应如意的脖颈,然后狠狠捏住他的面庞。 

我看到应如意惊惧在眼神里像洪水一样流过,下一刻就是他的整个头颅像是泄了气的皮囊一样瘫软

\newpage
下去。 

这是温良借给我的手。这是我特意为了应如意
准备的极致盛宴。 


我笑着说,这下,你永生啦。 

那手像软泥一样疯狂的倾泻进应如意空空如也的头颅里,我的袖口有如一团乱根般窜出皮肉向应如意身体涌去。他的头又饱满起来,恢复了原来的面目

我说:“让你把头嵌进这么小的地方,委屈你
了前辈。这右手,你随意取用。” 

这一刻跟我说话的,是拿了应如意皮囊的温良


温良摆了摆自己的右手说 

“不用了,我拿回了自己的右臂,要你的右手
有何用?” 


\newpage

我说,那好,前辈,愿你善待这天下。 

温良笑而不答。过了半响,他说,也愿天下善
待我。 

他开门走出,大声道:“反贼程善已被就地正
法!。” 


20. 

等我再次修好自己的时候,已经是满头银发。

我从皇宫离开时,温良说可以让我尽享荣华。
我说不了,已经累了。 


我什么都不想要了。 

没有亲人,没有自己。只有明彩的画,我留着

还有一块墨色的玉玦,像是太极的一边。这是
家传的古玉。 

\newpage


除此之外,皆无。 

我背着明彩的画卷走着,走在当年经过的山路上。又遇见同一伙山贼。也还是那个头目。他从山上
走下来说 


“程家少爷…,你的头发怎么…?” 


我笑着说,没事,权当被雪染了。 

他说:“少爷,当年我们不是要打劫你的。只是上面有令,他们说,当山贼,我不管。但是要是有
背长筒的少年,一定要留心。” 


我点点头说,没事,我不在意的。 


他说话的时候,我背后的画卷狂颤。 


我说,我先走了,有缘再见。 

那头目拜谢我说:“程大人宽宏大量,小的心
\newpage

领了。” 


我笑笑,没说话。 


我走了好远,一直走到无人的林间。 


扯开颤动的画卷,上面空空如也。 

耳边是梦中的歌声,是明彩在我耳边清唱。我
回头,林间恍若有霜雪飞舞。 

明彩披着白色大氅,持着一根画笔站在我身后

我不惊讶,我总是梦见她,我总觉得终有一日
我们会相见。 

她一直唱到“千般圣,千般魔,任由他人说。


她轻笑问我:好听么。 


\newpage

我点头说,好听。 


我答应她一定会说好听。 


她说,喏,我穿给你看了。 


我说,你真的是画师么? 


她脚步轻灵,恍若随风曼舞。 

她说:“我都说了,你有传家宝,我也有啊。


我说:“也是。明女侠不曾欺我。” 


她说:“当初你说的古训,都照做了?” 

我无奈苦笑,答道:“伞已经开了。信被温良掉了包,也不知道里面到底写的什么。只剩这一块玉
,还没来得及用。” 

她像是一团光,在我面前缥缈如雾,看不真切。她拿出一块白色的玉玦,正能与我那块严丝合缝。
\newpage


她说:“我的古训是这样‘遇危难,披氅。至境界,下笔。见故人,持玦。’。我平日只会画活物,是因为我的笔只能画魂。你老了,但好在你的魂还
年轻。” 

我说:“别管我了。你现在只是一团魂吧,将
来怎么办?” 

她说:“陪着你喽,家传的白氅可以保我魂魄不散,邪气不侵。我全等着你哪日给我做一副皮囊。

我摇头说:“这怎么行,铸人是有违天理的。

她说:“我画魂,修魂。是为魂匠。你铸人,修人。是为人匠。你我二人都未遭天谴,怎么谈有违
天理呢。” 

我笑出眼泪来,指着她说:“你看,又妄言了。这世界上哪有魂匠这一说。搞不好,你说的《云鬼词》,就是魂词吧?那我还要背一套《人词》不成?
\newpage


她飘过来轻吻我的额头,双手拂过我的白发。

她说:“你不信也罢。反正我千般圣魔,只与
你说。” 


完。 


谢谢大家的支持。 

因为预计于 20 节完结。所以在 15~16 之间压缩了三节内容。大概在下周末应该会有
状态去修缮这三节。 

因为以前一直在写长篇,第一次压缩到两万字左右的篇幅,掌控力还是不够。原本预计一万字完结,还是拖拖拉拉到两万字。水平有限,望大家见谅。

因此结局仓促突兀,是意料之中。无论如何,
人匠已经全部完结。 

\newpage


谢谢。 


真的谢谢。 


方糖。 

大家,下篇文再见。

\end{document}
