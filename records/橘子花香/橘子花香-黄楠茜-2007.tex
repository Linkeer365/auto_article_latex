\documentclass{article}
\usepackage[utf8]{inputenc}
\usepackage{ctex}

\title{橘子花香\footnote{Click to View:\url{https://web.archive.org/web/20221026034200/https://rentry.co/m8cpy}}}
\author{黄楠茜}
\date{2007-09}

% \setCJKmainfont[BoldFont = Noto Sans CJK SC]{Noto Serif CJK SC}
% \setCJKsansfont{Noto Sans CJK SC}
% \setCJKfamilyfont{zhsong}{Noto Serif CJK SC}
% \setCJKfamilyfont{zhhei}{Noto Sans CJK SC}
% \setlength\parindent{0pt}

\begin{document}
\CJKfamily{zhkai}

\maketitle


\Large

阳光暖暖地照射着整个孤儿院,风儿绕过院子,轻轻吟唱着古老的童谣。我坐在橘子树下,看着一朵朵的橘子花,有一种忧伤的心情弥漫在我的心头
。 

“橘子花很美,也很脆弱。”哲雨曾经这样说过。是的,橘子花很美,薄薄的花瓣包裹着静静的安宁,有着甜甜的香味,却也透着青涩的酸香。它没有迷人的外表,更没有高雅的气质,只是安安静静地开
着,一副与世无争的姿态。 

风轻轻地吹着,橘子树发出“哗哗”的声音,
就和小时候一样… 


\newpage

NO.1 

我坐在窗台上看书,手指在书页间繁忙地翻动着,其实我无心看书,只是想在书中找到些回忆,我怕我遗忘过去,更怕过去遗忘我。我的手指在书的某页停下,我轻轻地把夹在书中的花瓣取出。哦,是橘子花的花瓣,小小的,有着天然的淡粉红色。我把它放在指尖,慢慢地呼吸,生怕把它吹走。它是童年留
给我最好的礼物 

九岁,我被爸爸从孤儿院接到了这个家。当我打开门,我发现家中有一个与我年龄相仿的少年和-个女人。我这才发现我在这个家显得那么多余我是爸爸的私生女!爸爸把我拉到他们面前,说:“然然,以后这就是你的家,这是你妈妈,这是你杰哥哥。”我看了那女人一眼,我知道,她并不是诚心愿意接受我。那个男孩子对我充满敌意,爸爸让他给我拿糖,他恶狠狠地全摔在了地上。我知道,他们都恨我,恨
我的妈妈,可我又能怎么样呢? 

后来,只要爸爸不在家,他们给我的不是白眼,就是冷言冷语。他们当我就是一条没人要的流浪狗
\newpage
,一条只会向他们摇头摆尾的狗!我从来不和他们计较,我会躲在房间中,不再出来。一天晚上,我不小心打翻了那女人心爱的水晶花瓶,那女人的尖叫声让我难受,她像猛兽般向我扑来,顺手给了我一巴掌,
我忍无可忍,夺门而出 

黑夜中,路灯把我孤单的影子拉得好长,我坐在公园的石凳上无助地哭泣。我想,要是妈妈在身边,那该多好,至少茫茫黑夜,我会知道,我的家在哪
儿…… 


NO.2 

妈妈是个很柔弱也很坚强的女子,爸爸带着行李离开了我,离开了这个家,她却勇敢地选择了我。当初为了和爸爸在一起,她不惜和外婆决裂,从S城搬到爸爸工作的F城。她以为她和爸爸会永远幸福,可是,爸爸最后还是离开了她,但她却从未恨过爸爸。记得有一次我问她,当初为什么不让我跟爸爸走,她再去过自己本来应有的生活。她说,因为我是爸爸唯一的女儿,只有在我身上,她才会记起爸爸曾经也
\newpage
爱过自己。她爱得单纯,可是,有爱并不代表一切。爱在她美丽的容颜上悄悄落下痕迹,她疲倦了,于是
她选择了放手。 

六年,整整六年,我从没有觉得我是个单亲家庭的小孩,我一直都很幸福。她陪我玩,和我一起看书,一起画画。这么多年来,她把所有的爱都倾注到了我的身上,甚至忘了分一点给自己。我尽量少惹她生气,我怕有一天她会后悔,她会放弃我,可她却没有。我依恋她身上淡淡的橘子花香气,仿佛在她的身边,永远都是春天。她是个会魔法的天使,挥舞着星
星棒,带着爱悄然来到我的身边。 

妈妈喜欢橘子花,每到橘子花盛开的季节,她都会带我去收集橘子花的花瓣,洗净后,装在透明的玻璃瓶里,粉红的颜色有着暖暖的幸福感,柔美得让
人心痛。我和妈妈一样,都深深地喜爱着橘子花。 

橘子花很美,也很脆弱。我静静地想着,惋惜和悲伤在心中交织,兴许,爸爸妈妈的爱情也是如此。他们爱得轰轰烈烈,爱得单纯,爱得那么地辛苦,
\newpage
可最后依旧敌不过时间的磨炼,各自退出了彼此的生命。兴许人生也是如此,会拥有幸福,也会拥有无尽的伤痛。在那冷冰冰的玻璃瓶里,因为没有温暖的空气,一切才不会凋零,但一同掩埋的还有唯美凄凉的
爱,以及永远无法触碰的幸福 

我以为我会永远幸福下去,可妈妈却败给了寂寞和心伤。那个静静的黄昏,太阳收回了它所有的光芒,妈妈服下了一瓶安眠药,带着一脸的安详,微笑着离开了我。或许死亡对她来说并不可怕,反而是最
好的解脱。 

那晚我哭了,月亮躲在云朵里偷偷地哭泣,无家可归的孩子在哭泣。屋后的橘于花的花瓣悄悄飘落
,橘子花也哭了…… 


NO.3 

妈妈去世后,我被送到了孤儿院。那里有着一群和我身世相同的小孩至少,在孤儿院里,我不会觉

\newpage
得我有什么特别。 

那年我六岁。那本应该是天真烂漫的年龄,可我小小的眼眸中却透着不该有的忧伤。妈妈的死留给我太多伤害,有时半夜我会突然惊醒,在床上不停地哭喊着要妈妈。这个时候,哲雨就会托着小烛台到我的房间里陪我。心涵会从她的房间里走来,和我睡在
一个被窝里。 

七年前,我被送到了孤儿院,心涵趴在窗台上,小声对哲雨说:“你瞧,她长得多可爱,像不像言
儿?”哲雨微笑着点头,表示赞同 

有一次,我穿着深蓝色的小熊睡衣,端着牛奶穿过走廊,路过心涵的房间时,心涵突然冲出房间,抓住我的肩膀,不停地叫“言儿”。后来,我才知道,言儿是心涵的亲妹妹,可是不幸去世了。“她和我长得很像吗?”我抬起下巴,把头仰得高高的问哲雨,他在我眼中总是那么高。“是呀,和你很像。”哲雨摸摸我的头。他的确是个很好的男孩了,碎碎的刘海理得很干净。他会在某个不经意的时间,悄悄塞给你一大袋糖果,让你甜蜜好一阵子。雨天会撑伞陪你
\newpage
一起走,情愿自己湿透,却从不让你淋湿一点点。伤心时安慰你,落泪时哄你开心。看到他微笑,每个人都会有看阳光的感觉,他是这个世界上最优秀的男孩子哲雨是英俊的王子,而总有一天,他会骑着白马带心涵回到属于他的城堡,童话故事中总是这样的。在我眼中,也只有心涵配得上哲雨,她会是最美丽的公
主,一定 

虽然在孤儿院呆了很长时间,可我的寡言少语却一点未变。我仍是个安静的小孩子,我只有在和哲雨、心涵一起玩时才会笑,然后又恢复我的冷漠。有一次哲雨开玩笑地说我应该改名叫笑笑,不然就真成冷血动物了。心涵替我辩护说,然然才不要改名呢,她才不要像你一样老是笑,像小好人一样。哲雨笑嘻嘻地说,那你就是说然然是小坏人啰。他们俩连斗嘴都那么有意思,我似乎只是一个多余的陪衬。于是,我看我的书,他们做他们的事,三个人总是开开心心
的。 

春天到了,心涵开始忙碌起来。她的小抽屉里放满了花的种子,每天有空时,她就像小花匠一样去
\newpage
照料她的那些花。她把花的种子埋在橘子树下她说,待花开了,你坐在橘子树下看书,便也可以看见那些花了。我笑笑说好,心涵拉着我的手说,然然你笑的时候多好看哪,就和言儿一样。我听了,心中只有隐隐的疼痛。“言儿会在天堂想你的。”我轻轻地说。
 

橘子花慢慢开了,我爬上树贪婪地呼吸着橘子花的香气,依旧是淡淡的。我看着粉嘟嘟的花瓣突然伤感起来,失落,悲伤,孤单,想念,那一刻全从心底涌上心头。我趴在树干上哇哇大哭,哲雨爬上树坐在我身旁,小声问我怎么了。我呜咽着告诉他:“我
……想……·妈妈,我很想她。 

妈妈,我真的很想你,不知道你在天堂里看得到初开的橘子花吗?我现在过得很好,你看到了吗?妈妈,我好想回到过去,静静地趴在你怀里,即使只
有一会儿。 


NO.4 

\newpage

后来,那些花真的开了,开得很灿烂那个早晨我还没醒,心涵跑到我的床边轻轻唤醒了我。原来,她的姨妈找到了她,准备接她回家。心涵可以回家了,她很开心,不断地告诉孤儿院里的人,她要走了。本来我应该为她高兴,可是我却不希望她走,一点也不希望。可是我和哲雨能做的只是陪她度过她在孤儿院里最后的时光时光飞逝,很快就到了心涵要走的前一晚。那天的晚餐我胡乱找了个借口没有去餐厅吃饭,我躺在床上紧紧抱着心涵送给我的维尼熊绒毛玩具,舍不得松手。突然,我的房门被推开了,心涵拿着一小盘蛋糕走进来。“然然,你怎么了?不舒服吗?”心涵挨着我坐下,小声地问。我放下维尼熊一把抱住心涵,她有些吃惊,摸着我的头发问:“然然,你说话呀,是不是不开心啊?”我哭着问心涵:“心涵,告诉我,你可不可以不要走,不要离开我和哲雨?”心涵沉默了。我哭得更大声了:“心涵,你不可以离开我,不可以离开言儿,不可以。你种的那些花儿还等你去照顾它们呢,它们还没有升放呢!”心涵抱
着我,一直没有说话 

后来,心涵还是走了,她宁愿我伤心,也要回
\newpage
到她的亲人身边去,为什么她那么狠心啊!我只能放开手,让她从我身边飞走。对于她,我想,多年以后
,也许只是一个朦朦胧胧的回忆吧 

两个月后,心涵回来了,那天的阳光格外地晴朗,可我却分外悲伤。心涵静静地躺在她的床上,任我怎样哭哑了嗓子,她都不理我,我开始有些惧怕。我伸手去抓她的手,哲雨一把拦住我:“然然,不可以,你冷静一点心涵不希望你这样。”“不要,不要!你放开我!”我在哲雨怀里拼命挣扎,我不相信心涵就这么离开我。我被哲雨拖到椅子上,我不明白他们为什么不让我去碰心涵。“姐姐一一姐姐,你起来呀,告诉我,你和我开了个玩笑告诉我,然然保证不生气。你起来呀!”我绝望地哭喊着,屋子里所有的人都哭了,我这才清醒地意识到心涵已经永远地离开了我,眼泪从我的眼角不停地滑落,在明晃晃的地板
上绽出一朵朵伤心的花朵… 

在心涵的葬礼过后,老师交给我一封信,是心涵写的。我把信揣在口袋里,慢步走到橘子树下,我

\newpage
坐下打开信封一 


然然: 

你好!看到这封信时,我已经离开你,去了另
一个世界。不要怕,我可以在那里看着你长大。 

当初,我见到你时就觉得你很像言儿,真的。后来,我发现你和她有着一样的习惯,不爱笑,虽然你们笑起来都很可爱;有着一样的深蓝色的小熊睡衣;喜欢橘子花;喜欢维尼熊和咸蛋超人;沉默寡言………有时我甚至觉得你完全就是言儿,但你的确要比
她更可爱。 

我走的前一晚你很舍不得我,其实我也一样舍不得你。如果不是我的生命太短暂了,我一定会留下来陪你,陪你慢慢长大。可是,我和言儿一样,都得了先天性心脏病,这也是爸爸妈妈抛弃我们的原因。不要怪我为什么不早点告诉你,我怕你受不了打击,
因为我是多么的爱你,不愿你受伤害。 

然然,为了我,你一定要好好活下去。记得要
\newpage
照顾好我的花哦!心涵我的眼泪落在了信纸上,我抬起头,努力不让它们落下来。我看到了心涵的那些花儿,它们在阳光下开得很热烈,很漂亮,就如心涵一样漂亮。心涵,放心吧,我会好好照顾它们的,一定
会的。心涵…… 


NO.5 

我原以为,我虽然失去了心涵,但我会和哲雨
一起长大,安安静静地平淡而快乐。 

一天下午,我正坐在房间里看书,老师敲门进来,带我去了大厅。在大厅里,我见到了一个中年男人,噢,是爸爸。他微笑着张开手臂,试图与我拥抱,而我却只是冷冷地看着窗外的橘子树,因为我想到了妈妈。爸爸举起的手臂尴尬地停在了半空中,然后
回到了它们原来的地方。 

“然然,和爸爸笑笑,好吗?爸爸找了你好久。”爸爸轻声对我说。“抱歉,我不想这么做。”我面无表情地说。“然然,别这样。”老师在我背后推
\newpage
推我,“去吧,他始终是你爸爸。”我回头看看他,他一脸恳求地看着我。“对不起,我没有办法。我没有办法忘记你离开我和妈妈时的坚定,没有办法忘记是你骗了妈妈,走后还要害死她。”我尖叫着,“我不舒服,我想回房间。”“然然……”爸爸似乎还想说些什么,我早已梧着耳朵跑开了哦,爸爸,我不是妈妈,我没有办法不恨你,没法忘记妈妈的孤独寂莫。我只能逃避你,因为我怕我会原谅你,因为我身上
流着你的血,因为你是我这个世界上唯一的亲人。 

我摊坐在房门口,我宁愿我生下来就是一个没有父母的小孩,至少这样,妈妈和爸爸就可以和我无关了。或者,他们从来就没有认识过,妈妈可以有属于她的爱情,过着幸福的生活,有个漂亮的女儿或儿子。爸爸也一样。他们可以一辈子没有缘分,这样对
妈妈,对爸爸,对我都好。 

“小然然,怎么又在发呆啊?我陪你玩好不好?”哲雨抱着麻麻熊,低下头问我。我看着他,却不知该如何回答,告诉他一切吗?他会懂吗?我无助地哭着,哲雨好脾气地蹲在我身边陪我。我看着他,用
\newpage
一种几近哀求的语气说:“哲雨,即使再也没有人理我了,你也不可以离开我,好吗?”哲雨说:“好啊,好啊,我不离开你。”我那混乱的神经有了一种满
足感,哲雨从来不会骗人的。 

可是,后来一切快得让我无法想象,恍若云烟。先是爸爸办好了手续,准备接我回“家”,可他连问都没问我愿不愿意。再是只有两天的时间让我和孤
儿院里的一切告别,面对哲雨,我沉默了…… 

直到第二天早晨醒来,我才明白,我只剩下一天,不完整的-一个白大。那天中午的午餐,面对哲雨,我犹豫了。我的脑海中一遍遍重复着要和哲雨说的话,可是我却一个字也说不出来。每当看着他那透明的眼眸,我心中总会有一种淡淡的心痛。哲雨要是
我亲哥哥该多好,我便可以名正言顺地留下来了。 

我失去了心涵,我不想再一次失去哲雨,这种疼痛是我一辈子也不会忘记的。“然然,你没事吧?怎么脸色那么难看?”哲雨关心地问。“是吗·可能昨……昨天晚上没睡好吧。”我慌张地说。“哲雨,
\newpage
我……”我试图和他说明一切。“嗯,怎么了?”哲雨疑惑地望着我。“没……没什么。”我又一次放弃了,我实在不想看到哲雨难过,我怕我告诉他我要走
后,他会不再理我。 

夕阳慢慢沉没,我看见爸爸银白色的轿车在余光下镀上了一层橘红色闪着光,穿过篱笆,向孤儿院驶来。我在大厅的门口等他,我看见哲雨正在花园里找我,大声呼唤着我爸爸走进大厅,看见我笑了笑,问我:“然然,你的东西呢?我们要走了”我尽量平静地说:“嗯一一我能在这里再过一个晚上吗?”“为什么?’爸爸问我。“我想再和这里告别,就一个晚上,明天我就和你回去,好吗?”我恳求道。“可是,我们说好今天来接你的。”他似乎不太情愿。“我根本就不想去那个不属于我的家,我只想留在这里!”我愤怒地大叫。“好好好……我明天来接你,你要听话。”他终于答应了,然后转身离去,我目送他
离开。 

当我打开大厅的大门时,我发现哲雨正木木地站在那里,一动不动地看着我。“哲雨……”我看着
\newpage
他,不知所措。“然然,你要走了是吗?”哲雨问我。我含着眼泪点点头,低头看着地板,不敢直视他。“明天就走,对吗?”“哲雨,我不想走,我不想让你一个人留在这里…….”我哭着缩在墙角。“然然,没关系的,你不要哭了。”哲雨笑着说,“我不怕孤单的,你不要哭了,乖。起来啦,我带你去一个地方。”他伸手拉起我,其实我知道,他难过得想哭,
只是不想被我看见…… 

哲雨拉着我到了橘子树下。“哲雨,带我来这里干什么呀?”“嘘一闭上眼睛。”哲雨故作神秘。我乖乖地闭上眼睛。“抬头,好啦,睁开。”在我睁开眼的那一刻,我看到了还未完全暗淡的天空显现出闪亮的星星。哲雨在我耳边轻轻地说:“然然,你要走,我真的很舍不得,这满天的星星就当我送给你的礼物好了,想我的时候,小星星会代替我陪着你。”我幸福得眼泪都要掉下来了,我微微笑着对他说:“谢谢你,哲雨,我想我以后会有很多“小朋友’了。
”哲雨笑笑。 

那一晚是我在孤儿院最痛苦的一天,但也是最
\newpage

快乐的一天,小星星,嗯,都是我的…… 

第二天我就离开了孤儿院,走时哲雨递给我一个小袋子,他叮嘱我上车以后才可以打开。坐在爸爸的后车座,我倚着我软软的维尼熊,打开了袋子。啊,是满满一袋子的橘子花瓣,发出淡淡的香。一张小小的纸片躺在我的手心中:然然,最后一次给你送礼物,你要乖哦,哲雨。我把脸埋进橘子花中,悄悄地
哭泣,哲雨,我会很乖很乖的… 


NO.6 

后来,哲雨被一户很有钱的人家收养了。从此,他就再也没有回来过我要坚强勇敢地活下去,等有一天再次和哲雨重逢,我会告诉他,我依旧是那个很
乖,很安静的小然然,一直都是… 


NO.7 

很多年以后,我又回到了那个孤儿院,似乎一切都没有改变,橘子树小花园,土黄色的小秋千……
\newpage
…只是心涵不在了,哲雨不在了,只留下了当年的小
然然,仅此而已 

风缓缓地吹过红色的围墙,我的心在橘于花的香气中感到一丝温暖,我靠在墙角细数着过去的回忆
,妈妈,心涵,哲雨 

静静的,轻轻的,我坐在墙角重温我的过去,橘子花的香气中找寻属于我的幸福。

\end{document}
