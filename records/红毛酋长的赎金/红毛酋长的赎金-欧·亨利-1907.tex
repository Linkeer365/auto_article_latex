\documentclass{article}
\usepackage[utf8]{inputenc}
\usepackage{ctex}

\title{红毛酋长的赎金\footnote{Click to View:\url{https://web.archive.org/web/20221009130834/https://www.99csw.com/book/8234/288066.htm}}}
\author{欧·亨利}
\date{1907-07-06}

% \setCJKmainfont[BoldFont = Noto Sans CJK SC]{Noto Serif CJK SC}
% \setCJKsansfont{Noto Sans CJK SC}
% \setCJKfamilyfont{zhsong}{Noto Serif CJK SC}
% \setCJKfamilyfont{zhhei}{Noto Sans CJK SC}
% \setlength\parindent{0pt}

\begin{document}
\CJKfamily{zhkai}

\maketitle


\Large

看起来这是个好买卖;不过,你得等我把话说完。故事发生在我们——我和比尔·德里斯科尔——南下途中,经过阿拉巴马时突然起了这个绑票的念头。后来比尔把这说成是“一时糊涂”,但我们当时
并没有意识到。 

那地方有个小镇,地势平坦得宛如一张大饼,当然了,名字还是叫顶峰镇。镇上住的尽是些丰衣足食的农民,你完全可以想象得出这个阶层的人生活得
多么自在。 

我和比尔想合伙在伊利诺斯西部地区买块黑市地,但我俩总共只有六百来块钱资金,要实现这一计划,少不得还需要两千块钱。我们在旅店门口的台阶上坐下来商量。我们说,乡村集镇上的居民特别疼爱
\newpage
孩子;因此,再加上另外一些因素,在这里绑票比较容易得手,不像那些附近有报纸出版的地方,出了点事就被派去的记者搅得沸沸扬扬。我们知道,顶峰镇有几名警察,或许还有几条懒狗,案发后《农民周报》也可能登出一两篇文章,然而就凭这么点力量是抓
不住我们的。如此看来,是个好买卖。 

我们选中了镇上的头面人物埃比尼泽·多尔斯特的独生子作为我们的牺牲品。这位父亲很有地位,也很吝啬,经营建筑业,是个严肃认真的生意人。男孩子十岁了,脸上有些雀斑,头发的颜色像你赶火车时在报摊上买到的杂志的封面。我和比尔都认为,埃比尼泽至少也得给两千块钱的赎金,不过你还是等我
把话说完吧。 

离顶峰镇大约两英里路,有座草木茂密的小山
。后山上有个岩洞,我们的食品就储藏在里面。 

一天傍晚,太阳已经落山,我们驾着一辆马车从老多尔斯特的家门口经过,发现那男孩正在街上,

\newpage
朝对面人家栅栏上的一只小猫扔石子。 

“喂,小家伙!”比尔招呼说,“想不想吃袋
糖果,坐在车上兜兜风?” 

那男孩一甩手,一块砖头子儿击中了比尔的眼
睛,动作挺利落。 

“就这么一下子,你那老头子得额外多给五百
块钱。”比尔说着下了车。 

小家伙气势汹汹,像头半大不小的熊揪住我们一阵厮打,但最终还是被扔进车里,驰离顶峰镇。我们带着他到了山洞;我将马拴进树林,天黑以后又驾车赶到三英里以外的一个小村子将租来的车马还掉,
然后步行回山。 

比尔在脸上受伤的地方涂着药膏。洞口那块大石头后面已经生起火,男孩守在一旁看着一壶煮开的咖啡。我发现他的红头发上插了两根鸟尾毛。待我走近时,他举起手里的树枝指着我说:“哈哈!该死的

\newpage
白脸皮,你胆敢走进平原魔王红毛酋长的营地?” 

“他现在好了,”比尔说,又卷起裤脚看看腿上的伤痕,“我们扮印第安人玩来着。我们要让这小
子一辈子也忘不了在这儿玩的游戏。” 

真的,那孩子长到这么大,大概是头一回玩得这么开心。他觉得在山洞里住宿很有趣,早已忘记自己是给绑架来的了。他随即给我起了个名字叫蛇眼侦探,并宣布说,等他的那些印第安勇士打完仗回来,
日出时就将我捆在火刑柱上活活烧死。 

后来我们吃晚饭;他嘴里塞满肉片和肉酱以后便开始发表演说。他的席间谈话大致是下面这些内容
: 

“我很喜欢这样。我从来没在野外住过;不过我曾经有过一只可爱的野猫。我九岁的生日已经过了。我讨厌上学。吉米·塔尔博特的婶婶家,母鸡下的蛋给老鼠吃掉了十六只。这个林子里有没有真正的印第安人呀?我还想吃点肉酱。树动了是不是就刮风?我们家有五只小狗。你的鼻子怎么会这么红呢,汉克
\newpage
?我爹有很多很多的钱。天上的这些星星也是热的吗?上星期六我两次把埃德·沃克打败。我可不喜欢女孩子。没有绳子你就别想捉癞蛤蟆。公牛会叫吗?桔子为什么都是圆的?这个山洞里有床好睡觉吗?阿莫斯·默里长了六只脚趾头。鹦鹉会说话,猴子啊鱼啊
都不会。几乘以几等于十二?” 

每过几分钟,他一想起自己是个印第安人,就拿起那根树枝,像握着杆枪一样悄悄走到洞口搜索,看看有没有讨厌的白种人的侦探。他还时不时地发出一声喊杀声,老汉克听到这种声音就害怕。这孩子一
来就把比尔给吓唬住了。 

“红毛酋长,”我对孩子说,“你想回家吗?
” 

“咦,干吗要回家呢?”他说,“家里一点意思也没有。我讨厌上学;我喜欢野营。你不会把我再
送回去吧,蛇眼,是吗?” 

“现在不会,”我说,“我们要在这个洞里待
\newpage

些时候。” 

“好啊!”他说,“那就再好不过了。我长到
这么大从来没有这么痛痛快快地玩过。” 

我们睡觉时大约已是十一点了。我们在地上铺了几条又宽又厚的毛毯,让红毛酋长睡在我们中间。我们并不担心他会逃跑,可是一夜没有能够睡好觉。外面的树林里一有枝叶响动的声音,他那小脑袋瓜儿就以为有歹徒偷袭来了,于是一次次跳起身去取他那支长枪,并且在我和比尔的耳边一个劲地喊“伙计,你听”,害得我们三个小时未能入睡。最后我迷迷糊糊睡着了,却梦见自己遭了绑架,被一个凶神恶煞般
的红发海盗用铁链锁在一棵树上。 

天刚蒙蒙亮,我又被比尔的一阵极其尖利的叫声惊醒。你怎么也想不到一个男性发音器官里竟会发出这样的声音——既不是一阵吼叫,也不是一声长嚎,简直就像女人见了鬼或毛毛虫时发出的那种歇斯底里的、让人害怕而又难堪的一声声尖叫。一大早,又是在一个山洞里,突然听到一个壮汉如此尖声尖气没
\newpage

命似地叫喊,实在是不舒服。 

我翻身起床,看看到底出了什么事。原来是红毛酋长已经骑在比尔的胸口上,一只手揪着比尔的头发,一只手握着我们切肉用的快刀,正在为如何执行昨晚对比尔的判决而大伤脑筋,不知怎样才能完整地
割下他的头皮。 

我一把抢过孩子手中的刀,并强迫他重新躺下。但比尔从此变得丧魂落魄似的,在他的那一侧躺下后,因为有这孩子跟我们在一起,就再也没敢合眼。我虽然睡着了一会儿,在太阳快要出来时却想起了红毛酋长的话,日出时就要被绑在火刑柱上烧死。我倒不感到紧张,也不害怕;不过还是坐了起来,点上烟
斗,倚在身后的一块岩石上抽烟。 


“你干吗起这么早呢,萨姆?”比尔问。 

“我么?”我说,“噢,我肩膀这儿有点痛。
我想,坐着会好受些。” 

\newpage

“你在撒谎!”比尔说,“你害怕了。你给判了火刑,你害怕他会烧死你。要是他找着火柴的话,他真的会这样干的。这还不可怕吗,萨姆?你想,谁
肯出钱把这样一个小坏蛋赎回家呢?” 

“错不了,”我说,“做父母的就是喜欢这样淘气的孩子。喂,你跟酋长都起来做早饭吃,我去山
顶看看有些什么动静。” 

我爬上小山顶,将四下里的乡村扫视了一遍。朝顶峰镇方向眺望时,我本以为会看到有身强力壮的村民手执农具四处搜寻绑匪的,但映入眼帘的却是一幅宁静的风景画,唯一的点缀是一人一马在耕田。不见有人在河塘里打捞;也不见有人急匆匆来回奔走,报告焦急的父母说仍然没有消息。呈现在眼前的阿拉巴马整个儿还处于蒙眬的睡意之中。“或许,”我自言自语说,“他们还没有发现圈中的小羊已被狼叼走。老天保佑我们这两头狼吧!”我说着便下来吃早饭
。 

我走进山洞时却发现比尔背靠着洞壁站在那儿
\newpage
直喘气,小男孩举着半个椰子大小的石块威胁着要砸
他的脑袋。 

“他把一个滚烫的熟土豆放进我的衣领,烫我的背脊。”比尔解释说,“然后又把土豆踩在脚底下;我气不过给了他一记耳光。你身上带枪了吗,萨姆
?” 

我夺过孩子手里的石块,硬是阻止了一场争吵。“我会收拾你的,”男孩对比尔说,“打了红毛酋
长的人还没有一个不受惩罚的。你给我小心点。” 

吃完早饭,小家伙从口袋里掏出一块绳子捆着
的皮板儿,一边解绳子一边往洞外走去。 

“他又要搞什么鬼?”比尔忧心忡忡地说,“
他不会逃跑吧,萨姆?” 

“这倒不用担心,”我说,“他可不像是个喜欢待在家里的人。不过我们还是要拿出个讨钱的办法来。顶峰镇并没有因为他不见了而引起多大轰动;或
\newpage
许他们还没有意识到他被绑架了。他家里的人还以为他是在珍妮婶婶家或哪个邻居家过夜呢。但不管怎么说,今天总该想到要找人了。我们今晚一定要给他父
亲捎个信去,叫他拿出两千块钱把人赎回去。” 

就在这个时候我们听到了一声喊杀声,当年大卫很可能就是这样一声喊,甩出石块将勇士歌利亚击倒的。红毛酋长刚才从口袋里掏出来的皮板儿正是个
投石器,此刻正在他的头顶上挥舞着瞄准目标。 

我一跃而起,一声沉重的响声过后又听到比尔一声呻吟,像是马给卸下鞍子时的一声长嘘。一块鸡蛋大的石子击中比尔左耳后面,他全身散了骨架似地瘫倒在烧着洗碗水的热锅上。我把他拖到一边,往他
头上浇了半个小时的凉水。 

比尔终于慢慢坐起身,摸着后脑勺说:“萨姆
,你知道我最喜欢的《圣经》人物是谁吗?” 

“别紧张,”我说,“你已经清醒过来了。”

\newpage
 

“犹太王希律。”他说,“你不会走开,把我
一个人扔在这儿不管吧,萨姆?” 

我走到外面,抓住那小子的肩膀一阵猛摇,直
到我自己摇不动了才住手。 

“你要是还不听话,”我说,“我马上就送你
回家。你说,做个乖孩子呢,还是坏孩子?” 

“我不过是闹着玩的,”他哭丧着脸说,“又不是存心要伤害老汉克。可是他为什么要打我呀?我一定听话,蛇眼先生,只要你不赶我走,而且今天就
让我玩黑人侦察兵的游戏。” 

“这个游戏我不会玩,”我说,“那是你和比尔先生的事情。他今天陪你玩,我有事要出去一下。好吧,你进来跟他和好,你伤了人得先认个错,要不
你就回家,马上走。” 

我让他跟比尔握手言和,然后把比尔拉到一旁
\newpage
,告诉他走出山洞三英里有个小村子叫杨树湾,我想在那里打听打听顶峰镇对这起绑架有些什么反应。我还对他说,搞得好当天就给老多尔斯特捎封信去,直截了当提出要多少赎金,并指明交款的时间和地点。

“你是知道的,萨姆,”比尔说,“我俩一起玩牌,躲警察,抢火车,抵御龙卷风——上刀山,下火海,天大的困难我都跟你一起闯过来了。要不是抓了这么个小冒失鬼,我还从来不知道什么叫担心受怕哩。他已经弄得我寝食不安了。你不会出去很长时间
,让我一个人陪着他吧,萨姆?” 

“我今天下午肯定回来。”我说,“在我回来之前,你一定要好好逗着他玩,千万别把他惹翻了。
我们现在就给老多尔斯特写信吧。” 

我和比尔取出纸和笔准备写信,而此时的红毛
酋长,身上披了条毛毯,在洞门口来回巡视呢。 

比尔眼泪汪汪地求我把赎金从两千元减至一千五。他说:“我不想亵渎父母对子女的神圣的爱,但
\newpage
我们是在跟人打交道,按照人之常情,谁也不会为这个满脸雀斑四十磅重的野猫花上两千块钱的赎金。我宁可少要五百的好。你可以将这个差额记在我帐上。

为了让比尔安心,我同意了,于是两人你一句
我一句写成了下面这样一封信: 


尊敬的埃比尼泽·多尔斯特先生: 

我们已将你的宝贝儿子藏在一个远离顶峰镇的地方。别说你本人,就是最有本领的侦探也休想找到他。唯有答应以下条件才能使他回到你身边:给我们一千五百元大面额的钞票作为他的赎金;这笔钱可按照下述回信的方法,于今晚午夜放到同一地点的同一盒子里面。如同意这些条件,派一人于今晚八时半送来书面答复。在通往杨树湾的大路上,过了猫头鹰小溪后,路的右边沿麦田篱笆有三棵相距一百码左右的大树,第三棵树对面的篱笆桩底下放着一个小纸盒。

送信人将回信放入此盒后须立即返回顶峰镇。

\newpage

你要是背信弃义或拒不答应上述条件,你就永
远也别想再见到你的宝贝儿子了。 

你要是按要求交款,他将于三小时之内平平安安回到你身边。这些条件乃最后决定,即使有不同意
见,也不再联系。 


两个亡命徒启 

我在信封上写下多尔斯特的地址,将信揣进口
袋。正要动身,男孩走到我面前说: 

“喂,蛇眼,你说了你走了以后我可以扮黑人
侦察兵玩的。” 

“玩吧,完全可以。”我说,“比尔先生陪着
你玩。怎么玩法呢?” 

“我当黑人侦察兵,”红毛酋长说,“我骑马报信,通知寨子里的居民印第安人来犯的消息。我老是装扮印第安人,已经厌烦了。我想当个黑人侦察兵
\newpage

。” 

“行,”我说,“反正你伤不了一根毫毛。我还指望比尔先生会帮助你打退那些凶猛的野蛮人呢。

“要我做什么呢?”比尔不放心,眼睛盯着那
孩子看。 

“你来做马,”黑人侦察兵说,“给我趴下来
在地上爬。没有马骑我怎么能赶到寨子呢?” 

“你可别让他扫兴,”我对比尔说,“我们的
计划还没有开始实行呢。活动一下手脚吧。” 

比尔只得趴下,眼睛里流露出像兔子掉入陷阱
时的神情。 

“到寨子有多少路,小家伙?”他怯声怯气地
问道。 

“九十英里,”黑人侦察兵说,“你豁出性命
\newpage

也得准时赶到那里。现在就出发!” 

黑人侦察兵猛地跳到比尔背上,两只脚后跟还
在比尔腰上蹬了一下。 

“看在老天爷面上,”比尔说,“早点回来,萨姆,越早越好。早知道如此,我们不该把赎金定在一千元以上。喂,我说,你别踢我好不好?你要再踢
,我就起来揍你。” 

我赶到杨树湾,在那家兼卖杂货的邮局里坐下,见有进来买东西的当地老乡就凑过去聊上几句。有个胡子拉碴的家伙说,老埃比尼泽·多尔斯特的儿子也不知是走失了还是被人拐走了,顶峰镇乱成了一锅粥。行了,我就是想打听到这个消息。我买了些烟丝,又故意问问豇豆的价钱,走出邮局时趁人没注意将信投进了邮筒。听驿站长说,要不了一个钟头,过路
的邮车就会将这批邮件带往顶峰镇。 

我回到山洞时比尔和那小男孩却不见了。我在附近的地方一阵寻找,还大胆喊了两声也不见答应。
\newpage
我只好点起烟斗,坐在长满青草的土堆上等待事态的
发展。 

大约过去了半个钟头,树丛里传出窸窣的响声,比尔从里面钻了出来,拖着摇晃的身躯走上山洞前的那一小块空地。小男孩像个侦探轻手轻脚尾随其后,咧着嘴偷偷在笑。比尔站定后,脱下帽子,掏出一块红手帕擦汗。那孩子也止住脚步,离他大约有八英
尺远。 

“萨姆,”比尔说,“我想你也许会说我对不起朋友,但我实在是迫不得已啊。大丈夫能屈能伸,我已经逆来顺受惯了,但人总有个受不了的时候。那小子已被我打发回家了。全完了。古有殉道者,”比尔接着说,“他们干一行爱一行,宁死也不肯改弦易辙。可是他们当中没有一个受过我这样非人的折磨。我忍气吞声为的是信守我们共同商定的协议,但忍耐
毕竟是有限度的。” 


“出什么事了,比尔?”我问。 

\newpage

“我驮着他跑了九十英里赶到那寨子,没叫他走一步。后来,居民们得救了,给了我一点燕麦,毕竟地上的泥沙代替不了饲料。回来的路上,我又给他胡搅蛮缠了一个小时,反复向他解释为什么洞是空的,为什么一条路可以两头走,为什么草会发青。我敢说,萨姆,是人就经不起这么折磨。我揪住他的衣领硬是把他拽下了山。一路上我的两条小腿被踢得青一块紫一块;大拇指被咬了两三口,整个一只手都得找
医生治。” 

“不过他到底还是走了,”比尔接着说,“回家去了。我指着那条去顶峰镇的路,一脚把他送出去八英尺远。我很抱歉丢掉了一笔赎金,但如果不把他送走,比尔·德里斯特尔可就要被送进疯人院了。”

比尔说得直喘气,不过他那张红扑扑的脸看上
去却格外平静,说到最后才露出点满足的神情。 

“比尔,”我说,“你家里没人有心脏病,对
吧?” 

\newpage

“没有,”比尔说,“没人有这种病。除了疟
疾,那就是意外事故。你问这个干吗?” 

“那你不妨转过身,”我说,“看看后面是谁

比尔转过身看到了小男孩。他大惊失色,一屁股坐在地上,呆呆地抓弄起手边的青草和小树枝。我担心这样下去他脑神经会出毛病,考虑了一小时以后,对他说我已经有了个立即收场的办法,又说,要是老多尔斯特答应我们的条件,我们取了赎金连夜就离开。比尔这才缓过神来,勉强给了孩子个笑脸,并答
应身体稍好后就跟他玩俄国人打日本人的游戏。 

我有个安全的取款办法,不会落入任何圈套,应该介绍给以绑票为营生的弟兄们。我选中的那棵树——先在下面放回信,后在下面放赎金的那棵大树——离路边的篱笆很近,四周又有一大片空地。只要派几名警察在一旁守候,来取信的人在穿过空地甚至是在路上时老远就会被发现了。但这样反而不会出事,先生!我八点钟时已经躺在树上,像只树蛙似的坐等

\newpage
送信的人到来。 

果然很准时,一个半大的男孩骑着自行车从大路上来了。他在那篱笆桩子底下找着了纸盒,迅速塞进了一张折叠好的信纸,随即踩着自行车回顶峰镇去
了。 

我继续等了一小时,确信没有危险了,悄悄下地取了信,沿着篱笆溜进树林,半个小时后回到了山洞。我打开信,凑到灯前念给比尔听。信是用钢笔写
的,字很难认。主要内容如下: 


致两位亡命徒 

敬启者:你们的来信今天收悉。关于出钱赎回儿子一事,我认为你们的要求高了些,特提出反建议,谅能乐意接受。你们亲自将小孩约翰尼送回并付给我二百五十元现款,我就同意从你们手中接过孩子。不过你们还是趁夜晚来为好,因为邻居们都相信孩子是自己走失的,他们若发现被这样送回,会对来人采
取何种行动,我可担当不起。 

\newpage


埃比尼泽·多尔斯特谨启 

“简直是英国彭赞斯的海盗!”我说,“真他
妈的蛮横无理——” 

但我看了比尔一眼后,到了嘴边的话没有骂出口。他那苦苦央求的眼神太可怜了,我还从未在哪个人的脸上,无论是不能言语的哑巴或是会讲话的野兽
,见到过这样的神情。 

“萨姆,”他说,“二百五十块钱究竟算得了什么呢?这钱我们有。多留这小子一晚的话,我就会被送进疯人院了。多尔斯特先生只向我们要了这个价,我看他不但是个十足的绅士,而且是个慷慨仗义的
人。你不想放过这个机会,对吧?” 

“实话告诉你吧,比尔,”我说,“这个小兔崽子也已经叫我有点心烦了。我们把他送回去,赔了
钱就赶紧脱身。” 

我们当晚便送他回家。我们对他说,他父亲已
\newpage
经给他买了支银白色的来复枪,还特地买了印第安人的衣服,又说我们第二天要出去捕熊,才终于把他骗
上路。 

我们敲响埃比尼泽家的大门时,正好是夜里十二点。按原先的设想此刻本应由我从树下的纸盒里取出一千五百元赎金,而现在却是比尔数出了二百五十
元交到多尔斯特的手里。 

小男孩发觉我们要丢下他时,“哇”地一声哭了,哭声犹如狂风在呼号。他紧紧抱住比尔的腿,像只蚂蟥似的叮住不放。他父亲如同揭膏药一般慢慢把
他拉了过去。 


“你能拽住他多久?”比尔问。 

“我现在的力气也不如以前了,”老多尔斯特
说,“但我可以答应你们十分钟。” 

“足够了。”比尔说,“有十分钟时间,我就能穿过中部、南部和中西部各州,朝着加拿大边境飞
\newpage

奔了。” 

虽然天是那么黑,比尔又是那么胖,而我又可称得上是个飞毛腿,但是等我追上比尔时,他已经跑顶峰镇足足有一英里半远的路程了。

\end{document}
