\documentclass{article}
\usepackage[utf8]{inputenc}
\usepackage{ctex}

\title{花儿飞\footnote{Click to View:\url{https://web.archive.org/web/20221025145154/https://rentry.co/7qm37}}}
\author{史策}
\date{2005-09}

% \setCJKmainfont[BoldFont = Noto Sans CJK SC]{Noto Serif CJK SC}
% \setCJKsansfont{Noto Sans CJK SC}
% \setCJKfamilyfont{zhsong}{Noto Serif CJK SC}
% \setCJKfamilyfont{zhhei}{Noto Sans CJK SC}
% \setlength\parindent{0pt}

\begin{document}
\CJKfamily{zhkai}

\maketitle


\Large

那儿确实是一所好学校,自信的笑容弥漫在每一张陌生而年轻的面孔上。刚刚升学,大家都忙着结识朋友,尽快融人人群中去。在这“尖子生的学校,想争先可不像在初中那么容易。高翔必须抓住这一学期,崭露头角,这样才可以在同学们的心中先人为主。而找,是个默默的孩子,同学们在一起瞎唠时我也是安静地坐在一边,不聊天怎么能增进心情呢?可我就是不会看见什么就立刻发表几句俏皮的育论,我不幽默。同学说我有种孤做的气质。只有高翔温暖的
微笑是我惟一的慰藉。 

高翔在自我介绍时就被“闪"了二下。他微笑着对大家说:“我爱我的名字一高翔,它饱含了我的骄傲和希望,总有一天,我会自由地翱翔在属于我的

\newpage
天空上。 

下面响起了既不冷清也不热烈的掌声,同时一个自负的声音传来:“高翔,多么俗的名字。难道每一个高翔都能高翔吗?”声音不大,却字字真切地钻进高翔的耳朵。我看见高翔自信的笑容凝固在瞬间,
眼睛星星般地闪了一下,仍旧微笑着走下去。 


我们都记住了那个名字:马天成 

马天成的眼神很冷,目光锐利,但他也有着温暖的笑容,如果不看他的眼睛,是很令人愉快的。他非常聪明,而且初中时就当班干部,开学不久便当了班长。高翔是团支书,我由于噪子不错而被选为文委
。 

什么事在开始总是充满了干劲。然而这所好学校里汇集了各个初中的精英,大家都是一副骄傲的样子,学习的担子都很重。我的前几次小考成绩不突出,但在校艺术节上的表现使我备受音乐老师的青睐。我在班里没有特别亲密的朋友,高翔一有时间就会过来尽朋友的责任。高翔一直都很努力,成绩很好。日
\newpage

子像我的心情一样,平和又缓慢地过去了、 

第一次月考结束。高翔班内排第三,马天成第四一一我不知从何时就注意起马天成来。高翔在前面
回过头,我们相视而笑。 

“总有一天,我会在自己的天空上自在地翱翔
。”他张开双臂,阳光酒了一身。 

我们相约去看山上的秋叶。吹来一阵风,满山的植物使哗啦啦地唱起歌,又吹来一阵风,便有许许
多多红的黄的叶儿飞翔,扬起我们的好心情 

“翔,除了你,这个班里我就是孤家寡人了。
 

“没关系,这样也少了很多烦恼啊。日子还长,等你一点点地放出光芒你会得到全班的拥护和认可
。" 


\newpage

我微笑。 

他也绽放出好看的笑容:“我会给你最美好的
祝福。" 

我们一起遥望水灵灵的天空,一起吹山风,一起看山泉汨汨地穿林而过。我忽然觉得我们很美。“
青春啊,就是这样子的。”高翔说 


余光里忽然出现一个黑色的身影。 


马天成。 

那怪异的眼神让我不舒服。马天成不像个孩子,倒像个大人,而且始终穿着那件父辈们曾经很流行的黑色呢子大衣,更显得成熟稳重。他和同学们在一起说笑的时间是短暂的,他学习太用功了,这是他除了管理才能外更让人佩服的。他的内心,是什么样的
呢: 

不明白,也无法了解,我们说的话加起来都不超过十句。他让我有种压抑的感觉。可偏在这时,我
\newpage
和高翔碰见了马天成。马天成与我的目光撞上,正犹
豫着该走还是该打招呼 

“马天成!你也常来这儿玩吗?”我先招呼了
一声。 


“哦,是。” 


高翔回过头来,笑道:“夷,你也在?" 


“考完试散散心。 


“我们也是!” 


“这儿很安静。 

“是啊,自然的力量,让你的心不平静都不行
。吹,让人充满敬意,又精神抖数。" 


“这是你成绩那么好的原因吗? 

\newpage

高翔笑了:“也许吧。林静也特别喜欢这里。


马天成看看我,我笑了笑。 

马天成也笑了,我移开目光。“感谢上苍吧高
翔,你很幸福。” 

马天成下山的路与我们的正相反,往山后延伸
。他为什么朝那儿走呢山后只有一座精神病院。 

我很高兴高翔在班里表现优秀,他已开始准备“一二·九”的演讲。班级秩序很好,因为马天成和高翔都很负责。小王老师自然也干劲十足,看到我们这样更是信心百倍。只是马天成说话时。总喜欢问高翔;“我说的对吗?'高翔的回答总是“是,对”之类,因为马天成说的话像是经过筛选,每句都正确得体。马天成似乎也越来越“重用”高翔。“高翔,这个给老师送去。“高翔,这件事由你来做。”……仿佛高翔真成了他的“得力助手”一一我看更像秘书。
高翔对此,也隐隐觉得不舒服。 

\newpage


真让人头疼。 

终于有一天,高翔说:“你自己去就行,为什么非得我来做?”马天成面容婴时变得很严肃。他说:“为班级服务,你有什么理由推辞?我是班长,我还有我的事情。”这话的意思仿佛是:我们的政策是
我指挥你。高翔望着他吡咄逼人的眼睛,没有回答 

王老师大概无事可做,班会上要听听每个人的志向。高翔说,我要抓住青春,挑战自己,努力做好每一件事,让爱我的人为我骄做。马天成说,我要让所有人承认,我永远是最好的。王老师显然对他们的回答很满意。她喜欢这样的学生,她需要这样的学生为她赢得荣誉。而我说过之后,她只是微了撤嘴。我说,我长大后要挣足够的钱,然后和我的父母一起在一个有青山碧海的地方,种几块地,养些小动物,采
野味,观海,过安稳平静的生活 

第二天,便是演讲比赛的日子,每班有两名学生参赛。高翔和马天成坐在一起,我坐在高翔旁边。马天成轻声对高翔说,如果我们都得了特等奖那该是
\newpage
多美的事啊。高翔淡淡地说,你忘了,特等奖只有一名。马天成笑了,说,是啊,如果没有竞争,会幸福多少人啊。然后他们起身,很快便要上场了,同学们喊着口号为他们加油。马天成转过身,意气风发,向着大家说:“我要让所有人承认,我永远是最好的!”他盯着高翔的眼睛,微微笑了:“好好发挥,你的实力也很强。”随即叹了口气。“我们的竞争是不可避免的。不是你第一,就是我第--。哈,这样的角
逐,会比较有意思。 

高翔站在那儿,神色芒然地看着马天成的背影
直至消失。我走过去问他:“你们说了些什么?" 

同学们的欢呼和掌声突然爆发,马天成已站在台上,微笑着扫视下面的可学。高翔的叹息被掌声淹
没:“我根本没想同谁争,为什么会这样?' 

马天成的演讲和他平时一样,颇有领导风范,声调抑扬,剑眉微料,目光炯炯有神,手势恰当。不消说,那是个精彩的演讲。他走下台来,眼里流出关,空着高翔。高期,专心演讲!”找只匆匆嘱咐了这
\newpage

一句,他已经走上去了 

我不知道我是以什么样的心情看着高翔两次忘词,额上的汗珠在观众们的私语声中闪亮,干巴巴地
终于讲完了,迈着僵硬的步子下了台。 


王老师的脸色很难看。 

他低着头站在我面前,一句话也不说,我知道这次失败给向来优秀的他的打击。他是不愿面对全班
同学和王老师的眼睛,尤其是马天成的 

这对一向自信甚至自负的高期来说,着实是摔了个大跟头。高翔急于想找回自己的风采,可这学期再没有活动了,他心急,也只能忍耐。马天成对高翔的指使也更加威严,更无法抗拒。高翔第一次感到前
所未有的压抑 


到山上去吹风,越吹越痛 

天阴了,要下雪吗?凛视的风啊,吹走我们的
\newpage

烦恼吧。 

我和高翔在山脚下分手。在绕山的小道上,我又一次看见了马天成的背影。他为什么去后山?难不
成是去精神病院室 


和高翔分手后,我突然想去后山看着。 

精神病院的白色高墙分开了两个世界,两种人。站在椎近大门的高墙下,炫目的白色让我感觉那就是心理崩溃的边缘。正对着大铁门的是治疗大楼,楼后有个大操场,操场左边是医护人员的宿舍,右边关着精神病患者。这些,都是往常在山上俯瞰到的。这时,一个黑色的身影从大楼后面绕出来,朝大门走来
。那人果真是马天成 

马天成脸色平静,目光沉郁,对收发室老大爷点了一下头,走出大门看着他黑色的背影,我心里突
然沉重起来 


\newpage

“小姑娘,看人找人啊?先登记。" 

“大爷,您可知道刚才出去的那个人为什么来
这儿?” 

“来这儿的还能干什么?夷,你问他干吗?"


“大爷,我是他同学。他家里怎么了?" 

“他,唉……这孩子,喷喷……”老大爷叹了
一口气,又摇了摇头,最后点了支烟,说道: 

“这孩子,我们这儿的人,没有不知道他的。当爸的跑了,抛弃了糟糠之委,还有辛辛苦苦养了这么大的孩子,两年多了。这王八羔子!他妈疯了,一直住这儿。天成这孩子,懂事儿,一个月里看他妈得来个六七回。穷人的孩子·早当家呀!对人讲礼貌,学习还特别好,用功!在班里还当大班长昵,对吧?将来肯定有出息!就是命苦了点儿,叫人疼啊……哎,姑娘,份可别在班里说出丢啊,别伤人自尊心啊…
…” 

\newpage

漫天大雪纷飞,像天使因愤怒撕碎的纸片,扬
起漫天的悲歌 

开学了。王老师每次考试后都要按考试成绩重新分座位。第一排给那些眼睛不好的同学留着,第二排和第三排是给成续优异的同学的,第四排是奖励进步大的同学的。“高翔……。”王老师迟疑着。“就坐第四排吧,次没考好而已。”我小声说。“高翔,你坐那儿,那儿。"“不会吧,第五排?"有人惊讶的声音。“后面不是还有三排嘛。”老师不满地瞪了那个同学一眼。“高翔啊,这是咱们班的规定,你期末没有考好嘛。老师不能因为你是好学生、班干部就偏袒你,老师得公平嘛。你可不要有什么想法。”同学们都在议论高翔真是倒霉,王老师也真硬得下心来


“竟争就是这么残酷嘛。”王老师说 

倒霉的高期,他可知道,高中的老师爱分胜过
爱人,哪管你是谁! 

高翔的笑容少了。是不是每个急着努力进取的
\newpage

人都是这样严肃: 

然而,在开学后第一次月考中,高翔再次失利

我独自上了山,发现高翔蹲在巨石上,猫着腰


“翔,你在做什么?" 


“林静,来。 


我过去,他面前是一大堆奖状 

“我要把它们烧掉。让过去的一切都逝去吧,
重新开始。 

当这句话冲破他的喉咙时,一张张奖状雪花般飞舞起来,翻滚着投向山座。高翔的身影消失在树林
中,天空的脸突然变得很阴很阴 


肆虐的春风中,桃花就要开了 

\newpage

高中的孩子们,大了,可不像小学生那样,“领导”说一是一,说二是二。面对成绩下滑又失宠的高翔,谁还在乎?自习课上,当高翔喊出“都别说话了,安净点儿"的剩那,屋里一片寂静,然而大家仿
佛商量好似的"装”的一声,帽闹依旧 

“高翔是个不错的学生,但他在班里好像也没
有什么很要好的朋友,尤其这阵子心态不是很好。 


高翔?我躲在了楼梯口 


“是吗?我倒没注意。他平时与谁常接触? 

“呃……林静是他初中的同学,看起来关系不
错,体育课自由活动时间也常在一起交流” 

王老师没有说话,过了一会儿,那个声音小心
翼翼地说:“我常常在北山散步时看见他们。” 


“哦?” 

\newpage


…… 

外面的风很大。我仿佛听见了那瘦弱的花瓣离
开花朵的声音 

王老师是新班主任,出现了“问题”怎能不急于解决?高翔终于从办公室挪了出来。我望着他苍白的脸,告诉他:我永远是你的朋友,无论什么时候我
都会陪在你身边。 

高翔惨笑了一下:“王老师要找我妈随便聊聊

接下来的班会上,王老师做了一周的总结,同时也不很明了地稍稍提及了“那类事”。我什么也没
听见 


第二天,高翔背着书包被王老师堵在门口 


“你妈呢?" 


\newpage

“她不会来。” 

“高翔!老师和家长时常沟通是必须的。你那
倔脾气得改改,否则对你没什么好处。 

高翔冷笑一声,低下他苍白的脸,盯着王老师轻轻说:““你是很要强,可你一点都不懂得教育学
生的方法。" 

“你?”王老师的脸签时变了颜色。短暂的争吵后,她气急败坏地叫道“那么你就回家,你妈不来
,你也不用来!" 

高翔转头就走,王老师还气咻咻地。念叨;“
我还治不了你了?我还治不了你了!" 


高翔就再也没来过学校。 

我找又机起高翔站在巨石上,张开手件,迎风
微笑的样子来 


\newpage

在巨石下的缝除里,我找到了高翔的信 


林静 

你是我最好的朋友,从前是,现在是,永远都
是。 

上了高中我发现,所有的孩子仿佛一夜之间长大了,单纯不复存在。我真实地感受到生活并不简单

我怕的东西真的很多。我从不堡我的妈妈,地不容易。她为供我上学挨家借钱,说尽好话,看人家的险色,成绩不好时还要受别人笑话。妈妈发狠要我争气,好好供我,所有的亲戚朋友也都眼巴巴地看着我是不是有出息。我那么爱她,我一辈子也无法回报
她为我付出的一切,又怎能看着地失望,看她流泪 

所以我害怕失败,我怕一次次的失败会浇灭我妈妈的希望。面对自己的失败和马天成的压制,面对
众人一双双期待的眼睛,你知道我有多急吗? 

往事历历在目,我们常常说生活怎样怎样,其
\newpage

实我们都被玩弄在生活的股掌之中。 

我怎能把那件事告诉妈妈,让她去学校受比她小很多的王老师的“教育"?那天晚上,她对巷镜子找到了一根白发,要我拔下来,我翻着她的发丛才发现里面隐藏着她郡么多辛劳的代价。我一根一极地拔
,颤抖着手,泣不成声。 

我走了,我不知该如何面对妈妈,马天成少了我这个对手,终于可以毫无顾忌地说“我是最好的"
了。 


原谅我的离开。 

我算深深地体会到了,学校就是个缩小的社会,人情百态,一应俱全。我走了,你还要继续面对他们,你的生活又会有什么波溯呢?我不牌让你也社会
化,世故,但又怕你吃亏。你要保护好自己啊! 

我会回来的,为了我的妈妈,为了再与你在一

\newpage
起 


原谅我,保重! 


一辈子的朋友 高翔 

我早已净不开眼睛,眼前的信,模糊了又清晰,清晰了又模糊。他曾经是那么骄傲,那么自信,有着明亮的梦想和温暖的笑容,浪漫的单纯的心;现在,却如飞扬的花瓣,在风中离开了花朵,我不愿猜测它的结果。我多想再看到高翔昂着头,自信的微笑挂
在好看的脸上。 

我站在那块巨石上。手中的信折成了纸鹤,向
着我们遥望的方向飞去。 

“总有一天,我会自由地朝翔在属于我的天空
上。 

我仿佛又看见高翔站在巨石上,张开双臂,迎徽笑的样子。

\end{document}
