\documentclass{article}
\usepackage[utf8]{inputenc}
\usepackage{ctex}

\title{整形溢出·第六章·我们该相信谁\footnote{Click to View:\url{https://web.archive.org/web/20221012142258/https://rentry.co/e8bxm}}}
\author{徐西岭}
\date{2020-03-30}

% \setCJKmainfont[BoldFont = Noto Sans CJK SC]{Noto Serif CJK SC}
% \setCJKsansfont{Noto Sans CJK SC}
% \setCJKfamilyfont{zhsong}{Noto Serif CJK SC}
% \setCJKfamilyfont{zhhei}{Noto Sans CJK SC}
% \setlength\parindent{0pt}

\begin{document}
\CJKfamily{zhkai}

\maketitle


\Large

2018年8⽉5⽇,星期天。对绝⼤多数已经结束⾼考,并收到了录取通知书的⾼中毕业⽣⽽⾔,此刻正是他们苦读⼗⼏年来难得的放松时间。然⽽,某些考⽣的暑假却并不平静,例如郑州市第⼀中学的苏⼩妹(化名)以及其余三位来⾃河南各地的同学。他们信誓旦旦地声称,⾃⼰的⾼考成绩与平时成绩差距过⼤,其中必有第三⽅因素作祟。以苏⼩妹为例,在⾼考前两个⽉的各种测试中,她⼀向排在年级前100的位置,分数在600分左右。⾼考结束后,她⾃⼰给出的最低估分是627分,但她的真实成绩仅有335分。另外三⼈的分数和⾃⼰的估分也有不⼩的落差。这些考⽣据此认定⼀个事实:⾃⼰的⾼
考答题卡被⼈偷换了。他们现在要讨回公道。 

从关于此事件的最早⼀篇公开报道当中,我们
\newpage
可以看到更详细的情况。四位考⽣的家长实名举报“河南⾼考,滥⽤职权⾼考作弊、⾼考试卷偷梁换柱”,⽽苏⼩妹的家长苏洪先⽣本职即为政法系统的⼀名检察官,他感慨道“⾃⼰办案⽆数,却按照法律法规得不到政府公开信息。”在此之前,他们已经多次向河南省招⽣办公室进⾏了申诉,但对⽅的答复⼀直是“答题卡原件与扫描件对应⼀致,评卷系统对该⽣信息的读取与识别准确⽆误”。考⽣及家长对这种答复均不满意。⽽当考试院将答题卡展⽰给考⽣后,他们发现了不少更有⼒的证据:苏⼩妹“从来没有写过也没有见过放在苏⼩妹⾼考试卷名下的这段作⽂”;另⼀位考⽣余⼩芳(化名)的答题卡则“姓名、准考证号、座号多处严重涂改,根本就不是孩⼦本⼈的笔迹……考⽣条码多处涂改”;这些考⽣的家长认为“⾼考试卷偷梁换柱,⾼考分数肆意窜改掉包,致使河南⾼考考⽣⼤学梦碎,这类不法⾏为是……滥⽤职权、组织考试作弊内外勾结造成的”,他们“⼀怒联名向纪检监察部门举报河南省教育厅招⽣办涉嫌舞弊”。截⾄发稿时,“已有河南省纪委督办并交办河南省纪委驻河南省教育厅纪检组成⽴了专案组开展调查……

\newpage
⼒争还事实⼀个真相。” 

这篇新闻⼀经发布便引起了全国上下的注意,在各⼤⽹络平台上⼴为传播。显然,这是因为其中包含着相当引⼈注⽬的要素:在⾼考这样⼀个对中国⼈极为重要⽽严肃的程序当中,竟然可能出现偷换试卷这种骇⼈听闻的现象。各⼤社交媒体关于此事众说纷纭。绝⼤多数⼈都“希望彻查此事,给全国⼈民⼀个满意的交代”。但官⽅调查⾄少也需要经过数天才能下定论,在此之前便是⽹友们各展神通的时间。在知乎上,⼀位名为Crucialize的⽤户煞有介事地对可能的作案⼿法给出了分析:“……河南省教育考试院⾥有位领导X,他受关系户委托办事……X暴⼒涂改了4位关系户、4位提前选好的⾼分考⽣的考试信息,并破坏了条形码,为的是让扫描机抛出异常……扫描员发现考⽣信息被暴⼒涂改,⾃⼰做不了主,向上级领导X抛出异常……X批准按照试卷上的考⽣信息认定考⽣⾝份。”这篇回答很快便获得了该问题(如何看待⽹传「河南四位家长质疑考⽣⾼考答题卡被调包,成绩与平时相差近 300 分」⼀事,是否属实?)下最多的赞数。也有⼀位⾃称多次参加⾼考监考的⽤户提出:“每⼀个⽼师收起试卷后会
\newpage
再次核对条形码和准考证号,如果出现问题会⽴刻上报……答题卡被改成这个样⼦,作为监考⽼师的我觉得难以置信。”实际上,这两位⽹友的回答代表了主流公众舆论对此事件的初步看法:信任并⽀持四位考
⽣,要求考试院⾃证清⽩。 

不过,随着争论的升级,越来越多的疑点浮出⽔⾯。⾸先,苏⼩妹声称不属于⾃⼰的试卷上的笔迹和她考后默写的作⽂笔迹相似度极⾼,且都出现了“不负年少”这个使⽤率相对较低的短语;其次,有⽹友发现条形码被涂改后读出的信息属于⼴东⽽⾮河南(也就是说,如果有⼈作案,那还是个跨省犯罪团伙),且⽤于涂改条形码的油墨反射率和印刷油墨不同,根本不会影响机器读取。到这时,⽹友们⼤致分成了两派:⼀边怀疑四位考⽣⾔论的真实性,另⼀边选择⽀持以苏⼩妹为⾸的四位考⽣。其中尤为显眼的是和苏⼩妹同校的郑州⼀中学⽣群体。为了⽀持苏⼩妹,他们拿出了不少看似⽆懈可击的论据,包括她平时部分考试的成绩证明和她通过北京师范⼤学⾃主招⽣初审的证明。虽然这些证据在⼀定程度上帮他们扳回了⼀局,但当时没⼈料到这种⾏为会造成多么严重的
\newpage
后果。也有极少数郑州⼀中学⽣匿名表⽰了对苏⼩妹的质疑,但很快便遭受⼤量同学的⾮议,甚⾄“很多同学在班群⾥都对该答主进⾏了辱骂”。在这种情况下,这位同学不公开⾃⼰的⾝份是可以理解的⾏为。
 

新信息产⽣的速度令⼈⽬不暇接。⼀位⽹友通过技术⼿段,调取了苏⼩妹及余⼩芳两位考⽣在⾼中阶段的答题卡及考试成绩。在对她们的成绩波动幅度以及字迹的相似度作出分析之后,这位⽹友得出结论:“整个事件是她⾃导⾃演的⼀出戏,出于某种⽬的(例如本⾝成绩就没有那么⾼,或者是其它原因,不能随意猜测)⽽做了这些。”⽽当事⼈的表现也显得奇怪:苏⼩妹⼀家在事件⾼潮阶段⼿机关机,拒绝媒体的采访;⽽另外两名考⽣此时却在翻答题卡,鉴定字迹。再者,如果真的存在这么⼀个犯罪团伙,那他们需要“买通⾼考流程中起码数⼗⼈协助其作案,且能够同时买通河南省教育厅,纪委,公安局三个部门”,其⽬的仅仅是为了“调换⼏个号称500-600分的试卷”,这怎么看都显得有点⼩题⼤做,匪夷所思。⽽条形码的问题也很快有⼈下了定论:“语⽂
\newpage
和数学的条形码完全⽆误,仅数学⼿写考号有误,不影响条形码扫描;后两门考试,理综和英语条形码被涂改……然⽽“答题卡正反均有校验识别信息”→只涂条形码没⽤,还是会扫出考⽣本⼈的信息。”在多⽅⾯详实客观的分析下,舆论的风向标正⼀点点偏转
过来。真正的⼤结局也不会来得太远了。 

终于,8⽉11⽇,河南省纪委监察委发布正式通报。调查结果清晰明了:“4名考⽣各科⾼考场次的试卷和答题卡在启封、发放、回收、押运、⼊箱、封箱、出库、扫描、识别等关键环节均按规定程序规范操作,不存在⼈为调包试卷和答题卡现象……专案调查组委托权威专业司法鉴定机构进⾏了笔迹鉴定……答题卡上条形码及个⼈信息涂改系本⼈所为,不存在他⼈模仿笔迹作答和调包现象。”到这⼀步,试卷调包的闹剧可谓已经尘埃落定。然⽽由此衍⽣出的另⼀个问题却要严重得多——这都得归功于苏⼩妹那些急于求成的同学们。他们迫不及待地公开了她⾃主招⽣的通过证明,成功地将不少⼈的注意⼒吸引到了这⽅⾯来。根据⽹友们的查询,苏⼩妹通过了北师⼤⾃主招⽣的初审,但未通过复试,参加了中央传媒⼤
\newpage
学的⾃招,但未通过审核。更加令⼈关⼼的是她申请⾃主招⽣的材料:两篇分别名为《计算机⽹络技术在电⼦信息⼯程中的应⽤分析》和《天⽂学的基本性质与发展规律》的论⽂。⼀名⾼中⽣在完成学业的同时还能抽空在信息学和天⽂学两⼤领域分别完成并发表⼀篇论⽂,这种有悖常理的事不禁使⼈怀疑其中的蹊跷。果不其然,⼀名⽹友使⽤Paperpass⽹站提供的查重服务得出她《计算机⽹络技术在电⼦信息⼯程中的应⽤分析》这篇论⽂的重复率⾼达35%。⽽苏⼩妹⽗亲对此的回应则令⼈哑然失笑。他声称“天下论⽂⼀⼤抄……查重率超过30%才能叫抄袭。”但他却没有对这个检测结果给出任何看法。意犹未尽的⽹友们又在学术⽹站上查找所有郑州⼀中的学⽣创作的论⽂,结果出乎所有⼈的意料:这所⾼中的学⽣在完成教学任务之外,还能够⾼质量地完成⼤量横向课题;研究⽅向涉及语⾔学,电⼦信息⼯程,经济学等,其论⽂发表在国内多种期刊。其中有些学⽣的“能⼒”令⼈叹为观⽌,例如在⾼⼆⾼三的⾼强度学习下还能抽空参与⼗篇关于野⽣动物习性研究的论⽂撰写的刘某,以及在和⼀名博⼠共同⼯作的情况下还能担当第⼀作者的杨某。任何⼀个对现代科研体系
\newpage

有些许了解的⼈都能看出其中的耐⼈寻味之处。 

或许郑州⼀中的学⽣们到现在终于发觉了⾃⼰的问题,但这世上并没有后悔药可吃。在又有4名学⽣的论⽂被曝光抄袭之后,8⽉10⽇,河南纪委和调查组正式进驻郑州⼀中调查⾃主招⽣舞弊情况。⼴⼤⽹友对此可能闻所未闻,但在近⼏届的⾼中⽣看来,这种情况并不罕见。⼀位⾃称“在某⾃招初审⼈数多年位列全国TOP3的⾼中就读”的匿名⽤户写道,“……⾃招除了奥赛和作⽂⽐赛之外就靠专利和论⽂了,专利每年都有介绍会,⼤概是1500块钱左右⼀个专利可以⾛⾃招初审。论⽂也差不多。”甚⾄有学⽣声称“全国重点中学⼤半都是这样的,⽽且是和⾼校招⽣组达成默契的。”这种有恃⽆恐的⾔论很难不吸引别⼈的注意⼒。与此同时,也有⼈客观地分析了这种现象的成因。从这个答案列出的数据中,我们可以很容易的发现2013年这个拐点:在那⼀年之后,河南郑州各⼤⾼校的论⽂数量都出现了⼤幅度的上升。究其原因,还是要归结到我们之前提到过的2014年保送⽣名额的⼤幅收紧,导致不少本有机会参加保送⽣考试的学⽣不得不转⽽参与⾃主招⽣,
\newpage
使得竞争加剧,造成这种不择⼿段地争抢⾃招名额的乱象。但⾃主招⽣也不仅仅只针对郑州⼀个城市,全国各地的上千所⾼中都参与其中。他们会不会真的像
之前那位学⽣所说“⼤半都是这样”的呢? 

这个问题的答案应该在不少⼈的意料之内。8⽉16⽇,⼀篇名为《九省市⾼中名校学⽣论⽂涉嫌造假,或涉⾃主招⽣⿊幕》的⽂章登上了知乎⽇报。这篇⽂章的作者选取了九所全国知名⾼中的学⽣所发表的论⽂进⾏查阅,发现每所学校都出现了⾃主招⽣舞弊现象,且⼿法多种多样。他举出了不少⽣动详实的例⼦,其中有直接抄袭者,有多⽂拼凑者,也有更“⾼明”的和他⼈联合署名⽽将⾃⼰列在⾸位者。有理由相信,即使是这篇⽂章揭露出的问题也仅仅只是冰⼭⼀⾓⽽已。⾛到这⼀步,局势已经超出了所有⼈的控制范围。然⽽,对更多重磅消息翘⾸以盼的⽹友们要失望了:纪委和调查组的调查及处理结果并未公布,⼀段时间内也没有这⽅⾯的新进展。整个事态被
冷处理了。 

有⼀个问题值得我们注意:如此⼤规模的舞弊
\newpage
现象是如何不出差错地运⾏那么久,以⾄成为了某种“潜规则”的?⼀个很突出的原因就是了解这种“潜规则”的⼈绝⼤多数都是该体系的受益者;其余⼈则由于跟这部分⼈的种种联系(同学,师⽣等)从⽽选择对这种⾏为⼼照不宣。再者,就算某⼈决定举报揭发这种乱象,他也将⾯对跟河南纪委和调查组同样的难题:舞弊现象牵涉的范围过⼤,⼈数过多。即使不谈这些利益团体会如何阻⽌调查的进展,要对所有牵连⼈员全部加以处理也近乎不可能:这是从2014年起的数届⾼中毕业⽣中成绩相对顶尖的⼀批⼈,⼈数⾄少上万,且分布在各⼤985/211⾼校。这种法不责众的⼼态(也是现实)⽆疑在不少当事⼈的⼼中合理化了此种⾏为——但这并不能说明它有⼀丝⼀毫的正确性。每⼀位申请⾃主招⽣的学⽣都要抄写如下的承诺书:“我承诺,本⼈提交的所有材料客观,真实。如有虚假内容,⾃愿接受以下处理:取消今年⾃主招⽣的报名,考试和录取资格,同时取消今年⾼考报名,考试和录取资格,并视情节轻重3年内暂停参与各类国家教育考试。”那些私底下造假的学⽣在抄写这段话时究竟怀着什么样的⼼情,我们不得⽽知。虽然2018年的毕业⽣们暂时没有任何⿇烦,
\newpage
但⾃主招⽣却已经被推到了风⼜浪尖。⼀场改⾰势在
必⾏。 

实际上,如果不是因为“调包试卷”的噱头成功吸引了全国群众的注意⼒,⾃招乱象可能在很长⼀段时间内都会延续下去。从这个⾓度来讲,苏⼩妹和她的同学们为⾼校招⽣流程规范化和透明化做出了卓越的贡献。本事件和信息学竞赛并不存在什么联系,然⽽它对⾃主招⽣政策的后续影响却关系到了千千万万竞赛⽣的命运。和⼤多数⼈⼀样,竞赛⽣们在这场戏⾥扮演的是旁观者的⾓⾊。不过,很快他们就有机会⾛到台前,去⾯对部分⽹友的舆论冲击——尽管这
本质上也是⼀场闹剧。 

2018年8⽉17⽇,第27届全国中学⽣⽣物学竞赛在长沙市第⼀中学开幕。当天下午,选⼿们就参加了时长两个⼩时的理论考试,并在结束后参观了实验考试的场地。第⼆天,他们开始进⾏实验考试——动物解剖。今年的考题和往常有所不同:⽣物学竞赛的惯例是解剖诸如蛔⾍,蚯蚓,虾⼀类的⽆脊椎动物,但这次给出的题⽬是解剖⼀条鲫鱼,并准确
\newpage
地按要求找出它体内的五块⾻骼。不少选⼿看到题⽬时“⼀脸懵逼”,但还是认真地完成了这场考试。在决出50名国家集训队队员,并完成了签约环节之后,本届⽣物学竞赛于8⽉20⽇闭幕。对参赛选⼿们⽽⾔,除去感叹⼀下题⽬的反常和集训队员们的出⾊表现,这场赛事似乎已经没有什么值得讨论的地⽅了。然⽽,在⽐赛结束后的第⼆天(即8⽉21⽇),⼀篇耸⼈听闻的新闻标题成功占据了所有⼈的视野:“浙江4名⾼中⽣提前保送清华 原因是成功解剖鲫
鱼”。 

平⼼⽽论,除去标题之外,这篇新闻的内容并没有任何不得体之处,在开头就写明了解剖鲫鱼仅仅是⽐赛的实验环节。⼤部分篇幅被⽤于介绍四位进⼊国家集训队的浙江选⼿的学习经历和师友对他们的评价。多数⽹友对此的评论相对正⾯,例如“给我⼀条鲫鱼,我还你⼀碗浓汤”和“以后要转发鲫鱼了”此类具有调侃性质的⾔语。然⽽,也有不求甚解的群众提出了类似“解剖鲫鱼就能上清华?”的质疑,更有⼀位微博⽤户态度恶劣地写道:“杀鱼弟每天杀⼏百条鱼却⽆缘⼤学,官⼆代富⼆代杀条鲫鱼就进清华北
\newpage
⼤……⽤这样拙劣的⼿法来表演所谓的⾃主招⽣的科学独到,真是不要脸到了极点……现在是中国教育从
所未遇的⿊暗时刻。” 

这种⾔论很难不引发竞赛⽣的愤怒。⽹友们之所以会产⽣这种看法,有很⼤⼀部分责任要归咎于那个为了博⼈眼球不惜在标题⾥扭曲事实的记者。因此,竞赛⽣们的反击⽭头主要对准这些“⽆良媒体”。⼀位数学竞赛⽣认为:“⼀些媒体为获得流量⽽(⾃⼰掌握的信息不⾜,未经考证或考证⽽断章取义)写出歪曲事实、有辱⽣物竞赛和⾃主招⽣尊严的⽂章……在普通⼈看来,相较于⽣物教练和参赛者等作出的澄清,⼈们更加愿意相信阴谋论,相信这⾥仿佛真的有⿊幕……媒体错误的舆论引导,可能会让科普⼯作进⾏得⽆⽐艰难。”还有⼀名选⼿直接向中国动物学会举报了“微博营销号侮辱贵学会及竞赛⽣”的现象,得到的回复如下:“没必要理会这种肆意造谣的⾔论……学科竞赛得到社会和⾼校的认可,是学⽣的优秀换来的社会认可……我们学会⼯作多⽅⾯,专职⼯作⼈员少,⼯作都很忙,没有时间去理会这种造谣。”虽然选⼿们的愤慨⼀时间难以消除,但最好的应对
\newpage

⽅式的确就是动物学会建议的做法:不去理会。 

媒体在关于学科竞赛及竞赛⽣的报道中出现偏颇与不实之处早已不是第⼀次。在2018年5⽉份,就有⼀篇名为《奥数天才坠落之后》的⽂章受到了各⽅⾯的争议。这篇⽂章的被访者是曾经的IMO(国际数学奥林匹克)⾦牌得主付云皓,现在是⼀所师范学校的数学教师。作者把他的执教⽣活描写为“重复的讲课和做题让他的思维不得不拉低到学⽣层⾯……他很久没有体会到数学的乐趣了。”字⾥⾏间流露出⼀种对天才虚掷的惋惜。与这篇⽂章的看法截然相反的是付云皓本⼈对此的回应:《奥数天才坠落之后——在脚踏实地处 付云皓⾃⽩书》。按照他本⼈的总结,“该⽂章的作者笔下传递的观点是:优秀的⼈从事基础⼯作,就是⼀件很可耻的事情……成了付云皓这种去给“⼆本师范⽣”讲课的⼈,那就是天才坠落了。”但付云皓的观点却是:“我只想尽⾃⼰的⼒量,让初等教育越来越专业化越来越有⽔平,提⾼师范⽣的教学能⼒,让尽量多的孩⼦受到正确的引导……现在的我就是稳稳地在平地耕耘的我。没有所谓的⾃⽢堕落,没有所谓的“伤仲永”。”⽆独有偶,当
\newpage
时的知乎热榜上还有另⼀个引⼈注⽬的问题:“如何看待两次 IOI ⾦牌,⼀次 ACM 全球总决赛亚军的清华⼤学计算机系毕业⽣胡伟栋去⾼中当信息学教师?”提问者认为“他这样的条件……可以申请去国外名校读博甚⾄任教,可以去⼯业界顶尖的公司当科学家或码农……这么好的条件去做信息学教师
是不是浪费了他的计算机天赋?” 

归根结底,这反映了公众对竞赛⽣的⼀种普遍看法。学科竞赛的难度相对较⾼,这⼀点⼤家有⽬共睹,因此竞赛⽣在以学业成绩为核⼼的价值评价体系中毫⽆疑问地占据着⾼位。出于⽂化传统等种种原因,这种评价体系给⼈的印象是如此之深刻,以⾄于⼈们(有时包括竞赛⽣⾃⼰)理所当然地认为在竞赛中取得佳绩的学⽣就应当能在科研或技术领域做出⼀番事业,拥有同样优越的社会地位;要是和普通学⽣⼀样做了教师,就等于是“堕落”,“浪费了天赋”。这种看法的另⼀⾯,便是像那位微博⽤户的⾔论⼀样,将竞赛⽣统统称作“官⼆代富⼆代”;将竞赛视为“官富⼆代进名校的腐败通道”。此种“竞赛⽣权贵论”的荒谬之处⾃不必多说。我们感兴趣的地⽅在于
\newpage

,为何会有⼈对这类说法坚信不疑? 

应当承认的是,竞赛学科的发展的确存在很⼤的地域不平等性和资源⾼度集中性。在2018年的五⼤学科竞赛决出的468块全国⾦牌中,排名前列的七个省份(湖南,浙江,四川,上海,湖北,河北,⼭东)就占了超过半数(250块);⽽如果按⾼中计算的话,排名前10%的学校获得了超过40%的⾦牌。五⼤竞赛国家集训队的情况也与之类似:前五分之⼀的学校赢得了139个名额,是总数(260⼈)的⼀半以上。简单地将这种现象完全归咎于经济上的不平等是⽐较⽚⾯的看法。(这⾥并不否认经济对教学资源分配的重要影响。)另⼀个关键点在于所谓“竞赛传统”。它可以这么解释:如果⼀所学校曾经培养出过优秀的竞赛选⼿,那么这所学校就很有希望继续培养出更多优秀的竞赛选⼿。这个概念乍看上去没什么值得解读的地⽅,但对于部分竞赛弱校⽽⾔,它的意义却⼗分重⼤。为了证明它的正确性,我们需要⼀个⾜够有说服⼒的例⼦,⽽这个例⼦的主⼈公是中国信息学竞赛发展史上最具有传奇⾊彩的⼈物

\newpage
之⼀,其事迹和毅⼒令⼈叹为观⽌。 

2016年7⽉26⽇,四川省绵阳南⼭中学。在NOI2016的闭幕式上,NOI科学委员会主席王宏博⼠正在作最后的致辞。除去⽼⽣常谈的总结和展望,他特别点名了⼀位选⼿,原因是他进⼊了国家集训队。这个理由相当奇怪——每年的集训队员有整整五⼗位之多。可想⽽知,除去这⼀重⾝份以外,被点名的选⼿本⾝必有其特别之处。事实上也的确如此。他是NOI开赛⼆⼗多年来唯⼀⼀位打⼊国家集训队的⽢肃选⼿;⽽在他之前,⽢肃省的选⼿们获得的最⾼奖项是⼀块⼗年前的NOI2006银牌。
这位选⼿就是来⾃西北师范⼤学附属中学的吕欣。 

从吕欣⾃⼰的退役记⾥,我们可以对他的OI⽣涯有⼀个初步了解。他在⼩学六年级时学了⼀点C++语法,然⽽初中三年间都没有“学过什么算法,做过什么题”。直到进⼊⾼中后,他才开始系统地学习信息学竞赛。虽然他第⼀次NOIP只得了230分,他仍然获得了“全省⼀等奖,第三名”,并在省选中以全省第⼀的成绩顺利进⼊省队。不出意外的是,即使他的⽔平在⽢肃省内绝对领先,相较于NOI
\newpage
的要求也差了太多。不过,参加NOI2015的经历还是让他“见了世⾯”,“知道了外省竞赛⽣的那些不曾想象过的福利”。⾼⼆的⼀年⾥,他“翘晚⾃习,翘⾃习课,甚⾄翘体育课,找到任何可以利⽤的时间刷题”;想去参加CTSC&APIO,却被“⽢肃派不出指导⽼师”这种理由所拒绝;到后期他甚⾄向学校请假,独⾃待在家中做题。顶着“退到百名开外,勉强爬回来,又退到年级两百……和班级、学校荣誉绝缘”的压⼒,吕欣在THUSC2016当中和清华招⽣办签订了⽆条件降⼀本协议。他终于可以发出“⾃⼰的努⼒没有⽩费,感觉⾃⼰有了依靠……被认可的感觉真好。”这样的感慨。接下来发⽣的事情就是上⼀段中所提到的:吕欣以全国第18位的名次打进国家集训队,创造了⽢肃省前⽆古⼈的成就
。 

对他的母校,西北师范⼤学附属中学来说,吕欣的事迹是值得⼤书特书的。⽽尤其令⼈欣喜的是,他之后⼏届的同学们也取得了相对可观的成绩。⾃2016年起,西北师⼤附中的省级⼀等奖⼈数及NOI奖牌数都逐年增长。毫⽆疑问,这跟吕欣本⼈有很
\newpage
⼤的关系,不仅由于他对⾃⼰后辈们的指导,也源于这个榜样对后来者的激励作⽤。这就是所谓“竞赛传
统”的⼒量。 

从本质上讲,竞赛发展不平等的根源是信息的不对称。有浓厚竞赛传统,或经费充裕的学校可以通过种种⼿段获取更多的竞赛教学资源,以避免这种不对称;但两者都不具备的学校在这⽅⾯就显得捉襟见肘。这些学校的学⽣即使有⼼参加竞赛,也往往找不到门路,甚⾄有⼈在⾼中时从未听说过还有学科竞赛可以参加。但即使如此,把竞赛⽣全部指责为“权贵”也是毫⽆道理的。撇开吕欣这样的个例不谈,在竞赛教育资源丰富的⾼中就读的学⽣也很显然不⼀定具有优渥的家境,⽽竞赛需求的长时间学习和极⾼的淘汰率也注定了它并⾮⼀条所谓的“捷径”。公众会对竞赛⽣群体产⽣种种误会,最重要的原因还是由于没有深⼊了解。虽然主流媒体很快就对所谓“解剖鲫鱼上清华”的报道做出了澄清,但这件事不会那么容易就被忘记。在之后的很长⼀段时间⾥,它都会作为“社会舆论打击竞赛”的⼀个主要论据在竞赛⽣群体内被反复提及;⽽令⼈遗憾的是,⽐它更有⼒的证据还
\newpage

会越来越多。 

秋天已经到来,距离冬季降临还有⼏个⽉的时。

\end{document}
