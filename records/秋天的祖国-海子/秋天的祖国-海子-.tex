\documentclass{article}
\usepackage[utf8]{inputenc}
\usepackage{ctex}

\title{秋天的祖国\footnote{Click to View:\url{https://web.archive.org/web/20230730012652/http://m.wyzxwk.com/content.php?classid=23&id=278010}}}
\author{海子}
\date{}

% \setCJKmainfont[BoldFont = Noto Sans CJK SC]{Noto Serif CJK SC}
% \setCJKsansfont{Noto Sans CJK SC}
% \setCJKfamilyfont{zhsong}{Noto Serif CJK SC}
% \setCJKfamilyfont{zhhei}{Noto Sans CJK SC}
% \setlength\parindent{0pt}

\begin{document}
\CJKfamily{zhkai}

\maketitle

\setlength\parindent{0pt}


\Large


他说“一万年太久” \\ 


一万次秋天的河流拉着头颅 犁过烈火燎烈的城邦\\
心还张开着春天的欲望滋生的每一道伤口\\
  \\
秋雷隐隐 圣火燎烈\\
神秘的春天之火化为灰烬落在我们的脚旁\\
携带一只头盖骨嗑嗑作响的田徒\\
让我把他的头盖制成一只金色的号角 在秋天吹响\\
他称我为青春的诗人 爱与死的诗人\\
他要我在金角吹响的秋天走遍祖国和异邦\\
从新疆到云南 坐上十万座大山\\
秋天 如此遥远的群狮 相会在飞翔中\\
飞翔的祖国的群狮 携带着我走遍圣火燎烈的城邦\\
\newpage

如今是秋风阵阵 吹在我暮色苍茫的嘴唇上\\
土地表层 那温暖的信风和血滋生的种种欲望\\
如今全要化为尸首和肥料 金角吹响\\
如今只有他 宽恕一度喧嚣的众生\\
把春天和夏天的血痕从嘴唇上抹掉大地似乎苦难而丰盛

\end{document}
