\documentclass{article}
\usepackage[utf8]{inputenc}
\usepackage{ctex}

\title{百桥镇\footnote{Click to View:\url{https://web.archive.org/web/20221213142450/https://rentry.co/q4qvr}}}
\author{范锡林}
\date{2011-10}

% \setCJKmainfont[BoldFont = Noto Sans CJK SC]{Noto Serif CJK SC}
% \setCJKsansfont{Noto Sans CJK SC}
% \setCJKfamilyfont{zhsong}{Noto Serif CJK SC}
% \setCJKfamilyfont{zhhei}{Noto Sans CJK SC}
% \setlength\parindent{0pt}

\begin{document}
\CJKfamily{zhkai}

\maketitle


\Large

整整一夜,阿颖都在赶路,一刻也没有停过
。终于,在上午十点时分,来到了这座江南古镇 

望着镇口那座精雕细刻、苔痕斑斑的石牌坊,牌坊上镌刻着三个古篆字:百桥镇。阿颖抹去了额头上的汗水,长长地吁了一口气:“到了,终于到了。
” 

此刻,到古镇来观光游玩的人已经熙熙攘攘,络绎不绝了。这座古镇最吸引人的地方,就是步步见河,处处有桥。据说总共有一百座桥,所以叫“百桥
镇” 

在这里,河是弯弯曲曲、碧水澄澈的小河,桥是形形色色、造型各异的小桥。人们到这里来,就是
\newpage
来看这一座座桥。可以在河边慢慢走着看桥,可以在船上悠悠坐着看桥,可以斜倚在长廊里远远的看桥,也可以站在这一座桥上看那边的一座桥,甚至可以看
到如珠儿般串在这河上的好几座桥。 

可是,阿颖并不是来看桥的,他也没心思看桥,尽管这里的桥是那样的多彩纷呈,那样的雄健挺拔
,那样的精巧婉约。 

他是来走桥的,他要一座座桥去走一遍。从现在开始,到日落之前,他要将这镇上一白座桥,一座不漏地全部要走遍,而且,每座桥只能走一遍,不能
重复。 

牌坊下,一位穿紫花长裙的导游小姐正在用清亮如歌般的话语向大伙介绍:“如果有谁能在太阳落山之前,将这镇上所有的桥都走过一遍,那么在他身上,就会出现一个奇迹,那就是,他就可以回到他的昨天,而他在昨天所做的一切,就可以重新来过。”
 

\newpage


“真的吗?” 


“可能吗?” 


“啊,这太神奇了!”游客们七嘴八舌。 

“这是个传说,是这镇上的老一辈传下来的传
说。”导游小姐笑盈盈地回答。 

“有谁试过吗?”游客中有人问。“不知道,不过,你们谁有兴趣的话,不妨试试看,也许是真的
吧!”导游小姐不无调侃地说道。 

游客们发出一阵快活的哄笑,,从那笑声中,可以听出,他们谁也不相信这是真的,谁也不想去试
一试。 

然而,站在一旁的阿颖,他却是相信的,他就是因为执著的相信,这是真的,所以才连夜赶到这个
地方,来走这一百座桥的。 

\newpage

昨天发生的那一幕,至今还清晰得叫人心碎地闪现在他眼前·昨天早上,阿颖如往常一样,在明媚的阳光中,背着沉甸甸的书包,匆匆往学校走去,走到车水马龙的十字路口,正待要穿过马路时,他一眼
督见了一个骑自行车的纤细身影。 

这个身影是那么熟悉!她不是别人,就是与阿颖同桌了三年的榴榴,一个有着一对如星星般大眼睛
,和一对时隐时现小酒窝的女孩子。 

他们同桌的三年是阿颖感到读书最快乐的三年,尽管阿颖家境比较好,而榴榴的父母只是打工的,
但这丝毫也不妨碍他俩成为最要好的朋友。 

他们常常合用一张草稿纸,一块橡皮,一起分吃阿颖家里带来的巧克力,榴榴家里带来的地瓜干、葵花籽,常常一起走出学校门,走到这十字路口,两
人才分手。 

原以为,这样的快乐会很久很久,直到永远。可没想到的是,有一天早上,阿颖发现,旁边的座位
\newpage
上空了,榴榴没有来上课。老师告诉他说,榴榴是跟
她父母去打工了,再也不会来学校了。 

从那以后,阿颖就再也没有见到她,除了在梦里。而每一次在梦里见到榴榴的时候,阿颖总是有那么多的话要说,有那么多的事要问:“榴榴,你走的时候,为何不跟我说一声?榴榴,你现在在哪里?榴
榴,你现在一切都好吗?榴榴…… 

可往往没等到榴榴回答,阿颖的梦就醒了。如今,阿颖终于见到了她的身影,眼看她骑着自行车就要从马路那边掠过,阿颖忍不住激动万分地喊了起来
:“榴榴!” 

尽管十字路口是那么的喧闹,可这一声包含着那么多企盼的呼唤,如同一支射出的箭一样,还是让榴榴一下子就听到了,她情不自禁地一回头,自行车
龙头也随之往旁边一歪。 

就在这时,一辆从斜刺里开过来的汽车正好从她旁边疾风般地飞驰而过,猛然一下子,就把她连人
\newpage

带车撞了出去,撞出去很远很远。 

阿颖惊叫了一声,疯一般奔了过去,可是,眼
前的情景让他顿时惊呆了…… 

现在,阿颖就是为了榴榴,来走这一百座桥的。百桥镇内的河是条条相通,条条相连,主要的儿条河成了一个大大的“田”字形,而在这大的“田”字形之间,还有一些纵横交叉的小河,要不然,怎么用
得着这么多的桥呢? 

阿颖从东面开始走起,每走一座桥,他就在刚
买的百桥镇地图上标一个记号,记下一个数字 

一座、两座、三座……走了十几座桥之后,他就发现,这地方河连河,桥靠桥,所以,要想一座座桥走过来而绝不重复,是很难很难的。有时候必须要远远地绕一个很大的圈子,避开那些已经走过的桥;有时候甚至得坐上一只乌篷船,请船娘把他从河这边送到河那边去,为的是不重复走那几座己经走过的桥

\newpage

时间在分分秒秒地过去,眼看己经到响午时分了,阿颖一刻不停地在一座桥一座桥地走着,两腿都
已经累得拖也拖不动了。 

他在桥上的石栏上坐下想歇一口气,但一低头看到桥下清澈的河水里,倒映出来的一朵朵云彩,已经从棉花般雪白,开始染上嫩嫩的、淡淡的橘黄色了,他心里使说:“不,不能歇,太阳很快就要落山了
!” 

阿颖一咬牙,赶紧站起来,继续走,一座桥一座桥地继续走。这个百桥镇的桥也真多啊,大部分的桥都是明明白白的,架在纵纵横横的小河上,连接着
一条条青石板铺成的窄窄老街。 

可也有一些桥,却是隐藏在很僻静、很幽深的小巷里,那些桥,与其说是桥,其实仅是一条短短的麻石板,搁在浅浅的小沟上,称之为“步桥。可那也是一座桥,也不能少,也必须要走一遭。这就得细细
地去寻,细细地去找。 

\newpage

还有几座是廊桥,它们已经与古色古香的风雨长廊连在了一起,上面是绵延不断的廊顶,边上是一样的雕花栏杆,脚下是一样的青石板,如果不留心,压根儿不知道脚下其实是一座桥。这可就要特别当心了,绝不能糊里糊涂地走了一回,再来走一遭,走重
复了,那就前功尽弃了。 

天边的一轮夕阳,酒出万道霞光,映照在这一座座层层叠叠姿态各异的桥上,绝对是一幅用最神奇的色彩画出来的最美的画。可是,阿颖此刻哪有心情去欣赏,去品味?他在地图上数着自己已经走过的桥,其中有一半的桥,地图上都没有,是阿颖走过之后
自己标上去的。 

到现在为止,已经走了九十八座桥!还有两座桥,才满一白座。可是这两座桥在哪里呢?阿颖拿着标出自己已走过的那些桥的地图,去问本镇上的人:“大嫂,麻烦您了,您看,我今天已经走了你们镇上
的九十八座桥,还有两座桥,您知道在哪里吗?” 

那位扎着花头巾的大嫂看了那地图一眼,就笑
\newpage
了:“傻孩子这百桥镇的“百’是说我们镇上的桥多,并不一定就真的有一白座嘛。你走了九十八座桥,已经很了不起了,我嫁到这镇上十几年了,也没有走
过这么多的桥呢?” 

阿颖不死心,又找到一位戴着乌毡帽的中年汉子,再问:“大叔,打扰了,你瞧,我已经走了你们镇上的九十八座桥,还有两座桥,您能不能告诉我,
在哪里?” 

“喔唷,小兄弟,你很不简单嘛,能在一天里走了九十八座桥,要说我们这百桥镇嘛,恐怕也不一定真的有一百座桥,不过,据我所知,九十九座桥是
肯定有的。” 

“九十九座,还有一座,在哪里?”阿颖一听,赶紧问。那位大叔认真看了一下阿颖的地图,便指点着说:“对了,这条七星巷里,有一户姓鲁的,他
家的后面天井里有一座桥! 

“真的,太好了,谢谢您,大叔!”因为,在
\newpage
这一天里,阿颖已经跑遍了这镇上的大街小巷,所以
,他很快就找到了七星巷那户姓鲁的人家。 

“奶奶,我想走一走你们家后面天井里的那座小桥。”那位正戴着老花眼镜在门口纳鞋底的老奶奶一听,乐了:“哎你怎么知道我家后面天井里有座桥的,就是这镇上的人,也没有几个知道的呀,行,我
带你去!” 

来到她家后面,果然有一个天井,这天井其实是高高围墙里的一个院子,不过,尽管很小,里面居然有一条从外面引进来的水沟,上面当真有一座小巧玲珑的就像玩具积木一样的桥,只需一步就可跨过去
,可一本正经的装着矮矮的木桥栏呢 

走过了这座桥,就已经走了九十九座桥了。眼看天色已经渐渐暗下来了,最后一抹晚霞也即将在西
边的天际褪去了。 

难道这镇上真的就只有九十九座桥?难道那个走遍了一百座桥就可以回到昨天的传说,当真只是个
\newpage
虚幻的传说?阿颖心中的焦虑、沮丧、失望,交织在一起,但他仍不肯罢休他问这位额头上刻满皱纹的老奶奶:“奶奶,听说你们镇上只有九十九座桥,是不
是真的?” 

“谁说的,就是有一百座桥的嘛,要不然,怎
么能叫百桥镇呢?”老奶奶毫不犹豫地回答。 

“可是,我已经走了九十九座桥了,还有一座
桥,在哪里呢?”阿颖一听,赶紧问道 

老奶奶指点着说:“还有一座桥,一般人确实是不知道,那是一座天桥,是架在如意街两边楼房顶上的一根方木梁,当年是作过街天桥来用的,现在,那上面挂了些广告牌,所以,人们也就不知道它本来
是一座桥了………” 

没等老奶奶说完,阿颖就兴奋不己地说了声:
“谢谢!”转身就飞一般往街上奔去。 

转眼之间,阿颖就己经找到了那座架在如意街
\newpage
两边青瓦粉墙的楼房之间的天桥,他不由分说,一口
气登上了楼房的屋顶 

那根称之为天桥的木梁,仅仅只有一尺来宽,年代久远,一踩上去,嘎嘎作响,但阿颖还是义无反顾地跨了上去。看见那桥面上清晰地刻着“如意天桥”四个字,阿颖怀着一种必定能如愿的信念,一步一
步,既坚定又沉着,朝那一边走去。 


底下有人看到了,仰着脸惊呼: 


“啊,这孩子想干什么?” 


“这孩子疯了吗?” 

“小心,走稳了!”就在天边仅剩的一点霞光也融到夜幕里去的一刹那间,阿颖在那天桥上也恰好
走完了最后的一步。 

在街上抬着头观望的人们,惊讶地发现,走在那天桥上的孩子不知怎么,在他们的视线中一下子消
\newpage

失了。 

难道是因为天色昏暗,他们眼睛发花了,还是那孩子走过了天桥,立刻就从楼房顶上某一处离开了
? 

而阿颖呢,当他从天桥上走完最后一步时,他的眼前突然一亮,天桥、楼房、老街、古镇,一下子
全不见了。 

他正背着书包,站在清晨的明媚的阳光之中,眼前,是十分熟悉的十字路口,车水马龙,一片喧闹
。 

于是,他明白,他成功了,那个神奇的传说在他身上应验了。他在太阳落山之前,如数走完了百桥镇上的一百座桥,他果然已经回到了昨天,而在昨天
他所做的一切,就可以重新来过。 

就在这时,他看到了,一个熟悉的纤巧身影,

\newpage
轻快地骑着自行车,在街对面出现了。 

是榴榴,是他时时在思念的朋友,他最好的同桌!然而,阿颖没有呼唤,他只是站在那里,默默地看着她,看着她的身影从街对面驰过去,最后消失在
拥挤的人流之中。 

他相信,将来总还会有那么一天,他会再遇到的。到那时候,要不要将这段经历告诉她呢?

\end{document}
