\documentclass{article}
\usepackage[utf8]{inputenc}
\usepackage{ctex}

\title{安州、北川--春天里的蜀道行\footnote{Click to View:\url{https://web.archive.org/web/20221009125316/https://www.sohu.com/a/546068523_475768}}}
\author{周瑄璞}
\date{2022-05}

% \setCJKmainfont[BoldFont = Noto Sans CJK SC]{Noto Serif CJK SC}
% \setCJKsansfont{Noto Sans CJK SC}
% \setCJKfamilyfont{zhsong}{Noto Serif CJK SC}
% \setCJKfamilyfont{zhhei}{Noto Sans CJK SC}
% \setlength\parindent{0pt}

\begin{document}
\CJKfamily{zhkai}

\maketitle


\Large

我曾于2012年春天受邀去安县,参观踩
桥节。 

踩桥节是四川北部的一种民间风俗,每年春天举行,时间定为春社日(即立春后第五个戊日,也就
是立春后五十天,又叫逢社)。 

发出邀请的是鲁迅文学院同学安昌河,四川省绵阳市安县人。2010年我们一起上“鲁院”。他是比较沉默的一个,不太合群,常常独自思考着什么,好像还没有从当年的汶川地震中缓过神来。学期快结束时,院里为他的长篇小说《我将不朽》举办研讨会。书很厚,叙述手法颇为先锋,很符合四川作家的文风,奇崛诡异,充满神秘主义,像是沾染了大山密林潮湿的巫气。整本书里充满着各式各样的杀戮,评
\newpage
论家和发言的学员也指出了这一点,可他好像不以为
然。 

安昌河本名何长安,安昌河是他家乡安县的一条河,流经绵阳市和安县。安昌河初中没有上完,十六岁跑到山西挖过两年煤,亲眼看到矿难死人,再也不愿从事这项工作,回到家乡,自修中文,并开始写作,曾经写过两篇关于煤矿的小说,几部小说被改编
为影视剧,时在安县文化馆工作。 

2012年的踩桥节是3月18日。安昌河邀请了好几位同学,到跟前却各有原因,去不了了,只有我和丈夫及外甥三人,于3月17日由西安出发,
前往安县。 

下午五点多,在绵阳下了高速路,安昌河一家三口站在路边等待。夕阳下,胖胖圆圆的他、高挑亮丽的妻子、学龄前的儿子,并排而立,手拉手翘首以待的样子很是感人。看到我们的车,他带头跑过来,三个人还是拉着手不肯松开。那时安昌河还不到四十岁,胖嘟嘟的身躯稍显沉重,步态矫健不起来,带着
\newpage
一点沧桑和憨厚。此后每当电视里播放四川的广告片
,“熊猫故里”那个词,都让我想起安昌河。 

他打一辆出租车,头前带路,我们的车跟在后面,在宽阔笔直的辽宁大道上行驶了十多公里——这是地震后辽宁省援建的一条公路——进入安县县城。县城所在地花荄镇,本世纪之初才从原来的安昌镇迁到这里,是一个崭新的县城,道路宽阔,设施齐备。
 

我发现安昌河的妻子是河南口音,细问,原来是河南周口人。我说,一个四川,一个河南,你俩怎么认识的?安昌河神秘一笑,不作回答。我想,或许归功于网络。安昌河原有一段婚姻,一个女儿。新妻子名叫周丹,个头比他还略高一点,披肩发,白皮肤,容长脸,算得上漂亮女人,儿子白白圆圆,健康聪
明,继承了两人的优点。 

第二天,起个大早,到雎水镇上踩桥。满眼嫩黄颜色,一路油菜花香,一路粪便气味,两种气味都异常浓烈而明确,让你无可选择要哪个不要哪个,这
\newpage
正像是一个哲理,烘托出三月的川北大地。这种混搭气息不由得让人感慨土地的宽容与博大。春风拂面,路有弯道与缓坡,汽车从一个慢坡上向下冲去,像是一头扎进油菜花的海洋,进入一个不真实的梦境。一入小镇,但见人群乌泱泱只往一个方向流去,路边各种小吃摊点夹道欢迎。停好车走到镇街背后,眼前景象吓人不轻,成千上万的人簇拥着一座拱桥,变成了
人体之桥,血肉之桥。 

安昌河领着我们,挤入人群。不必用自己的脚走路了,只被前后左右的人推拥着,有一阵被架空起来,连脚都挨不到地。警察手拉手形成人墙,也不顶用,被人挤得忽东忽西,不能左右自己脚步。夹在人群中,半天也挪不动,喘气都困难,后悔已晚,退不出来,只好被人群架着拥着推着,听天由命。桥下的河里,水本就很少,此时铺满了钞票和衣服。丢钱是祈福,衣服是病人的,由家里人拿来,在桥顶丢下去,去病免灾。安昌河说有一年,一个有钱人,在桥上向河滩里扔百元大钞,一沓沓往下丢,天空下起了金
钱雨。人们也不挤桥了,都扑向水中捡钱。 

\newpage

用了一个多小时,从桥这头移到那头,又从人堆里挤出来,终于逃开人群,来到河边半人高的油菜花地里照相。路边地垄上,穿新衣的女人排成行走过,像是电影中的画面。从来没有见过这么多油菜花,花香给人以幸福感,再看不远处桥上桥下那些彩色人
群,想起一句话:火热的生活。 

晚上安昌河叫来几位当地文友,在一个半露天的地方用餐,有廊有水,灯光迷离。肉食用大盆盛着,味道极美。其中有一位叫林辉的男士,双眼皮,大眼睛,戴眼镜,总是锁着眉头,一股忧郁气质,张口说话,桀骜不驯。我们喝酒吃肉,大声谈笑,一任天冷下来,我又回房间加了衣服,大家也没有散去的意思。说到四川人的爱吃、会吃,安昌河说,曾经因为家里十天没有吃到肉,他父亲把锅砸了,冲母亲大发脾气。我说,在上世纪七八十年代,我们河南乡下,只有过年才会吃一次肉,此外常年吃不到肉的,也没一个人为此表现出异议,如果一个人爱吃讲究吃,会被人瞧不起,觉得你不会过日子。好吃懒做、不务正
业、歪门邪道这些帽子,会扣到头上。 

\newpage

周丹说,刚嫁来四川,很不习惯,见身边的人,天天都在外面吃喝,也不知哪儿来的钱。而安昌河
认为,你不吃喝,钱也没见省下多少。 

第二天,安昌河带我们去北川老县城地震遗址,他说,胆小的人,还是不要去。我不信这个,我想那一定是风声。假如去一趟能替那些死去的人承担一
些痛苦,也是应该的。 

北川老县城在一个山谷里,四面环山,山又非常之高,通向县城之路,就像是往一个大坑里走。遗址保留着当时震后的面貌,只是将道路修整出来,路边立了护栏,供人参观。各种单位门口,仍然立着白底黑字的竖长牌子,另有统一标牌写着单位简介,还有殉难职工照片及名字。有的房子歪斜,有的半倒,有的建筑塌成一堆;有的门面房卷闸门掉下来,一辆机动小三轮砸在里面;一家商店门口只剩下“嘉陵摩”三字,“托”字不知去向;有个窗户里面甩出来一半窗帘,贴在外墙上;还有一个窗户内,绳子上挂着一件洗净的衬衫;有一个小学,完全被山上滚下的石

\newpage
头盖住。 

不知道那些活着的人,还会不会回到这里,站
在楼下看看自家窗口,回忆从前的生活。 

县城很小,十多分钟走完。来到城边上的公墓
,我们买了菊花献上,向遇难同胞三鞠躬。 

安县老县城在安昌镇,重要机构迁走,如今只有镇的各种设置,变得十分安静。我们在公园里大树下喝茶聊天,有擦皮鞋的人趁机来揽生意。记不得是林辉还是哪位文友,邀请我们大家擦了皮鞋。林辉很
健谈,激动地抒发着他对文学和现实的看法。 

北川新县城精巧而美丽,城边有商业开发的羌族风情旅游区,几条街上卖工艺品、豆腐干、当地自酿酒之类。林辉执意要给我们买些手工挂面,死活拦挡下了。他趁我们不注意,进到另一个店里,不一会儿,手里提着两个盒子,沉甸甸走出来。夕阳斜照,风吹动他枣红色西服的右边衣襟,张开来像一个翅膀,瘦弱身姿稍显弯曲,一幅挺悲壮的样子。我心里大为不忍,听安昌河说他因为爱喝酒,爱招待朋友,经
\newpage
济常常吃紧,有时借钱过活,夫妻关系非常糟糕。这两天陪着我们的行程中,朋友们话里话外,多有责备
之意,劝说他不要再这样下去,他似乎并不在意。 

第二年,从安昌河的博客上,看到悼念林辉的文章,得知他在我们安县之行半年后,因脑溢血去世。他妻子有自己的生活,基本对他不闻不问,当然他也有自身的一堆问题,喝起酒来没有节制,每月工资入不敷出,所以死得很是仓促凄凉。林辉生前,是安昌河又爱又恨的朋友,他很是崇拜在文坛小有名气的安昌河,把他当好哥们,信赖有加,只要他在的场合,没人敢说安昌河一个不好。可也给他惹了不少麻烦,安昌河夫妻常常因为他闹得很不开心。每个小城,好像都有一两个林辉式的人物,热情、仗义、豪放、浪漫,总有自身克服不了的弱点,总是一身伤痛一堆
不如意,这仿佛是他们现实生活的标配。 

2017年11月的一天,安昌河来电,托我请贾平凹老师题写两个书名。我让他先写好短信发来,说出他这两个书名的重要性,请贾老师题写的必要

\newpage
性,总之,就是要打动名家。 

几分钟后,一条微信发来:冯翔是北川县委宣传部副部长,2008年汶川大地震中,他失去了近百位亲戚、同学、朋友,最疼爱的儿子也在这场浩劫中罹难。2009年4月,冯翔选择极端方式离开了这个世界。他生前创作的反映羌族百年风云的长篇小说《策马羌寨》和散文集《风居住的天堂》,2010年5月由长江文艺出版社出版发行,马上将由四川人民出版社再版。他的孪生兄长冯飞为了更好地纪念他,特别想请他们兄弟二人都非常喜欢和敬爱的贾平
凹老师题写书名,并敬奉润笔。 

经过与贾老师联系和等待,终于拿到了大作家题写的书名,贾老师分文不取。字还未干,摊地板上晾着,立即拍照发给安昌河。几十分钟后,他估计我离开贾老师处回到家,打来电话说,冯飞非常高兴,要乘高铁到西安亲取墨宝。我说不必跑来一趟,明天就快递去。他说,那你啥时来四川玩吧,冯飞和朋友在成都开餐饮多年,有好几家店,你来吃噢。我说,正有意春节期间四川行,不只为去成都吃美食,主要是想再去安县,写一写安县的你。他说好啊,安县现
\newpage

在改名安州区了。 

2010年5月的一天,我推开社长办公室的门,看见一个小巧亮丽的女孩子。社长向我介绍,她叫王佳,四川人,爱好写作,陕西师范大学研究生,来应聘编辑岗位。她后来找到了更好的就业岗位,没有来出版社工作,但一直保持联系。没想到她竟然是
安昌河的老乡。 

2017年,王佳已经是两个女孩的母亲,偶尔打电话问我中短篇小说投稿的问题、孩子上小学的事情。我顺便告诉她,春节期间可能会去你们安县。她说太好了,住到我家里吧,我家有房子,只是冷,我去买电暖气。我家在安昌镇,原是安县老县城,地震后划给北川县了,所以我现在是北川人……王佳说话很快,麻辣脆,用《红楼梦》里形容王熙凤的话,
就像是倒了核桃车子。 

随着春节临近,王佳过一段时间就问我啥时到她们那里,电暖器已经买好。已经放寒假回家的王佳,在微信群里发了安昌镇的早餐米粉,油汪汪红鲜鲜
\newpage
,很是诱人,还说,笋子米粉最好吃。安昌河说,对于老安县人来说,美好的一天,是从早餐一碗肥肠粉
开始的。 


人还没有入川,就被美食吸引。 

初一早上,我们一家三口,开车一路西南,过陕南的汉中和川北的广元、绵阳,路上已经见到油菜花羞涩矜持地点缀山坡,向我们宣告南方春早的消息
。 

冯飞他们在成都的餐馆春节不营业,说好在北川等待我们。初三早上,安昌河便在群里问我怎么安排,我说初四早上到。于是他开始了精心布置。冯飞说他刚才开车去镇上又采购了一些东西,现在在家专
心等待,快到时通知他,他下山来接。 

我们七点多从成都出发,九点从绵阳下了高速。眼前一个大花坛,分开两条路,想起六年前,安昌河一家三口,手拉手站在夕阳下的这个花坛前面等待我们。于是记忆激活,上左边这条路。时隔六年,我
\newpage
们再次行走在辽宁大道。二十分钟后,看到道路上方一个蓝色牌子,上写安州界,拿出手机欲给安昌河打电话,见他已经发来微信照片。走到跟前,果见一辆白色北京吉普停在路边。下车招呼后,他前头带路,
我们一起向安昌镇去接王佳。 

十点多来到安昌,在一个只有两幢楼的小小家属院里,站在楼下,安昌河大声呼喊王佳,听到她清亮的声音,来喽。于是我俩上楼,见她家大门敞开,夫妻二人大袋小包地提着吃的用的,丈夫怀里抱着小
的,妻子手中牵着大的,一起下得楼来。 

经过北川新县城,安昌河专门进入,绕了一圈,为让我们看看新城风貌。县城边上,还是那个羌族风情旅游区,想起六年前,林辉是从哪个店里出来,
手里提着两盒挂面,风吹起他的西服衣襟。 

冯飞在群里发出照片,充满羌族风情的吊脚楼,屋前的小阳台,阳光照耀,石桌上摆好了水果瓜子
,纸杯子放了两排。 

\newpage

又行几十分钟,道路边上,停着一辆小车,冯飞站在路边等待我们。狭窄的道路使我们无法下车,他挥挥手,让我们跟上。一百八十度拐弯,走到上山去的一条更小的路,只能容一辆车通行,路边的大山被硬切下来,像一堵高墙。羌族最早为北方游牧民族,历史上为躲避战争,从北方一路向南,逃往大山,多居住在山之高处。我们的车拐了无数个弯,快要走到山顶了,进入一个村庄。山里的村庄,居住都很分散,这里一家,那里一户,所谓邻居,是目之能见,声之可闻的百十米处,一个村子要扯出几里地。水泥路修到每家门口,房子自己盖好,政府负责外装修,统一为羌族风格,墙上贴着三坪村村规民约十四条。三辆车只能一个跟一个停在路上。一座约两百平方米的吊脚楼,进去之后,就像迷宫一般。冯飞的妈妈和两个女人在忙碌,灶台很长,连着坐了大中小三口锅,烧的木柴,一只烟囱由灶台通向房顶,各式凉菜已经装入盘子,看来是要好好招待我们。七八个房间被好多个门连接起来,穿过之后,来到屋前的——也可以说屋后的平台上。阳光普照,碧空中走着白云。冯飞的家人忙着招呼我们,绿茶是他妈妈自己种、自己炒的,水是山泉水。我问一位五十来岁的男人,你是
\newpage
冯飞的哥哥吗?那人笑答,我是他爸爸,今年七十二了。啊?众人一阵惊呼,都来围观冯叔叔,个头不高,身板挺直,眼角上挑,目光有神,尤其神奇的是,
一头细致柔软的全黑头发,在阳光下闪着亮光。 

左手平台外边,下坡处长着一簇新竹,刚刚蹿高的光竿上,还顶着笋皮,右边高坡上,种着青菜与草药。喝着绿茶,吃着瓜子水果,我们很是期待这顿丰盛的午餐。来了两个男人,和冯叔叔一起坐在门廊的太阳下聊天。过一会儿,两张圆桌,凉菜摆上,冯飞从屋里抱出一坛酒,拿来一个大茶缸倒出,清亮亮,淡黄色,边沿上冒出几个小泡。周丹已经将酒杯筷子洗好,安昌河在旁边给她和我们大家拍照。在向每个小杯里倒酒之前,冯飞说,今天不用开车的,之前我安排你们住在禹里镇最好的酒店里,房间都订好了。可安哥说,住在家里,体验真正的山中夜晚,周老师你愿意吗?我说,倒是非常愿意,只是这样给你们增添了麻烦,我们这么些人,哪里来那么多被子呀?冯飞说,这个你不用操心,看,下面那所房子,是我两个舅舅家,晚上你们住在那里。我们山里都是这样的,谁家有亲戚住不下,就领到别人家里去。这才知
\newpage
道,坐在廊檐下和冯叔叔说话的两个男人,是冯飞的
舅舅。 

冯飞父亲的家,在离此几里地的另一座山上叫杨家岭的小寨子。因冯叔叔年轻时在外当兵,复员后当了小学老师,常年不在家,冯飞妈妈为了家里有个照应,就将家搬到娘家这里,也就是说,冯飞的家,其实是舅家。现在厨房帮忙做饭的两个女人,是冯飞
的二舅妈和姐姐。 

凉菜上齐,大家围着两个圆桌落座,两位舅舅坐下来招呼大家喝酒。圆脸红润的是二舅,腼腆话不多,长脸黑黄的是幺舅,村文书兼五、六组小组长,也就是生产队长,热情开朗,颇见过世面的样子。这
位幺舅,还是个传奇人物,容我待到晚上细说。 

菜是自己种的,鸡是自养土鸡,腊肉是自己腌制,酒是用苞谷自酿,连豆腐都是自家磨的,切成大片子,和大叶白菜煮在一起,清水里捞出来,蘸着有辣酱的调料吃,不蘸也很好吃,更有豆腐的清香。冯叔叔不喝酒,只吃了一点菜,就抱走了王佳的小女儿
\newpage
,让王佳好好吃饭。头顶蓝天和阳光,喝酒,品菜,聊天,我们吃了一顿羌族待客的过年大餐。我问冯飞,想把你写进我的文章里,能用你的真名吗?冯飞爽快地说,当然可以,大名小名都能写,我小名叫健娃子,冯飞是我的曾用名,父辈取名的寓意是让我和弟弟连在一起“飞翔”,我现在用的名字是冯维政,只是大家还习惯叫我冯飞。我猜想,他的意思是,弟弟
没了,他也不飞了。 

几十分钟后,估计我们吃完饭了,冯叔叔把孩子抱回来,交给王佳。他又拿起扫帚打扫战场,冯飞也帮着收拾,冯叔叔说,我来弄,你带他们到后面山
上玩一玩。 

天气热了起来,几人换了薄一些的衣服,冯飞和安昌河背起相机,一群人下了他家的坡道,沿着山
路往后面走。 

北川县为国家级贫困县,但冯飞说,他们这里,基本不知贫困是啥滋味,空气好,山上物产丰富,从不挨饿,现在各种政策都好。他爸爸教书几十年,
\newpage
从教师岗位上退休,村子里祖孙三代都是他爸爸学生的家庭随处可见,爸爸思想单纯,受人尊重,没吃过苦,所以显得年轻。路过一家屋前,小小的一块三角形平地上,围着几人在绑竹竿,像是要做一个什么工具。冯飞从路上跳下去打招呼,掏出烟给几个男人挨个敬一支。继续往上走,又一户人家,门关着,两个门鼻上,横插着一根竹片,这就相当于锁子了。冯飞
说这里民风淳朴,没有发生过偷窃事件。 

路过村委会,冯飞的幺舅站在楼前空地上晒太阳。小小的两层楼门前,挂着三块牌子:中国共产党北川羌族自治县禹里镇三坪村支部委员会,北川羌族自治县禹里镇三坪村村民委员会,北川羌族自治县禹里镇三坪村日间照料中心。路边空地上,一块照壁,上书两行大字:听党话,跟党走,脱贫奔康有盼头。照壁旁边,倒下一棵大槐树,树皮已无,只有光溜溜的树干。冯飞说这棵槐树至少有几百年历史,上世纪六七十年代,革命群众聚在这片空地上开会,冬天太冷,在树下架起火堆烤火,次数多了,树被烤死,慢慢倒下,像一道拱门横在路上,人们还可从树干下弯腰通过。从这条路下去,走几里地,就是他爸爸的老
\newpage

家。 

再次感叹,山上修路,实为不易,要将大山一点点削开,一边是壁立千仞,常有碎石落下,一边是万丈悬崖,钢板拦着也让人心惊。安昌河与妻子儿子时而手拉手,时而搂着肩,三人差不多一般高,在山路上走成了一堵墙。孩子已经十一岁,夫妻还如此恩爱。周丹说,四川男人懂得疼人,平常家里,都是安昌河做饭,两人每年回一次河南,安昌河在饭桌上公然给她夹菜,娘家人瞪大了眼睛看。在河南乡下,很少有丈夫这样关照妻子的。周丹的舅舅私下说,这孩子是不是怕小丹不跟他了?为啥每次小丹回娘家,他
都要跟来,形影不离,难道怕她不再回川? 

回到村里,太阳已在西天,失去了热力。路边一小块地里,一位老人坐在凳子上翻拣一种白色草根,我们凑上去看。冯飞说这是韭菜根,地里长得太密,老人拣一些腌咸菜,夹馍吃特别香。它们被翻出来,晾一晾,过些天春分后再埋进去,就能长出韭菜。我问,拿回我们家,种在花盆里也能长出吗?冯飞说,能的。我惊奇,这看起来皱巴巴蔫了的根,又要离
\newpage
开土地几十小时,能重新发芽吗?冯飞说,生命力是很顽强的。刚好我口袋里有个小塑料袋,就拣了两块疙瘩根装进去。那位大叔回屋子里,拿一个稍大些的
塑料袋,给安昌河装了一满袋子。 

太阳又下去一点,天更凉了一些,感觉山里的
时间走得慢而宁静。 

晚饭仍是两个舅舅相陪,冯飞的妈妈、舅妈、姐姐在厨房忙碌,中午没吃完的肉类摆上来,新炒了几个素菜。大米稀饭里,煮着大块金黄瓤红薯,又甜又软。在不太亮的灯下,女人喝稀饭,男人喝酒,话也多了起来,时不时冒出深情表白。安昌河说,他和冯翔本是好友,冯翔多次说起,他有个双胞胎哥哥,在成都干事业,有机会介绍他认识,却一直没有实现。2009年,在冯翔的追悼会上,安昌河见到冯飞,抱住大哭,从此两人成为铁哥们。冯飞说,在北川的多个场合,人们见了他,都是突然一愣,吓得不说话,他知道对方将他当作了冯翔。地震和死亡,是一个残酷的话题,我们似乎都有意回避,不愿轻易触及

\newpage

喝得脸通红的冯飞,头脑还保持清醒,安排今晚的住宿,说得井井有条:王佳因娃儿小,不易走远,就住他家,她带着小女儿,住在冯飞身后这一间,王佳的丈夫王博士,带着大女儿住在旁边另一间;我们两家呢,住到坡下两个舅舅家,安昌河一家住在二舅舅家,妻子和儿子住一间,安昌河单独一间;我们一家住幺舅家,我和女儿住一间,丈夫单独住一间。他似乎怕我们不理解,或者说怕我们不遵守,特意说明,他们这里风俗,客人到来,不能夫妻住在一处。
我们纷纷表示理解并坚决贯彻。 

安昌河我们两家人被两个舅舅和冯飞引领,王佳陪着,手机照明,顺小路一阶一阶下了山坡,来到一座大房子前。两个舅舅的屋子连在一起,二舅夫妻俩常年在外打工,过几天就得出门,屋子里有些简陋。幺舅这里,屋里屋外装修挺好,家里摆设也很现代化,卫生间和冯飞家一样,冲水马桶、洗澡设施、太阳能、浴霸、洗衣机,一应俱全。不用说,这里一切用水,包括冲马桶,都是山泉水。幺舅带着我们参观屋里屋外,屋前空地,花圃里种着花草,具体品种在灯下看不真切。穿皮夹克的幺舅,细长身材窄长面孔
\newpage
,挥舞着手臂介绍他的领地,院子上空有几个摄像头,女儿给安的。现在咱们站在这里说话,一举一动,成都我的女儿都看得真切。一行人、连同二舅回到幺舅的客厅,大家意犹未尽,好像有一个重要的话题,怎么也不该越过去的,哪怕千回百转绕树三匝,它总是刻在北川人的心中,只不过是用了一种洒脱而通达的方式讲起。五十四岁的幺舅先起的头,在某一个或许是他自己铺就的水到渠成的茬口,突然说,我年轻时候是武警战士,看押犯人的,后来我自己成为犯人,被武警看押。他从手机相册里调出身着军装的照片,细长得像一根稍显弯曲的竹竿,头上军帽显得很大

起因是他一时迷了心窍,和别人一起,捉了一只金丝猴,卖了一千多元钱,几人分了。金丝猴是国家一级保护动物,凡捕捉必判刑。幺舅被判九年刑,关在北川县监狱,毕竟他有着武警战士的良好素质,狱中表现好,减了几次刑,2008年7月就可出狱

汶川大地震发生时,冯飞在成都工作,通讯中断,他找到成都交通广播电台发布消息,他将开车回北川,有需要搭车回家的北川人,于晚上十点在某个
\newpage
路口集合。因地震原因,成绵高速晚上十一点才开放。他在新都钟楼带着几个同乡,于第二天凌晨赶到安县的辕门坝,再冒着落石走回北川的擂鼓镇。在绵阳接到幺舅的两个女儿,告诉她们,你爸肯定是没搞了(死定了),县监狱和县医院相连,医院全部倒塌,人员伤亡惨重,监狱也不可能幸免。天刚亮的时候,从擂鼓镇走到接近曲山镇的凉风垭山口,冯飞见到妈妈和弟弟冯翔从对面走来,抱住他失声痛哭:冯翰墨
(冯翔的儿子)没了! 

他们怎么也想不到,此时幺舅正在忙着救人,他突然恢复成一名战士,顾不上与家人联系,指挥几个幸存的狱友,不停地从砖瓦堆里向外背人,致使电视新闻镜头里,有好几处他奔忙的身影。地震抢救工作结束后,家里人整理了电视新闻上幺舅背人的画面,他被宣布当场释放。回到村上的幺舅,因见过大世
面,担任了村里文书及两个村民小组的组长。 

活着真是太好了,每一天都是这么好,我现在活一天就开心一天。幺舅挥舞着长长的胳膊,龇着长

\newpage
长的牙齿说。 

夜深了,大家散去,幺舅招呼我们一家洗漱。在院子里的水管,接了热水洗脸刷牙。屋子里,幺舅夫妻俩给我们准备洗脚水,一个大塑料盆,哗啦啦往里倒热水。当我想到这是山泉水时,有种暴殄天物之感,可这里找不来不是山泉的水呀。大盆周边放了三只小凳子,三双新的棉拖鞋在门后排成一队,舅妈又拿来新的擦脚毛巾。我站了几秒钟,观望幺舅,他并没有离开的意思。在中原文明覆盖区域的河南陕西,女人洗脚,男人是要回避的,但远在巴蜀之地的羌民族或许没有这个讲究(就像冯飞把他称为小舅舅一样,在我们河南老家,小舅是骂人话),这位经历丰富的羌族汉子也不在意这些。那我也就入乡随俗吧,一家三口变成幼儿园的孩子,在他夫妻二人的全程注视下,乖乖脱了鞋袜,将脚伸进热水之中。夫妻二人坐在沙发上,幺舅还在滔滔不绝,牙齿闪着亮光,表达
着他对生活的热爱,那样子单纯得像个孩子。 

第二天早上,在冯飞家吃了早饭,安昌河一家和我们一家去北川老县城遗址。王佳孩子太小,不适合去。中午约在桑枣镇吃焦鸭子,到时冯飞带着王佳
\newpage
一家过去。这里的饭馆,常以姓氏和经营品种结合而起名,简洁明了,透着实诚:周蹄筋,陈排骨,杨肥
肠,宋包子,马油条,党米粉…… 

安昌河说,从前小城非常美丽,夏天的晚上,天热睡不着,三五好友,开车来北川喝啤酒唱卡拉OK,闹到半夜方休。从山上回望,北川县城灯光明亮,小巧玲珑,像一个盆景坐在谷底。县城最早在禹里镇,新中国成立后,南下干部来到这里,觉得禹里在山中,交通不便,他们到绵阳开会太远,而这个地方,在公路边,出行方便,建议将县城迁到这里。当时形成两种意见,地质专家说此处不适合居住,大山包围,川道里两面夹住,一旦发生泥石流,将无处可逃。但县领导一点点将一些重要机构建在这里,时间长了无法再回到禹里,危险说便一直存在。近年也一直有意搬迁,据说就在地震发生前不久,县城搬迁到擂
鼓镇的申请刚获批准。 

故地重游,又是春天。这里已经变成人气旺盛的观光旅游区,偌大的停车场几无虚位,好多工作人员分散在各个角落忙碌,车辆进出繁忙,牌号随便一
\newpage
扫就有十多个省份。导游的讲解声此起彼伏,前呼后
应。 

公路仍然从城边通过,每天有各种车辆穿行。日出日落,生活如常,昨日不再,人去楼空,这些曾经上演过各种人生故事的建筑物静静伫立,大树缀满绿叶,几株碧桃以废墟和坍塌为背景,绽放粉红色花
朵。 

离开的车上,周丹说,地震后,她娘家那里几个干部随河南方面救援队来参与建设,很是生气地说,我们在忙着干活,本地人却在吃吃喝喝,支起桌子打麻将。她刚嫁来时,也很是看不惯,觉得安昌河每天都在外面吃喝玩耍,周末这样倒也罢了,平常上班,也在河岸上支起桌子耍了起来。我说,那肯定是“八项规定”之前,现在恐怕不敢这样。周丹说,大地震后,那些活下来的人,都看开了,婆婆天天杀一只
鸭子,把安昌河的女儿吃成了小胖子。 

安昌河说,这与四川历史上多灾多难有关,人民常有朝不保夕之忧,一方面,成都平原的天府之国
\newpage
富饶丰美,人们会享受,吃的花样巨多;周边山地交通不便,苦寒交加,最主要是远离中原文明的儒家文化,不受正统约束,豪放达观,天真烂漫,形成了“蜀国人”能吃苦也会享受的性格,今朝有酒今朝醉,
也不失为一种积极的人生态度。 

生活滚滚向前,倒塌的,伤痛的,残缺的,丢失的,都将收起眼泪,合拢伤口。每一个春天来临,新开始新的生活。

\end{document}
