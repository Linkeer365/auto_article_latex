\documentclass{article}
\usepackage[utf8]{inputenc}
\usepackage{ctex}

\title{洗尘\footnote{Click to View:\url{https://web.archive.org/web/20221012144121/https://www.istitutoconfucio.unimi.it/wp-content/uploads/2016/03/di-an_racconto.pdf}}}
\author{笛安}
\date{2012-11-05}

% \setCJKmainfont[BoldFont = Noto Sans CJK SC]{Noto Serif CJK SC}
% \setCJKsansfont{Noto Sans CJK SC}
% \setCJKfamilyfont{zhsong}{Noto Serif CJK SC}
% \setCJKfamilyfont{zhhei}{Noto Sans CJK SC}
% \setlength\parindent{0pt}

\begin{document}
\CJKfamily{zhkai}

\maketitle


\Large

按道理讲,请客吃饭,一张桌子上不应该有六个人。连主人带宾客,五个人可以,七个人也没有任何问题,可是一直以来,龙城人的确有个说法,一张宴席的饭桌,六个人围着坐,有些不妥。没人说得清究竟哪里不妥,于是这个规矩就这样流传着。每个
人只有在小的时候,才会问“为什么”。 

可是今天这顿饭,非得六个人不可。一个主人,五个客人。虽然只要随便再拉来一个什么人,就躲闪过了那个古老的忌讳——但是,还真不大方便。第一个客人走进来,他们彼此对望的时候还是有点恍惚,尽管他已经在心里排练了很多次,他知道客人也早有准备——可是在看见彼此的那个瞬间,还是觉得难
以置信,怎么已经过去了那么多年。 

\newpage

“你老了。”客人说,声音里似乎还夹着户外的寒气。然后第一个客人又加了一句:“今天真是冷
。” 

“彼此彼此。”他笑笑,然后又说,“你看着
还好,我知道我自己变了太多。” 

客人也笑:“不用这么客气。三十年,谁能不
老?” 

往下就不知道该寒暄什么了。但是真维持着沉默,也不成体统。说什么呢?总不能说:上个月同学聚会的时候,听说你得了癌症。可是这位客人自己将外套随意地丢在一旁空着的椅子上,神色坦然地说:“没错,你用不着不好意思,肝癌,查出来的时候就转移了,大家都知道的,没救,不过习惯了就觉得也好。”他尴尬地说:“你能想得开就是最好的,什么也比不上能放下。”话音没落,他自己也觉得这句话接得太糟糕,紧张地命令自己住嘴,顺便端起面前的茶壶想替客人倒茶,水歪歪扭扭地砸到了茶杯的边缘上,像条可怜的瀑布,一分为二了,小小的一股流进
\newpage
了杯底,更多的顺着杯壁浸润到了桌布上。他突然笑了起来——见鬼了,可是他控制不了这个笑,渐渐地,笑得前仰后合了起来,他只好尽力修改一下笑声,
企图笑出些自嘲的味道。 

还好客人也跟着朗声大笑了。他们就这样对着笑了一阵子,茶杯在颤抖的笑声里被危险地斟满了。第二个客人进来的时候,就只好莫名其妙地看着他们,似乎觉得既然已经这样了,他初来乍到,不跟着笑有些失礼,但实在不知道这二位在笑什么,所以只能挂着一个对于应酬来说太温暖些了的微笑,等到室内重新恢复寂静的时候,第二位客人用一种轻手轻脚,过于谨慎地姿势走到他们俩跟前,拿走了那个摸上去
还烫着的茶壶。 

“这院子景致不错。”第二位客人选了一个离门最近的位子,安静坐下来。他须发皆白,是个耄耋
老者。 

“我也是找了好久。才找到这个地方。视野很好,正好能看见一整面山坡。春天的时候,花全开了
\newpage
,才最好看。”主人终于恢复了正常的神色,“好久
不见,沈老师。” 

“是不是该介绍一下?”第一位客人看着他。
 

“沈老师。我初中时候的班主任。教我们数学
。”主人转了一下脸,“这位是……” 

“鄙姓曲,沈老师,曲陆炎。我是他的大学同学。”面对老者,第一位客人的眉宇间有种自然而然
地恭顺。 

“大学。”第二位客人神色似有些复杂,“你
去上大学的那年,正好是若梅……” 

“1977 年。”主人打断了第二位客人,
“沈老师,若梅怎么没和您一起来。” 

“她还是老样子,害怕跟生人说话。临出门的

\newpage
时候,我想想还是算了。” 

“若梅是沈老师的小女儿,”主人拿起茶壶,
往沈老师喝了一半的茶杯里再斟了一些, 

沈老师有些慌张地欠了欠身子,“你不知道,”主人对第一位客人说,“沈若梅那个时候,是我们
龙城出了名的美女。” 

沈老师接着喝茶,眼睑垂下来对着茶杯底,完
全看不出表情。 

“1977 年的时候,她多大?”第一位客
人的语气里带着“什么都明白”的洞察。 

主人把菜单放在第二位客人面前:“沈老师先点菜吧,我对这儿也不熟,您喜欢吃什么,随便点。”接着扫了第一个客人一眼,看似轻᧿淡写地说,“
23。” 

第一位客人笑笑:“沈老师的女儿来不了,今天咱们还是只有五个人,不正好避过去你们龙城的忌
\newpage

讳?” 

“你怎么连这个都知道——我走了这么多地方,好像真的只有龙城才有这个规矩。”主人惊诧道,
其实他暗自庆幸话题终于可以离开若梅。 

“你自己告诉我的。”第一位客人,曲陆炎说,“有一年暑假,我跟着你回龙城玩,在你们家住了两个多星期,你妈妈还教我说了好几句龙城话,那时
候,你我无话不谈。” 

三十几年前,他们无话不谈。这似乎是一个不
错的,用来当作故事开头的句子。 

直到有一天,曲陆炎的女朋友成了他的新娘。

“要是今天有六个人,那再等最后两位来,就可以开席了,那两位是一起的。”主人的眼睛从曲陆
炎的脸上挪开,看着沈老师。 

“不急,不急。”沈老师笑道,“现在我们谁
\newpage

都不需要赶时间了,还急什么。慢慢等吧。” 

“林宛现在好吗?”曲陆炎似乎不打算继续粉饰太平。林宛就是他的妻子,也是曲陆炎最初的恋人
——是他们的女人。 


“我也不知道。”他诚恳地笑笑。 

“你今天为什么要请我们吃饭?”曲陆炎看似
漫不经心地环顾四周。 

“因为我们都死了。”主人回答,“这理由还
不够么?” 

沈老师死了,八年前死于脑出血之后的深度昏迷;曲陆炎也死了,去年冬天死于肝癌,这是他上个月才从同学聚会上听来的;他也死了,十天前的事情,算是俗称的“尸骨未寒”,死于突发性的心肌梗塞——他也是死了以后才知道自己原来有心脏病的。沈老师的小女儿,若梅也死了。死于 1977 年。

\newpage

葬礼之后,活着的人都还热热闹闹地活着;那么,死了的人也该一起吃顿饭才对。他不知道这边的世界里有没有这些习惯,只是他刚死没多久,还不适
应那种寂寞。 

主人推开门,招呼走廊上的服务生:“上凉菜吧,也把酒打开。”然后,他回过头,对曲陆炎说:“我知道,你心里肯定想过,到死也不再跟我说话。可现在大家都已经死了,所以,我们可以坐下来吃顿
饭了。” 

曲陆炎笑了:“没错,自从死了以后,我就不
恨你了。” 

主人摆摆手:“不ᨀ这些,恨不恨的,跟死活也没关系。我们今天不醉不归。你多久没好好喝酒了
?反正你现在用不着再担心肝脏。” 

“我倒是没那么馋。”沈老师笑道,“活着的时候整天偷着喝酒,现在想怎么喝就怎么喝,反倒没

\newpage
什么意思。” 

他在 1977 年的那个傍晚,最后一次看见若梅。若梅穿着一件很旧的白色衬衣,上面隐隐地撒着一些看不出色泽的碎花,深蓝色的布裤子——满大街的女孩都会这么穿,但是到了她这里就有了种袅娜。她在通往他们母校的街口徐徐地转过身,对他漫不经心地笑笑:“你是不是也去考了大学?”若梅的眼睛直视着他的脸,语气横冲直撞——那时他早已听说了若梅的病,人们早就在传的,病是生在脑袋里,说是心里,也对——总之,根治是不大可能的,跟她多说几句话就能发现她不对头,可惜了,一个那么美
的姑娘。已经是红颜了,估计也只好薄命。 

他依然把若梅当成了一个正常人。他告诉她,没错,参加了高考,并不是只有他一个人,好些人都参加了,那谁,那谁,还有谁谁谁,有谁去了北京,有谁考上了名校,又有谁意外地被分配到了某些在他们眼里非常浪漫的远处,而他自己,还行吧,接纳他的那所大学没那么显赫也没那么传奇,不过好歹是所有根基的老学校——聊的都是沈老师过去的学生,若梅全都认得的。他站在那个黄昏里跟若梅聊了足足半
\newpage
个小时,历数所有考上了大学,即将开始全新生活的故人们。他是故意的。曾经,沈若梅心比天高,没兴趣正眼瞧他们。他自认为也在注意自我克制,并没有在这个患了精神病的女孩子面前炫耀他们的锦绣前程——若梅安静地听,听完了,嫣然一笑:“真好呀,真好。”他略带错愕地望着她潋滟的笑容,心想她果然是脑子有问题了,居然如此心无杂念地替别人欢笑
着。 


就在那天晚上,若梅跳了楼。 

他跟沈老师碰了一杯,他说:“沈老师,我们不劝酒,大家随意。”沈老师沉默着也举起杯,在半空中停滞了一瞬,表情庄重,这一瞬也因此有了风骨。与沈老师的这一杯,他一饮而尽。他早就想好了,微醺之际,告诉沈老师有关那个黄昏的事情。为什么要告诉他呢?肯定不是道歉,并不是他的错,至少他不是存心的。他只是想稍微挫一下那个女孩的骄傲。因为她也曾经深深地挫败过他的傲气。她那么美,这对他本身就是伤害。一个人只有在喝多了的时候才能

\newpage
清晰地表达出这些。 


只是他不知道,死人是不会醉的。 

客人们还没告诉过他这件事。“活人”和“死人”之间的区别有很多,千杯不醉只是其中之一。其实也不用刻意说明,当死人当久了,自然都会知道的
。 

和曲陆炎碰杯的时候,他认真地思索了一下,要不要说一句,对不起。可是终究说不出口。曾经他说过的,他和林宛都说过一千次,不过这种事,即便曲陆炎当真说了“没关系,算了”,他们也承受不起。刚毕业的那些年,旧日的同学们一起同仇敌忾地孤立了他和林宛,他们二人也知趣地不和大家联络。可是多年过去,曲陆炎在同学圈子里始终销声匿迹,同学们跟他们逐渐恢复了走动,尤其是——当他们俩的孩子和同学们的孩子渐渐长大的时候,他们不知不觉有了太多共同的烦恼和困惑。于是后来,曲陆炎反倒成了大家眼中,那个不那么懂事的人。所谓人走茶凉
,说的大概就是这个。 

\newpage

沈老师装作对他和曲陆炎之间那些细微的尴尬浑然不觉,坐在那里细细端详着上来的六道凉菜。似乎是在从色泽品评着厨子的水准。沈老师一直都是个生活得细致的人。他似乎记得,某个火热的夏天里,校园里满墙的大字报,有一张是骂沈老师的,罪状是他家里的书架上,若干年前有一本撕了封面的,1949 年版的《雅舍小品》,作者是一个名叫梁实秋的反动文人。那里面有些写怎么吃东西的散文,被沈
老师翻得很旧。 

“沈老师,您不用客气,先尝两样小菜下酒。
”他招呼着。 

“那不用。”沈老师摇头,“我吃点蚕豆就行。别的菜,动了不好的。”随后沈老师解围似地说,“这家馆子水准好像还不错。比好多人间的馆子都强。不过想想也没错,有水准的厨子们就算是死了,不
做菜,也太闷了。” 

“你们这些年过得好不好?”他听见了曲陆炎

\newpage
的问题,语气平缓。 

“还行。就是孩子不争气。是个男孩子,淘气
得很。”他微笑。 


“我知道。”曲陆炎说。 

他怔了怔,不大明白曲陆炎知道他和林宛有个男孩,还是知道那孩子很不争气。不过他决定不追究这个了,他无奈地笑:“现在不同了,我一走,他就
得学会顶门立户。” 

“这个我懂。”曲陆炎挪动了一下身下的椅子,“我唯一安慰的其实也是——我看着我女儿嫁了人
,在澳洲安了家,她过得不错,我就放心了。” 


“你比我有运气。”他说的是真心话。 

最后两位客人终于来了。服务生把他们领进包
间的时候,看得出在压抑脸上的惊讶。 

那是一对夫妻。丈夫没有双臂,将用旧了的拐
\newpage
杖夹在腋窝下面,用一种看起来危险的平衡支撑自己行走,那是经年累月跟自己的残肢磨合出来的默契。他用一个夸张的角度,将额头远远地放在残臂上,乍一看以为他要攻击谁,其实只是略微擦擦脸上冒出的汗。身上的黑色薄棉衣旧得发亮,不过双臂处的确是被精心地改制过,像是真的从什么地方买得到一件双臂只有婴儿那么长的成人外套。不过这位丈夫脸上的笑尽管腼腆,却比他的妻子坦然。妻子倒是四肢健全,微胖,手指短而粗,半长的头发草草梳了个马尾,满脸惊诧,似乎不知道该把自己的身体放在哪儿,只好死死地抓着她男人的拐杖。抓得越紧,神情就越奇
怪。 

沈老师站起身来,把一把椅子拉开,招呼这丈夫坐下。曲陆炎冲着这对夫妻习惯性地伸出右手:“幸会。”他对着丈夫愣了一下,把手略略移开,明确地向着妻子,妻子的眼睛在曲陆炎那只悬空的手上扫了一下,就挪开了,维持着一脸呆若木鸡的表情,好像因为自己的男人没有手,所以长在别的男人身上的手都不大吉利。丈夫却礼貌地对着曲陆炎点头:“她

\newpage
脑子有点慢,”丈夫周全地说,“不大好见人。” 


“他们是我的邻居。”主人解释道。 

“快坐着。”沈老师把菜单放在离他们近些的桌面上,一阵椅子在地板上拖泥带水的声响过去后,这夫妻二人好不容易坐定了。这时候妻子却不知道该把丈夫的拐杖怎么办,只好抱在怀里,像是抱着一个过于硕大的宠物。拐杖斜斜地横在她胸前,有很长的一部分像个路障那样,延伸出去一个小小的斜坡,直抵墙面。曲陆炎凝神望了她一眼,用一种前所未有的耐心,弯下身子对她解释:“拐杖交给我吧,我帮你放个舒服的地方。”——看得出,他也很讨厌这样说
话的自己。 

主人从曲陆炎手中接过拐杖,以合适的角度靠在丈夫的椅背,丈夫轻微挥动两只短小残臂的样子虽然滑稽,可是他非常认真的社交的神情却让所有人不由自主地将他当成是二人中的领导者。丈夫的眼睛选中了沈老师,略略欠身的样子像卡通片里的什么人物:“她小的时候淘气掉到水里去,差点淹死,昏了好几天,醒来以后反应就不快了。不过也是认生,跟熟
\newpage

人,不是这样的。” 

“他们在我们小区门口摆水果摊。”主人淡淡
地说。 

“是。”丈夫补充道,“他一直都特照顾我们的生意。”说话间,左臂——准确说是左臂剩下的那一点点在他和主人之间的空气里划了一下,看上去像
是抽搐,实际是在表示“我们”。 

两年前的夏夜,因为天气热,他们收摊也晚。他的儿子喝完大学的毕业酒回来,那辆新买的车就像它的主人——那不知轻重的小王八蛋一样,直直地对着水果摊撞了过去。双臂残疾的摊主当场毙命。那没出息的孩子吓得六神无主,拿起电话打给林宛,深夜的电话机里传出的先是语无伦次的说话声,跟着就被
他自己的嚎啕大哭打断了。“妈,我怎么办……” 

又能怎么办。当他和林宛准备好了把半生积蓄全赔进去换他的自由身的时候,却知道了残疾摊主的智障妻子,得到噩耗的当晚,静静地一个人走进了小
\newpage
区花园的湖泊。她终究还是死在了水里。他们夫妻没有孩子,乡下来的亲戚们拿了赔偿金,懒得再去打官司。这对残缺辛苦的夫妻至死都不知道谁是肇事者。丈夫根本没来得及看清楚,妻子没有弄明白整件事的能力——她不识数字,水果摊的账一直都是男人在算的,她唯一的工作就是把一颗颗水果放进秤里,直到丈夫说:“可以了。”然后在把这些“可以了”的水果倒进塑料袋,但她总会对顾客笑一下,那是她唯一不需要她男人来指导,就能做好的事情。她珍惜这个


这边,对他们,也许是个好地方。 

“今天来的,都是我的老朋友。”主人佩服自己,能如此真诚地看着那对夫妻说出这句话,“我们就是——好不容易聚起来了,一定要见个面吃一顿才
行。” 

“吃饭。”那女人突然明白了过来,然后开始掏自己的口袋,“吃饭前得吃药。”她看着主人,曲陆炎,以及沈老师的脸,看了一圈,用力地说:“他

\newpage
血压高。得吃药。” 


“现在不用吃了吧?”曲陆炎怀疑地问。 

丈夫打断了他:“反正她兜里带着我的那瓶药,我就一直吃着,吃完了算。她不知道我们俩都死了
,得慢慢跟她说。” 

“无所谓说不说。”曲陆炎道,“她只要能看见你,在这边还是在那边,估计也都没什么分别。”

女人把药瓶拧开,糖衣药片是一种像交通灯一样的绿色。她不小心倒了一大捧在手心里。她丈夫在旁边拖长了声音,有一点想叹气的意思:“两片,两片就行了,不能这么多。”女人的手指对于那些药片来说可能过分粗大了,她只好用右手的食指点着左手的手心,那只紧张的右手好像随时准备戳到什么人的额头上去骂人。一不小心,还是将三四粒划了出来,她丈夫耐心地重复着:“两片,教过你,再想想……”她努力地想,微颤的食指在那一小撮药片上犹豫不决,鼻翼间的呼吸差点把一片势单力薄的药片吹掉了。男人的残肢又像是在抽搐,其实是在指挥她:“两
\newpage

片,对了,马上就对了——” 

窗外天色越来越暗了。主人有些不顾礼节地给自己倒满了一杯,一饮而尽。他的一生亏欠的人,不止这几位,可是剩下的那些,都还活着。于是他就觉得那些歉意的确都不能算数了。他在想,怎么还不醉呢?脸上就连一点热度都感觉不到。他像是掩饰什么
,放下杯子,对沈老师一笑:“天太冷了。” 

沈老师配合他:“是。晚来天欲雪,能饮一杯
无。” 

真的有一些白点开始在窗玻璃上蜻蜓点水。神许他们的世界,下雪了。

\end{document}
