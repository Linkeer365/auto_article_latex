\documentclass{article}
\usepackage[utf8]{inputenc}
\usepackage{ctex}

\title{第一次玩海龟汤就上头}
\author{吐维}
\date{2010-12-01}

% \setCJKmainfont[BoldFont = Noto Sans CJK SC]{Noto Serif CJK SC}
% \setCJKsansfont{Noto Sans CJK SC}
% \setCJKfamilyfont{zhsong}{Noto Serif CJK SC}
% \setCJKfamilyfont{zhhei}{Noto Sans CJK SC}
% \setlength\parindent{0pt}

\begin{document}
\CJKfamily{zhkai}

\maketitle


\Large

大家玩过海龟汤吗? 


第一次接触到海龟汤(TurtleSoup)这种游戏,是我的T女友A教给我的。海龟汤其实是一种叫做「水平思考拼图」的游戏,光是这样讲,各位一定还是一头雾水,就举A第一次给我玩的汤为例好了。 

她给我玩的是最普遍的那个汤,叫「湖中无水草」。 

汤面是「一名男子在湖边看到一个上面写着『湖中无水草』的告示牌,回家之后就自杀了,请问为什么。」而想知道故事背后的原因,也就是俗称的「汤底」,就要经由不断地询问出题者,才能找到真正的答案。 

但说是询问,问题的型态却是受限制的,参与
\newpage
者只能询问出题者非黑即白的问题,也就是所谓的「Yes/No Question」。 

你不能问出题者:「湖边除了告示牌,还有什么其他的东西?」而只能问出题者:「湖边还有其他的东西吗?」 

两个问题听起来类似,其实天差地远,前者出题者可以告诉你:「湖边除了告示牌,还有A、B、C、D……」 

但后者他只能回答你:「湖边有其他东西。」至于是什么东西,你还得自己想办法问出来。 

这样的游戏极度考验想像力,而事实证明我是个极度缺乏想像力的人。第一次和女友玩的时候足足花了三十分钟,问到A子都笑着引导起我来。 

后来我知道了汤底后,才知道自己拐了多大一个弯。这碗汤的汤底是「因为当年男子的女友在湖底溺水,男子跳下湖去救,摸到了女友的头发,但却误认那些头发是水草,因此没有把女友救上来,导致女友淹死在湖底,因而发现真相后羞愧自杀。」 

但我在玩的时候,却觉得他一定是在湖边发生了什么让他绝望的事。我什至猜想湖中无水草会不会是一个谜语,比如男子的情人在信上写了什么「见水
\newpage
草如见我」,「我们在没有水草的地方相会吧!」之类的遗言,所以男子才会自杀。 

这种游戏就是这样,一但方向偏离,问的问题就会越来越远。 

A说有些海龟汤,在Yes/No之外,容许「这个问题不是重点(Not important)」这种答案,如此可以阻止参与者将答案越带越远。 

但正统的海龟汤其实是不能这样答的,因为问题是否重要,必须凭借参与者自己的思考与推理,而不是由出题者告诉我们。 

「这根本就强人所难嘛!」 

我向A抗议,我和她玩到最后,往往都变成A对我循循善诱,半暗示半回答地将我引导向最后的答案。 

A是一个很聪明的人,我从开始和她交往就这么觉得。她平常很喜欢看一些推理小说,还有日剧里的刑侦推理,有时看得入迷,还会用笔记下故事里的细节,画成表格之类的东西,然后煞有其事地分析。 

像我就完全无法理解她的行为,我也不喜欢看
\newpage
推理剧,只要她不在的时候,我总是把电视转到韩剧,动脑这类事情完全不适合我。 

但大概是所谓近朱者斥,和A玩这种小游戏玩久了,我也渐渐有一点喜欢起这类动脑的游戏,除了海龟汤,有时也会陪A一起玩玩乙案侦查之类的活动
有一次,A说要带我去见她的新室友。 

她最近搬了新家,搬进一间ShareHouse,里头因为某种原因,除了她那个新室友外,其他人都搬走了。 

初次见面,我就吓了一跳,不单是A和男人同租一间屋子这件事,这个男人非常奇怪,他有着一张帅脸,但头发像是很多年没洗一样,乱糟糟地散在脸颊旁。明明是第一次和女生见面,竟然只穿了件榇衫,以一个婆的眼光看来,真的是浪费了他那张脸。 

A介绍他叫作Q,我们聊了一下天,叫了披萨和可乐当晚餐,在等披萨来的时间中,A就说想玩海龟汤。 

「海龟汤?」颓废的男人不解地问。 

「就是TurtleSoup啊,你有在逛PTT吧,我之前明明有叫你去看。」A说。 

「喔,那个啊,我只看了一下规则。」Q先生
\newpage
懒洋洋地说。 

「那个东西你一定会有兴趣,那可是结合了推理和想像力的游戏。」 

「推理本来就需要想像力。」 

「总而言之现在刚好有三个人,你就和我女友比个赛怎么样?」 

A兴冲冲地问,Q先生好像没有反对的意思,稍微点了一下头。 

我觉得紧张起来。「请、请多指教。」 

A先出了一个简单的题目,那好像是经典题库的题目之一,但我还没有玩过。 

汤面是「有个男人头下脚上地倒插在沙漠里,手上拿着一根烧尽的火柴棒。」A要我和Q一人轮番问一个问题,直到猜出汤底。 

基于女士优先,Q先生让我先问。 

「那个男人死了吗?」我问。 

「Yes。」A很快地答。 

轮到Q先生了,我看见他稍微直起身来,看了A一眼。 

「他周围没有其他东西?」 

「Yes。」A答。 

\newpage

我不太明白Q为什么要问这个问题,而且还是用否定问句,好像一开始就有定见他周围没有其他东西似的。 

「他是从顶楼掉下来吗?」我问。 

A笑了一声。「No。」 

我「喔」了一声,有些失望。 

「火柴棒不是用来做它本来的用途吗?」Q先生问。 

A迟疑了一下。「Yes。」 

我准备要问下一问题,Q先生却忽然揉了一下太阳穴,说: 

「我知道汤底了。 」 

他在我惊讶的目光下打了个喝欠。 

「男人和朋友坐热汽球横越沙漠,因为热汽球重量太重,所以他们抽签决定谁要被扔下去减轻重量,他们划燃了一根火柴,在把烧过的火柴夹在其他没烧过的火柴里,谁抽到那根烧过的火柴,谁就得牺牲自己被扔下去。后面的事情应该不用我解释了。」 

我瞪大了眼睛,A无奈地摊了摊手,「Yes,完全正确。」 

A看起来似乎很不甘心,嘟着嘴说:「我知道
\newpage
这种题目难不倒你,但你也应该顾全一下我女友的情面,干嘛这么快就讲答案出来,人家是女孩子耶。」
「妳也是女孩子啊,你平常怎么就不要我顾情面。」 

「我是我,她是她,不能混为一谈。」 

我听着他们吵嘴,不由得也觉得有趣。A平常是个非常酷的人,要我来说的话,简直像那些冷硬派侦探故事里,抛家弃子的那种男人。 

我和她从求学时代交往至今,知道她过去有很多不好的经验,因而变得性格上有些孤僻。除了我这个伴侣,我也很少见A有什么朋友,更别说像这样和另一个男性打闹。 

这让我很担心,我认为A除了女朋友之外,是该有几个能够听他谈心的朋友才对。因为即使是最亲密的人,也有很多事情是无法分享的,有些话题还是跟朋友聊比较好。 

「等一下,为什么你会这么快知道汤底,你玩过这题目吗?」我忍不住问Q。 

「我没有玩过,但是这个题目并不难猜,因为他出得很好。」 

「出得很好?」A问。 

\newpage

「我看过海龟汤的范例和规则,也看过一些实际玩起来的状况。这游戏的秘诀,说穿了就是两个。」 

我看Q直起了身,感觉他和刚才颓废的样子判若两人,整个人精神起来。 

「一个就是选择问题的能力,要问什么样的问题才能快速接近答案?例如谜面里一个男人死了,你不能散焦地问『他是被刀杀死的吗?』、『被车撞死的吗?』,而应该先问『有人杀死他的吗?』如果答案是No,再进一步去问『事故而死的吗?』、『是自杀的吗?』,就像数学的圆一样,从大而小,从远而近,这部分是逻辑推衍最基本的功夫。」 

Q屈起一根手指,又说, 

「其次是组织资讯的能力,这游戏因为全是Yes&No Question,所以很容易问到后面,就忘记前面问过什么问题。」 

Q说的一点都没错,我就经常犯这样的毛病。
「例如前面『事故死』已经被否决了,但参与者却在游戏后期又问出『难道是掉下去摔死的?』,这种事情在游戏中很常见。脑袋里对于问题的顺序、关连,在游戏期间必须有张清楚的蓝图,否则你就会
\newpage
一直重覆类似甚至相同的疑问。」 

「比如你一天到晚问我有没有看见你的袜子。」A插口。 

「这是人的记忆对于不重要资讯取舍的问题,和组织能力无关。总而言之,海龟汤要玩得快狠准,至少要具备上述两个基本功夫,不过这些都只是表面的基础而已。」 

「表面的基础?」我忍不住问。 

「没错,这两点基础,可以拿还应付所有的海龟汤谜面,但有的时候,根据题目的不同,有的时候还是可以耍一些小技巧。」 

「作弊吗?」 

「说作弊也不全是作弊,只是一点点小小的推理。首先,这种游戏最有趣的地方就在于,越好的题目,往往越容易被猜中。」 

Q望着A。「就用刚才这个题目当例子好了,一开始出题者给了我们那些资讯?」 

「资讯?是说男子头下脚上插在沙漠里吗?」我问。 

「不只是这样,在玩海龟汤时,要注意出题者offer的关键字,也就是所谓的key wor
\newpage
d,例如在这个题目里,你第一次看到这个谜面时,会注意到哪些特别的词语?」 

「啊,你是说像这样:头下脚上、沙漠、燃尽的火柴棒,像这样?」A反应很快。 

「对,首先一定会注意到的就是头下脚上了。好的海龟汤谜面,通常就是起于一个超乎常识的点,这个题目里最超乎常识的点就在这里,因为一般正常的人不会头下脚上,看到这样的人,你首先会想到什么?」 

「他绝对不是自己把头埋进沙漠里的。」 

「这很难说,海龟汤的题目千奇百怪,搞不好他是探头看他的温泉蛋煮好了没也说不一定。」 

Q先生笑笑,他又继续说。 

「不过的确这点给了我们一点暗示,暗示这个题目里可能有其他的外力,才会让一个人头下脚上地插在沙漠里。再来,火柴棒也是一个超乎常识的东西,因为一般人死掉的时候手里并不会握着火柴棒。」
「但是就算知道火柴棒也古怪,一般也会想到跟火有关的事情吧?」我问。 

「的确如此,事实上这个谜面并非和火完全无关,我说过了,他是一个很好的题目,也因此他的关
\newpage
键字和故事之间必定有合理的连结性,我首先会想的不是他拿火柴棒点燃什么东西,而是他从哪里拿到火柴棒,以及为什么是火柴棒。」 

「啊,你是说热气球……」我恍然。 

「对,出题者在想要用什么东西当签时,他心里已经知道男人坐的是热气球了,事实上他可以选择用午餐的免洗筷子之类的做签,或是美乃滋的瓶盖之类的,但他最后却选择了火柴棒,就是因为他已经受到既定谜底的影响。」 

Q先生又把背靠回沙发上,喝了一口我放在桌子上的茶。 

「要知道出题者是知道谜底的,知道谜底的人,无论他再怎么刻意隐瞒,都没有办法在抹消他脑里谜底的情况下出题。某些方面,这和测谎有点像,你们知道测谎吗?」 

Q先生似乎来了聊兴,身子从沙发上直起来。我和A对看一眼,A好像已经习惯她的室友这样,就摊了一下手。 

「就是那种把人绑在可以反映心跳的机器上,问他一些问题,然后测验他说谎时心跳会不会加速的东西吗?」A说。 

\newpage

「对,但也不完全对,很少人知道测谎的真正流程。首先是测谎的题目,出题的人会先准备很多受测者一定会回答的问题,例如你叫什么名字、你今年几岁等等,应该说测谎的题目中,百分之八十以上都会是受测者一定知道的问题。」 

「那还需要测谎吗?」我好奇地问。 

「这就是出题者的诡计啊,真正需要测谎的题目,往往夹杂在那些理所当然的问题里,例如问过你住哪里、高中念什么地方,当你反覆地面对这种不需要思考、理所当然就能回答出来的提问时,脑子里戒心就会自然而然降低。」 

「这时候再问『人是不是你杀的?』吗?」A说。 

Q先生笑了起来。「大致是这样,当你乍然听到需要思考的问题,就好像好梦正酣的人被惊醒一样,一度迟钝的脑袋要恢复敏锐的思考是需要时间的,那时候说谎的反应就会远比平常准备好的时候要大。
「好诈喔。」我忍不住感慨。 

「测谎的艺术还不仅于此,当遇到需要测谎的关键问题,打个比方好了,凶手杀完人之后就把凶器藏起来,但出题者却不知道凶手把他藏在哪里。」 

\newpage

「嗯。」 

「这时候出题者就会这样问:『凶器藏在什么地方呢?』,但他不会要受测者直接回答,而是要他用选的,出题者会替他把所有可能的答案拟好。他会用这种问法:『凶器藏在你家里吗?』。」 

「这时候受测者当然回答『No』,然后出题者再紧接者问:『凶器埋在花园里里吗?』受测者再答『No』,出题者再问『凶器被快递寄出去了吗?』、『凶器交给朋友了吗?』、『凶器丢到马桶里了吗?』或是『凶器烧掉了吗?』……」 

「为什么要这样问啊?」我忍不住举手。 

「因为大部分去接受测谎的人,其实心底都有拟一个虚假的答案。」 

Q慢慢地说:「比如出题者问:『你和被害者最后一次见面是什么时候呢?』,他就会回答『是去年的秋天啊。』要知道虽然大部分人说谎的时候,生理都会出现反应,但是这种生理反应,是可以经由训练来消除的。」 

Q笑了笑。「比如A如果一直对着镜子说:我是美女,我是正妹,总有一天她就能够脸不红气不喘地说出这句话,这个道理是一样的。」 

\newpage

「所以说,这样可以防止受测者事先演练。」
A听得专心,她竟然没有生气,我却不禁在旁边偷笑起来。 

「没错,而且要是出题者的选项里出现实际为真的答案,也就是本来受测者应该要答『Yes』,却不得不说谎答『No』的选项时,通常就会出现动摇,再配合前面说的那种穿插法,受测者很难不起反应,这样出题者就会知道,这题的答案其实是Yes。」 

「但是也有可能出现根本没有正确答案的情况不是吗?」A提出质疑。 

「没错,但是通常这很少见,就算不是完全正确,但因为受测者心里已经有一个答案了,所以只要你提到类似的事情,他就会不由自主地往那方面想。
这时候门铃响了,原来是送披萨的来了,A主动站起来去开了门,拿了两个特大号披萨进来,而Q先生仍旧在沙发上一动也不动。 

这样的状况也让我觉得新鲜,因为我和A在一起的时候,通常都是我在服务A的,她某些方面是个有点大男人主义的家伙,但搞不好我就是喜欢她这一点。 

\newpage

A把披萨放到桌上,我和Q就一人拿了一片,边吃边继续聊起来。 

「人的联想力是非常可怕的,就算你问『凶器埋在花园里吗?』,但实际凶器埋在后山上,但受测者一听见『埋』这个字,还是会马上起生理反应。」
Q一面吃着手里的夏威夷海鲜披萨,一边说。
「等到确认『凶器在花园里』是Yes时,出题者就会再进一步问:『那么凶器是什么呢?』,『是刀子吗?』、『是碎玻璃吗?』、『是剪刀吗?』,如果受测者对『剪刀』出现Yes的反应,就再进一步问:『那把剪刀是谁的呢?』,『是被害人自己的吗?』、『是你带去的吗?』、『是别人给你的吗?』就这样一步步逼近真相。」 

「简直就好像海龟汤一样嘛!」我忍不住叫了出来。 

「对,我一开始看到海龟汤这个游戏,就觉得他非常像测谎。」 

Q咯咯笑了起来,伸手拿可乐灌了一口。「只是和测谎不同的是,海龟汤的出题者是被设定为绝对诚实的,也就是他的肯定否定必须要是与谜面相符的
Q举起一根手指。 

\newpage

「但说是这样说,海龟汤还是会出现和测谎相同的盲点,那就是我最开始说的,『出题者已经知道谜底』这件事。」 

「我不懂。」我摇了摇头。 

「这很简单,你观察刚刚A在说谜面时的状况,首先她强调了『火柴棒』这三个字,你还记得吗?
老实说我不太记得,大概是我太专心在听谜面的关系,所以没注意到A子语气有什么古怪。 

「我那时就想,火柴棒一定是这个谜面的关键,非但是关键,以她喜欢整我的个性,出题的时候一定尽量以误导我为乐,所以我一听到她强调这个字,就知道这个火柴棒肯定有鬼,至少不会是火柴棒一般的联想。」 

Q笑着说:「所以我打从一开始,就放弃去想谜面里的主角,到底拿火柴去烧什么东西了。」 

「真对不起,我就是喜欢整你啊,萌绘。」 

「哪里,我习惯了犀川老师。再来就是当你问A第二个问题的时候。」 

Q忽然转向我,我吓了一跳。 

「咦,我吗?」 

「你不是问她:『是从顶楼摔下来的吗?』那
\newpage
时候A笑了一声。在海龟汤的游戏里,出题者如果出现笑容,特别是A这种心胸狭窄的出题者,那绝对不会是因为你的问题接近答案了。」 

「而是因为我猜得太离谱了是吗?唔,平常都是这样……」 

我恍然大悟,和A迷上海龟汤也有一段时间了,每次我只要越猜越远,A就会对着我笑个不停,脸上露出又是得意又是怜惜的表情。这种时候我就知道自己问了奇怪的问题,但A这个坏胚子,每次都以看我的窘境为乐。 

不过我也有点佩服,照理说和A相处比较久的应该是我才对,但Q却很快地注意到这一点了。 

「你这是犯规吧,哪有这样设计人的。」A不满地说。 

「没有犯规啊,海龟汤是一种互动游戏,最初就是一群无所事事的人,在公众场所里想出的把戏,只要是互动游戏,就一定无法摆脱『人』这个因素,脱离人的解谜游戏只是单纯的理论,也会变得不好玩喔!」 

Q的话激起我些许思绪,事实上我也听A说过一点海龟汤的缘起,虽然也是属于往路传言的层次就
\newpage
是了。 

听说第一碗汤就是真实故事改编的。一个从战场上退役的军人后喝了海龟汤泪流满面,餐厅里的人关心他,问他怎么了。 

那个军人就当场出题,反问大家他为什么会哭,但是条件是他只能答点头或摇头,因为军人被限制不能泄露军中的机密。 

餐厅里的人纷纷提问,后来谜底是军人的父亲和他一起上战场,但因为被困在营地里,弹尽粮绝。军人的父亲于是就杀死自己,央求好友把他的肉煮成汤,让伙伴和儿子分食,再骗儿子那碗汤是海龟汤。
儿子当初不疑有他地喝了,多年以后尝到真正的海龟汤,才惊觉原来当年喝的根本不是这种汤,细思之下马上明白了残酷的真相,当场在餐厅里痛哭失声。 

后来这个游戏在当地流传了下来,为了纪念这位大兵,就沿用这个故事的典,称这种游戏叫「海龟汤」。 

姑且不论这个故事是不是真的,但我觉得这样听过之后,原先单纯的解谜游戏,确实如Q先生所说的,多了许多人的气息,这或许也是我甘心被A耍着
\newpage
玩的原因之一。 

「那时候我听见A笑,就知道『顶楼』这个答案一定离汤底很远,」 

Q继续说下去。 

「加上知道主角是摔死的,其实人可以摔死的地方并不多,要嘛就固定的建筑物,要嘛就是会动的交通工具,既然A的反应告诉我建筑物相去甚远,那我几乎就可以确定,他是从什么交通工具上摔下来的了。」 

「知道是交通工具后,之前火柴棒的提示就派上用场了,我把所有飞空的交通工具想了一轮,飞机、直升机、魔毯、滑翔翼……这里面唯一和火柴棒有关的东西,就只有热汽球了。再接下来,只要运用一点想像力,答案几乎就呼之欲出了。」 

「哈啊……」我长长吐了口气,还在头晕脑胀中。 

「什么嘛,原来是靠老娘我的反应,这根本就是作弊不是吗?而且要是在网路上玩怎么办?」A还是很不满。 

「网路上玩也是一样的,出题者经常会多答一些是或否以外的闲聊,从这些闲聊获得的讯息,往往
\newpage
都快比从答案本身得到的要多。」 

Q大方地说,「我倒是觉得,要是真的像电脑程式一样,遇见问题,只机械性地答Yes/No,那就不是游戏,是在做方程式解题了。」 

「解谜游戏就是因为有各种变数,所以才有趣,这就跟看推理小说一样,就算是因为推理小说的套路,『这个家伙是个美女,还一出场就跟侦探有暧昧,肯定是凶手。』用这种方式找到真凶,也不失为一种解谜的乐趣。」 

Q笑得天真无邪,我觉得他的印象和初次见面时似乎有些改了。我本来以为他是个懒洋洋,对凡事不感兴趣的阿宅。 

但现在我竟觉得他有点像小孩子,单纯得可爱。只是有时候有点脱离现实就是了。 

「唉,解谜还是推理什么的,实在太复杂了,我一辈子也弄不懂。」 

我叹了口气,其实跟A玩这些解谜游戏的时候,我就有这种感觉。A总是能想到我所想不到的方向去,智慧的差距,在这类游戏中最容易残酷地被体现
而现在我知道除了A以外,这世上还有比A更聪明的人。这让我不禁觉得,自己是不是一点也不适
\newpage
合做这种动脑的活动,做了只是自取其辱而已。 

Q安静了一会儿,忽然说:「那我也来出个题目好了。」 

我和A都惊讶地看着他,他就指着玄关问:「我的谜面很简单,你们知道这扇门为什么要向外开吗?」 

「向外开?门不都是向外开的吗?」我看着ShareHouse往内敞开的房门。 

Q却摇了摇头,他边比划边解说着。 

「不是这样的,在台湾或许比较不明显,但在欧美国家,门一定都是向内开的,相反的,如果你去看日本的人家,他们的们则一定都是向外开。台湾的话,因为本来就是个文化混杂的地方,所以才会有的门是向内,有的门是向外开。」 

「是这样吗……?」 

我有点讶异,明明是每天面对的门,但我却从未注意这一类的事情。 

「这是海龟汤吗?」A问。 

Q笑笑。「是海龟汤啊,我出的海龟汤。你们可以用海龟汤的形式问我。」 

「是因为气候不同的关系吗?」我马上问。
\newpage

 
「No,跟气候无关。」Q一本正经地答。 

「是因为建材的关系吗?铁门或木门之类的。
「No。」 

「还是因为日照?」 

「就说跟气候无关了,妳也要问我袜子放哪了吗?」Q笑说。 

连续三个问题答案都是「No」,原本有点轻视这问题的A,也发觉没有想像中容易,低头沉思起来。 

「跟日本或是欧美的习俗有关吗?」我开口问
这回Q竟点了头。「Yes,可以这么说。」
我精神一振,仔细想了一下,欧美的习俗和亚洲的习俗有什么不同,但脑子里却一片空白,感觉只想得到汉堡和寿司的差异而已。 

「因为欧美人比较开放,所以向内开表示我家欢迎客人,日本人比较内向,所以向外开表示对外人有戒心,是因为这样吗?」A在一旁问。 

「照你这样说,反过来解释也是可以啊,欧美人开放,所以对外开表示张开双臂欢迎客人,而日本人内向,所以向内开表示保护家人,这样也可以说得
\newpage
通。」 

Q笑了笑,A像吞了只青蛙一样,哑口无言的样子,让我和Q都笑了。 

但A不甘示弱,她立刻接着问:「是因为门的形式不同?日本人过去都用纸门不是吗?所以会习惯往内拉。」 

「纸门也可以往外推开啊,我认为并没有差别。」Q摇了摇头。 

「还是因为日本人手短,所以只能用推的,往内拉门会不舒服,欧美人手长,所以有余裕可以向内开门。」 

「这话有种族歧视啊这位太太,亏你自己还是亚洲人。」Q笑着对A说。 

我安静地想了一下,忽然灵光一闪。 

「难道说……是开门的空间问题?」我叫了出
「Yes。」Q回答,给了我一个鼓励的眼神
我精神大振,继续说:「是因为这样吗?日本人在室内放了榻榻米,所以没有空间往内开门,因此习惯向外开?」 

「接近了,但不对。」Q先生摇了手指。 

「因为他们在玄关放了佛坛,所以不能往内开
\newpage
门?」A马上接着说。 

「哪个日本人会在玄关放佛坛啊?」 

「有啊,我家以前就是这样。」 

「台湾人不算,台湾人的玄关可以放任何东西
   「是因为某种经常放在玄关里的东西吗?」我问。 

「是的。」 

「是雨伞?唔,还是高尔夫球杆?」 

Q又笑起来。「都不是。」我忽然福至心灵,从椅子上站了起来, 

「……是鞋子?」 

我感觉脑子里有条线贯通起来,好像堵塞许久的马桶忽然畅通一样。 

「啊……日本人的家里,或是一些亚洲人的家里,多半是要脱鞋子才能入内的,所以鞋子大部分会放在玄关,所以如果向内开门的话,会打到鞋子!是这样吗?是这样吗?」我忍不住冲着Q大叫起来,Q坐在沙发柄上点了点头。 

「就是这样,而且亚洲人的玄关通常比较小,且和家里地板间有段差,客人只能在相当狭小的地方脱鞋子,要是门向里开的话,客人就必须一边闪门,
\newpage
一边艰难地把鞋子脱下来,所以相当不便。那么欧美向外开的理由呢?」 

「因为他们不需要脱鞋子就可以进屋子不是吗?」A接口。 

「可是这样的话,还是可以向里面开啊,不脱鞋子的话,照理向外向内都没差。」 

我和A都愣了一下,确实有道理。A「唔」了一声。 

「果然是天气关系吗?欧美比较冷,有时会下雪,如果向外面开的话,雪会堆积在门边,清扫不便
A边说还边从沙发上站起来,模拟开门的状况
「你看,像这样往里开的话,雪从旁边吹过来时,就不会被门挡住,堆在门口,增加铲雪的困难。
「这样的确有可能,但是东京也下雪啊,不用说东京,北海道的门也多是向外开门,你想说北海道不会下雪吗?」Q笑着说。 

A一副被难倒的样子,我却忽然想起来。 

「和欧美人习惯穿鞋子进屋这点有关吗?」 

「是的。」Q赞许地看了我一眼。 

「穿鞋子进屋的话……这和欧美人没有高差的玄关也有关对吗?」 

\newpage

「是的,看来妳的小公主已经想到了啊。」Q笑着对A挑衅。 

「把门往内开的理由,也是怕会打到什么东西对吗?」 

「Yes。」 

「……我知道了,是脚踏垫。」 

我交握着双手,感觉自己心口有什么东西点燃起来。 

「因为穿鞋子进屋,容易把房子里面弄脏,所以欧美人习惯在门外放一块脚踏垫,让客人可以撮掉鞋子上的脏污……啊啊,原来是这样!如果门向外开的话,就会一直打到脚踏垫,很不方便,因此欧美人才选择把门做成往里开。」 

「Exactly,你喝到汤底了,恭喜你。
我看见Q笑得无比温柔。我还沉浸在喝得汤底的余韵中,老实说过去所有的海龟汤,对我来说都太难了,几乎都是在A不断提醒下,我才找到谜底。那时候我脑子早就已经一团混乱,不要说喜悦,往往只有种松了口气的虚脱感。 

但这是我第一次,享受到靠着自己的力量,把什么东西解开后的快感。 

\newpage

「很有趣,对吗?」我发现Q先生对我眨了眨眼,我忍不住红着脸猛点头。 

「这哪算什么海龟汤啊。」A似乎还有所不满
Q就搓着手说,「没人说这样不可以是海龟汤啊!其实还有很多呢,像是你们知道插头的两个铁片上,为什么要有两个孔吗?」 

我愣了一下,还来不及说话,Q闭上眼睛又说
「还有像是为什么斑马线是横的而非直的,为什么警车下部总是黑色,却又不全部涂成黑的。为什么烤丸子总是三个一串、布丁总是三个一盒,为什么信封的折口总是要多削两个角,而不干脆保持完整的长方形……」 

「这些全都是有原因的,而且他们全都发生在我们面前,光是开门的事情,我们每天都看着门在我们面前开开关关,但却很少思考他们为什么会如此。
Q张开眼睛,我看见他的双目闪闪发亮。 

「而我认为去思考这些事情背后的原因,就是推理最初的本质和源头,解谜绝对不是聪明人的专利,也不需要特殊的学历还是智商,那应该是每一个活在世界上的人,都应该勇敢去做,也乐于去做的一件事。」 

\newpage

「那也得要像你这么闲才行啊。」 

A还是忍不住吐嘈,Q也不反驳,只是不动声色地拿走最后一片披萨。 

「发现谜题的存在,需要的是日常生活的观察力。而去推敲谜题可能的解答,需要的是人与生俱来的想像力。在解谜的过程中,即使不断地失败,仍然想要追求真相、不肯放弃的那种情绪,则是人永远不该忘记的童心。」 

我还记得那时候Q的声音,变得完全不符他外形地深邃、温柔。 

「观察力、想像力还有童心,只要有这三项,就足以解开世界上所有的谜了。」 

我听着Q的话,想起刚才灵光一闪,推测到鞋子瞬间的那种喜悦之情。虽然只是个小学生程度的谜,但说真的,经由思考,靠自己找出答案的感觉,真的很棒。 

和A交往日久,我在她的影响下,也看了不少推理小说和推理剧,有些推理小说确实很有趣。 

但有时候有些故事太过复杂,经常一个案子死六、七个人,光是记起人名就耗尽我的脑浆,我又不像A这么勤劳,还会拿笔把人物和地图都记下来。有
\newpage
时候作者甚至还附表格给我,什么火车时刻表,还是建筑物平面图之类的。 

看A解的津津有味,但对我还有我一些不常动脑的朋友来说,常让我觉得很累。比起累更大的是挫败感,那种输给作者、输给其他读者的自卑之情。 

像我就经常听到一些不看推理小说的朋友说:「推理小说?那种东西太难了啦,我讨厌动脑。」或是自谦地笑说:「我脑袋不好,小时候数学都考不及格,要我做推理这种事,不如直接翻解答还比较快。
这让我想起小时候考试,总有一些小学老师以考倒学生为乐,他们会出一些上课没有教、课本上也没有写的问题,等到学生答错了,再狞笑着当掉他们
有时我会觉得很呐闷,考试的目的,不就是为了测验学生懂得多少吗?出一些根本不可能解开的问题,或百分之九十九以上的学生都解不开的问题,这种考试真的有意义吗?但很多老师依然乐此不疲。 

某些方面来讲,我觉得海龟汤真的是一种很好的游戏。他让人不会惧怕动脑,不会惧怕解谜,就像数独之于数学一样,他让推理变得更加平易近人。 

我想起Q临走前跟我说的:『经由观察发现问题,经由想像力找到可能的答案,抱持着童心找到最
\newpage
后的解答,这样所有海龟汤都难不倒妳。』我忽然有种放松的感觉,我想我应该放下推理小说,去研究一下信箱上的洞为什么总是椭圆形而不是方形的好了。
「总觉得……我有点迷上他了耶。」有一天我还故意跟A说。 

「迷上谁?那个死阿宅吗?」A不客气地批评,把我揽过来吻了一下。我笑嘻嘻地没有答话,任凭A在我背后叫嚣着吃醋也不予理会。 

A后来在那间ShareHouse安稳地住了下来,虽然他们经常斗嘴,但我想他们应该会成为很好的朋友。 

值得一提的是,后来我和Q先生又见了一次面,那已经是A搬进去三年后的事情了。 

那时A和我迷上了一种叫「杀手」的扑克牌游戏,那也是非常单纯有趣,藏着各种推理技巧的小游戏,总而言之就是指定一张牌的花色,再由大家抽牌,由抽到那张指定花色的人当杀手。 

然后主持人会叫大家闭上眼睛,拿到杀手花色的人则张开眼睛,告诉主持人他想杀死哪一个参与者
然后等大家张开眼睛,主持人就会公告刚刚是谁被杀手杀死了,然后请他推出杀手是参与者中的哪
\newpage
一个。 

游戏以杀手被人猜出是谁作结,如果一直没人猜出杀手是谁,那这场游戏就是杀手获胜。这是非常考验演技、人性还有对参与者理解程度的游戏。 

A对这种游戏非常擅长,只要她当杀手,最后我和朋友一定死光光。她不只是个聪明人,还是天生的骗子,所以我才会一生被她骗得死死的。 

「演技某些程度也是一种推理。」 

A还曾经得意洋洋地跟我这样说:「明明不是那个人,却要演成那个人的样子,明明不是真实发生的事,却要假装那件事在你眼前发生的样子。要做到这件事,观察力、想像力和童心,缺一不可啊,不单只是骗人而已。」 

看吧,A果然是个大骗子。 

A熟悉这个游戏后,脑子很快就动到Q先生身上,我知道她自从认识这位室友后,就一直以在推理上击败他为毕生志业。 

但是杀手这游戏要一定人数才玩得起来,Q先生又很别扭地不想和我的朋友一块玩,根据我从A口中得到的资讯,这个脑子灵活的男人其实很怕见生人
「那你就叫你的那一位带他的朋友来一起玩嘛
\newpage
!」 

我听见A对Q先生这样说,起居厅马上就传来Q窘迫的声音。 

「什……什么那一位?」 

「少装了,你跟对方在交往吧?瞒不了我的啦
「不是妳想像的那样。」 

「少来,我上次都亲眼看见他送你回家了。俗话说丑媳妇总是要见公婆,我们又不是陌生人了,迟早都要见上一面的嘛!」 

「就说不是妳想像的那样了……」 

我在玄关听着,感到有几分惊讶,不单是Q先生这样的人有了交往对象的缘故,虽然他极力否认,但像我这样的笨蛋也听得出来,Q先生只是单纯害羞而已。 

但我觉得除了害羞之外,他和三年前的样子也不一样了。怎么说,感觉更温暖、更有人性了一些。我想他说的是对的,人只要保持这一颗乐于解谜、乐于追根究柢的心,就会有动力不断地向前迈进。 

不过我想,这样的Q先生,玩起杀手来,一定敌不过我那奸诈狡猾的A。 

「我帮你打电话给他,你在那边给我乖乖待着
\newpage
吧,有栖。」 

「妳给我住手!还有谁是有栖了?」 

算了,就让A小小的复仇一下也无妨。 

毕竟现在的Q先生,一定有人会为他讨回公道,不是吗?

\end{document}
