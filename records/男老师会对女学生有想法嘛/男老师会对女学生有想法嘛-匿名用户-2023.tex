\documentclass{article}
\usepackage[utf8]{inputenc}
\usepackage{ctex}

\title{男老师会对女学生有想法嘛\footnote{Click to View:\url{https://web.archive.org/web/20230418050106/https://paste.ubuntu.com/p/wxH2HcKzbv/}}}
\author{匿名用户}
\date{2023-02-24}

% \setCJKmainfont[BoldFont = Noto Sans CJK SC]{Noto Serif CJK SC}
% \setCJKsansfont{Noto Sans CJK SC}
% \setCJKfamilyfont{zhsong}{Noto Serif CJK SC}
% \setCJKfamilyfont{zhhei}{Noto Sans CJK SC}
% \setlength\parindent{0pt}

\begin{document}
\CJKfamily{zhkai}

\maketitle


\Large

这问题刷到很多次了,一只不敢回答,上午
没课,试着写写吧。 

不是高中老师,我教的C++,在贵州一所二本院校的计科系任教,这个系的女生不多,喜欢我的课的女生就更少得可怜了,有十年了吧,两只手肯定
数得过来。 

那孩子算是个例外,是截止今日,唯一一个会
在我课上主动打断我,问问题的女生。 

她和我妻子很像,不是外貌,是整个人的感觉,无论是说话的口吻还是语气,以及肢体上的小动作,真的,非常非常像,有时候听她说话,我甚至都会

\newpage
愣神。 

她是15届的,软件工程专业。说来可能有人觉得夸张,2011年时,竟然还有人关电脑是直接
把显示屏关掉的。 


我问她,之前上学有没有上过微机课? 

她笑得很甜,回我,上过的老师,然后把我电
源拔了,还要补一句,这样就可以了吧? 

然而就是这么一个人,竟然可以在毕业答辩中
获得98分的高分,至今还没有学生可以超越。 


也可能是我这学校不咋地吧。 

我不是贵州人,是因为我妻子才来到贵州的,没想到的是,妻子从谈恋爱开始,劝了我多年我都咽不下去的凉拌折耳根,在第一次和那孩子吃饭时,就
一股脑的咽下去了一大把。 

她和妻子每一次劝说时的用词和表情都一样,
\newpage

眉头紧锁,咬牙切齿,真的真的非常好吃! 


后话了。 


 

第一次见到那孩子是在我的课上,2011年
,那年她19,我30。 

我在黑板上写下我的名字后,她在前排咯咯咯
咯笑出了声。 

我问她笑什么?她说想到了好笑的事情,我问
她什么好笑的事情,她回答不敢说。 


我哪管这个,义正词严,让她必须说。 

是我名字的谐音笑话,她说完我就后悔了,全
班笑塌。 

我其实还是比较开得起玩笑的,便让他们以后
\newpage
就叫我这个吧,她哈哈哈哈鬼笑一通,就差拍桌子了
。 

但其实我也是比较记仇的,心里想的是你看你
这个学期挂不挂科吧。 

我实际上教的是两门课,C和C++,倒不是学校师资不够哈,是我自己跟系上申请的,所以同届同专业我只带一个班,C是大一开课,C++大二才
会开。 

因为我的教学方式有些不同,我不喜欢按着书上的顺序去教,特别在我们这个学习浓度没有那么深的学校,如果开篇学生就没有兴趣了,那后期大概率来上课也是人在教室心在外,所以这两门课必须得我
一起上。 

我的方式是上来就先让他们看我玩20分钟的游戏,游戏是我做的,之前会定期更新年度排行榜上的日本动漫元素加入到游戏中,现在会再增加一些实

\newpage
事热门梗,大多是取材于B站。 

现在吸引学生兴趣比之前简单很多,就开局打个小怪,小怪被打叫一句你干嘛啊,学生就能开开心
心看你玩一天。 

等他们都看得入了迷,再告诉他们,这游戏是我做的,大家也能做,只要把这学期的课上完,就能做一个更棒的,下面我们就一步一步来,做一个属于自己独一无二的游戏。接着剩下的时间,就让学生们踊跃发言,阐述自己的创意,比如在游戏中加入什么
元素,增加什么玩法等等。 

自然是画大饼,但对99%的男生都能秒杀,之后的每一节课都不会落下,屡试不爽。但也有点对女生不负责,因为大部分女学生是不太感兴趣的,倒是尝试过很多办法,不过都不能完全吸引到她们,即
便她们人数很少。 

所以那孩子还是那个例外,才看了几分钟,她
就疯狂举手示意,表示她想亲自操作。 

\newpage

我没让,当时的Bug还很多,我怕她给我暴
露了。 

最后是发现我想多了,在她装模作样的假哭下
让她尝试了,第一个坑就没跳过去。 

其他学生看到后纷纷跃跃欲试,直接打乱了我
的教学节奏。 

不仅是教学节奏,人生节奏应该也是被她打乱
了,因为如果不是她,我可能早就离开了贵州。 


事实上,早在10年底,我就决定要回北京了,也征求了岳父岳母的意见,他们也理解我的决定。
 

系主任说让我带完11届毕业班的答辩,说我比较会引导学生,夸了我一堆好的,我心里其实清楚自己几斤几两,但脸皮薄经不得别人劝,就同意暂时

\newpage
留下了。 

我其实真不傻,主任的用意再明显不过,不然
不可能再给我安排11级的课程。 


他不是怕我真的离开,他是怕我真的离开。 

他知道,我如果离开学校,离开学生,没有精神上的寄托,根本不可能再支撑得下去,就像我两年
前第一次要辞职,他陪我喝了一个通宵那样。 

这大概就是贵州人的善良,他们总是以一种表面上几近道德绑架的方式,让你接受那种内心深处无
法抗拒的温暖,甚至,从不试图戳破你的伪装。 

然而,真正让我决定要继续再呆下去的一个契机,是某节课前,那孩子和她的室友在半路与我偶遇,闲谈中,大概从我的口音判断出我不是本地人,于
是询问我是哪里人,为什么会想要来她们贵州。 

我那天不知道哪根经不对劲,没有像以前那样嘻嘻哈哈乱扯一通,反而是直接说出了妻子想要回贵
\newpage
州的初心,为了振兴家乡建设,实现自我人生价值,
只是把家乡二字换成了贵州。 


有时候,说真话,其实更像说笑话。 

那孩子却没有和其它学生那样偷偷发笑,而是挤到我面前,竖起大拇子,瞪大了眼睛,用纯正的贵
州口音对我说到,老师,你好他妈牛逼喔! 

给我还整羞愧了,半天才挤出一句,女孩子家
家的,怎么说脏话? 


从她的眼神中,我相信她相信了。 

正如妻子所说,人生的意义,是你自己赋予的,它像一张空白的白纸,你用笔,在纸上画满了黑色
的线条。 

而当有第二个人,第三个人,或者更多的人,认可你的定义时,它将发生质的改变,你会发现,那

\newpage
画上的线条,变成了彩色。 


在她的眼睛里,我看到了那副画,是彩虹。 

我假借接电话让她们先走,因为当时我已明显
感觉到,自己已经泪眼模糊。 

我确实是陪妻子来贵州了,我支持她的想法,也尊重她的决定,并且也以实际行动配合她的目标,哪怕是进入一所非一流的普通院校,我也毫无怨言,只是直到那一刻,我才恍然意识到,即便我做了一切我力所能及的事情,我似乎也从来没有让妻子在我的
眼睛里,看到过那道彩虹。 

那种感觉无以言表,可能也没人能够理解,是
心酸,也是心酸。 

所以我决定要留下来,更像是一个不成文的约
定。 


事情总是这样,当你信心满满决定要大干一场
\newpage

的时候,总有人能在你面前竖起一道墙。 

那孩子最擅长做这样的事情,她甚至能直接立
起一座山。 

她课后问我的第一个问题是,老师,英语不好
会不会影响学习编程。 

这个问题,其实还真重来没被人问过,要说会吧,编程的关键字也就那么些个,好像也确实影响不大,要说不会吧,后期需要的一些参考文献和资料都是全英文的,虽然也有翻译版,但多少英语不好还是
有些影响。 

我本着德智体美劳全面发展,然后回答她,当
然。 

她回我,那糟糕了,她看见英文字母就头疼,她本来对编程是很有兴趣的,但还要英语好才能学,
那学不了了,她要跟学校申请换专业。 

\newpage

实话实说,在她问我这个问题之前,我真的已经把她当成心里的浪波万来培养了,这才起了个头,她就给我整这出,我自然不乐意,然后告诉她,也不
用很好,有个四级水平也是可以的。 

四级其实也是我自己加的,我们学校当时甚至都不强制要求,学生必须过四级才能毕业,就是,我打心里觉得她跟妻子很像,而妻子学生时代最擅长的便是英语,所以理所当然的觉得,她应该也要英语好
才行。 

她连连摇头,说死也不做不到。她高考英语才考了40几分,如果高考不考英语,她说不定能上一
个一流的一本大学。 

我问她为啥不喜欢学英语,她说就是讨厌,听
到英语就很烦,学不进去。 

很久很久之后,我才得知她痛恨英语的原因,源自与她初中的英语老师,当着全班同学的面,在她朗读完课本上的对话段落后,开玩笑说她发音像美国
\newpage

农民在放牛。 

她心气很高,本就是农村出生,被老师随口这么一说,同学私下叫了她两年美国农民,美国难民。她从此再不肯学英语,嘴里最多也就嘟囔几句那些鬼打架的英文句子,比如什么,来是抗母,去是够,坟
头烧纸,漏漏漏。 

还有些骂人的,都是些类似顺口溜之类的句子
,扯远了。 

我当时自然是不知道她这些原因,就说了一些大道理,总而言之就是不能偏科,且强调了英语的重
要性,都是捡些有利的因素来说。 

她哪里听得进去,甚至强调自己不学英语是因
为爱国。 


最后逼得我承认,编程,与英语好坏无关。 


\newpage

我对她产生浓厚兴趣的原因是一节上机课,我也正是那节课认定了她,一定能成为一名程序员鬼才

那节课要讲解的是0-100中,如何分别打印出奇数与偶数,这个大部分同学应该都知道,其实非常简单,也就是 数字%2==0 即偶数,否则为奇数,其实就是以数字除以2的余数是否为0来作
为判断。 

学生们很快就写完了,我则挨个走过去检查,等我走到她电脑面前时,直接瞪大了眼睛,几行代码
就能写完的内容,她竟然满满写了一屏幕。 


我问她写的是啥,这么大一堆? 

她没有说话,只是直接运行了结果,笑得很得
意。 


结果正确。 

我仔细看了一眼她的代码,她不知道什么原因
\newpage

,并不知道有%这个可以直接求余数的运算符。 

她的做法是,一个整数A,先除以2,得到整数结果A1。然后把这个整数A 转换为浮点数B,B除以2,得到浮点数结果B1,最后比较A1与B1,如果相等,说明是偶数,如果不等,说明是奇数。(简单来说就是,整数类型会舍去小数部分,比如3/2=1,而浮点数会保留小数部分,比如3/2=1.5,所以如果是偶数,结果不会有小数,所以
A1、B1相等,反之。) 

所以她才会写一大堆的代码,其实是为了实现
%运算符的功能。 

她笑得很放荡,我没有打断她,虽然她的行为完全是因为不认真听讲造成的,但是凡事皆能看到好坏。好的是,她能用自己的方式,解决实际需求问题,这正是一个合格的程序员所需要具备的基本素养,鬼才是,她能用十几行代码实现一行就能实现的功能
,结果还是对的,你还不好骂她。 

\newpage

等她笑完,我才让她和旁边的男生互看对方的
代码,两人皆叹,我靠,还可以这样?! 

顺带一提,她的毕业论文7万字,查重率0.3%,自己口述代码量大概有十万行,我没细看,我不是她的指导老师,鬼知道里面有多少离谱的写法,
说不定一万行就能写完。 


先写到这,要去上课了。 


手上事情有点多,非常抱歉,感谢提醒我下课
的同学,拖堂惯了,还拖到知乎上来了,老毛病。 

2012年前,其实我和那孩子的交集并不多,除了上她们班的课,最多也就偶尔在学校里碰见,
打打招呼。 

她看起来文文弱弱,温温柔柔,打招呼时可完全没有女孩子家家那种温文尔雅。隔着老远的距离就开始大喊,拼命挥手示意,总让人产生一种她有什么
\newpage
大事想要迫不及待告诉你的错觉,结果每次走近都是
,您去上课吗?您下课了啊? 

我多半情况下都是装没看见,或装没听见,有时等她靠近了还要演一出好巧喔的样子。实话实说,真不是我高高在上或者故意摆什么长辈的架子,我总不至于也像她那样,手举得老高摇得飞快,一边跑向
她,还一边答应吧?好歹路上还有这么多学生呢。 

她的头发多数情况下都是乱糟糟的,还有一点点男孩子气,藏不住笑,遇到一点点好玩的事情,总是哈哈哈哈笑出声来,倘若是能让她得意的事,那更是能笑到山呼海啸,直至喘不上气来。这点和妻子稍显不同,虽然放荡不羁后的笑声几乎完全一致,但妻
子至少只在与我独处时才敢如此毫无保留。 

我天生对这种傻不拉几的鬼笑声没有抵抗力,即便能让她们发笑的事情原本并没有那么可笑,我也会不由自主地被那些笑声感染,稍稍控制不好还会跟着笑出声来。这也是我对那孩子格外关注的另外一个重要原因,时至今日,我依然这么觉得,有些人的笑
\newpage

,就是能让你感觉到这个世界的轻松。 

2012年,下学期开学后的第一节课后,那孩子找到我,很兴奋,说想买一台笔记本电脑,让我
推荐推荐。 

我起初是很吃惊的,因为计算机相关专业的学生,不说100%,基本上99%的学生都在入学前或开学后1个月内就会买电脑,更别说还是软件工程。毕竟实操大于一切脑补,总不至于用草稿纸写代码
吧? 

也许是我有些刻板印象,觉得即便她是个女生,也不至于已经上了一个学期的计算机相关课程还没有自己的电脑,而且期间还通过邮件交过编程的作业,所以转念变便觉得应该是,电脑坏了,电脑不好用
,想换个新的。 

说真的,我唯独没朝没钱买这个方向上去想。虽然当时已年过30,但归根结底还是涉世未深,很多事情上的结论都是想当然。在我当时的认知里,心
\newpage
里知道,的确是有很多学生家庭条件困难,且贵州这边乡镇考上来的学生也偏多,但怎么说呢,能交得起大学的学费,能选择到计算机相关的专业,再怎么不
济孩子父母应该也能凑得出一台电脑的钱吧? 

其次是无论是她的言谈举止,还是她天真烂漫的笑容,都始终夹带着她从骨子里透出来的自信,让你本能的觉得,这个女孩生活的世界必定阳光明媚,
风和日丽。 

于是我问她,大概要买什么价位的。她回答我
最好不要超过2500。 

2500要是放到现在,选择不少,谈谈价甚至还能买到不错的,性价比其高的,而放到当年,直
接是,买不到。 

我没过脑子,直接蹦出一句,咋,二手的啊?
她笑着点了点头,回我二手的也可以。 

我当时着急,走得很快,只是告诉她2500
\newpage

怕是买二手的都老火,让她去电脑城问问。 


她小声哦了一声,脸上的兴奋完全没有了。 

当天下午上课,不知怎么的,她瞬间失落的表情时不时就突然印在我眼前,就是那种笑容戛然而止的画面,一遍又一遍,直到下午吃晚饭,还是会莫名其妙就蹦跳出来,接着又开始乱想,她去了电脑城,买不到电脑,她去了电脑城,买了个二手电脑,她去了电脑城,被骗买了个破烂电脑,越想越离谱,越想越心惊,还莫名其妙愧疚起来,也不知道自己在愧疚
什么。 

也许,仅仅只是也许,当时自己对她就有了一些不一样的情愫,只是当时的自己并没有擦觉,也没有往任何方面去想,只是单纯的觉得自己有可能几句无心的话,伤了小孩子的心。一直到晚上上床前,始终被那种奇奇怪怪的愧疚感缠住不放,逼得最后给她们辅导员打电话,问了那孩子的手机号,并问她有没有去电脑城买电脑。她在电话那头嘻嘻哈哈的答复我还没有去,因为她也问了其它同学,确实是买不到,
\newpage

打算等再攒一些钱再去。 

我临时起意,告诉她我有一个不用了的旧电脑,编程啥的完全没问题,可以1500块钱卖给她,问她要不要。她在电话那头呼喊连天,一口气说了N句带脏字的捧语,一时间搞得我不知道她是在骂人,
还是在谢人。 

我确实有一台额外不用了的旧电脑,可当我翻出来想要把电脑中的数据备份并重装一下系统时,我卡住了,数据很多,我不可能整盘复制,我也不敢一一点开挑选,僵持了很久,最终还是放弃了。但已经跟那孩子约好了明天一早在我办公室交货,无奈起了个大早,去市里的电脑城打算新买一台交差。想得很好,天才蒙蒙亮便出发,8点前便能返回,结果人家
10点多11点才陆陆续续开门。 

一切都不在计划之中,一切却都像是最好的安
排。 

她不相信我的电脑是旧的,即便我提前在电脑
\newpage
上装了很多有用无用的软件,放了很多日期是之前的资料文稿图片,我自认为天衣无缝,装软件和新建文件前,提前修改了系统时间,还顺带更换了一张奇丑的电脑桌面,把电脑上的标签、贴纸角抠破一部分,
又在地上抹了半天灰在电脑壳内壳外擦了好几遍。 


我高估她了,她压根不看。 

吴老师,我感觉你这电脑像新的一样,是新的

怎么可能?你看这文件的时间,这些软件安装
的时间……不是你看看啊?? 


你那台电脑都掉漆了。 


…… 

场面很尴尬,我简短交代一下,总而言之,就是狂誉了某惠字开头的品牌,以表达爱不释手,猛贬
了某联字开头的品牌,以解释历久弥新。 

\newpage

说到最后,她信没信我不知道,我自己反正是
信了,这么多年了,还在用惠某的笔记本。 

就像学生问的问题我突然回答不了那样,假装有急事,要么下回再议,要么蒙混过关。我催她赶紧拿了电脑走人,我还急着要去开会,我脚上抖得厉害,她不敢耽搁,从兜里把钱掏出来递给我,1500
块,钱很新。 

临走前她千恩万谢,脸上挂着花,直言我帮了她的大忙,要请我吃顿饭,好好感谢一下,我答复她
可以可以,等以后看看时间。 

后来约了我好几次,确实是有事情给耽搁了,有的时候也不在学校,只是自那以后的逢年过节,她都会给我发来短信问候,小到端午,大到除夕,中间还通过几次简短的电话,内容都是关于电脑和编程的
问题。 

那顿所谓的感谢饭,也就一直拖到了那个学期

\newpage
的期末。 

她话很多,和妻子一样,说起来就没完没了。我老捧哏了,很擅长不让对方冷场,嗯,对,可以,
是吗?啊?怕不会喔! 

那天她和我聊了很多,从学习目标说到人生规划,聊了当下,聊了未来,俩人却很有默契,对过去只字未提。我问她专业为什么选了软件工程,她告诉我她们村里有个大哥哥学的土木工程,赚了很多钱,那哥哥告诉她现在土木没那么好赚钱了,软件才好赚钱,于是当她知道有个专业叫软件工程时,便毅然决然的选了它。事实上,在此之前,她都没怎么接触过
计算机。 

她的目标很明确,赚钱,赚大钱,赚很多很多
的钱。 

我告诉她不能把钱看得太重,又顺着话题说了一大堆连自己都不相信的大道理,比如目光的长远,人生的意义,生命的价值,她听得很认真,频频点头

\newpage
回应,若有所思,不时露出恍然大悟的表情。 

她的表情我再熟悉不过,每次妻子给我念叨这些道理时,我应该都是这样的表情。听进去了,却也没听进去。那时我才明白,什么狗屁道理,我只是想不停的跟你说话,让你不停的听我说话,我所描述的一切,也许我自己都做不到,但是,会让你觉得我很
厉害,很有想法。 

她被我吃完折耳根后的样子逗得咯咯发笑,那笑容中没有半分幸灾乐祸,反是满心欢喜,好似完全因为自己的鼓励,把一个腿脚不好的人推上了珠穆朗
玛峰顶。 

我被折耳根独有的味道呛得满眼泪光,真的没
有想到,原来是满心欢喜。 

道别前,她郑重其事的夸我,觉得我是她遇到的最好最棒的老师,她开学第一天就后悔选了这个专
业,但是上了我的课,感觉打开了新世界的大门。 

我面上从容淡定,连连摆手否认,心里确也是
\newpage

乐开了花。 

于是挥手再见时,我朝她喊了一句,都一样。



我上个课的功夫,咋来了这么多人… 

确实是都一样,时至今日我依然这么觉得。她就是我遇到的最好最棒的学生,没有否定我其它学生的意思,但确实是只有她,让我也打开了新世界的大
门。 

2012年下半年,应该是开学有一两个月了,她抱着电脑来办公室找我,说是有几个问题想要问我。当时办公室里还有其它老师,我也真的以为她就是来问问题的,主要是她问得很逼真,皱着眉头,还
要时不时发出几句喔喔喔,或者喔~哦! 

只是问的问题都是之前才教过的,所以我多少有些小怨气,觉得为啥上我的课还敢不认真听讲。但也没有很不耐烦,还是稍有耐心的给她讲解答复。结
\newpage
果,办公室的另外两个老师前脚刚走,她就突然打断我的激情演讲,说吴老师,我查了,这个电脑要50
00多。 

说完,从兜里掏出厚厚一坨对折的人民币,攥在手里,非常得意的说,我再补你4000哈,接着
就把那坨钱推到我脸前。 

很突然,完全没有心理准备,直接语无伦次,又是摆手,又是摇头,甚至还跺了一下脚。我属于那种反应比较略带延迟的类型,就是习惯性先在脑子里梳理一条清晰的逻辑线,逻辑线虽然会有分支,但每一条分支…,算了,不编了吧,就是有点呆。前一秒还在说 if else,后一秒就给你整到给你一把钱你要不要吧?多少有点,跳转不过来。她真的,
很是擅长搞这样的事情。 

想不到任何词汇来描述当时的心情,很复杂,说尴尬吧,好像也不能叫尴尬,说心虚吧,又好像还有点让人生气。她的眼神中对我充满了敬意,可嘴角边那股藏不住的得意劲,分明就是在对我说,噢,我
\newpage

尊敬的懵逼者吴老师,我是你的破壁人。 

我们僵持了很久,过程就是那坨钱在半空被推过来又推过去,夹杂着几句您收下吧以及真的不用了。办公室房门大开,好在没有其他人经过,那场面,
看着可真有些像是我在收学生的礼。 

她是得了理不饶人,还说什么多的就当是给您买烟抽了。我可真的是诶哟喂了,这能是多了少了的问题吗?我教了几年的逻辑,没想到最后被她给逻辑了。她直言,要是我不收,就把电脑退给我,但是,
我要退她1500块现金。 


挑不出半点毛病。 

最后收了吗?收了,一点脾气也没有。她还给我找台阶下,说什么我教育她的,不要把金钱看得那么重,什么要不是我她要这个学期才能买电脑,肯定什么都跟不上了,最最离谱的是,说我把她电脑,对,这下就理直气壮她电脑了,把她电脑的标签扣破了,她以后不好转卖,所以少给我500,当时不知怎
\newpage
么就觉得她说得特别有理有据,合情合理,在她拿回去500块钱以后,还真就莫名其妙收得心安理得了

说真的,真是科技改变生活啊,但凡当年有个微信支付,或者但凡支付宝再早推广个一年半载,咱
也受不了这个气。 


同年12月的平安夜,那天已经上完课回到家了,接到那孩子的电话,说她们寝室的女生们有礼物要送给我,问我还在不在学校。我千恩万谢,表示已经回家了,让她可以明天再给我。她在电话那头吼了起来,说是平安夜的苹果,明天吃就没有效果了,一定要今天交到我手里,说着说着就说要给我送到家里
来。 

她论起理来像机关枪一样,丝毫不考虑我一个30出头的人,会不会,可不可能,相信她吃了平安夜的苹果就能一辈子平平安安。但是怎么可能犟得过她,光是在电话那头,唉…怎么怎么,唉…又如何如何的叹气,就能把你磨得脑瓜子嗡嗡作响,甚至还要
\newpage

不明所以自责起来。 

自然是不可能让她送到家里来的,最后没有办法,只能说让她在学校门口等我,我马上打车过去。好歹是个长辈吧,也不能空着手啊,正好楼下就有个水果店,于是让老板给我挑了一小箱苹果。当时想的是就算她们人多,一箱也应该够分,现在想来,确实是有点老直男那味儿了,要不妻子也不至于总是念叨
我不懂浪漫。 

我刚下车,就看见她就举着一个包得五颜六色的那种苹果朝我跑了过来,和之前一样,隔着老远就开始一连串的喊吴老师吴老师吴老师,真不是我有意诋毁她的形象,只是她举着的东西,已经不像苹果,
更像火炬,一蹦一跳的,头发乱飞,像在骑马。 

我还没责怪她让我跑这么远来拿她那非主流的玩意儿呢,她先以质问的口气问我,您不会想让我把这一箱都搬回去吧?相互道谢,满口祝福。我目送她的背影直至消失在路口的尽头,弯着腰,驼着背,再

\newpage
没有之前风一样的女子那般模样。 

除了那个花里胡哨的苹果,她还交给我一张那种可以折合的圣诞贺卡,虽然当时就比较好奇贺卡中写了什么,但也直到她的背影完全消失,又打了一俩出租车后,才迫不及待地在出租车上把贺卡打开。我很快读完了贺卡上的内容,却突然止不住声,哭了出

我依稀记得司机师傅,面对我突如袭来的举动时的反应,是纯正的贵阳口音,诶哟天,囊还哭喽?屁大点事,东边不亮,它西边点撒?莫在一颗树上吊
死! 

我当时沉浸于莫名的忧伤情绪之中,并没有来得及反驳师傅,他打开了话匣,及其热心,用男人之间的安慰方式,不断鼓励着我,他说得动情,甚至搬出了自己被人拒绝的过去,我总结了他的中心思想,
她不要你的苹果,你可以送给别人。 

贺卡中的文字并不多,开头是一段对我的感谢,中间是一大段手写的C++代码,用了几个简单的逻辑判断和方法调用,如果输出的话,内容大概会是
\newpage
什么选择【快乐】就能【幸福】,如果【不开心】就会怎么怎么样这种。我一眼便看见了几个BUG,少
分号,缺参数,还有可能出现空指针异常。 

代码之后还有一句话,吴老师,希望我们能成
为你的药。 

这句话,应该只有那一届的学生能够看懂,其实还是源自于她开学第一天,公然嘲讽我名字时,留
下的那个谐音笑话。 

既然写到这了,那还是直接说了吧,藏头去尾,总感觉会表达得奇奇怪怪。我的名字是育志,育字不是指教育,只是祖上传下来的字辈,当老师也不过是个巧合而已,其实没有什么联系,确实只是一个平
平无奇的名字,打死我,我也联想不到能有什么。 

离谱的是,育在贵州的发音是,油,药的发音
是,哟。 

所以联想鬼才当着全班同学的面用贵州方言大
\newpage

声示范,你们读快一点,是不是就是,无药治? 


我当时就给她鼓掌了,左手扇右手。 

所以她想表达的可能是,既然我是无药治,那
她们就来当能治好我的药。 

而我会突然止不住情绪的原因是,这句话让我想起了大四那年,妻子在雪地里跟我说的一句话,跟
你在一起,连感冒都是开心的。 

所以我看到的,其实和不知道名字梗的人看到
的一样,她是药。 

说矫情也好,说脆弱也罢,像我这种神经敏感的人,确实非常容易泣不成声。按现在的词来形容,那应该是叫,被破防了,但当时的我却有不同的心境
,大概是,感觉重新穿上了铠甲。 


13年春节,大概初三初四的样子,接到那孩
\newpage
子的电话,说要送一些她妈妈做的香肠和腊肉给我。我当时人不在贵阳,于是与她约好元宵节前一天在学
校附近碰面,想的是请她吃顿饭以表示感谢。 

吃饭时,我问她为什么这么早就来学校了,她回我,来体验生活。鬼的体验生活,说完她便笑了,我也笑了,没有丝毫遮掩地跟着她笑出了声,失了心智,笑了很久。她打两份工,一份在饭店,从下午4点到晚上9点半,一份在夜市摊,从晚上10点到凌晨2、3点,结束后在肯德基呆到早上6点半,坐最
早一班公交回学校。 

我问她人家不把你赶出来吗?她很吃惊,挤着眼睛反问我,为什么要赶她。我回答不出来,她开始给我上课,轻描淡写,话里话外却透露着这个世界以及这个世界上的人,其实并没有我想象中那么糟糕。

对口相声很快变成单口,她开始绘声绘色地讲述她的打工经历,从高中时期的第一份工,到寒暑假的临时工,有的轻松到坐着发呆,有的辛苦到站着睡着。她脸上始终带着笑,激动时口水乱飙,眼眶中时
\newpage

不时发出几道闪烁,我看不清是泪是光。 

讲到最后,她突然改口问我,吴老师,你现在是不是特别想要资助我?我连忙点头,一口气说了好几个对,甚至强调自己数十分钟之前就有了这个想法。她哈哈大笑,是那种得意的山呼海啸,仿如站在山
尖尖的勇士,昂首挺胸。 

她笑了很久,直到旁边的客人都转头看向她。她收起笑声,问我那是不是觉得她过得特别惨。我摇头否认,嘴上还带着一连串的没有,没有,没有。她纠正我,说我应该觉得她特别惨,应该觉得她特别不
容易,这样才能显得她特别厉害。 

我朝她竖起大拇指,手还在半空摇了摇,用蹩
脚的贵阳口音对她说了一句,牛逼喔~ 

她再次咯咯咯的笑出声来,同时回应我,是牛
批喔~ 

很像,不是外貌,是整个人的感觉,无论是口
\newpage

吻还是语气,以及肢体上的小动作。 

她最小,上面还有两个姐姐,用她的原话来说,她父亲认为她们都是赔钱货。她收到录取通知书那天,得到的不是父亲眼含热泪的赞可,而是父亲眼含
热泪的辱骂。 


她拒绝了我的资助,她缺的从来就不是钱。 


13年下旬,她大三上学期开学。当时微信已经逐渐推广起来了,我和她也在不久之前互加了好友。她在微信上给我发来了一堆图片和语音消息,图片
内容是一堆代码片段,语音消息是哭爹喊娘。 

几句话就能说完的内容,她发了几十条语音,其中大半都是啊啊啊,要死了,咋回事啊。总结下来是,她们自制的抢课软件不知什么原因出了BUG,她不仅没有抢到我的Java选修课,还被分配到乒乓球课上去了,连取消按钮都没有,她们班好几个同

\newpage
学扬言要把她杀了,求求我救救她。 

她自然找不到她软件的BUG,如果我不说,到死,她,以及她的师兄师弟们,都不会知道,当时她们口中那个外表憨厚,慈眉善目,身形胖丢丢圆溜溜的王老师,现在已经不在我们学校了,是我见过最最最最喜欢搞恶趣味的老师,没有之一,暗地里最喜
欢跟她们那群计科系的生瓜蛋子玩斗智斗勇。 

大部分模仿人机交互的同学都难逃老王的魔掌,更别提像她那种无脑发Http请求的,分分钟就被逮到IP,然后接受王的制裁。我告诉她,有可能是学校网站的系统BUG,也有可能是重新分配了课程的系统编号,让她接受现状,下次不要再投机取巧,老老实实选课,这样对其它系的学生才公平。自然不可能给她透露实情,其一是不可能出卖老王,其二是老王问我这次怎么惩罚那些作弊的小崽子时,我随
口说了句调到乒乓球? 

C++只教到大二下,她大三时,我已经没有她们班的课了,所以她想来听我课的唯一途径,就是选到我开的Java选修。当时还只有我一个老师开
\newpage
了这门课,所以很受欢迎,一些非计科系的同学也会加入到抢课的队列中,自然就变得有些难抢了。她早在数天之前就信心满满的告诉我一定会来上我的选修课,我当时其实意识到她极有可能编写了抢课程序,但却也没有提醒她,因为内心深处,还是有那么一丝丝期许,她能掀翻老王,甚至连台词都想好了,是极
其无心的,咦,这个好像是我的学生吧? 

她那次是真哭了,给我打电话,梨花带雨,疯狂抽泣,表示她书都买好了,还在肯德基自学了数十个夜晚。我让她解释一下C++和Java中堆栈的区别,她哭得更伤心了。她哭得更伤心了,我也就心软得心安理得了。废了很大的功夫才联系到体育系的老师,连夸带捧,最后商议了她可以不去上他的课,也给她合格的成绩,而在我这边,她能来上我的课,
但没有成绩。 

兴许是我跟她接触得多了吧,夸人捧人的话变
得信手拈来,害她乒乓球课得了95分。 


\newpage

她每次都坐第一排,没有一次把头发捆好。 


我们常常在微信上聊天,聊编程时像师生,她
彬彬有礼,聊生活时像朋友,我落落大方。 

她毕业前最后一次与我见面,她问了我一个问
题,C++和Java应该怎么选择。 

我回答得很认真,分别例举了两者的利弊,学习路径上的差别,当下的就业场景,以及未来的发展趋势,至于现阶段应该如何选择,其实还是最好遵从
于自己的兴趣。 

她没有再彬彬有礼,对于我的回答极不满意,表示我在敷衍她,虽然说了一大堆,却没有给到她实
质上的建议。 

我收起了老师的架子,哈哈笑过之后告诉她,我可不敢替你做决定,因为不管你现在选了什么,大概率以后加班的时候都会骂我,当年让你选了个什么

\newpage
玩意儿。 

她抿着嘴笑,说没想到被我识破了,还想着以
后吃不上饭可以来找我混顿饱的。 


我让她安心,天塌饿不死搬砖人。 

她离开贵阳前,也问了我一个问题,C++和
Java应该怎么选择。 


我想了很久,最后没有回答。 

我在第一次接触Java时便被它深深吸引,
对我来说,它太熟悉了。 

Java和C++很像,不是写法,是整个语言的感觉,无论是循环的关键字还是语法结构,以及结尾上的小字符,真的,非常非常像,有时候写Ja
va代码,我甚至都会愣神。 

而Java独有的一些机制,用法,以及设计,常常让我很兴奋,很惊喜,我不可否认,我爱上了
\newpage
这门语言,它让我重拾了探索的欲望,恢复了学习的
热情。 

但我很容易陷入一个怪圈,因为它们太像,我常常在写Java代码时便不由自主的把C++的语法强行带入,于是反复发问,为什么它会没有指针,为什么它不可以多继承,为什么它不支持运算符重载
,…… 

可是,它为什么需要有这些?凭什么就需要有
这些?就因为你是一个教C++的老师吗? 

最后的最后,她去了北京,成了一名Java后端开发工程师,不仅是合格,应该算得上优秀。我还在贵阳,还是一名C++老师,谈不上优秀,应该得上合格。

\end{document}
