\documentclass{article}
\usepackage[utf8]{inputenc}
\usepackage{ctex}

\title{北饮大泽\footnote{Click to View:\url{https://web.archive.org/web/20221113144842/https://linkeer365.github.io/Linkeer365ColorfulLife2/1376189623/}}}
\author{卡塔兰}
\date{2021-03-26}

% \setCJKmainfont[BoldFont = Noto Sans CJK SC]{Noto Serif CJK SC}
% \setCJKsansfont{Noto Sans CJK SC}
% \setCJKfamilyfont{zhsong}{Noto Serif CJK SC}
% \setCJKfamilyfont{zhhei}{Noto Sans CJK SC}
% \setlength\parindent{0pt}

\begin{document}
\CJKfamily{zhkai}

\maketitle


\Large

幸运的疯了的闲杂人等,颈子向后兜售人生
 

时代的暗涌是冰镇而苏醒而死的水,水是死在
水与冰骑墙的墙外的 

人并不是水,人在把自己当闲杂人等的那些日
子里,人是能感受到了自己的活与时代的死, 

组成墙的或许也曾经是水,曾经源于沧海、源于神话、源于源于的水,对生死不加置喙的水,像哑
在过去日子里的人 

因为水是依托与墙,或许它们正依托于不再作为水而感知而存在的水,或许不,但有一点将是清楚
\newpage
的,即湿了人的手、人的衣物、人的大清早的好心情的,一定不是冰,在水以外冰是最无忧与无辜的存在
,在人生以外的人同样幸福得令人咋舌 

然而时代的水,与墙的水人的水冰的水水的水却又有不同,时代的水用疯癫来包裹严肃,用死亡来装点清醒,用边角料与边角料的边角料来填满爱与希
望 

水的味道如同水一样,人们从水中找到血,水从水中找到冰,冰是无言的,聚合了阳光把骑墙的墙、曾经可能由部分的冰或部分的水组成的墙,点燃了,冰看到了自己的血,人看到了自己的人生,他们都
感到幸福,他们都不停流血 

冰从后面轻轻抱住已经根本止不住融化的冰,

(我下辈子做墙吧,真要是你下辈子也不能不
流血,我就把我的人生输给你) 

(人拆掉墙再正常不过了,西墙大概或许坏了
\newpage
、抑或没坏也好,东墙都不能阻止人把他的头足肢干和血液,涂抹到西墙上去;在季孙出面的时候,西墙
上也多了他的血) 

(墙固然易拆,可封印在墙中曾经作为冰或者水的那些存在,将不死不灭;从西到东由北向南,从圣诞到末法,从野人到非人,它们都将存在着期待着
,希求再作为新的江河与洪水) 

(人同墙又何异呢,人拿走人的好心情,人拿走人的物品,人拿走人的器官,人拿走人的生命,然而化作洪水的墙中之水,化作江河的墙中之冰,化作信托与代言人的,在墙以外的同源于水的一切存在,终究是能摆弄得人们愿疯而不得,只是每天把自己比作幸运的长颈鹿,兜售作为人的应有的幸福、爱与希
望…) 

(幸福那种东西怎样都无所谓吧,但我真是好喜欢好喜欢太阳…我知道是我不配得到太阳的热,我太傻了…事到如今你不也满手是血,是我害得你变脏

\newpage
的…) 

(冰的泪是从来不脏的,这也正是你所珍贵的
地方,你是从来没有半滴肮脏的血的…) 

(谢谢你,这是最后的有着太阳的味道的我对你所下的命令,请你一定要忘掉我,忘掉与我之间乱七八糟的阴差阳错,却也请你记住这珍贵的阳光,记
住就算被这样玷污也没有丝毫编织的阳光…) 

后来我如愿地做了墙,放弃了所有言语,可我总也想不明白,曾经那个啰嗦到死的爱着太阳的冰,什么时候(弃其杖,化为邓林)…

\end{document}
