\documentclass{article}
\usepackage[utf8]{inputenc}
\usepackage{ctex}

\title{老A\footnote{Click to View:\url{https://web.archive.org/web/20221128014250/https://rentry.co/hpoet}}}
\author{张玉清}
\date{2013-12}

% \setCJKmainfont[BoldFont = Noto Sans CJK SC]{Noto Serif CJK SC}
% \setCJKsansfont{Noto Sans CJK SC}
% \setCJKfamilyfont{zhsong}{Noto Serif CJK SC}
% \setCJKfamilyfont{zhhei}{Noto Sans CJK SC}
% \setlength\parindent{0pt}

\begin{document}
\CJKfamily{zhkai}

\maketitle


\Large

老A入学的时候,细高个儿,长头发,戴窄
框金边变色眼镜。 

我想他那时候总该是有些毛病吧,或者是语言不如女生们所希望的那么文雅,或者是作风不够她们所期待的那么庄重,也许因为他经济上比较宽裕穿着打扮比一般人总是好一些吧,反正是入学后不久便由
女生们送了个外号一老阿! 

“阿”即是指阿飞的“阿”。那时我们男生和女生之间还没有融洽到互相呼唤外号的程度,一开始“老阿”只是在女生中秘密叫开,几个月之后才被老A知晓。他歪着脖子回忆了一下说并没有什么地方得
罪她们呀 

\newpage

我说从小学到中学到师范,女生都是一群又可
爱又可恶的东西! 

老A摇着头说这样的断语不好。我说嗤,你就
是在女生面前这么贱,怪不得她们叫你“老阿”。 

老A一笑此后老A见到女生仍然打招呼说话,面部也不改色有一个星期六下午,那是学习时间,大家讨论传统文化,有人肯定有人否定更多的人说“扬弃”。老A站起来批判孔圣人,他说圣人的语录里有很多谬论,应该大力批判,比如圣人云:“唯女子与小人为难养也!”他把这句话重复了三四遍,然后详详细细地解释,一边解释一边问大家明白了没有。他啰里啰唆地解释啰里啰唆地问,等到大家说明白了全
明白了他就坐下了。 

有人说你还没批判呢。老A才又站起来说批判批判当然批判孔圣人是不对的嘛,说完他又坐下了。
他说这句话时不加任何标点。 

男生们大笑。女生们此后好多天咬着牙大骂“
\newpage
老阿”。老A得意,得意了好多天,却不知不久后遇到了入团问题。我们师范,每一个学年都要发展一批新团员。我们班,一共四十七名同学,有二十五名是团员,从初中加入的,剩下二十二名不是团员。我在二十五名里,老A在二十二名里。老A入学时的中考
成绩在他们本县位列第五名,却不是团员 

申请书是二十二名同学都应该写的,候选名额是六人,六人里有老A。但在团全体会议上表决时,老A被“表”了下去,全体女团员一致取否定意见,“老阿?哼!……”结果六名候选人只有五名入了团
。 

如果不是得罪了女生,老A这次人团是不成问题的。他在男生里很有人缘,男生不在乎他文雅不文雅作风庄重不庄重。但是团员中女性的比例占绝对优
势 

事后,我将情况告诉老A,老A潇洒地甩了甩
头发挥了挥手说“无他!何其奈我尔!” 

\newpage

这句话是老A别出心裁地从《卖油翁》里生发
出来的,原句是“无他,但手熟尔!” 

后来当我们班男女生的关系融洽到可以公开互相呼唤外号了,我们男生郑重向女生递交了关于更改老A外号的提案,因为被女生公开地“老阿老阿”地呼来唤去,不但老A本人无法接受,就是其他男生也觉得脸上无光。但是可恶的女生们仍对老A前嫌难释,宁死坚持原判。最后折中,改“老阿”为“老A”,内涵是:“阿”的汉语拼音为“a”,按书写规则用于人名当大写为“A”,所以叫“老A”。这样叫起来就比较隐晦了,全体男生包括老A本人在内勉强接受了这个东西方文化交流渗透互补所造就的外号。从此老A除了在花名册上和将来的毕业证上仍沿用原
名齐海军之外,光天化日之下便一律叫“老A”。 

但“老A”却没有达到预期的效果,相反倒更诱使人们想弄个究竟,结果是所有知道“老A”的人也都知道它的出处,呼唤的时候自然在心里所认可的含义就是“老阿”。很快,“老A”又传扬到外班去,渐渐地蔓延到学校的各个角落。真是弄巧成拙,如
\newpage

果一直就喊“老阿”,倒不一定传得这么快。 

那个提出折中方案的同学觉得有点儿对不住老A。但既成事实,老A便用了甩头发挥了挥手说:“
无他!” 

他当时根本没能预料到这个外号将给他造成深远的影响。老A什么都不在乎,我这做朋友的却时时为他感到不平。我比别人更了解老A,在我眼里老A这人很不错,讲义气,乐于助人,热情而不计较得失。比如朋友、同学有困难向他借钱,他总是慷慨解囊,并且从来不让还,只说:“就算我请你喝酒了!”拯救大熊猫运动无记名捐款,他一下子捐了三十元。只不过他的性格中有一些不拘小节的成分,这大概就
是容易被人误解的因素吧 

而且老A还很善良。一天我和老A上街,见一群人围着一个老乞丐看。这是个拄双拐的残疾人,他坐在地上,双拐扔在一边,面前放一张硬纸板,上写自己生活艰难向四方求援的文字。老A面色沉重,叹一口气便往口袋里掏钱,他那天没带多少钱,便将口
\newpage

袋里仅有的六元钱全放在老乞丐身前的破帽盔里 

围观的人看到这小书生一下子便扔出六元钱,立刻一片喷喷声起,议论纷纷。老A丝毫也不理会众
人的褒贬,拉着我挤出人群。 

回到学校我对几个要好的人讲了老A的事,他们很感动也很佩服。我忽然灵机一动,认为这是改变老A在人们心目中形象的好机会,便有意将这事在人们中传开可没想到又是弄巧成拙,因为传来传去不知为什么失去了本来面目,把老乞丐传成一个姑娘,后来竟有人传说老A上街看到一个年轻漂亮的姑娘丢失了车票(这是小说和电影里常有的),便慌忙从兜里掏出了十块钱……到后来便演化成了一个荒唐而戏谑
的故事。真见鬼! 

我说老A呀老A呀……老A说无他!恰巧当时搞“五讲四美三热爱”,大树典型。学校主管“五、四、三”的干部听到了这个乱纷纷的故事,也是缺少工作经验,竟然跑去问老A哪种传闻是真的,说如果是年老的男乞丐就树他为典型,如果是年轻的姑娘那
\newpage

恐怕师生们都不服气。 

老A气得差点儿暴跳起来,但他很快稳住了自己,故意笑微微地说:“是一个很漂亮的姑娘,要不
要我拿照片给你看?” 

“五四三”气得歪着鼻子悻悻而去。我埋怨老A太不慎重太无城府,这次如果能当上典型那么到二年级发展团员就保证能入上去了,那时我们已经快升
二年级了 

老A挥着手说无他无他!我气得说你老这样无他无他终归会有“有他”的时候不幸真让我说中了暑假以后我们就二年级了,老A此时已经十八岁半,他青春骚动,一进二年级就老跟我夸邻班一个叫刘菲的
女生。我知道他快 


进入角色了 

果然老A在经过了几天的愣征恍惚和心事重重之后,有一天响午肃穆地对我说要给她写一封信。那
\newpage
时太阳白亮亮地挂在天上,晒着他脸上渗出的细小汗珠,我看着他脸上少有的肃穆直想笑。我说你不要给自己找麻烦给朋友丢脸。老A说那么多人都在写信干马我就不行?我说我总觉得刘菲这人不可靠。老A不
高兴了说我本想要你来支持我可你……吾意已决! 

过了两天老A发出了精心设计的一封信,虽然“吾意已决”可信发出去后却是忧心忡忡。一连三天没见回信。老A计算往返时程知道如果有回信早该到了,因为我们校内寄信很简单,只要将信放在传达室窗前的待领栏前,在栏目上写上领取人姓名就完事大吉既省时间又省邮票。我安慰老A说人家在慎重考虑
 

一连五天没有回信,我和老A焦躁不安。第六天,传来消息说刘菲将信交到了教育处。当晚得到证实教育处传老A到处。半节课后老A回来,垂头丧气。我问他教育处怎么着,老A说不怎么着只是拿出那封信问是不是他写的,又问了些细节就放他回来了。

没批评?没批评。我想这下子坏事了,暴风雨
\newpage
前的寂静,糟糕的在后头。我这样想没对老A说。过了两天便满城风雨,刘菲还到处扬言说如果这信不是老A而是别的什么人写的她不会告到教育处,因为是老A她才告的,听那口气好像还是迫不得已的。真够浑的!我到现在也不明白刘菲为什么正因为是老A就告,你有什么了不起!你向乞丐的破碗投过面值超过
五分的钱币吗? 

刘菲的话当然也传到了老A的耳朵里,他听了甩用头发挥了挥手,我以为他又要说“无他”,但他垂下了手很沉重很沉重地说“我真的没有想到我的名
声会这么坏,真的…… 

我叫一声:“老A……”却再也说不下去,我此时多么希望他能潇洒地说一声“无他”,可是现在
我知道了世界上任谁也有潇洒不起来的时候。 

第八天,班主任把我找了去要我提供关于老A的情报,并且为了对我表示信任,还在叮嘱我保密的前提下,小声向我透露了学校准备给老A处分正在整他的材料的消息。吓了我一跳,我认为很糟糕却没想
\newpage
到会这么严重。我用一些无关紧要的情报敷衍了班主
任回来就着急忙慌地找老Ac 

“学校要给你处分!”我直截了当地说,忘了
应该照顾一下他的情绪。 

老A呆了呆,良久苦笑了一下,说:“给我处分我倒不在乎。你相信吗我真的不在乎什么处分不处
分。” 

我说:“我明白。”此后几天,老A像等判刑似的等处分。调查正在接近尾声,材料已整得差不多了。学校这次要狼一家伙了,因为目前连同邻近一些学校在内男女生之间通信成风(学校当局对外从不承认早恋,学生自己也不承认,双方都不承认存在早恋,只存在“通信”,可这种“通信”同样令当局恼火),当局恼火却抓不住把柄,现在终于有了原告了,
当局要杀一做百! 

老A憔悴了。本来就是一个瘦人,这一憔悴失形便似连站立走路都是勉强支持了,摇摇晃晃地像一
\newpage
根没有生命的竹竿。很快好多同学都知道了学校要给他处分,他走过哪里总要带来一片小声的议论,平时要好的朋友则来安慰他。老A一概表示沉默,不说无他,只偶尔对来人说一两声谢谢。恐怕只有我一个人知道老A其实并不是在乎什么处分不处分,因而也更
为他感到一种辛酸 

我实在不明白刘菲为什么要告发老A,不明白一个女孩儿为什么接到这种信后要去告发。老A也不
明白吧?不然他为什么会如此的憔悴呢? 

埃,老A,老A呀……那辆豪华型小轿车开进学校的时候,谁都看见了,因为平时学校出来进去的只有一辆老掉牙的破吉普车,那是校长的坐骑。小轿车只一会儿就又开了出去,第二天再来时后面跟着好
几辆大卡车。 

大卡车雄壮地停在院子里,小轿车径直开向校
长室,校长老远就迎出来。 

过一会儿,教育处主任亲自来找老A,让他诧
\newpage
异的是主任的脸色分明很和蔼。老A也莫名其妙,因为他并没有忘记那天在教育处里主任那铁青色的面孔

老A只一会儿就回来了,脸色不难着,我赶紧迎上去问他。他有些高兴却又真的很淡然地说:“我
爸爸来了,给学校送原料。” 

“就是坐小轿车的?”“嗯。”我早知道老A
的爸爸是个体户,却不知道是坐豪华车的个体户。 

“叫你去做什么?”“无他。不知道是校长叫
我去见我爸爸还是我爸爸叫我去见校长。” 

我想起一个月之前校长的《告全校师生书》,内容是说学校校办工厂由于没有原料濒临倒闭,希望广大师生都来关心工厂的危亡云云。那原料的名称我
没记住,只知道是市场上的紧缺物资。 

老A告诉我说他当时就给他爸爸写了封信,但对他爸爸能否弄到那些紧缺原料他也没有把握,就没

\newpage
对别人讲 

第二天班主任又找到我,告诉我对老A的调查已经结束,理由不充足,不准备给他处分了,并一再嘱咐我不要对老A讲学校曾经要给他处分。看着他那么郑重其事地嘱咐我,我真想笑他的迂腐我说:“学校里早就传开了,老A早知道了。”但他只说了一句话就让我知道了其实迁腐的是我:“那些只是传言,懂吗?”我说懂了同时心里比被女同学骂做笨蛋时还惭愧。但不管怎样,仍然高兴我听到“曾经”二字便
放下心来,知道事情已经过去。 

不但事情已经过去,而且老A从此还有了一段中兴的历史就在卡车开进学校大约一个月之后,那时老A和刘菲之间的风波连舆论上都巴平息,时来运转,学校团委书记同志开始找老A谈话,而且是短短的两个星期中就谈了三次,中心思想是希望老A好好努
力向团组织靠拢。 

谁都明白有了这样的谈话老A便是已将团证稳
稳地握在了手里。 

\newpage

很多人都知道了那挽救了校办工厂危亡的原料是老A的爸爸送来的,老A真让人既佩服又羡慕。老A简直成了学校的红人。和刘菲邻班,常碰到,每次刘菲竟然都显得很惭愧,在老A面前埋着头走过去,
原先那趾高气扬的气势不知道哪里去了 

老A前景辉煌。有消息透露说他将来很有可能留校。嘿,留校对于普通的学生来说那是连梦也不敢
做的。 

久已被人忘却的大名齐海军又有人呼唤了,是校长。一天晨操后人们看见校长站在操场入口处叫:
“齐海军,齐海军同学一” 

整整一节早自习,老A从校长处回来,我问他谈了什么,他说校长问了他的生活和学习情况,说了
些勉励的话语,无他,以示关心尔。 

我真为老A高兴,想想看,全校上千名学生有几个能得到校长接见的殊荣呢?三年的师范生涯,老

\newpage
A算是达到了鼎盛时期。 

但可惜这鼎盛太短了,命运有时好像是在故意捉弄人。比老A的突然发迹更令人感到意外,谁都没
有料到他会那么快就从中兴走向没落。 

那一天天很阴,有人传话要老A去教育处一趟。老A潇洒地去了,沮丧地回来。他回来时自习课还
没有散,全班都让他的沮丧闹得一愣。 

老A跟我说他一进教育处就发现主任的脸比外面的天还要阴,主任严肃地直截了当地对他讲学校已经对他给女同学写信一事做出处理决定,决定给他记
过处分 

老A听了错愕了一下,又愣征了一会儿,他百忙中在心里做了一番计算一一现在距离他给刘菲写信
事发整整是六个月又两星期 

主任拿出处分的复印件,轻飘飘的一张纸,说是这个就交给他了,要他保存好。老A不明白为什么要保存好却也没问。主任文说学校鉴于老A对校办工
\newpage
厂做的贡献就不准备将处分开大会公开了,说只通知到教育处、团委、学生会、有关领导和老师、老A所在班的班委会就行了,说这样做是为了避免给老A扩大影响。主任用极其严肃极其平板的语调向老A讲完了这些话就不再说什么,好像除此之外也没有什么话好说了。沉默了半分钟主任将处分塞在老A手里放他
回来了 

无他无他,老A说,大家别担心,无他。过了几天,大家和老A得到了一个消息。原来现在市场上的变化是一日千里,六个月之前的紧缺物资现在大批投放市场,学校当局刚刚知道,在价格上,六个月之
前他们傻子似的被老A的爸爸狼敲了一笔! 

接到处分后不到两个星期,我们在师范第二学年的发展新团员工作开始了。有了处分当然不能入团
了,老A又一次没了指望。 

这一次有七名同学入团,全班只剩下十名非团员了。我想不出用什么话来安慰老A,只说:“这次也有人提名你,只是没有通过。好在还有三年级呢,
\newpage

再争取吧。” 

老A突然暴怒起来,瞪着眼嚷道:“你真幼稚!”过一会儿,他平静下来:“我不想入团了,暂时不想了,毕业后再说吧。没想到,真没想到,你知道我虽然常说“无他’,但一直把希望放在三年级,我想三年的时间人们总会理解我……没想到,真没想到……毕业的时候我二十岁,周岁才十九岁,还有好几
年的时间呢…” 

老A给他爸爸写了一封很长的信,大意是对他爸爸赚学校的钱表示不满。他爸爸没有回信,大约他
想老A还太小太幼稚。 

我们升入三年级后日月如梭。老A不再做红人,也不再怕做黑人,倒也轻松。日月如梭,那么快就过完了秋天,那么快就过完了冬天,那么快就过完了春天,那么快师范生涯的最后一个夏天就到了,我们
临近毕业 

在这期间,班里又有八名同学加入共青团。这
\newpage
样,到我们毕业的时候,全班四十七名同学中就只有两名非团员了:老A和另外一个外号叫“佐罗”的同
学 

在距离毕业还有两个月的日子,老A经历了最
后一场磨难。这磨难是因为刘菲也是为了刘菲。 

学校附近有林荫路,夏天的傍晚,这里满是散步的人们。我和老A也常去,刘菲也常去。相遇时,谁也不理谁,老A尽量做出自然的样子,刘菲则意态
漠然,仿佛什么也没有发生过。 

这天是太阳还没有落山的时候,散步的人还不多,刘菲和几个女伴走在我们前面。忽然迎面鬼使神差地走来一只威风凛凛的大狗,它大概是挣脱了锁链
跑出来的,脖颈上还带着一截断链。 

在一般情况下,狗是怕人的,绝不敢跟人比肩而过。然而这条狗大约是与众不同的厉害而勇敢的狗,而刘菲们又是女孩儿,那条狗高傲地阔步擦过刘菲们的身边,本来是井水不犯河水的样子。但也许是那
\newpage
狗晃动着的毛茸茸的尾巴扫了一下刘菲的裙子,可能还蹈到了她的小腿,让刘菲意识到了脚下狗的存在,
她立刻惊叫起来一边叫一边手忙脚乱地往一旁退让 

那狗一开始让刘菲的突然惊叫吓了一跳,本能地往旁一闪,很惶惑。但刘菲惊慌失措的样子使它受了鼓舞,它停一下,看了刘菲几眼。也许是她那有意做出的弱不禁风的样子太动人,连狗都被打动了,或者是她的大惊小怪让狗感到很有趣,那狗抖数了一下
,意气风发地向她接近。 

我敢说这条狗此时根本没有咬她的意思,公道
地说它只不过是对她比较感兴趣 

可是刘菲这下真的是害怕了,她猛地转身就跑,仓皇至极,边跑边“妈呀妈呀”地瞎叫,声音里满
是哭腔 

那狗见刘菲一跑,大受激励,勇气倍增地追上去。刘菲当然没有狗跑得快,只几步狗便已追上。那狗虽有些愤怒,却也并没有当即下口便咬,它只是将
\newpage
冰凉的湿嘴去吻刘菲的小腿。假如刘菲此时停止一切动作,那狗也就会停下来,至少不会咬她。又假如她此时勇敢一些对狗实施一些有效的攻击,那狗也会知难而退。然而刘菲偏偏只是哭叫挣扎,一味地给那狗以激励。于是那狗终于被激发得斗志昂扬,在刘菲胡乱踢动的丰满雪白的小腿上牛刀小试地咬了一口。霎
时鲜血涔涔,刘菲扑倒在地。 

那狗一开始调戏刘菲时,我和老A是站在一边看热闹的。但是后来刘菲的刺耳尖叫声让老A极为心疼,站立不稳。及至后来演化到刘菲逃跑那狗追上去欲行不轨,老A便一跃而起,不知怎么那么快就从路
边寻到了半块砖头,握在手里十万火急地冲上去。 

老A举着砖头,要砍出去却又怕误伤刘菲,只得奋勇当前。那时刘菲身旁的女伴早已吓得“疯狗疯狗”地乱嚷着四散逃开,近旁的两个男同学也在“疯狗”声中畏缩不前。只有老A瘦长的身影高举着砖头奋不顾身地冲上去。他没有什么战斗经验,那高举砖头大踏步疾奔的姿势并不是什么有效的作战动作,如果没有这种紧急状态的烘托,他的姿势简直是万分滑
\newpage
稽可笑。但,此时,他那夕阳中奋勇的剪影,着实令
人感动 

三十多米的距离,老A眨眼已到跟前。刘菲扑倒在地,老A的传头派不上用场,刻不容缓,他提起
腿向狗踢去。 

若从力量上对比,那狗对付老A这样一个瘦长得杆子一样的人物是不在话下的,但老A那以死相拼的架势实在让这只威风凛凛的雄狗发毛,它仓皇地在老A踢过来的腿上咬了一口后狼狈地逃之天天。老A
百忙中将砖头砸出去,丢在离狗尾巴三米远的地方 

老A小腿剧痛,鲜血涌出,他被咬得比刘菲重多了。他顽强地支持住不倒下,将全身重量移到那条好腿,伸出双手去欲将刘菲拉起,却不知是不灵便还是别的什么原因,伸到一半便停住,看着她的女伴们
过来将她抱起搀扶着远去。 

我跑上去扶住老A,搀他回校。老A人缘很好,听说他挨了狗咬很多人都来看他,大多是男生。虽
\newpage
然天快黑了,还是有好几个人踊跃去医院给他买狂犬疫苗。可是两个小时后买疫苗的人空手而归,他们跑遍了市里所有的大小医院和卫生防疫站,都没有狂犬疫苗。那时期狂犬猖獗,狂犬疫苗早已售光了。我们面临了一个极为严峻的现实:如果那狗是狂犬,那么
老A… 

这真是一个沉重的打击。那条狗主动咬人,十有八成是疯狗老A失魂落魄,我也是心往下沉,好像我的好朋友已命在俄顷。这一夜都睡不好觉,第二天阜早起来,我对老A说我和几个人去附近的大城市撞撞运气。老A说也只好如此了,他嘱咐我们一定要买
双份。我明白他的意思。 

正这时,几辆摩托车呼啸着开进学校。原来是老A在校外交的几个朋友听说老A被咬赶来看他。我们简单讲述了情况。其中一个首脑性的人物说去大城市没用,大城市不养狗更没有狂犬疫苗,要去就去农村,农村狗多,想必可以买到疫苗。简单几句话,说得我们恍然大悟。不再耽搁,几辆摩托车呼啸着冲出了校门,分头向农村扑去。我和几个同学没摩托,不
\newpage
能跑农村,商量了一下,还是去大城市,去总归是比
不去强。 

紧张是不用说的,因为狂犬疫苗必须在咬后二
十四小时内打下第一针! 

下午,我们几个去城市的同学都两手空空地返回。后来,有几辆摩托也相继垂头丧气地回来了。眼看就要过二十四小时了,我们都已绝望。终于最后一辆摩托赶回来了,后座上载着一个人,那人下了车从挎包里小心地捧出一盒针剂,说是一手交钱一手交货。待他说出货的价钱着实让人吃了一惊:一千元!而
那盒针剂的正常零售价应为三元八角 

一旁的首脑火冒三丈,说我打你个丫挺的而那“丫挺的”只一笑,说那我就摔掉那最后一辆摩托说别吵了,已经讲定的价钱,咱们的老A还不直一千元
吗? 

老A说拿药来,两盒,给你两千元但卖药人只拿得出一盒,他自己也很遗憾。老A转向那最后一辆
\newpage
摩托,气急败坏地吼道:“我怎么嘱咐你的?!你怕
我拿不起两千元吗?” 

最后一辆摩托很委屈:“他只有这最后一盒,这方圆几百里恐怕就这最后一盒了,要不怎么会卖到一千元呢?老A,你一盒也够用了……老A呆了呆,颓然坐在床铺上。老A把自己所有的钱都拿出来,只
三百多一点儿,还亏将近 

七百。大家便凑,临近毕业了,谁的钱也不多,是二十多人凑的。老A声明一定还大家,但大家谁
也没有报数目,那意思是不用还。 

时间不能再等,老A去打针。他只要我一个人
陪他去医务室别人也要陪,他坚决不许。 

走到半路,在一个僻静处,老A停住了说我有
一件事要你去办你一定要替我办。 

我说你说吧老A说你去将这盒药给刘菲,快去

\newpage
,时间不多了,别说是我给的。 

什么?老A!我愣住了。从他只要我一个人陪他去医务室,我就预感到这里面有问题,因为以他的性情是喜欢众人陪着的。我沉默一下说老A,我完全理解你的心情,可是你想过没有,这是生死攸关的事
情,生死攸关…… 

我想过。他说“但是我决不会替你做,我也决不允许你这样做!”我知道老A已昏了头,没法跟他
讲道理。 

“你不替我送,我自己去送。”“不行!”老A说:“不然我也不会自己打。”我说:“不存在不然!”老A不再理我,绕过我径自走,我第一次看到他这么坚决。我死死拉住他,把他往医务室拖。老A,你别犯傻,老A你听我的话老A生死攸关生死攸关
哪,老A你疯啦! 

“放开!”老A暴躁地嚷道,“还有二十分钟
,把她耽误了我不会饶你的!” 

\newpage


我不放! 

你放开,你不是说理解我吗?你不是说理解我
吗!你放开,求你了! 


我不放!! 


还有十五分钟了,求你了求你了 


老A猛地趁我不备狂怒地挣脱了我,一拳打在我的胸口上,狼狠骂道:“还有十分钟了!去你的!
” 

我被他打得梧住胸口蹲在地上动弹不得,从他的眼神里我看出来再也拦不住他。我只得看着他的背
影一瘸一瘸地向女生宿舍区走去。 

老A,老A呀,这可是你自愿向地狱走去了一
不过,他也许是走向天堂…… 

\newpage

不知道老A是怎么将这盒针剂交给刘菲的,反
正后来是刘菲打了而且没耽误一分钟时间。 

但老A也没有死。原因很简单,那条狗不是疯狗。这样的结局实在很幸运,也似乎有点儿令人扫兴。然而确实只是后来人们才知道那条狗不是疯狗,因
为老A没有死,所以人们知道那条狗不是疯狗。 

老A直到毕业没有再理我,毕业后连他分配在哪里我都不知道。我知道他一定认为我对不起他,也
一定认为他对不起我. 

但后来我还是知道了老A的消息,那时我们已经毕业两年了说他分配在乡下一所偏僻的小学校里,是他自己要求分配到那里的。说他干得不错,最近还了团。

\end{document}
