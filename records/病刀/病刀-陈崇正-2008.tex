\documentclass{article}
\usepackage[utf8]{inputenc}
\usepackage{ctex}

\title{病刀\footnote{Click to View:\url{https://web.archive.org/web/20221009011754/https://www.meidekan.com/meiwen/2077274.html}}}
\author{陈崇正}
\date{2008-05-15}

% \setCJKmainfont[BoldFont = Noto Sans CJK SC]{Noto Serif CJK SC}
% \setCJKsansfont{Noto Sans CJK SC}
% \setCJKfamilyfont{zhsong}{Noto Serif CJK SC}
% \setCJKfamilyfont{zhhei}{Noto Sans CJK SC}
% \setlength\parindent{0pt}

\begin{document}
\CJKfamily{zhkai}

\maketitle


\Large


我等桃花开,你等桃花落 


春风来,北风去,风把风吹老 


你在斜阳里,你在斜阳外 


你在斜阳埋骨处,相思莫相扰 


一 

这是十二指街今年的第一场雨。雨丝如千万个修长的指头轻轻触摸着大地。集结在屋檐上的雨水慢悠悠滴落在街面的青石上,也有一点懒洋洋。( 文章阅读网:www.meidekan.com )

\newpage
 

黄昏时分,天变得更加阴沉,街道的尽头已经被吞在黑色之中,一片死寂。这样湿漉漉的雨天,谁还愿意出门,此时在家里温二两黄酒对窗赏雨,最恰
当不过。 

假如你是个细心的人,就会发现今天十二指街的安静,不单单因为雨——马铁匠今天没有像往常一样发出叮叮当当的打铁声。打铁铺没有拉风箱的声音
,也看不见红色的炉火。 

打铁铺在一棵大榕树底下。这是十二指街最大的一棵树,树枝上长长的树根垂到打铁铺屋后的池塘
里。 

一只小鸭从池塘上了岸以后,打铁铺的门终于吱呀一声开了一条缝,一个男孩黑色的头探了出来,一对乌溜溜的眼睛骨碌一转,头又缩进去,门也关上了。过了不久,这扇破落的小木门又打开了,小男孩
闪身而出。 

\newpage

小男孩缩着脑袋,双手护在胸口,走在雨中,走在湿漉漉的青石板路上。他在街角一所院落门前停住脚步,踮起脚尖,伸手去扣那一对大门环。“嘭!嘭!嘭!”等了很久,没人答应。天这么阴沉,安静总是令人害怕。小男孩站到门槛上去,重新敲响门环。这时,里面传来脚步声。门开了,一只手伸出来,
把小男孩一把提了进去。 

小男孩一直低着头,缩着脖子,映入他眼帘的是一双黑色的布鞋,鞋帮处打了一个蝴蝶状的补丁,右脚的白袜子上破了一个黄豆大小的洞。小男孩一直跟着这双黑布鞋,穿过了院子,这时前面的黑布鞋换成了皮靴,便听到有人叫他的名字,才抬起了头。“
黑子,你不去拉风箱,跑来敲我的门做什么?” 

他看到穿黑布鞋的婆婆已经走开了,站在他面前的是一位书生。是樊秀才,没错,就是樊秀才!于是他将藏在胸口破衫底下的那只手拿了出来,将一锭金子交到了樊秀才手里:“樊先生,我爹要您去!”


\newpage

“你爹怎么了?找我喝酒吗?” 

“我爹快死了,”黑子吸了吸鼻子,“我爹说
,让你带上笔墨。” 

“生病了应该找大夫啊,怎么……我懂了。”

樊先生摸了摸黑子的后脑勺,点了点头。他对内堂叫道:“韩三婆,让小玉把我的画笔带上,到马
铁匠家去!” 

内堂应了一声,黑子又看到了黑布鞋在走动,
韩三婆穿过大厅,朝后院走去。 


二 

马铁匠一匹骆驼一样卧在床上。往日团团隆起的肌肉,现在全部舒展开,像被泡过的茶叶。樊秀才侧目凝神正在为他把脉,一手轻轻揭开他胸口的粗布衫,一个黑色的掌印赫然入眼。樊秀才又让马铁匠翻身,查看后背,果然,那个掌印穿胸而过,在后背也

\newpage
留下了一个红色的掌印。 

马铁匠轻轻一笑:“樊秀才,我是让你来画画的,咳咳,我是活不过今夜了,现在,你相信了吧。
” 


樊秀才点了点头,长长叹了一口气。 

樊秀才又问:“这是黑砂掌,还是朱砂掌?”

马铁匠说:“不是黑砂掌,也不是朱砂掌。这两路掌法虽然沉着阴毒,但只能碎胸,力道不足穿胸
。” 


“那究竟是何种掌法?” 

“你是铁笔世家,竟不知道这露掌法?这是宁波府花红紫的桃花斜阳掌。黑掌有如黑夜,红掌有如
桃花,一夜花落人亡,悄无声息。” 


樊秀才脸上掠过一丝不安:“仇家?” 

\newpage


“爱恨情仇,谁说得明白!” 

这时,铁匠铺的门被推开,一个女孩的声音大叫:“爹,这地方又黑又臭,你怎么不把铁匠请到我
们家里去,况且我那支铁帘钩也得修理一下。” 

在小玉背后,韩三婆提着樊秀才平时用的布袋,站在那里一动不动,听小玉这么说,便轻声说:“
小玉,不得胡闹,过去问候马叔叔!” 

小玉却不搭理韩三婆,转头向背后的黑子,问
:“你爹怎么了?” 


“我爹快死了。” 

小玉这才对樊秀才说:“爹,您就救救铁匠吧,我们家不是有很多膏药么,救救他,这十二指街要
是没有了铁匠,那以后的锅鼎刀剑坏了找谁呀!” 

樊秀才仍然低头叹息,让小玉把文房四宝铺开,再吩咐小玉研墨。他将那锭金子放在铁匠枕边:“
\newpage
街坊之间,画一幅画像,举手之劳,哪里需要什么报
酬!黑子还小,需要钱。” 

“樊先生,我并不需要什么遗像,请先生来,是想先生破例一次。我有一套祖传刀法,名为病刀,
想请先生写录下来。” 


樊秀才面有难色。 

“我知道自从尊夫人因武林纠纷去世之后,先生就封笔,发誓不会为江湖上的武功秘笈再执画笔,所以这些年过去了,江湖上只知道有武功秘笈,却不知道武功秘笈背后的铁笔世家。没有铁笔,哪来武功秘笈?八年前,你我结伴隐居十二指街,彼此心照。

樊秀才目光也变得悠远:“人世无常,谁能料到当年李自成马前第一勇将,竟然成了一名铁匠,最阳刚之人却死在江湖最阴毒的武功之手,唉——罢了,小玉,将那支血貂铁笔取出来,你和韩三婆回避吧

“无须回避,看着吧,让孩子们都看一看。黑
\newpage

子,取刀来!” 

黑子走到熄灭的火炉前,手一伸,在炉灰中一抓,一把黑色的大刀就把拉出来。黑子扛在肩上,吃
力地抬到他爹床前,咣当一声放下。 


马铁匠又喝道:“黑子,酒!” 


一坛酒也搬到马铁匠面前。 

马铁匠举起酒,一饮而尽,脸上红光闪闪。只听他一阵猛烈的喘息,哼了一声,站了起来,身体摇摆了一下,方才站稳。樊秀才面露担忧的之色,跨出两步。马铁匠举起一只手,示意他没有事,自己能行
。 

马铁匠又向前迈出一步,深吸了一口气,闭目运功,双手一抬一压,脚下的小煤块竟然嘶啦作响。
樊秀才不禁脱口而出:“太极内功!” 

马铁匠调息完毕,一伸手,刀已在手。樊秀才
\newpage
也叫一声好,手中画笔没入墨水之中。马铁匠的刀稳稳向前推出,慢慢一招一招使出,步法中似乎摇摇晃晃,都在欲倒未倒之际换步:“樊先生,这第一式,
叫病入膏肓。” 

一听有人给自己的招式取这样的名字,小玉不禁扑哧一声笑了。樊秀才出言制止:“莫笑,你看不出这刀法的奥妙。不偏不倚,也邪也正,正是阴阳开
合中打开的一扇门户。” 

只见马铁匠的刀法越使越歪,樊秀才却是眼如闪电,笔走龙蛇,行云流水。小屋之中,烛影摇曳,铁匠的刀稳如泰山,秀才的笔疾似猿猴,一快一慢,
看得黑子目瞪口呆。 

小玉在旁边看着,不禁问:“爹,这路刀法这
么难看,这么慢,能杀人么?” 

黑子听小玉这么说,也忍不住回应:“你懂什么?你不知道在把刀杀了多少人,多少人听到这套刀

\newpage
法就闻风丧胆!” 

就在两个小孩喋喋不休的争论中,马铁匠已经把二十六式刀法演练了三遍,三遍之中,似乎相同,又似乎不同。马铁匠边练边说,心法口诀,尽数让樊
秀才记了下来。 


刀停笔落,二人相视哈哈大笑。 

樊秀才说:“快哉!八年来一直给人画遗像,都画腻了,还是这笔下的武功让人痛快。”樊秀才又神色严峻,一字一顿地说:“马将军,这套绝世的刀
法,为什么取如此难听的名字?” 

“八年了,八年来第一声马将军。这刀法原来的名字太惹人注目,病刀不病,病在朱门食肉之人!

樊秀才点了点头,在封面上写上“病刀”二个字。此本刀谱于这一夜写就,一直流传,几经辗转,一直到林则徐虎门硝烟,不知何人将之垫在鸦片箱底,被徐公付之一炬,令人叹惋。此前此后,皆无人知刀谱为何人所创,如天底下的许多拳经剑谱一样,只
\newpage

知少林武当,却不知绘者何人。 

闲话休提,却说马铁匠将病刀刀法一口气练了三遍之后,脸上的红光已经渐渐退去,只听得他的呼吸又重新变得安静了,像一滴水滴入大海。 #p#
副标题#e# 


三 

这时夜风吹着雨滴从窗口散入屋内,颇有一阵寒意。马铁匠斜斜靠在一条木凳上,他已经没有任何力气了,他看着樊秀才,笑了一下:“其实我知道是你。”樊秀才一脸诧异,突然之间他明白了,马铁匠
的眼光正越过她,投射在韩三婆身上。 

马铁匠的眼睛已经有一点迷离,他打了一个酒嗝,看上去非常疲惫,他依旧喃喃地说着:“我认得你的鞋,从你进门我就知道是你,多少年了,你还是用蝴蝶为鞋打补丁,你就不能变一下么?你说,即使我死的时候,你也一定不会回来,你终于等到了,你

\newpage
却也……终于还是来了。” 

韩三婆还是站在那里,并不挪步,她望着窗檐上的雨滴,淡淡地说:“是她下的毒手,我还能说些什么呢,妻子打死了丈夫,都是你们一家人的事,与
我何干?” 


“她也是一时失手……” 

“什么叫一时失手?”韩三婆终于提高了声调,“她就是想把黑子带走,她就是一心想要黑子去继承她花红紫的九指神教,你还为她辩护,好啊!我这个局外人,来于不来,又有什么关系呢?”韩三婆竟
然呜咽起来。 

“死之前,能把病刀传下来,传给黑子,我就心满意足了,更何况,还能见到你,三十年了,你原
来是我的邻居!” 

“每一夜,我都是听着你的打铁声睡去的,现在年纪大了,没有那个狐狸精年轻漂亮,现在年纪大就会失眠,我就整夜整夜地听,听你一下又一下地敲
\newpage
打着红色的铁块,听你磨刀的声音,就如当年在江边那个醉酒的夜晚一样。三十年了,我说三十年不见面,果然就三十年不见了,想想,我真后悔年轻时候发
的毒誓,年轻真好,还可以犯错,还可以犯错!” 

樊秀才一看情况发生了转变,拉着小玉和黑子,准备出去,但黑子不愿意,他站在那里也如一块黑
色的木炭。 

“你们都不要走,你们给他收尸吧,”韩三婆用手帕擦了一下眼泪,“该走的人是我。我知道出了这个门,暴露了身份,也许就活不过明天了,那个贱人一定会来杀我的,我一直都在等她。”说罢,韩三婆退开那扇破旧的木门,依旧用手帕捂着嘴巴,小跑
着出去,就仿佛是被黑夜吸了出去。 

马铁匠突然不只哪来的力气,坐了起来,吼了一声:“花红紫,你出来,你杀不了我,能杀我的只有我自己!”马铁匠的手往地上的刀把一拍,那把大刀就如一条黑色的鱼从地上跳了起来,划了一道优美

\newpage
的弧线,跳入马铁匠的胸膛,鲜血也溅了起来。 


四 


安静。良久。 

樊秀才抚着黑子的头,说:“黑子,跟我回家
吧。” 


黑子摇头。 


“明天我们再为你爹办丧事。” 



“你先同小玉姐姐一起回家。” 


“这屋子又黑又脏,你爹又……你不能待在这
屋里。” 


“这是你爹的刀谱,我给你叠好,长大了你要
\newpage
好好练,为你爹报仇。”樊秀才这么一说,也知道不对,为他爹报仇,难道要他去杀他娘么?果然是病刀,连情和爱,也都是病的,樊秀才不禁想起了寒冬里
的梅花。 


黑子还是摇头。 


樊秀才对小玉说:“这孩子怕是吓傻了。” 

樊秀才想伸手去提黑子,但黑子避开了。他抬起头,对樊秀才说:“樊先生,您出去吧!我想陪着
我爹。” 


“刀谱要不要我为你先带着?” 

黑子没有回答,他移动沉重的脚步,爬上了床,侧身躺下,蜷缩着身子,两眼愣愣地望着前方,一
颗颗泪珠滑过眼角,渗进枕头里。 

樊先生长叹一声,收拾了刀谱笔墨,带着小玉走入黑夜之中。四周又重新安静了下来,烛光变得越
\newpage

来越暗,越来越模糊。黑子终于沉沉地睡去。 

待他慢慢睁开眼睛的时候,蜡烛不知什么时候已经被风吹灭了,打铁铺中一片黑暗。他爹的尸体还躺在地上,保持着他倒下的那个姿势,这使黑子明白睡去原来并不能使眼前的一切发生有所改变,心头一
阵悲怆。 

但就在这时,床头暗角处传出一个女子悠悠地
一声叹息:唉—— 

黑子这一惊非同小可,嘭地一声坐了起来,汗毛直竖。四周又说不出的安静,黑子努力地睁大眼睛,但除了黑暗,他什么都没有看到。只听到自己的心
跳。黑子一拉被子,缩到了床角。 


“唉——”又是一声若有若无的叹息。 


“你是谁?” 

这时黑影浮动,像月光下婆娑的柳树,一个黑
\newpage

衣女子站在床前。她和黑夜一样黑。 

女子说:“如果有人伤害了你的亲人,你应该
怎么做?” 

黑子摇了摇头:“我不知道,我爹没有教我。


“你不恨么?” 


黑子摇头:“我不知道。” 

女子冷冷一笑:“孩子,你应该学会仇恨,就
如你要学会爱一样。” 

静了一会儿,女子又说:“你最想做的是什么
?你最想要的是什么?” 


黑子依然抱着被子。 


“你应该有一点欲望吧?” 

\newpage


“我想死。” 

女子愣住了。良久才说:“那你为什么不死?

“我怕痛。能不能死,但不害怕,也不痛?”

“可以,你听我的话,我就让你死。不害怕,
也不痛。” 


“嗯。”黑子连连点头。 

月光下,黑衣女子从背后的包袱中取出一架小
巧的玉琴,盘膝而坐,对着马铁匠,扶琴而歌: 


我等桃花开,你等桃花落 




琴音缭绕,缠绵悱恻,一缕伤感涌上了黑子的心头,刚刚被恐惧逼走的伤感,一瞬间全都回来了。
\newpage
想起父亲往日种种情状,不由百感交集,潸然泪下。

一曲《风把风吹老》终了,黑衣女子将玉琴重新放回包袱中,拉着黑子的手,从窗户中一跃而出。

这一夜,铁匠铺火光冲天。第二天一早,雨势加大,只有那棵老榕树依然立在池塘边,一半枝叶随
大火变得碳黑。 


五 


“娘,带我去江湖吧!” 


“好,你先把樊秀才给杀了!” 


“那小玉呢?” 


“娶她做老婆。” 


“但她比我大。” 

\newpage


“你爹原来的老婆,也比他大。” 


“你是说韩三婆吗?你杀了她了吗?” 

“没有。我一直羡慕她,但我也讨厌她,所以
,记住,你讨厌一个人,就让她活下来。” 


“好。” 

“黑子,你走得太慢,别再背这把病刀了,太
重。” 

“娘,等我长大了,我用这把刀杀了你,因为不讨厌你。”

\end{document}
