\documentclass{article}
\usepackage[utf8]{inputenc}
\usepackage{ctex}

\title{贝尔纳·斯蒂格勒:思想的行者\footnote{Click to View:\url{https://web.archive.org/web/20220629020755/https://ptext.nju.edu.cn/7f/1a/c13164a491290/page.htm}}}
\author{张一兵}
\date{2020-08-22}

% \setCJKmainfont[BoldFont = Noto Sans CJK SC]{Noto Serif CJK SC}
% \setCJKsansfont{Noto Sans CJK SC}
% \setCJKfamilyfont{zhsong}{Noto Serif CJK SC}
% \setCJKfamilyfont{zhhei}{Noto Sans CJK SC}
% \setlength\parindent{0pt}

\begin{document}
\CJKfamily{zhkai}

\maketitle


\Large

法国哲学家贝尔纳· 斯蒂格勒(Bernard Stiegler)于当地时间8月6日自杀辞世。他经历传奇,曾因抢劫银行入狱,在监狱中开始哲学研究;他以《技术与时间》等著作重新讨论了技术在人类本性中的作用和地位,是当代西方社会
批判理论阵营中最重要的代表人物。 

我与斯蒂格勒的相遇,缘起于2013年他的前妻马拉布到访南京大学,这位当代法国著名女哲学家也是德里达的弟子,那天,她在讲课间隙到我办公室里喝水,看到我与德里达2001年的合影,瞬间眼泪就下来了。也是那一天,我从她口中知道了她的前夫竟然是斯蒂格勒。当时,我着实大吃一惊,因为不少年前读到斯蒂格勒的《技术与时间》(La Technique et le Temps),曾
\newpage
经深为海德格尔在技术存在论中的变形而惊叹过。那天的午宴,成了我向她追问斯蒂格勒近期思想的讨论
。 

马拉布哲学的关键词是存在和思想的可塑性,她在欧美大陆哲学界的影响,是她将形而上学的目光投射到了遗传基因和脑科学。应该是德里达、海德格尔和对技术的共同关注,使她与斯蒂格勒走到了一起。我们正在翻译的一本她的名著,就是《海德格尔的变化》。在斯蒂格勒《技术与时间》第一卷的序言里,我们可以看到这样的文字:“我们被柔情一同系于严谨的哲学追求,这种追求使我们聚合,同样也带来一种争斗的气氛。”显然,他们的爱情基础是对哲学的共同兴趣,可斯蒂格勒也承认,他与同样好胜的马拉布之间存在“一种既富创造性而又具有危险性的竞争”,可能,也是这种竞争让他们最后分开。马拉布似乎有些遗憾地告诉我,“他现在的思想变得十分激进,现在写的书全是批判当代资本主义新型技术统治的东西”。这下子,却让我这个正在研究欧洲激进思
潮的人对斯蒂格勒更加有了兴趣。 

\newpage


一、在当囚犯的时候爱上了哲学 

2015年,斯蒂格勒应邀第一次访问南京大学,我们一起开设了工作坊。这是我们相互了解的开始。那时候,他把我当成一个研究马克思的官方化的怪物,说话时总是小心翼翼;而我则将他视作研究海德格尔技术观的德里达的学生,所以,总是想从存在论和解构理论的构境接近他。这样,我们之间好奇性试探多于实质性的交流。
说起斯蒂格勒研究海德格尔和遭遇德里达,一定会先碰到那个流行的段子,即德里达到监狱中探望因抢劫银行而入狱的斯蒂格勒,传说前者发现这个写信给他并声称研究海德格尔的囚犯竟然真的孺子可教,于是答应作为他的博士论文指导老师。一次吃饭间,我开玩笑似的问斯蒂格勒,为什么会想起抢银行?他毫不掩饰地说:“年轻气盛,只是想把属于自己的钱拿回来。”那是一个特殊的时代,即后1968时代。斯蒂格勒说,1968年的五月风暴“只能作为反抗,而不是革命”。冲上街头,青年人只是在一种无目的的激进和拒绝性的批判幻境中表达不满,但所有人都不知道为了什么起来闹事。所以一旦充满激情
\newpage
的造反狂欢过去,大家都非常失落。寻求新的刺激和宣泄伪境中,就有犯罪。不过,斯蒂格勒去抢银行还
有具体的生活背景。 

1952年4月1日,斯蒂格勒出生于法国法兰西岛大区埃松省的伊韦特河畔维勒邦(Villebon-sur-Yvette)一个普通工薪阶层的家庭,爸爸是一名电子工程师,妈妈则是银行员工,家境平实。斯蒂格勒小的时候,在法国的萨塞勒(Sarcelles)长大。1968年巴黎爆发“红色五月风暴”时,还在读高中二年级的斯蒂格勒参加了街上的路障活动,并在五月风暴后加入法国共产党。1969年,斯蒂格勒高中毕业之后,开始在法国电影自由学院(CLCF)学习做导演助理,但他并没有完成学业。因为在中学毕业前,只有16岁的他就已经在外打工。似乎,因为不喜欢某种体制内的固定职业,所以他一直到处打零工,做过农场工人、酒保等工作。1972年,斯蒂格勒在农场工作,存下一些钱,买下了一间小餐馆。头脑聪明的斯蒂格勒很快就挖到了自己的第一桶金,于是心大的他将原来的小饭店卖了出去,转而买下了一间更赚钱的爵士乐
\newpage
酒吧。欧洲人常去的酒吧大多数只在晚上营业,这种场所少不了三教九流的人,这当然也就成了警察经常盯着的地方。一次,警察为了指证一名常来酒吧的顾客,要求酒吧老板斯蒂格勒出面做证人,被他当场拒绝。于是,警察就找借口查封了斯蒂格勒的店,这也就断了斯蒂格勒的生路。一气之下,借着母亲熟悉银行内部情况的优势,他一连抢劫了三家银行。依自己的说法,这一得手,就像上瘾般地无法停下来,一直到第四次作案,正好遇上在街上巡逻的警察,当场被抓个现行。斯蒂格勒自己说,按他的罪行应该坐15年的牢,但因为找到一位很好的律师,他只被判入狱8年。于是,1978~1983年,斯蒂格勒在图卢兹的圣-米歇尔监狱和米雷看守所服刑,并于第五
年提前出狱。 

斯蒂格勒自己说,他真的是在当囚犯的时候爱上了哲学。一般的人,进了监狱服刑,从心理场境和生存态度上就会陷入崩溃情境和下行状态。斯蒂格勒也有过一个类似的短暂时刻,他说自己在刚被捕的两个星期,是很绝望的,就像很多人一样,也想到过自杀,但他很快就振作起来了。这种恢复中,就包含着
\newpage
对哲学的向往。一个朋友听说斯蒂格勒想学习哲学,就开始为他送一些哲学书。后来,他在监狱里通过函授的方式进入图卢兹大学学习哲学。依斯蒂格勒自己的回忆,他每天早上起床的第一件事是读马拉美的诗,然后收听广播,再开始阅读胡塞尔、海德格尔和西蒙栋等人的哲学文献并思考。马拉美的诗已经是从抽象的象征和意象对粗俗不堪的现实所进行的悬置,而哲学本身就是远离生活的形而上学,所以对这一段理应黑暗不堪的岁月,斯蒂格勒却生成了另一种常人不能进入的纯粹思想构境状态。恰恰是在这种封闭和安静的状态下,他体悟到了现象学所说的悬置构境,即把常人的那种自明性的常识用括号隔出,以造成一种纯粹地回到事物本身的初遇。监狱中的隔世环境倒是一种天然的隔绝,这正好制造了一种海德格尔所说的周围上手世界的断裂,这有可能让斯蒂格勒原初地回到哲学思想的本真构境中。多年以后,斯蒂格勒真的以《技术与时间》作为博士论文,在德里达的指导下
,通过了巴黎高等社会科学院博士学位的申请。 


二、看到支配了周围世界的无所不在的技术 

\newpage

2016年,斯蒂格勒如约再次来到南京大学,这一次,我们聘任他做了马克思主义社会理论中心的兼职教授,并且签了第一个三年合作协议,约定每年他来南京大学开一个英文课程。所以在接下来的三年中,斯蒂格勒都在4~5月之间到南京为南大的学生开设研讨课,几年下来,他分别开设了“从《德意志意识形态》到《自然辩证法》——从人类纪的视角阅读马克思和恩格斯”“在后真理时代阅读柏拉图”“超批判与超唯物主义认识论”等课程,授课期间,他也不定期地面向全校学生开一些关于自己最新研究的报告会,比如关于“人类纪社会的来临”“数字化资本主义批判”等主题。也因为时间多了起来,有时我也会陪他在南京城到处看一下,品尝南京百姓的小吃。我注意到,在生活中的斯蒂格勒是十分随和和平易近人的,他从来不把自己伪饰为大师,一个简单的玩笑,他会笑得前仰后合,一个被焚毁的大报恩寺遗迹,会让喜欢艺术的斯蒂格勒沉浸于那些复建的东方式建筑和逝去的生活情境之中。那时,他是多么地开心。
慢慢地,我们就逐渐熟悉了起来,我与斯蒂格勒之间的交流变得随意和深入起来。我们在一起谈论
\newpage
我们共同喜欢的海德格尔、德里达和马克思,因为观点上的不同,有时我们也会争辩,好像,长几岁的斯蒂格勒常常让着我的强势。面对我不断地强调文本细节和学术项目很困扰精准性,他友善地对我说:“我是一个思想上的行者,而不是一个大学的学院派教授。”当然,我们也会聊共同爱好的摄影和健身,与我跑到健身房撸铁不同,斯蒂格勒主要是骑山地车锻炼,并且,他悄悄告诉我:“我就是在骑车的过程中,通过头戴式耳机话筒口述自己的新书,然后再由夫人录入电脑。”这是他几乎一年一本新书的秘密。这期间,我们还就共同关心的学术问题进行过多次学术对话,比如“技术、知识与批判”“第三持存与非物质劳动”“人类纪的熵、负熵和熵增”,以及“认识论研究前沿”等。我们的思想碰撞彼此影响,斯蒂格勒开始关注我所研究的马克思的文本,并将其扩展为自己的讲课内容,我正在进行的认识论研究专题则成为他将要开始的《技术与时间》新一卷的主题;而我则开始系统研究斯蒂格勒的主要代表作《技术与时间》三卷本,并在他随时的面对面的直接答疑中完成了《
斯蒂格勒〈技术与时间〉》一书。 

\newpage

在我与斯蒂格勒的交流中,他从来没有从正面回答那个关于德里达探监的传说,他只是说德里达算是指导博士论文的老师,我也能感觉得到,斯蒂格勒并不愿意人们仅仅将他看成德里达的弟子,似乎更想以一个原创性的当代思想家自居。谈起他在德里达指导下完成的博士论文《技术与时间》,他承认最先被海德格尔的《存在与时间》一书所吸引,在海德格尔看到存在者背后的存在的地方,斯蒂格勒看到了今天
已经支配了周围世界的无所不在的技术。 

《技术与时间》第一卷的副标题为“埃庇米修斯的过失”(faute d’ Épiméthée),这个构境背景缘起于人们不太说起的一个关于埃庇米修斯的故事:在马克思所喜欢的希腊神话人物普罗米修斯为人类盗火的英雄事迹背后,隐藏着他兄弟埃庇米修斯的一桩“劣迹”,他在负责分配所有生物各有种系属性时竟然忘记了给人类留下一种天生的专长,如鱼能游水,鸟可飞翔,猴会上树,马擅奔跑,等等,可人却什么都不会。他的这一“遗忘过失”造成了人类存在的一种在生物种系遗传中的“原始性缺陷”,从而使人与其他动物相比,初生下来却无一
\newpage
技可依,倒成了一个没有任何可遗传的种系发生的生存技能的存在物。斯蒂格勒认为,也由于埃庇米修斯的这种“滞后”(相对于普罗米修斯的“先行”),人类才发明了技术这一后种系生成(épiphylogénéyique)和全新非生物的外部义肢(prothèse)持存。这里的后种系生成,是指与动物物种天生具有的生存能力相异的社会历史赋形的人类独有的生存能力,这也就是说,人与动物生存最大的不同,是人通过自身外部的广义技术(包括文字)发展来充抵自己的不足。这当然只是一个隐喻。
 

远一些,斯蒂格勒是从人类历史发端之初发的燧石打火开始讨论体外工具的持存,在此,他援引了德里达的“延异”概念,动物直接吃树上的果实和其他生物的身体,而人则会通过外部持存工具制作和延迟的熟食来创造一种新的存在方式。近一些,斯蒂格勒会将我们手中的音乐CD或者手机中的硬盘看作技术持存的最新体现,他把胡塞尔在《内在时间意识的现象学》中的时间结构观当作逻辑踏板,比如,现在我们从音乐厅现场听到由鲁宾斯坦演奏的肖邦《夜曲
\newpage
》第一首的钢琴曲,演奏者弹奏琴键发出的每一个音响发声之后都会消失,但在我们的听觉体验中,当下一个音符弹奏响起时,上一个已经消失的音符仍然滞留在我们的脑海中,甚至这种遗存的音符还会进一步回溯到一个已经消失的音响链,这种不断消失和在场的音响踪迹在我们的主动音乐构境中连续起来,才让我们听到完整的肖邦经典杰作的优美旋律。在胡塞尔那里,他将这个由主体在现场刚刚听到的正在消逝的音响指认为主体体验中的初始持存,也叫第一持存或原生持存。这是音响在人体内部的直接在场。可是当我们在音乐会结束后,在脑海中再回响起这首《夜曲》音乐的时候,它已经不是当下发生的听觉经验,而已经是对我们过去音乐记忆的重新激活,胡塞尔将其指认为“重新回忆中的第二持存”。但是,胡塞尔简单排除了主体从没有直接体验过的“过去的踪迹”,比如那时刚刚开始出现的音乐唱片,它们并不是我们曾经经历过的过去(体验),而是作为我们体外的后种系生成的增补存在的义肢性技术。我们今天从CD或者硬盘中重新播放的《夜曲》,既不是主体直接体验的原生记忆,也不是主体回忆自己直接体验的间接发生的第二记忆,而是一种外在于主体体验的第三记
\newpage
忆。斯蒂格勒说,这正是胡塞尔忽视掉的东西。他极其兴奋地将这种第三记忆命名为“第三持存”。这也是他自己全部技术哲学的重要基础。斯蒂格勒认为,海德格尔也忽视了这个作为支配了我们存在本质的作
为第三持存的外部技术。 


三、对现实生活的批判性反思 

在提出自己原创性的第三持存论之后,斯蒂格勒迅速转移到对现实生活的批判性反思中来。在斯蒂格勒看来,今天资本主义的数字化生存中,许多重要的存在构序都来自于最基本的电子化第三持存载体的深刻变化,斯蒂格勒试图从现象学和存在论的视角剖析这些技术义肢的本质,以铸就对当代数字化资本主义批判的全新理论基础。在斯蒂格勒看来,今天的数字化资本主义彻底改变了传统社会生活的构式方式和基本质性,模拟-数字化网络信息技术构序所生成的先天现实综合筑模,造成了存在本身的脱与境化和光速急迫的突现特征,这种对此在在世的非现实重新构序,远程登录的在场性,同时消解了空间意义上的领土和现实关系中人的在场状态。今天网络虚拟现实中
\newpage
的义肢性是一个全新的问题。主体存在本身被义肢化了!在我们每从智能手机上的微信登录中,主体在场已经不是真的此在在场,而是一个电子化的虚拟主体以网络方式在场,数字化构序存在替代了生命负熵。海德格尔那个“此时”,“在这儿”全部被延异化了。我们明明坐在父母亲的饭桌上,却低头于自己的智能手机,这是可怕的在场中的不在场。一个网上界面上的两个马甲的相遇,可能是美国东部时间和中国北京时间的构序的同时性,也会是两个完全异地的电子持存共在。在这种新型的虚拟网络存在中,形而上学迷失了什么?存在本身迷失了什么?如果在康德那里,先天观念综合构架给予了我们在时空中被座架的现象直观,那么,由“影像工业、远程在场工业和虚拟现实工业”构序起来的他性综合,也像一种先天构架
构序出一个虚拟时空中全新的直观世界。 

其实,斯蒂格勒的观点并不只是理论假设,而是发生在我们身边的现实。我们从醒来到睡去的所有清醒的时候,通过智能手机终端和电脑屏幕,可以在任何一个时间和空间中登录,不同的匿名主体可同时在北京时间和伦敦时间的异地在场,全屏的拟真影像
\newpage
存在和内爆的大量信息,成为我们直观和知识世界的先天综合构架,在我们遭遇世界之前,这一已经无法摆脱的数字化先天综合已经通过自动整合座架了我们可能看到、听到和触到的世界和一切现象。康德那个“自然以一定的形式向我们呈现”的观点,现在被改写成存在以网络上的智能手机和电脑屏幕的构序形式向我们呈现。斯蒂格勒说,今天的资本正是利用了数字化网络这一新的外部持存中的主体异化,让我们成为新型的无脑儿,受制于金钱和物欲,成为智力上的
无产阶级贫困化。 

一方面,斯蒂格勒说,现在我们已经逃脱不了网络信息技术构序起来的令人炫目的资本主义新世界,在数字化的时空综合中,构序个性化的主体性时间流被彻底摧毁了。作为个性化内部结构支撑点的欲望(désir)被摧毁了。你以为是自己想要的东西,其实不是。拉康曾经讨论过的那个“伪我要”(虚假需要),现在以全新的数字化幻境制造出来。如果你不想与世界隔离,你就得一遍遍地跟着更新电脑和手机中的操作系统,否则你将不能使用你看到听到世界的数字化综合构架。但当你顺从地不断更新时,你
\newpage
则被编码进数字化资本编织的“编程工业的巨流”中,成为任人宰割的用户群体之中。我们可以想一下,完全可以使用的电脑和手机,在硬件换代和频繁的系统升级中被人为地宣判死刑,正因为我们都不想落伍
,都不想被这个网络信息世界革除。 

另一方面,如果说早期的殖民主义背景下的世界贸易,多为明目张胆的强买强卖,再不行就直接武装开道,有如1840年前后,老牌帝国主义打开中国国门的样式,鸦片战争和八国联军对北京屠城中的表现。可是在20世纪帝国主义争夺殖民地和世界霸主地位的两次世界大战结束之后,后来的资本主义世界贸易的手段则改为温柔的说服技巧,直接的抢夺和欺骗不见了,资本家变成了最会讲诱人故事的高手。在斯蒂格勒看来,今天已经构序起来的网络和编程技术筑模之上出现的多媒体视听手段,将这种说服技巧上升到一个无以复加的高度。人们为什么会发疯一般地购买那些毫无个性的世俗化、均质化的工业文化产品?有如中国的一些妇女在世界各地奢侈品店中抢购名牌包,全世界的“果粉”永不停息地追逐苹果公司的各类新产品。斯蒂格勒的答案是,今天数字化资本
\newpage
主义中的图像和声音的技术与信息技术、电子通信相互结合,在讲述故事时具有了一种极为特殊且前所未有的力量,激发了人们对故事的信仰。人们并不知道,美国式的快餐遍及全世界, 看起来,它只是在卖汉堡和炸鸡,但却从孩子们初步开始构序的生活中深
深地布展一种“麦当劳化”的讲故事的方式。 

有时候与斯蒂格勒讨论学术问题,讲着讲着就想起马克思在《资本论》中常说起的那句话:“阁下,说的正是你的事情。”斯蒂格勒不同于传统西方马克思主义学者的地方,是他会想到了就亲自践行。近年来,他在法国成立的组织集合了不同背景的人士,例如工程师、哲学家、经济学家等,联合研究以及寻找一种新的工业精神。2006年,他在巴黎蓬皮杜中心成立了一家叫“创新与研究中心”的非营利机构,组织了一批工程师、编程人员发展研发出不少以合作为主题的软件,包括音像材料的合作性注记,建基于推特的辩论平台等等。这些都是在尝试去探究非资产阶级意识形态之下的后工业社会发展的前景,以及科技的解药性,进而引导一种新的个性化,从根本上超越数字资本主义的奴役,建设一种全新的知识共产
\newpage
主义。我以为,这是斯蒂格勒的激进思想和实践中最令人心潮澎湃的部分。这一改传统西方马克思主义左翼学者那种将批判构境仅仅停留在书本和空洞的激愤中的乌托邦憧憬,这是一种对现实革命实践道路的实验和有益探索。2018年夏天,我们在巴黎蓬皮杜中心旁的会议室共同举办学术研讨会时,我见到了他
的这群各行各业的战友们。斯蒂格勒真的很努力。 

贝尔纳,我还记得去年对话结束时你那充满自信的目光,你曾经是那么阳光向上的抗争者,可是,你怎么就放弃了?用策兰的诗性话语,“你已在那光亮之中,希望烟飞云散之处,洒向世界的,还是人性
中的柔情和不屈”。 \par 愿上苍厚待每一个致善的灵魂!

\end{document}
