\documentclass{article}
\usepackage[utf8]{inputenc}
\usepackage{ctex}

\title{绿山墙的安妮·峰回路转\footnote{Click to View:\url{https://web.archive.org/web/20221205032315/https://www.nunubook.com/yuyantonghua/8335/321325.html}}}
\author{露西·莫德·蒙哥玛利}
\date{1908-04}

% \setCJKmainfont[BoldFont = Noto Sans CJK SC]{Noto Serif CJK SC}
% \setCJKsansfont{Noto Sans CJK SC}
% \setCJKfamilyfont{zhsong}{Noto Serif CJK SC}
% \setCJKfamilyfont{zhhei}{Noto Sans CJK SC}
% \setlength\parindent{0pt}

\begin{document}
\CJKfamily{zhkai}

\maketitle


\Large

第二天玛丽拉上镇上去了,傍晚才回来。安妮到果园坡去找戴安娜,回家后发现玛丽拉手撑着脑袋,坐在桌旁。不知怎么的,一见她那垂头丧气的样子,安妮打了个寒战。安妮从未见过玛丽拉这样没精
打采地呆坐着。 


“你累了吧,玛丽拉?” 

“是的……不,我说不上。”玛丽拉抬起头,疲倦地说,“我想我是累了,可我还没有想到这份儿
上来。问题不在这里。” 

“你见过眼科大夫了吗?他怎么说?”安妮急
切地问。 

\newpage

“是的,见过了。他检查了我的眼睛。他说,要是我不再看书和做针线活,不再做任何有伤眼睛的事,要是我注意不掉眼泪、戴上他配的眼镜,那么他认为我的眼睛不会再继续坏下去,我的头痛病也会消失。他说,要是不这么做,他说我的眼睛肯定会在六
个月内瞎了。瞎了!安妮,你想过这话吗?” 

安妮惊叫了一声,接着陷入了片刻的沉默。她觉得自己实在无法回答。过了一会儿她勇敢地说:“玛丽拉,别再想这事了!你知道,大夫已经给了你希望。要是你多加注意,你是完全不会失明的。要是他配给你的眼镜能治好头痛,那就是天大的好事了。”
 

“我可不认为希望会有多大,”玛丽拉痛苦地说,“要是我既不能看书,也不能做针线活,什么事都做不成,活着还有什么意思?那还不如瞎掉的好——死了的好。要说掉眼泪,每当我感到孤独时,我忍不住要掉泪。得了,这事儿还是不说的好。你去给我倒杯茶吧。我累坏了。眼下这事你千万别对任何人说起。那样人家就会跑来问长问短,说些同情的话,没
\newpage

完没了的,我可受不了。” 

玛丽拉吃好晚饭,安妮劝她睡觉去。安妮自己也回到东山墙,泪水涟涟,心情沉重。黑暗中,她在窗旁坐了下来。自她回家后的那晚以来,发生了多少令人痛心的事!当时她满怀希望与欢乐,未来似乎是光辉灿烂的。安妮觉得此后自己已生活许多年了。不过在她上床睡觉前嘴角还是露出一丝微笑,心情也平静下来。她勇敢地正视自己的责任,且把它看作自己的朋友——当我们坦然直视责任时,责任就始终成自
己的朋友。 

数天后的一个下午,玛丽拉刚在前院与一位来客说了一阵话后,慢慢地走了进来。安妮一眼就认出来客是卡莫迪来的萨德勒。安妮不知道他说了什么,
害得玛丽拉的脸色这么难看。 


“萨德勒先生干什么来的,玛丽拉?” 

玛丽拉在窗口坐了下来,眼望着安妮。尽管眼科大夫再三叮嘱她不要哭泣,她的眼睛里还是泪汪汪
\newpage
的,连说话的声音都变了:“他听说我要把绿山墙卖
了,想买下来。” 

“买下来!买下绿山墙!”安妮怀疑自己是不是听错了,“哦,玛丽拉,你是不是说要把绿山墙卖
了?” 

“安妮,此外,我不知道还有别的办法。我已仔细考虑过了。要是我的眼睛没问题,我能在这里待下去,雇上个得力的帮工,料理好事务,管好这个家。可从现在的情况看来,我没法维持好这个家了。我可能完全失明,那时怎么也管不了事了。哦,我压根儿没想到过在我的有生之年会变卖自己的家产。可情况将会每况愈下,到时候就不再有人想买绿山墙了。我们家的每分钱都放到银行里去了。还有几张去年秋天马修签的单据,要偿还。雷切尔太太劝我把农场给卖了,住到别的地方去——我想跟她一起住。绿山墙卖不了多少钱——规模太小了,房子又很旧。不过我估计,得来的钱还是能维持生活的。幸好你有一笔奖学金,安妮。遗憾的是你放假时,无家可回了。情况

\newpage
就是这样,不过我想你好歹能对付过去的。” 


玛丽拉说罢忍不住号啕大哭起来。 


“你不能卖绿山墙!”安妮坚决地说。 

“安妮,我也不想卖呀。可情况就是这样,你也看得很清楚。我不能孤孤单单一个人住在这儿。种种困难和孤独会逼得我发疯的。再加上我的视力会—
—我知道准会的。” 

“你不会独自一人待在这儿的,玛丽拉。有我
陪着你呢。我不准备去雷德蒙德了。” 

“不去雷德蒙德!”玛丽拉双手捂着那憔悴的脸,这时放了下来,同时抬起头,望着安妮,“什么
,你这话是什么意思?” 

“我说得很明白了。那份奖学金我不要了。昨天晚上你从镇上回来后我就作出了这个决定。你为我付出了这么多的心血,你不应该认为我会在你遇到困难的时候丢下你一走了之的。我想了又想,盘算很久
\newpage
了。请听听我的打算吧。芭里先生想在明年把咱们家的农场租下来。所以你用不着为这事操心了。我打算去教书。我已向这里的学校提出了申请——我估计不能得到这份工作,因为我知道理事会已答应把这职位交给吉尔伯特•布莱思了。那我就去卡莫迪的学校——昨晚布莱尔先生就跟我说过。当然这就没有在阿丰利学校合适方便,可至少在天气暖和的时候,我可以住在家里,来回自己驾车。即使在冬天,星期五我可以回家。这样咱们得留下一匹马。哦,我全想好了,玛丽拉。到时候我可以给你念念书,让你快乐,你就不会寂寞孤单了。你我在一起会非常舒心和幸福的。
” 


玛丽拉仿佛身处梦境,呆呆地听着。 

“哦,安妮,要是有你在身边,我的日子就会非常快乐的,这我知道。但我不能让你为了我牺牲自
己的前程。那就太可怕了。” 

“说到哪里去了!”安妮开心地笑开了,“说不上是牺牲。没有比卖掉绿山墙更糟糕的了——没有
\newpage
什么比这更伤我的心了。咱们一定要守住这个心爱的老地方。我不准备去雷德蒙德了。我就要留在这儿教
书。你丝毫也不要为我操心。” 


“可你的抱负——和——” 

“我照样有自己的抱负。只是我改变了抱负的目标而已。我要做个好教师——我也打算保住你的视力。此外,我想在家里自学,自修一些大学的课程。哦,我的打算有一大堆哩,玛丽拉。这都是我足足花了一个星期想出来的。我要把一生中最美好的东西都献给这里,我相信我也会得到最丰厚的回报。我离开女王学院时,我的面前似乎展现一条康庄大道,我以为沿着它走下去,就能看见许多的里程碑。现在我只是遇到一个弯道。我不知道过了这弯道那边有什么,但我深信,一定是最美好的。这条弯道自有它迷人之处,玛丽拉。我想知道过了弯道通向何处——是不是有碧绿而轻柔的光华和变幻无常的光彩和阴影——崭新的风光——陌生的美景——接下去是不是有多道弯
、多座山。” 

\newpage

“我觉得不该让你放弃才是。”玛丽拉说。她
指的是不该放弃奖学金。 

“可你阻止不了我。我已经十六岁半了。我这个人,像雷切尔太太说的,‘固执得像驴’。”安妮说着笑了起来,“哦,玛丽拉,你用不着怜悯我,我不喜欢别人怜悯,也不需要怜悯。我一想到能留在心爱的绿山墙打心眼里感到高兴。没有人像你我那样爱
它——所以咱们一起要守住它。” 

“老天保佑你这姑娘!”玛丽拉只得同意了,“我觉得你像是给了我新的生活。我认为我本应该坚持让你去上大学的——但是我知道,我拗不过你,所以不再打算劝你了。不过,我会让你得到补偿的,安
妮。” 

安妮•雪莉放弃去上大学,准备留在家乡教书的消息传开后,在阿丰利闹得沸沸扬扬,议论纷纷。大多数好心人,由于不了解玛丽拉眼睛的病情,以为安妮太傻了。可阿伦太太不这么想,所以她说了不少表示赞同的话,这让这姑娘高兴得热泪盈眶。好心的
\newpage
雷切尔太太也持有相同的观点。有天傍晚她来到绿山墙,看到安妮和玛丽拉在花香扑鼻的暖和的夏日暮色中,一起坐在前门口。每当暮色苍茫之时,花园四周白蛾飞舞,清新的空气里薄荷飘香,她俩总喜欢坐在
那里。 

雷切尔太太一副疲惫的神情,舒了口气,她那壮实的身躯在门旁石凳上坐了下来。石凳后面是一排
高高的粉红色和黄色的蜀葵。 

“不瞒你说,到底有个地方坐坐了,真高兴。瞧我这两条腿,压着个两百多磅的身子,整天跑来颠去的,够累人的。不发胖的人才叫有福气哩,玛丽拉。我希望你好好珍惜。嗯,安妮,听说你放弃了上大学的念头,我非常高兴。你现在受的教育够高的了,一个女人,能做到这一步该满足了。我可不相信姑娘家跟小伙子一起上大学,满脑子装了拉丁文、希腊文
之类的乱七八糟东西有什么有好处。” 

“可我还是照样要学拉丁文和希腊文呢,雷切尔太太。”安妮笑着说,“我准备就在绿山墙里学习
\newpage

文科课程,把大学里要学的全学会。” 


雷切尔太太惊讶得举起了双手。 


“安妮•雪莉,你会累死的。” 

“哪能呢?我会健健康康的。哦,做事我会量力而行的,正像‘约西亚•阿伦的太太’说的,我会‘悠着点’的。漫长的冬天晚上,我有的是空闲的时间,我天生就不喜欢干编编织织的活儿。知道吗,我
要去卡莫迪教书。” 

“这我没听说。我看你准在阿丰利教书。理事
会已决定让你来阿丰利的学校任教了。” 

“雷切尔太太!”安妮跳了起来,意外之余大声说道,“不是吗,他们已答应吉尔伯特•布莱思了
!” 

“他们是答应了。不过吉尔伯特一听说你提出了申请,便跑去找他们——昨天晚上他们在学校里开
\newpage
了事务会——他对他们说,他要收回自己的申请,还建议接受你去任教。他说要到白沙镇去教书。不用说,他放弃这里的学校完全是为了你好,因为他知道你多么希望能和玛丽拉待在一起。我得说,这小伙子的心地就是好,想得也周全,就这话。他也够有自我牺牲精神的,因为那得多付出一笔在白沙镇的食宿费用,大家都知道,他得自己赚够钱好去上大学。就这样理事会决定聘用你了。托马斯回家把这事跟我一说,
可把我给乐坏了。” 

“我认为我不应该接受,”安妮喃喃低语,“我的意思是说,我不应该让吉尔伯特为了——为了我
而作出牺牲。” 

“我看你现在没法阻止了。他已跟白沙镇方面签好了合同。就是你拒绝了,对他也没有好处。不用说,你会接受这所学校的。现在这里已没有派伊家的孩子在上学,你会干得顺顺当当的。乔西是他们家最后一个来上学的孩子,也是难对付的主儿,就这话。最近二十年来,阿丰利学校陆陆续续都有派伊家的孩子在读书。我觉得他们活着就是让教师记住,这里可
\newpage
不是他们容身之地。天哪!芭里家那闪光到底是什么
意思?” 

“是戴安娜给我发的信号,让我过去。”安妮笑道,“知道吗,我们一直还保持老习惯呢。失陪了
,我得过去,看看她有什么事。” 

安妮像只小鹿,跑下长着三叶草的山坡,消失在“闹鬼的林子”中的冷杉树阴影中。雷切尔太太宽
容地打量着她的背影。 


“有的地方她看上去还完全是个孩子。” 

“从另一些地方来看,她完全是个成熟的女人
了。”玛丽拉又用过去那种口吻,毫不含糊地说。 

但是,正如那天晚上雷切尔太太对自己的托马斯说的,现在的玛丽拉,说话毫不含糊不再是她突出
的性格特征了。 

“玛丽拉•卡思伯特变得温和了,就这话。”
\newpage


第二天傍晚,安妮来到阿丰利的小墓地,给马修的坟头换上新鲜的花束,又给苏格兰玫瑰浇了水。她在那里盘桓了很久,直到暮色很浓才回家。她留恋那一小块地方宁静和温馨的氛围,白杨树友好地对她沙沙低语,自由自在生长的青草说着悄悄话。最后她离开坟地,顺着向“闪光的湖”的下坡走去,太阳已经下山了。她面前的整个阿丰利笼罩在梦幻般的余晖之中——“古老的宁静永不消失的地方”。空气中有一股新鲜的气息,好像是风刚刚吹过三叶草的田野带来的甜蜜清香。宅院四周的树木丛中闪烁着明明灭灭的灯光。远处是大海,轻雾蒙蒙,紫气氤氲,而它那永无休止的低吟浅唱始终在耳际萦回。西方的景色柔和而色彩斑斓,投入池塘中的倒影显得越发柔和而迷离。面对这良辰美景,安妮心潮澎湃,她怀着一颗感
恩的心向它们尽情吐露自己的心曲。 

“亲爱的世界,”她低声道,“你多么美好,
我庆幸活在你的怀抱中。” 

下坡途中,从布莱思家大门走出一位高个的小
\newpage
伙子,他边走边吹着口哨。他是吉尔伯特,一认出迎面而来的安妮,嘴边的口哨声便消失了。他很有礼貌地抬了抬帽子。要不是安妮停下脚步主动伸出手去,
他会一言不发擦肩而过的。 

“吉尔伯特,”她红着脸,说,“我想谢谢你。你为了我放弃了这里的学校。你太好了——我想让
你知道,我对此非常感激。” 


吉尔伯特热情地握住安妮伸出的手。 

“这并非我特别善良,安妮。我很高兴能为你尽绵薄之力。此后我们能不能成为朋友?你真的原谅
我以往的过错了吗?” 


安妮笑了,她想抽回手,但没有成功。 

“你帮我从池塘边上了岸,那天我就原谅你了,只是我自己没有意识到。我可是个固执的小傻瓜。我一直——我还是彻底承认了吧——从那以后,我一

\newpage
直很后悔。” 

“今后我们会成为最好的朋友,”吉尔伯特喜滋滋地说,“我们天生就应该是好朋友的,安妮。你一直在阻挠命运的安排,够久了。我知道我们在许多方面可以互相帮助。你准备继续学习,是不是?我也
是。来,我送你回家。” 


安妮走进厨房,玛丽拉好奇地望着她。 


“跟你一起从小路过来的是哪个,安妮?” 

“吉尔伯特•布莱思。”安妮答道,她发现自己的脸热了起来,很是恼火,“我在芭里家的山冈上
遇到了他。” 

“我没有想到你和吉尔伯特•布莱思好到这样的程度,居然在大门口跟他说了半个时辰。”玛丽拉
生硬地笑了笑。 

“我们向来不是——我们向来就是死对头。可是我们已理智地作出决定,将来成为好朋友。我们刚
\newpage
才真的在那边待了半个小时了?好像只有几分钟。可是,你瞧,我们已有五年没说话了,有多少话要说呀
,玛丽拉。” 

那天晚上安妮久久地坐在窗口,心满意足,喜气洋洋。风在樱桃树枝间轻轻吹拂,送来阵阵薄荷清香,山谷里,冷杉的尖尖树梢上星星眨巴眼睛,戴安
娜的灯光透过古老的缝隙仍然在闪烁着。 

安妮从女王学院回来的那天晚上坐在窗口以来,她的活动天地变窄了。可是,即使她脚下的小路是狭窄的,她知道,这一路上仍然开放着恬静的幸福之花。真诚的工作带来的欢乐,有价值的追求,志趣相投的友情都将属于她。任何东西都无法夺走她那与生俱来的想象权利和梦幻的理想世界。总有峰回路转之
时。 

“苍天在上,愿万事美满!”安妮轻声祝愿道
。 



\end{document}
