\documentclass{article}
\usepackage[utf8]{inputenc}
\usepackage{ctex}

\title{人造现实\footnote{Click to View:\url{https://web.archive.org/web/20221009034216/https://listingk.com/society/421906.html}}}
\author{范先慧}
\date{2009-06}

% \setCJKmainfont[BoldFont = Noto Sans CJK SC]{Noto Serif CJK SC}
% \setCJKsansfont{Noto Sans CJK SC}
% \setCJKfamilyfont{zhsong}{Noto Serif CJK SC}
% \setCJKfamilyfont{zhhei}{Noto Sans CJK SC}
% \setlength\parindent{0pt}

\begin{document}
\CJKfamily{zhkai}

\maketitle


\Large

灰白的冷墙夹杂着霉味的血腥。耿伟感到一股粘稠的液体正顺着冰冷的肌肤缓缓流下来。是血?
还是汗?黑暗中,传来一阵开心而凄厉的笑声。 

他拼着最后的一口气,艰难地伸出小手:“妈
妈,救我!妈——” 


一 

“小伟,没事了,妈妈在这儿!”静溪坐在床
边,紧握着儿子的双手。 

“没用的,他现在听不见任何人说话。”靳雷
站在她身后说,“他正处于功能性精神分裂中。” 

\newpage


“什么!”静溪睁大了惊惧的眼睛。 

“受到极大的精神惊吓和刺激而出现间歇性精
神紊乱和心理休克……” 


“治愈的可能性有多大?” 

“很难……”他低下头,“听天由命吧——他还是个孩子……临床成功的例子不多,即使痊愈,也
会留有严重后遗症……” 

静溪默默地低下头。即而,她又抬起头来,定定地看着靳雷:“治好他,你一定有办法!小伟,他
还只有4岁……” 


“别难为我……” 

“如果小伟果真不能恢复,我们将一辈子良心难安!” 静溪那双掩映着长长睫毛的眼睛望着靳雷,眼里尽融了悲哀。“雷,求你,我求求你……我就

\newpage
只剩下小伟这一个孩子了……” 

靳雷的嘴角难以觉察地抽搐了一下,他的心里仿佛在喃喃地说,你唯一的孩子?那我呢?我唯一的
孩子呢…… 

然而看到静溪满面泪痕,他强迫自己缓和了下情绪,“静溪,小伟是从小我看着长大的,我对他的
爱丝毫不亚于你……放心,我一定竭尽全力!” 

“谢谢……”她脆弱的神经捕捉到了一丝淡淡
的希望。 


二 

静溪疲惫不堪地坐在靳雷对面。虚弱的表情和
微红的双眼昭示着又一个个不眠之夜。 

太多的不眠之夜了。她睡不着,也不敢睡。每当闭上眼睛,她就会看见一张稚嫩而又扭曲的小脸,
听见那一阵阵撕心裂肺的开心的狂笑。 

\newpage

“喝杯水吧。”靳雷把桌上的杯子推向静溪,
“小伟已经没事了,手术非常成功。” 

“靳雷,我真不知道说什么才好……” 静溪初露喜色的脸瞬间又蒙上了厚厚的哀伤,“那孩子…
…唉,我欠你的实在太多太多了……” 

“别去想吧。现在小伟没事了,然而可这并不
代表以后永远高枕无忧。” 


“怎么?” 

“我用手术抹去了他对'那件事'的记忆,我们只能寄希望于,他以后永远也不再想起发生过的一切。否则。否则,对他的打击……我不能保证他不会
旧病复发,甚至更糟……” 


静溪垂下长长的睫毛,沉默。 

“算了,不要想了,相信不会发生,一切都过去了……一切。总会好起来的……别担心……别担心
\newpage
……” 靳雷深沉而温存地说。不知道是宽慰她还是
在宽慰自己。 


三 


15年后。 

“妈!从今天开始,我就要到靳伯伯的科研室实习了!”一大早,耿伟就兴冲冲地隔着桌子对坐在
桌子另一边的静溪叫道。 

“从昨天开始就不知重复多少遍了!”静溪望着儿子微笑,“实习时要一切听导师安排,知道吗?
” 


“没问题!” 

静溪把儿子送到门口。看着小伟跳跃着远去的
背影,她忽然莫名其妙地轻叹了口气。 

天,转眼间耿伟已经变得这般高大。这些年来
\newpage
,靳雷一直是他心中的偶像、导师,甚至父亲。是的,就像是“父亲”。耿伟的生父在他很小的时候就去世了。伴随他从小到大的,是靳雷。静溪能清晰的感
到靳雷在耿伟心中特殊而不可替代的位置。 


四 

“小伟,今天的表现相当出色!”法政医科研
究所门口,靳雷轻拍着耿伟的肩。 

“当然!”耿伟自信地笑,“我什么时候会给
您丢过脸!” 

“下班了,找个地方一起吃饭去?”靳雷疼爱
地看着他。 

“不啦!我约了沙军,待会您来我家吧!今天是我实习第一天,妈为我们准备了丰盛的晚餐,请您
一定要来!” 

靳雷笑了:“好,我也好长时间没尝你到母亲
\newpage

的手艺了!” 


“正是!” 


“太不可思议了!绝不可能是自杀……” 

“是的,绝无可能。首先自杀不可能从背后那么准确地从背部刺中肺叶中央。而且,自杀者由于对死亡本能的畏惧通常都会留下缓冲迹象的犹豫伤,女
孩是被一刀命中要害,没有半点犹豫。” 


“狠毒至极。”沙军喃喃地说。 

“更奇怪的是,从面部表情僵硬度及声带受损
分析报告,死者临死前应该是在笑——” 

“笑?”沙军久经沙场,还是感到一丝不易察
觉的凉意。 

“对,而且还是一种歇斯底里的狂笑。” 耿

\newpage
伟补充。 

“不可能!”沙军从椅子上跳起来,凑过来看
。 

“的确令人匪夷所思……”耿伟放下报告。他
显然也被这宗离奇的怪案吸引住了。 

他走到桌边信手翻阅厚厚的卷宗:死者小敏,12周岁。住所不详。家庭状况不详。无人认尸。现
场无任何其他可证实其身份的有效证件…… 

耿伟的目光继续往下移,报告下方赫然印着一
张死亡现场的放大照片。 

犹如一道惊怪的闪电划过心尖,耿伟不自觉地震了一下,胃仿佛在一瞬间被人狠狠捏了一把,顿时感到一阵恶心。他不由自主靠着椅子剧烈呕吐起来。
 

“怎么了?”沙军扶住耿伟,“这可不像是未

\newpage
来法医专家的素质啊!” 

“哦,不,我从不这样的。今天是怎么了?”耿伟勉强支起身子,“沙军,这几页验尸报告暂时放
在我这儿吧,我会尽力帮你,一定发现线索!” 


“够哥们儿!”沙军响亮地拍了耿伟一记。 


六 

让人匪夷所思的验尸报告平摊在桌上。耿伟坐
在桌旁。不知怎么的,他有种莫名的诡异感觉。 

早晨,他百无聊赖地翻看材料,厚重的扉页中忽而飘出一张彩色半身照片。耿伟捡起来,认出正是被害者。这是一张根据死者遗容制作还原的死者生前
的电脑合成图片。 

照片上是个非常漂亮的女孩子。雪白的面庞、高直的鼻子、削峭的薄唇,这一切都使她看起来那样的坚毅和早熟。尤其是那双大大的眼睛,点缀着长长

\newpage
的睫毛,让耿伟莫名其妙感到有些似曾相识。 

耿伟看着那双眼睛,四目相对。他觉得那双眼睛也在凝视他,仿佛她认识他。不,这怎么可能?!耿伟使劲地甩甩头。可能女孩被害的惨状给自己的印
象太深了,实习也实在太忙,太累,太过于紧张。 

“小伟,饭做好了,来吧!”静溪在客厅忙着
布置满桌的菜肴。 

“都是你最喜欢的。还有这个——京酱肉丝!小伟,今天你也尝尝靳伯伯的手艺!” 靳雷在厨房
得意地招呼。 

“就来!”耿伟把相片塞进厚厚的材料。材料
上的这一页记录着案发现场:团圆路13号。 


七 

“团圆路13号。”耿伟轻念着。在他面前的
,是一幢废弃已久的大宅。 

\newpage

几天他一直心神不宁,常常在睡梦中被噩梦惊醒。梦里总会浮现女孩那张灿烂儿明净的笑脸。那甜甜的笑意会在一刹变得阴森可怖,疯狂地尖叫,让他
在突然觉醒,心惊肉跳。 

耿伟走到门前,推开宽阔的门扉。门和转轴相互倾轧发出的尖锐的摩擦声儿,随之而来的,是一阵扑面而来的呛人烟尘。屋内所有的一切,仿佛都被岁
月的无情昏暗所吞噬。 

耿伟觉得自己从没来过这儿,却闻到了一股熟悉的气息——那是死亡的气息。他在实习期间多次遇到过,却没有像这一次一般深沉而阴暗。他疑惑地在这片残宅中轻踱。大厅深处光线渐弱,灰尘在他脚下轻扬。猛然,他触到一个冰凉的东西——一面灰白的断墙。枯槁的灰白和冰冷的触感让他有种触电般的恐
惧。刹那间,他仿佛听见一声悠长而尖厉的笑声。 


八 

“小伟,你今天上午去哪了?靳伯伯说你没去
\newpage

研究所。”静溪把午饭端上桌。 

“妈,对不起,我没胃口。”耿伟起身推开碟
子。 


“怎么了?”静溪关切地问。 

“没什么,妈。我想休息一会儿。”耿伟一边说,一边抱起沙军的资料走进房间。一张相片从厚厚的资料中滑下,跌落在地上。静溪把照片拾起来,看
了看,脸色刹时变得惨白。 

“怎么,妈?您知道照片上的女孩?”耿伟问

“啊,不,没什么……这孩子,太像我认识的
一个孩子了。” 

“是吗,怎么认识的?邻居家的?我怎么不知
道?”耿伟注意到了母亲的慌乱与失态。 

“伟,这张照片哪来的?”静溪死死地抓着照
\newpage

片的一角。 

“是沙军的。他正在调查一个十几年前的旧案
。” 


静溪呆呆的,不知所措地陷在沙发里。 


“他的东西怎么会在你这儿?” 

“沙军让我帮忙。妈,你要知道,相片上的女孩子只有12岁!您知道12岁的女孩有多高,有多可爱吗?凶杀应该得到应有的惩罚!我今天打算再到
犯罪现场察看一下。” 

“不!我不准你去!”静溪激动地大叫,抓住
耿伟的手。 

耿伟奇怪地望着静溪:“妈,您今天怎么了?
您平时不是最赞成惩恶扬善吗?” 

“不……我只是不希望我儿子卷到一场这样的
\newpage

凶杀案中,可怕极了!”静溪的脸色苍白而忧郁。 


“妈,您怎么知道是一起凶杀案件?” 


“别管那些,总之我不允许!” 


“妈?好了,别担心。还有别的事儿吗?” 

“啊,不,没了。”静溪尽量显出平静的神色
,松开了抓住耿伟的手。 


“下午早点去研究所。靳伯伯等着你呢。” 


“恩。”耿伟转身走回房间。 

看着耿伟高大的身影,静溪眼前仿佛浮现出另一个异常清晰的影象。那是一个瘦弱、单薄、轻盈美
丽的女孩子的背影……她的双眸不知不觉间微湿。 


九 

\newpage

夜近黄昏。耿伟又一次立于夕阳下的那栋凶宅

母亲一定隐瞒了什么。而自己呢?何以对面前这座素未谋面的建筑有如此熟悉?好奇心在他心底无
限弥散开来,沉淀下无数迷惘和困惑。 

朽碎的地板在他脚下发出断裂的呻吟。他在想,他在努力回忆,在他心灵深处始终有一个记不得的
梦…… 


十 

冷硬的墙,黑紫的霉腥,碎裂的地板,这一切让他不寒而栗。一只冰冷生硬的胳膊使劲勒住他的脖子,使得他呼吸困难,头晕目眩。他害怕地挣扎,向无尽深邃的黑暗伸出小手:“妈妈,救我!妈——”

“住嘴!”一声尖声的呵斥由黑暗的深处传来

一只纤细的胳膊紧紧勒住他的脖子,并在他稚

\newpage
弱的颈骨上加重了的力量。 

“我不准你这样叫!妈妈是我的,是我一个人
的!恨死你!我恨你!” 

锋利的尖刀在他后颈上划出了一道窄长的血口
。他受痛,发出一声小兽般的惨叫。 

“小伟!”他在黑暗中听见一男一女同时惊呼


“敏敏,别这样!”女人哭叫着。 

“妈……妈妈……”朦胧间,他依稀认出了女
人。 

“妈妈!你肯终于来啦!”黑暗中的声音透着
疯狂的惊喜。 


“敏敏,冷静点儿!”男人大喊。 


“放开小伟,就当妈求你了!” 

\newpage

“你给我过来,放开你弟弟,他无辜的!”男
人的声音焦急而严厉。 

瘦弱轻盈的身影拖着小耿伟向后退了一步,勒得他生疼,“我明白了,原来,你们抛弃我,完全是为了这个孩子!他有那么重要吗?比我重要?妈妈,
你说,妈妈,你爱我吗?” 

“敏敏……别做可怕的事……妈妈求你了!”
女人哭得几近虚脱。 

“我们当然爱你!我们是你的父母啊!可现在,你必须冷静下来,来,放开那小男孩!”男人上前
一步,向声音张开双臂。 

“不!”声音继续向黑暗深处退去,“为什么,为什么你们都那么在乎这孩子!为什么他可以光明正大地活,我却活得像个影子?你们不敢认我,不敢见我,不敢爱我,你们都是胆小鬼!我恨这小崽子,为什么他拥有的一切我却没有!我要杀了他!为你们自己的自私懦弱付出代价,我要让你们身败名裂!”
\newpage
这疯狂的声音紧箍着他,毫不留情地将尖刀举到空中
,刺向他孱弱的咽喉。 

“小伟!”女人不顾一切向声音扑过去,男人紧追其后。混乱中,他感到自己被女人用力拉出去,摔在厅中的那面灰墙上,顿时感到一阵刺骨的冰凉。


“放开我!”声音在挣扎,在狂叫。 

“按住她!让她冷静下来!她已经失去理智了
!”男人对女人喊。 

“妈……”朦胧中,他意识模糊,奄奄一息。

没有人搭理,没人顾得上他。昏惑中,他看见男人蹲在地下,按住一个黑影。女人则紧紧地抱住那
个在地上不断扭曲的影子…… 

是妈妈!黑暗中,妈妈掰开那只疯狂的握住尖
刀的白皙的手,紧握在空中……” 

\newpage


“妈……”他努力往前爬去…… 

“妈妈!!!!!”即而,他忍不住惊声尖叫,一股红色的湿气飞溅在他脸上,令人作呕的腥味顿
时弥漫了整个狭窄的空间…… 

就在那一刹那,他看到一双仇恨的眼睛在黑暗中死死地盯着他,“我恨你,我恨你!恨你们!”他看见血不断从她嘴角喷出来,把他的视野糊成一片…
… 

“太不公平了!我放不过你们!我会报复……咳咳咳……”急促的咳嗽和着暗红的血色在空气中回
荡,咳声随着血气的浓烈渐渐减弱,最后是没有。 

当咳嗽声都听不到了,他听见一声疯狂而歇斯底里的笑声:“爸爸妈妈是我的,总有一天他们是我的!我要报复你,报复!你不会有好下场!呵呵呵呵
呵……” 


\newpage

那双死亡的眼睛至死都死死地盯着他。 


十一 


耿伟从梦中惊醒。 

残余的阳光从旧宅的罅隙透射进来。木色的地板依然泛着熟悉的霉味。刚才究竟发生了什么?梦吗?不,不是梦。一幕幕情景竟如身临其境般的熟悉逼
真。绝不是梦。 

他依然徘徊在大宅里,踉跄地直起身,踩着枯朽的地板,地板的一处断面在他践踏下猛然横向裂开,发出一声惨烈的怪叫。他一下失去平衡,摔倒在地


地板的断面下,一个灰亮的物体显现出来。 

这是一枚银色的、中空的鸡心挂坠。小巧的坠
面正反镌刻着几个秀巧熟悉的字型: 


敏敏12岁,生日快乐,幸福安康! 

\newpage


——妈妈 


耿伟打开了鸡心……血液在刹时凝固。 


十二 

静溪有些心神不宁。这种心神不宁在她心底铺
开一层浓厚的不详。 


电话铃响。静溪三步并作两步奔过去。 

“小伟,是你吗?靳伯伯打电话告诉我,你下
午没去他那儿。” 

“是。妈。”电话彼端,耿伟的声音冷且硬。
“我去了团圆路13号。” 

“……”静溪的心刹时透过脊梁感到了刺骨的
寒意,“不是叫你别去那儿吗?” 

“您只是让我别插手沙军的案子,别去凶案现
\newpage

场,我并没有告诉您团圆路13号是凶案现场。” 


“……”静溪的手心渗出细汗。 

“是的,那是因为您知道那儿就是案发现场!
15年前,那儿发生过一桩惨不忍睹的血案。” 

“伟,你误会了。妈妈只是一时情急……这样
吧,你马上回家,妈妈再详细问你……” 

“妈,今晚我住沙军家。而且……妈妈,我不
会再回来了……”耿伟哽咽地吞了口气。 


“说什么!?”静溪对着电话大叫。 


“妈,我全想起来了!” 


“小伟!” 


“您想让我把所有事实说给您听吗?” 

\newpage

“不,小伟,你冷静点,听妈妈说,那不是真
的!” 

“我也希望不是真的。”耿伟的声音开始打颤,“15年前,当时我只有4岁。一个叫闵敏的女孩绑架了我,她,就是你和靳雷的私生女。你们闻讯赶去,她用我要挟你。她憎恨我,我有合法的父亲和母亲,而她没有。在她要杀我的时候,你冲上去把我从她手中拉出来,靳雷把她按在地上。本来……此时你们就可以住手了。但你们为了保住名誉,你举起尖刀,刺穿了那女孩的肺叶……”耿伟的声音由平诉转为激烈,“她是您的亲生的孩子!我的异母姐姐!你们是双手沾着孩子鲜血的父母!是你,恶毒地亲手杀死
了自己的女儿!” 


…… …… 

电话那边历经了非常非常长久的一段沉默。耿
伟仅仅听到一声如释重负般的叹息。 


\newpage

“这就是你要说的吗?”母亲终于开口问。 


“是!” 


“那,你打算怎么办?” 


“等你解释。” 

“没有解释。终究会有这天。但我没料到,竟
会是这样……” 

“……”耿伟沉默了一会儿,良久,他从牙逢里挤出几个连自己也觉得可怕的字眼:“我要去告你
们!” 

“好吧,孩子;再见,孩子。”静溪挂掉了电话。她镇定地站了一会儿,然后,开始拨号:“喂?
靳雷吗?马上过来,有重要的事。” 


十三 


\newpage

“他想起来了。”静溪背对着靳雷。 


“什么?” 

静溪转过身,“敏敏的事……他记起来了……

“什么!”靳雷焦灼地看着静溪:“那他……

“听我说,”静溪迅速地打断靳雷的话,“他
说,是我杀了我亲生的孩子。” 

一瞬间,靳雷用惊恐地望着静溪,“怎么会这
样?小伟……” 

“这是不争的事实!”静溪用手紧紧地抓住靳雷的双臂,“从下一刻起,我就要为这个而付出代价
了。” 


“这不可能!那……” 

“靳雷……”静溪忽然一下抱紧面前这个鬓发苍苍的男人,“请你再帮我一次——再原谅我一次,
\newpage

原谅我的自私,就像以前一样……” 

“不!”靳雷忽然激动起来,“我从没没原谅过你!你是这个世界上最无情、最心硬、最不通人情的女人!你的丈夫并不爱你,你却坚持要嫁给他!”

“雷,这个时候,你还要和我争论这个无谓的话题吗?”静溪深情地趴在靳雷宽厚的臂弯中,“我终于不必再担惊受怕,也不必承受任何来自道德的枷锁和作为一个母亲的压力。在我一生中,真正觉得最幸福的就只有这一刻了——他很快会去告我,我时日
无多……” 

靳雷怔怔地看着静溪,很认真地问了一句:“
决定了?” 


“恩。因为,我只有小伟这一个孩子了。” 


十四 

时间与生命还有所有的一切仿佛都静止在一片
\newpage

黯然的阴森里。 


团圆路13号。 

耿伟一个人独自坐在屋内一张散发着霉味的朽
桌前。 

桌上放着一枚灰亮的坠饰。坠饰的盖开着,里面是一张温馨的照片——一个可爱的少女靠在年轻的静溪怀里,一个英俊高大的男人站在她们身后,用宽阔的肩膀环抱这对母女。是靳雷。坠饰下压着一张当
天的报纸,黑色的标题触目惊心: 

官员遗孀亲手杀死私生女法医院长丑闻败露畏
罪自杀 

耿伟站起来,从破损的窗台边拾起一片尖锐的
玻璃,重新走到桌前,坐下。 


呆若木鸡。 

\newpage

几分钟前,当他再次踏入这空寂房间的一刹那
,记忆的洪流如开闸的潮水,奔涌而出…… 


十五 

静溪冲过去夺下闵敏手中的尖刀扔在地下,不
顾一切搂住她的脖子哭泣着。 

“敏敏,好了,没事了,别怕,妈妈在这里。

“走开,放开我!”失去理智的孩子尖叫着,
“恨你们!你放手!” 

“不!我不会放的!我的女儿,敏敏!”静溪把女孩搂得更紧,任凭女孩把她的手臂咬得鲜血淋漓

“妈……”耿伟恍然间叫着,幽暗中他看见了
静溪满是鲜血的伤臂。 

“妈!”他努力支撑起细小的身子。黑暗中,

\newpage
谁也没有注意到他。 

这个小小的男孩扶着冰冷的墙,摸到了掉落在地下的尖刀……他慢慢地移动着身子,他的小手在微微颤抖。妈,我来救你,他浑身发抖,黑暗中,他猛地举起了那把沉重的尖刀,用尽全身力气向女孩的后
背捅下去…… 


“啊!”静溪和靳雷齐声痛叫。 

刹那间,他听见了一声刺耳而开心的笑声,顿
时失去知觉…… 


十六 

耿伟把玻璃片的锋刃轻轻地压在自己手臂的动
脉上。 

周围的一切很静,很静。深色的液体缓缓地桌上,无声地浸润了坠饰里的照片,照片下压着载有妈
妈死讯的报纸。 

\newpage


妈妈…… 

耿伟的心跳逐渐在寂寥的空间中稀薄,减弱,
接近听阀…… 

阴暗深寂的尽头,仿佛隐约传来微弱的笑声,
开心的笑。 


咯,咯咯咯…… 

听,是姐姐在笑……

\end{document}
