\documentclass{article}
\usepackage[utf8]{inputenc}
\usepackage{ctex}

\title{若是春天真的能予人重生\footnote{Click to View:\url{https://web.archive.org/web/20230510130358/https://site.douban.com/108119/widget/articles/158181/article/10874691/}}}
\author{于潇湉}
\date{2009-08}

% \setCJKmainfont[BoldFont = Noto Sans CJK SC]{Noto Serif CJK SC}
% \setCJKsansfont{Noto Sans CJK SC}
% \setCJKfamilyfont{zhsong}{Noto Serif CJK SC}
% \setCJKfamilyfont{zhhei}{Noto Sans CJK SC}
% \setlength\parindent{0pt}

\begin{document}
\CJKfamily{zhkai}

\maketitle


\Large

雪下得紧,到了夜半也还是没有停的样子。
 

白色皮毛的兔子在雪地上止步不前,把长长的脚印留在了身后。一时间,她有些分不清楚,那是给这大地留下的印记,还是从自己身体里遗失的什么东
西。 

落雪的时候并不太冷,雪花掉落在睫毛上甚至起了一层湿润的水汽。深吸一口气,她向上耸了耸自
己身上背着的东西。 

马上……记得前边有座庙可以躲躲雪的。今天
晚上也只能在那里过夜了吧。 

\newpage

只是,似乎没有东西可以盖呢。白兔心想,若是可以把影子扯下来盖在身上倒是不错呢。一边想着
,一边已经望见了破败的庙宇那片灰色的瓦楞。 

推开了快要掉下来的木门的时候,风没来由地大了起来。雪片旋转着飞进眼睛里,兔子闭上眼睛,
又睁开来——哎呀,难道有人? 

还好,在庙里昏暗的光线中慢慢现出的是另一只兔子的轮廓。那是一只灰兔。此时灰兔正抖落自己
身上的雪,玲珑的眸子转过来,望着白兔。 


看这样子……不也是刚进庙的吗? 

灰兔不发一言,径自走到墙角边,拾了些干草
,非常熟练地生起火来。 

“有火的话,就可以煮点东西来吃了呢。”白兔微笑着,想与这个将跟自己共同分享一个夜晚的陌
生人打破僵局。 

\newpage

灰兔这才又抬起头看了白兔一眼。白兔尽可能友好地笑着,虽然面对陌生人的时候她总是紧张。这样的担心和紧张几乎伴随了她一生,怎么都改不掉的。不过白兔倒是知道,当自己笑起来的时候,上下左
右无论哪个角度看上去都绝对温顺可亲。 

“你有吃的?”灰兔的眼睛又冷又清澈,像浸过水的葡萄。说着话的时候,光芒就像飞雪一样一片
一片旋转着飞出来。 

“其实,只有锅子,没有什么材料。”白兔指
指自己身上背的那口沉重的锅。 


“哦,我们来做个交易如何?” 

“什……什么交易?”白兔疑惑地问。难不成
面前这位是个商人吗? 

“我这里有干蘑菇还有一点米,你用你的锅给
我做点东西吃吧。” 

\newpage

“不用交易的,我的锅本来就可以给你用啊。
” 

“我不想占人便宜。”灰兔将一个口袋扔过来
,“我只有这个,好了,剩下的就看你的了。” 

白兔把锅架到火上,又去捧了雪来化开。把米和蘑菇扔进去,从自己的行囊里掏出了瓶瓶罐罐,一
长排都摆在地上。挨个儿地往那锅水里倒。 


“哎,哎——你放什么呢?” 

“调料啊,放心吧,不是毒药。”白兔一笑,她发现灰兔长得堪称俊秀,就是那身灰色的皮毛平添
了股桀骜不驯。 

似乎来了点兴趣,灰兔靠着火暖着自己,“你
是要做什么啊?” 


“粥。” 

\newpage


“粥啊……要是有肉的话那就最好了啊。” 

白兔看了灰兔一眼,突然不再笑了,“我做各
种粥,但就是不做带肉的粥。” 

“你吃素啊?那是旧时代的兔子了。现在的兔
子不带油味的萝卜都不啃一口呢。” 

“不管你怎么说,我就是不做带肉的粥的。”
白兔坚持起来。 

等到水沸腾起来,香味儿也飘散了出来。白兔从自己包里取出一只碗来,盛了满满的一碗先递给灰
兔。 

“你……不会是厨师吧?”灰兔看着白兔的动
作突然好奇了。 

“啊?差不多吧。可是我别的都不做,就只做
粥。” 

\newpage

“而且还必须是全素粥?怎么会有这样的坚持
呢?” 

白兔不想再多答话,只是勉强笑了一下:“你
呢?你看起来像只游手好闲的家伙哦。” 


“我啊,我是画故事和讲故事的。” 

“哦?”白兔挑了挑眉毛,“我曾经也喜欢讲
故事的,而且还喜欢写故事哦。” 


“那为什么又改行了?” 


“因为,心里有乌云。” 


“怎么说?” 

“我啊,每次写故事的时候,无论那是谁的事情,总是要自动代入到自己身上。时间久了,每次写故事我都会哭,太悲伤了。哦,我只写悲剧故事。”

\newpage

“你自虐啊?”灰兔喝完了一碗粥,白兔就自动把碗接过来,给他又盛了一碗,可是自己却没有动
过。 

灰兔沉吟了一下,看着暗下来的天,“这雪,
今天停不了呢。” 


“是啊。” 

“喂,你不是说你曾经是写故事的吗?讲个故
事给我听吧?” 

“你还不是一样?画故事而且讲故事,应该是
你讲给我听吧。” 

“不呀,你拿我的东西做了这么好喝的粥,是该你答谢我的嘛。”灰兔弄了些干草,堆成一堆,自
己舒服地躺去,“喂,讲吧,我听着呢。” 

“好吧。”白兔拿起一根棍子,看似无心地拨

\newpage
起火来,神情却一片肃穆。 

那是遥远的山上的故事。是很高很大的一座山。山里铭记着鸟、河流还有风的声音,偶尔也有孩子
们碎碎的嬉笑声。 

然而山里很少有孩子去玩耍,山是寂寞的山。


在寂寞的山里,住着两只同样寂寞的兔子。 

可能寂寞就像阳光的碎片一样,映在了兔子的心上。在那么大的山里边,两只兔子从来也没见过面
,也没说过话。也没见到过其他什么动物。 

 他们在路上走的时候,耳朵里听到的都是山内心的声音。那些风啊、水啊、孩子们的嬉笑啊,让兔子越听越难过。虽然那声音很好听,但是却是不知
道为什么,越好听就越觉得心里酸酸的。 

“很有艺术家气质的兔子。”灰兔作为一个听
众,非常懂得互动。 

\newpage

 灰心在出去找食物的时候,看到了一串兔子的脚印。尽管知道这山里有同类存在,他也没有想过
去找她。 

因为他觉得,他已经习惯了孤独,如果见到另
一只兔子,谁知道会发生什么事情呢? 


也许另一只兔子会让他根本后悔见面呢! 

所以灰心看了看脚印,就回家过自己的日子去
了。 

“幸亏是灰心,要是灰狼的话一定毫不犹豫就追着脚印过去了……”灰兔看了白兔一眼说,“你继
续讲。” 

白菜小,饭量少,但她喜欢在山里到处逛逛。当然顺便看看萝卜都长在哪儿,哪儿有大萝卜之类的
。 

结果白菜不止一次发现,她前一天看中的萝卜
\newpage
,第二天去看时,只剩一个坑了。旁边还有凌乱的兔子脚印。白菜停下来思索了一会儿,她认定这只总是先它一步拔走萝卜的家伙一定与她非常相似——连对萝卜的喜好和品位都一样。白菜寂寞的时候很想哭,但是又哭不出来。其实仔细想想,她觉得自己非常希望能遇到同类,哪怕就是远远看一眼。但是白菜很胆怯。她害怕自己被嫌弃。她想,自己是这样一只小小
的、不通世故的兔子。 

所以白菜也没有去找那只拔走胡萝卜的兔子。她在家辗转反侧,不断想,那一只兔子到底长什么样
儿? 

“可以悄悄守在萝卜附近跟踪嘛……我是说,这真是只纯洁的小白兔。我的这个想法貌似是大灰狼的思维。”灰心摸摸自己的脸,好让自己不笑得那么
明显。 

灰心为了避开白菜,不再去那片留有另一只兔子脚印的萝卜地。白菜是同样的思维,她想,该去找另一片萝卜地了。两只兔子各自从家出发,去寻找新
\newpage

的萝卜,却意外地撞见了。 

“山不转水转,真是感人的相遇。传说中的金风玉露一相逢,便胜过人间无数。”灰兔依旧喜欢冷
不丁插上一句。 

两个人中间隔着很大的距离,当看到彼此后,就停了下来,都想扭头就走,可是不知道为什么,却
没有挪动脚步。 

白菜一紧张,就喜欢拿一只脚蹭另一只脚。这种小动作却给她带来了麻烦,她把左腿放下来时,被
一大团草给绊着了,很丢人地摔倒了。 

灰心第一反应是,这是一只多么笨的兔子啊!
不过,他立刻跑过去扶起了白菜。 

那一刻,他们没有了距离。掌上的温度让两只
兔子心里都很舒服,好像有些什么东西融化了。 


\newpage

“都害羞得很可爱呢。” 

是啊。于是,两只兔子成为了朋友,此后,一天都没有分开过。慢慢的,白菜发现,灰心是只很出色的兔子,无论是找萝卜还是做窝都干得极漂亮,就连人懂得的那些知识,灰心也都很精通的样子。这让白菜很自卑,她觉得平凡的自己是不配拥有这么出色
的朋友的。 


白兔的声音一点点沉下去—— 

有一天,灰心和白菜被一个猎人追杀。白菜跑得太慢,而灰心本来有很多逃跑的机会,却因为要停
下来等白菜而放弃了。 

猎人追上来的时候,灰心扯着白菜跳进了一个他早就打好的洞里,才逃了这一劫。可是白菜却非但没有感激灰心,反而说了这样一句话:“以后,我们
再也不要在一起了。” 


灰兔抱住胳膊,不再说话。 

\newpage


一阵沉默过后,灰心问:“为什么?” 

白菜含着眼泪说:“因为跟我在一起,会浪费你许多时间。而且,我也需要独立闯荡世界,需要时
间来磨练我自己啊!” 

白菜心想,她要等到与灰心一样强大的时候,
就再也不用担心会连累到灰心了。 

可是灰心却气疯了,他反复问一句话:“为什
么明知道这样会让我伤心,你还要这么说呢?” 


“我不知道!”白菜颤抖着大声地答道。 

“噢,那真是抱歉,我这么在乎你,你居然都
看不出来。我现在知道了。” 

可是,可是……一点都开心不起来。白菜明白,这是在辜负朋友的友谊啊,但是她又有什么办法呢
? 

\newpage


眼泪就快流下来了。白菜却倔强地忍住了。 

从前有过很多次,白菜也有自己的朋友,可是当他们发现她是一只普普通通、一无是处还尽拖人后
腿的兔子后,就慢慢疏远了她。 


“你不信任我!” 


“不……不信任?”白菜问。 

“对,不信任!”灰心答道,“你认为有一天我会抛弃你,觉得我会像别的兔子一样,可别的兔子
是别的兔子,我是我啊!” 

白菜愣住了,她真的没有这样想过。先前她只是在不信任自己而已,此刻才发现,原来真的也没有
信任过灰心。 


这么想着,白菜就软了下来。 


\newpage

“白菜,我也有话要对你说。” 

白菜抬起头,有些迷茫地看着灰心,等着下文

“从此以后,我也不会再和你一起了!我每次都只给别人一次信任的机会!我很害怕哪一天你想不通再次说要离开,你刚才让我非常伤心……其实,我本来决定要和你成为唯一的知己的,可我现在不敢再
信任你了。” 

“那白菜又如何了呢?”灰兔前倾着身子,专
注地看白兔。 

“白菜啊……白菜又变回了那只弱小的兔子,另外,她还多了一份沉重的内疚。倒是白菜,她很想
知道灰心后来的情况呢。” 

灰兔淡淡地笑了:“灰心吗?他后来很长时间一直独来独往,直到他又遇到了另一只特别的兔子。其实希望永远不该被放弃,于是灰心再次拥有了朋友
。” 

\newpage


“哦……你怎么知道的?” 

“我就是灰心。”灰兔摆了摆耳朵,将他的皮毛故意给白兔看。“你讲的不就是你自己的故事吗?
所以你是白菜吧!” 

 白兔低下头来,算是默认。雪的声音一片一片飘过来,细碎的,干燥的。许久,她低声说:“那
还好……那灰心和那位朋友……还好吗?” 

“实际上,并没有比我和你的故事好多少。”灰心眯起眼睛看着火苗,眼神迷离,“只是,当不信
任出现的时候,我又多给了她一次机会。” 

白菜眼睛里迅速有光芒一闪而过,迅疾的,令人无法察觉,那水滴一般的光芒很快渗入了眼角两侧的皮毛。“这样,非常好呢……真的,很好呢。”然
后,她又悄声说,“那只兔子,好幸运……” 

灰心站起来,张望着门外,“啊,雪停了呢。

\newpage
而且,讲着讲着故事,居然就过了一个晚上呢。” 


“你要走了吗?”白菜突然察觉到了什么。 

“是的,我的那位朋友还在等着我呢。你呢?
还是独自住在山上?” 

“啊?不是不是,我和家人生活在一起,他们
对我非常好。总之很温暖。” 

灰心仔细看了看白菜,笑了一下,“那么,我
这就走了。” 

白菜开始专心拢着快要灭掉的火,她不断往火里添干草和木头,弄得自己满脸是灰,而且剧烈地咳
嗽起来。 

 “其实我有件事不明白啊,白菜,你怎么做
起粥来了呢?” 

 白菜又露出她惯有的甜美而带些忧伤的笑容,她看着咕嘟咕嘟的锅子说:“其实,我一直想做的
\newpage
职业就是这个啊。粥总是给人带来温暖,无论是谁,
喝下去都暖暖的,不是很好吗?” 


“哦。” 

“灰心,再喝碗粥吧,早饭是非常重要的呢。

灰心看见白菜又把碗递了过来,“原来昨天晚
上还没喝完啊?” 


“啊,嗯……总之,喝了暖暖胃吧。” 

灰心接过,那碗粥的香气让他觉得有些恍惚,
忍不住一饮而尽,“呀,这味道……是肉?” 

“不是肉!我怎么会放肉呢?我是个只做素粥
的厨师!好了好了,你快走吧!” 

“呀,这会儿又赶我走了吗?喂,你这人毛病
真多,怎么就喜欢赶我走啊?!” 

\newpage

“啊,这次不同嘛,你还有人等着呢。不要让
人伤心啊!” 


“那我……就走了!” 

白菜回过身去,开始默默收拾餐具,一边大声
答着:“再见!” 

直到灰心的脚步声渐渐消失,白菜才慢慢回过头来。看着灰心的两排脚印——到底那是什么呢?是
留下了什么,还是放下了什么呢? 

白菜把火踩熄,把碗收起来,再次背起自己的
锅子。 

开始走路时,皱了皱眉头,她不得不停下来,把用围裙遮住的地方掀起来。那片地方一片濡湿——
全都是血。 


果然,真的很疼呢! 

\newpage

白菜不得不喘息着,用围裙把伤口狠命包扎起来。在来的路上,白菜遇到了猎人设下的夹子,她拼着命把腿拔下来的,可是却连皮带肉一起扯下了一块

她把那块肉用雪擦洗干净放进包里,也不是为了别的,就是想自己的东西终究是不要留在荒郊野外
的好。 

不过,就在刚才,在给灰心做最后一碗粥的时
候,她把那块肉悄悄丢了进去。 

丢那块肉进去的时候,白菜仿佛看到了她和灰心做朋友时的情景。那一天是腊八,人类都在喝粥。

白菜乐颠颠地跑去找灰心,“灰心,灰心,你
喝粥了吗?” 


“今天流行喝粥吗?” 

“今天是腊八呀,腊八要喝粥啊,我弄了一点

\newpage
来,你喝了吧。” 

灰心把那小小的碗里的粥舔了个干净,最后皱起了眉头,“怎么连点肉末儿都没有啊,至少有点油
花也成吧。” 

“不要紧的。明年,明年这时候我给你做粥喝
。我给你做带肉丁的粥,只给你喝哦。” 

“那么……明年快点到来吧!”灰心向往地看着天空。仿佛时间可以一下子缩水,把下一个腊八立
刻送到他面前似的。 


可是……可是,没有明年了啊。 

到了第二年腊八的时候,灰心不在白菜身边,而白菜却默默学着做粥。她做得越来越好,却坚持着
不肯做一碗带肉的粥。 

白菜想到这里,看了看远方。大地一片素白,
偶尔有飞鸟掠过,呼啦——很快的一下。 

\newpage

白菜带着伤,朝与灰心那串脚印完全相反的方向走去。前边并没有什么家在等着她,白菜只有她自己,一直以来,因为太害怕会再遇到伤害别人的事,
她坚持独来独往。 

在那座破庙即将离开白菜视野的时候,白菜再
一次停下来回望大地。 

有风吹来,将她白色的皮毛掀动。白菜的伤心
,是谁都不可能看得到的。 

这一切,都发生在那漫长的漫长的几年里。开心的时候,痛哭的时候;天晴的时候,天阴的时候,白菜一直用它那有点忧伤而疲倦的眼神默默注视这个
世界——很大很大的世界……无边无际的世界…… 

把世界包容在其中的那颗心也很大很大,把心
包容在其中的悲痛也无边无际…… 

白菜看着灰心的脚印,轻声说:再见了,再不

\newpage
相见…… 


那是多久以前的事了? 

曾经有两只兔子在落雪的日子里等待着春天。

“春天的时候,美丽的事物就会复苏了呢。”
灰色的兔子说。 

“若是春天真的能予人重生……就好了啊……
”白色的兔子有点忧伤,迟疑着答道。 


那是灰心和白菜。 

那是心灵里最初和最后的光芒,明亮,却惟独敢碰触的光芒。

\end{document}
