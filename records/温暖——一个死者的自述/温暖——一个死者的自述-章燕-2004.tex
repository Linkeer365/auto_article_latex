\documentclass{article}
\usepackage[utf8]{inputenc}
\usepackage{ctex}

\title{温暖——一个死者的自述\footnote{Click to View:\url{https://web.archive.org/web/20230411132344/https://paste.ubuntu.com/p/nPsMdyCbWb/}}}
\author{章燕}
\date{2004}

% \setCJKmainfont[BoldFont = Noto Sans CJK SC]{Noto Serif CJK SC}
% \setCJKsansfont{Noto Sans CJK SC}
% \setCJKfamilyfont{zhsong}{Noto Serif CJK SC}
% \setCJKfamilyfont{zhhei}{Noto Sans CJK SC}
% \setlength\parindent{0pt}

\begin{document}
\CJKfamily{zhkai}

\maketitle


\Large

我叫程姣,原来是安徽省绩溪县临溪小学三年级的学生。和其他同学一样,我憧憬未来,也曾用多彩的鲜花编织过理想的花环。可是无情的再生障碍性贫血过早地夺去了我的生命。我是一个不幸的人,可我又是一个幸福的人,在我短短的几年里,我享受
到了人间真情,体会到了祖国大家庭的温暖。 

去年暑假的一天,我的大腿上突然出现了两个紫块,过了几天,手臂上也出现了许多斑点,并晕倒在地。到屯溪抽骨髓化验,我得了再生障碍性贫血,转送到上海大医院治疗。医生安慰我,常夸我勇敢。
护士给输液时,生怕弄疼了我,总是小心翼翼的。 

回到家里全村的男女老少都来看我。张家大娘送来了鸡蛋,说是给我补补身子;李家大伯送来钱物
\newpage
,说是给我看病;程家大姐送来水果,说给我解馋……连来村里玩的两位外地青年也给我送来了五百元钱。记得一天中午,太阳火辣辣地炙烤着大地。我无意中向妈妈说:“我想吃鱼。”来看望我的章爷爷听见后,二话没说就走了。两小时后,年近古稀的章爷爷提着十几条活蹦乱跳的鱼,笑嘻嘻地对我妈妈说:“
这是刚从自家鱼塘里钓来的,快烧给孩子吃吧!” 

望着章爷爷皱纹里浸满的汗水,我的鼻子酸酸
的。 

我忘不了学校的老师、同学对我的关心和帮助。暑假里,余校长常常顶着烈日,驱车来看望我,为我的医疗费四处奔波;班主任程老师陪我看病,下屯溪,往上海,消瘦了一大截;其他老师和同学也都纷纷来看望我……开学的第一天,我躺在病床上,大队长递给我一个大纸包,我打开一看,原来里面是一大堆钱:有十元的,五元的,也有一角的,两角的。面对同学们的颗颗爱心,我的眼泪一下子夺眶而出,再也抑制不住了,真想飞到学校说一声:“谢谢。”在我去世的第二天,学校又从有限的经费里拿出六百元
\newpage

慰问金,送到我家里,安慰我的亲人。 

我的妈妈失去了自己养育九年的孩子,真是痛不欲生,整天以泪洗面,同学们就利用节假日或课余时间,来陪伴她,用一颗颗天真的童心愈合一位母亲
流血的伤口。 

妈妈,别哭了,看见了他们就像看见了我,他们就是您的亲儿女,会像我一样爱您的。我们的祖国是一个温暖的大家庭啊。

\end{document}
