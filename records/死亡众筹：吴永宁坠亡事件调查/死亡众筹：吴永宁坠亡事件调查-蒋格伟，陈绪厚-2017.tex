\documentclass{article}
\usepackage[utf8]{inputenc}
\usepackage{ctex}

\title{死亡众筹:吴永宁坠亡事件调查\footnote{Click to View:\url{https://web.archive.org/web/20221012104925/https://www.thepaper.cn/newsDetail_forward_1907239}}}
\author{蒋格伟,陈绪厚}
\date{2017-12-16}

% \setCJKmainfont[BoldFont = Noto Sans CJK SC]{Noto Serif CJK SC}
% \setCJKsansfont{Noto Sans CJK SC}
% \setCJKfamilyfont{zhsong}{Noto Serif CJK SC}
% \setCJKfamilyfont{zhhei}{Noto Sans CJK SC}
% \setlength\parindent{0pt}

\begin{document}
\CJKfamily{zhkai}

\maketitle


\Large

11月8日13时许,吴永宁在湖南长沙华远国际中心坠楼。当地警方通报称,其死亡原因系高空坠亡,排除他杀。如果没有这次意外,他将在11月10日携继父和母亲前往湖北女友金金(化名)家
送彩礼。 

吴永宁,长沙宁乡市人。过去十个月时间里,他用ID名“极限咏宁”在火山小视频、陌陌、YY、花椒、快手等多个视频直播网站陆续发布了301条自己攀爬地标性建筑的视频,地点涉及重庆、长沙
、武汉、宁波、上海。 

惊险、刺激的表演,让他拥有了百万粉丝和一些广告商。吴永宁的继父冯福山回忆,11月4日吴永宁回到老家告诉他,自己要出名了,“我会有很多
\newpage

钱,等挣了钱,带妈妈重新去治病”。 

高空挑战、更新视频账号、洽谈生意、收款……澎湃新闻了解得知,成名后的吴永宁,不再像以往
那样在微博上频繁记录自己的情绪,他开始忙碌。 

与数年前在工厂打工、当群众演员、替身相比,他仿佛看到成为自己期待的“土豪”的大门已经开
启。 

11月8日上午9点22分,吴永宁接亲属电话,得知已经为他筹集了2万元作为去武汉的彩礼(
共预计8万元礼金)。 

约4小时后,吴永宁独自来到华远中心。各项准备工作完成之后,他攀着62楼楼顶墙体边缘,费力地做了两个引体向上。然而,之后两次竭力上爬均未成功。他往下看了这个世界最后一眼,攀住墙体边
缘的双手无力地松开,从高空堕落。 

同样喜欢极限高空挑战的丁鹏(化名)是吴永
\newpage
宁朋友,他向澎湃新闻(www.thepaper.cn)叹息,“也许每一个打赏过永宁的粉丝和催促过他爬楼拍摄的广告商,都参与了这场‘死亡众筹
’”。 

一、辞职:“3500元一月,我怎么给妈治
病,养老婆” 

吴永宁的人生路与多数同龄人相比,是不幸的
。 

1991年4月,他出生在宁乡南田坪乡锡福村的一个农村家庭。其继伯父冯胜良介绍,吴永宁读完高二就辍学了。辍学原因一是“心思没有在学习上,成绩差”,另外一个是,这一年,他的生父患病离
世,母亲精神又出现问题。 

冯胜良认为,父亲的去世,对吴永宁打击很大,尽管他从未在这位继伯父面起提及这些事情。在吴永宁的微博里,他曾写到,“我亲爱的父亲!当我遇到挫折和困难时,我第一个想到的就是您,当我遇到
\newpage
快乐大转盘时,我第一个告诉的也是您。我的父亲,我爱你。爸,你在另外一个世界还好吗,你儿我想你
了!” 

继父冯福山说,妻子何小飞与自己结婚之前,就有精神方面的问题。2009年6月,何小飞还办
理了残疾证,写有“精神残疾人”字样。 

12月10日,何小飞见有记者来访,听到记
者用本地方言交流,她起身走出了卧室。 

何小飞面色憔悴,低着头呆呆看着手机里与儿子的合影。当提及儿子生前过往,她缓缓抬头,右手摸着额头,试图想起点什么,然后又说“我记不起了
,记不起了.........” 

父亲去世后,吴永宁随母亲改嫁到冯福山家。尽管何小飞每日服食的精神病药物费由政府承担,但
家里还是过得很拮据。 

在邻居眼里,冯福山是一位不善言辞的农民,
\newpage
种田干农活之余,他还到工地干泥瓦工谋生。在他的计划里,是想带着继子吴永宁一起进工厂打工挣钱,
为他娶个老婆,传续香火。 

在与何小飞结婚四年里,吴永宁见到冯福山尽管还是喊一声“爸爸”,但不愿与他多交流。每次外出打工回来,吴永宁都会塞几百元给他妈妈,然后去
亲生奶奶家住。 

“大概从2012年上半年开始,他(指吴永宁)开始跑北京和浙江当群众演员。”继伯父冯胜良说,吴永宁自小性格内向,不愿与陌生人说话,但对妈妈很孝顺。不知道从何时开始,吴永宁对武术产生了兴趣,经常躲在家里练习一些武术基本功,却不在
外人面前展现,也不愿多提自己当群众演员的事。 

与何小飞结婚后不久,2014年年初,冯福山见继子外出务工没挣到钱,便要求他一起到广东中
山某知名空调企业生产线上班。 

工厂打工,时间长,强度大,但每月能拿到3
\newpage
500元,“两人加起来就是7000元,我约定每个月给他800元零花钱,其余由我保管,攒够了钱
就回去给他娶老婆”。 

工作数日后,吴永宁担心母亲独自在家生活不能自理,提出要接母亲来中山同住。他的提议被冯福山否决了,“我们刚来,等安定下来再去接你妈妈”

800元零花钱,吴永宁大部分拿去买彩票,却没中过奖。冯福山说,自己的计划并没有得到继子的认可。在工作20天后,吴永宁瞒着继父向工厂递交了辞职信。直到吴永宁在工厂宿舍收拾行李时,他
才告诉继父自己已经辞职。 

“3500元,这么低的工资,我怎么给妈妈治病,怎么养得起未来的老婆?”吴永宁临行时向继
父说了自己辞职的理由。 

二、群众演员:“可以说我是跑龙套的,但不
可以说是臭跑龙套的” 

\newpage

不幸家庭成长的吴永宁,一直努力向上“爬”
,试图改善自己的经济状况。 

中文名:吴永宁,别名:八戒,身高:176cm,体重:55kg……职业:演员;意向:跟组、替身、特约、小角色;特长:街舞、咏春、跑酷、空翻速成;代表参与作品:《雪豹坚强岁月》饰小勇、《犀利仁师》、《枪火》、《欢喜县令》、《新神
雕侠侣》、《与狼共舞》等等。 

这是一份保存于吴永宁QQ邮箱里的求职简历,这份简历在2013年11月到2016年5月之
间频繁显示“已发送”。 

赵小旭(化名)是吴永宁在横店影视城认识的
朋友,他们同为群众演员。 

“只要能跟组,能上镜挣钱,吴永宁都会去争取。甚至一些危险的替身活,他都答应。”赵小旭回忆,吴永宁是一位只要有机会跟剧组就会去争取的年轻人,“和他一起跟组几个月,他只说过自己是长沙
\newpage

宁乡人,却从未说过自己家里情况”。 

吴永宁辞职后,有邻居电话给冯福山说,“你老婆睡的房间24小时亮着灯,一天经常就是一餐饭
”。 

闻讯而归的冯福山发现,自己外出时买的20斤米,一个多月后,剩下了一大半,而且米已经发霉
。从此之后,冯福山很少再出远门务工。 

继伯父冯胜良说,从与吴永宁交流的一些只言片语中判断,他当群众演员的几年时间里过得很辛苦
:吃盒饭,廉价薪酬,得不到尊重。 

吴永宁微博记录的群众演员生活佐证了冯胜良的判断。2015年3月,吴永宁在微博上发布一条“你可以说我是跑龙套的,但是你不可以说我说是臭
跑龙套的”。 

微博后的第二条即贴出了四张剧组照片。其中一张剧照上,他脚上踏着一双黑色布鞋,穿着一身军
\newpage

装,腰间挂着褪色的水壶,满身是血。 

赵小旭说,每个群众演员都心中藏着一个王宝强,想一步一个脚印往上爬,从群演到主演,最后衣锦还乡。吴永宁钱包里曾放着一张王宝强做群演时的
照片,以此来勉励自己。 

“事实上,群演能做到王宝强的机会,不及万分之一。不是肯拼就够了,没有背景和好的出身,就得靠伯乐,但横店排着队的群演,哪有那么多伯乐?
”赵小旭感叹。 

在横店的群众演员,一天下来薪资不会超过80元,工作时间8小时,而编外武行则有100元一天的,挨打武行200元一天。赵小旭回忆,吴永宁为了更多表现和多拿报酬,经常会主动要求充当挨打
武行。 

2014年1月7日,当了大半年群演的吴永宁在微博名为“演员吴咏宁”的微博中写下了对自己的评价:“……人生有大起大落!为什么至今我还落
\newpage
着呢?我想拼一把,可我已不知道从何拼起!演戏演技烂,我不是专业的,我没有学过表演。既然我来到了这个世界,不在这个世界留下点什么,我觉得对不起自己的生命!我家庭条件不好,不是土豪,我只是
向着土豪的目标迈进!” 

三、极限:“第一我不敢说,但我一定是玩得
最狠的那一个” 

数年的群演和替身生涯并不能帮吴永宁挣钱给妈妈治病,更谈不上成为“土豪”。而演武行练就的
一身本领,为他开启了另外一扇门。 

今年2月10日,吴永宁第一次将自己极限挑战视频上传到火山小视频。在一栋大楼的十层楼顶,他双脚踏着一辆白色平衡车,谨慎缓慢地在楼顶边缘滑行,身子微微向室内一侧倾斜。与此前当群演的小
视频相比,脸上少了一些俏皮。 

在这条视频后面,不少粉丝留言表示担心“孩子,这种玩儿法!失误只有一次!谨慎!”对此,吴
\newpage
永宁回复“嗯”。不过,多数网友给出了“厉害”、
“666”、“牛牛牛”等赞许的评价。 

据媒体报道,这次表演,为吴咏宁带来了131.6元的收入。同时,他把微博名改成了“极限咏
宁”。 

随后,吴永宁前往重庆、长沙、武汉、宁波、上海多地,攀爬从100米至468米高度不等的建筑高楼,留下301段极限挑战视频。他曾在微信公号“极限咏宁”里写到,“极限运动国内第一人我不敢说,因为现在国内玩这个的实在太多了,但我一定是玩得最狠的那一个,因为我每天都在爬,我是在玩
命儿”。 

对于极限挑战,吴永宁却瞒着家人,只字未提

他生前有两部手机、两个手机号,其中只有一部手机是加有母亲和继父微信号,且极限挑战的内容都对父母屏蔽。直到今年9月份,冯福山买了新手机重新加了吴永宁微信号。吴永宁不知是继父,并没有
\newpage
设置屏蔽,他极限挑战的视频和照片才被家人知晓。
 

“极限是什么意思?”“那些视频和照片是怎
么回事?”冯福山质问吴永宁。 


“工作需要。”吴永宁回复。 

“我做泥瓦工,超过两米高度都要做安全措施
。”冯福山有些生气了。 


“以后不会这样了。” 


吴永宁再次把继父冯福山屏蔽。 

在多个网络平台中,吴永宁的账户信息介绍重
点强调:高空挑战无任何保护。 

如吴永宁在美拍的账户为“极限-咏宁”,其账户介绍信息为,“国内无任何保护极限挑战第一人,挑战全世界高楼大厦”,并可楼盘炒作,商业合作
\newpage
。他的微博账户@极限咏宁 在简介中写着,“国内极限高空运动挑战第一人,无任何保护的情况下完成每一个能完成动作,认真的拍好每一个挑战视频,目
标:无任何保护挑战全世界高楼大厦。” 

从事食品生意的叶革(化名)关注高空挑战有七八年,他也是咏宁的一位忠实粉丝。叶革向澎湃新闻表示,从相关视频可以看出,咏宁没有团队、独行,不做任何保护措施,但国外很多都有保护措施的,而且他身材较为瘦弱,常常做一些高危动作,看起来
惊心动魄,这超出了他的能力。 

“他的视频比国外的都吓人。为了博取眼球,他真的是拼了。”叶革说,咏宁坠亡后,他已取消关
注其账户。 

吴永宁坚持高空挑战不用保护措施的原因,尚不清楚。澎湃新闻注意到,在吴永宁坠亡前,众多网友看了他的挑战视频后,认为过于危险,曾留言提醒
过他。 

\newpage

今年6月,吴永宁曾上传一段失败的“坠楼”视频,并配文“失误,上不去了”。视频中,他在高
处做引体向上动作时掉了下来。 

对于网友们的疑问,吴永宁回复说,“没,视频是真的,下去也是真的,只是只有十几米高。”很多粉丝劝他要注意安全,他当时回复称,“不会,百
分百把握。” 

四、曾经的追随者:吴永宁的我行我素让人感
到害怕 

在业界拥有影响力后,吴永宁很快拥有了自己
的追随者和合伙人。 

他生前两部手机的聊天记录显示,在小有名气
之后,他与重庆籍的聋哑人童虎来往甚密。 

聊天记录显示,吴永宁在生活上对童虎颇为关照,不但教他如何吸引广告商,如何与直播平台打交

\newpage
道,还多次借钱给童虎。 

9月11日下午13点35分,吴永宁与童虎
的聊天记录显示: 


”你有多少钱?“ 


“1000” 


“够了” 


“不,很快花完了” 


“住不要你花钱,房钱我出。” 

两人之间聊天,一直延续到11月6日上午1
0点多。 

冯福山称,吴永宁曾向家人解释,自己之所以什么事情都愿意帮童虎,是因为童虎是聋哑人,“我
不帮他,就没有人帮了”。 

\newpage

吴永宁坠亡后,童虎接受媒体采访时多次提到,“每次看见他这些危险动作,我都看不下去”。童虎称,自己告诫过吴永宁动作不要太危险。他的我行我素让童虎感到害怕。“如果他真的掉下去了,我怎么跟他的家人交待?”童虎考虑再三,最终离开了吴
永宁。 

如果说称呼“咏哥”的童虎是追随者,罗齐则
是吴永宁生前的“合伙人”。 

罗齐与吴永宁一样,同为宁乡人。同时,罗齐还是“长沙星启源电子科技有限公司 ”的法人代表
。该公司直接负责“极限咏宁”的微信账号。 

冯胜良介绍,吴永宁此前认识了罗齐的外甥。罗齐的外甥原来在传媒公司工作,懂视频剪辑,与吴永宁是搭档。通过罗齐的外甥,吴永宁接触到了罗齐

罗齐称,他和吴永宁只见过两次面。第一次是在8月的一个晚上,随外甥一起驱车到吴永宁家里玩。针对冯福山多次对媒体提到的,吴永宁的最后一次
\newpage
高空极限运动,就是给罗齐的公司做视频一事,罗齐
否认。 

冯福山说,8月份,罗齐等几人曾来家里找吴永宁玩。其间,罗齐谈到一个8万元的业务,公司和吴永宁各分4万元。冯福山追问是什么业务,众人马
上闭上了嘴。随后,去了二楼房间继续谈。 

12月10日,澎湃新闻查阅吴永宁的两个手机,没有发现任何吴永宁与罗齐的交集。吴永宁生前
的一部手机,11月8日前的所有短信不见了。 

12月13日,澎湃新闻再次打开“极限咏宁”公众号,发现账号内的所有极限运动视频均已删除。该账号10月24日发布第一条推送,最后一条于11月8日发布。账号运营16天,其间9条推送全
部是关于吴永宁极限挑战的内容。 

五、生意:“每星期发一次,一个月发四次,
一万元” 

\newpage

9月12日下午3点,某直播平台6.0版本
正式上线,给用户和粉丝特意准备了一份礼物。 

活动方案显示:产品经理发布会现场连线“咏宁老师”,要求“咏宁老师”站在一处大楼楼顶,将手机交给同伴,露出在大楼楼梯上事先贴好的“某直播平台6.0上线”,并同时做两个动作,全程都要
露出直播平台字样。 

这是存于吴永宁手机内的一份某直播平台官方
邀请吴永宁直播表演的方案。 

事实上,从手机内保存的聊天记录和邮件往来显示,吴永宁与国内热门的数个直播平台都有着较为紧密的联系。他们或约拍视频,或邀请入驻,在言语措辞上都显得尤为尊重,多用“您”、“咏宁老师”
等词汇。 

10月30日下午,一位名为“丙凡铲屎官 
小咖秀”与吴永宁微信聊天显示: 

\newpage

“我这边最近YY在招募,要不要试试呢?就是一个月(发)20个有效视频。1000块钱?”


“我与陌陌已经签约了。” 

手机内保存的一份“8月21日-9月20日视频奖励前三”绩效显示,“极限咏宁”的陌陌号在这个时间段排名第二,税前总业绩为5305.84
元。 

群众演员赵小旭介绍,事实上,吴永宁从事极限挑战后,他的广告收入远比直播中粉丝打赏收获多

吴永宁的手机资料显示,从7月下旬开始,与他洽谈广告的客户越来越多。他们多为户外品牌或运动饮料的推广,要求是,吴永宁在极限高空挑战中,穿戴着指定品牌的鞋子、头套、上衣、围脖等。在挑
战中,要求看似无意地将客户的品牌展示出来。 

这些客户均会在洽谈后,给吴永宁寄送样品,然后打预付款。随后与吴永宁约拍摄地点,催促加快
\newpage
拍摄,并要求赶紧后期制作,经确认后再发到他直播
账号展示。 

其中,一名疑似某知名运动饮料广告方在给吴永宁支付宝打款后提出,“想要这些在塔尖上的,您看可以的话……”“拍三段短视频,你分三次发给我

一位重庆户外运动鞋广告商与吴永宁的广告合
作,体现了他与广告商交往的日常。 

10月28日,重庆广告商给吴永宁寄送鞋子后,开始催促他去拍摄作品。吴永宁称,自己手上暂
时没有作品了。 

29日,吴永宁收到鞋子,去选地方拍广告。当日下午4时21分,吴永宁告诉重庆广告商,作品已经拍好,晚上能剪辑好发布。吴永宁提到,“广告包月1万元,只给你一个人做。”“每星期发一次,
一个月四次,发到火山小视频上。” 

30日,广告商抱怨,广告没有效果,“鞋子
\newpage
5双都没有卖到”,“粉丝100人都没有涨到,还
比之前找快手(做广告)的要贵”。 

与广告商谈合作时,吴永宁表现出他善辩的一
面。 

广告商埋怨价格太贵时,他会说“一个月1万
元,我已经给你优惠好几千了”。 

为了取消广告商的疑虑,他经常发一些自己点击较高的视频链接给客户看,还会说,“我是做武行出身的,也拍过电影,照片和视频肯定拍得很漂亮”

随着吴永宁的极限挑战在各大视频网站被追捧,运动类品牌的广告商并不想和吴永宁搞僵关系。即便那位聊天中抱怨“广告效果不佳”的广告商,聊天
最后还是说道“我看好你,人气越来越高了”。 

11月8日下午1点30分左右,吴永宁来到湖南长沙华远国际中心楼顶,他攀着62楼楼顶墙体边缘,费力地做了两个引体向上后,两次竭力往上爬
\newpage

,但均未成功。最后,不幸坠楼身亡。 

12月6日下午,吴永宁曾经的队友在短视频平台上发布了一条动态消息:国内极限第一人,行走在生死边缘,最火帅小伙“极限咏宁”失手……同一天下午,他的女朋友金金公开承认男友失手去世的事
实。 

澎湃新闻发现,随着舆论的发酵,各大直播平台上已无法检索到“极限咏宁”账号及极限挑战视频。各平台也先后回应媒体称,对于吴永宁的遭遇表示惋惜和同情,此类极限挑战视频目前未被中国现行法律法规所禁止,平台会根据实际情况对审核政策进行
完善与改进。 

“他太想证明自己,太迫切需要钱了。”对于吴永宁的失手坠亡,好友丁鹏尤为痛心,他直言不讳地说,“也许每一个打赏过永宁的粉丝和催促过他爬
楼拍摄的客户,都参与了这场‘死亡众筹’”。 

一位不愿具名的国内知名视频平台工作人员告
\newpage
诉澎湃新闻,同样的高空极限挑战视频,吴永宁会在微博、微信公众号、美拍、火山小视频、快手等众多平台分发,且还在某视频平台做过直播,粉丝众多。

保守估计,吴永宁在各大平台拥有的粉丝数超130万。其中,吴永宁的微博账号有4万多粉丝,微信公众号单篇阅读数最低也破8000,美拍账户有24万粉丝。目前,快手、火山小视频已无法检索到吴永宁的账户,据快手工作人员向澎湃新闻提供的截图,吴永宁在快手有2.5万粉丝;另据微博网友晒出的截图,吴永宁在火山小视频有99.1万粉丝

在吴永宁的微博评论中,多数网友表示惋惜之情,部分网友认为,各视频平台发布咏宁的高空挑战视频过于危险,不宜发布传播。少数网友称,早在事
发之前,就曾在相关视频平台举报过此事。 

吴永宁坠亡后,部分网友认为相关视频平台存在责任,并提出质疑。12月9日至10日,美拍、快手、火山小视频先后回应澎湃新闻称,对于吴永宁的遭遇表示惋惜和同情,此类极限挑战视频目前未被
\newpage
中国现行法律法规所禁止,平台会根据实际情况对审
核政策进行完善与改进。 

美拍表示,为避免类似事件发生,平台不应再鼓励此类内容。快手表示,吴永宁的快手账户已于今年9月被限制传播。火山小视频表示,已和家属沟通
,会尊重家属意愿对相关视频进行处理。 

澎湃新闻检索发现,目前,美拍、快手和火山
小视频已无法检索到吴永宁的账户及其相关视频。 

据中新网12月9日报道,面对国内外不断有人参与其中的“爬高楼”举动,如何管理成为重中之重。中国人民大学危机管理研究中心主任唐钧此前在接受媒体采访时表示,要尽快建立健全与户外攀爬有关的法律法规。出台一个类似于负面清单的管理制度,对于一些比较危险的爬高楼行为要明确禁止,包括
明确坚决禁止做这些户外活动的地方。 

唐钧表示,在此之外也还要注意人性化管理。从目前的情况来看,爬高楼似乎正在成为一种潮流,
\newpage
在国外也有人爬高楼。可以考虑找一些适合攀爬的地方予以开放,在确保安全的前提下,也可以做一些类
似的户外活动。 

上海政法学院社会管理学院院长章友德向澎湃新闻表示,在中国传统文化中,冒险精神比较缺少,吴永宁通过高空挑战成为网络红人,这并不奇怪,迎
合了大众的追求冒险和刺激的心理。 

章友德认为,粉丝的围观以及围观带来的收入,很可能导致了一种结果,促使吴永宁不断增加挑战
难度,从而增加了风险。 

在章友德看来,假如要进行类似的极限挑战,需要做充分的准备,进行长期的训练,特别要做好防止意外的预案,同时也要有依法依规的自觉性,“不
能为了收入而给自己带来不可想象的风险”。 

中山大学传播与设计学院副教授周如南表示,吴永宁的坠亡,让他想起两个概念,即“景观社会”和“娱乐至死”,在景观社会中,人们处于资本逻辑
\newpage
下的景象消费之中,而娱乐至死精神则进一步消解生活意义本身。在直播技术进步的前提下,网红经济、猎奇心态、围观消费糅合在一起,让死亡变成围观者的狂欢,而死者生命消逝本身也成为一种消费主义下
的宿命和隐喻。 

11月10日,一位微信名为“某国际名表”
的广告商从下午5点到第二天中午,一再追问: 


“在不在?” 



“?” 


“不赚钱了?” 



\end{document}
