\documentclass{article}
\usepackage[utf8]{inputenc}
\usepackage{ctex}

\title{水自无言\footnote{Click to View:\url{https://web.archive.org/web/20221025123432/https://rentry.co/uwte9}}}
\author{pp}
\date{}

% \setCJKmainfont[BoldFont = Noto Sans CJK SC]{Noto Serif CJK SC}
% \setCJKsansfont{Noto Sans CJK SC}
% \setCJKfamilyfont{zhsong}{Noto Serif CJK SC}
% \setCJKfamilyfont{zhhei}{Noto Sans CJK SC}
% \setlength\parindent{0pt}

\begin{document}
\CJKfamily{zhkai}

\maketitle


\Large

王园子的“哑巴”是特指的,就是那个老在小学门前放羊的小男孩,黑油黑油的,浑身闪亮着一种光泽。那光泽从他欣喜而好奇的眼神、一蹦三跳的走姿和笑吟吟的面容里无遮无拦地流泻出来。与普通孩子相比,哑巴只是少张嘴,少双耳朵,少个名字。
 

哑巴当然不是真的没长嘴巴和耳朵,只不过长与不长没什么两样,他不会说话,也听不见别人说话。但是不会说话的哑巴,与王园子小学那帮能说会道的孩子之间真正的交往,却是从相骂相打开始的。生
活就是这么奇怪。 

哑巴家住在王园子小学门口。王园子小学周围有四条河,蜿蜿蜒蜒,互联互通,像顽皮的孩子用四
\newpage
根火柴棍搭成的歪歪斜斜的井字。庄户人家傍河而居,零零星星地散落在那四条河外围。惟有三户人家并排在小学正对门,和学校一起被圈定在水中央。这三
户人家分别是哑巴、惠惠和秋桃家。 

王园子的孩子打小就在庄稼地里撒野,没有幼儿园的学前训练,到了年龄直接关进王园子小学。开始骨头自然作痒得难受。放学铃一响,他们就像熬过了冬寒第一趟下水的鸭鹅,不仅扑腾腾地动作,喉咙里也嚷出放肆的声音。男孩子互相冲撞,弄场架打打,或者往河里漂石子。女孩子唧唧喳喳,她们沿着河堤回家,不是发现河岸的芦苇抽穗了,田螺上岸了,
就是看到浅水滩里有螃蟹在打盹。 

河堤是新修的,宽崭崭,平坦坦,月光下看起来,像上好的滑亮布缎。除了惠惠和秋桃,王园子大部分孩子每天都要顺着河堤往返在家和学校之间。但对哑巴来说,河堤只是他下地的通道,更多的时候,
他在河堤上孩子们的嬉闹之外默默充当着观众。 

王园子的孩子最难捱的是冬天。北风呼啸起来
\newpage
的时候,大人们不用下地,纷纷窝在家里捂火盆烤花生,只有孩子们一早顶着寒风出门。他们被冻得呼哧呼哧的,但找乐子的脾气不改。放着宽崭崭的河堤不走,成群结队地蜿蜒在堤下结了冰的水边。岸上芦苇早枯了,一片干燥的黄,高高的河堤成了屏障,把西
北风挡得远远的。 

孩子们走在这条自己踩出来的新道上,别提有多欢欣。男孩子走着走着会往冰层上扔砖头,有时候当着女孩子的面龇牙咧嘴地嚼冰块儿,冰层厚的时候
他们干脆跑上去装模做样地溜冰。 

也有掉进冰窟窿湿了身子出洋相的,于是王园子的老师和家长一律反对他们从河堤下面走。王园子的孩子裹着头巾帽子戴了手套重新回到宽崭崭的河堤上,北风刀子样地穿刺着,他们缩着身子一路逃窜。从此只能眼巴巴地羡慕校门口的惠惠、秋桃和哑巴,
被下了禁令后的冬天,是他们的冬天。 

秋桃是个懒洋洋的男孩,看不到他调皮捣蛋也看不到他乖巧。老瓦匠家的饭桌上并不萧条,但秋桃
\newpage
就是吃不胖。他走路总是蔫蔫地,脚跟拖着地,发出很响的声音。磨烂了鞋跟不算,每双布鞋的前头还都
长了眼睛。 

冬天的早上,王园子的孩子裹了头巾帽子戴了手套来到学校,可他们的鼻子还是冻红了,脸还是冻紫了,他们跺着脚呵着手哇啦哇啦地读书。上课铃快
响的时候,秋桃这才夹着书包拖着鞋子走进教室。 

秋桃既不戴帽子也不裹围巾,他两手插在裤兜里,脖子露在外面很长,头发在头顶舒松地晃悠,一看就知道他刚从被窝里出来,哑巴和孩子们都能闻见
他身上热烘烘的被窝里的气息。 

惠惠不迟到。惠惠的书念得很好,她的爸爸妈妈给她订了好多书报。她是王园子小学唯一拥有课外书的孩子。她不仅书比别人多,她的花裙子和花鞋子也名目繁多。惠惠上学放学总是被爸爸妈妈一左一右牵着,那架势在哑巴和孩子们眼里总有另外一层意思,好象在暗示,“瞧瞧,这是老师家的孩子。”可是阳春白雪的惠惠却看不出一点点快乐,她的小眉心总
\newpage
是蹙在那里,好象存心要作难王园子那些没心没肺孩
子。 

惠惠最惹人眼馋的是冬天早上第二节课后。那时候,王园子的孩子熬过了大清早的冻劲儿,肚子里开始唧咕唧咕地唱空城计。于是他们疲软地趴在课桌上,再没力气疯。每到这个时候,教室的窗口就会趴上一个缺门牙的老太喊惠惠。老太一脸和善的笑容,嗓音里存了蜜糖似的。她就是惠惠的奶奶,她只露一个脸儿,只喊那么一声,递进一个热乎乎的纸包包,就含羞似的走了。惠惠蹙着眉,当众挑开纸包,里面不是冒着热气的烤红薯,就是一只肉包子,或者是喷香的饼干和面包。惠惠不太情愿地吃着,王园子的孩子却趴在桌子上频频地咽口水,有的受不了,只好出
去走走。 

王园子小学的孩子羡慕惠惠自是不用说,他们也羡慕秋桃,秋桃可以比他们多赖一会儿被窝。他们其实最羡慕哑巴,因为哑巴根本不用上学,一年四季都在度假,想怎么赖被窝都行。孩子们想不太明白,哑巴男孩不用上学,不用赶早,他为什么天天要起得
\newpage
那么早。秋桃在课堂上就曾跟天竹咬过耳朵,要是他也哑巴了就好了,不仅不要天天来学校坐牢,而且也不必应付没完没了的考试。天竹觉得是,好象也不是
。 

天竹不想成为哑巴,她当然知道哑巴意味着什么,她无法想象那个没有声音没有语言的世界,是沉静的漆黑,还是鲜亮的明黄,还是虚无的苍白——隐隐地,天竹害怕和回避着那个莫测的世界,可又抑制
不住那份好奇。 

也许是天竹带的头,王园子的孩子一时都对哑巴这个缺陷的生理成因发生兴趣。她们回去问大人,哑巴为什么不能说话,哑巴嘴里的结构到底跟别人有什么不同。大人们不耐烦,说哑巴就是吃糖吃多的。村里的老婆婆顶真地说,哑巴才不会是吃糖吃的呢,哑巴不会说话是因为他的舌头比别人短,而且品不出来味儿。天竹她们不相信,就问秋桃,秋桃说:对,
哑巴家一年四季不吃调味品。 

这个消息让孩子们更好奇了,他们决定找机会
\newpage
试试哑巴。为了试验成功,他们开始有意无意地凑近哑巴。孩子们来找哑巴玩,哑巴显得很乐意,哑巴喜欢他们,尤其喜欢那个爱说爱闹的天竹。孩子们看出来了,待时机成熟就让天竹逗他吃辣椒。哑巴当着天竹和大家的面,拿起顶辣顶辣的钉子椒,摆出满不在乎的样子,往嘴里扔了大把,大模大样地嚼着,好象一点不感觉到辣。他果然没有味觉吗?!天竹她们在
旁边看得嘴巴直渗水,眉眼皱得挤到一起。 

秋桃感到纳闷,他怕哑巴吃的辣椒不辣,于是也试着咬了一口,但他很快把辣椒吐出来,捂着嘴巴在草地上打滚。这下轮到哑巴嘲笑他了。孩子们眼见为实,相信了老婆婆的话。其实哑巴那天回去后胃就开始疼了,舌头被辣得很不利落。他不知道孩子们为什么要玩吃辣椒的游戏,但他依然高兴,因为他们高
兴。 

天竹知道哑巴的缺陷后,对哑巴无端地生出来许多同情。她经常去找他一起玩,有一回她给他一块大白兔奶糖。天竹以为他会像吃辣椒一样毫无感觉,谁知他吃了一口就很满足地乐了,然后吐出来一半,
\newpage

包起来比划说要带给他妈妈。 


天竹奇怪地问他“好吃?” 

哑巴点头笑。天竹举着辣椒问,“跟这个一样?”哑巴红着脸摇摇头。原来哑巴是有味觉的呀!天竹感到上了老婆婆的当,而且谋害过哑巴。她惭愧得
不得了,她把口袋里的大白兔都掏给了哑巴。 

如果不上学,天竹他们也许会跟哑巴多些接触。可是自从进了学校,他们就跟哑巴拉开了距离。哑巴看着天竹他们上学心里有些空落,哑巴上不了学不怪别人,他那样的缺陷着实不适合进王园子小学。好在哑巴和哑巴妈妈没这个奢望。天竹、秋桃和惠惠他
们上学的时候,哑巴就提着草篮子跟他妈妈下地。 

跟秋桃不同,哑巴虽没好的吃喝,但长得很皮实,体格也高大。他爱跑爱跳爱爬树,像只调皮的黑马猴。哑巴妈妈下地干活的时候,哑巴就在田埂上跳窜,饿了扳根玉米,渴了拔根萝卜。再大一点的时候,哑巴妈妈就给他抓了两只羊,哑巴就经常带着羊在
\newpage

学校附近的草坡上放。 

不知道是哑巴喜欢学校才在校门前的河款上放羊,还是因为长期在那里放羊,对学校生出了喜欢。哑巴放羊的时候,眼神总是往校园里漂。他不厌其烦地望着一茬茬进进出出的孩子。他们一律打他眼皮底下经过,哑巴比谁都清楚这些孩子的喜乐烦忧,常常身不由己地跟着他们激动。但哑巴再激动也不凑上前,他始终保持着那段距离,尽管他跟天竹她们并不生
疏。 

哑巴不明白秋桃为什么要天天起那么迟,他常常担心他要睡过头,甚至他等到秋桃拖着鞋子懒洋洋地出现时悬着心才松下来。有阵子,哑巴对惠惠奶奶提过去的纸包包特别感兴趣,他管不住腿脚地相跟了去,好象那是一幕有意思的戏。哑巴爱趴在窗台上看惠惠不情愿地吃点心的模样,只是他后来发现,自己看得越过瘾,惠惠的眉心蹙得越发地紧。渐渐地,哑巴似乎有些明白,于是就略略知趣了些。王园子小学的孩子在羡慕惠惠、秋桃和哑巴的时候,压根儿没想到哑巴在暗暗地羡慕他们,他们一丁点儿也没想到。
\newpage


2.自从村上放了《三家巷》的电影后,小学门前的这三户人家就被孩子们叫做三家巷。这个名字
是天竹先叫起来的。 

三家巷里,惠惠家住在左边,惠惠的爸爸妈妈都是王园子小学的老师,房子看上去整洁亮堂。秋桃家住在右边,秋桃爸爸是村里的老瓦匠,房子也不差,但门前屋后垒了鸡窝和兔笼,那里成年飘散出臭烘烘的难闻味道。夹在他们两家之间的是哑巴家。哑巴家两间草房,矮塌塌的被左右两幢砖房架着,像个不
能自立的瘸子。 

哑巴家两口人,就是哑巴和他的哑巴妈妈。王园子小学的孩子不知道哑巴为什么没有爸爸。庄上的老婆婆说,哑巴爸爸去了遥远的城市,去了就没有回来。不知道是死了,还是对他们母子两个哑巴绝情了

老婆婆们闲聊时候说过,哑巴的爸爸长得不错,能说会道,知书识礼,早年差点当了小学里的老师。只是他的出身有问题,连女人都找不到,最后碰上
\newpage
了嫁不出去的哑巴女人。哑巴女人模样还算周正,除了少张嘴。哑巴爸爸一眼就相中了她,刚开始日子还不错,可等哑巴生下来后,他就有了外心。算起来哑巴爸爸进城快十五年了,想是不会回来了。王园子村寨不大,人也不算多,该是个云淡风轻的所在。但因了些旧事传奇,又常年烟笼雾绕,倒无来由地添了层
亦真亦幻的神秘。 

如果不是哑巴妈妈人缘好,像哑巴这样的孩子早就是其他孩子欺负玩弄的对象了。哑巴妈妈高高大大的,水色好,五十岁的人一点不显老,尤其是笑起来,大概想着自己的缺陷,分外谦和娇羞,依稀可见昔时风韵。哑巴妈妈能发出咿咿呀呀的声音,也能听懂别人的意思。可惜她那个儿子不如她,说不出也听不到。没人知道哑巴女人的名字,也没人来为她的哑巴添名定姓。日子长了,人们为了方便,就喊她儿子为哑巴,她就自然地成了哑巴妈妈。哑巴直到上学才有了自己的名字,不过,这已经是后话了。其实不管哑巴有着什么样的身世,哑巴叫什么名字,在当时他是不可能成为王园子孩子心目中的主角的。他基本上被忽略在孩子们的视线之外。他和他们过着完全不同
\newpage

的生活。 

孩子们读完了一年级,基本适应了学校的生活,彼此间也有了交情,有的还结成了死党,校里校外地粘着,或者干点秘密勾当,或者交流点私底话儿。反正外人是插不进的,甚至连水都泼不进去。哑巴在他们面前完全跟不上趟儿,即使在熟悉的天竹面前,
也不例外。 

好在哑巴喜静,他喜欢一个人远远地呆着,很满足很安详。哑巴最爱看孩子们在操场上课。王园子小学有三种课在那里进行——体育课、活动课和劳动课。孩子们在操场上排队做操,或者三五成群地踢毽子抓么儿跳房子——哑巴看得有滋有味。哑巴悟性好,什么游戏他看一眼就会了。哑巴看着看着羊跑出去
很远都不知道。 

那个明朗的下午,逢上天竹、秋桃和惠惠他们出来上体育课。这节课与往常都不一样,因为操场中
央放了一个跳高架。 

\newpage

孩子们好象被跳高架上竹竿的高度吓住了。男孩子还好,大都跳过去了,女孩子不行。教他们体育的黄老师已经为她们把竹竿降了又降,可她们好象还是感到为难,尤其那个从来都不看哑巴一眼的惠惠。

哑巴觉得好玩,他看了一会就按捺不住了,索性壮起胆子从河款的草坡上挪到操场边上。黄老师没有在意他的靠近,他含着金属的哨子,严肃地给孩子
们做示范。孩子们也没心思注意他。 

女孩子们怕归怕,但强不过黄老师的指令,她们加足了马力跑,小脸憋得通红,有的跳将过去,一把摔在地上;有的跑上前却抱着竹竿冲出去,很狼狈的样子。围观的男孩子笑得前仰后合,黄老师也忍不住吐出哨子笑。哑巴在一边也憋不住,把手指含在嘴巴里傻笑。惠惠看那阵势,跟黄老师较上了真,她死
活没肯跳。 

哑巴就在那瞬间,心里突然升起了一个念头,这个念头把他膨胀得脚跟踮踮的。他瞅准他们准备解散的时候,趁虚而入,将竹竿移到最顶一格,然后顽
\newpage
皮地纵身一飞,容易得跟吃豆子一样。王园子小学的好多男生都看傻了眼,黄老师也吃了一惊。黄老师知道哑巴不会说话,就和善地对他笑了笑,还竖了竖大拇指。哑巴不好意思,他笑着缩起脖子兔子样急速窜回青色的草坡。哑巴初次尝试得了甜头。这以后,他只要看见黄老师带了孩子们出来上体育课,就慢慢地挪到操场边上去。他很谨慎,丝毫没打扰他们的意思,只是独自找块干净地方坐下。黄老师默认了他的介入,偶尔会招呼他过去帮着拿教具,或者拉皮尺丈量孩子们的跳远成绩。哑巴殷勤而爽气,黄老师看他一眼,他就得了赏似的跑上前来。就这样,哑巴卷进了
孩子们的学校生活。 

秋桃的跳远是男孩子里最差的,这好象是天生的,跟哑巴天生不会说话一样。没人觉得奇怪,但他自己却不服输。他蔫蔫地往前一跳,常跳不出女孩的及格成绩。他讨厌哑巴帮着黄老师拉尺子量他成绩的
样子,那场面特别让他不服气。 

哑巴偏偏不识趣,秋桃明明得了很差的成绩,他仍旧冲他竖大拇指。当着那么多人的面,哑巴这份
\newpage
热情令秋桃无比尴尬。有天放晚学的时候,秋桃看见草坡上的哑巴,不经意间想出一个好玩的游戏。他歪着脖子伸出舌头,翘起小拇指指着舌头,故意引逗草
坡上的哑巴。 

哑巴果然很不痛快地皱起了脸。男孩子们醒悟过来,他们和秋桃一起,朝哑巴做那样一个富有含义的动作。哑巴果然被激怒了,他像麋鹿样冲他们杀将过来,手里还捏着一团泥块。那泥块当然砸不到人,最后惨惨地碎裂在地上。孩子们觉得有趣,于是嬉闹
得更厉害。 

哑巴被他们戏弄了几回,坐在草坡上就渐渐地失去了悠闲。哑巴会趁孩子们上课的时候去寻几片碎瓦,或者结实的泥块,作为自卫反击的武器放在身边。等那几个调皮的男孩子出来,哑巴就虎着脸,拳头握着碎瓦或泥块,眼睛警惕地梭罗着。可毕竟寡不敌众,孩子们不怕他的武器,倒像有意要看他的愤怒和反抗,仿佛他是只马戏团的猴子。哑巴越是在意他们闹得越带劲笑得越疯狂。这群孩子在戏弄人方面真的别具天赋,他们筹筹胜出。秋桃一向闷皮,这种时候
\newpage

他早退到了游戏之外,好象他根本不知道似的。 

哑巴也不是好惹的,孩子们把他逼到毫无退路后,他就从家里带了根铁叉出来放羊。孩子们一出来,他就拿着铁叉虎视眈眈地站着,做出随时准备出击的样子。这一招真灵,孩子们的嚣张气焰被压下去了,至少再没人敢当着哑巴的面指舌头。从此哑巴也不去操场参加他们的体育课了。他拿着铁叉站在草坡上的样子像个孤独的斗士,跟操场上那根气宇轩昂的旗
杆差不多。 

王园子小学的孩子由此都知道,哑巴嫉恨人朝他吐舌头,或者说王园子最恐怖和最刺激的游戏就是对着哑巴吐舌头。天竹听说后跃跃欲试。经过三家巷的时候她有意无意地寻找哑巴。她很想对哑巴吐回舌头看看,女孩子里有这种想法的不止她一个。其实只
是想看哑巴的反应,倒不存在歧视的意味。 

那天她们叽叽喳喳地一起回家,看见哑巴一个人在草坡上放羊。因为好一阵子没人挑衅,哑巴放羊早就不带着铁叉。孩子们放学出来的时候,他省了明
\newpage
目张胆的仇视,换成躲躲闪闪的戒备。那种神色下他们之间出现了暂时的空白,空白里游来游去的是不安分的寂静。天竹她们彼此交换了一个微笑,于是苗着腰一起往前跑,等跑出了一定的距离,她们就一齐朝哑巴喧闹地吐舌头,然后再没命的往前跑。哑巴不出所料地跟踪追击,天竹觉得哑巴的黑手臂就快抓到自己了,女孩子们边跑边叫。可是不久哑巴追上来的脚步声停住了,大家随即也停下来。她们看见哑巴远远地停在那里。女孩子们松了口气,又发出解嘲般刺激的哄笑。哑巴在对面平静地望着她们,他慢慢地往后退,脸上挂着很奇怪的表情,总之一点也不恐怖。天竹她们的笑就倏地失去了滋味。从那以后,女孩子和哑巴之间不仅消除了距离,而且还生出了一些什么。

活动课的时候,天竹她们玩累了,会跨过小河,到草坡上逗哑巴的羊。哑巴的羊雪白干净,吃草的样子像个贵夫人,而且特别会发嗲,看见她们过去就哞哞地叫唤。女孩子看它们好玩,就纷纷采鲜草来喂。哑巴知道她们没有敌意,也不吝啬,他给她们尽情地逗羊,自己则拿起她们的毽子,很娴熟地踢着。女孩的舌头就是活络,哑巴踢毽子的功夫很快传开了。
\newpage
后来活动课时女孩子习惯了朝着草坡上的哑巴招手,哑巴也不忸怩,让来就来。惠惠和秋桃始终没有跟哑巴走近,可能是他们本来就靠得太近的缘故。这年暑假结束的时候,王园子小学发生了小小的变化,老校长光荣退休,新校长走马上任。这新校长不是别人,
是大家熟悉的惠惠爸爸。 

哑巴不知道从哪里知道惠惠爸爸当上了校长,他特地下河摸了一篮子田螺和河蚌,天黑前送到校长
手里。 

哑巴接连送了三天,隔壁的校长就坐不住了。哑巴妈妈也不想让哑巴继续泡在河里,因为受了寒,
哑巴已经拉了两天的稀。 

哑巴妈妈很抱歉地看着新校长,然后拉着哑巴指着小学,依里哇啦地一通比划。校长终于听明白了
,哑巴想上学。 

哑巴因为校长猜出了他的意思而害羞,他低着头卷着衣角,偶尔抬头瞄一眼校长。校长跟他们家隔
\newpage
壁隔,对他们的情况知根知底。哑巴要上学没错,学费交不上可以减免,可学校没法教他东西,总不能专门为他请个哑语老师吧,而方圆几十里又没设残疾人学校,就算有,他家也没那个条件。这个问题可把新
校长难住了。 

校长含笑朝他们无奈何地摇了一下头。没肯定
也不像否定。 

秋桃爸爸老瓦匠听说了这件事,他热心地过来劝哑巴妈妈,让她别听孩子瞎说,去了听不了课,认不了字,叫什么上学,不是花冤枉钱吗?他说哑巴要上学可能是想和孩子们玩,玩就让他玩呗,何必花钱
去玩。 

哑巴看懂了老瓦匠脸上的意思,很不高兴。老
瓦匠前脚走出去,哑巴就在后面掩上门。 

第二天哑巴照常下河去摸田螺,惠惠奶奶没办法,只好把田螺一家一家地分给人家。所以当哑巴晚上再拎着湿漉漉的田螺来到惠惠家时,惠惠不由得生
\newpage

气了。 

惠惠妈妈看这样下去不是个办法,于是给校长出了个主意,让哑巴上学,象征性地收他的书费,其他费用免了。这样他即使学不了什么,也能得几本书
看看。 

这年开学的时候,王园子小学因为走进了背着书包的哑巴,好象酝酿出了一个特别可乐的事情,大
家上学放学都特别喜庆。 

哑巴妈妈给他做了套新衣服,白衬衫黑裤子,哑巴头发也理过了,理得短短的。哑巴看见天竹她们好象不太好意思,黑眼睛里藏着秘密样地垂着,只是到了活动课,他踢起毽子,跳起房子,又成了过去的
哑巴。 

班主任对这个特殊的新同学比较有耐心,他给哑巴娶了个名字叫王自立。但那只是书面意义上的,平时王园子的大人孩子们照样喊他哑巴,没人使用这

\newpage
个名字。 

哑巴对书本非常爱惜,他像其他孩子一样,选了上好的旧年画,裁好了包书皮。上课的时候,他从不做小动作,也不干扰其他人,像能听懂一样,专心
致志地看着老师。看得出他很想学出点模样来。 

老师们也试着喊他回答过问题,可他说不出来,只能手舞足蹈地比划,老师又看不懂,同学们就哄笑,老师和哑巴都很尴尬。这样磨合了半学期,哑巴和大家形成了默契。他不参加课堂讨论,老师提问不找他。文化课的考试试卷发给他,他交不交无所谓,大都他也做不出来。老师对他没有要求,有空闲的时候才想起来跟他交流交流。一年下来,哑巴学会了写王自立三个字,学会了表示数字。哑巴妈妈对他的表
现很满意。 

哑巴最爱上两种课,就是体育课和劳动课。哑巴好表现,特别喜欢老师给他竖大拇指。不知道是不是因为他在文化课上无法得到关注的缘故。劳动课上,他常常喧宾夺主代替老师充当指挥,王园子小学的孩子最怕的劳动就是扫厕所。但怕也怕不掉,五个班
\newpage
级一个个轮流。碰到那样的日子,老师也是拈手拈脚的。可是轮到哑巴那个班就不一样了,哑巴像个孤胆英雄,从家里拿来工具。然后一个人冲进厕所,其他人只在外面做他的帮手。哑巴干活大刀阔斧,校长就亲眼看到他用手掏过厕所。等他浑身带着异味从里面出来后,男厕所干净得像新的一样,大家都有些不习惯。全校的师生都不得不佩服哑巴崇高无私的劳动积极性。校长在全校师生大会上表扬他时,哑巴听不懂,天竹她们就拿胳膊肘捣他,叫他看校长特地给他竖的大拇指。哑巴知道自己被表扬了,就很不好意思,抓着头咧着嘴巴笑。笑得脸蛋红红的,女孩子们都觉
得哑巴好玩。 

哑巴读到三年级的时候,个子格外地高,不仅比他们班的孩子长出一头,比毕业班的秋桃惠惠他们也要高一些。他长胳膊长腿地挤在课桌之间很难受,但他不怕,哑巴怕另外一种难受。随着功课难度的加深,哑巴根本不知道老师们讲解的什么,哑巴专心致志坐在课堂上的样子很傻,渐渐地他自己都觉得尴尬起来。他怕上文化课。他怕那种茫然无知置身事外的

\newpage
空洞感。 

老师早已习惯了把他放在一边,同学们学习和做功课的时候从来想不到他,他一个人寂寞地翻着课本,坐着坐着思想就跑出很远。他的眼前出现了一个无声的美丽世界,那里所有的人都和他一样,有美好的心灵,有机灵的大脑,他们一律不使用语言,他们一起安逸地游移在语言之外,彼此理解,互相关爱。他们一起笑,一起哭。在里面没有被抛荒的焦虑,没有被隔离后的寂寞。哑巴认定这个美丽的世界一定与河水相关——那么无声无息,却一圈又一圈漾着美丽
而神秘的涟漪。 

哑巴这样想着的时候,看见身边的同学在激烈地使用嘴巴,老师的嘴巴也在翕合,突然他们一起笑得手舞足蹈,老师也笑得趴在讲台上。他们完全在另外一个陌生的世界。往常哑巴在这个时候会受他们感染,跟他们一起笑,尽管比他们慢一拍,尽管不知道他们笑什么。哑巴不喜欢自己落伍,不喜欢被抛荒,他渴望跟大家保持一致。哑巴这样想着就笑不出来了,他凄凄恻恻地念及刚才的幻想,不知道那样美丽的

\newpage
世界存不存在。 

哑巴被这个问题折磨了一次后,接着就开始没完没了地陷在里面。当老师用信任和热切的眼睛望着那个学习好的班长,当同学们从校长手里接过三好学生的奖状,当同学们热火朝天地在争论一个不知名的话题时。哑巴缩着过长的手脚呆在座位上就特别自卑
,他觉得自己的存在已经是个多余。 

好在体育课和劳动课偶尔会弥补一下他的失落
,让他的心忽悠地闪亮出一串快乐的火花。 


3.  

那是一个平常的冬日午后,校长召集大家在操场上开大会。他宣布了一个激动人心的消息,说全校将选送五名运动健将去参加乡里三年一度的小学生冬季运动会,选送队员的事情将在一周内结束,具体地
由教体育的黄老师负责。 

黄老师是个急脾气,学校的体育是他一个人教的,王园子小学的运动健将在他心里都有眉目。会议
\newpage
结束黄老师就拟出了名单,并把名额通知到各班的班主任。因为忙乎这事,黄老师这晚上就比别人回去得
迟了些。 

黄老师在校门口看到了哑巴,他好象在等人。黄老师本能地喊“王自立”,然后对他补充了一个微
笑。 

哑巴等的就是黄老师。放学的时候,天竹她们几个女孩子鼓励他主动去找黄老师报名,还比划说他
参加跳高,肯定能拿个冠军回来。 

哑巴从她们的言谈中看出来,冠军是很了不起的事情,再说跳高对他来说太容易了,他一直跳最高一格的,每回都轻轻松松,好象比黄老师还轻松。他
总嫌那高度不够,要是再往上挪一挪就更好玩了。 

哑巴被他们说得激动起来,妈妈已经做好了晚饭,可他无心吃,他看见黄老师的那辆蓝坐垫的自行
车还在,于是发狠劲等在学校门口。 

\newpage

他望着黄老师,摸摸头皮。黄老师看出他有话
要说,于是耐心地看着他。 

哑巴一着急用树枝在沙子上画了个跳高架,然后指了指远处的方向。黄老师笑了起来,然后把手放在哑巴的肩上,点点头。哑巴高兴得蹦起来老高。他看着黄老师离去的背影,真恨不得自己能跟着他一起
飞出去。 

哑巴回家的时候路过校长家,校长拿着故事书在给惠惠讲故事,他们互相交换了一个微笑。惠惠像没看到哑巴似的,蹙着眉心,陷在故事里。秋桃端着碗面条在漫不经心地吃。哑巴知道秋桃最怕吃这种费劲的食品。秋桃不知道哑巴高兴什么,他费解地领受了哑巴一个灿烂的微笑。第二天哑巴把这个好消息告诉了天竹,校园里的孩子很快都知道哑巴要去参加乡运动会,他们都朝哑巴微笑,有的给他竖大拇指,有的在他肩上拍一掌。王园子的孩子都觉得哑巴去对了
,他准能稳笃笃地拿个冠军回来。 

这一天无疑是哑巴这么多年来最快乐的一天,
\newpage
他张着嘴巴回报大家的善意,同时把快乐分散给大家。他似乎觉得自己已经融入了周围这个世界,他们不再陌生不再有藩篱阻隔。带着这崭新的心情,哑巴在
课堂上听得特别虔诚,他还在纸上写写划划。 

活动课的铃声一响,黄老师就来找哑巴。操场上已经有人在练习短跑跳远和跳高。哑巴跟在黄老师后面勤奋地练习。操场上拥挤了好多孩子,他们不时
地为哑巴喝彩,因为他的跳高简直是特技表演。 

哑巴在围观的人群里找到了天竹她们,她们比哑巴自己还开心。哑巴已经想好了,明天带两根长钉
子来在架子上打眼,这样他想跳多高都可以。 

第二天,寒流已经入侵了王园子,哑巴一早起床就在操场上跑步,天竹她们裹着大衣围巾来上学时,看见穿着单衣的哑巴在操场上跑得一头一脸的汗。

校长走进学校时,照样和他夫人牵着惠惠。他
们一起看见了汗流浃背的哑巴在操场上用功。 

\newpage


校长说,“这孩子发什么疯呢?” 

校长夫人说,“他不是学校准备选送的运动员
吗?这孩子可真尽心,就是少个嘴和耳朵。” 

校长没说话,他坐进办公室喝了一杯浓茶,然
后将黄老师叫进校长室,并且掩上了校长室的门。 

“你不能把他选去,他是个残疾孩子,你应该首选健康正常的孩子,这是小学生运动会,不是残疾
人运动会。” 

“可通知上没说残疾孩子不能参加呀?再说王
自立的跳高是可以绝对保证为学校拿奖杯的。” 

“奖杯倒是其次,关键是我们学校的名声,堂堂一个学校都找不出一个健康的运动员,别的学校会
怎么看我们的工作?” 

“可王自立在这方面有特殊才能,为什么不能给他一个表现的机会呢,这对他来说很重要。我们应
\newpage

该鼓励他,他毕竟也是我们的学生。” 

“我已经够鼓励他的了,换了一个校长,他根
本不可能有这个上学的机会。” 

“那是因为我们这个地方条件落后,假如条件
好,他是有机会接受教育的。” 

“那是假如,你得正视现实,我们怎么也不能让他去丢脸。再说是运动会,又不是什么文化课比赛,差一点教育局也不会计较的,关键他们是看你的考
试成绩。” 


“我已经跟他说了,他——” 

“去跟他解释,就说乡里不同意残疾孩子参加
。” 

黄老师一整天都绷着脸,晚上他照旧带哑巴去
操场,好象校长没跟他谈过话似的。 

\newpage

校长比较生气,吃过晚饭天都黑了他还看见哑巴在那里练。校长好不容易等到哑巴气喘吁吁地回来

校长当着哑巴和哑巴妈妈的面,把情况解释得一清二楚,哑巴没打断他,他瞪着黑漆漆的眼睛,像两把锋利的锥子。校长知道跟他说不清楚,他跟他妈
妈点点头就出去了。 

哑巴哇地发出一串动物样的呜咽,他把那两根铁钉子摔在饭桌上,震出很剧烈的响声。哑巴妈妈很生气,她随手打了哑巴一巴掌,哑巴抱着书包出门埋到草屋后的黑暗处抽泣起来。冬天的夜晚总是来得特别的仓促。这会儿秋桃早吃过饭,在他爸爸的监督下有气无力地做着功课,惠惠做好了功课,捧着一本《小学生优秀作文选》在看。外面静悄悄的,连鸡鸭也在笼子里开始打瞌睡了。哑巴蹲在地上,死死地抱着
那包对他毫无用处的书,无望地哭着。 

第二天清早,气温更低了,大人们给孩子准备了简单的早餐后又缩回了被窝。哑巴妈妈也是,她不知道哑巴昨晚哭到什么时候,总之哭得她头都要炸了
\newpage
。她伤心地想,这有什么好哭的,要是这么容易哭,那自己这辈子还不早就哭死了。哑巴妈妈决定给他一个狠狠的教训,任他怎么闹也不去哄他。关键是人家
校长对咱们已经够好的了。 

哑巴妈妈煮好稀饭,看见哑巴一反常态地埋在被窝里不起床,她叹了口气,也不去喊他,自己回到
床上。 

哑巴已经醒了,早上他特别容易醒,他不喜欢赖被窝,寒暑都一样。他知道现在该是孩子们上学的时间,天竹她们一定裹着大衣,沿着光秃秃的河堤往学校走。他真不想再见到她们,不想再见到任何一个同学,那是多么丢面子的事情,他像一只被拔了毛的鸡。校长明晃晃地在他和其他同学之间划开来一条河,这条河会越来越宽,他将跟他们越来越远。哑巴想到这儿,又呜呜地哭起来。这是哑巴上学以来第一次
逃学。 

这个不平等永远不能消除,任凭他怎么伤心怎么哭泣,也无能为力,天竹不能帮助他,黄老师也不
\newpage

能帮助他。 

哑巴妈妈闷头忍受了他一会儿,可哑巴的哭劲
一时好象下不去,他连学也不想去上了吗? 

哑巴妈妈下了床,饭也没吃就出门下地了。她看不惯儿子懦弱,既然读书读成这副样子,读不读也
无所谓。 

哑巴一天都没来学校,晚上天刮大风,还夹了雨滴,锻炼暂时取消。黄老师来到哑巴家的草屋。他
看见哑巴的书包被摔在床头的地上。 

哑巴感觉来人不是妈妈,他坐起来,看见了黄老师,瞪着的黑眼睛就委靡起来,接着泪水一串串地往下流。黄老师也不劝他,他掏出一根烟,靠在哑巴
的床头抽起来。 

奇怪那呛人的香烟像有止泪功能,哑巴哭着哭着就停了下来。他看见黄老师在烟雾中的脸,从没这么灰暗。他觉得非常内疚,他不应该让老师不高兴。
\newpage

哑巴不哭后,就抱歉地望着老师。 

黄老师抽完了烟,很无奈地拍了拍哑巴的肩,然后帮他把书包拣起来,拍掉包上的泥灰,放在哑巴
的手里,走了。 

第二天哑巴照常去学校了,学校里跟平常一样喧闹,大家看见他没觉得有什么异样。好象大家早猜到这么个结局,好象他们没有对哑巴流露过信任和希望。哑巴的失意没影响周围的气氛,哑巴本来就没有
影响过这里的气氛。 

天竹下午才遇到哑巴,她一脸沉重地望着哑巴,那沉重熨贴着哑巴的心情,好象这时候他才真正感受到失去尊严的滋味,那么真切,那么痛心。可惜天竹的沉重是无济于事的。哑巴坚持天天去上学,他觉得不能后退,只要退下来,就永远也起不来了。所以他坚守着教室里那个位置,像个忠于职守的哨兵。好在运动会终于过去了,哑巴仍然在学校,他还属于他们中间的一个。好象他自己已经反败为胜了,毕竟他的黄老师在体育课上一如从前地流露对他的器重,天
\newpage
竹这些女孩子在活动课上还来找他踢毽子。可就在他苦苦坚持的过程中,上学的快乐真的从他身边逝去了,他不知道自己怎么丢掉那些快乐的。反正他一天天厌倦起学校来,一切都是不能自已,就像在破裂的冰层上无法站立一样。甚至他开始回避天竹的关心,他
觉得其中含夹了太多的同情。 

期终考试转眼到了,这次是县里统考,老师特别重视,体育课劳动课活动课暂停,全部用来复习。同学们很紧张,下课了他们脑袋也凑在一起,互相看看笔记讨论讨论习题。哑巴这个时候是可有可无的,他不需要紧张。有一回他凑到人堆里,被人家一把推
出来,“去,去,去,有你什么事。” 

班主任找过他,意思是他可以提前回家准备过年。这里委实没他什么事了。这样规模的竞赛,哑巴肯定没有资格参加。他自己也知道不能参加,他的确
考不出什么来。 

哑巴回家后烦躁得很,每到上学放学的时候,他就关上门。妈妈对他的表现不满意,依里哇啦地对
\newpage

他嚷嚷,怪他读书没长进,不如小时候懂事。 

妈妈批评时,哑巴没反应,他殷勤地帮妈妈打下手。妈妈为了给他凑学费,做了些豆腐卖。可王园子人到了年关都要做豆腐,哑巴背着豆腐篮子绕过学校去卖时,大家期期艾艾的,象征性地买几块,但那不痛不快的表情刺伤了他。他出去了几次就不干了。

哑巴就在那几天决定不读书了。他还没来得及跟妈妈讲,新年已经到跟前了,学校已经放了假,妈
妈却病倒了。 

哑巴妈妈的病很怪,她不停地咳嗽发烧,王园子的赤脚医生给她用了最好的药,还不见效。偏偏又到了过年,妈妈不肯去住院。赤脚医生看他们也住不起,就说慢慢来吧。哑巴妈妈的病慢慢好了后,家里的钱早用得精光,而且欠了校长和老瓦匠一笔债。这
时候新年已经过去,学校又要开学了。 

哑巴不肯去上学,哑巴妈妈看他那个坚决的样子也没办法。于是她去找老瓦匠商议。老瓦匠一向就
\newpage
反对哑巴上学,他的意思是他们在把钱往水里扔。他对哑巴妈妈又是点头又是摇头,哑巴妈妈约莫知道了他的意思。哑巴妈妈等了几天,她看哑巴没有反悔的意思才去找校长。校长没发表意见,他没法保证哑巴在学校里能学到知识。就像当初他们迫切地要求上学一样,他无奈地笑了笑,然后摊摊手。哑巴从此不上学了,而且坚持不到学校去,好象跟学校结仇了似的。他看见学校的孩子一律冷冷地虎着脸。天竹有次在路上拦住他,他没停下,头也没抬从天竹面前绕过去了。没人知道哑巴怎么啦。王园子的孩子也懒得去分析他,黄老师看他那样,干脆放任不管。好象他有意
要让他吃苦头走弯路一样。 

哑巴妈妈给他多买了几只羊。哑巴不再在学校门口的草坡上放羊了,他拿着长长的竹竿,把羊赶到很远很远的大河边。来来回回,哑巴总设法回避着王
园子小学的孩子。 

有回天气突变,哑巴早早收工回来,正巧碰上学校放学。孩子们看见哑巴有些兴奋,大家帮他赶乱跑的羊。可哑巴很古怪,他谁也不看,狠狠地挥着竹
\newpage
竿赶着羊群,好象谁再帮他他就要抽谁似的,天竹看
到他那蛮横样子感觉非常怪异。 

有孩子掩着嘴说,哑巴大都是神经病,发起脾气来特别吓人,大家可不能乱跟他开玩笑。从此大家看到他都规矩了起来,有让着他的意思。他呢,也就
越发地气势汹汹。 

可是天竹又听人说,哑巴在没人的大河边放羊的时候,老抱着书看,不知道他看的什么书,好象还挺用心的呢。下田干活的大人都看到过。秋桃妈妈就不止一次地拎着秋桃的耳朵说,“你要有哑巴那个用功劲儿就出息了。”不仅秋桃妈妈,越来越多的孩子
被逼着像哑巴那样用功。 

王园子的孩子看到哑巴就更觉得神奇,他脸上不动声色的冷漠就镀上了高贵的光彩。哑巴不知道这些,他依然我行我素,冷着他的黑眼睛,包括校长在
内,他谁都不理不睬。 

哑巴勤快能吃苦,家里家外帮了妈妈不少忙。
\newpage
哑巴妈妈看他顺眼多了。她跟老瓦匠说好了,等哑巴
再出落两年,就跟着他去拎灰桶做徒弟。 

哑巴的羊群因为服侍得好,又是放养在外,长得比别人家快。半年过去,羊儿们一个个膘肥体壮,大家都觉得这群羊能为他们捞一个好收成。可校长一
家不喜欢这群羊。尤其是炎炎夏日到来之际。 

校长家没养任何家禽,门前就是门前,屋后就是屋后。惠惠奶奶每天都要拿上笤帚扫上三遍,所以老远就看见校长家前前后后一片宽敞干净,干净得一
尘不染,跟庄户人家完全两样。 

哑巴家的羊圈靠在校长家这一边。没办法,老瓦匠那边早已经垒了他家的兔笼了。校长一家对他那群羊的厌恶哑巴早看出来了。每次他赶羊回来,他们就赶紧把门关上,生怕羊会跑进她家里。不仅如此,惠惠看见他就捏起鼻子,好象他哑巴也是羊一样。这让他特别窝火。惠惠奶奶也是,他的羊从他家门口经过一次她就拿出笤帚扫一次,哪怕没什么,她也要扫。他们一家人对他这群羊的嫌恶,做得已经非常露骨
\newpage

了。 

哑巴爱他的羊超过一切。他容不得校长一家这么不尊重他的羊。他不喜欢校长,是校长没让他去参加那次运动会。隐隐地他记恨他。如果他去了一定能拿个跳高冠军回来,他就不会在学校呆不下去。上不上学倒在其次,关键校长把他跟王园子的其他孩子分别开来了。哑巴体恤羊,他知道羊群不喜欢呆在污躁黑暗的羊圈里。蓝天白云下面,那翠绿的草皮清粼粼的河水,多舒畅。哑巴经常在水边把它们洗得雪白干净,他觉得他的羊比惠惠还要漂亮。是日周末,哑巴
像往常一样去放羊。 

哑巴赶着羊出去时,惠惠在门口看书。哑巴看羊慢腾腾地走,便用竹竿抽在校长家的空地上。惠惠像受了惊吓,跑进家门。校长旋风般地出来了,他一定以为哑巴给他惠惠苦头吃了,因为他带着一脸怒气。偏偏哑巴的羊在他家干净得像台面的后院里拉了一串屎,像几粒丑陋的黑蜘蛛,让校长和哑巴之间迅速织就一张紧张的网,怎么也脱不了干系。校长应该发火,哑巴想。可哑巴也想反抗,尽管已经没有道理。
\newpage


校长像在学校开大会批评人似的,他凭空点着他的右手,左手有气势地叉在腰际。看不出他点的是
地上的黑蜘蛛还是散发着味道的羊群。 

惠惠奶奶出来了,迈着小脚找来一摊泥,把黑蜘蛛盖起来扫掉了。但她不罢休,恶狠狠地拿眼挖了
哑巴和哑巴的羊群。 

哑巴就在这时候,凭空撕裂般地挥下了一竹竿。竹竿像是铁鞭,在校长脚边炸开,把校长家门口的平整炸了个缺口。惠惠奶奶急得跳起来,校长拎起哑巴的耳朵,很凶地把他摔了出去。老瓦匠和秋桃听到了动静。老瓦匠像董存瑞扑炸药包似的老远扑过来,拉住准备还击校长的哑巴。哑巴挥舞着竹竿点着校长,校长开始向大家控诉哑巴的倒行逆施。哑巴被老瓦匠的铁手臂掐着,他跳腾着挣扎着,不服气得厉害。

“这孩子反了他,没老子教育就是没老子教育
的样子,你再疯我今天就给你点颜色看看。 

\newpage

老瓦匠咬牙切齿地叫骂。围观的人多起来。大家都觉得哑巴没有人教育不行了,大家身上都滋长出
了一份责任和义务。同仇敌忾似的。 

校长指指他脚边的地说,“我没说他什么,他就拿竹竿准备打我,要不是我让得快,我还得吃他这
一下子呢!” 

哑巴不服气,他指着风度翩翩的校长跳跃着,
嘴巴里发出豹子样地呜咽。 

周围的人说,“这孩子不是忘恩负义吗?校长对他那么好,破天荒地让他上学,他怎么恩将仇报呀
?这孩子,要打,这孩子该教育!” 

哑巴妈妈被人家拽了过来,她一看那阵势,什么也不说,拿起地上的竹竿,当众抽起了哑巴。老瓦匠看哑巴妈妈要教育儿子,就配合般地把哑巴往旁边的空地上一推,大家自觉地后退,给哑巴妈妈腾出空间来教育儿子。哑巴妈妈代表着大家的愤怒,把竹竿

\newpage
狠狠地落在哑巴身上,生怕不彻底似的。 

哑巴像只在火堆里跳腾的铜豆子,竹竿邦邦邦地落在他身上,落下去再弹回来。哑巴不哭不喊,没有眼泪。人们诧异地看着这颗铜豆子被竹竿上下炒翻踢打,他们奇怪他怎么不哭。慢慢的哑巴身上有了条条红杠杠,哑巴的脸也红得像涂了猪血。他们这才觉
得哑巴不是颗铜豆子,哑巴身上是皮肉。 

大家的心气顺畅了,有人上去拦竹竿了,更多的人上去抱住哑巴妈妈。哑巴妈妈可能力气耗尽了,毕竟她已年逾五十,毕竟她刚刚大病一场。哑巴妈妈好象是被他们推倒在地上的,她俯在泛着热气的泥土
上号啕起来,边哭边抓着揉着拽着那泥土。 

哑巴没上来,也没动,人们好心地说他,“孩子呀,你妈妈不容易,你要懂事啊!”也有老太太把
自己的眼泪都说下来的。 

哑巴捡起了妈妈打自己的竹竿,去找受了惊吓的那群羊。哑巴把羊赶到大河边,哑巴那天没有看书,他对着大河边的一棵老树,发疯地抽他那支竹竿,
\newpage

直到抽断了为止。 

那晚太阳落了西山哑巴才回家,人们看见他换
了根崭新的青竹竿。 

哑巴妈妈像受了哑巴的启发,找来很多青竹竿,在后院里拦出一个栅栏,把羊圈拦在里面,这样可
以保证羊不去侵犯校长家的院子。 

4.哑巴挨打后变得更内向了,除了他的羊,他谁都不看,谁都不理。尤其是对隔壁的校长一家,像结了很深的仇一样。哑巴看见他们就故意要高高地挥起竹竿,只要不在他家院子里挥,哑巴怎么弄出动
静,他也不好管。校长家的人都小心提防着哑巴。 

王园子的孩子跟惠惠一样,看见哑巴就要捏鼻子。天竹也发现,成天和羊混在一起的哑巴身上是有一股很重的羊膻味儿,天竹不捏鼻子,但她和哑巴之
间无可挽回地生分了。 

夏天王园子人是在午睡里过去的,过得糊里糊
\newpage
涂的。夏天过去后,王园子又发生了些变化。惠惠秋桃和天竹她们考上了镇里的初中,开了学,他们就带上被子行李,自己蹬着自行车,住扎到了镇上。偶尔回来,好象也染上外乡人的习气,像来客一样。不仅庄户人家感觉到他们长大了,连他们自己说起王园子小学,也是以一种过来人的口气去调侃。大家偶尔也会说到哑巴,但仅仅是提一提就转到新话题上去了。

哑巴也觉得自己长大了,他卖了那群羊,给妈妈扯了两身新衣服,剩下的钱又领了更多的羊,他想
他应该对妈妈好一点。 

夏天快过去时,新学期开始了,又有一批野孩子被关进了王园子小学。哑巴对学校已经没有了感觉,也许是因为惠惠秋桃天竹他们离开的缘故吧,他对
学校的芥蒂说没就没了。 

哑巴身体不舒服的时候,会把羊散在学校附近的草坡上,自己四仰八叉地躺在草地上。他还会大明大放地走进学校,去他曾经打扫过多遍的厕所。王园子小学的孩子背着哑巴偷偷地耳语,“他是哑巴,他
\newpage
是个可怕的哑巴。”也有胆大调皮的男孩向他发起挑战的,哑巴就呼地过来,拿着青竹竿狂追,非把那家伙吓得下次不敢才罢。渐渐地小学里的孩子都怕哑巴,没上学的孩子更怕,看见羊群老远的就让道。年轻的女人常常这样吓唬孩子,再不听话,不听话就把你
送给哑巴。小孩子保准收了哭声。 

在这么平静的日子里,三家巷发生了一件算得
上事情的事情。 

校长一块名贵的手表不翼而飞。校长这块表王园子独此一只,是校长在瑞士的哥哥越洋给他寄回来的,不仅价值不斐,而且牵连着校长兄弟之间的一份深情。校长很着急,校长一家都在紧张地寻找这块表

校长家闷头找了三天,把屋子的旮旮旯旯都翻
检了个底朝天,可就是找不出那块表。 

校长夫人更仔细,她发动大家在校园里找,他
们又花了三天的时间,手表愣是不出来。 

\newpage

学校里的老师和老瓦匠纷纷知道了这件事,他
们一起帮校长回忆丢表前后的细节。 

那只手表大家都见过,的确不同一般。校长珍爱倍至,基本上表不离身身不离表。校长只在晚上下水洗澡前拿开手表,洗好澡随即戴上。那几天秋老虎光顾,气温特别地高。校长一天要洗几回澡,他洗澡
肯定要脱下手表,洗完了戴没戴上就不知道了。 

校长在课堂上看时间发现手表不见了,他以为手表一定忘在家里。可等他放学回家却怎么也找不到了。他回忆来回忆去,觉得手表肯定是在家里丢的。老师们也记得,那天下午校长主持会议时,手表还在他的手腕上。校长开的是新学年工作计划会,会开得很晚,提起来大家都记忆犹新,尤其是记得校长带手
表挥舞的样子。 

老瓦匠说,丢在家里就不怕,贼应该不会太远
。老瓦匠让他去报案。 

校长在村里毕竟是有头脸的人物,这种事情最
\newpage
好不能轻举妄动。于是他给各班布置了寻表通知,想看看表是不是被学生拣走了。校长特地声明,交上手
表不仅不罚,而且有重赏。 

这下学生都知道校长的手表丢了,学生回家告诉了家长,王园子的人都知道校长的外国手表丢了。那可是不得了的事,拿谁的东西也不能拿校长的东西,拿什么东西也不能拿那只表。那是王园子唯一的外国货,价值连城呢!校长的焦急传染到了各家各户。大人们都怕这丑事出在自己头上,于是个个关上门先把自家孩子盘问了个透,保证孩子没拿后,开始义愤
填膺地议论。 

那么昂贵的手表不可能丢在学校,校长是学校的什么人?谁不知道那外国手表是校长的?要是哪孩
子哪老师拣到敢不送给校长? 

通知发下来三天,果然没有下文。手表一定是
丢在校长家里了。 


\newpage

有人偷了校长的外国手表?! 

这可是名副其实的大案子。王园子多少年没失窃过了,大家不由得兴奋。这个贼既然有胆量偷校长家,下回不知道还要偷谁呢?庄户人家穷归穷,没什么值钱的东西,但照样怕小偷进门。王园子吵得沸沸
扬扬的。整个村庄都笼罩在一团紧张的疑虑里。 

村长不管校长愿意不愿意,给他去镇上的派出所报了案。派出所非常重视,当天就有两个戴大盖帽的警察跟着村长进了三家巷。专案组就设在旁边的小
学里。 

警察在校长家里做了一通侦察,然后就开始找人谈话。校长谈过,跟校长一起开会的老师谈过,校长夫人惠惠奶奶都谈了一遍。惠惠在镇上读书,中途没回来,自然跟她没关系。这些都是例行公事,王园子人想也想得到的。关键就看一下步他们找谁,找谁
谁的嫌疑就最大。 

在王园子人的紧张关注下,警察拦住了赶羊群

\newpage
的哑巴,他们把哑巴带进了小学一个单独的房子。 

那些警察有能耐,能看得懂哑语。但哑巴很愤怒,他不配合,强烈地要求出去看羊。警察对他的表现起疑了,几下交涉过后,哑巴还是坐不下来跟他们说话。警察没办法,只好动粗了。趴在窗外看热闹的人于是分头嚷嚷,手表肯定是哑巴拿的,警察已经对
他左右开弓了。 

王园子的人一下醒悟过来,手表应该是他拿的,肯定是他拿的。他对校长有意见,校长没让他参加运动会,上回挨他妈妈毒打也与校长有关,肯定是那
孩子报复校长呢! 

而且他有下手的机会呀,就那么三户人家,白天老瓦匠两口子都去工地做工,哑巴妈妈下地,校长两口子教书,秋桃跟惠惠一起在镇上。三家巷只有惠惠奶奶和哑巴在家。哑巴肯定是趁惠惠奶奶不注意下手的。王园子的人那几天激动得很,个个都觉得自己成了神探,警察找哑巴谈了一天没收获,继续找其他
人做工作。 

\newpage

他们接着找了哑巴妈妈。哑巴妈妈又哭又叫的,她一个劲地说他儿子没拿手表。她眼泪巴巴地望着警察,翻来覆去说这句,说着还跪下了。警察看审不
下去,只好让她回家。 

哑巴知道校长的手表丢了,暗地里高兴。他觉得真够解气的,虽然不明不白地被警察关了一天,还挨了两下子。他挽着清凌凌的河水给羊洗澡的时候笑得都抖起来。他不知道警察已经敲着桌子对校长说,
哑巴的嫌疑最大。 

校长见是这么个情况,他心里有了底。其实他掂量来掂量去就是想逃避这么个结果。自从上回哑巴因为自己被他妈妈打了后,那块阴影一直覆盖在他心头。他知道哑巴恨他,而他也无力挽回,他想时间迟早会消除这一切的,以后尽量不要和他们发生摩擦就
是了,省得别人以为自己欺负他们呢。 

他自己也觉得手表八九不离十是哑巴拿的。因为他最顺手,而且他也恨自己,警察不说他也是这么

\newpage
猜的。 

校长在床上辗转反侧到深夜,他决定这事就这么让它过去,可又觉得自己窝囊,堂堂一个大校长就这么被一个哑巴拿捏在手里,真是活见鬼。何况那手表是哥哥送自己四十岁生日的礼物。他特别喜欢那只
表,不仅仅是因为他的价值。 

校长觉得这事儿让了哑巴,以后怎么办,他想起哑巴看见自己挥竹竿的野蛮样子。校长觉得哑巴是自己美好生活里的一粒硌人的沙子。要拿掉它就舒坦
了,不拿掉,终归硌人疼。 

校长第二天见到专案组的人,也没说好也没说坏,沉着一张疲倦有余的脸。专案组以为他对他们的工作有意见呢,于是赶紧加大审讯力度。他们来到村
里做群众工作。 

群众的眼睛是雪亮的,他们把哑巴的性格怪异、穷困潦倒以及与校长家的恩怨都说了出来。有细心的人说哑巴去镇上给她妈妈买了新衣服,而且买了好

\newpage
多羊。他们怀疑哑巴那群羊到底值不值这个价。 

警察当晚来到哑巴家,强行把哑巴带走了。他们把哑巴带到镇派出所,走的时候王园子人看见哑巴
的手被铐起来了。哑巴妈妈跟在后面哭出去好远。 

王园子沉寂了,只有哑巴妈妈不管不顾的哭声牵扯着人们,大家觉得她的命真苦。没有人能帮得上她这个忙,校长一家很为难。校长夫人和校长一起过去劝慰她,他们表示要是哑巴被带走了,他们将会照
顾她的,大家仍然是好邻居,互相帮助是应该的。 

哑巴妈妈觉得对不起这家人,自己过年看病欠下的债还没还人家呢!哑巴妈妈无法安定,她把家里的所有家当都搬出草房子,一件件地翻,她要找出那块表。老瓦匠也帮她找,但他们找遍了都找不到。哑
巴把那块表放哪里去了呢?校长一家人也纳闷! 

第三天早上,哑巴被两个警察开着三轮摩托送回来了。哑巴不像王园子人想象的那样,被打得遍体鳞伤。哑巴好好的,而且手上没戴手铐。送他回来的

\newpage
是另外两个警察。 

他们要哑巴交出那块手表,不管是扔到哪里,得找到它才能结案。警察带着哑巴一起来到他家,把家里家外甚至羊圈都找过了,后又来到他放羊的大河边。警察找来抽水机,准备对大河开展一个大工程。


王园子人跟在后面看热闹。 

警察在王园子翻检了三天,包括那条大河,结果什么也没找到。他们只好把哑巴放回家。王园子被手表事情闹翻了天,关键是学校无法正常上课,孩子们全部沉浸到这个事情里面去了。校长觉得再不结束
调查他都快要崩溃了。 

在校长一家的强烈要求下,案子告一段落,专
案组撤消了。剩下的就是人们的谣言。 

哑巴妈妈没有追究儿子,她不相信他会做那样的事。他们恢复到平常的日子,两人一起下地干活,
羊群慵懒地跟在他们后面。 

\newpage

但王园子人看他们的表情复杂了。现在不用哑巴虎着脸跟他们拉开距离,而是他们见到哑巴就躲。不是前一阵小孩子捏着鼻子跑的躲法,而是夹着恐惧的一团慌张,或者带着戒备的嫌弃,好象哑巴真的是
偷表的贼。 

哑巴沉默了,他本来就只会用眼睛收视这个世
界,现在这个世界满目的荒凉和冰冷。 

哑巴帮妈妈忙完了地头的体力活,就继续一个人坐在河坎上放羊。阳光的移动逼迫着他从一个树阴挪到另一个树阴。可常常,他坐着坐着,忘了头顶的毒日头。他觉得这如火如荼的阳光真好,他被晒得真
痛快啊。 

哑巴是不是呆了,有人看见他坐在毒日头下半
天不挪动,凝望着大河出神。 

一个月后,人们从手表事情里走出来准备投入秋收的时侯,哑巴走进了王园子那条古老的大河,一

\newpage
去再没回来。 

哑巴妈妈那晚上睡得特别香,哑巴半夜起床给她挑满了一缸水,给她烧热了一壶茶,给她熬烂了一锅粥,她一点都没被吵醒。哑巴做完了这一切才去了
他放羊的大河边。 

曾经被抽水机抽干的河床重又漫上了水,甚至比先前还要丰盈纯净。河面无声无息,优雅地吐纳着水雾。漫漶的水雾苍茫茫的接天连地升腾开去。哑巴由衷地看见了那个美丽的无声世界,那里的人和他一样,那里没有冷漠,没有嘲笑没有怀疑没有歧视——他很憧憬地往其中走去,河水被激荡出一片喑哑,隐
约感动着哑巴。 

人们后来发现了哑巴压在枕头下面的那张纸条,纸条上歪歪扭扭地写了三个字:“不是我”。这是
哑巴用文字与这个世界进行的唯一一次交流。 

人们都为哑巴流了泪。哑巴妈妈把他葬在屋后的草坡上,那里朝着学校,是他小时候最爱逗留的地

\newpage
方。 

哑巴下葬的那天晚上,惠惠听见老瓦匠在恶狠狠地骂秋桃,说:你这孩子太没志气,还不如人家哑巴。天竹后来听到好多大人这么教训孩子。王园子从
此多了一条不成文的口训。 

王园子的大河一时却断了丁丁冬冬的吟唱,只是在风起雨飞的夜晚,依稀听见它们喑哑的声息,很
轻很淡的。 


据说持续了很久很久。 



\end{document}
