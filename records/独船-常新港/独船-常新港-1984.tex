\documentclass{article}
\usepackage[utf8]{inputenc}
\usepackage{ctex}

\title{独船\footnote{Click to View:\url{https://web.archive.org/web/20230505213311/http://www.zh61wx.com/Article/Class1/6410.html}}}
\author{常新港}
\date{1984-11}

% \setCJKmainfont[BoldFont = Noto Sans CJK SC]{Noto Serif CJK SC}
% \setCJKsansfont{Noto Sans CJK SC}
% \setCJKfamilyfont{zhsong}{Noto Serif CJK SC}
% \setCJKfamilyfont{zhhei}{Noto Sans CJK SC}
% \setlength\parindent{0pt}

\begin{document}
\CJKfamily{zhkai}

\maketitle


\Large

在北方,这种河流数不过来,地图上找不到
。小黑河,就是这样一条河。 


1. 三独 

几年前,这里连下了几天罕见的暴雨,河槽里的水一下子盛满了。中午时,河岸上站着一个妇女,手端着一大盆脏衣服。她在岸边来回走了几趟,怎么也找不到埋在河边上的平平的大青石。那青石上常站
着洗衣的和钓鱼的人。 

她终于按着熟悉的、被人们踩硬的土路走向水边,找到了那块青石。青石只露着一个边角,其余部分都被水淹没了。她脱下黑布鞋,赤着脚踩在青石上。她回身把儿子的衣服拿在手里,刚一蹲下,脚下的
\newpage
大地好像滑动了。她没来得及叫一声,就落入水里,
被急流卷走了。原来青石被水冲得松动了。 

岸上有人看见,急忙呼喊着,追赶着水里若隐若现的人踪向下游跑去。水,太凶猛了。没有人敢贸然脱衣下水。在下游,一个河湾处,这女人的尸体被打捞上来。苍白的手还抓着儿子那件不大的湿漉漉的
衣服。 

“我来晚了!我来晚了!”这女人的丈夫张木头赶到了,一手握着妻子遗落在岸上的一只鞋,一手捶打自己的胸口,重复地唠叨着:“我要是在,你就
不会死……” 

有人扶着张木头的肩:“张大哥,别难受了。大伙不是不救,如果有船,大嫂也许能救上来。单靠
人下水救,谁也别想活着从水里爬上来。” 

“我不信,我不信。我来晚了,我要是在,你
不会死的!”岸上,回荡着张木头哭哑了的声音。 

\newpage

不久,人们发现河面上出现了一条船,这是小黑河上的第一条船。挂在船帮上的桨,是用红漆仔细涂抹过的。有人看见,这条船的主人张木头和儿子张石牙经常坐在小船上,漂向下游,下好夜网。然后,父子俩背着纤,拖着船,逆水而上。第二天,再划船
去取鱼。 

村里实行生产责任制,开始分地时,张木头包了河边上的一块水田。他不顾村上人的劝说,决计把
家迁到远离村子的河边。 

张木头断绝和人们的一切交往,一心一意守着
自己的独屋、独船,还有独生儿子张石牙。 

“爸爸!这儿离镇上中学太远了。咱们搬回村里去吧!”有一天,张石牙跟父亲说。因为他要上中
学了。 

“远了好!”张木头眼睛看也不看儿子,干巴
巴回答他。 

\newpage


“我要走很多路!”儿子解释。 

“两条腿生着,就是走路的!”张木头顶着儿
子。 


“我没有伴!” 

“一天见不到一个人影更清静!”张木头没注意到儿子那束怨恨的眼光,“去!到河边守着船,别
让人随便用!听没听见?快去!” 


2. 结怨 

人们疏远了张木头,尽管他是一个比以前更加
勤劳能干的人。 

有一天,张木头赤着泥脚,从水田里走出来,把手搭在额头上,往河上一望,发现船桩上系船用的缆绳搭拉在水上,船没有了。他心里一惊,飞快地顺着河岸向下游跑去。在河流转弯的地方,看到了那只船。船上有几个穿裤头的半大孩子,正四仰八叉躺在
\newpage
船板上,一边哼着歌,一边舒服地晒着太阳,任船向
下游漂去。 

张木头脸发青,怒吼了一声,吓得几个孩子翻身从船板上站了起来。他们一看岸上奔过来的汉子,以及那身结实的黑疙瘩肉,心里暗暗叫苦,有人认识
张木头。 

“王猛,王猛!快靠岸,快靠岸!”几个孩子
慌张地向握桨的那个孩子叫起来。 

“怎么啦?”那个叫王猛的孩子回头望了望,看见岸上的张木头已经脱去了衣服,正准备下水,便
叫起来:“你们怕啥?他咬人咋的?别怕!” 

“这船动不得,谁动他的东西,他就跟谁拚命。天!这回让他撞见了!”几个孩子把衣服缠在脖子上,下饺子一样跳下水,向岸边游去。一上岸,头不
回,撒开脚丫跑了。 

王猛,这个愣头儿青,正是啥都不服气的年龄
\newpage
。他仍旧坐在船头上,看着张木头挥着两条黑鱼一样颜色的胳膊,劈开顶头浪,向船游来。当他看清张木头那气势汹汹的脸时,他心虚了,想把船划开去。但
,张木头是从船的前头游来的,已经把船拦住了。 

王猛糊里糊涂地被张木头从摇晃的船上掀下水,好半天才在水里辨认出岸边的方向。亏得这是水势平缓的地方,没有大浪头。王猛还是灌了几口浑水,费了九牛二虎之力,快要抽筋的脚尖才触到岸边的浅滩。他哆嗦着爬上岸,一屁股坐在地上,又吐又喘,擦了一把脸上的水,看见那条船停在不远的挂同处,张木头正得意地扯起一条大狗鱼,根本没把他王猛的
生死放在心上。这老家伙太少见了,简直没人味! 

王猛憋足劲,对船上的张木头喊:“你个老不死的,等我长大了,非把你的船用斧头劈碎了当柴烧
!老东西!” 

张木头被骂得在船上直跳脚。突然,他喊了一
句:“石牙子!你给我抓住这挥小子。” 

\newpage

王猛回头一看,岸上正奔过来一个跟自己年龄相仿的少年。吓得他气没喘匀,匆忙站起身,迈动着
疲劳的腿跑了,还回头恶狠狠地瞪了石牙子一眼。 

石牙站住了。刚才王猛仇恨的一瞥,使他心里很难受。刚才父亲把王猛掀下水的情景,被他看到了
。他同情父亲,又恨父亲做事太绝。 


3. 隔阂 

张石牙扛着行李,一走进陌生的学生宿舍,就感到一股冷意,把初上中学的新奇和兴奋的情绪冲淡了。有几个同学对他冷冷的,把上铺一个漏雨的角落让给了他。他听见下铺几个学生小声嘀咕:“他爸就
是张木头!”“对!他没有妈!” 


“河边上那间独屋是他家的!” 


“还有那红桨独船也是他家的!” 

“喂,”一个声音从门外传进来,拍了拍张石
\newpage

牙的床铺,“洗洗脸!”那人端着一盆水。 

张石牙心里涌出一股感激之情,急忙从上铺跳
下来。 

当四目对视时,张石牙愣住了,这个端水的人就是被爸爸从船上掀下水的王猛!王猛长着一头刷子
样直立的头发。 

王猛也认出了他,扭头把一盆水“哗”地泼到
门外。 

以后,张石牙感到了王猛在同学中的权威性。
他越来越感到自己孤独了。 


出早操,没人叫他。 


他的衣服从晾衣绳上落下来,没人拾。 

踢足球时,场上明明缺少队员,王猛也不让他

\newpage
上场。 

一天,张石牙一进宿舍门,迎面掉下雨点。低
头一看,白褂上染上一小串蓝墨水。 

“你怎么能这样?”张石牙看见王猛正在摆弄
手里的钢笔。 

“对不起,我的笔不出水,甩了两下,凑巧你
进来。” 


张石牙忍住了。 

下午踢足球,人太少了,王猛才让石牙上场。石牙憋足劲玩命踢,想让同学们知道他踢得很好。可
惜,一大脚,竟把球踢到操场边上的水泡里去了。 

“就这点本事!真无能!”“败兴!没劲!”有人双手叉腰,用眼斜瞪着石牙,吐着唾沫,不满地嗦叨着。石牙红着脸,连衣服都没脱,跳到水泡里,把球捞出来。当他拧着湿衣服,在球场上来回奔跑时,他发现,同学们不再把球传给他了。他慢慢站住了
\newpage

,默默退出球场,呆呆地看着欢笑的同学们。 

晚上,石牙刚走进宿舍门,屋里传出窃窃笑声。石牙听出那个粗嗓门是王猛的:“谁也别说,谁说
是小狗!” 

石牙一出现在门口,几个同学都愣住了。他们踢完球,正在用一块毛巾轮流洗脚。那毛巾正是石牙
洗脸用的,这是一块带着红白方格的毛巾。 

石牙久蓄在心底的泪水终于涌出来,扭头冲出门去。这污辱和歧视使他忍受不了了。他知道这一切都是父亲和王猛结下的私怨带来的,可为什么把恨都
发泄在他身上?就因为自己是父亲的儿子? 

有人拉他的衣服。他一回头,是黑小三,班里
最小的同学,王猛的影子。 

“石牙!别哭。我也用它擦脚了,一共擦过两次……刚才,我用香皂把你的毛巾洗了。你要不愿意

\newpage
,我给你买一条!” 


张石牙哭得更厉害了。 


“你还怨我吗?”黑小三哀求地小声说。 


“不!我怨我爸爸!” 


4. 惩罚 

王猛从来不知愁,这两天却愁了。石牙有好几次感到王猛想主动跟他说话,但又不把肚里的话全说
出来,还掩藏着什么。 

石牙问黑小三:“王猛怎么啦,他好像有事?
” 

黑小三说:“他妈病了,想吃鱼,到处买不到。他知道你家有船,你爸又会挂鱼。可他不好意思张
嘴求你!” 

“你告诉他,明天我们划船去取鱼。我爸每天
\newpage

都把挂网提前下好,不会空网。” 


“石牙,你真是个……好人!” 

第二天星期日,这群孩子悄悄爬上那条船,向
下游划去。 

王猛一声不响坐在船上。他不敢看石牙的眼睛。当黑小三转告了石牙的主意时,王猛心里难受了好一阵。他想,一定找个机会向石牙道歉,郑重邀请石
牙踢球。尽管他王猛从没向别人说过软话。 

他们看见了露出水面的挂网,看见了挂网在抖动。石牙脱了上衣跳下水,一边踩水,一边从网底摘
下一条尺把长的鲫鱼,扔到船板上。 

“坏了!爸爸来收网了!”河里的石牙爬上船
,把桨抓在手里。王猛和黑小三都慌了。 

“别急。我把船靠在岸上,王猛提着鱼,赶快

\newpage
回家!” 

张木头跑近时,孩子们已经上岸了。张木头看见王猛手里提着一条大鱼,急了,脱了鞋,提在手里,咒骂着撵王猛。撵了半天没追到,才气淋淋转回来
,怒气冲冲盯着船上的儿子。 


“败家仔!”张木头喷出一句带火的话。 


儿子不回答。 

张木头几步蹿上船去,劈手夺过船桨,狠命向儿子砸去。石牙一偏头,船桨砸在右肩上,被划开一道血口子。石牙捂住肩膀,眼里流着泪:“爸!你不
要太绝了!” 

“你敢顶嘴?拉纤,把船给我拖回去!”张木
头挥着手里的桨,脚把船跺得鸣鸣响。 

石牙背起纤绳,微弓着背,一手捂住肩头,在岸上走着。张木头坐在船头上,看着儿子拉纤的背影,拉长了脸说:“今天我罚你,我教训你,你就得听
\newpage
着!我掉的汗珠子比你吃的饭粒子都多,过的桥比你
走的路都长。你听见没有?” 


没有回答。 

“你这小子,越上学越坏了。明天,把行李从
学校取回来,甭上学了。在家帮我干活!” 


儿子站住了。船也停住了。 


“怎么不拉了?”张木头瞪着眼睛。 


“爸!你说什么我都听,别让我辍学!” 

“那好。你听我说,你妈死时,没有一个人下河去救。我去晚了,不是亲人,谁也不会舍命。你知
道我的意思吗?” 


“知道!” 

“如今世上好人少了,活在世上别太傻,你知
\newpage

道吗?” 



“你背上怎么了?” 

石牙低头看了一下自己的肩膀,血口子张开嘴
,涌出的血把衬衣染红了。 

张木头从船上跳起来,跨到岸上:“你怎么不
告诉我?”他撕开衣服,给儿子包扎上。 

儿子含泪的眼睛使他受不了:“你有啥话就说
!怨爸爸手狠。可都是为了咱家好!为了你!” 


“爸!把船借我用一用吧!” 


“干啥?” 


“我的同学王猛……” 


\newpage

“闭嘴!这船是我的!不是你的!” 

石牙擦了一把泪,咬着牙,背起纤绳向前走了
。张木头疑惑地盯着儿子的背影。 


5. 大水 

又是几天的暴雨,河槽注满了水。小黑河发怒
了。这是石牙肩头受伤后在家养伤的第三天。 

张木头也惧怕这场暴雨。面前的情景,使他想起几年前那场大水。他铁青着脸,回头命令儿子老老实实呆在屋里,不许走出家门一步。他拎着一把铁锨,耳朵听着河水的吼叫,奔到水田里。他要把所有的
土埂都挖开一个个缺口,把积水放掉。 

河水太满了。隔夜的挂网被水冲得没了踪影;水棒草只剩个头,可怜地摇晃着;岸边上的独船不安地摆动着船尾,像一匹被主人抽打而要奋力挣脱缰绳的烈马;那块大青石终于被水卷走了,留下一个漩涡;一条黑鱼拖着一根钓竿从上游茫然地冲下来,近了,才能看清鱼已经死了……岸边上没有了淡淡的水草
\newpage
香味,只能闻到从上游泻下的浑浊的泥水带来的水腥
气。 

张木头根本没想到,此时,河边上那间独屋的
门被人突然打开了。 

黑小三哭过的脸出现在张石牙的面前:“石牙
!不好了,王猛叫水冲走了,快划船去……” 


“这么大的水还游泳?” 


“不是,他织了个网,想给他妈挂鱼!” 

两人奔到船边。石牙解缆绳时,发现缆绳被父亲紧紧拴到木桩上,像长在木桩上一样,系着死扣。石牙马上跑回屋,操起菜刀返身冲出来,把绳子砍断了。船马上顺着水势向下游漂去。黑小三飞跑到岸上
,引着船向王猛被淹的地方奔去。 

岸上有人看见了石牙,都大声喊起来:“石牙

\newpage
来了,石牙划船来了!” 

“我来了。”石牙在心里回答了一声。他第一次感受到同学们对他的尊重,把他当作一个有用的人
。这是一种呼唤亲人的感觉,是石牙久已期待的。 

突然,水面上浮现出一个头影。他立刻认出是王猛刷子一样的头发。王猛的头若隐若现,像在潜泳。他想把手里的桨伸给王猛,可王猛的手无力地在水
面上举了举,又沉底了,形成了一个水涡。 

石牙突然大喊一声。当时,谁也记不得石牙子喊了一句什么,便传来了“扑通”一声。岸上的孩子们看见船上的石牙消失了,船板上只滚动着那根红漆
木桨,还有石牙刚脱掉的白褂。 

船失去了控制,顺着水势缓慢地转了一个头,倒退着向下游移动,仿佛也在回头留恋地朝小主人下
水的地方投射最后一瞥。 

石牙没有摸到王猛,正准备冒出水面缓口气,他的腿被昏迷的王猛抓住了,两人一起沉到水里。这
\newpage
时,石牙感到水从鼻腔里像针一样扎进了自己的胸腔
,他被无情的水呛了。 

王猛借助刚才石牙身体的浮力,把头冒出水面
,昏迷中抓住了从身边漂过的独船…… 

在河湾,当年打捞出石牙母亲的地方,孩子们把石牙捞了上来,静静地放在船板上,洗去石牙身上
的泥,呆呆地围住了这只独船…… 


6. 儿子 

“石牙子!……把尸体从船上掀下去!……我
的船上不能摆死人!” 

岸上跑来了张木头。他刚才听说又淹死了人。他用嘶哑的声音命令儿子。当他跑到船板上时,后退
了一步,呆住了。 

几个光身子的孩子跪成一圈,仿佛在等待躺着的人睡醒,这个一动不动的孩子赤裸的肩膀上,有一
\newpage
道刺目的泛红疤痕。啊,这是自己的儿子!张木头傻
了。 

王猛慢慢爬起来,爬到石牙面前,胆怯地伸手去抚摸石牙的脸。突然,他把手缩了回去,害怕地问:“石牙!石牙!你怎么啦?你怎么啦?石牙……”当发现船板上那件染上蓝墨水的白褂时,王猛一把抓在手里,把脸埋在上面,哽咽地哭出来:“我还有话
跟你说,石牙!……” 

水仿佛变得凝固了,像黏稠的液体在缓慢流动。岸上的孩子跟在逆水而上的独船的后面,默默地走
着。 

张木头自己背着纤,拖着船。他不让别人拉纤。他一步一回头,看见儿子的身躯,仰卧在船板上,随着浮动的船起伏着,像在水里仰泳。他想起了几天前儿子捂住肩膀拉他时的情景,默默地在心里呼喊:“我为什么要惩罚儿子?”他双膝突然一弯,背上的纤绳滑落下来。他趴在岸上,手捂住脸,声音从指缝

\newpage
里挤出来:“石牙子!你……” 

他一面悲怆地哭着,一面重复着几句话:“你太傻了!我的儿子,你真是太傻了!就剩我一个人啦
!就剩下我一个人啦!” 


“爸爸!” 

张木头猛然听见一声喊,抬起泪眼一看,王猛
跪在自己面前。 



紧跟着,黑小三也跪下了。 

张木头呆住了,好半天,才用手捶打着地上湿漉漉的泥:“石牙子!这船是你的,我答应你了!这
船是你的了,你听见没有?你怎么不站起来!” 


孩子们都哭了。 

没过几天,村上的人都拥到河边,把张木头的

\newpage
小屋迁回了村里。人们尊敬他。 

王猛一直保存着石牙那件白褂子。他经常去看
张木头,做一些石牙活着时应该做的活。 

人们常常看见张木头蹲在河边,守着那条独船。一遇到人,他就迎上去:“你们用船吧?你们上船
玩吧?这是我家石牙子的船!” 

人们都不愿轻易去使用这条船,这条小黑河上一的船……

\end{document}
