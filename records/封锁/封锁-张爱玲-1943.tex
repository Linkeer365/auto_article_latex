\documentclass{article}
\usepackage[utf8]{inputenc}
\usepackage{ctex}

\title{封锁\footnote{Click to View:\url{https://web.archive.org/web/20150126163305/http://www.kanunu8.com/book3/7114/155379.html}}}
\author{张爱玲}
\date{1943-08}

% \setCJKmainfont[BoldFont = Noto Sans CJK SC]{Noto Serif CJK SC}
% \setCJKsansfont{Noto Sans CJK SC}
% \setCJKfamilyfont{zhsong}{Noto Serif CJK SC}
% \setCJKfamilyfont{zhhei}{Noto Sans CJK SC}
% \setlength\parindent{0pt}

\begin{document}
\CJKfamily{zhkai}

\maketitle


\Large

	
    开电车的人开电车。在大太阳底下,电车轨道像两条光莹莹的,水里钻出来的曲蟮,抽长了,又缩短了;抽长了,又缩短了,就这么样往前移——柔滑的,老长老长的曲蟮,没有完,没有完……开电车的人眼睛盯住了这两条蠕蠕的车轨,然而他不
发疯。 

    如果不碰到封锁,电车的进行是永远
不会断的。封锁了。 

    摇铃了。“叮玲玲玲玲玲,”每一个“玲”字是冷冷的一小点,一点一点连成了一条虚线
,切断了时间与空间。 

\newpage

    电车停了,马路上的人却开始奔跑,在街的左面的人们奔到街的右面,在右面的人们奔到左面。商店一律地沙啦啦拉上铁门。女太太们发狂一般扯动铁栅栏,叫道:“让我们进来一会儿!我这儿有孩子哪,有年纪大的人!”然而门还是关得紧腾腾的。铁门里的人和铁门外的人眼睁睁对看着,互相惧
怕着。 

    电车里的人相当镇静。他们有座位可坐,虽然设备简陋一点,和多数乘客的家里的情形比较起来,还是略胜一筹。街上渐渐地也安静下来,并不是绝对的寂静,但是人声逐渐渺茫,像睡梦里所听到的芦花枕头里的赶咐。这庞大的城市在阳光里盹着了,重重地把头搁在人们的肩上,口涎顺着人们的衣服缓缓流下去,不能想象的巨大的重量压住了每一个
人。 


 

    上海似乎从来没有这么静过——大白天里!一个乞丐趁着鸦雀无声的时候,提高了喉咙唱
\newpage
将起来:“阿有老爷太太先生小姐做做好事救救我可怜人哇?阿有老爷太太……”然而他不久就停了下来
,被这不经见的沉寂吓噤住了。 

    还有一个较有勇气的山东乞丐,毅然打破了这静默。他的嗓子浑圆嘹亮:“可怜啊可怜!一个人啊没钱!”悠久的歌,从一个世纪唱到下一个世纪。音乐性的节奏传染上了开电车的。开电车的也是山东人。他长长地叹了一口气,抱着胳膊,向车门上一靠,跟着唱了起来:“可怜啊可怜!一个人啊没
钱!” 

    电车里,一部分的乘客下去了。剩下的一群中,零零落落也有人说句把话。靠近门口的几个公事房里回来的人继续谈讲下去。一个人撒喇一声抖开了扇子,下了结论道:“总而言之,他别的毛病没有,就吃亏在不会做人。”另一个鼻子里哼了一声,冷笑道:“说他不会做人,他把上头敷衍得挺好的
呢!” 

    一对长得颇像兄妹的中年夫妇把手吊
\newpage
在皮圈上,双双站在电车的正中,她突然叫道:“当心别把裤子弄脏了!”他吃了一惊,抬起他的手,手里拎着一包熏鱼。他小心翼翼使那油汪汪的纸口袋与他的西装裤子维持二寸远的距离。他太太兀自絮叨道:“现在干洗是什么价钱?做一条裤子是什么价钱?
” 

    坐在角落里的吕宗桢,华茂银行的会计师,看见了那熏鱼,就联想到他夫人托他在银行附近一家面食摊子上买的菠菜包子。女人就是这样!弯弯扭扭最难找的小胡同里买来的包子必定是价廉物美的!她一点也不为他着想——一个齐齐整整穿着西装戴着玳瑁边眼镜提着公事皮包的人,抱着报纸里的热腾腾的包子满街跑,实在是不像话!然而无论如何,假使这封锁延长下去,耽误了他的晚饭,至少这包子可以派用场。他看了看手表,才四点半。该是心理作用罢?他已经觉得饿了。他轻轻揭开报纸的一角,向里面张了一张。一个个雪白的,喷出淡淡的麻油气味。一部分的报纸粘住了包子,他谨慎地把报纸撕了下来,包子上印了铅字,字都是反的,像镜子里映出来

\newpage
的,然而他有这耐心,低下头去逐个认了出来: 


    案几妗…申请……华股动态……隆重登场候教……”都是得用的字眼儿,不知道为什么转载到包子上,就带点开玩笑性质。也许因为“吃”是太严重的一件事了,相形之下,其他的一切都成了笑话。吕宗桢看着也觉得不顺眼,可是他并没有笑,他是一个老实人。他从包子上的文章看到报上的文章,把半页旧报纸读完了,若是翻过来看,包子就得跌出来,只得罢了。他在这里看报,全车的人都学了样,有报的看报,没有报的看发票,看章程,看名片。任何印刷物都没有的人,就看街上的市招。他们不能不填满这可怕的空虚——不然,他们的脑子也许会活动
起来。思想是痛苦的一件事。 

    只有吕宗桢对面坐着的一个老头子,手心里骨碌碌骨碌碌搓着两只油光水滑的核桃,有板
有眼的小动作代替了思想。 

    他剃着光头,红黄皮色,满脸浮油,打着皱,整个的头像一个核桃。他的脑子就像核桃仁
\newpage

,甜的,滋润的,可是没有多大意思。 

    老头子右首坐着吴翠远,看上去像一个教会派的少奶奶,但是还没有结婚。她穿着一件白洋纱旗袍,滚一道窄窄的蓝边——深蓝与白,很有点讣闻的风味。她携着一把蓝白格子小遮阳伞。头发梳
成千篇一律的式样,唯恐唤起公众的注意。 

    然而她实在没有过分触目的危险。她长得不难看,可是她那种美是一种模棱两可的,仿佛怕得罪了谁的美,脸上一切都是淡淡的,松弛的,没有轮廓。连她自己的母亲也形容不出她是长脸还是圆
脸。 


    在家里她是一个好女儿,在学校里她是一个好学生。大学毕了业后,翠远就在母校服务,担任英文助教。她现在打算利用封锁的时间改改卷子。翻开了第一篇,是一个男生做的,大声疾呼抨击都市的罪恶,充满了正义感的愤怒,用不很合文法的,

\newpage
吃吃艾艾的句子,骂着“红嘴唇的卖淫妇…… 

    大世界……下等舞场与酒吧间“。翠远略略沉吟了一会,就找出红铅笔来批了一个”A“字。若在平时,批了也就批了,可是今天她有太多的考虑的时间,她不由地要质问自己,为什么她给了他这么好的分数:不问倒也罢了,一问,她竟涨红了脸。她突然明白了:因为这学生是胆敢这么毫无顾忌地
对她说这些话的唯一的一个男子。 

    他拿她当做一个见多识广的人看待;他拿她当做一个男人,一个心腹。他看得起她。翠远在学校里老是觉得谁都看不起她——从校长起,教授、学生、校役……学生们尤其愤慨得厉害:“申大越来越糟了!一天不如一天!用中国人教英文,照说,已经是不应当,何况是没有出过洋的中国人!”翠远在学校里受气,在家里也受气。吴家是一个新式的,带着宗教背景的模范家庭。家里竭力鼓励女儿用功读书,一步一步往上爬,爬到了顶儿尖儿上——一个二十来岁的女孩子在大学里教书!打破了女子职业的新纪录。然而家长渐渐对她失掉了兴趣,宁愿她当初在书本上马虎一点,匀出点时间来找一个有钱的女婿。
\newpage

 

    她是一个好女儿,好学生。她家里都是好人,天天洗澡,看报,听无线电向来不听申曲滑稽京戏什么的,而专听贝多芬瓦格涅的交响乐,听不懂也要听。世界上的好人比真人多……翠远不快乐。

    生命像圣经,从希伯莱文译成希腊文,从希腊文译成拉丁文,从拉丁文译成英文,从英文译成国语。翠远读它的时候,国语又在她脑子里译成
了上海话。那未免有点隔膜。 

    翠远搁下了那本卷子,双手捧着脸。
太阳滚热地晒在她背脊上。 

    隔壁坐着个奶妈,怀里躺着小孩,孩子的脚底心紧紧抵在翠远的腿上。小小的老虎头红鞋
包着柔软而坚硬的脚…… 


    这至少是真的。 

\newpage

    电车里,一位医科学生拿出一本图画簿,孜孜修改一张人体骨骼的简图。其他的乘客以为他在那里速写他对面盹着的那个人。大家闲着没事干,一个一个聚拢来,三三两两,撑着腰,背着手,围绕着他,看他写生。拎着熏鱼的丈夫向他妻子低声道:“我就看不惯现在兴的这些立体派,印象派!”他
妻子附耳道:“你的裤子!” 

    那医科学生细细填写每一根骨头,神经,筋络的名字。有一个公事房里回来的人将折扇半掩着脸,悄悄向他的同事解释道:“中国画的影响。现在的西洋画也时兴题字了,倒真是‘东风西渐’!

    吕宗桢没凑热闹,孤零零地坐在原处
。他决定他是饿了。 

    大家都走开了,他正好从容地吃他的菠菜包子,偏偏他一抬头,瞥见了三等车厢里有他一个亲戚,是他太太的姨表妹的儿子。他恨透了这董培芝。培芝是一个胸怀大志的清寒子弟,一心只想娶个略具资产的小姐。吕宗桢的大女儿今年方才十三岁,
\newpage
已经被培芝睃在眼里,心里打着如意算盘,脚步儿越发走得勤了。吕宗桢一眼望见了这年青人,暗暗叫声不好,只怕培芝看见了他,要利用这绝好的机会向他进攻。若是在封锁期间和这董培芝困在一间屋子里,
这情形一定是不堪设想 

    他匆匆收拾起公事皮包和包子,一阵风奔到对面一排座位上,坐了下来。现在他恰巧被隔壁的吴翠远挡住了,他表侄绝对不能够看见他。翠远
回过头来,微微瞪了他一眼。糟了 


    这女人准是以为他无缘无故换了一个座位,不怀好意。他认得出那被调戏的女人的脸谱——脸板得纹丝不动,眼睛里没有笑意,嘴角也没有笑意,连鼻洼里都没有笑意,然而不知道什么地方有一点颤巍巍的微笑,随时可以散布开来。觉得自己太可
爱了的人,是熬不住要笑的。 

    该死,董培芝毕竟看见了他,向头等车厢走过来了,满卑地,老远地就躬着腰,红喷喷的
\newpage
长长的面颊,含有僧尼气息的灰布长衫——一个吃苦耐劳,守身如玉的青年,最合理想的乘龙快婿。宗桢迅疾地决定将计就计,顺水推舟,伸出一只手臂来搁在翠远背后的窗台上,不声不响宣布了他的调情的计划。他知道他这么一来,并不能吓退了董培芝,因为培芝眼中的他素来是一个无恶不作的老年人。由培芝看来,过了三十岁的人都是老年人,老年人都是一肚子的坏。培芝今天亲眼看见他这样下流,少不得一五一十要去报告给他太太听——气气他太太也好!谁叫
她给他弄上这么一个表侄!气,活该气 

    他不怎么喜欢身边这女人。她的手臂,白倒是白的,像挤出来的牙膏。她的整个的人像挤
出来的牙膏,没有款式。 

    他向她低声笑道:“这封锁,几时完哪?真讨厌!”翠远吃了一惊,掉过头来,看见了他搁在她身后的那只胳膊,整个身子就僵了一僵,宗桢无论如何不能容许他自己抽回那只胳膊。他的表侄正在那里双眼灼灼望着他,脸上带着点会心的微笑。如果他夹忙里跟他表侄对一对眼光,也许那小子会怯怯
\newpage
地低下头去——处女风韵的窘态;也许那小子会向他
挤一挤眼睛——谁知道? 

    他咬一咬牙,重新向翠远进攻。他道
:“您也觉着闷罢? 

    我们说两句话,总没有什么要紧!我们——我们谈谈!“他不由自主的,声音里带着哀恳的调子。翠远重新吃了一惊,又掉回头来看了他一眼。他现在记得了,他瞧见她上车的——非常戏剧化的一刹那,但是那戏剧效果是碰巧得到的,并不能归功于她。他低声道:”你知道么?我看见你上车,前头的玻璃上贴的广告,撕破了一块,从这破的地方我看见你的侧面,就只一点下巴。“是乃络维奶粉的广告,画着一个胖孩子,孩子的耳朵底下突然出现了这女人的下巴,仔细想起来是有点吓人的。”后来你低下头去从皮包里拿钱,我才看见你的眼睛,眉毛,头发。“拆开来一部分一部分地看,她未尝没有她的一种
风韵。 

    翠远笑了。看不出这人倒也会花言巧
\newpage
语——以为他是个靠得住的生意人模样!她又看了他一眼。太阳光红红地晒穿他鼻尖下的软骨。他搁在报纸包上的那只手,从袖口里出来,黄色的,敏感的——一个真的人!不很诚实,也不很聪明,但是一个真的人!她突然觉得炽热,快乐。她背过脸去,细声道
:“这种话,少说些罢!” 


    宗桢道:“嗯?”他早忘了他说了些什么。他眼睛盯着他表侄的背影——那知趣的青年觉得他在这儿是多余的,他不愿得罪了表叔,以后他们还要见面呢,大家都是快刀斩不断的好亲戚;他竟退回三等车厢去了。董培芝一走,宗桢立刻将他的手臂收回,谈吐也正经起来。他搭讪着望了一望她膝上摊着的练习簿,道:“申光大学……您在申光读书!”

    他以为她这么年青?她还是一个学生
?她笑了,没做声。 

    宗桢道:“我是华济毕业的。华济。”她颈子上有一粒小小的棕色的痣,像指甲刻的印子
\newpage
。宗桢下意识地用右手捻了一捻左手的指甲,咳嗽了
一声,接下去问道:“您读的是哪一科?” 

    翠远注意到他的手臂不在那儿了,以为他态度的转变是由于她端凝的人格,潜移默化所致。这么一想,倒不能不答话了,便道:“文科。您呢?”宗桢道:“商科。”他忽然觉得他们的对话,道学气太浓了一点,便道:“当初在学校里的时候,忙着运动,出了学校,又忙着混饭吃。书,简直没念多少!”翠远道:“你公事忙么?”宗桢道:“忙得没
头没脑。 

    早上乘电车上公事房去,下午又乘电车回来,也不知道为什么去,为什么来!我对于我的工作一点也不感到兴趣。说是为了挣钱罢,也不知道
是为谁挣的!“翠远道:”谁都有点家累。“ 

    宗桢道:“你不知道——我家里——
咳,别提了!”翠远暗道: 

    袄戳耍∷太太一点都不同情他!世上
\newpage
有了太太的男人,似乎都是急切需要别的女人的同情。”宗桢迟疑了一会,方才吞吞吐吐,万分为难地说
道:“我太太——一点都不同情我。” 

    翠远皱着眉毛望着他,表示充分了解。宗桢道:“我简直不懂我为什么天天到了时候就回家去。回到哪儿去?实际上我是无家可归的。”他褪下眼镜来,迎着亮,用手绢予拭去上面的水渍,道:“咳!混着也就混下去了,不能想——就是不能想!”近视眼的人当众摘下眼镜子,翠远觉得有点秽亵,仿佛当众脱衣服似的,不成体统。宗桢继续说道:“你——你不知道她是怎么样的一个女人!”翠远道:“那么,你当初……”宗桢道:“当初我也反对来着
。她是我母亲给订下的。 

    我自然是愿意让我自己拣,可是……她从前非常的美……我那时又年青……年青的人,你
知道……“翠远点点头。 

    宗桢道:“她后来变成了这么样的一个人——连我母亲都跟她闹翻了,倒过来怪我不该娶
\newpage
了她!她……她那脾气——她连小学都没有毕业。”翠远不禁微笑道:“你仿佛非常看重那一纸文凭!其实,女子教育也不过是那么一回事!”她不知道为什么她说出这句话来,伤了她自己的心。宗桢道:“当然哪,你可以在旁边说风凉话,因为你是受过上等教育的。你不知道她是怎么样的一个——”他顿住了口,上气不接下气,刚戴上了眼镜子,又褪下来擦镜片。翠远道:“你说得太过分了一点罢?”宗桢手里捏
着眼镜,艰难地做了一个手势道: 

    澳悴恢道她是——”翠远忙道:“我知道,我知道。”她知道他们夫妇不和,决不能单怪他太太,他自己也是一个思想简单的人。他需要一个
原谅他,包涵他的女人。 

    街上一阵乱,轰隆轰隆来了两辆卡车,载满了兵。翠远与宗桢同时探头出去张望;出其不意地,两人的面庞异常接近。在极短的距离内,任何人的脸都和寻常不同,像银幕上特写镜头一般的紧张。宗桢和翠远突然觉得他们俩还是第一次见面。在宗桢的眼中,她的脸像一朵淡淡几笔的白描牡丹花,额
\newpage

角上两三根吹乱的短发便是风中的花蕊。 

    他看着她,她红了脸,她一脸红,让
他看见了,他显然是很愉快。她的脸就越发红了。 

    宗桢没有想到他能够使一个女人脸红,使她微笑,使她背过脸去,使她掉过头来。在这里,他是一个男子。平时,他是会计师,他是孩子的父亲,他是家长,他是车上的搭客,他是店里的主顾,他是市民。可是对于这个不知道他的底细的女人,他
只是一个单纯的男子。 

    他们恋爱着了。他告诉她许多话,关于他们银行里,谁跟他最好,谁跟他面和心不和,家里怎样闹口舌,他的秘密的悲哀,他读书时代的志愿……无休无歇的话,可是她并不嫌烦。恋爱着的男子向来是喜欢说,恋爱着的女人向来是喜欢听。恋爱着的女人破例地不大爱说话,因为下意识地她知道:男
人彻底地懂得了一个女人之后,是不会爱她的。 

    宗桢断定了翠远是一个可爱的女人—
\newpage
—白,稀薄,温热,像冬天里你自己嘴里呵出来的一口气。你不要她,她就悄悄地飘散了。她是你自己的一部分,她什么都懂,什么都宽宥你。你说真话,她
为你心酸;你说假话,她微笑着,仿佛说: 


    扒颇阏庹抛欤 

    宗桢沉默了一会,忽然说道:“我打算重新结婚。”翠远连忙做出惊慌的神气,叫道:“
你要离婚?那……恐怕不行罢?” 

    宗桢道:“我不能够离婚。我得顾全孩子们的幸福。我大女儿今年十三岁了,才考进了中
学,成绩很不错。”翠远暗道: 

    罢飧当前的问题又有什么关系?”她冷冷地道:“哦,你打算娶妾。”宗桢道:“我预备将她当妻子看待。我——我会替她安排好的。我不会让她为难。”翠远道:“可是,如果她是个好人家的女孩子,只怕她未见得肯罢?种种法律上的麻烦……”宗桢叹了口气道:“是的。你这话对。我没有这权
\newpage

利。 

    我根本不该起这种念头……我年纪也太大了。我已经三十五了。“翠远缓缓地道:”其实,照现在的眼光看来,那倒也不算大。“宗桢默然。半晌方说道:”你……几岁?“翠远低下头去道:”二十五。“宗桢顿了一顿,又道:”你是自由的么?“翠远不答。宗桢道:”你不是自由的。即使你答应了,你的家里人也不会答应的,是不是?……是不是
?“ 

    翠远抿紧了嘴唇。她家里的人——那
些一尘不染的好人——她恨他们!他们哄够了她。 

    他们要她找个有钱的女婿,宗桢没有
钱而有太太——气气他们也好!气,活该气 

    车上的人又渐渐多了起来,外面许是有了“封锁行将开放”的谣言,乘客一个一个上来,坐下,宗桢与翠远给他们挤得紧紧的,坐近一点,再

\newpage
坐近一点。 

    宗桢与翠远奇怪他们刚才怎么这样的糊涂,就想不到自动地坐近一点,宗桢觉得她太快乐了,不能不抗议。他用苦楚的声音向她说:“不行!这不行!我不能让你牺牲了你的前程!你是上等人,你受过这样好的教育……我——我又没有多少钱,我不能坑了你的一生!”可不是,还是钱的问题。他的话有理。翠远想道:“完了。”以后她多半是会嫁人的,可是她的丈夫决不会像一个萍水相逢的人一股的可爱——封锁中的电车上的人……一切再也不会像这样自然。再也不会……呵,这个人,这么笨!这么笨!她只要他的生命中的一部分,谁也不希罕的一部分。他白糟蹋了他自己的幸福。那么愚蠢的浪费!她哭了,可是那不是斯斯文文的,淑女式的哭。她简直把她的眼泪唾到他脸上。他是个好人——世界上的好人
又多了一个 

    向他解释有什么用?如果一个女人必须倚仗着她的言语来打动一个男人,她也就太可怜了
。 

\newpage

    宗桢一急,竟说不出话来,连连用手去摇撼她手里的阳伞。她不理他。他又去摇撼她的手,道:“我说——我说——这儿有人哪!别!别这样!等会儿我们在电话上仔细谈。你告诉我你的电话。”翠远不答。他逼着问道:“你无论如何得给我一个电话号码。”翠远飞快地说了一遍道:“七五三六九
。” 

    宗桢道:“七五三六九?”她又不做声了。宗桢嘴里喃喃重复着:“七五三六九,”伸手在上下的口袋里掏摸自来水笔,越忙越摸不着。翠远
皮包里有红铅笔,但是她有意地不拿出来。 

    她的电话号码,他理该记得。记不得
,他是不爱她,他们也就用不着往下谈了。 

    封锁开放了。“叮玲玲玲玲玲”摇着铃,每一个“玲”字是冷冷的一点,一点一点连成一
条虚线,切断时间与空间。 

    一阵欢呼的风刮过这大城市。电车当
\newpage
当当往前开了。宗桢突然站起身来,挤到人丛中,不见了。翠远偏过头去,只做不理会。他走了。对于她,他等于死了。电车加足了速力前进,黄昏的人行道上,卖臭豆腐干的歇下了担子,一个人捧着文王神卦的匣子,闭着眼霍霍地摇。一个大个子的金发女人,背上背着大草帽,露出大牙齿来向一个意大利水兵一笑,说了句玩笑话。翠远的眼睛看到了他们,他们就活了,只活那么一刹那。车往前当当地跑,他们一个
个的死去了。 

    翠远烦恼地合上了眼。他如果打电话给她,她一定管不住她自己的声音,对他分外的热烈
,因为他是一个死去了又活过来的人。 

    电车里点上了灯,她一睁眼望见他遥遥坐在他原先的位子上。她震了一震——原来他并没有下车去!她明白他的意思了:封锁期间的一切,等于没有发生。整个的上海打了个盹,做了个不近情理
的梦。 

    开电车的放声唱道:“可怜啊可怜!
\newpage
一个人啊没钱!可怜啊可……”一个缝穷婆子慌里慌张掠过车头,横穿过马路。开电车的大喝道:“猪猡
!” 

    吕宗桢到家正赶上吃晚饭。他一面吃一面阅读他女儿的成绩报告单,刚寄来的。他还记得电车上那一回事,可是翠远的脸已经有点模糊——那是天生使人忘记的脸。他不记得她说了些什么,可是
他自己的话他记得很清楚——温柔地: 

    澳恪-几岁?”慷慨激昂地:“我不
能让你牺牲了你的前程!” 

    饭后,他接过热手巾,擦着脸,踱到卧室里来,扭开了电灯。一只乌壳虫从房这头爬到房那头,爬了一半,灯一开,它只得伏在地板的正中,一动也不动。在装死么?在思想着么?整天爬来爬去,很少有思想的时间罢?然而思想毕竟是痛苦的。宗桢捻灭了电灯,手按在机括上,手心汗潮了,浑身一滴滴沁出汗来,像小虫子痒痒地在爬。他又开了灯,

\newpage
乌壳虫不见了,爬回窠里去了。 

    (一九四三年八月)

\end{document}
