\documentclass{article}
\usepackage[utf8]{inputenc}
\usepackage{ctex}

\title{狼牙跑道\footnote{Click to View:\url{https://web.archive.org/web/20220630024359/http://www.ciyuku.com/ertongwenxue/2737.html}}}
\author{北董}
\date{2005-07}

% \setCJKmainfont[BoldFont = Noto Sans CJK SC]{Noto Serif CJK SC}
% \setCJKsansfont{Noto Sans CJK SC}
% \setCJKfamilyfont{zhsong}{Noto Serif CJK SC}
% \setCJKfamilyfont{zhhei}{Noto Sans CJK SC}
% \setlength\parindent{0pt}

\begin{document}
\CJKfamily{zhkai}

\maketitle


\Large


农家女娃一辈子忘不了那个九月一日。 

做过种种揣想,有了一百样准备,该去城里的“大”中学报到了,她还是往返了三趟村南那巍巍颤颤的独木桥。这跟往日打草背柴过桥的滋味大不一样啊!“妈,卡子!”“妈,带几块白薯,红心的!”“妈,裤衩的备用松紧带儿!”妈妈嗔她你这个丢三落四的丫头。她喜得颠脑瓜,用两只眼睛一齐与妈妈打趣。“我不可能个个礼拜往家溜呀!”妈妈也喜,
村里人把城里的中学叫做“大中学”哩! 

她是搭了邻家伯的小四轮一溜啵啵啵来到城里的。伯,再见!有空到我们中学来!邻家伯眼前这两辫朝天的土丫,仿佛已成了“我们中学”的女校长,那么自豪,还有二分五的矜持。伯说你是咱们村的女
\newpage
娃梢梢,状元坯胎,伯等喝你念成大书做成大事的还乡酒。女孩点下颏,自信,两只眼睛一齐宣告,伯你
只管等着。 

当时报到的很多,许多学生或者家长都嘟噜着许多东西。她背负着大行李卷7字形地看罢那两大张“报到须知”,撸两把鼻子上的汗豆,甩甩,就撵到了宿舍。妈吔,洋楼!宿舍不是楼,她指的是双层铁架床。她在一“楼”试试躺,颠颠,铁架床吱吱叫,很好。又爬到二“楼”,又试试躺,也吱吱叫,很好。养兔就这样,她想,咱属兔,这回才真正地“兔”
起来啦! 

她终于发现铺架上贴了黄纸条,纸条写着名字,便找到自己的铺位,也是二“楼”。她知道了这屋里有唐冰冰、马静,周蕤等12名伙伴。唐冰冰,这
名字多棒! 

她蓦然想起自己还没去交这费那费,那卷票子可别耍丢了。她跳下床就往外跑,不料撞到了一个人的怀里。农家娃没有说“对不起”的习惯,只对人家
\newpage
笑笑。人家的两只上眼皮往下做了个重合,说真讨厌
! 

她就愣了。我怎么讨厌了?难道这就讨厌了吗
? 

但这是心里的打鼓,她没有说出来。她还闹明白了,说她真讨厌的并不是那挨撞的女孩,是旁边的一个。她的脸色血红,难过得想哭。那路见不平的女
孩还不肯宽容,说“没长眼睛啊?” 

我没长眼睛,也看见了你脸上有几粒家雀屎!
 

她心里说,依然埋在心里。那挨撞的女孩去拿行李中的什么东西。她极美,美得像弗莉苏尔小兔—
— 

她就这么想了。她在家养着两对弗莉苏尔兔,她深为小兔的美丽倾心。其实,小兔的美,不仅在外表,而且在心灵。她不只一次地看到,纵使只有一片
\newpage
树叶,一寸薯秧,两只兔也不会抢夺厮打,它们各嚼一头,渐渐推进,直到两只嘴巴相吻。多么和睦多么
善良的弗莉苏尔呀!她多少回赞叹。 

弗莉苏尔显然也不满,这更让人不好受。农家女孩步子快然,极迟慢地走到摆在教导处窗子外面的
一溜长桌前。 

人很簇拥,个个朝前挤。中国人喜欢簇拥,喜欢朝前挤,她在路途中看人们上公共汽车,就是这样。她不想去挤,尽管她是个急性子。妈妈常常说她,鸽子你叫三声狗狗不来,你敢把屎吃了!她说妈你甭糟踏人,我叫三声狗不来,我把狗宰了!现在,她确实不想去挤。她欣赏着那一片各种颜色的脊背和后脑
勺儿。挤什么?迟早还不能交? 


“学费,冰冰我俩的!唐冰冰,马静!” 

她听见雀斑女士异常响亮的呼叫。又见那美丽的弗莉苏尔公主在前面站着。显然,雀斑女士是后者

\newpage
的干将,横冲竖挤,所向披靡。 


“书费,冰冰我俩的!” 

她和她是一对朋友呢。她想。人家相好。我可还没朋友。我在亮甲营有的是朋友。在这里一个也没
有,只有两个“熟人”叫讨厌。 


“伙食费,冰冰我俩的!” 

她很研究地打量弗莉苏尔的倩影。她注意她雪白的颈上挂着纤细而粲然的项链,项链使领间那片V形地带笼罩了神秘的色彩。弗莉苏尔的背影也美极了,水红色泡泡肩小褂,系在湖绿色短裙里边,裙底上,摇曳一片清枝秀叶的竹影。人家那袜是雪白的,与
雪白的皮凉鞋非常相配。 

她不由自主地俯视了一眼自己的绿胶皮鞋,毛
蓝裤,—声妈吔,响亮在心的角角落落。 

回到宿舍以后,她不知道是人家的短裙生了胶,还是自己的目光生了胶,反正目光和短裙就粘在一
\newpage
起,分不开了。直到好几年以后,她还懊悔下面这个难脱鲁莽的动作——她在唐冰冰的背后弯下腰,她要弄明白,人家裙子上竹的图案是织出来的,还是印出来的。她也想知道那裙的质料是绵纱,是化纤,还是丝。她伸出手,轻轻撩起那裙的下摆。(标新立异)

如果不是突然有人惊叹好大的疤,那么她怎么也不会让人目光脱轨,转到那精美得一如艺术品般的大腿上去的。真的,公主的右腿根上有块疤盘踞着,瓶盖大小,边缘不齐,颜色青红,表面光亮而不平。这样的大疤,使她想到的头—个字不是丑,而是痛。弟弟肚脐旁小疱如豆,还妈呀妈呀地地哭个没完呢。为这一片“痛苦的遗址”,农家女娃一下子想到了姑夫那儿的鹿角、鹿血、鹿茸。治疤有没有特效药呢?

“谁?”那美丽的艺术品突然一跺,裙摆也随即嘭地一击,疤被裙遮了,恰到好处。那一声断喝之后,农家女孩一惊,差点跌倒。那—声断喝之后,唐冰冰的如笋的玉指已经指定了她的眉心。“你讨厌极了!你!你真不知耻!”马静显然并不知道发生了什

\newpage
么,但马上配合火力,说没教养的土佬滚远点儿! 

她泪花盈眶地解释,说我不是看疤的,我想看看裙子。弗莉苏尔已不屑理她。雀斑说人家的裙子,
用你看啦?你打票了怎么的! 


下午派座。 

“唐冰冰!李鸽!”班主任的食指定乾坤,—勾一点指挥着,农家女娃就配了美丽的弗莉苏尔。李鸽便怀了既高兴又惴惴不安的心情,小心地把书包放进桌肚。岂料雀斑女士包打天下,硬把—名小男生史
公长配与李鸽,拉去了唐冰冰。 

“好得很嘛!”小男生史公长铜喉铁嗓,大叫着我爸先养蝎子后养蜜蜂带刺儿的玩艺咱见多啦;也见厌啦!”说罢,往前吹桌面,尘土便—飞扬,前桌的雀斑便只剩白眼不见黑眼。史公长又叫道:“鲶鱼配鯻头!”雀斑女士不示弱,连发子弹射击—般:“
那可真是那可真是那可真是……” 

胖胖的女孩叫刘炼,抬抬眼镜说“静一点吧静
\newpage
—点吧!暂时的胜利常常是永久的失败,最锋利的伤
害最可能殃及自己!” 

鸽有些头晕。城市里的人太学问,咱以后怎么活呢?亮甲营小学过了六年,原来是一片寸草不生的
白地…… 

幸好开学伊始先劳动。他们初一(1)班的任务是修跑道。李鸽就高兴。她最爱劳动,也最会劳动
。她希望在劳动中展示风采。 

不知哪位明白人提出,跑道内外要用砖圈起来,镶成齿形的花边儿。同学们便献砖。鸽头一趟回家就为了这事儿。她扒了鸡窝,摘了烟囱,嘎吱嘎吱将138块砖用独轮木车推到学校。她刨沟也能,埋砖也能,一人顶三人。班主任夸奖她立下了汗马功劳。她听了心里格外舒服。可惜马静和唐冰冰让人扫兴,
一个悄悄说“汗驴”,一个窃笑说“汗熊”。 


不知谁最先给这跑道取名叫狼牙跑道。 

\newpage

到后来,李鸽才知道给跑道安上狼牙是何等愚蠢。那简直是愚蠢的平方乘以愚蠢的立方,再乘以愚
蠢的四次方、五次方…… 

前三周没上体育课,据说老师到医院生孩子去了。鸽喜欢体育课,她能跳绳,也能摔跤,最能掰手
腕。 

新老师来了,人高马大,男的,人很严厉,有句“你甭解释”的口头禅。第一天有人就送他一个熊的外号。头一节课谁都提心吊胆,但是他的双杠和吊环使人五体投地。不知为什么,他喜欢唐冰冰,他叫她不曾带过一个“唐”字。“我任命冰冰做体育委员
!”他宣布。雀斑女士带头鼓掌,很疯,很真诚。 

地球上根本没有这种鬼跑道!甭解释讥讽地叱道。他显然不通或者忘了语法。鸽便惭愧。队列里又有了“汗驴”、“汗熊”的窃窃声,鸽觉得当初的功
劳变成了如同做贼的耻辱。老师忽然轻松地说: 

“谁愿意踩着狼牙往前走,倒还可以练练平衡
\newpage

。谁试试?冰冰,走走看!” 

鸽没有受到任何人的鼓励,一种与生俱来的表现欲促使她摆摆地走上狼牙。这时的队伍已经散开,所以比较随便。她走得十分乖巧,准确而平衡,如岩羊一样。道理很简单,她从三岁就走摆村南虎羊河上
的独木桥。 

史公长又是铜喉铁嗓,说:“棒!升亮甲店村
旗!” 


马静白他一眼,唇线鄙夷成弧。 

弗莉苏尔娇羞地笑着,鹤般的长腿前后叉开,脚儿小鱼般地在空气中游,慢慢踏向砖棱。她两臂婀娜地摇,走得趔趔趄趄,口里咿呀地叫着,总统视察
般的高贵。 

鸽跑到她的后面。一半是欣赏,一半是惦记。

她只怕不识事的风儿吹来,掀起那裙,亮出那
\newpage
片痛苦的遗址。弗莉苏尔太美了,鸽实在觉得那片疤
盘踞在弗莉苏尔身上“真缺德”。 

雀斑女士嘿嘿地赞美。同学们为她这种一文不值的捧臭脚鄙夷摇头。近来,不知谁送她一个外号叫做“侍者”。   人高马大的体育老师,在他的选手前面倒退着鼓励,伸出双臂以空气传导着支持。“嘿,看我们冰冰!坚持,坚持,不要慌,多美的一头
小鹿!” 

漫无边际的褒奖使“小鹿”自己也笑了。她喘息着跳下狼牙。鸽不合时宜地傻冒了一句:“这就像小鹿?小鹿才不是这个样子呢:!”显然,她对老师的评价不予赞同。鸽对鹿的了解比对城里娃还熟的。她每年要四五次去姑家。姑夫是养鹿的把式。她知道按年龄、犄角鹿有毛槽子、二杠子、三*子、怪角子,知道鹿血、鹿茸、鹿这鹿那都是好药材;知道公鹿打架以角相抵,母鹿打架如人而立,四“臂”相搏,“巴掌”击得呱嗒响。鹿们在走路这样东倒西晃?哎
,你们见过鹿吗?除非鹿得了美尼尔!(词语库) 

\newpage

(李鸽的舅舅得了美尼尔综合症,总是头晕。
) 

唐冰冰美丽的脸盘凝成了肉冻。侍者及时地啐出一口不含痰的液体。液体落到一齿狼牙上,狼牙便
也有了一块“痛苦的遗址”。 

鸽很悔。人家又不高兴了。啧啧!她那么希望
与这美丽的人儿相好,自己却胡说了没用的话。 

熊老师肯定早已扫兴,但没发脾气。他粗壮的
食指抡着哨子的系绳,说下课了甭整队了解散! 

鸽很久以后回家的时候,伙伴们问她在“大”中学开心不开心,她说,开心。她不忍把一个梦般的向往说破。“大”中学是伙伴们心中的一个谜。李鸽
是亮甲营唯一的一个“大”中学生。 


日子一天天过去。 

鸽获得了许多友谊。但是,她始终不能获得弗
\newpage
莉苏尔和侍者。鸽是个朋友迷,她为得不到她们而痛苦。几年以后她回忆这段生活,还说她当时得不到弗莉苏尔,就像害了单相思,简直无法接受那个事实。

史公长看破了天机。“你这个家伙干吗老想交
朋友呢?” 

不知道。鸽只想交朋友,却没想过交朋友“干
吗”? 

刘炼则说:“友谊也许本来就是吃快餐,吃着
就吃,吃不着换店!” 

鸽的眼睛张得大大,怅惘起一片迷离的亮光。史公长又说:“心壳子一米厚的人,你甭理她。除非
你有钻探机!” 

冰冰对于鸽,犹如痛苦的遗址对于冰冰——成了一个不幸的存在。世界上有许多不幸的存在,真是没有办法。鸽夜里常失眠,脑际常有弗莉苏尔的冰冷和侍者的“含蓄”。她梦见自己在狼牙跑道上走狼牙
\newpage
,无尽无休地走着,累一身大汗,把被子打得精湿。她向冰冰道过歉,解释撩裙子的事情。人家不听,只说无聊,神经病!鸽想帮助冰冰,怎奈冰冰用不着她帮助,人家是万事不求人。好在做值日、接力赛、周六劳动的时候她可以多出些力,不过人家根本不被她
感动。 

她编算着回家周,她想去姑家,让姑夫给想想治疤的药方。如果治好了冰冰的疤,冰冰就会变成另一个人,那也许就跟她相好了。她很快就否定了回家周去姑家的打算,因为包括周六下午才一天半的时间
,根本完不成一个往返。便盼着国庆节。 

国庆节珊珊而来,鸽不回家,她乘汽车,倒火车,还徒步30华里,半夜才赶到那个三面环山的小村。姑姑、姑夫被她的到来吓坏了,以为家中出了不幸。她说漂亮女孩有疤,有疤的漂亮女孩心壳太厚,不肯交朋友,她说不相信交朋友就如同吃快餐。姑姑和姑夫都听得一塌糊涂。她按下渴和饿重说了一遍。她求姑夫给些“鹿药”,去治好一位同学的疤,她答

\newpage
应寒假来给表弟补习功课。 

养鹿人听罢哈哈大笑。“你这娃儿念书念了个明白!天底下有治疮的没治疤的,疤不疼不痒,你惹
它干吗?来,我看看疤在哪儿?” 

鸽说:“我不是说了嘛,疤不是我的,是我的同学的!”疤,鸽浑身没有一个,如果说有,只心上
那一颗。那是一颗隐隐作痛的疤。 

这次徒劳的远行她从不对人讲起。她固执地绝
望就是因为世界上无药治疤 

鸽终于知道冰冰是“大家闺秀”,家庭条件是她用半个世纪也无法追及的。可是鸽无法明白冰冰为什么拒绝友谊。鸽多少次在无月的夜晚徜徉于狼牙跑
道上,她没有找到那闭合图形的终点。 

那回,她从表姑家回来已是夜下十一时了,她打烂铁门喊破嗓子,才弄醒了看门的老头。来到宿舍外,见里面还亮着灯光。走近门前听里面正在朗诵。唐冰冰朗诵得真好听,还唱了一段英语的《白桦林》
\newpage
。大家没睡,这使得鸽很踏实,因为只有这样她才不必担扰人休息的过错。她等《白桦林》唱完以后才推门进屋,她随着女同胞们的鼓掌而鼓了掌。她说唱得
真好听。 

女同胞们打招呼,跳下床到她兜里翻吃食。刘炼打趣说李小姐幽会真是忘返呀!鸽说别瞎说啊,把
姑姑给的核桃枣子分给大家尝。 

唐冰冰和侍者都仰躺下去。鸽把核桃枣子送到她们的枕边。人家一个说倒牙,一个说硌呀,核桃枣
子一动不动。 

鸽酸酸地爬到“楼上”躺,伤心地琢磨那一层
厚厚的心壳。 

体育老师为男性,这对女生来说有点不便。甭解释习惯地岔腿而立成为一座雄伟的金字塔,他说今后你们女孩子每月的事情课前报告给冰冰,我会做出适当安排的。没有‘事情’却装傻充愣打马虎眼,我可一律不客气!你该跑就得跑,该跳就得跳,甭解释
\newpage


每个女同胞都红着脸,看并没有云彩的蓝天。

唐冰冰站在队列外头,体育委员嘛。她朝金字
塔点点头,表示我明白,老师。 

大概就是说这话的下一节体育课,老师点了一串名字,无疑属于“有事情”者。她们去拔净茉莉花畦的杂草,然后给体育教研组去擦擦玻璃。鸽不明白
老师为什么忘了自己的名字,怎么办?她好为难。 

眼看人家都走出操场的边缘了,她鼓鼓勇气,
尾随过去。 

“李鸽!回来!”熊老师的声音如惊雷落地。


“要混水摸鱼吗?回来!” 

鸽觉得腰骨被那声音震断了。觉得脸在淌血。

她的耳边掠起狂风,故乡的大树枝摧干裂。她
\newpage
几经努力才回想起,她是在厕所里跟唐冰冰讲过的,
此时她朝唐冰冰投去了求助的目光。 

唐冰冰也有“事情”,她也走出去了。但是操场风云吸引她转脸驻足。鸽和熊老师都看着她,她的
表态将是一枚开闭攸关的键钮。 


她就那么历史性地摇了摇头。 

熊老师大光其火。上节的规章这一节就泡汤,那简直是用鞋底打他的脸。“李鸽我罚你跑30圈!
 向右——转!跑步——走!” 


“老师,……”她掰起指头,嗫嚅着。 


“甭解释!跑步——走!” 

鸽很犹豫。们又不敢拖延。她丢了心的躯壳逆时针地沿了狼牙跑道颠下去,每一步都令她羞辱难堪。她的泪水禁不住流下来,每一齿狼牙从她的视线中划过,她都感到视网膜的刺痛。她并不知道跑了多久
\newpage
,看见一个模糊的冰美人在茉莉花畦边蹲着望她,侍
者摇块手帕扇着自己和她主人般的朋友。 

操场比邻着初一年级4个班的教室。鸽知道此刻每个窗门都贴满了眼睛。谁都会看到一个挨罚的女
孩,正处在一个倒霉的“现在时”。 

有人批评过第四节上体育课违反科学。当时鸽非常饿,胃里泛起恶心。她的重心渐渐落在脚跟上了
,头很重,很晕。 

“冰冰,你给她数着圈儿!”熊老师把哨子抛
给体育委员,他踱回宿舍去了。 

唐冰冰机智地斜向坐在篮球架下,不像监视,又不像没监视。但是她的脚边一道一道蹭划着痕迹,
用的是一粒尖角的石子。 

30圈是个什么距离呢?每圈400米。鸽的腿越来越酸软,小腹坠坠的。有人喊熊回洞了。有人

\newpage
喊80圈超了。有人喊“其一犬坐于前”。 

鸽悲哀地想起了鹿。姑夫的鹿王曾经—跳跃过
了三米高的围墙…… 

太阳恶毒地放出火箭,报复着—名被它误认的夸父。迎面而来的狼牙跳动起来;鸽的大地轰然塌陷;太阳尖叫着裂成碎片,每—片都化成一只萦飞的小
虫…… 

如果不是凑巧,老师罚学生跑步这样的区区小事不值一提。当时省里派下—个教育考察团,抽签“碰”’到了这所学校。不知哪位同胞写了个纸条放到奥迪轿车的方向盘上,“狼牙跑道事件”就成了个小
话题。省里人、局里人和校长先生“三堂会审”。 

李鸽出院了。晕倒的时候一齿狼牙咬了她的额头,颅骨轻度损伤,额头上便有了一块疤痕。她的目光里有了—种东西,那种东西过去的确不曾属于她。

“我……有错。”熊老师的金字塔不那么雄伟了。“李鸽同学逃避体育锻炼,……体育委员唐冰冰
\newpage
证明她没有……女孩子的事情,我罚她……是我脾气
不够好……” 


他很沉痛的样子。 

鸽的眼睛鹞般地喷射冷焰,她无畏地逼视住唐
冰冰,她要她吐出真诚。 

“没有,我没有摇头。”唐冰冰从容不迫地说,还像有一点微笑。“我也不知道李鸽的‘事情’。
” 


熊老师吃惊地望住他的女弟子。 


女弟子镇定的面孔似一盘冷月。 

“冰冰,你别慌啊,你慌什么呢?你是不是叫
医疗费给吓住了?”熊老师的口气是循循善诱的。 

“我慌什么?我没摇头就是没摇头!我从不用

\newpage
摇头表示否定!” 

“冰冰,你告诉大家,你当时知道李鸽的身体
状况吗?” 

“不知道!没人向我说,也没人问我!”(瑕
疵) 


李鸽陡地站起。 


校长说李鸽你坐下。 


熊老师把手指弄得格崩一声。 


桌上一只马蹄表疯狂地跋涉。 


熊老师猛地站起,质问唐冰冰: 

“那么,你,”他没说完,唐冰冰已经笑了,那笑容极富调侃味道,这使她的老师下了某种决心。

“我要罚她三四圈,你为什么要罚她三十圈呢
\newpage

?你和她到底有什么成见?” 


李鸽又站起来,却听唐冰冰说: 

“你说让她跑三四圈,我也说让她跑三四圈呀
!——不,对了,我根本什么也没说!” 


“审判员”们有些交头接耳。 

李鸽已经用鹞般的眼球盯着唐冰冰,她浑身剧烈地发抖,朝唐冰冰走过去。谁也不知道她想问什么,或者想干什么。唐冰冰故作轻松地吸着鼻子,—只脚尖敲起什么曲子的拍节。李鸽的鹞眼谁也不看,只看唐冰冰。挨得近了,唐冰冰有些骇然,急忙问李鸽你要干什么?李鸽不答,恶狠狠抽出手,盯牢冰美人那张娇嫩如膏的脸,人们估计得出,那将是一个电光
石火惊天动地的嘴巴。 

不料,像被无形的剑斩断了臂,那只手突然颤颤抖抖地垂落下去,它的主人一声无字的长啸,口吐

\newpage
白沫,倒在地上。校医来来了,喊她,久久不醒。 

后来,李鸽还是活了。一切都大白,因为群众是真正的英雄。三个人都受了处分,李鸽的医疗费由教育局“特事特办”地予以解决。史公长从此滋生了种恶习——骂人,挨了好几回批评都改不掉。

\end{document}
