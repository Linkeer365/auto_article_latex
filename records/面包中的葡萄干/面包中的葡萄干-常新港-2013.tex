\documentclass{article}
\usepackage[utf8]{inputenc}
\usepackage{ctex}

\title{面包中的葡萄干\footnote{Click to View:\url{https://web.archive.org/web/20220710145411/http://libgen.is/book/index.php?md5=7EA10189A554C5BC2F32519D5DC0062C}}}
\author{常新港}
\date{2013-05}

% \setCJKmainfont[BoldFont = Noto Sans CJK SC]{Noto Serif CJK SC}
% \setCJKsansfont{Noto Sans CJK SC}
% \setCJKfamilyfont{zhsong}{Noto Serif CJK SC}
% \setCJKfamilyfont{zhhei}{Noto Sans CJK SC}
% \setlength\parindent{0pt}

\begin{document}
\CJKfamily{zhkai}

\maketitle


\Large

小蚁在小学五年级时,第一次登台演出。其实,小蚁的那个节目很一般,所有的学生都不记得那个节目了。她在台上讲了一个故事。故事也不长,有三分半钟。当时,她站在台上一动不动,嘴也张合得不大,坐在后面的学生像是在听小蚁的录音。小蚁的节目是替补。因为班主任紧紧抓住的一个节目无法演出,那是一个演唱节目,节目的女主角嗓子突然哑了。发不出一点儿声音。吃药、吃梨和吃冰糖都救不了她的嗓子。最后。小蚁才登了台。小蚁记得,那个自己讲述的三分半钟的故事,感动得自己流了眼泪,故事讲得很流畅,没有中断过,一气呵成。小蚁觉得讲完时,台下响起了轰鸣的掌声。过了一些日子,她才觉得自己听到的不是掌声,是自己脑子里紧张的轰鸣声,让她产生了错觉,判断有误。班主任在讲评演出的节目时,只有一句提到小蚁:“我们班的小蚁也登
\newpage

台表演节目了。” 

小蚁的脸很红,渴望老师再多说两句自己的登台表演,但是。班主任不说了,而是说到了上台表演
的别的同学。 

小蚁永远不会忘记自己在五年级那个元旦的演出。窗外还在飘雪,她的内衣,都让紧张和兴奋的汗浸透了。这让她的细若游丝的记忆变得棒棒糖一样的甜。这棒棒糖就是放在最温暖的地方。也不会融化。
 

小蚁在十岁那年的一天,突然问爸爸和妈妈:
“我为什么叫小蚁?” 

这问题被提出来,有点突然。因为,在爸爸和
妈妈看来,这不应该成为一个问题被认真提出来。 

爸爸和妈妈相互看了一眼,妈妈的头朝小蚁的方向轻轻摆了一下。让爸爸回答小蚁的问题。爸爸像个不听话的孩子,在回答问题之前,先顽劣地笑了一
\newpage
通,然后才回答小蚁的问题:“你是问为什么叫小蚁
不叫大象吧?” 


小蚁等不急了:“爸爸快点说。” 

爸爸说:“小蚂蚁吃东西时,一点声音都没有。我和你妈妈买回什么吃的零食,你都不让零食过夜,总是无声无息地吃光。你妈妈有一次说,这孩子是
什么时候吃光了零食的?怎么像个蚂蚁?” 

“真的像蚂蚁哎!”妈妈也感叹着。妈妈对小
蚁的这个小名无怨无悔。 


小蚁问:“你们不能再给我起个小名?” 

爸爸说:“难!小蚁这个名字已经叫出去了,
收不回来了。" 

妈妈说:“难!我们叫了这么多年,改不了口
了。” 

\newpage

那天,小蚁还在卫生间洗脸,突然听见爸爸声
音不大不小地叫一个人的名字:“倩倩!倩倩!” 

小蚁匆忙擦了一把脸,探出头来:“家里来客
人了?” 

她看见爸爸和妈妈脸上出现了一种坏坏的笑意
,她问:“谁是倩倩?” 

爸爸说:“经过我和你妈慎重考虑,今后,你
的小名叫倩倩了。” 

“倩倩?!”小蚁突然觉得“小蚁”是一个跟随自己这么久的一个朋友,它要告别自己,不再回来,心里不舍,嘴上就排斥“倩倩”了:“倩倩是多恶
心的名字?你们连这种名字都能想出来?” 

爸爸就笑起来,妈妈也用手捂着嘴笑起来,指着爸爸说:“是你爸爸故意恶心你的,都是他的主意
。” 

\newpage

小蚁一下子安静下来,继续在卫生间洗漱。觉得自己留住了一个叫“小蚁”的亲切朋友,并把它永
远留在自己的生活中。 

小蚁是在六岁时开始离开妈妈,单独睡觉的。第一个晚上,她跟妈妈腻了好久,不肯到自己的小屋里去。小蚁从来没见妈妈如此狠心地说:“今天,小蚁必须自己睡了,因为你已经六岁。你一定要经过这
第一次。” 

小蚁从爸爸和妈妈的眼里看出她躲不过这“第一次”了。就抱着自己的枕头很委屈地走进了那间小屋。小蚁的这间屋子只有七平方米,但是,小蚁就是觉得它太大了。大得让她不能放心,大得太不安全。小蚁站在门口不动,心里在拼命挣扎。妈妈盯着小蚁的背影,就知道小蚁的眼泪已经掉出来了。妈妈走进小屋,教给了小蚁一个办法,让她别关掉床头上的灯
。说夜色最怕的是光线,有了光,夜色就躲开了。 

尽管妈妈讲得有声有色,小蚁还是怕面对黑暗,她觉得黑暗就藏在离自己很近的地方,一有机会就
\newpage

冒出来,淹没自己。 

小蚁坚持到半夜也不睡觉,她觉得自己一闭上
眼,黑夜就醒了,专跟她作对,爬到她床上…… 

小蚁在第二天早上醒过来,认真地对爸爸和妈妈说:“昨天夜里。我听见有人敲我的窗户。当当当。当当当……你们就没听见?”爸爸说:“自己吓唬自己。谁能爬上七楼窗户去吓唬你呢?它就不怕从七
楼掉下去?” 


小蚁故意找的理由被爸爸轻易就识破了。 

但是,她很快找到了一个生存办法,以便度过提心吊胆的漫漫长夜。当然了,小蚁的这个办法在她
上一年级后,就羞于启齿了。 

她把自己那七平方米的小屋搞得很乱,乱得像是天天被小偷洗劫过。妈妈站在小蚁的门口克制着自己的情绪:“小蚁啊小蚁。你真的要变成一只蚂蚁吗

\newpage
?太乱了太乱了……” 

妈妈问她,为什么把屋子搞成这样?小蚁说:
“弄成这样,我才不怕黑夜。” 

“我不懂,你说清楚点。”妈妈确实不懂,一个女孩子把屋子搞成这种见不得人的样子,就不怕黑
夜了? 


小蚁说:“我可以躲到洞里的。” 

“洞里……”妈妈一看,被子和褥子让小蚁搞成了三个洞穴,枕头和床头也弄成一个可以藏住半个身体的小洞。妈妈又气又好笑地把头伸进一个“洞穴”里,看见洞的四壁都是小蚁吃水果时蹭在上面的各种颜色的果汁。“把被子染成这样,洗都洗不掉了……”妈妈的脸上没有笑,只剩下急了,她开始动手拆小蚁的被褥,把小蚁吓哭了。爸爸一直看着眼前发生的事,见小蚁妈妈生气了,就说了一句话:“你把小
蚁的洞毁了,让小蚁在夜里睡哪里?” 

小蚁本来是准备好好大哭一场,用眼泪和哭声
\newpage
声讨妈妈的暴行,因为爸爸的一句话,她不哭了,走
上去拉住爸爸的手,哀求道:“爸,我的洞……” 

那天晚上,在小蚁要睡觉之前,小蚁让爸爸帮助她在床上搭起了几个“洞穴”。小蚁钻进洞里要睡觉时,她要求爸爸在洞口趴一会儿,看着她睡着之后
,再离开。爸爸答应了,就趴在小蚁的洞口。 

小蚁半闭着眼,看见爸爸趴在洞口守着她,小蚁感到很踏实很幸福。但是,她却睡不着了。不知过了多久,小蚁看见爸爸趴在洞口不动,就伸手揪爸爸
的耳朵,却发觉爸爸已经疲劳地睡着了。 

这样,洞里的小蚁瞪着大大的眼睛,小心守护着洞外的爸爸。为了更安全,她把爸爸的一只手,轻轻拉进洞里,然后,她攥住爸爸的右手食指,睡着了
。 

小蚁一岁的时候,常被爸爸架起来骑马一样坐在爸爸的肩膀上,她的位置很高,紧紧抓住爸爸的右

\newpage
手食指,就会让自己很安全了。 

小蚁上幼儿园时,一个男孩子老是欺负她,把鼻涕偷偷抹在她身上,到最后,见小蚁不反抗,就公开把小蚁的衣服当手绢使用了。小蚁不想去幼儿园了,她说,除非带上爸爸的右手食指。在她说这句话时,她的手正紧抓住爸爸的右手食指不放。爸爸就认真地左顾右盼,在找什么东西。小蚁问:“爸,你找什
么?” 

爸爸说:“我要找一把锋利的刀,把我的右手食指切下来,让我的小蚁攥着。爸爸带着另外九根手
指去工作,还要挣钱养活我的小蚁呢。” 

听了爸爸的话,小蚁哭了:“我没有非要爸爸的手指,爸爸不要找刀了,小蚁去幼儿园,小蚁听话

“这么说,小蚁不要爸爸的右手食指了?”爸
爸擦掉小蚁的眼泪。 


小蚁说:“现在不要,晚上回家再要。” 

\newpage

爸爸笑了:“好好好,爸爸的右手食指给小蚁
留着。” 

小蚁也是一岁时第一次吃面包的。那是吃早餐。她嘴里有一粒东西软软的又实实的,她伸手把它拿出来看,不认识。小蚁很快又把它放进嘴里,她尝出了它的微微的甜味。因为她把它咬破了。她不想让它在自己嘴里消失,她喜欢它丝丝缕缕的甜味,不像糖
那样甜得坏牙把嘴里搞酸,让人不想去回忆它。 

小蚁就在嘴里含着它,怕尖尖的牙齿损坏它,就把它放在舌根下保管起来。整整一个上午,小蚁的嘴都在动,像是吃东西,其实,她在保护那粒东西。

一直都在密切关注小蚁牙齿健康的妈妈警觉起来:“小蚁啊,你一直在吃什么东西?每天只能吃两颗糖。可是,一个上午了,我都看见你在吃啊吃的。张开嘴让妈看看。”妈妈的牙齿从小就没保护好,结果很糟,早早就掉了三颗,修了三颗,到现在,又有
三颗活动了,还要看牙科医生。 

\newpage


小蚁紧紧闭着嘴,不接受妈妈的检查。 

妈妈对小蚁牙齿的管理很严格:“张开嘴!”


小蚁还是闭着嘴巴。抵抗妈妈的威胁。 

妈妈张开自己的嘴,让自己那口沧桑的牙齿暴露在小蚁的眼前,给小蚁来一场活生生的现实教育:“看看妈妈的牙,不保护自己的牙,你的牙就会像妈妈的牙一样,一天到晚什么都不用干了,天天看牙医
,让牙钻钻,让扳子夹,让锤子砸……” 

小蚁屈服了,把嘴张开了,妈妈描绘的治牙场景,简直血淋淋的骇人听闻。妈妈把手掌摊开:“吐
出来!” 

掉在妈妈手掌上的东西已经面目皆非了,让妈妈根本判断不出那是什么。“这是什么?”妈妈认不出来,又怕吃坏了小蚁,妈妈就急了,她把手掌心凑近鼻子,想闻闻它的味道,识别出它是什么食物。可

\newpage
是,什么也闻不出来了。 


小蚁已经把它的味道留给了自己的记忆。 

小蚁的初中同学娇娇坐在对面,把可乐里的冰块嚼得嘎巴嘎巴响:“你吃的是面包里的葡萄干吧?
” 


小蚁说:“现在想起来,是葡萄干吧。” 

“肯定是葡萄干。”娇娇没有多想,再说,她也不愿意对小蚁一岁时的记忆抱有幻想,一岁时的味觉还能储藏在脑子里?她觉得小蚁在这种小事上费了
太多的心思。 

小蚁说:“可有一点我不明白,那么大一个面
包,为什么只有一粒葡萄干呢?” 

娇娇已经倒进嘴里几块冰,听见小蚁的话,就把嘴里的冰吐进杯子里:“这更简单了,面包店里的面点师傅同时做了很多种面包,其中一种面包是夹有很多葡萄干的,还有一种就是你一岁时吃的普通面包
\newpage
,也许你是两岁时三岁时吃的也说不定。面包师傅做普通面包时,手上粘了一粒葡萄干,揉在了普通面包里,这个幸运的面包就被烤熟了,你这个叫小蚁的幸运人的妈妈就买到了它。你们一家三口人吃那顿有特
别意义的早餐时,就被很幸运的你吃到了……” 

现在,幸运的小蚁为自己一岁时的幸运举起了可乐杯子,跟娇娇撞了一下,说道:“为那粒葡萄干
喝一口吧。” 

娇娇把杯子里的冰再次倒进嘴里。制造噪音一样嘎巴嘎巴嚼起来:“你很怪的,不为自己的幸运碰
杯,倒为那粒葡萄干碰杯。” 

小蚁说:“我当然为那粒葡萄干碰杯了,到今
天为止,我再也没有吃过那样味道的葡萄干了。” 

娇娇瞪着漂亮的大眼睛问道:“真的假的?”
她转身问一个服务生:“店里有葡萄干吗?” 

服务生说:“有七八种葡萄干,都是从原产地
\newpage
直接进来的,有新疆南面的北面的,尤其有吐鲁番的
葡萄干……” 

娇娇让服务生把各种葡萄干拼成一小碟端上来。男服务生转身一走,小蚁问娇娇:“你让我过葡萄
干的瘾来了?” 


娇娇说:“我给你治病来了。” 


“我有什么病?” 


“童年葡萄干妄想症。” 


“我今天吃了七八种葡萄干,病就好了?” 


“你小蚁起码不提那粒葡萄干了。” 

小蚁不吭声,觉得娇娇是个没感觉的人。对什么该记的什么不该记的都搞不懂,也太粗了太马大哈了。小蚁想,越是一天到晚张着嘴傻乐的人,越是觉

\newpage
得幸福。我怎么会忘掉那粒葡萄干呢? 

除了娇娇,小蚁结识了霏霏。霏霏是别的班的
,娇娇认识了霏霏,就把她领来跟小蚁认识了。 

过去,小蚁经常在学校的走廊里见到霏霏。霏霏留着短短直立的头发,跟男生说话时,常用拳头捶打男生,用脚去踢男生。开始,小蚁从霏霏的背影上,还判断不出霏霏的性别。小蚁从心里不大喜欢她。
能跟霏霏在一起吃东西看电影,都是因为娇娇。 

有霏霏在场,娇娇有些抢不上话,小蚁又是一个比较少话的人,就只听霏霏一个人说书了。小蚁在霏霏哇拉哇拉不停地说话时,才意识到,霏霏要经常
出现在自己的生活中了,她就有点不知所措。 

霏霏终于说累了,她看着小蚁:“小蚁,你说点什么让我听听。”霏霏明明知道娇娇也在场,可她
就偏偏说“让我听听”,就这么霸道。 

有一次,娇娇和小蚁两个人在一起时,小蚁问

\newpage
娇娇:“你怎么跟霏霏认识的?” 

娇娇不知道小蚁对霏霏是有看法的,就不在意地说起霏霏:“我忘了什么时候和她在一起聊天的,那天有男生有女生,都挤在走廊里,我跟霏霏就挤到了一起。反正,我跟霏霏很谈得来,有共同语言。”

“什么共同语言?”小蚁想知道娇娇对霏霏的
心理感受和她对一个人的判断。 

娇娇想了想说道:“比方说,我烦我爸。霏霏
烦她爸,还烦她妈。” 


“就这个理由?”小蚁问。 


“就这个理由!”娇娇说。 


这是什么理由? 


小蚁在心里问道。 


\newpage

娇娇的表情告诉小蚁,我的理由充足吧? 

没多久。小蚁从霏霏的嘴里,就知道了很多霏霏家里的事。霏霏的妈妈是个有自己生活空间的人。霏霏妈妈不仅抽时间练瑜伽,还学跳肚皮舞,把身材修炼得像个小姑娘。霏霏经常责问自己的妈妈:“你整日迷恋小姑娘才玩的肚皮舞,为什么啊?你想干什
么啊?” 

霏霏妈妈的回答一点都不含糊:“我喜欢啊!

霏霏跟娇娇和小蚁说起妈妈练肚皮舞这件事,像是说一个几千年前臭名昭著的老妖婆复活来到人世
兴风作浪:“也太不正经了。” 

受了三个人当时说话时的气氛影响,小蚁也觉得霏霏妈妈玩得太过了。回到家里,正赶上妈妈催促她认真刷牙,要上下刷左右刷,不要应付,那是自己的牙齿,将来牙齿遭不遭罪,只有自己知道。这种督促已经有十几年了。妈妈不断地重复,小蚁突然火了:“妈,你能不能不管我的闲事啊?我都初中了,你还管我怎么刷牙,我是弱智儿童吗?你天天盯着我刷
\newpage
牙,小考中考大考等着看我的成绩表,还天天为我将来上什么大学提前操心,你累不累啊?你也像人家那样练练瑜伽,跳跳肚皮舞,除了热爱厨房事业,当我的专职警察,这样多好。你把自己吃得胖胖的,听人家说你长相富态,你就真的觉得自己一生满足,世界上就你最幸福了!你朝左边看见我天天长个,朝右边看见我爸爸事业还顺利,你就有了一种成就感!妈,你自己有时间把自己的肚囊练下去,为自己的健康多
想想吧!” 

妈妈猛然听见小蚁一下子说出这么多的话来,
身上一向 

安逸的肉突然间出现了局部痉挛和反叛,变得极不舒服了。小蚁看见妈妈撩起自己的衣襟,把小蚁说的肚囊露出来。看了一会儿,对小蚁说道:“我为什么有肚囊?我想有肚囊吗?我哪里有时间盯着它啊?你让我练瑜伽?还玩什么肚皮舞,那是我玩的吗…
…” 

两人正在半辩论半发泄,爸爸下班回来了,见
\newpage
小蚁和妈妈的情绪都很激动,有点吃惊:“你们两个
吵架了?少见啊!” 

小蚁不想再说什么,转身进了自己的屋子。妈妈大声跟爸爸学刚才发生的事,想让爸爸听得清,也
想让屋子里的小蚁也能听见。 

小蚁有点担心爸爸听到这些事会对自己有看法。没想到,爸爸听完了,对妈妈说:“我觉得小蚁是关心妈妈,她想让妈妈给自己留点空间,把自己的生活质量搞好一些。”爸爸大声对小蚁的屋门喊道:“
是不是,小蚁?” 

小蚁心里有点感动,就悄悄走出来,看了妈妈
一眼:“我就是这个意思。” 

“我说对了吧?”爸爸笑意满脸地看着小蚁。

那时,小蚁就走近了爸爸,用手攥住了爸爸右手的食指。妈妈的气没消,想起一件事来问小蚁:“小蚁,你说谁的妈妈在练瑜伽,玩肚皮舞?什么馆教
\newpage

这种东西?你给我打听打听。” 

爸爸说:“小蚁上学那么忙,哪里有时间打听这种事?我明天给你问问。说到锻炼,我还要表扬小蚁两句,小蚁都发现我们的生活缺少质量了,我跟你生活了这么多年,都没意识到,还是小蚁提醒的。好
了,今天多做两个菜,答谢我们的小蚁。” 

爸爸说这些话时,小蚁一直攥着爸爸的右手食指,一岁时吃到的那粒葡萄干的味觉记忆又爬了出来
,让小蚁的心里又暖又甜。 

娇娇对小蚁说:“能不能让霏霏在你家住两天?住一天也行!”小蚁一愣:“霏霏怎么了?她家的
楼拆迁了?还是漏水不能住了?” 

娇娇说:“霏霏已经在我家住了两天了。她是
被她妈赶出来的。” 


“霏霏惹她妈生气了?” 

\newpage


“霏霏骂她妈妈。” 

“她凭什么骂自己的妈妈啊?”小蚁感到吃惊。尽管霏霏当着小蚁和娇娇的面,说她妈妈练瑜伽跳肚皮舞,只是发发牢骚而已。她张嘴骂自己的妈妈,
还是令小蚁吃惊不小。 

“我不想让霏霏到我家来住。”小蚁说出了自
己的态度。 

娇娇也没说小蚁不够朋友,而是说:“霏霏也是的,为什么骂妈妈啊?她在我们家住了两天,也不洗澡,也不洗脚,说话前不顾后不管的,只顾自己哇啦哇啦说痛快,都过了夜里十二点,还朝我要啤酒喝。我说,我们家没人喝酒,所以冰箱里从来就不准备啤酒。霏霏还大声问,连你爸都不喝酒啊?一个男人不喝酒,这辈子不白混了吗?有什么意思?……结果。霏霏的话让我爸听见了,我爸这晚就没睡着觉,也许是不敢睡觉,他把霏霏当成了定时炸弹。今天早上,我爸跟我说,必须让霏霏走,我不想看见这种女孩子!你说,我怎么跟霏霏说?我找你,也是没办法了
\newpage
。真的让霏霏上你们家住两天,还不知道会发生什么事了。好了,我去劝霏霏回家跟她妈妈道歉,再野的
鸟。也该回窝了,免得让大家都不安生。” 

小蚁见到霏霏时,霏霏对她很冷淡。小蚁想,肯定是娇娇把她拒绝霏霏来家里投宿的话学给霏霏听了。霏霏在这件事上,根本就没认真想过她妈妈什么感受,只想到自己想做什么就做什么,太自我了,也太自私了。小蚁觉得自己失去霏霏这种朋友,谈不上
友谊情感上的损失。 

霏霏本来想对小蚁公开表现出一种冷淡态度,表明自己生气了,引出小蚁心里对霏霏的歉意来。但是,小蚁也同样表现出一种冷淡,对霏霏的行为并不在意,就像一个路人抽烟,把烟吹到了她的鼻子里,她只是挥一挥手,把飘在眼前的烟雾挥到一边去了。

霏霏这种女孩子很怪,她不在乎那些天天跟她泡在一起的男生女生。但她会在意跟她保持一定距离的人。小蚁就是跟她保持着一定距离的朋友。两人相互冷淡了几天,霏霏憋不住了,就跟娇娇说,叫上小
\newpage

蚁,咱们去比乐街吃炸臭豆腐串! 

比乐街的油炸臭豆腐,香遍了一条街。娇娇知道霏霏主动拉近跟小蚁的距离。娇娇很积极地跑去跟小蚁说,霏霏约咱俩去比乐街吃炸臭豆腐!小蚁说:“我没兴趣。”娇娇觉得没帮助霏霏做好外交工作,就哀求小蚁:“去吧,小蚁,霏霏就是为了请你才让我叫你的。”小蚁不吭声了,保持沉默。娇娇缠着小蚁不罢休:“去吧小蚁,我们三个在一起多好啊!我
不能看着好朋友突然就散了……” 

小蚁突然说道:“你问问霏霏,她请我们吃东西的钱是捡来的。还是从她妈妈那里要来的?如果是从她妈妈那里要来的钱,她还敢张嘴骂她妈妈,还是
人吗?” 


这几句话。把娇娇说得呆住了。 

小蚁说完转身就走了。其实,她早想说类似的话。娇娇把小蚁的话跟霏霏学了一遍之后,霏霏也闷

\newpage
闷地呆住了,变得无话可说。 

小蚁爸爸乘坐单位的一辆小车去外地签一份很重要的合同,回来的路上,差一点翻车,被护栏卡住了,车上的几个人都有了点轻伤。小蚁听说后,吓坏了,非要去医院看爸爸。妈妈说,没什么事,只是爸
爸的手不慎被车门夹伤了。 

小蚁不放心,逼着妈妈带她去医院。在医院里,爸爸正躺在床上看书,他的右手被白纱布缠得厚厚
的。像是戴着过冬的大棉手套。 

爸爸看见了小蚁,就笑了一下:“我不想住在医院,一会儿就出院了。小蚁没放学就跑来做什么?
担心爸爸啊?” 

小蚁看着爸爸受伤的右手说:“爸爸没事就好

爸爸用左手抚摸了一下小蚁的脸,安慰她:“
没事没事。” 

那天晚上,妈妈一边做菜,一边嘀嘀咕咕:“
\newpage
万幸啊万幸,万一那车要是翻了呢……”小蚁不想让妈妈说下去了:“妈,行了,别说那些没发生的事好
不好?” 

“好,我不说了。小蚁啊,你爸真的是万幸啊
……” 


“你又说这件事!” 

吃饭时,爸爸用左手使筷子,动作很笨,夹不起食物,老是掉在桌子上。小蚁就用那种少有的担忧眼神看着爸爸的左手。爸爸说:“别用那种眼神看爸爸,爸爸用右手拿筷子用了几十年了,改用左手,便
于开发另一半大脑。这是好事。” 

十几天后,小蚁才知道爸爸失去了一根手指——右手的食指。爸爸不让小蚁看,就背到身后去,小蚁预感到了什么,突然就喊起来:“你给不给我看?
拿出来让我看!” 

妈妈躲到厨房流眼泪,她怕看见这一幕。爸爸
\newpage
就把右手塞到小蚁的手里。小蚁没有摸到自己熟悉的那根食指。内心极度悲伤。小蚁没哭。她流泪的通道
被什么堵塞了。 

小蚁一个人站在街上。北方城市气候无常,一会儿是风,一会儿就是雨。街心花园有一座白色少女雕像,在她俊美的脸上,有雨冲刷成的泪痕,永远挂在了她的脸上,雕像少女就有了摆脱不了的伤心事。


小蚁就觉得自己成了伤心的雕像。 

爸爸坚持用左手吃饭,把右手放在餐桌底下。他不想让吃饭的小蚁看见自己没有食指的右手。小蚁也在有意躲避爸爸的右手,躲避爸爸眼睛的视线。爸爸面前的桌面上,老是掉食物,衣服上也滴上了菜油
星子,像一个刚刚学拿筷子吃饭的幼儿园孩子。 

有一次,小蚁终于跟爸爸说:“爸,你还是用
右手拿筷子吧,别难为自己了。” 

爸爸故作平静地说:“我想用左手……用左手
\newpage

可以开发大脑……” 

“爸,你不用多说了,我知道你心里想的是什
么,你是……怕我想起它……” 

爸爸的眼睛湿了,很歉疚地说:“我没看好它
,偏偏把它弄丢了。” 


小蚁号啕大哭:“你非让我哭出来吗?” 

一直到了落雪的冬天,霏霏和娇娇还有小蚁,三个人才重新坐在一起吃东西聊天了。时间,让她们淡忘了三人之间的隔阂。那时候,霏霏和娇娇都很安静,想听小蚁说些什么。小蚁就说了一岁时第一次吃面包,发现了那粒葡萄干的故事,也是她在五年级第一次上台给大家讲过的那个故事。娇娇听过了,这个故事没有波澜,不曲折,只是一个女孩子的心事而已。娇娇为了让霏霏能够感受到小蚁的内心世界,坐在一边,一点声息都没有,静得像一只不爱多事的小龟

霏霏说,你当时在台上讲的就是这个故事?当
\newpage
时,我还在台下跟别的同学说,台上的那个女生在讲什么?怎么把自己讲哭了?你该早点面对面跟我讲这个故事!你还有故事吧?再讲一个行吗?你肯定有很
多的故事。 

于是,小蚁又讲了一个女孩子跟她的爸爸右手食指的故事,这个故事很长、很细,有些琐碎,最后,那个女孩子失去了爸爸的食指。讲这件事时,小蚁的表情很平淡,却把娇娇和霏霏讲得眼睛发红。娇娇和霏霏擦完眼泪之后,都问小蚁,这个故事在哪里读
到的?拿来也让我们看看! 

小蚁说,有些事,不是在书上读到的。这些事藏在心里,忘不掉的。

\end{document}
