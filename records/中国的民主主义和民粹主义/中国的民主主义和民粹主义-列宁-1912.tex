\documentclass{article}
\usepackage[utf8]{inputenc}
\usepackage{ctex}

\title{中国的民主主义和民粹主义\footnote{Click to View:\url{https://web.archive.org/web/20220430084030/https://www.marxists.org/chinese/lenin-cworks/21/056.htm}}}
\author{列宁}
\date{1912-07-15}

% \setCJKmainfont[BoldFont = Noto Sans CJK SC]{Noto Serif CJK SC}
% \setCJKsansfont{Noto Sans CJK SC}
% \setCJKfamilyfont{zhsong}{Noto Serif CJK SC}
% \setCJKfamilyfont{zhhei}{Noto Sans CJK SC}
% \setlength\parindent{0pt}

\begin{document}
\CJKfamily{zhkai}

\maketitle


\Large

中华民国临时大总统孙中山的一篇文章(我们是从布鲁塞尔的社会主义报纸《人民报》上转载来
的)使我们俄国人非常感兴趣。 

俗话说:旁观者清。孙中山是一位非常有意思的“旁观”者,因为他虽然是个受过欧洲教育的人,但是显然完全不了解俄国。可是这位受过欧洲教育的人,这位代表已经争得了共和制度的、战斗的和胜利的中国民主派的人,在完全不管俄国、不管俄国经验和俄国文献的情况下,提出了一些纯粹俄国的问题。这位先进的中国民主主义者简直象一个俄国人那样发表议论。他同俄国民粹主义者十分相似,以至基本思
想和许多说法都完全相同。 

旁观者清。伟大的中国民主派的纲领(孙中山
\newpage
的文章正是这样的纲领),迫使我们,同时也给了我们一个方便的机会再一次根据新的世界事态来研究亚洲现代资产阶级革命中民主主义和民粹主义的相互关系问题。这是俄国在从1905年开始的俄国革命时期所面临的最重大的问题之一。从中华民国临时大总统的纲领中,特别是把这个纲领同俄国、土耳其、波斯和中国的革命事态的发展对照一下,就可以看出不仅俄国面临这个问题,整个亚洲也面临这个问题。俄国在许多重要方面无疑是一个亚洲国家,而且是一个
最野蛮的、中世纪式的、丢人地落后的亚洲国家。 

俄国资产阶级民主派,从它的早期的单枪匹马的先驱者贵族赫尔岑起到它的群众性的代表——1905年农民协会会员和1906—1912年的头三届杜马中的劳动派代表止,都具有民粹主义色彩。现在我们看到,中国资产阶级民主派也具有完全同样的民粹主义色彩。这里我们试就孙中山的例子来考察一下,目前已经完全卷入全世界资本主义文明潮流的几万万人的深刻革命运动所产生的思想的“社会意义”
究竟在什么地方。 

\newpage

孙中山的纲领的字里行间都充满了战斗的、真诚的民主主义。它充分认识到“种族”革命的不足,丝毫没有忽视政治问题,或者说,丝毫没有轻视政治自由或容许中国专制制度与中国“社会改革”、中国立宪改革等等并存的思想。这是带有建立共和制度要求的完整的民主主义。它直接提出群众生活状况及群众斗争问题,热烈地同情被剥削劳动者,相信他们是
正义的和有力量的。 

我们现在看到的是真正伟大的人民的真正伟大的思想;这样的人民不仅会为自己历来的奴隶地位而痛心,不仅会向往自由和平等,而且会同中国历来的
压迫者作斗争。 

人们自然可以把亚洲这个野蛮的、死气沉沉的中国的共和国临时大总统与欧美各先进文明国家的共和国总统比较一下。那里的共和国总统都是受资产阶级操纵的生意人、是他们的代理人或傀儡,而那里的资产阶级则已经腐朽透顶,从头到脚都沾满了污垢和鲜血——不是国王和皇帝的鲜血,而是为了进步和文明在罢工中被枪杀的工人们的鲜血。那里的总统是资
\newpage
产阶级的代表,那里的资产阶级则早已抛弃了青年时代的一切理想,已经完全变得寡廉鲜耻了,已经完全把自己出卖给百万富翁、亿万富翁和资产阶级化了的
封建主等等了。 

这位亚洲的共和国临时大总统则是充满着崇高精神和英雄气概的革命的民主主义者,这种精神和气概是一个向上发展而不是衰落下去的阶级所固有的;这个阶级不惧怕未来,而是相信未来,奋不顾身地为未来而斗争,这个阶级憎恨过去,善于抛弃过去时代的麻木不仁的和窒息一切生命的腐朽东西,决不为了
维护自己的特权而硬要保存和恢复过去的时代。 

这是怎么一回事呢?这是不是说唯物主义的西方已经腐朽了,只有神秘的、富有宗教色彩的东方才光芒四射呢?不,恰恰相反。这是说,东方已完全走上了西方的道路,今后还会有几万万人为争取西方已经实现的理想而斗争。西方资产阶级已经腐朽了,在它面前已经站着它的掘墓人——无产阶级。在亚洲却还有能够代表真诚的、战斗的、彻底的民主派的资产阶级,他们不愧为法国18世纪末叶的伟大宣传家和
\newpage

伟大活动家的同志。 

这个还能从事历史上进步事业的亚洲资产阶级的主要代表或主要社会支柱是农民。农民旁边还有一个自由派资产阶级,它的活动家如袁世凯之流最善于变节:他们昨天害怕皇帝,匍伏在他面前;后来看到了革命民主派的力量,感觉到革命民主派就要取得胜利时,就背叛了皇帝;明天则可能为了同某个老的或
新的“立宪”皇帝勾结而出卖民主派。 

没有真诚的民主主义的高涨,中国人民就不可能摆脱历来的奴隶地位而求得真正的解放,只有这种高涨才能激发劳动群众,使他们创造奇迹。在孙中山
的纲领的每一句话中都可以看出这种高涨。 

但是在这位中国民粹主义者那里,这种战斗的民主主义思想首先是同社会主义空想、同使中国避免走资本主义道路即防止资本主义的愿望结合在一起的,其次是同宣传和实行激进的土地改革的计划结合在一起的。后面这两种思想政治倾向正是构成具有独特含义的(即不同于民主主义的、超出民主主义的)民
\newpage

粹主义的因素。 

这两种倾向是怎样产生的?它们的意义如何?
 

如果没有群众的革命情绪的蓬勃高涨,中国民主派不可能推翻中国的旧制度,不可能争得共和制度。这种高涨以对劳动群众生活状况的最真挚的同情和对他们的压迫者及剥削者的最强烈憎恨为前提,同时又反过来产生这种同情和憎恨。先进的中国人,所有经历过这种高涨的中国人,从欧美吸收了解放思想,但在欧美,提到日程上的问题已经是摆脱资产阶级而求得解放,即实行社会主义的问题。由此必然产生中国民主派对社会主义的同情,产生他们的主观社会主
义。 

他们在主观上是社会主义者,因为他们反对对群众的压迫和剥削。但是中国这个落后的、农业的、半封建国家的客观条件,在将近5亿人民的生活日程上,只提出了这种压迫和这种剥削的一定的历史独特形式——封建制度。农业生活方式和自然经济占统治
\newpage
地位是封建制度的基础;以这种或那种方式把中国农民束缚在土地上,这是他们受封建剥削的根源;这种剥削的政治代表就是封建主,以皇帝为整个制度首脑
的封建主整体和单个的封建主。 

因此,这位中国民主主义者的主观社会主义思想和纲领,事实上仅仅是“改变不动产的全部法权根
据”的纲领,仅仅是消灭封建剥削的纲领。 

孙中出的民粹主义的实质,他的进步的、战斗的、革命的资产阶级民主主义土地改革纲领以及他的
所谓社会主义理论的实质就在这里。 

从学理上来说,这个理论是小资产阶级反动“社会主义者”的理论。这是因为认为在中国可以“防止”资本主义,认为中国既然落后就比较容易实行“社会革命”等等的看法,都是极其反动的空想。孙中山可以说是以其独特的少女般的天真粉碎了自己反动的民粹主义理论,承认了生活迫使他承认的东西:“中国处在大规模的工业〈即资本主义〉发展的前夜”,中国“商业〈即资本主义〉也将大规模地发展起来
\newpage
”,“再过50年,我们将有许多上海”,即拥有几百万人口的资本家发财和无产阶级贫困的中心城市。

试问,孙中山有没有用自己反动的经济理论来捍卫真正反动的土地纲领呢?这是问题的全部关键所在,是最重要的一点,被掐头去尾和被阉割的自由派
假马克思主义面对这个问题往往不知所措。 

没有,——问题也就在这里。中国社会关系的辩证法就在于:中国的民主主义者真挚地同情欧洲的社会主义,把它改造成为反动的理论,并根据这种“防止”资本主义的反动理论制定纯粹资本主义的、十
足资本主义的土地纲领! 

孙中山在文章的开头谈得如此娓娓动听而又如此含糊其辞的“经济革命”归结起来究竟是什么呢?

就是把地租转交给国家,即通过亨利·乔治式的某种单一税来实行土地国有化。孙中山所提出和鼓
吹的“经济革命”,决没有其他实际的东西。 

\newpage

穷乡僻壤的地价与上海的地价的差别,是地租量上的差别。地价是资本化的地租。使地产“价值的增殖额”成为“人民的财产”,也就是说把地租即土
地所有权交给国家,或者说使土地国有化。 

在资本主义范围内实行这种改革有没有可能呢?不但有可能,而且是最纯粹、最彻底、最完善的资本主义。马克思在《哲学的贫困》中指出了这一点,在《资本论》第3卷中详尽地证明了这一点,在《剩余价值理论》中与洛贝尔图斯论战时非常清楚地发挥了这一点。【注:见《马克思恩格斯全集》第4卷第180—191页,第21卷第904—917页和第26卷第2册第163—176页。——编者注】

土地国有化能够消灭绝对地租,只保留级差地租。按照马克思的学说,土地国有化就是:尽量铲除农业中的中世纪式的垄断和中世纪关系,使土地买卖有最大的自由,使农业最容易适应市场。历史的讽刺在于:民粹派为了“防止”农业中的“资本主义”,竟然实行一种土地纲领,它的彻底实现会使农业中的

\newpage
资本主义得到最迅速发展。 

是什么经济上的必要性使得最先进的资产阶级民主主义土地纲领能够在亚洲一个最落后的农民国家中得到推行呢?是把各种形式各种表现的封建主义摧
毁的必要性。 

中国愈落在欧洲和日本的后面,就愈有四分五裂和民族解体的危险。只有革命人民群众的英雄主义才能“振兴”中国,才能在政治方面建立中华民国,在土地方面实行国有化以保证资本主义最迅速的发展
。 

能不能做到这一点,能做到什么程度,——这是另一个问题。不同的国家通过自己的资产阶级革命所实现的政治方面和土地方面的民主主义,在程度上是不同的,而且情况是错综复杂的。这要看国际形势和中国各种社会力量的对比而定。看来皇帝大概会把封建主、官僚、僧侣联合起来,准备复辟。刚刚从自由主义君主派变成自由主义共和派(能长久吗?)的资产阶级代表袁世凯,将在君主制和革命之间实行随风倒的政策。以孙中山为代表的革命的资产阶级民主
\newpage
派,正在发挥农民群众在政治改革和土地改革方面的高度主动性、坚定性和果断精神,从中正确地寻找“
振兴”中国的道路。 

最后,由于在中国将出现更多的上海,中国无产阶级也将日益成长起来。它一定会建立这样或那样的中国社会民主工党,而这个党在批判孙中山的小资产阶级空想和反动观点时,大概会细心地挑选出他的政治纲领和土地纲领中的革命民主主义内核,并加以护和发展。

\end{document}
