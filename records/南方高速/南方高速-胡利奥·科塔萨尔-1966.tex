\documentclass{article}
\usepackage[utf8]{inputenc}
\usepackage{ctex}

\title{南方高速\footnote{Click to View:\url{https://web.archive.org/web/20230121032908/https://www.99csw.com/book/10596/383025.htm}}}
\author{胡利奥·科塔萨尔}
\date{1966}

% \setCJKmainfont[BoldFont = Noto Sans CJK SC]{Noto Serif CJK SC}
% \setCJKsansfont{Noto Sans CJK SC}
% \setCJKfamilyfont{zhsong}{Noto Serif CJK SC}
% \setCJKfamilyfont{zhhei}{Noto Sans CJK SC}
% \setlength\parindent{0pt}

\begin{document}
\CJKfamily{zhkai}

\maketitle


\Large

起初,开雷诺王妃的女孩还坚持要打开计时器,可是开标致404的工程师觉得那都是无所谓的事。人人都可以看自己的手表,可是,无论是右手腕上的时间还是收音机里的“哔哔”声,此刻都好像已经与时间无关,时间的概念只属于那些还没有愚蠢到选择星期天下午从南方高速返回巴黎的人,刚过了枫丹白露,他们就不得不走走停停,隔离带两侧各排起六道长龙(众所周知,一到星期天,整条高速都留给了返回首都的车辆),启动汽车,开上三米,停下来,和右手边那辆双马力上的两位修女聊聊天,和左手边王妃上的姑娘搭搭话,再从后视镜里看一会儿后面开大众凯路威的脸色苍白的男人,而王妃后面是一辆标致203,上面坐了一对夫妻,正在逗自己的小女儿玩,说说笑笑,不时吃点儿奶酪什么的,其乐融融,出乎意料地教人心生羡慕,标致404前面是一辆
\newpage
福特西姆卡,坐在那上面的两个小伙子吵吵嚷嚷,令人不耐,有时车停得久了些,工程师还会下车四处转转,但不能离车太远(因为说不准什么时候前面的车就重新启动,必须三步并两步跑回来,否则后面喇叭声叫骂声就会响成一片),就这样他走到了一辆福特陶努斯附近(后面就是王妃,那个姑娘在不停地看表),车上是两个男人,带着一个金黄头发的男孩,此情此景中,男孩最大的乐趣就是让一辆玩具小汽车在陶努斯的座椅和靠背上纵横驰骋,他和那两个男人抱怨一番,调侃几句,看上去前面的车都没有要动弹的意思,他壮起胆子多走了几步,看见一辆雪铁龙ID上坐着一对老夫妇,不禁心生怜悯,两人好似漂浮在一口巨大的紫色浴缸里,老头双臂倚在方向盘上,一副逆来顺受的疲惫神情,老太太啃着一只苹果,不像
在享受吃苹果的滋味,倒像是在完成什么任务。 

就这样反复折腾了四次,工程师决心不再下车,而是等警察来疏通拥堵。八月的高温齐着一只只轮胎的高度弥散开来,车一动不动,人们越发萎靡不振。到处都是汽油味儿,西姆卡上那两个小伙子的尖叫声肆无忌惮,车窗玻璃和镀铬部件反射出刺眼的光芒
\newpage
,最糟的是这种自相矛盾的感觉,初衷是载人飞驰的机器,却把人困在了这机器丛林中。从中间的隔离带数过来,工程师的标致404在右边第二车道,算起来,他的右手边还有四列车,左手边则有七列,而实际上他能看见的只有四周的八辆车以及车上的人,一切细节他都已经记得清清楚楚,看得厌倦了。他和所有的人都聊过,只除了西姆卡上的那两个小伙子,他实在看他们不顺眼。这些两两交谈涉及了这次堵车的各种细枝末节,总的印象是,一直到科贝尔——埃松内都会这样走走停停,不过从科贝尔到茹维希那一段,一旦直升机和摩托骑警把拥堵的路段疏通,车就可以开快一点。大家都确信那段路上一定发生了什么严重事故,否则没法解释如此可怕的堵车。就这样,议论议论政府,骂骂炎热的天气,对税收发几句牢骚,再抱怨抱怨交通部门,话题一个接着一个,开上三米,又停在了一起,再开上五米,不时会有人冒出一句
精辟的格言,或是一句含蓄的诅咒。 

双马力上的两位修女指望在八点之前赶到米利——拉福雷,因为她们带了一篮子蔬菜给那里的厨娘。标致203上的那对夫妻最操心的是别误了晚上九
\newpage
点半的比赛直播;王妃上的姑娘对工程师说过,晚一点到巴黎她倒不在乎,她抱怨的是这荒唐的现实,把好几千人搞得像骆驼队一样慢腾腾。几个小时里(这会儿该有五点钟了,可热浪还是把他们压得喘不过气来),按照工程师的估算,他们总共才前进了五十来米,陶努斯上的其中一个男人牵着孩子走过来聊天,孩子手里还拿着他的玩具小汽车,男人不无讽刺地指了指路边一棵孤零零的梧桐树,王妃上的姑娘记起来了,那棵梧桐树(也许是棵板栗树)一直和她的车排在同一条线上,她现在连手表都懒得去看,计算时间
已经毫无意义。 

太阳仿佛不肯落下,路面和车身上晃动的阳光让人头晕目眩。或者戴上墨镜,或者头上顶着洒了古龙水的手帕,大家想出各种办法躲避刺目的反光,躲避每行进一步都会从排气管里冒出来的尾气,这些凑合而成的举措渐趋完备,成为众人交流和评估的主题。工程师还是下了车,想活动活动腿脚,修女的双马力前面是一辆阿利亚纳,车里坐着一对乡下人模样的夫妻,他和他们聊了几句。双马力的后面跟了辆大众,坐着一名军人和一个姑娘,看上去像是度完蜜月归
\newpage
来。第三车道往外他不想去看了,怕离自己的标致404太远,出什么问题;他看见的车各式各样,有奔驰、ID、4R、蓝旗亚、斯柯达、莫里斯微型车,简直是汽车博览会。往左边看去,对面车道上有雷诺、福特安格利亚、标致、保时捷和沃尔沃,延伸到无尽的远方;实在了无趣味,最后,在和陶努斯上的两个男人闲聊了几句、想和凯路威上的独身男人交换一番感想却没能谈成之后,他别无选择,只有回到自己的标致404,和王妃上的姑娘重新拾起老话题,谈
谈时间呀,距离呀,电影什么的。 

偶尔会走来一个陌生人,从对面车道或是从右边外侧的车道沿着汽车夹缝穿行而来,带来某个真假难辨的消息,这些消息会从一辆车传到另一辆车,顺着滚烫的公路散布开来。陌生人看到自己带来的消息得以传播,听到一扇扇车门打开关上砰砰作响、人们争先恐后各抒己见,心中十分得意,可是片刻之后传来一声喇叭响,或是引擎启动的声音,陌生人拔腿便跑,在车辆之间曲折奔行,为的是重新钻进他自己的汽车,以免暴露在别人理所应当的愤怒中。整个下午,人们都议论纷纷,先是说有一辆雷诺弗洛里德在科
\newpage
贝尔附近撞上了一辆双马力,三人死亡,一名男童受伤,又说有一辆雷诺货车把一辆满载英国游客的奥斯丁撞得稀烂,接着又有一辆菲亚特1500连环撞上了这辆货车,还有人说是一辆奥利机场的大巴翻了车,上面坐满了从哥本哈根乘飞机来的游客。在科贝尔附近甚至巴黎近郊一定是发生了什么严重的事故,否则交通绝不至于瘫痪到如此地步,但工程师仍然可以断定,所有或者几乎所有消息都是谣言。阿利亚纳上的乡下人在蒙特罗附近有家农庄,他们对这个地区很熟悉,据他们说,前些日子,也是个星期天,这里的交通堵塞持续了五个小时,可那点时间和现在比起来真的算不了什么,此刻,太阳正一点一点向着公路左侧落下去,给每一辆车都泼洒上一层金黄色的浆汁,金属像在燃烧,令人目眩,身后的树木好像伫立不动,永远不会消失,前方远处若隐若现的树影却永远无法接近,简直感觉不到车流在挪动,哪怕只挪一点点,哪怕是不断地发动、停车、急刹车,哪怕永远不能摆脱一挡,也不能摆脱令人恼火的失望,一次又一次地从一挡变成空挡,不断地踩刹车,拉手刹,停车,
就这样,一次又一次,仿佛没有尽头。 

\newpage

有一回,在一段漫长得没有尽头的静止中,工程师闲极无聊,决定去隔离带左边一探究竟。在王妃左边,他看见了一辆奥迪DKW,往前是又一辆双马力,还有一辆菲亚特600,他在一辆迪索托旁停了下来,同一个来自华盛顿的忧心忡忡的游客交谈了几句,那位几乎一句法语也听不懂,可是八点钟他必须赶到歌剧院去,你听得懂我的话吗,我老婆会担心的,真该死,后来从DKW下来一个像是旅行推销员的人,说刚才有人带来一个消息,一架Piper Cub坠毁在公路中央,死了好几个人。美国人对Piper Cub的事儿毫无兴趣,工程师也顾不上这消息了,因为这时他听见喇叭声响成一片,得赶紧回到标致404上,顺便把这些新闻传递给陶努斯上的两个男人和标致203上的那对夫妻。他把更详细的解释都留给了王妃上的姑娘,这时车辆都缓缓前行了几米(王妃先是稍稍落后标致404,过了一会儿又稍稍超到了前面,可实际上十二道车龙正像个整体似的向前移动,仿佛公路尽头有个隐身的宪兵在发布命令,让大家齐头并进,任何人都不得领先)。小姐,Piper Cub是一种观光用的小型飞机。哦。在一个星期天的下午,坠毁在公路正中央,这真是糟
\newpage
糕透顶的事。这都是些什么事儿呀。要是这些倒霉的车里不是这么热,要是右手边那些树都能最终退到身后,要是里程表上那最末位的数字能最终掉进那个小黑窟窿里,而不是像现在这样没完没了地悬着,那该
多好呀。 

有那么一会儿(天色渐渐暗了下来,由车顶组成的地平线被染上一层淡淡的丁香色),一只大大的白蝴蝶歇在了王妃的前挡风玻璃上,在它短暂停留的美妙一刻,姑娘和工程师都对它的一双翅膀赞叹不已;他们满怀惆怅看着它一点点飞远,飞过陶努斯,飞过那对老夫妇的紫色ID,飞向从标致404上已经看不见的菲亚特600,又飞到西姆卡上方,从那车里伸出一只手想捉住它,但没能成功,飞到那乡下人夫妻的阿利亚纳,那对夫妻好像在吃什么东西,它友好地扇了扇翅膀,最后在右边消失不见了。天黑下来的时候,车流第一次前进了一段不错的距离,几乎有四十来米吧;工程师不经意地看了看里程表,6已经下去了一半,7露了一点头。几乎所有的人都打开了收音机,西姆卡上那两位把音量开到了最大,嘴里还唱着摇摆舞曲,身体摇摆着,连车子都跟着摇个不停
\newpage
;两位修女拨动着念珠,陶努斯上的那个孩子已经睡着了,脸靠在车窗玻璃上,手里还拿着玩具小汽车。又过了一会儿(天已经完全黑下来了),过来了几个陌生人,他们带来了新消息,和此前已经被人遗忘的消息一样自相矛盾。不是一架小型飞机,是一位将军的女儿开的滑翔机。雷诺货车把奥斯丁压扁了这事儿不假,可根本不是在茹维希,而是在离巴黎很近的地方;有一个陌生人还告诉标致203上的夫妻说,伊格尼那边高速公路路面坍塌,有五辆车前轮陷了进去,都翻了车。这种自然灾害的说法也传到了工程师耳朵里,他耸耸肩,不做评论。又过了一会儿,他正回忆着天黑下来的这段时间人们总算可以舒舒服服地喘口气了,突然又想起来,自己曾经把胳膊从车窗伸过去敲了敲王妃,把那姑娘叫醒,她已经趴在方向盘上睡着了,毫不在意车流能不能再往前走。大概是在夜半时分,修女中的一位可能是觉得他饿了吧,怯生生地递给他一份火腿三明治。工程师出于礼貌接受了(其实此刻他很恶心,想吐),征得同意之后,他把三明治分了一半给王妃上的姑娘,姑娘欣然接受,三口两口吃完,她左手边DKW上的旅行推销员递过来一块巧克力,她也吃光了。又有好几个小时没能前进一
\newpage
步了,车里太热,很多人都下了车;人们开始感到口渴难耐,瓶子里的柠檬水或是可口可乐都喝得见了底,就连车上带的葡萄酒都喝光了。第一个渴得受不了的是标致203上的那个小女孩,于是军人和工程师都下了车,帮小女孩的父亲一起去找水。西姆卡的收音机放得正欢,工程师看见它的前方是一辆波利欧,开车的是一位上了点岁数的妇人,眼神惶恐不安。没有,我没有水,但我可以给小女孩几粒糖果。ID上的老两口商量了一番,老太太把手伸进一只袋子里,掏出一小听果汁。工程师谢过老两口,问他们肚子饿了没有,自己能不能帮点儿什么忙,老头摇了摇头,老太太没说什么,但看上去是给了个肯定的答复。接下来的时间里,王妃上的姑娘和工程师一起顺着左面几列车寻找了一番,他们也没敢走太远,回来的时候,给ID的老太太带来几块饼干,刚好赶在一片疾风
暴雨般的喇叭声中跑回自己的车上。 

除了这有限的几次出行外,人们能做的少之又少,时间几乎一动也不动,显得分外漫长;有那么一阵,工程师真想把这一天从自己的记事簿上删去,他强忍住没有哈哈大笑起来,可过了一会儿,当那两位
\newpage
修女、陶努斯上的两个男人以及王妃上的姑娘把时间算成了一笔糊涂账的时候,他想还真不如当初就打开计时器。地方广播电台都停止了播音,唯有DKW上的那位旅行推销员有一台短波收音机,还在一个劲地播送股票消息。快到凌晨三点的时候,大家都心照不宣地达成了某种默契,决定休息休息,就这样,直到天亮,车流一动也没动过。西姆卡上的小伙子卸下两张充气床垫,在车旁躺了下来;工程师把404前排座椅放倒,请两位修女来躺躺,被她们拒绝了;刚躺下没一会儿,工程师想起了王妃上的姑娘(她安静地趴在方向盘上),便若无其事地向她提议换个车,天亮再换回来;她拒绝了,说她不管坐着躺着都能睡得很香。有那么一阵,能听见陶努斯上的小孩在哭,他睡在汽车的后排座椅上,一定热得不行。修女们还在祈祷,工程师已经一头倒在自己的卧铺上,慢慢睡着了,然而他睡得一点儿也不踏实,最后浑身大汗、心烦意乱地醒来,一时间竟弄不清自己身处何方;他舒展了一下身体,发现车外模模糊糊有些动静,一团黑影朝公路边移动着;他猜到了原因,接着也悄无声息地下车,去到路边方便了一下;路边没有树,连围栏都没有,只有黑漆漆的田野,天上一颗星星也看不见
\newpage
,就像有一堵看不见的墙围困着泛白的路面,路面上的车像一条停滞不动的河流。他差一点撞上了阿利亚纳上的乡下人,那人嘴里嘟囔了一句什么;燥热的公路上本来汽油味就够难闻,现在再加上人体排出来的骚味,工程师赶紧回到了自己的车上。王妃上的姑娘趴在方向盘上睡着了,一绺头发搭在眼睛上;回到404之前,工程师在黑暗中愉快地端详了一番姑娘的侧影,猜想着她弯弯的双唇是如何轻柔地呼吸。在另一边,DKW上的男人静静地抽着烟,也在注视着这
个姑娘。 

上午,车还是没能前进多远,可已经足以使人满怀希望,想着到了下午通往巴黎的道路就会疏通。九点钟,有个陌生人过来,带来了好消息:前方塌陷的路面已经垫好了,交通很快就能恢复正常。西姆卡上的小伙子打开收音机,其中一个还爬上了车顶,又叫又唱。工程师告诉自己,这消息并不比昨天的那些更靠谱,那陌生人只不过是想趁这群人兴高采烈之际,从阿利亚纳上的夫妻那里讨到一只橘子罢了。后来又过来一个陌生人,想故伎重施,可谁都不肯给他东西了。天越来越热,大家都情愿待在车里等更确切的
\newpage
好消息。中午时分,标致203上的小女孩又哭了起来,王妃上的姑娘去和她玩了一会儿,还和那夫妻俩交上了朋友。203上的那对夫妻运气不佳:他们右边就是那个开凯路威一声不吭的男人,对周围发生的事情漠不关心,左手边又得忍受开弗洛里德那家伙的满腹牢骚,好像这堵车全是冲着他一个人来的。那小女孩又说口渴的时候,工程师突然想到可以去同阿利亚纳上的乡下人谈谈,他们车上肯定有不少吃食。他没料到那两位十分和气,通情达理,说在这样的情况下人们就该互相帮助,他们还有个想法,要是有人出面把这一群人的事儿管起来(说这话时那女人用手画了一个圆圈,把他们周围的十几辆车都包括了进来),那他们坚持到巴黎是没什么问题的。工程师生性不爱出头露面、充当组织者的角色,便叫来了陶努斯上的两个男人,同他们还有阿利亚纳上的夫妻开了个小会。接下来,他们分别征求了这一小群体的意见。大众上的军人立刻表示同意,标致203上的夫妻把自己所剩不多的给养贡献了出来(王妃上的姑娘已经给那小女孩弄到了一杯石榴水,现在那小女孩正在嬉笑玩耍)。陶努斯上的其中一个男人去向西姆卡上的小伙子征求意见,他们倒是同意了,但摆出一副嘲弄的
\newpage
神情;凯路威上脸色苍白的男人耸了耸肩,说他无所谓,你们爱怎么办怎么办。ID上的那对老夫妇和波利欧上的妇人明显很高兴,好像这样一来他们就有了依靠。弗洛里德和DKW上的人都没有发表意见。迪索托上的美国人带着惊讶的神情看了看他们,又说了句什么“上帝的意志”之类的话。工程师没费多大劲就提议让陶努斯上的一个男人负责协调各种事务,他基于直觉对这人有一种信任感。吃的东西眼下谁都不缺,问题是得弄到水;他们的头儿(西姆卡上的两个小伙子为了好玩儿,干脆就把他叫作陶努斯了)请工程师、军人还有两个小伙子当中的一个到周围去转转,看能不能用食物换点儿喝的东西。陶努斯显然深谙领导之道,他算了一下,照最不乐观的估计,需要准备最多足够一天半的吃喝。修女们的双马力和乡下人的阿利亚纳上有充足的食物来应付这一段时间,只要出去侦察的那几位能找到水,就万事大吉了。可是只有那个军人带回来满满一壶水,水的主人要求换取够两个人吃的食物。工程师没找着能提供水的人,但出去转了这一趟,他得知除了他们这个群体之外,还有人也在组织起来解决类似的问题;有一回,一辆阿尔法·罗密欧的车主拒绝和他谈水的问题,说要谈得到
\newpage
这列车往后第五辆找他们这个小组的头儿。又过了一会儿,西姆卡上的小伙子回来了,他也没弄到水,可陶努斯估计给两个孩子、ID上的老太太以及其余几个女人的水已经足够了。工程师正在给王妃上的姑娘讲自己在周围转了一圈碰到的事情(这时已经是下午一点钟了,阳光把大家都困在车里),姑娘突然做手势打断了他的话,又朝西姆卡指了指。工程师三下两下便跳到了西姆卡跟前,一把抓住其中一个小伙子的胳膊,这家伙正舒舒服服地靠在座位上大口大口地从水壶里喝水,原来他回来的时候把水壶藏在了夹克衫底下。看见小伙子恼羞成怒的神情,工程师抓他胳臂的手更用劲了;另一个小伙子下了车朝工程师扑来,工程师退了两步,几乎是带着怜悯的神情,等他扑过来。这时军人从天而降,修女们的叫声惊动了陶努斯和他的伙伴;陶努斯听了事情的经过,走到拿水壶的小伙子身边,给了这家伙两记耳光。那小伙子又喊又闹,哭了起来,另一个嘴里嘟嘟囔囔,再也不敢掺和进来。工程师夺过水壶,递给了陶努斯。这时又响起了喇叭声,每个人都回到自己车上,可这回也没多大
进展,车流向前走了还不到五米。 

\newpage

到了午睡时间,阳光比起前一天来更加炽热,一位修女除下头巾,她的同伴替她在太阳穴那儿涂了些古龙水。女人们纷纷担当起助人为乐的角色,一辆车一辆车地照顾孩子,好让男人们腾出手来;没有人怨天尤人,当然这种一团和气的氛围也很勉强,不过是建立在千篇一律的文字游戏和心存疑虑的好言好语之上。对工程师和王妃上的姑娘来说,最难受的事情莫过于浑身臭汗、脏兮兮的;但他们每次到乡下夫妻那里和他们商量事情或是去告诉他们什么最新消息的时候,那两个人都对自己胳肢窝下散发出来的臭味浑不在意,这令他们钦佩不已。黄昏时分,工程师偶尔从后视镜看过去,只见开凯路威的男人一如既往地脸色苍白,没有丝毫表情,一副事不关己的样子,和开弗洛里德的胖子一个样。他觉得这人的脸比之前更瘦更尖,暗想这人是不是病了。可后来当他过去同军人还有他的妻子聊天的时候,他就近看了看,才发现这人不是生病,而是另外一回事,一定要有个说辞的话,也许就叫作不合群。大众上的军人后来告诉工程师,自己的妻子有点害怕这个一声不吭的男人,这人从不离开他的方向盘,就连睡觉都好像睁着眼睛。各种猜测都有,人们实在闲到无聊,甚至编出了一则怪谈
\newpage
。陶努斯和203上的两个孩子已经成了好朋友,一会儿吵架,一会儿又和好;小孩儿的家长也互相走动,而王妃上的姑娘过一会儿便会去看看ID上的老太太还有波利欧上那位妇人怎么样了。入夜之际突然刮起一阵大风,从西方涌上一片乌云,遮住了太阳,大家都开开心心的,心想这下可以凉快了。雨点落下来的时候,正好车流向前走了一百来米;远方划过一道闪电,天越发闷热了。空气里充满了电荷,陶努斯出于本能直到天黑也没再让大家做些什么,仿佛担心在如此的疲劳和炎热下会出意外,对他这种本能,工程师嘴上没说什么,心里却十分赞赏。八点钟,女人们负责分配给养;大家早先已经决定把那对乡下夫妻的阿利亚纳作为总储备所,修女们那辆双马力作为补充库存。陶努斯亲自去和另外四五个邻近团体的头儿商谈,谈妥之后,在军人和203男人帮助下,他们给别的小组送去一些食品,换回了更多的水,甚至还有一点葡萄酒。人们还做出决定,让西姆卡上的两个小伙子把他们的充气床垫让给ID上的老太太和波利欧上的妇人;王妃上的姑娘给她们拿去了两条苏格兰毛毯,工程师把自己的车贡献出来给需要的人使用,他把它戏称为卧铺车厢。没想到王妃上的姑娘欣然接受
\newpage
了他的盛情,这天夜里,她和一位修女共享了标致404的卧铺,另一位修女去到标致203车上,和小女孩以及女孩的妈妈睡在一起,那丈夫则裹了条毯子在公路上过夜。工程师一点儿也不困,和陶努斯还有他的朋友一起玩掷色子游戏,阿利亚纳上的乡下男人一度也参加进来,他们一边谈论着政治,一边喝上两口白兰地,这酒还是这天早上乡下男人上交给陶努斯的。这一夜过得不坏。天变凉爽了,云朵之间还有几
颗星星在闪烁。 

天快亮的时候他们都困了,东方泛白之际,人最需要有个遮风蔽雨的地方。陶努斯在后座上孩子旁边睡了下来,他的同伴和工程师在前排座位上休息了片刻。在变幻的梦境之间,工程师仿佛听见远处传来了叫喊声,还看见了模模糊糊的亮光。另一个小组的头儿过来对他们说,往前大概三十来辆车的地方,有辆埃斯塔菲特着了火,起因是有人想悄悄煮菜吃。陶努斯对这件事说了两句玩笑话,就逐一去查看大家这一夜都过得怎么样,该吩咐的话一句都没落下。这天早上,车流很早就开始挪动,大家都急急忙忙把床垫和毯子收起来,但因为各处情况都差不多,没有人着
\newpage
急上火,也没有人乱按喇叭。到中午的时候,车流走了差不多五十米,公路右侧隐约看见一片树林的影子。大家都对那些此时能有好运气享受路边阴凉的人心生羡慕;兴许那儿还会有条小溪,或是有个出饮用水的龙头什么的。王妃上的姑娘闭上双眼,想象着自己在冲澡,水流顺着脖颈后背流下来,一直流到腿上;工程师悄悄看了她一眼,看见两滴泪珠顺着她的脸颊
流下来。 

陶努斯刚才已经往前走到了ID那里,这时回来找几个年轻的女士帮忙去照顾一下那位老太太,她有点儿不舒服。后面第三组的头儿管辖的人群里有一位医生,军人立即跑步过去找这位医生。工程师一直怀着略带嘲讽的善意注视着西姆卡上的两个小伙子努力改变,他尽量让自己原谅他们的不懂事,知道该给他们一次改过的机会。两个小伙子用一顶帐篷的篷布把标致404蒙了起来,卧铺车变成救护车,这么一来,老太太就可以在暗一些的环境下休息。她丈夫躺在她身边,握着她的手,人们让老两口单独和医生待在里面。之后两位修女也过来照顾老太太,她感觉好了很多,工程师尽力打发掉下午的时间,走访其他的
\newpage
车辆,太阳实在热辣的时候,他就在陶努斯的车里休息一会儿;总共只有三次他不得不跑到自己的车那里,老两口好像都睡着了,他随着车流把车往前开上一点儿,直到再一次停下来。就这样一直到夜幕降临,
他们也没能前进到那片小树林。 

凌晨两点左右,气温降了下来,有毯子的人都暗自庆幸可以把自己裹住。看上去这车流天亮以前是动不了了(从夜风里就可以感觉得到,它正从天边一动不动的汽车丛林那里刮来),工程师和陶努斯,还有阿利亚纳上的男人和军人坐了下来,一边抽烟一边聊天。陶努斯原先做的估计现在看来与现实不太相符,他很坦率地承认了这一点;天亮以后必须得做点什么,多弄一些吃的喝的。于是军人便去找邻近几个小组的头儿商量,那几位也没睡觉,他们压低嗓门把这些问题讨论了一番,不想把女人们吵醒。几个头儿又把范围扩大到八十或是一百辆车的半径,和远处的一些小组负责人商量了一番,最后确信各组的情况都大同小异。乡下男人对这一带比较熟悉,他建议等天一亮就派出两三个男人到附近的农庄里去买食物,在此期间由陶努斯指定司机来开那些没了主人的车。这主
\newpage
意不错,在场的人没费多少事就把钱凑够了;他们决定由乡下男人、军人和陶努斯的同伴一起去,带上所有能用的提包、网兜和水壶。其他小组的头儿们也纷纷返回各自的单元,去组织类似的出征。天亮以后,他们把实情告知各位女士,只要车流能继续向前行进,该做的他们都做了。王妃上的姑娘告诉工程师说,老太太已经好一些了,坚持要回到他们那辆ID上去;八点钟,医生过来了,他觉得老夫妇俩回自己车上没什么不合适的。尽管如此,陶努斯还是决定把标致404专设为救护车;两个小伙子为了好玩儿,自制了一面红十字小旗,挂在404的天线上。已经有好一会儿了,人们都尽量不从自己的车上下来;气温继续下降。到了中午,天上下起了雨,远远地,还能看见闪电。乡下男人的老婆赶紧拿出一只塑料广口瓶和一个漏斗接水,西姆卡上的小伙子看得很开心。工程师把这一切都看在眼里,他俯向方向盘,那儿摊开着一本书,他并没认真去看,而是暗自思索,为什么出去的那几个人还迟迟没有归来。过了片刻,陶努斯悄悄叫他到自己车上去一趟,两个人都上了车之后,陶努斯告诉他,事没办成。陶努斯的同伴提供了更多的细节:要么是农庄废弃了,要么就是农户拒绝卖给他
\newpage
们任何东西,说是有规定不能把东西卖给个人,而且怀疑他们是稽查人员,故意利用这种情况来引他们上钩。尽管如此,他们还是弄回来一点水和一点吃食,也说不定是军人顺手牵羊的成果,他在一旁笑眯眯的,根本不参加这些细节的讨论。当然,不会再堵很长时间了,可是他们手头的这些食物并不适合两个孩子和那位老太太。下午四点半左右,医生又来看了一趟病人,他露出懊恼困倦的神情对陶努斯说,在他那个小组里,其实在周围所有的小组里,出征都不顺利。收音机里早就在说要采取紧急措施来疏通公路,但只有天快黑的时候来过一架直升飞机,转了一小会儿就走了,除此之外再也没见其他的措施。不管怎么说,天越来越凉快了,大家似乎都在等候夜晚的到来,好用毯子把自己裹起来,在睡梦中把等候的时间缩短几个小时。工程师坐在自己的车里,听着王妃上的姑娘和DKW上的旅行推销员聊天,那推销员给她讲故事,姑娘勉勉强强地报以笑容。突然,他们吃惊地看见了波利欧上那个从不下车的妇人,于是工程师下了车,问她有没有什么需要帮忙的,可那位妇人只是想打听一下最新的消息,她和修女们聊了会儿天。天黑了,大家被一种莫名的厌倦情绪所笼罩;与其去听那些
\newpage
永远自相矛盾的假消息,不如屈服于倦意。陶努斯的同伴悄悄走了过来,把工程师、军人和203上的男人叫了过去。陶努斯告诉他们说,弗洛里德上的司机刚刚逃走了;西姆卡上的一个小伙子看见那车上没了人,他也实在闲极无聊,便去找这车的主人。谁都不认识弗洛里德上的那个胖子,只知道第一天他嚷嚷得最欢,后来却像那个开凯路威的人一样沉默不语。到了早上五点钟,那位弗洛里德(这是西姆卡上的小伙子对他的戏称)确实是逃走了,随身带走了一只手提箱,把另一只装满衬衣和内衣的箱子扔在了车上,于是陶努斯决定,让西姆卡上的一个小伙子去负责这辆被遗弃的车,免得妨碍整个车流的行进。人们都隐隐觉得,这人在漆黑的夜间逃走,情况有点不太妙。旷野里,这个弗洛里德能跑到哪儿去呢?除此之外,这个夜晚似乎还有别人做出了重大决定。工程师躺在404的卧铺上,觉得好像听见了一声呻吟。起初他以为是军人和他妻子在做点什么,在这漆黑的夜晚,又处在这样的情况下,他们做点什么完全可以理解,后来他又仔细一想,便把盖在车后窗上的帆布掀了起来;在暗淡的星光下,一米五开外,他看见了凯路威那永恒不变的前挡风玻璃,玻璃另一面,那人变了形的
\newpage
脸仿佛贴在了上面,歪倒在一旁。他不想弄醒两位修女,于是悄悄从左边下了车,走近凯路威。然后他叫来了陶努斯,这时军人飞奔去找医生。很显然,这人服下什么毒药,自杀了;记事本上用铅笔写下了几行字,是给一个叫伊薇特的女人留下的一封信,这女人在维尔松把他甩了。幸好在车里睡觉已经成了定例(夜里太冷,谁也不会想待在车外面),也很少有人会去操心有没有人穿过车林,走到公路边去方便方便。陶努斯召集了一次紧急会议,医生对他的提议深表赞同。把尸体就近放在公路边,会吓着后面过来的人,至少也会让他们不太舒服;把尸体扔远一点吧,扔到田野里去,又怕会遭到当地居民的强烈排斥——前一天晚上,另一组有个年轻人去找吃的,就被那帮人连骂带打地收拾了一顿。阿利亚纳上的乡下人和DKW上的旅行推销员倒是有工具能够封死凯路威的后备厢。这两位开始动手干活的时候,王妃上的姑娘也过来了,她浑身颤抖,紧紧拉住工程师的胳膊。他压低嗓音把刚才发生的事情解释给她听,等她平静下来一点,便把她送回她自己的车上。陶努斯和他的帮手们把尸体塞进后备厢里,军人用手电筒照着,旅行推销员用透明胶带和胶水把后备厢紧紧封住。因为标致20
\newpage
3上的女人也会开车,陶努斯决定让她的丈夫来开凯路威,反正车就在203的右边;就这样,天亮以后,203上的小女孩发现爸爸又有了一辆车,就不停地从一辆车爬上另一辆车,而且把她的一些玩具也放
在了凯路威上。 

第一次,大白天人们也觉得冷,谁也不想把外套脱掉。王妃上的姑娘和两位修女清点了本组的几件大衣。碰巧又在几辆车里或是在提箱里找到几件套头毛衣,还有几条毯子、风衣和薄外套之类。大家制定了一个优先使用名单,外套也分发了下去。现在又面临缺水的问题,陶努斯派出去三个人,包括工程师,试图和当地百姓周旋一番。不知道为什么,外面的人对他们反感透顶;他们只要一离开公路边,便有石块从四面八方像雨点般投向他们。深夜里,不知是什么人把一柄大镰刀砸向DKW的车顶,落到了王妃旁。旅行推销员被吓得脸色发白,没敢从车里出来,但是,开迪索托的那个美国人(他并没有加入陶努斯这个小组,可大家对他的好心态还有爽朗的笑声都十分欣赏)迅速跑过来,把镰刀抡了几圈,用尽全力扔了回去,一面还高声叫骂着。然而,陶努斯认为不宜这样
\newpage

加深敌对情绪,或许以后还能从那里换些水来。 

现在没人再去记这一天或者是这几天到底前进了多少米;王妃上的姑娘觉得应该在八十到二百米之间;工程师倒没有她那么乐观,但他乐于做出各种复杂的估算,拖延与女邻居一起计算的时间,DKW的旅行推销员正施展自己的职业本领对她大献殷勤,工程师觉得时不时有点儿事情打打岔也挺有意思。就在这天下午,负责驾驶弗洛里德的小伙子跑过来告诉陶努斯,有一辆福特水星正在高价卖水。陶努斯拒绝了,可是到了天黑的时候,一位修女为ID上的老太太向工程师要一口水喝,老太太不曾抱怨,但她的确很难受,一直握着丈夫的手,由两位修女和王妃上的姑娘轮流照看着。水只剩下半升,女人们把水全给了老太太和波利欧上的妇人。这天晚上,陶努斯自掏腰包买了两升水;福特水星答应第二天再弄些水来,只是
价钱要翻番。 

现在要想把大家都召集在一起商量点事太难了,天气太冷,若非迫不得已,谁都不想离开车子。电瓶里的电也用得差不多了,不可能整天都把暖气开着
\newpage
。陶努斯做出决定,把两辆各方面配置最好的车留出来以备不时之需,给病人使用。每个人都用毯子把自己裹起来(西姆卡上的两个小伙子把自己车上的椅垫扯下来做成背心和帽子,已经有人开始仿效他们了),尽量少开车门,好存住些许热气。就在这样一个寒冷的夜晚,工程师听见王妃上的姑娘在低声抽泣。他静静地、一点点打开车门,在黑暗中摸索着,触摸到一张被泪水打湿的脸庞。姑娘顺从地跟他上了标致404;工程师扶着她在卧铺上躺下来,把唯一的一条毯子盖在她身上,又给她盖上自己的风衣。帐篷布遮住了两边的车窗,这辆救护车里面显得格外暗。工程师又放下两扇遮阳板,把自己的衬衣和一件套头毛衣挂在上面,使这辆车与外面完全隔绝开来。天快亮的时候,她在他耳边轻声说,在哭泣之前,她觉得,在
右手边,自己远远地看见了城市的灯火。 

也许那真的是一座城市,可清晨的浓雾让人连二十米外都看不清。奇怪的是,这一天车流倒前进了不少,大约有二三百米之多。这和最新的广播一致(现在谁都不去听广播了,除了陶努斯,他觉得自己有义务随时掌握最新情况);播音员们一再强调正在采
\newpage
取特殊手段来疏通道路,他们还提到,交通巡视员和警察都已经累得筋疲力尽。突然,一位修女开始说胡话。她的同伴惊恐地看着她,王妃上的姑娘则用剩下的一点香水在她的太阳穴上涂抹,那修女说起了世界末日的善恶大决战、第九日、朱砂串什么的。中午开始,天上下起了雪,雪一点一点地把车围了起来,医生在雪中艰难行走,很晚才到。他为手头没有镇静针剂深感遗憾,建议把这位修女转移到暖气好一些的车上去。陶努斯把她安置在自己的车上,那小男孩去了凯路威那里,标致203上他的小伙伴也在那辆车上;他们一起玩着玩具小汽车,玩得兴高采烈,因为只有他们没有挨饿。这一整天,加上接下来的几天,雪一直下个不停,当车流能前进几米的时候,人们还得
想各种办法清除车辆之间的积雪。 

现在,不管用什么办法获取食物和水恐怕都不会有人感到惊奇。陶努斯唯一能做的只有管好公有资金,在以物换物时争取最大的收益。每天晚上,福特水星和另一辆保时捷就会过来兜售口粮;陶努斯和工程师负责根据每个人的身体状况把口粮分发下去。不可思议的是,ID上的老太太活了下来,只是陷入了
\newpage
昏睡,女人们正在想办法。波利欧上的那位妇人几天前还时不时呕吐昏厥,但她随着气温下降彻底康复,现在成了那位修女的得力助手,帮助照顾后者那虚弱不堪甚至还有点精神错乱的同伴。军人的妻子和203的妻子负责照看两个孩子;DKW上的旅行推销商眼见王妃上的姑娘更情愿和工程师待在一起,也许是为了寻找安慰吧,一连几个小时给孩子们讲故事。到了夜里,各小组都有自己的私密生活;车门会静悄悄地打开,一个个冻得发木的黑影进进出出;谁都不去看别人,各人都像自己的影子一样,对一切视而不见。蒙在脏兮兮的毯子下,手上的指甲疯长,浑身散发出多日困在狭小空间未换衣服的气味,却有欢愉处处蔓延。王妃上的姑娘没有看错:远处确实有一座灯火辉煌的城市,而且越来越近。每到下午,西姆卡上某个小伙子就爬上车顶,身上东一块西一块地裹着椅垫的碎片和绿色的麻布,不知疲倦地瞭望着。在无望地注视远方的地平线之余,他千百次地把目光投向周围的车辆;他不无忌妒地发现王妃上的姑娘竟然在标致404上,先是热吻,接着以一只手爱抚另一个人的脖颈结束。这时他已经重新获得了404的友谊,完全是出于玩笑,他冲着他们大声叫喊,说车队又该挪
\newpage
动了;于是王妃急忙从404车上下来,钻进自己车里,可稍过片刻她就又会钻过去寻求温暖,西姆卡上的小伙子当然也希望能从别的小组里带个姑娘到自己车里来,可是如今饥寒交迫,这种美事儿想都别想,更不用说陶努斯小组与前面一个小组已经因为一罐炼乳结下了不解之仇,除了与福特水星和保时捷有生意上的往来之外,和其他的小组绝无交往的可能。因此,西姆卡上的小伙子一面满怀惆怅地叹息着,一面继续瞭望,直到冰雪与严寒使他不得不浑身哆嗦地钻回
自己的车里。 

寒意渐消,紧接着是一段风雨交加的日子,不但消磨着人们的意志,也给物资供应增添了困难,再往后便是凉爽的晴天,人们又可以走出车子,互相串门,修复和周围其他小组的关系。各组的头儿已经在一起讨论了局势,最后,他们同前面那个小组也达成了和解。人们都在议论着福特水星的突然消失,这在很长一段时间里成了人们纷纷议论的话题。谁也说不清这车到底出了什么事,但保时捷依然定期前来,把控着黑市交易。有了这些交易,水和罐头从来没短缺过,但小组的资金在一点点减少,工程师和陶努斯有
\newpage
时会自问,真到了没钱给保时捷付账的那一天该怎么办。有人提出偷袭,把他抓起来,逼他说出这些供给的来源,然而这些天车流向前移动了很长一段距离,各组的头儿们觉得最好还是再等等看,不要因为一个带暴力色彩的决定把一切都搞砸了。工程师已经进入一种近乎愉悦的无动于衷的境界,当王妃上的姑娘羞羞答答地把那事告诉他的时候,一时间他还是吃了一惊,可随后他就想开了,这种事在所难免,想到会和她有一个孩子,工程师觉得这事儿再正常不过,就和每天晚上分发食物,或是偷偷摸摸走到公路边去方便一样正常。ID上老太太的死亡也没有人觉得意外。只是深更半夜的,大家又不得不忙活了一阵,她的丈夫接受不了这样的现实,也得有人陪伴他,安慰他。前面有两个小组起了冲突,陶努斯不得不去充当仲裁的角色,勉勉强强算是把事情摆平。随时都有可能发生情况,毫无规律可言;在谁都不再指望的时候,最重要的事情发生了,而且是最无所事事的那一位最先发现的。在西姆卡的车顶上,那位兴高采烈的瞭望哨突然觉得地平线那边有了些变化(正值日落,橙黄色的斜阳那微弱的光线逐渐暗淡),一个几乎令人难以置信的异象发生了,就在五百米、然后是三百米、二
\newpage
百五十米外。他把这消息大声喊给404,404对王妃说了句什么,她迅速回到了自己车上,这时,陶努斯、军人、那个乡下人都已经飞奔而至,小伙子还站在西姆卡的车顶上,用手指着前方,一遍又一遍地重复着他的宣言,仿佛是想说服自己他双眼所见是实实在在的景象;这时他们听见一片骚动,一股沉重然而不可遏制的迁徙浪潮把车龙从无休无止的昏睡中猛然惊醒,试探着它的力量。陶努斯大声命令各人回到自己车里,波利欧、ID、菲亚特600和迪索托同时发动了。双马力、陶努斯、西姆卡和阿利亚纳紧跟着动了起来,西姆卡的小伙子还陶醉在自己的成就里,他转过头来朝着404挥了挥手,这时,404、王妃、修女们的双马力和DKW也同时开动了。可一切还取决于这种状态能持续多长时间;开到和王妃并排的时候,404几乎是习惯性地如此思量,还朝那姑娘笑了笑,给她打气。在他们后面,大众、凯路威、203还有那辆弗洛里德同时慢慢启动,在用一挡行进了一小段之后,都挂上了二挡,一直在二挡,可是毕竟不用像先前那样总要松开离合器了,大家都把脚踩在油门上,等待着换成三挡。404把左胳臂伸出车外,去够王妃的手,却只勉强碰到了她的指尖,
\newpage
他在她的脸上看到了一丝微笑,仿佛不敢相信有这样的好事,他想,他们很快就会到巴黎了,要先好好洗个澡,一起随便到哪里去,到他家,或她家,先洗个澡,再去吃饭,要洗个没完没了,要吃饭,还要喝点儿什么,要有家具,一间带家具的卧室,还要带浴室,能涂上剃须膏好好刮刮胡子,还得有抽水马桶,有食物,有抽水马桶,还有床单。巴黎就意味着一个抽水马桶和两条床单,热水冲洒在胸口和腿上,一把指甲刀,白葡萄酒,接吻之前必须喝点儿白葡萄酒,身上还要有薰衣草精油和古龙水的味道,然后他们钻进干干净净的床单中间,在明亮的灯光下充分地相知相识,再去浴室里嬉闹一番,相亲相爱,再冲个澡,喝点儿什么,去一家理发店,再去浴室,抚弄床单,也在床单里互相爱抚,在肥皂泡沫、薰衣草精油和毛刷之间相亲相爱,然后再去考虑接下来要做的事情,考虑孩子,考虑其他问题,考虑他们的未来,这一切都要取决于车别再停顿下来,车流能继续前进,哪怕还不能挂上三挡,就这样挂着二挡开吧,只要能继续前进就行。404的保险杠蹭到了西姆卡,404身子后仰靠到座位上,觉得速度在加快,他感觉可以更快些,还不至于碰到西姆卡,西姆卡也在提高车速,不
\newpage
用担心会撞上波利欧,他感到凯路威紧跟在自己后边,大家都在一点点地加速,可以换三挡了,不会磨损发动机,变速杆奇迹般地挂上了三挡,车开得更平稳,也更快了,404向左面投去惊喜而温情的一瞥,想捕捉王妃的眼神。很自然,以这样的速度跑起来,各列车队很难并驾齐驱,王妃现在领先近一米,404只能看见她的后脑勺和一点点侧影,正在这时,她也转过头来看他,看到404越来越靠后,姑娘露出惊奇的神情。404微笑着以示安慰,猛地加速,可几乎立刻就踩下了刹车,差一点就撞上了西姆卡;他短促地按了一下喇叭,西姆卡的小伙子从倒车镜里看了他一眼,做了个无能为力的表情,又伸出左手指指前面的波利欧,两车几乎贴在了一起。王妃现在领先三米,和西姆卡并排,203和404开在了一起,车上的小女孩挥着手,让他看自己的小洋娃娃。右手边一团红色的影子分散了404的注意力;不是修女们开的那辆双马力,也不是军人的那辆大众,而是一辆陌生的雪佛兰,雪佛兰也超过去了,跟着是一辆蓝旗亚和一辆雷诺8。左边,一辆ID和他并行,后来也一米一米地和他拉开了距离,ID被后面一辆403取代位置的时候,404还勉强能看见前面的20
\newpage
3,王妃被它挡住了。他们的小组就这样散开,已经没有什么小组了,陶努斯应该在前面二十多米远的地方,它后面是王妃;这时,左边第三列也落后了,因为本来该是旅行推销员的DKW的位置,现在他看见的是一辆黑色的老式货车的车尾,可能是辆雪铁龙,也说不定是辆标致。车都挂着三挡,随着一列列车流的节奏,时而超到前面,时而又落到后面,浓雾和夜色中,公路两边的树木房屋都向后方闪去。前面的车打开灯,后面的也相继打开了红色指示灯,夜幕一下子降临了。时不时有喇叭声响起,速度盘上的指针越升越高,有的车列开到七十公里,也有的开到六十五或六十。在不同车列的进退之间,404还心怀一线希望,希望能追上王妃,可时间一点点流逝,他慢慢认清这是徒劳的念想,小组已经无可挽回地解散了,他们再也不能每天碰头开会,无论是例行会议还是在陶努斯车里的紧急会议,他再也不能感受到宁静的清晨里王妃给予他的爱抚,听不到孩子们玩小汽车时的嬉笑声,看不到修女们手捻念珠的情景。当前面西姆卡的刹车灯亮起的时候,404心怀一股荒唐的渴望,他停住车,匆匆拉起手刹,跳出车子,向前跑去。除了西姆卡和波利欧外(凯路威应该在他后面,但这
\newpage
对他来说无关紧要),没有一辆他认识的车;各式各样的车窗玻璃后面,一些他平生从未见过的面孔看着他,带着震惊,甚至带着愤慨。喇叭一阵乱响,404不得不回到自己的车上,西姆卡的小伙子对他做了个友好的表情,仿佛表示能理解他的举动,鼓励般指了指巴黎的方向。车流继续前行,开始几分钟前进得很慢,到后来,高速公路仿佛完全放开了。404的左边跑着一辆陶努斯,有那么一瞬间,404以为他们的小组重新聚合起来了,一切又恢复了先前的秩序,不必以打破为代价而继续前行。可这辆陶努斯是绿色的,而且方向盘后面坐的是个戴墨镜的女人,她目不转睛地盯着前方。这时候,只能随波逐流,机械地跟上周围车辆的速度,什么也不去想。他的皮夹克应该是落在军人的大众上了。他前几天看的那本小说在陶努斯那里。一瓶几乎空了的薰衣草精油落在了修女们的双马力上。他这里倒有王妃上的姑娘当吉祥物送给他的长毛绒小熊,他不时伸出右手摸一摸。荒唐的是,他无法抛却这些念头,九点半钟该去分发食品、探望病人,还得和陶努斯以及阿利亚纳的乡下人一起分析形势;然后天黑了,王妃会悄悄来到他的车上,满天的星斗和云彩,这才叫生活。是的,生活本该这
\newpage
样,一切不能就这样告终。也许军人能弄到些水,最后那几个小时水实在稀缺;不管怎么说,只要能按照那家伙的要求付钱,还是可以指望保时捷的。车前的天线上,红十字旗帜还在猎猎飘扬,车已经跑到了每小时八十公里,前方的灯火越来越明亮,只有一件事他不明白,为什么要这么匆忙,为什么深更半夜在一群陌生的汽车中,在谁都不了解谁的人群中,在这样一个人人目视前方、也只知道目视前方的世界里,要样向前飞驰。

\end{document}
