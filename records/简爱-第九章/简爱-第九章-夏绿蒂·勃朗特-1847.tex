\documentclass{article}
\usepackage[utf8]{inputenc}
\usepackage{ctex}

\title{简爱-第九章\footnote{Click to View:\url{https://web.archive.org/web/20221128030416/https://www.99csw.com/book/2116/61673.htm}}}
\author{夏绿蒂·勃朗特}
\date{1847-10}

% \setCJKmainfont[BoldFont = Noto Sans CJK SC]{Noto Serif CJK SC}
% \setCJKsansfont{Noto Sans CJK SC}
% \setCJKfamilyfont{zhsong}{Noto Serif CJK SC}
% \setCJKfamilyfont{zhhei}{Noto Sans CJK SC}
% \setlength\parindent{0pt}

\begin{document}
\CJKfamily{zhkai}

\maketitle


\Large

然而,罗沃德的贫困,或者不如说艰辛,有所好转。春天即将来临,实际上已经到来,冬季的严寒过去了。积雪已融化,刺骨的寒风不再那般肆虐,在四月和风的吹拂下,我那双曾被一月的寒气剥去了一层皮,红肿得一拐一拐的可怜的脚,已开始消肿和痊愈。夜晚和清晨不再出现加拿大式的低气温,险些把我们血管里的血冻住。现在我们己受得了花园中度过的游戏的时刻。有时逢上好日子,天气甚至变得温暖舒适。枯黄的苗圃长出了一片新绿,一天比一天鲜嫩,使人仿佛觉得希望之神曾在夜间走过,每天清晨留下她愈来愈明亮的足迹。花朵从树叶丛中探出头来,有雪花莲呀、藏红花呀、紫色的报春花和金眼三色紫罗兰。每逢星期四下午(半假日)、我们都出去散步,看到不少更加可爱的花朵,盛开在路边的篱笆下

\newpage
。 

我还发现,就在顶端用尖铁防范着的花园高墙之外,有着一种莫大的愉快和享受,它广阔无垠,直达天际,那种愉快来自宏伟的山峰环抱着的一个树木葱笼绿荫盖地的大山谷;也来自满是黑色石子和闪光漩涡的明净溪流。这景色与我在冬日铁灰色的苍穹下,冰霜封冻、积雪覆盖时看到的情景多么不同呀!那时候,死一般冷的雾气被东风驱赶着,飘过紫色的山峰,滚下草地与河滩,直至与溪流上凝结的水气融为一体。那时,这条小溪是一股混浊不堪、势不可挡的急流,它冲决了树林,在空中发出咆哮,那声音在夹杂着暴雨和旋转的冻雨时,听来常常更加沉闷。至于
两岸的树木,都己成了一排排死人的骨骼。 

四月己逝,五月来临。这是一个明媚宁静的五月,日复一日,都是蔚蓝的天空,和煦的阳光,轻柔的西风和南风。现在,草木茁壮成长起来。罗沃德抖散了它的秀发,处处叶绿,遍地开花。榆树、岑树和橡树光秃秃的高大树干,恢复了生气勃勃的雄姿,林间植物在幽深处茂密生长,无数种类的苔鲜填补了林中的空谷。众多的野樱草花,就像奇妙地从地上升起
\newpage
的阳光。我在林荫深处曾见过它们淡谈的金色光芒,犹如点点散开的可爱光斑。这一切我常常尽情享受着,无拘无束,无人看管,而且几乎总是独自一人。这种自由与乐趣所以这么不同寻常,是有其原因的、而
说清楚这个原委,就成了我现在的任务。 

我在说这个地方掩映在山林之中,坐落在溪流之畔时,不是把它描绘成一个舒适的住处吗?的确,舒适倒是够舒适的,但有益于健康与否,却是另一回
事了。 

罗沃德所在的林间山谷,是大雾的摇篮,是雾气诱发的病疫的滋生地。时疫随着春天急速的步伐,加速潜入孤儿院,把斑疹伤寒传进了它拥挤的教室和
寝室,五月未到,就己把整所学校变成了医院。 

学生们素来半饥半饱,得了感冒也无人过问,所以大多容易受到感染。八十五个女生中四十五人一下子病倒了。班级停课,纪律松懈。少数没有得病的,几乎已完全放任自流,因为医生认为他们必须经常参加活动,保持身体健康。就是不这样,也无人顾得
\newpage
上去看管她们了。坦普尔小姐的全部注意力已被病人所吸引,她住在病房里,除了夜间抓紧几小时休息外,寸步不离病人,教师们全力以赴,为那些幸而有亲戚朋友,能够并愿意把她们从传染地带走的人,打铺盖和作好动身前的必要准备。很多已经染病的回家去等死;有些人死在学校里,悄悄地草草埋掉算数,这
种病的特性决定了容不得半点拖延。 

就这样,疾病在罗沃德安了家,死亡成了这里的常客;围墙之内笼罩着阴郁和恐怖;房间里和过道上散发着医院的气味,香锭徒劳地挣扎着要镇住死亡的恶臭。与此同时,五月的明媚阳光从万里无云的天空,洒向陡峭的小山和美丽的林地。罗沃德的花园花儿盛开,灿烂夺目。一丈红拔地而起,高大如林,百合花已开,郁金香和玫瑰争妍斗艳,粉红色的海石竹和深红的双瓣雏菊,把小小花坛的边缘装扮得十分鲜艳。香甜的欧石南,在清晨和夜间散发着香料和苹果的气味。但这些香气扑鼻的宝贝,除了时时提供一捧香草和鲜花放进棺材里,对罗沃德的人来说已毫无用
处。 

\newpage

不过我与其余仍然健康的人,充分享受着这景色和季节的美妙动人之处。他们让我们像吉卜赛人一样,从早到晚在林中游荡,爱干什么就干什么,爱上哪里就上哪里。我们的生活也有所改善。布罗克赫斯特先生和他的家人现在已从不靠近罗沃德,家常事也无人来有问,啤气急躁的管家己逃之夭夭,生怕受到传染。她的后任原本是洛顿诊所的护士长,并未习惯于新地方的规矩,因此给得比较大方。此外,用饭的人少了,病人又吃得不多,于是我们早饭碗里的东西也就多了一些。新管家常常没有时间准备正餐,干脆就给我们一个大冷饼,或者一厚片面包和乳酪,我们会把这些东西随身带到树林里,各人找个喜欢的地方
,来享受一顿盛宴。 

我最喜欢坐在一块光滑的大石头上。这块石头儿立在小溪正中,又白又干燥,要淌水过河才到得那里,我每每赤了脚来完成这一壮举。这块石头正好够舒舒服服地坐上两个人,我和另一位姑娘。她是我当时选中的伙伴,名叫玛丽·安·威尔逊,这个人聪明伶俐,目光敏锐。我喜欢同她相处,一半是因为她机灵而有头脑,一半是因为她的神态使人感到无拘无束
\newpage
。她比我大几岁,更了解世情,能告诉我很多我乐意听的东西,满足我的好奇心。对我的缺陷她也能宽容姑息,从不对我说的什么加以干涉。她擅长叙述,我善于分析;她喜欢讲,我喜欢问,我们两个处得很融
洽,就是得不到很大长进,也有不少乐趣。 

与此同时,海伦·彭斯哪儿去了呢?为什么我没有同她共度这些自由自在的舒心日子?是我把她忘了,还是我本人不足取,居然对她纯洁的交往感到了厌倦?当然我所提及的玛丽·安·威尔逊要逊于我的第一位相识。她只不过能给我讲些有趣的故事,回对一些我所津津乐道的辛辣活泼的闲聊。而海伦呢,要是我没有说错,她足以使有幸听她谈话的人品味到高
级得多的东西。 

确实如此,读者,我明白,并感觉到了这一点。尽管我是一个很有缺陷的人,毛病很多,长处很少,但我决不会嫌弃海伦,也不会不珍惜对她的亲情。这种亲情同激发我心灵的任何感情一样强烈,一样温柔,一样令人珍重。不论何时何地,海伦都向我证实了一种平静而忠实的友情,闹别扭或者发脾气都不会
\newpage
带来丝毫损害。可是海伦现在病倒了。她从我面前消失,搬到楼上的某一间房子,已经有好几周了。听说她不在学校的医院部同发烧病人在一起,因为她患的是肺病,不是斑疹伤寒。在我幼稚无知的心灵中,认为肺病比较和缓,待以时日并悉心照料,肯定是可以
好转的。 

我的想法得到了证实,因为她偶尔在风和日丽的下午下楼来,由坦普尔小姐带着步入花园。但在这种场合,她们不允许我上去同她说话。我只不过从教室的窗户中看到了她,而且又看不清楚,因为她裹得
严严实实,远远地坐在回廊上。 

六月初的一个晚上,我与玛丽·安在林子里逗留得很晚。像往常一样,我们又与别人分道扬镳,闲逛到了很远的地方,远得终于使我们迷了路,而不得不去一间孤零零的茅舍回路。那里住着一男一女,养了一群以林间山毛榉为食的半野的猪。回校时,己经是明月高挂。一匹我们知道是外科医生骑的小马,呆在花园门口。玛丽·安说她猜想一定是有人病得很重,所以才在晚间这个时候请贝茨先生来。她先进了屋
\newpage
,我在外面呆了几分钟,把才从森林里挖来的一把树根栽在花园里,怕留到第二天早晨会枯死。栽好以后,我又多耽搁了一会儿,沾上露水的花异香扑鼻。这是一个可爱的夜晚,那么宁静,又那么温煦。西边的天际依旧一片红光,预示着明天又是个好天。月亮从黯淡的东方庄严地升起。我注意着这一切,尽一个孩子所能欣赏着。这时我脑子里出现了一个从未有过的
想法: 

“这会儿躺在病床上,面临着死亡的威胁是多么悲哀呀!这个世界是美好的,把人从这里唤走,到一个谁都不知道的地方去,会是一件十分悲惨的事。
” 

随后我的脑袋第一次潜心来理解已被灌输进去的天堂和地狱的内涵,而且也第一次退缩了,迷惑不解了,也是第一次左右前后扫视着。它在自己的周围看到了无底的深渊,感到除了现在这一立足点之外,其余一切都是无形的浮云和空虚的深渊。想到自己摇摇晃晃要落入一片混乱之中,便不禁颤抖起来。我正细细咀嚼着这个新想法,却听得前门开了,贝茨先生
\newpage
走了出来,由一个护士陪同着。她目送贝茨先生上马
离去后,正要关门,我一个箭步到了她跟前。 


“海伦·彭斯怎么样了?” 


“很不好,”回答说。 


“贝茨先生是去看她的吗?” 


“是的。” 


“对她的病,他说了些什么呀?” 


“他说她不会在这儿呆很久了。” 

这句话要是昨天让我听到,它所表达的含义只能是,她将要搬到诺森伯兰郡自己家去了,我不会去怀疑它包含着“她要死了”的意思。但此刻我立即明白了。在我理解起来,这句话一清二楚,海伦在世的日子已屈指可数,她将被带往精灵的地域,要是这样的地域确实存在的话。我感到一阵恐怖,一种今人震
\newpage
颤的悲哀,随后是一种愿望,一种要见她的需要。我
问她躺在哪一个房间。 


“她在坦普尔小姐的屋里,”护士说。 


“我可以上去同她说话吗?” 

“啊,孩子!那不行。现在你该进来了,要是
降了露水还呆在外面,你也会得热病的。” 

护士关了前门,我从通往教室的边门溜了进去。我恰好准时,九点刚敲,米勒小姐正吩咐学生上床

也许过了两小时,可能是将近十一点了,我难以入睡,而且从宿舍里一片沉寂推断,我的同伴们都已蒙头大睡。于是我便轻手轻脚地爬起来,在睡衣外面穿了件外衣,赤着脚从屋里溜了出来,去寻找坦普尔小姐的房间。它远靠房子的另外一头,不过我认得路。夏夜的皎洁月光,零零落落地洒进过道的窗户,使我毫不费力地找到了她的房间。一股樟脑味和烧焦的醋味,提醒我己走近了热病病房。我快步走过门前
\newpage
,深怕通宵值班的护士会听到我。我担心被人发现被赶回房去。我必须看到海伦——在她死去之前必须拥抱她一下——我必须最后亲吻她一下,同她交换最后
一句话。 

我下了楼梯,走过了楼底下的一段路,终于毫无声响地开了和关了两道门,到了另一排楼梯,拾级而上,正对面便是坦普尔小姐的房间,一星灯光从锁孔里和门底下透出来,四周万籁俱寂。我走近一看,只见门虚掩着,也许是要让闷人的病室进去一点新鲜空气。我生性讨厌犹犹豫豫,而且当时急不可耐,十分冲动——我全身心都因极度痛苦而震颤起来,我推开门,探进头去,目光搜索着海伦,担心遇见死亡。
 

紧靠坦普尔小姐的床铺,被白色的帷帐遮去了一半的是一只小床。我看到了被子底下身子的轮廓,但脸部被帷幔遮住了。那位在花园里同我讲过话的护士坐在一把安乐椅上,睡着了。一支灯芯未剪的蜡烛幽幽地在桌子上燃着。没有看到坦普尔小姐。我后来知道,她已被叫到热病病室,看望一个昏迷不醒的病
\newpage
人。我往前走去,随后在小床旁边停了下来,我的手伸向帷幔,但我宁愿在拉动之前开口说一下,我们人
仍然畏缩不前,唯恐看到一具尸体。 


“海伦!”我轻声耳语道,“你醒着吗?” 

她动弹了一下,自己拉开帷幔,我后到了她的脸,苍白、憔悴,却十分镇静,她看上去没有什么变
化,于是我的恐惧心理顿时消失了。 

“真是你吗,简?”她以独特的柔和语调问。

“啊!”我想,“她不会死,她们搞错了,要是她活不了啦,她的言语和神色不会那么镇定自若。

我上了她的小床,吻了她一下。她的额头冰冷,两颊也冰冷,而且还很消瘦,她的手和手腕也都冰
冷,只有她那微笑依旧。 

“你为什么到这儿来,简?已经过了十一点啦

\newpage
,几分钟前我听见敲的。” 

“我来看你,海伦。我听说你病得很重,我不
同你说句话就睡不着。” 

“那你是来同我告别的了,也许许来得正是时
候。” 

“你上哪儿去吗,海伦?你要回家是不是?”


“是的,回到我永久的——我最后的家。” 

“不,不,海伦,”我顿住了,心里很难过。我竭力咽下眼泪,这时海伦一阵咳嗽,不过没有吵醒护士。咳完以后,她精疲力尽地躺了几分钟,随后轻
声说: 

“简,你都光着你的小脚呢,躺下来吧,盖上
我的被子。” 

我照她的话做了。她用胳膊楼住我,我紧偎着

\newpage
她,在沉默了很久之后,她继续低声耳语着说: 

“我很愉快,简,你听到我已经死了的时候,你可千万别悲伤。没有什么可以感到悲伤的。总有一天我们大家都得死去。现在正夺去我生命的疾病并不痛苦。既温和而又缓慢,我的心灵已经安息。我不会让任何人感到太悲痛,我只有一个父亲,他新近刚结婚,不会思念我。我那么年纪轻轻就死去,可以逃脱大苦大难。我没有会使自己在世上发迹的气质和才能
。要是我活着,我会一直错下去的。” 

“可是你到哪儿去呢,海伦?你能看得见吗?
你知道吗?” 


“我相信,我有信仰,我去上帝那儿。” 


“上帝在哪儿?上帝是什么?” 

“我的创造者,也是你的。他不会永远毁坏他所创造的东西。我毫无保留地依赖他的力量,完全信任他的仁慈,我数着钟点,直至那个重要时刻到来,

\newpage
那时我又被送还给他,他又再次显现在我面前。” 

“海伦,那你肯定认为有天堂这个地方,而且
我们死后灵魂都到那儿去吗?” 

“我敢肯定有一个未来的国度。我相信上帝是慈悲的。我可以毫无忧虑地把我不朽的部分托付给他,上帝是我的父亲,上帝是我的朋友,我爱他,我相
信他也爱我。” 


“海伦,我死掉后,还能再见到你吗?” 

“你会来到同一个幸福的地域,被同一个伟大的、普天下共有的父亲所接纳,毫无疑问,亲爱的简
。” 

我又再次发问,不过这回只是想想而已。“这个地域在哪儿?它存在不存在?”我用胳膊把海伦楼得更紧了。她对我似乎比以往任何时候都要宝贵了,我仿佛觉得我不能让她走,我躺着把脸埋在她的颈窝
里。她立刻用最甜蜜的嗓音说: 

\newpage

“我多么舒服啊!刚才那一阵子咳嗽弄得我有点儿累了,我好像是能睡着了,可是别离开我,简,
我喜欢你在我身边。” 

“我会同你呆在一起的,亲爱的海伦。谁也不
能把我撵走。” 


“你暖和吗,亲爱的?” 



“晚安,简。” 


“晚安,海伦。” 


她吻了我,我吻了她,两人很快就睡熟了。 

我醒来的时候已经是白天了,一阵异样的抖动把我弄醒了。我抬起头来,发现自己正躺在别人的怀抱里,那位护士抱着我,正穿过过道把我送回宿舍,我没有因为离开床位而受到责备,人们还有别的事儿要考虑,我提出的很多问题也没有得到解释。但一两
\newpage
天后我知道,坦普尔小姐在拂晓回房时,发现我躺在小床上,我的脸蛋紧贴着海伦·彭斯的肩膀,我的胳
膊搂着她的脖子,我睡着了,而海伦死了。 

她的坟墓在布罗克布里奇墓地,她去世后十五年中,墓上仅有一个杂草丛生的土墩,但现在一块灰色的大理石墓碑标出了这个地点,上面刻着她的名字“Resurgam”这个字。

\end{document}
