\documentclass{article}
\usepackage[utf8]{inputenc}
\usepackage{ctex}

\title{pqjw\footnote{Click to View:\url{}}}
\author{nebwk}
\date{}

\maketitle
\newpage

\begin{document}
\CJKfamily{zhkai}

% : comm which is only poems edit in-use %


\Large

今天闲来无事,拜读了联合光子所写的《中国和日本的真实差距》一文。该文字字珠玑,军刀忍不住就写了个读后感。

光子在文中主要谈到了三点:

1、日本的GDP和人均GDP都远远超过中国。

2、日本的专利技术申报的数量要超过中国

3、光子在回贴中大胆地提出了“炎黄子孙历来斗不过小日本,各位还是歇歇吧,大东亚共荣不是挺好的吗,非要和日本斗,斗输了还得去当伪军,损害你们的形象”的观点。

军刀针对以上观点反驳如下:

1、光子在回帖中谈及自己是北大经济学专业的,军刀斗胆冒犯虎威来谈一下GDP的问题。根据年增长率来计算,我拿日本目前GDP4.6万亿美元和中国GDP1.4万亿美元,及我在网上查到的,日本年增长2003、2004年度实际GDP分别为百分之0.2和百分之0.9(日本大和证券数字)和保守下调后的中国年增长百分之7的数据计算,中国的经济总量有希望在2030年前后追平日本。(当然,此计算方法将有些中国zf非得为日本GDP增长做贡献的因素排除在外)

2、光子在文中提及科技方面,军刀承认,确实在很多领域,中日之间存在差距。但有些领域,中国是超过日本的。不说中国,就拿另一个国家——俄罗斯举例。别看俄罗斯现在满目疮痍,你让日本打一下俄罗斯试试。所以GDP和商业范畴的科技能力不见得一定对国力和军力构成绝对的支持。

3、光子在文章大胆地“炎黄子孙历来斗不过小日本,各位还是歇歇吧,大东亚共荣不是挺好的吗,非要和日本斗,斗输了还得去当伪军,损害你们的形象”的观点。

对此军刀只能说,你愿意怎么办,那是你的自由,军刀无权管。大街上还有愿意当妓女的,那有什么辄啊,你说是不是。此引用一段我曾经在其他文章的结尾中写过的一段话:

在过去的一百年中,无数的中国人为了摆脱屈辱,为了这个民族可以坚强的站立而献出了生命。在过去的一百年中,无数的中国士兵浴血奋战,出生入死,他们用意志和钢铁较量,他们用刺刀,用勇气,用冲锋,用坚强打败所有企图主宰我们的人。无数的中国人,用平型关嘹亮的冲锋号,用堵住地堡时胸膛弹孔喷射的热血,用朝鲜三八线踩过敌人头盔的草鞋,用金门决不低头轰鸣的炮声,用喜玛拉亚山脉印军绝望的惨呼,用刺进越军身体的刺刀告诉这个世界,那个中国人引颈待戮的年代已经过去了……

对于日本,我们要学习它,研究它,甚至模仿它,但最终的目的不是建立个什么狗屁圈,而是要在经济、军事等领域最终压制它,做掉它。

军刀无能,不能早生几十年在昆仑关杀敌,和平年代,但商场如战场军刀愿意用平生所学为中国的经济尽最微薄的一点力。



\end{document}
