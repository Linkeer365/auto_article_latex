\documentclass{article}
\usepackage[utf8]{inputenc}
\usepackage{ctex}

\title{宁有故人,可以相忘\footnote{Click to View:\url{https://web.archive.org/web/20120715022030/http://www.zh61wx.com/Article/Class1/10935.html}}}
\author{李秋沅}
\date{2009-01}

% \setCJKmainfont[BoldFont = Noto Sans CJK SC]{Noto Serif CJK SC}
% \setCJKsansfont{Noto Sans CJK SC}
% \setCJKfamilyfont{zhsong}{Noto Serif CJK SC}
% \setCJKfamilyfont{zhhei}{Noto Sans CJK SC}
% \setlength\parindent{0pt}

\begin{document}
\CJKfamily{zhkai}

\maketitle


\Large


(一) 


木棉岛上的林家园已荒了许久。 

林家园与外公的老屋仅一墙之隔,园内枝籐蔓
布,荒草离离。 

“林家园主人是位将军。”外公如是说。夏夜里,他摇着蒲扇,躺在竹摇椅上纳凉,我紧挨着他,睁大眼,期望他能继续往下说林家园的故事,可他只长叹一声,话音戛然而止。在如此的夏夜里,我曾听他说过莫家大院,说过李家庭院,说过许许多多木棉岛上那在时光中渐行渐远的往事。往事如烟,已被时光抽去了骨血,寄生在外公的唇齿间。外公的言语在

\newpage
如水月色中游走,却绕开了林家园。 

外公身边有一本英文版的西洋老歌集子,是自他青年时代留学英国就带在身边的《The 101 best songs》,外公喜欢吟唱其中的《Auld Lang Syne》(《友谊地久天长
》): 

“Should auld acquaintance be forgot,and never brocht to mind?.....
.” 


唱罢英文,外公随即用中文重复一遍: 

“宁有故人,可以相忘,曾不中心卷藏?宁有故人,可以相忘,曾不睠怀畴曩?我尝与子乘兴翱翔,采菊白云之乡。载驰载驱徵逐踉跄,怎不依依既往?我尝与子荡桨横塘,清流浩浩汤汤,永朝永夕容与
倘佯,怎不依依既往?” 

“外公,《友谊地久天长》的中文歌词不是这
\newpage
样的。”我纠正,“应该是‘怎能相忘,旧日朋友.
....’” 

“我唱的,是林家园的林太太所译。译文讲究‘信达雅’,林太太的译词,才算是好的。”外公若
有所思、若有所失。 

我曾壮着胆子翻过林家园攀满爬山虎的院墙,小心翼翼地在离离荒草中寻一条落脚之道,走近林家宅院。宅门紧闭,破败不堪,中间朽败露出一个大缝。钻过缝,我闯进林家无主的大屋。屋内弥漫着一股腐浊之气,尘土在透隙而入的光柱中飞扬,局促不安。四周静谧,大屋无语静默着,我陷入了恍惚之中,时间止步,我的思维仿佛也被寂静所凝固。好一会儿
,我才回过神来,大喘着气逃离林家园。 

“林家的主人呢,怎么不回来?”我问外公。
 


外公静默不语,神色萧肃。 

\newpage

外公的书房墙上挂着好几幅小幅油画。画有年头了,画上的色彩似被时光罩上了一层灰蒙蒙的纱,画中的景物模糊不清。大扫除时,母亲曾想取下好好
清洗清洗。 


“别碰那些画,别碰!”外公慌忙阻止。 


“为什么?” 

“我们是外行。弄不好会毁了画的。画的主人
会回来的,就快回来了......” 

日子就这么一天天过去了,林家园内,草木枯荣,林家的主人没回来,画的主人亦不见踪影。外公活到了八十岁,一天天颓败下去。终于有一天,他老得说不出话来。我为他整理衣物时,在箱子的底层发
现了一个漆木小匣子。 

“SHUANG CHOU”。木匣上铭刻着这几个字母。外公瞪着我看,张大嘴,却什么也说不

\newpage
出。叹了口气,闭上眼。 

外公离去那天,天出奇的冷。外公去世后,我同父母搬出了老屋。在我们离开老屋的那一刹那,厅
堂瓷砖突然开裂。 


(二) 

我将原本挂在外公书房里的油画挂在新居厅堂里。我曾以为自己再也不会想起外公老屋的人与事。可事实上,我似乎从未离开老屋。在我的梦境中,老屋时常屹立其中。外公留下的匣子里,有一沓纸边已泛黄的西洋曲译稿,有一柄镌刻着“林实”名字的短剑,还有几幅与厅堂里所挂油画风格类似的老画..
.... 


“林实是谁?”我摩挲着短剑问母亲。 

“就是林家园主人啊。”母亲回答,“外公连这都没告诉你?那柄短剑,应该就是你外公常提起的国民党‘军魂剑’。林实可是一名国民党的高级将领

\newpage
。” 


“外公似乎与林家渊源极深?” 

“林家与外公施家原本就是世交,林实与外公从小一起玩大的。林实娶的太太,恰巧又是外公在英国的校友。与林家的这层关系,后来你外公可没少受
罪。” 

我缠着母亲告诉我林家园的故事,母亲为难地摇摇头,“外公解放后就离开木棉岛,多年辗转颠簸。我六岁随你外公离开,现在才回来,对木棉岛,对
林家园并无多深印象。” 

我始终无法将匣中这些东西理出一个头序来。它们真实地存在着。在它们的存在之下,埋藏着我所
不知的故事。 

木棉岛上,莫家园林宏伟气派,依山伴海,蔚为壮观;李家花园,纤巧典雅,为民国才女李士奇故居;苏家长廊小筑,中西合璧......林家园是一处被忽略的风景,淹没于林林总总的建筑之中。倘
\newpage
若真如外公所言,林家园住过一位将军,怎又默默无
闻无人所知? 

偌大的林家园,面朝大海屹立山头。沉默不语
,独自凋零。 

从老屋搬出后,家人将外公的老屋租给了木棉岛上艺术学校的学生们。大学生们时常放着摇滚,在院子里习画立雕塑。每月一次,父母差我去收房租。我也乐得与那些比我大不了多少的大学生们厮混。我喜欢看他们捏泥塑、画油画。混熟了,好客的母亲答
应我的请求,邀他们到家中做客。 

张诚,一位学油画的学生被厅堂里外公留下的
画所吸引。 

“这原是挂在老屋的画,应该是幅很老很老的
油画了。”母亲无意地说。 


“谁画的?”其他人全都好奇地凑过来。 

\newpage

“不知道。别人给外公的。外公认识一位将军
!”我开始吹牛。 

“仔细看看,画作者肯定会在自己的作品上留记号的。”张诚凑近画,细细找寻,“诺,在这里,”他指了指画布的左下角,“签名几乎被背景的色彩盖过去了,‘SHUANG CHOU 1930’

“SHUANG CHOU?这应该是位中国画家的音译名。”张诚簇了簇眉,摇了摇头,“不认识。”他的眼舍不得离开画,“此画雅致大气,应该是出自名家之手吧。油画作于1930年,画者应该是中国油画界的先驱人物了,李为夫、徐悲鸿、卫天霖、颜文梁、庞熏琴、刘海粟、洪瑞麟、常书鸿、吴作人、潘玉良、林枫眠、丁衍庸、冯钢白......”他喃喃自语,将百年来国内油画大家一一数过,
“SHUANG CHOU,是谁呢?” 

过了几天,张诚的老师,艺校名声在外的油画
家唐岳教授闻风亲自过来看画。 

\newpage

“这画,在西方印象派基础上融入了中国浅绛山水画的特点,减少色相的种类,降低色彩的饱和度,提高明度,柔和清新。作画之人,西洋画功底深厚,同时,深谙中国山水画精髓。”唐岳簇眉沉思,“‘SHUANG CHOU’,难道这是周裳的画?前几年,美国西雅图美术馆曾向我打听过这人。他们说,他们正为中国人周裳举办画展,收集的画作还不太齐整,少了他某些年段的的作品。木棉岛上的人与物是他的画作中频繁出现的主题。他们期望从中国同行那里得到周裳更多的信息。这是我第一次听人说起‘周裳’的名字。在此之前,周裳在国内,完全不为
人所知。” 

大家一听,好奇不已。父亲拿出了上好的铁观音茶,亲自为大家泡茶。茶香四溢间,唐岳启开话匣
: 

“周裳的父亲是一位热心新教育的人,积极支持孙中山的国民革命,与国民党权贵王廷凯更是好朋友。周裳就是王廷凯的教子。周裳自小深受中国文化传统的熏陶,18岁赴英国留学,1928年毕业于
\newpage
英国皇家美术学院。1938年与王廷凯的孙女结婚,活跃于英美美术界,1948年,在他的事业最盛之际,他突然宣布与王氏离婚。随后悄然淡出名利圈,1970年,周裳去世时,未留下任何子女,也几乎未留下任何照片、文字等个人资料。一位天才画家
就此陨落,悄无声息。” 

周裳的画,怎么会到了林家园,最后又到了外
公手中?张诚与我百思不得其解。 


(三) 

周裳的画再次钩起了我对林家园的好奇。我上网搜“林实”的名字,在网上出现的189条记录中,与“将军”挂钩的仅有10条。其中仅有一条简单的记录吸引了我:“林实,1900年生,黄埔军校第三期学员,抗战期间,因屡立战功,升任国民党陆军中将。抗战胜利后,遭蒋介石猜忌,借腿疾之故带职在木棉岛休养。1948年1月赴重庆任“军事参议院院长”虚职,实被蒋介石软禁,半年后失踪。”

\newpage

“1948年”,我倒抽一口冷气。我仿佛陷入一个黑暗的迷阵之中,我能感受到出口处曦微的天
光,却无法寻得路径。 

我疯狂地上网搜寻木棉岛林家园的资料,空白,一片空白。林家园巍然屹立于时光之中,屹立于故人的记忆之中,但是,它却被历史抹杀了,一笔钩销。林家园的失语,宛若外公最后的叹息,令我不安却
无能为力。 

“您知道林实么......知道林家园......”母亲带我拜访黄老太时,我再一次提起林家园。一有机会,我便向木棉岛上的老人们询问林家园的往事。而我,也渐渐长大,对历史的浓厚兴趣令
我选择了历史作为大学专业。 

“傻小子,你怎么会对那些旧事感兴趣呢?该死的人都死了,该老的人都老了......林家园?就是据说女主人被人打了三枪的林家园?林家园的太太,可是位风雅人物啊。对了,你外公和她很熟悉啊,都是留英回来的。”八十岁的黄老太,谈起当年
\newpage
往事,眼眸里忽然焕发神采,“当年,林家园常开PARTY。那时,我还是十几岁的小姑娘,有一次随父亲前去,林太太宴客,长长的蜡烛插在烛台,高脚玻璃杯斟满了红酒,镀银的刀叉,雪白的四方餐巾.
.....” 

“唉,都是什么时候的事了,”黄老太太眼神暗淡下来,“林太太最喜欢穿翠色旗袍。我后来再没见过别人把翠色穿得那么服贴。她说起话来,轻声细语,她是真正的淑女。琴技书画俱佳,西文更是出色。我记得,她在宴会上,曾唱过自己翻译的西洋歌曲......好好地,怎地就被杀了呢。弄不清林太太是军统杀的,还是地下党这边杀的,要不,就是日本人报复林将军。林实虽是国民党高官,却不避嫌,与各路人士交好,因此,似乎很不得老蒋欢心。杀人凶手后来也没找到,此事不了了之。林太太出事后不久,林先生就带着孩子离开林家园了。解放后,因为林先生曾是国民党高官,林家园的事再没人多提..
....”她突然噤声,沉默。 


\newpage

(四) 

夜里,我久久难以入眠,从阁楼上拿出外公给
我的木匣子。 

在夜灯下,黑色的漆木匣子泛着清冷的光,翻
开匣子,林太太的译稿静静躺在匣子里, 

“愿言与子携手相将,陶陶共举壶觞,追怀往日引杯需长,重入当年好梦!往日时光,大好时光,我将酌彼兕觥!往日时光,大好时光,我将酌彼兕觥
!” 

我倦怠地闭上眼。夕阳中,林家园内在浅风中飘荡的离离荒草;外公凝固在眼中的惆怅;黄老太太
欲言又止的无奈一一涌上心头。 

梦中,林家园光鲜如昨,满院花开正好,一位身着翠色的女子打开林家园的大门走出,她向我伸出手,“进来吧......”我一踏入园子,园内突然变了颜色,满目苍凉,那绿衣女子俶忽消逝,只听耳边有延绵耳语声,“林家园,林家园......
\newpage
林,林,林......”声音由弱转强,忽地尖锐
如山顶夜风呼啸。 

“铃......”铃声大作,我愣了半天,想确认自己不在梦中。起身一看时间,凌晨四点。是张诚的越洋电话。张诚从木棉岛艺术学院毕业后,便
去美国继续深造。 

“老兄啊,现在是北京时间四点正。”我嘟囔
着抱怨。 

“我实在等不及了,太兴奋了。我身边就是西雅图美术馆的里奥先生,就是主办周裳画展并对他展
开研究的美术馆。他们收集了一些周裳的资料。” 


“啊!” 

“他有位姐姐,叫周晨。周母早丧,周父常年在外,姐姐与他相依为命,感情极其深厚,他们一同在教会学校长大,后又一同出国留学。姐姐回国后,尊父命嫁给了国民党内重要人物林实。1947年1
\newpage
0月,周晨在林家园遭枪击身亡。1948年,姐夫林实在重庆失踪,年仅8岁的外甥被人刺死在家中。周裳极少对外界提及姐姐一家的惨剧。姐姐去世后,他曾为姐姐画了一幅肖像画,这幅画一直跟随着他直到他离开人世。1948年,周裳与王氏离婚,外界对他这一做法极其不解。王家当时在美国上流社会亦是一极引人瞩目的华人家族,周裳与王氏离异,对他的事业是极大的伤害。之后,周裳似乎一度陷入了抑郁困境,事业一落千丈。他没再婚,亦无子女,19
70年孑然一身地在宾州逝世。” 

“你,你快把他的资料带传给我看看!”我兴
奋的喊。 


“我快回去了,到时候我将图片带给你。” 


(五) 

我几乎可以这么推断:林实一家实际上是遭军统所暗杀。而周裳与王氏的决裂,也直接表达了他对与军统渊源极深的王氏家族的不满。林实一家遇难之
\newpage
际,正是临解放国共力量较量的关键时期。林实在抗战期间,英勇领导将士抗日,民望甚高,极有可能成
为共产党争取的对象 

我四处搜寻解放前夕木棉岛共产党地下组织的历史资料,期望能从中查得林实与地下组织的联系,但却一无所获。1947-1948年,木棉岛地下组织所属的闽浙赣区城市地下党组织经历了重大冤案“城工部案”的惨痛打击,大批优秀的地下组织骨干成员被错杀,大批的地下组织联络员失去应有的组织联系,地下组织受到重创。而木棉岛,正是此冤案风
暴的影响区。 

我喟然叹息,如此看来,林家即使与地下组织有所联系,因木棉岛地下组织自身的历史原因,也查无实据了。时光前行,林家园缄默,历史在林实一家
的鲜血面前背过身去。 

张诚回国,果真带了许多周裳的画作像片过来。周裳画笔下的林家园与我所见的林家园是多么的不同啊,园内繁花似锦,阳光融融,林家园的木棉树上
\newpage
,鲜红的木棉花朵朵傲立枝头......画中的周晨神情安详、眼眸温和,身着一袭白衣,静静地透过重重时光注视着我。时光止步,故人故园在周裳的画作中终于寻到了不受打扰的庇护所,获得了永恒的安
宁。 

1998年木棉岛遭遇了特大台风。名为“FRAGRANT”(芬芳)的台风却一点也不温柔,年久失修的林家园主楼坍塌了一角。政府将它定为危房,在它的院门外竖起一个醒目的警示牌:“危楼,
注意安全!” 

在外公的老屋租住的学生换了一茬又一茬,续租者由众渐寡。与林家园仅一墙之隔的老屋在外公去世之后,衰老的速度惊人。在一个风雨之夜,老屋的破漏终于令最后一位租房的学生忍无可忍,无奈弃离
。 

林家园、老屋、林实、周裳、周晨、外公......那些人,那些事,渐行渐远。我企图抓住他们曾经鲜活的过往,却见他们的影像在时光中淡去,
\newpage

支离破碎,惟有隐约耳语声轻唱: 

“宁有故人,可以相忘,曾不中心卷藏?宁有故人,可以相忘,曾不睠怀畴曩?我尝与子乘兴翱翔,采菊白云之乡。载驰载驱徵逐踉跄,怎不依依既往?我尝与子荡桨横塘,清流浩浩汤汤,永朝永夕容与

我始终弄不明白,外公是怎么得到那个漆木匣子,而又是谁将画交托给他的,是周裳?还是林实?外公一辈子,等的就是那个“失踪”了的林将军重返故园么?他的等待,难道仅仅是对往事的缅怀与深深
的追思? 

2008年,在外地工作的我重返木棉岛,林家园已完全淹没于荒草藤蔓之中。听说,不久之后,木棉岛政府将出资修复一些有碍瞻观的无主荒院并出租,以租养院,以恢复岛上旧貌,林家园亦在名单之列。当新的园子建成,又有谁能记得故园往事?又将谁将成为它的新主人呢?......

\end{document}
