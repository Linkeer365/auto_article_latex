\documentclass{article}
\usepackage[utf8]{inputenc}
\usepackage{ctex}

\title{蚂蚁街的山楂案\footnote{Click to View:\url{https://web.archive.org/web/20230510141823/http://old.zh61wx.com/Article/Class210/6330.html}}}
\author{余雷}
\date{2007-09}

% \setCJKmainfont[BoldFont = Noto Sans CJK SC]{Noto Serif CJK SC}
% \setCJKsansfont{Noto Sans CJK SC}
% \setCJKfamilyfont{zhsong}{Noto Serif CJK SC}
% \setCJKfamilyfont{zhhei}{Noto Sans CJK SC}
% \setlength\parindent{0pt}

\begin{document}
\CJKfamily{zhkai}

\maketitle


\Large


刘知县今天很高兴。 

刘知县高兴的时候脚就发痒。脚一发痒他就想
在屋顶上跑一跑。 

“现在可不是出去的时候啊。”刘知县轻声嘟囔着,尽管脚底下似乎有无数只手在轻轻地挠,挠得
他心烦意乱,刘知县也只能使劲憋着。 

刘知县现在正陪客人说话。客人是知府大人派来的官差,他们给刘知县带来了一封知府大人的亲笔
信。信中的内容,让刘知县非常高兴。 

刘知县一边和客人说话,一边悄悄把藏在官服下的两只脚提起来,脚底相对,轻轻地搓揉。但不管
\newpage
怎么搓揉,脚底依然奇痒难耐,甚至连脚趾头们都在官靴里兴奋地跳动着。刘知县只好不停地用一只脚踩住另一只脚,心里说:“即使现在没有客人也不能到屋顶上去啊。如果青峡镇的百姓在大白天看到他们的
知县大人在屋顶上奔跑一定会吓坏的。” 

但脚底的痒不管这些,似乎现在不到屋顶上跑一跑,脚趾头们就会自己从刘知县的脚上走下来。刘知县只好跺了跺脚,身子稍稍后仰,膝盖用力,轻轻翘起坐着的椅子,用两只椅子腿压住了自己的双脚。
 

“唉,当年没有走火入魔就好了。”刘知县悄
悄叹息道。 

青峡镇没有人知道矮胖的刘知县竟是江湖上赫赫有名的神捕教头最得意的弟子。十年前,刘知县练得最好的功夫是“神龙游”。即使再高再窄的地方他也能行走自如。要不是那一次练功时走火入魔,双臂从此筋脉僵硬,无法伸展自如,身体也迅速变形,从一个高挑英俊的少年变成了现在矮胖的样子。今天的
\newpage

刘知县应当是一名闻名天下的捕快。 

不能做捕快做一个知县也不错。但因为练功走火入魔,刘知县身上有许多别的知县没有的坏毛病。例如:高兴的时候要到屋顶上跑一跑;着急的时候要喝水,喝很多的水;思考问题的时候要闭着一只眼睛
;吃到好吃的东西时,耳朵会忽扇…… 

“刘知县,时候不早,我们告辞了。”客人站
了起来。 

“好的!好的!”刘知县站起身,却险些摔倒
在地,他忘了自己的脚被压在了椅子腿下。 

刘知县站稳身子,对两位客人拱了拱手:“二位慢走。请转告知府大人,我一定会竭尽全力,把目
前的事情做好。” 

刘知县所管辖的青峡镇,是位于云贵川三省交界处的一个小镇。青峡镇背靠青峡山,左临黔水河,是三省货物运输的必经之地。商人们于是把这里当作
\newpage
物资交流的中转站。青峡镇上只有一条名叫蚂蚁的小街,很多交易都在这条小得像蚂蚁一样的街上完成。

蚂蚁街就成了方圆几十里最热闹的集市。每天天亮的时候,蚂蚁街就像一条涨水的河,人们从不同的方向汇集到这里,渐渐形成一股汹涌的人流,在窄窄的街道上涌动。有做生意的,有专门来看热闹的,车来车往,人喧马叫,热闹非凡。傍晚时分,这股人
潮才各自散去。 

人多,街窄,大家难免就会有些磕磕碰碰,还有一些不法之徒趁乱干些非法的勾当。县衙的人手不多,刘知县恨不能把每个人分成十个人来用。每天从天亮开始,刘知县不是在县衙审理因为拥挤产生的纠纷就是在蚂蚁街维持秩序,偶尔还要亲手抓一两个盗贼。尽管县衙的每个人都尽了最大的努力,蚂蚁街依然混乱,时时有事情发生。刘知县只好将这些情况如实上报知府大人,请教知府大人如何解决这个问题。

知府大人的信中说:由于蚂蚁街太小,交易的人太多,不仅影响交通,还带来许多管理的问题。目
\newpage
前解决问题的唯一办法,就是在青峡镇建一个更大的市场以满足老百姓的需要。他将拨专款建这个市场。在市场建成之前,会增派人手协助刘知县把蚂蚁街治
理好。 

修建新市场的消息自然使刘知县欣喜若狂。送走了知府大人派来的官差。刘知县长出了一口气,脚
上的难受劲儿却丝毫没有减少。 

“新市场建起来,蚂蚁街就不会那么拥挤和热闹。现在不去看以后就看不到了。”刘知县终于给自己找到了一个白天到屋顶上去的理由。看看四下无人,他脱下官服和官靴,用一块头巾包住头脸,一拧身
,跃上了屋顶。 

刘知县的双脚在踏上屋顶的一瞬间变得比猫还要轻灵,他矮胖的身子好像也灵活了许多。一眨眼的功夫,他已经攀上了蚂蚁街最高建筑迎客楼的屋顶。

太阳正当顶,每一块瓦片都被晒得暖洋洋的,光脚踩上去,温度正好,比烫脚还舒服。那些在夜里
\newpage
如波浪般虚幻的屋脊现在是如此真实,它们在阳光下显露出砖瓦青灰的颜色,错杂其间的绿树红花让整个
青峡镇看上去生气勃勃。 

刘知县趴在迎宾楼的屋顶上。屋檐上停着几只白色的鸽子,它们大概从来没有在屋顶上见过人,歪着脑袋“咕咕”叫着打量刘知县。从这儿往下看,蚂蚁街上熙熙攘攘的人群和忙碌的蚂蚁一样,各自埋头忙着做自己的事,根本没有人注意到屋顶上正有一个
人在看他们。 

一只胆大的鸽子啄了啄刘知县的脚趾头,这些露在外面的脚趾头像盛开的花瓣一样在阳光下自在地伸展着。刘知县懒得动弹,午后的阳光很温暖,他的身体像浸泡在一盆热水里,全身的每一个毛孔都舒坦,每一块肌肉都轻松,就在刘知县舒服得几乎要睡着
的时候,楼下的喧闹声让他清醒了。 

卖糖葫芦的二牛和卖西瓜的阿华不知为什么打

阿华把切开的西瓜一片片砸向二牛,二牛则抡
\newpage
起插着糖葫芦的草把子打向西瓜张。阿华虽然身材矮胖,像一个圆滚滚的西瓜,但他出手很快,西瓜片几乎连成一条线飞向二牛。二牛也不示弱,将插着糖葫芦的草把子抡得“呼呼”生风。周围的人纷纷躲避,但无奈人多,街窄,怎么躲也躲不开,许多人的身上都沾上了西瓜的汁水和糖葫芦。有人抓起身边的东西
砸向两个少年,眼看一场混战即将发生。 

“住手,不许打架!”刘知县大喝一声。房檐
上的鸽子吓得“扑楞楞”飞了起来。 

“知县大人来了。”大家停住手,四下寻找发出声音的刘知县,人们做梦也不会想到刘知县在屋顶
上。 

刘知县刚想站起身,脚下一滑,才意识到自己是在屋顶上,没有穿官服,赤着脚,头上还包了一块
头巾。 

“这副样子怎么下去呢?”刘知县犯愁了。但

\newpage
即使下不去,这件事也是要管一管的。 

他把身体紧紧贴在屋顶上,一动也不敢动。对着天空大声说:“你们不要管我在哪里,先说说到底
是怎么回事?” 

阿华大声说:“知县大人,您要给我做主。刚
才二牛不停地用东西打我。我才用西瓜砸他的。” 

“我没有。他看到我在他后面就说是我打他。后面那么多人,你凭什么说是我打你?”二牛喘着粗
气说。 

“哼,因为打在我头上的东西是糖葫芦,除了你,还有谁会用这东西打人。哎哟,是谁打我?”阿
华突然尖叫起来。 


“哎哟!” 


刘知县悄悄伸出头一看,不仅阿华捂着脑袋,

\newpage
二牛和其他人也捂着脑袋在原地打转。 

“是谁?有本事就出来。在背后偷袭别人算什
么好汉?”阿华的大嗓门吼得一条街都能听到。 

人群乱哄哄的,大家面面相觑,不知是谁在打
人。 

刘知县看到,混乱中,一个穿绿衣服的小女孩挤到二牛身后,飞快地抓起几串糖葫芦,又挤出了人群。她的个头很小,动作很快,二牛和周围的人居然都没有发现。如果不是在屋顶的位置,刘知县也看不
到这一幕。 

绿衣女孩在一个屋檐下站住,张嘴咬下一个糖葫芦,使劲嚼了嚼,皱着眉,吐了出来。一眨眼的功夫,她已经吃完了手里的几串糖葫芦。准确地说,不是吃完,而是每一颗都放进嘴里嚼一嚼,又吐了出来。女孩的表情似乎很不高兴。她四下看了看,一扬手,把吐在手里的糖葫芦撒向人群。女孩个子不高,手劲却不小,糖葫芦的碎渣从她的手里径直飞出去,竟

\newpage
然都打到了大家的脑袋上。 



“到底是谁,有种的站出来。”人群又喧闹起
来。 

绿衣女孩笑着,若无其事地扭身朝蚂蚁街的另
一头走去。 

“小小年纪就如此调皮,应当给她一点儿教训。”刘知县忍不住叫了起来:“快抓住那个穿绿衣服
的小姑娘。是她在打人。” 

下面又是一阵混乱,大家不再寻找刘知县在哪
儿,七手八脚忙着抓那个穿绿衣服的小姑娘去了。 

趁这功夫,刘知县连忙匍匐着身子,抄近路回
到了县衙。 

刘知县刚换好衣服,就听前面传来一阵喧闹声

\newpage
。 

这喧闹声中,有两个女孩的哭声很特别,一个尖利得如刀子一样,刺得人耳朵疼;另一个则像一把钝钝的锉刀在锉铁皮,刺啦,刺啦,让人头皮发麻。

刘知县连忙来到公堂上,往下一看,不由得吃
了一惊。 

堂下竟然有两个一模一样穿绿衣服,年纪大约七八岁的女孩。两人梳着一样的发髻,系着一样的头绳,一样的鼻子、眼睛和嘴巴,唯一不同的,只有两
人的哭声。 

刘知县发愁了,“怎么会是两个呢?刚才只看到她的模样,没有听到声音啊。”他一拍惊堂木,大
喝一声:“刚才是谁在蚂蚁街偷糖葫芦吃,说!” 


“哇——哇——” 


“呜呜——呜呜——” 

\newpage

两个女孩没有说话,哭声反而更大了。刘知县
吓了一跳,大家不禁捂住了耳朵。 

二牛看了看自己光秃秃的草把子,疑惑地问:“大人,您是说,我的这些糖葫芦都是她们偷吃的?
” 

“哼!光顾着打架,别人把你的糖葫芦都吃光了,你还不知道是谁吃的。”刘知县瞪了二牛一眼,
“我看得一清二楚,就是她们中的一个。” 

刘知县捂着耳朵走到两个女孩面前仔细打量着
,无奈两人长得实在太相象,根本没有办法区分。 

二牛捂着耳朵对刘知县大声说:“算了,算了,即使是她们偷了我的糖葫芦我也认了。放她们走吧
。大人。不知道的人还以为我们欺负小孩呢。” 

刘知县想了想,对两个女孩说:“既然二牛这
么说,本县当然没意见。就不追究了吧。” 

\newpage

两个女孩的哭声戛然而止,站起身就要往外走

“且慢,走之前你们必须说清楚,为什么要把
糖葫芦嚼碎了打人?” 


两个女孩一齐问:“说了就可以走么?” 


刘知县点点头。 

两个女孩又异口同声地问:“你说话算话吗?


刘知县笑了:“当然算话。” 

两个女孩瞟了二牛一眼,商量好似地一齐说:“我们问他,这糖葫芦甜不甜,他说甜。可是我们把他的糖葫芦全吃光了,也没有吃到一个甜的。”如果不是两个人的声音有高低区别,听到的人会以为只有
一个人在说话。 

二牛的脸红了,“做这些糖葫芦的时候,冰糖

\newpage
不多了,就放得少了点儿。” 

刘知县问:“他的糖葫芦不甜你也不能打人啊
。” 

“可是,他们如果不打起来,我们怎么可以不花钱就把每一个糖葫芦都尝一尝呢?”两个女孩理直
气壮地说。 


刘知县说:“吃糖葫芦是要花钱买的……” 


“可是我们的钱已经花光了。” 

“你们钱花光了就来找我的麻烦?”二牛哭笑
不得。 

“因为你的糖葫芦不甜,我们买错了,钱才花
光的。”两个女孩振振有词。 


“可是你们已经吃了啊。”二牛争辩道。 

“但我们的钱是买甜的糖葫芦的,不是买酸的
\newpage

糖葫芦。” 

二牛小声说:“你们觉得不甜可以退给我,打
人太浪费了。” 

“咬过一口的也可以退吗?”女孩天真地问。


“咬过的当然不可以。” 


“但没有咬过怎么知道不甜呢?” 

“嗯,要咬过才知道……”二牛的脸胀得比糖
葫芦还红,不知道该怎么回答女孩的问题。 


刘知县岔开话题:“你们打人就不对……” 

两个女孩认真地说:“我们姥姥说,做了错事
就该打。他的糖葫芦做得不好就应该打。” 

刘知县说:“这个世界上谁要是不满意了就打人,天下可就要大乱了。你们俩记住,以后不能这样
\newpage

做。” 

两个女孩答应着:“哦,知道了。我们可以走
了吗?” 

刘知县说:“问了半天,我还不知道你们俩叫
什么呢,告诉我你们的名字再走。” 

两个女孩一起皱了皱眉:“你刚才没说要知道我们的名字啊?你说话不算话!你是个骗人的知县大
人。”小嘴一瘪,好像马上就要哭出来。 


大家一边笑,一边捂住了耳朵。 

刘知县从来没有遇到过这样的案子,明明已经掌握了作案者的作案动机,了解了作案经过,人证物证俱全,不仅不能给案犯定罪,还连作案者的姓名都
不知道。 

但现在丢失糖葫芦的二牛说不追究她们偷糖葫芦的行为,自己又亲口承诺她们说出打人的原因就可
\newpage
以走。刘知县叹了一口气,摆摆手:“走吧,走吧,
以后不要再这样了。退堂!” 

两个绿衣女孩和人们一起退了出去。刘知县抹了一把额头上的汗水,长长地舒了一口气:“唉,审
案子比练功还累啊。” 

没等他坐下喝口水。刚刚退出公堂的人们又回
来了。 

人们是倒退着进来的。准确地说,他们是被一
双手推回来的。 

刘知县伸头一看,大门外,一个穿红衣服的老太婆张开双手,不知用了什么力气,将大家一起推回
了公堂。 

“谁说我的孙女偷别人的东西吃了?不说清楚
谁也别想走!” 

一个浑厚的声音像炸雷一样在公堂上炸响。刘
\newpage

知县面前的茶碗也一连蹦了好几下。 

“什么人?竟敢咆哮公堂!”刘知县大声喝道

“山楂姥姥我。”一团红影瞬乎从门口移到刘
知县面前。 

刘知县一看,险些笑了起来。这个穿一身红衣服的老太太真应当叫做山楂姥姥。她的身体就像三个串在一起的山楂。圆圆的头,圆圆的身体,就连腿也
是圆圆的罗圈腿。 

山楂姥姥气咻咻地问道:“刚才是谁说我们孩
子偷吃糖葫芦的?” 


刘知县问:“您的孩子是谁?” 

山楂姥姥一手一个,把两个绿衣女孩推到刘知县面前,“这是我的孙女山一楂和山二楂。刚才谁说
她们偷糖葫芦吃?” 

\newpage

世界上竟然有那么奇怪的名字,难怪刚才两个
女孩不愿意说。刘知县忍住笑回答道:“本县。” 


“你有什么证据?” 


“我亲眼所见。” 

“你看见她们哪只手里有别人的东西?捉贼捉
脏,难道你不知道吗?” 

“我,我……”刘知县咬住舌头尖才没让自己
说出是在屋顶上看到的。 

“你们谁看到她们偷糖葫芦了?”山楂姥姥问
堂下众人。 


大家纷纷摇头。 

“你自己说,她们有没有偷你的糖葫芦?”山楂姥姥从人群中揪出二牛,谁也没有看清楚她用了什么手法,一下就把二牛壮硕的身体轻巧地拎在手里,
\newpage

就像她随手提着个包裹一样。 

二牛瞥了一眼堂上的刘知县,因为被山楂姥姥
倒拎着,他只能看到刘知县的一双脚。 


“说话呀!”山楂姥姥抖了抖二牛。 

“我说,我说,我,我,我没有看见。”二牛
害怕得上牙直打下牙。 

“我问的是她们偷了没有?”山楂姥姥又抖了
抖。 


“不知道。呜……”二牛哭了起来。 

山楂姥姥把二牛扔在地上,“我还没怎么你呢,你就委屈成这样。你们冤枉我们孩子偷东西她们才
委屈呢。噢——噢——” 

山楂姥姥竟然哭了,她的哭声比刚才两个女孩的声音还要大,哭声在公堂上空回荡,像有人在大家
\newpage

头顶上推一扇大石磨,压得人喘不过气来。 

“啪!”刘知县使劲拍了一下惊堂木,“公堂之上,怎么能这样大吵大闹。你眼里还有王法吗?”

山楂姥姥的哭声顿住了,她用衣袖抹了抹眼睛,“知县大人,连你都说我们孩子是小偷,我还能让
谁给我做主呢?噢——” 

刘知县提高了嗓门:“这位姥姥,您说这两个
孩子没有偷糖葫芦,您的证据在哪里?” 

“你这个知县大人真是奇怪,你有本事让大街上的人都证明自己是好人。”山楂姥姥瞪着刘知县。

“如果有这个必要我会的。您先找出您的证据
吧。” 

山楂姥姥想了想,说:“我们孩子有钱买糖葫
芦,她们不会去偷。” 

\newpage

“可是她们刚才已经承认了,她们的钱花光了
,就让大家打起来,趁乱偷了糖葫芦。” 

“呵呵——呵呵——”山楂姥姥拍着巴掌笑起
来,“聪明孩子,还知道声东击西。” 


“您这是承认她们偷糖葫芦了?” 


“没有!”山楂姥姥大声说。 


“那您刚才说她们声东击西怎么解释?” 

山楂姥姥的嘴里像塞进了几十个糖葫芦,大张
着却说不出话来。 

“本县看在二牛不追究两个孩子的行为,她们也承认了打人的动机,就放她们走了。您老人家还有
什么要说的?” 

“我就不说什么了。”山楂姥姥转身拉着山一楂和山二楂往外走,刚走几步,她又回来了,“不对
\newpage
,我被你绕进去了。照你这么说,我们孩子还是小偷
?” 

刘知县看着眼睛瞪得比糖葫芦还圆的山楂姥姥,不知道该说什么。他只能闭上一只眼睛吩咐道:“
茶,给我倒茶。” 

刘知县思考问题的时候必须闭上一只眼睛;心烦着急的时候就想喝茶,喝很多的茶。这个习惯县衙的人都知道。每次刘知县审案子时,衙役阿都就会早
早把茶水预备好了。 

刘知县刚端起杯子,一股扑面而来的劲风使他睁着的一只眼睛也闭了起来,手里一松,杯子不知怎
么就到了山楂姥姥的手里。 

“今天你不说清楚就别想喝水吃饭。”山楂姥
姥把茶杯里的水一饮而尽。 

刘知县苦笑着舔了舔嘴唇:“您问问大家,我

\newpage
有没有说两个孩子是小偷?” 

山楂姥姥把目光转向大家,人们连忙摇头:“
没有!没有!” 

“您现在的行为我可以告您扰乱公堂。”刘知
县觉得嗓子干得不得了,说话都困难。 

“那,你对大家说,我们孩子不是小偷。”山
楂姥姥撅着嘴,声音小了许多。 

“二牛,那些糖葫芦值多少钱?山楂姥姥要付
钱。”刘知县有了一个主意。 

山楂姥姥疑惑地问:“我什么时候说过要付钱


二牛连忙摇手,“不用了,不用了。” 

刘知县说:“付过钱的东西就不能说是偷的了
,这个道理您应该知道吧?” 

山楂姥姥笑了:“你说的对。我明白了。”她
\newpage

转向二牛:“一共多少钱?” 


二牛还是摇手,“不用了,不用了。” 

山楂姥姥一把将二牛提起来:“怎么不用?你
不要钱是要让我们孩子担小偷的名儿吗?” 


二牛使劲点头,“是!是!是!” 

山楂姥姥把二牛举了起来:“你小子胆子不小
……” 

刘知县忙说:“您放下他吧,二牛的意思是您
说得对。” 

山楂姥姥掏出一把铜钱递给二牛,“够不够?


二牛说:“够了,够了。” 

山楂姥姥瘪了瘪嘴:“我把明天买糖葫芦的钱

\newpage
都给你了,怎么会不够。” 

两个女孩问:“姥姥,您今天的糖葫芦怎么办
呢?” 

山楂姥姥叹了口气,“你们把他的糖葫芦都全扔了,我还吃什么呢?我们现在到另一个集市去吧。

女孩担心地说:“最近的集市离这里有三十里地,天快黑了,等我们到那里恐怕没有糖葫芦卖了。

刘知县问:“能告诉我你们为什么一定要买到
糖葫芦吗?” 

山楂姥姥突然有些不好意思,原本红彤彤的脸
现在更红了。 

女孩抢着说:“我们姥姥有一个习惯,每天可以不吃饭,但不能不吃糖葫芦。我们每天都要出来给
姥姥买糖葫芦。” 

刘知县点点头:“我明白大家为什么称您是山
\newpage
楂姥姥了。”他转身问二牛:“做一串糖葫芦要多长
时间?” 

二牛说:“我家里有现成的材料,很快就可以
做好的。” 

刘知县说:“那你现在就回去做吧。山楂姥姥,您和两个孩子跟二牛一起回家,让他马上给您做。

二牛有些为难:“我家在青峡山上,路不好走
。姥姥您……” 

二牛的话还没说完,山楂姥姥“忽”地一下就把他提在了手里,“你小子竟敢小看姥姥,说,你家
在哪里?” 

二牛连忙说:“不敢,不敢,您把我放下来,
我自己走吧。” 

山楂姥姥笑道:“我可等不了那么长的时间。

\newpage
还是我提着你走快一些。” 

话音未落,山楂姥姥的人已经在大门外了。

\end{document}
